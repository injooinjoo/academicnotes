%%%%%%%%%%%%%%%%%%%%%%%%%%%%%%%%%%%%%%%%%%%%%%%%%%%%%%%%%%%%%%%%%%%%%%%%%%%%%%%
% Harvard Academic Notes - 통합 마스터 템플릿
% 모든 강의 노트에 적용되는 통일된 스타일
% 버전: 2.1 - 가독성 개선 (선택적 최적화)
% 최종 수정일: 2025-11-17
%%%%%%%%%%%%%%%%%%%%%%%%%%%%%%%%%%%%%%%%%%%%%%%%%%%%%%%%%%%%%%%%%%%%%%%%%%%%%%%

\documentclass[11pt,a4paper]{article}

%========================================================================================
% 기본 패키지
%========================================================================================

% --- 한국어 지원 ---
\usepackage{kotex}

% --- 페이지 레이아웃 ---
\usepackage[top=20mm, bottom=20mm, left=20mm, right=18mm]{geometry}
\usepackage{setspace}
\onehalfspacing                      % 1.5배 줄간격
\setlength{\parskip}{0.5em}          % 문단 간격
\setlength{\parindent}{0pt}          % 들여쓰기 없음

% --- 표 관련 ---
\usepackage{booktabs}              % 고품질 표
\usepackage{tabularx}              % 자동 너비 조절 표
\usepackage{array}                 % 표 컬럼 확장
\usepackage{longtable}             % 여러 페이지 표
\renewcommand{\arraystretch}{1.1}  % 표 행간 조절

%========================================================================================
% 헤더 및 푸터
%========================================================================================

\usepackage{fancyhdr}
\pagestyle{fancy}
\fancyhf{}
\fancyhead[L]{\small\textit{CSCI E-103: 재현 가능한 머신러닝}}
\fancyhead[R]{\small\textit{Lecture 14}}
\fancyfoot[C]{\thepage}
\renewcommand{\headrulewidth}{0.5pt}
\renewcommand{\footrulewidth}{0.3pt}

% 첫 페이지는 헤더 없음
\fancypagestyle{firstpage}{
    \fancyhf{}
    \fancyfoot[C]{\thepage}
    \renewcommand{\headrulewidth}{0pt}
}

%========================================================================================
% 색상 정의 (파스텔 톤 + 다크모드 호환)
%========================================================================================

\usepackage[dvipsnames]{xcolor}

% 밝은 배경용 파스텔 색상
\definecolor{lightblue}{RGB}{220, 235, 255}      % 부드러운 파랑
\definecolor{lightgreen}{RGB}{220, 255, 235}     % 부드러운 초록
\definecolor{lightyellow}{RGB}{255, 250, 220}    % 부드러운 노랑
\definecolor{lightpurple}{RGB}{240, 230, 255}    % 부드러운 보라
\definecolor{lightgray}{gray}{0.95}              % 밝은 회색
\definecolor{lightpink}{RGB}{255, 235, 245}      % 부드러운 핑크
\definecolor{boxgray}{gray}{0.95}
\definecolor{boxblue}{rgb}{0.9, 0.95, 1.0}
\definecolor{boxred}{rgb}{1.0, 0.95, 0.95}

% 진한 색상 (테두리/제목용)
\definecolor{darkblue}{RGB}{50, 80, 150}
\definecolor{darkgreen}{RGB}{40, 120, 70}
\definecolor{darkorange}{RGB}{200, 100, 30}
\definecolor{darkpurple}{RGB}{100, 60, 150}

%========================================================================================
% 박스 환경 (tcolorbox) - 6가지 타입
%========================================================================================

\usepackage[most]{tcolorbox}
\tcbuselibrary{skins, breakable}

% 1. 개요 박스 (강의 시작 부분)
\newtcolorbox{overviewbox}[1][]{
    enhanced,
    colback=lightpurple,
    colframe=darkpurple,
    fonttitle=\bfseries\large,
    title=📚 강의 개요,
    arc=3mm,
    boxrule=1pt,
    left=8pt,
    right=8pt,
    top=8pt,
    bottom=8pt,
    breakable,
    #1
}

% 2. 요약 박스
\newtcolorbox{summarybox}[1][]{
    enhanced,
    colback=lightblue,
    colframe=darkblue,
    fonttitle=\bfseries,
    title=📝 핵심 요약,
    arc=2mm,
    boxrule=0.7pt,
    left=6pt,
    right=6pt,
    top=6pt,
    bottom=6pt,
    breakable,
    #1
}

% 3. 핵심 정보 박스
\newtcolorbox{infobox}[1][]{
    enhanced,
    colback=lightgreen,
    colframe=darkgreen,
    fonttitle=\bfseries,
    title=💡 핵심 정보,
    arc=2mm,
    boxrule=0.7pt,
    left=6pt,
    right=6pt,
    top=6pt,
    bottom=6pt,
    breakable,
    #1
}

% 4. 주의사항 박스
\newtcolorbox{warningbox}[1][]{
    enhanced,
    colback=lightyellow,
    colframe=darkorange,
    fonttitle=\bfseries,
    title=⚠️ 주의사항,
    arc=2mm,
    boxrule=0.7pt,
    left=6pt,
    right=6pt,
    top=6pt,
    bottom=6pt,
    breakable,
    #1
}

% 5. 예제 박스
\newtcolorbox{examplebox}[1][]{
    enhanced,
    colback=lightgray,
    colframe=black!60,
    fonttitle=\bfseries,
    title=📖 예제: #1,
    arc=2mm,
    boxrule=0.7pt,
    left=6pt,
    right=6pt,
    top=6pt,
    bottom=6pt,
    breakable,
}

% 6. 정의 박스
\newtcolorbox{definitionbox}[1][]{
    enhanced,
    colback=lightpink,
    colframe=purple!70!black,
    fonttitle=\bfseries,
    title=📌 정의: #1,
    arc=2mm,
    boxrule=0.7pt,
    left=6pt,
    right=6pt,
    top=6pt,
    bottom=6pt,
    breakable,
}

% 7. 중요 박스 (importantbox - warningbox와 유사)
\newtcolorbox{importantbox}[1][]{
    enhanced,
    colback=boxred,
    colframe=red!70!black,
    fonttitle=\bfseries,
    title=⚠️ 매우 중요: #1,
    arc=2mm,
    boxrule=0.7pt,
    left=6pt,
    right=6pt,
    top=6pt,
    bottom=6pt,
    breakable,
}

% 8. cautionbox (warningbox와 동일)
\let\cautionbox\warningbox
\let\endcautionbox\endwarningbox

%========================================================================================
% 코드 블록 설정 (밝은 배경)
%========================================================================================

\usepackage{listings}

\definecolor{codegray}{rgb}{0.5,0.5,0.5}
\definecolor{codepurple}{rgb}{0.58,0,0.82}
\definecolor{backcolour}{rgb}{0.95,0.95,0.95}

\lstset{
    basicstyle=\ttfamily\small,
    backgroundcolor=\color{lightgray},
    keywordstyle=\color{darkblue}\bfseries,
    commentstyle=\color{darkgreen}\itshape,
    stringstyle=\color{purple!80!black},
    numberstyle=\tiny\color{black!60},
    numbers=left,
    numbersep=8pt,
    breaklines=true,
    breakatwhitespace=false,
    frame=single,
    frameround=tttt,
    rulecolor=\color{black!30},
    captionpos=b,
    showstringspaces=false,
    tabsize=2,
    xleftmargin=15pt,
    xrightmargin=5pt,
    escapeinside={\%*}{*)}
}

% Python 코드 스타일
\lstdefinestyle{pythonstyle}{
    language=Python,
    morekeywords={self, True, False, None},
}

% SQL 코드 스타일
\lstdefinestyle{sqlstyle}{
    language=SQL,
    morekeywords={SELECT, FROM, WHERE, JOIN, GROUP, BY, ORDER, HAVING},
}

%========================================================================================
% 목차 스타일링
%========================================================================================

\usepackage{tocloft}
\renewcommand{\cftsecleader}{\cftdotfill{\cftdotsep}}
\setlength{\cftbeforesecskip}{0.4em}
\renewcommand{\cftsecfont}{\bfseries}
\renewcommand{\cftsubsecfont}{\normalfont}

%========================================================================================
% 표 및 그림
%========================================================================================

\usepackage{graphicx}              % 이미지
\usepackage{adjustbox}             % 표/박스 크기 조절

% 표 캡션 스타일
\usepackage{caption}
\captionsetup[table]{
    labelfont=bf,
    textfont=it,
    skip=5pt
}
\captionsetup[figure]{
    labelfont=bf,
    textfont=it,
    skip=5pt
}

%========================================================================================
% 수학
%========================================================================================

\usepackage{amsmath, amssymb, amsthm}

% 정리 환경
\theoremstyle{definition}
\newtheorem{theorem}{정리}[section]
\newtheorem{lemma}[theorem]{보조정리}
\newtheorem{proposition}[theorem]{명제}
\newtheorem{corollary}[theorem]{따름정리}
\newtheorem{definition}{정의}[section]
\newtheorem{example}{예제}[section]

%========================================================================================
% 하이퍼링크
%========================================================================================

\usepackage[
    colorlinks=true,
    linkcolor=blue!80!black,
    urlcolor=blue!80!black,
    citecolor=green!60!black,
    bookmarks=true,
    bookmarksnumbered=true,
    pdfborder={0 0 0}
]{hyperref}

% PDF 메타데이터는 각 문서에서 설정
\hypersetup{
    pdftitle={CSCI E-103: 재현 가능한 머신러닝 - Lecture 14},
    pdfauthor={강의 노트},
    pdfsubject={Academic Notes}
}

%========================================================================================
% 기타 유용한 패키지
%========================================================================================

\usepackage{enumitem}              % 리스트 커스터마이징
\setlist{nosep, leftmargin=*, itemsep=0.3em}

\usepackage{microtype}             % 타이포그래피 개선
\usepackage{footnote}              % 각주 개선
\usepackage{url}                   % URL 줄바꿈
\urlstyle{same}

%========================================================================================
% 사용자 정의 명령어
%========================================================================================

% 강조 텍스트
\newcommand{\important}[1]{\textbf{\textcolor{red!70!black}{#1}}}
\newcommand{\keyword}[1]{\textbf{#1}}
\newcommand{\term}[1]{\textit{#1}}
\newcommand{\code}[1]{\texttt{#1}}

% 용어 설명 (인라인)
\newcommand{\defterm}[2]{\textbf{#1}\footnote{#2}}

% 섹션 시작 전 페이지 분리
\newcommand{\newsection}[1]{\newpage\section{#1}}

%========================================================================================
% 문서 제목 스타일
%========================================================================================

\usepackage{titling}
\pretitle{\begin{center}\LARGE\bfseries}
\posttitle{\par\end{center}\vskip 0.5em}
\preauthor{\begin{center}\large}
\postauthor{\end{center}}
\predate{\begin{center}\large}
\postdate{\par\end{center}}

%========================================================================================
% 섹션 제목 간격
%========================================================================================

\usepackage{titlesec}
\titlespacing*{\section}{0pt}{1.5em}{0.8em}
\titlespacing*{\subsection}{0pt}{1.2em}{0.6em}
\titlespacing*{\subsubsection}{0pt}{1em}{0.5em}

%========================================================================================
% 메타 정보 박스 명령어
%========================================================================================

\newcommand{\metainfo}[4]{
\begin{tcolorbox}[
    colback=lightpurple,
    colframe=darkpurple,
    boxrule=1pt,
    arc=2mm,
    left=10pt,
    right=10pt,
    top=8pt,
    bottom=8pt
]
\begin{tabular}{@{}rl@{}}
▣ \textbf{강의명:} & #1 \\[0.3em]
▣ \textbf{주차:} & #2 \\[0.3em]
▣ \textbf{교수명:} & #3 \\[0.3em]
▣ \textbf{목적:} & \begin{minipage}[t]{0.75\textwidth}#4\end{minipage}
\end{tabular}
\end{tcolorbox}
}

%========================================================================================
% 끝
%========================================================================================


\begin{document}

\metainfo{CSCI E-103: 재현 가능한 머신러닝}{Lecture 14}{Anindita Mahapatra \& Eric Gieseke}{Lecture 14의 핵심 개념 학습}

% 3.1 첫 페이지 구조 통일
\thispagestyle{plain}

\metainfo{Big Data Analysis \& Engineering}
    {Day 14 - Databricks Roadmap \& Review}
    {Prof. \& Teaching Staff}
    {Databricks의 최신 기능(Lakebase, AI/BI, Agent) 이해 및 Quiz 2, Assignment 3의 주요 개념(Spark 최적화, Streaming) 심층 리뷰}

\tableofcontents

\newpage

% =========================================================
% 4. 본문 내용
% =========================================================

\section{Databricks Data Intelligence Platform (Roadmap)}

이번 강의에서는 Databricks 로드맵에 포함된 최신 기능들을 다룹니다. 이 기능들은 주로 \textbf{Data Intelligence Capabilities}를 강화하여 플랫폼을 더 스마트하게 만들고 데이터 담당자의 업무를 줄이는 데 초점을 맞춥니다.

\subsection{주요 구성 요소 요약}

\begin{table}[h!]
    \centering
    \begin{adjustbox}{width=\textwidth, center}
    \begin{tabular}{l|l|l}
    \toprule
    \textbf{Category} & \textbf{Feature} & \textbf{Description} \\
    \midrule
    \textbf{Data Warehousing} & Lakebase (Postgres) & Lakehouse 환경 내에서 OLTP 트랜잭션 처리를 지원하는 관리형 Postgres. \\
    \midrule
    \textbf{AI \& Agents} & Agent Bricks & AutoML처럼 데이터셋만 지정하면 에이전트를 자동 생성해주는 프레임워크. \\
    & Genie & 정형화된 "What" 질문에 대한 답변을 제공하는 BI 도구. \\
    & Research Mode & 가설 기반의 "Why" 질문에 대해 Chain of Thought를 통해 심층 답변 제공. \\
    \midrule
    \textbf{Governance} & ABAC & 속성 기반 접근 제어(Attribute-Based Access Control). PII 태그 등에 따라 자동 권한 부여. \\
    \midrule
    \textbf{Apps} & Lakehouse Apps & 데이터가 있는 곳에서 바로 실행되는 Python(Streamlit 등) 기반 앱. 보안 및 확장성 보장. \\
    \bottomrule
    \end{tabular}
    \end{adjustbox}
    \caption{Databricks 주요 신규 기능 요약}
\end{table}

\subsection{Lakebase (Managed Postgres)}
기존의 Databricks는 OLAP(분석용)에 최적화되어 있었으나, \textbf{Lakebase}의 도입으로 OLTP(트랜잭션용) 워크로드까지 커버하게 되었습니다.

\begin{itemize}
    \item \textbf{구조적 특징}: Compute와 Storage가 분리되어 있습니다.
    \item \textbf{장점}: 
        \begin{itemize}
            \item 사용량이 늘어나면 Compute가 확장(Scale-up)되고, 사용하지 않으면 0으로 축소(Scale-to-Zero)되어 비용 효율적입니다.
            \item Production과 Development 브랜치를 기본 제공하여, 운영 환경에 영향을 주지 않고 테스트가 가능합니다.
        \end{itemize}
    \item \textbf{통합}: Unity Catalog와 연동되며, Lakehouse와 Lakebase 간 양방향 데이터 동기화(Reverse Sync)가 가능합니다.
\end{itemize}

\subsection{AI/BI: Genie \& Research Mode}
\begin{itemize}
    \item \textbf{Genie}: 정형 데이터에 대한 사실적 질문(Fact-based questions)을 SQL로 변환하여 답변합니다.
    \item \textbf{Research Mode}: "왜(Why)"에 대한 질문을 처리합니다. 단순 검색이 아니라 가설을 세우고 여러 경로를 탐색(Reasoning)하여 답변을 도출합니다. (예: "관세 인상이 포트폴리오에 미칠 영향은?")
\end{itemize}

\subsection{Apps \& Model Serving}
\begin{itemize}
    \item \textbf{Apps}: Streamlit, Flask 등으로 만든 앱을 데이터가 저장된 플랫폼 위에서 직접 구동합니다. 데이터 이동 없이 보안이 유지된 상태로 대시보드 이상의 상호작용(Write-back 등)이 가능합니다.
    \item \textbf{Model Serving}: 최신 LLM(Llama, Gemini, Claude 등)을 즉시 사용할 수 있도록 제공하며, MCP(Model Context Protocol) 서버를 통해 에이전트가 외부 도구나 데이터에 접근하도록 지원합니다.
\end{itemize}

\newpage

\section{Quiz 2 Review: Spark \& Delta Lake Concepts}

퀴즈에서 많이 혼동했던 Spark와 Delta Lake의 핵심 개념을 정리합니다.

\subsection{Spark Optimization \& Troubleshooting}

\begin{enumerate}
    \item \textbf{Spark 최적화 4대 요소}: Data Skew, Shuffle, Spill, Small Files.
    \item \textbf{Broadcast Variables}:
        \begin{itemize}
            \item \textbf{Immutable (불변)}: 생성 후 변경할 수 없습니다.
            \item \textbf{Local to Workers}: 클러스터 간 공유되는 것이 아니라, 각 워커 노드에 복사본이 저장됩니다.
        \end{itemize}
    \item \textbf{Repartition vs Coalesce}:
        \begin{itemize}
            \item 파티션을 \textbf{늘릴 때} (예: 12 $\rightarrow$ 24): `repartition()` 사용 (Shuffle 발생).
            \item 파티션을 \textbf{줄일 때}: `coalesce()` 사용 (Shuffle 최소화).
        \end{itemize}
    \item \textbf{Action 함수 주의사항}:
        \begin{itemize}
            \item `collect()`: 모든 데이터를 드라이버로 가져오므로 OOM(Out of Memory) 위험이 큼. 디버깅용으로만 사용 권장.
            \item `take(n)`: 필요한 $n$개 행만 가져오므로 대용량 데이터에서 안전함.
        \end{itemize}
\end{enumerate}

\subsection{Delta Lake Internals}
\begin{itemize}
    \item \textbf{Transaction Log}: `_delta_log` 폴더 내에 저장됩니다.
    \item \textbf{구조}: 각 트랜잭션은 \textbf{개별 JSON 파일}로 기록됩니다. (하나의 JSON이 아님). 10번째 트랜잭션마다 Checkpoint(Parquet) 파일이 생성됩니다.
    \item \textbf{Time Travel}: 별도의 복사본을 만드는 것이 아니라, 트랜잭션 로그를 통해 과거 시점의 데이터 상태를 조회합니다.
\end{itemize}

\subsection{Data Frame Operations}
\begin{itemize}
    \item \textbf{Explode}: DataFrame의 메서드가 아니라 `select` 문 내에서 사용해야 합니다. (예: `df.select(explode(col))`)
    \item \textbf{Drop Duplicates}: `df.dropDuplicates(['col1', 'col2'])` 형태로 리스트를 전달해야 합니다.
\end{itemize}

\newpage

\section{Assignment 3 Review: Streaming \& Architecture}

\subsection{Kafka vs. Kinesis}
\begin{table}[h!]
    \centering
    \begin{adjustbox}{width=\textwidth, center}
    \begin{tabular}{l|l|l}
    \toprule
    \textbf{Feature} & \textbf{AWS Kinesis} & \textbf{Apache Kafka} \\
    \midrule
    \textbf{관리 주체} & AWS 완전 관리형 (Managed) & Confluent (상용) 또는 Open Source (직접 관리) \\
    \textbf{확장 단위} & \textbf{Shard} (샤드 수에 따라 비용/성능 결정) & \textbf{Partition} (파티션 단위로 병렬 처리) \\
    \textbf{데이터 분배} & Partition Key 사용 & Round-robin 또는 Key-Value 기반 Partitioning \\
    \bottomrule
    \end{tabular}
    \end{adjustbox}
    \caption{Kinesis와 Kafka 비교}
\end{table}

\subsection{Slowly Changing Dimensions (SCD)}
\begin{itemize}
    \item \textbf{Type 1}: 과거 데이터를 덮어씀(Overwrite). 이력 관리 안 됨.
    \item \textbf{Type 2}: 새로운 행을 추가하고 유효 기간(Start/End Date)이나 플래그(Current)를 두어 이력을 관리함.
    \item \textbf{Delta Merge}: `MERGE INTO` 구문을 사용하여 변경된 컬럼(예: City)만 감지하고 업데이트/삽입 로직을 구현합니다.
\end{itemize}

\subsection{Spark Streaming Logic}
\begin{itemize}
    \item \textbf{Trigger 옵션}:
        \begin{itemize}
            \item `trigger(once=True)`: \textbf{Deprecated}.
            \item `trigger(availableNow=True)`: 권장됨. 데이터를 마이크로 배치로 나누어 처리하므로 클러스터 과부하를 방지합니다.
        \end{itemize}
    \item \textbf{Checkpointing}: 스트리밍의 핵심. 시스템 중단 후 재시작 시 중단된 지점부터 정확히 다시 처리(Exactly-once)할 수 있게 해줍니다.
    \item \textbf{Image Processing}:
        \begin{itemize}
            \item Spark Serverless 환경에서는 `display()`로 이미지를 직접 렌더링하지 못할 수 있습니다.
            \item `pillow` 라이브러리 등을 사용하여 바이너리 데이터를 이미지로 변환하거나 처리(Invert 등)하는 UDF를 작성해야 합니다.
        \end{itemize}
\end{itemize}

\section{Next Steps \& Action Items}
\begin{itemize}
    \item \textbf{Assignment 4}: 다음 수업 전까지 최종 과제(Assignment 4)를 확인하고 준비할 것.
    \item \textbf{Guest Lecture}: 다음 수업에는 업계 전문가 초청 강연이 예정되어 있음.
\end{itemize}

\end{document}
