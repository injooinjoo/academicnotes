%%%%%%%%%%%%%%%%%%%%%%%%%%%%%%%%%%%%%%%%%%%%%%%%%%%%%%%%%%%%%%%%%%%%%%%%%%%%%%%
% Harvard Academic Notes - 통합 마스터 템플릿
% 모든 강의 노트에 적용되는 통일된 스타일
% 버전: 2.1 - 가독성 개선 (선택적 최적화)
% 최종 수정일: 2025-11-17
%%%%%%%%%%%%%%%%%%%%%%%%%%%%%%%%%%%%%%%%%%%%%%%%%%%%%%%%%%%%%%%%%%%%%%%%%%%%%%%

\documentclass[11pt,a4paper]{article}

%========================================================================================
% 기본 패키지
%========================================================================================

% --- 한국어 지원 ---
\usepackage{kotex}

% --- 페이지 레이아웃 ---
\usepackage[top=20mm, bottom=20mm, left=20mm, right=18mm]{geometry}
\usepackage{setspace}
\onehalfspacing                      % 1.5배 줄간격
\setlength{\parskip}{0.5em}          % 문단 간격
\setlength{\parindent}{0pt}          % 들여쓰기 없음

% --- 표 관련 ---
\usepackage{booktabs}              % 고품질 표
\usepackage{tabularx}              % 자동 너비 조절 표
\usepackage{array}                 % 표 컬럼 확장
\usepackage{longtable}             % 여러 페이지 표
\renewcommand{\arraystretch}{1.1}  % 표 행간 조절

%========================================================================================
% 헤더 및 푸터
%========================================================================================

\usepackage{fancyhdr}
\pagestyle{fancy}
\fancyhf{}
\fancyhead[L]{\small\textit{CSCI E-103: 재현 가능한 머신러닝}}
\fancyhead[R]{\small\textit{Lecture 11}}
\fancyfoot[C]{\thepage}
\renewcommand{\headrulewidth}{0.5pt}
\renewcommand{\footrulewidth}{0.3pt}

% 첫 페이지는 헤더 없음
\fancypagestyle{firstpage}{
    \fancyhf{}
    \fancyfoot[C]{\thepage}
    \renewcommand{\headrulewidth}{0pt}
}

%========================================================================================
% 색상 정의 (파스텔 톤 + 다크모드 호환)
%========================================================================================

\usepackage[dvipsnames]{xcolor}

% 밝은 배경용 파스텔 색상
\definecolor{lightblue}{RGB}{220, 235, 255}      % 부드러운 파랑
\definecolor{lightgreen}{RGB}{220, 255, 235}     % 부드러운 초록
\definecolor{lightyellow}{RGB}{255, 250, 220}    % 부드러운 노랑
\definecolor{lightpurple}{RGB}{240, 230, 255}    % 부드러운 보라
\definecolor{lightgray}{gray}{0.95}              % 밝은 회색
\definecolor{lightpink}{RGB}{255, 235, 245}      % 부드러운 핑크
\definecolor{boxgray}{gray}{0.95}
\definecolor{boxblue}{rgb}{0.9, 0.95, 1.0}
\definecolor{boxred}{rgb}{1.0, 0.95, 0.95}

% 진한 색상 (테두리/제목용)
\definecolor{darkblue}{RGB}{50, 80, 150}
\definecolor{darkgreen}{RGB}{40, 120, 70}
\definecolor{darkorange}{RGB}{200, 100, 30}
\definecolor{darkpurple}{RGB}{100, 60, 150}

%========================================================================================
% 박스 환경 (tcolorbox) - 6가지 타입
%========================================================================================

\usepackage[most]{tcolorbox}
\tcbuselibrary{skins, breakable}

% 1. 개요 박스 (강의 시작 부분)
\newtcolorbox{overviewbox}[1][]{
    enhanced,
    colback=lightpurple,
    colframe=darkpurple,
    fonttitle=\bfseries\large,
    title=📚 강의 개요,
    arc=3mm,
    boxrule=1pt,
    left=8pt,
    right=8pt,
    top=8pt,
    bottom=8pt,
    breakable,
    #1
}

% 2. 요약 박스
\newtcolorbox{summarybox}[1][]{
    enhanced,
    colback=lightblue,
    colframe=darkblue,
    fonttitle=\bfseries,
    title=📝 핵심 요약,
    arc=2mm,
    boxrule=0.7pt,
    left=6pt,
    right=6pt,
    top=6pt,
    bottom=6pt,
    breakable,
    #1
}

% 3. 핵심 정보 박스
\newtcolorbox{infobox}[1][]{
    enhanced,
    colback=lightgreen,
    colframe=darkgreen,
    fonttitle=\bfseries,
    title=💡 핵심 정보,
    arc=2mm,
    boxrule=0.7pt,
    left=6pt,
    right=6pt,
    top=6pt,
    bottom=6pt,
    breakable,
    #1
}

% 4. 주의사항 박스
\newtcolorbox{warningbox}[1][]{
    enhanced,
    colback=lightyellow,
    colframe=darkorange,
    fonttitle=\bfseries,
    title=⚠️ 주의사항,
    arc=2mm,
    boxrule=0.7pt,
    left=6pt,
    right=6pt,
    top=6pt,
    bottom=6pt,
    breakable,
    #1
}

% 5. 예제 박스
\newtcolorbox{examplebox}[1][]{
    enhanced,
    colback=lightgray,
    colframe=black!60,
    fonttitle=\bfseries,
    title=📖 예제: #1,
    arc=2mm,
    boxrule=0.7pt,
    left=6pt,
    right=6pt,
    top=6pt,
    bottom=6pt,
    breakable,
}

% 6. 정의 박스
\newtcolorbox{definitionbox}[1][]{
    enhanced,
    colback=lightpink,
    colframe=purple!70!black,
    fonttitle=\bfseries,
    title=📌 정의: #1,
    arc=2mm,
    boxrule=0.7pt,
    left=6pt,
    right=6pt,
    top=6pt,
    bottom=6pt,
    breakable,
}

% 7. 중요 박스 (importantbox - warningbox와 유사)
\newtcolorbox{importantbox}[1][]{
    enhanced,
    colback=boxred,
    colframe=red!70!black,
    fonttitle=\bfseries,
    title=⚠️ 매우 중요: #1,
    arc=2mm,
    boxrule=0.7pt,
    left=6pt,
    right=6pt,
    top=6pt,
    bottom=6pt,
    breakable,
}

% 8. cautionbox (warningbox와 동일)
\let\cautionbox\warningbox
\let\endcautionbox\endwarningbox

%========================================================================================
% 코드 블록 설정 (밝은 배경)
%========================================================================================

\usepackage{listings}

\definecolor{codegray}{rgb}{0.5,0.5,0.5}
\definecolor{codepurple}{rgb}{0.58,0,0.82}
\definecolor{backcolour}{rgb}{0.95,0.95,0.95}

\lstset{
    basicstyle=\ttfamily\small,
    backgroundcolor=\color{lightgray},
    keywordstyle=\color{darkblue}\bfseries,
    commentstyle=\color{darkgreen}\itshape,
    stringstyle=\color{purple!80!black},
    numberstyle=\tiny\color{black!60},
    numbers=left,
    numbersep=8pt,
    breaklines=true,
    breakatwhitespace=false,
    frame=single,
    frameround=tttt,
    rulecolor=\color{black!30},
    captionpos=b,
    showstringspaces=false,
    tabsize=2,
    xleftmargin=15pt,
    xrightmargin=5pt,
    escapeinside={\%*}{*)}
}

% Python 코드 스타일
\lstdefinestyle{pythonstyle}{
    language=Python,
    morekeywords={self, True, False, None},
}

% SQL 코드 스타일
\lstdefinestyle{sqlstyle}{
    language=SQL,
    morekeywords={SELECT, FROM, WHERE, JOIN, GROUP, BY, ORDER, HAVING},
}

%========================================================================================
% 목차 스타일링
%========================================================================================

\usepackage{tocloft}
\renewcommand{\cftsecleader}{\cftdotfill{\cftdotsep}}
\setlength{\cftbeforesecskip}{0.4em}
\renewcommand{\cftsecfont}{\bfseries}
\renewcommand{\cftsubsecfont}{\normalfont}

%========================================================================================
% 표 및 그림
%========================================================================================

\usepackage{graphicx}              % 이미지
\usepackage{adjustbox}             % 표/박스 크기 조절

% 표 캡션 스타일
\usepackage{caption}
\captionsetup[table]{
    labelfont=bf,
    textfont=it,
    skip=5pt
}
\captionsetup[figure]{
    labelfont=bf,
    textfont=it,
    skip=5pt
}

%========================================================================================
% 수학
%========================================================================================

\usepackage{amsmath, amssymb, amsthm}

% 정리 환경
\theoremstyle{definition}
\newtheorem{theorem}{정리}[section]
\newtheorem{lemma}[theorem]{보조정리}
\newtheorem{proposition}[theorem]{명제}
\newtheorem{corollary}[theorem]{따름정리}
\newtheorem{definition}{정의}[section]
\newtheorem{example}{예제}[section]

%========================================================================================
% 하이퍼링크
%========================================================================================

\usepackage[
    colorlinks=true,
    linkcolor=blue!80!black,
    urlcolor=blue!80!black,
    citecolor=green!60!black,
    bookmarks=true,
    bookmarksnumbered=true,
    pdfborder={0 0 0}
]{hyperref}

% PDF 메타데이터는 각 문서에서 설정
\hypersetup{
    pdftitle={CSCI E-103: 재현 가능한 머신러닝 - Lecture 11},
    pdfauthor={강의 노트},
    pdfsubject={Academic Notes}
}

%========================================================================================
% 기타 유용한 패키지
%========================================================================================

\usepackage{enumitem}              % 리스트 커스터마이징
\setlist{nosep, leftmargin=*, itemsep=0.3em}

\usepackage{microtype}             % 타이포그래피 개선
\usepackage{footnote}              % 각주 개선
\usepackage{url}                   % URL 줄바꿈
\urlstyle{same}

%========================================================================================
% 사용자 정의 명령어
%========================================================================================

% 강조 텍스트
\newcommand{\important}[1]{\textbf{\textcolor{red!70!black}{#1}}}
\newcommand{\keyword}[1]{\textbf{#1}}
\newcommand{\term}[1]{\textit{#1}}
\newcommand{\code}[1]{\texttt{#1}}

% 용어 설명 (인라인)
\newcommand{\defterm}[2]{\textbf{#1}\footnote{#2}}

% 섹션 시작 전 페이지 분리
\newcommand{\newsection}[1]{\newpage\section{#1}}

%========================================================================================
% 문서 제목 스타일
%========================================================================================

\usepackage{titling}
\pretitle{\begin{center}\LARGE\bfseries}
\posttitle{\par\end{center}\vskip 0.5em}
\preauthor{\begin{center}\large}
\postauthor{\end{center}}
\predate{\begin{center}\large}
\postdate{\par\end{center}}

%========================================================================================
% 섹션 제목 간격
%========================================================================================

\usepackage{titlesec}
\titlespacing*{\section}{0pt}{1.5em}{0.8em}
\titlespacing*{\subsection}{0pt}{1.2em}{0.6em}
\titlespacing*{\subsubsection}{0pt}{1em}{0.5em}

%========================================================================================
% 메타 정보 박스 명령어
%========================================================================================

\newcommand{\metainfo}[4]{
\begin{tcolorbox}[
    colback=lightpurple,
    colframe=darkpurple,
    boxrule=1pt,
    arc=2mm,
    left=10pt,
    right=10pt,
    top=8pt,
    bottom=8pt
]
\begin{tabular}{@{}rl@{}}
▣ \textbf{강의명:} & #1 \\[0.3em]
▣ \textbf{주차:} & #2 \\[0.3em]
▣ \textbf{교수명:} & #3 \\[0.3em]
▣ \textbf{목적:} & \begin{minipage}[t]{0.75\textwidth}#4\end{minipage}
\end{tabular}
\end{tcolorbox}
}

%========================================================================================
% 끝
%========================================================================================


\begin{document}

\metainfo{CSCI E-103: 재현 가능한 머신러닝}{Lecture 11}{Anindita Mahapatra \& Eric Gieseke}{Lecture 11의 핵심 개념 학습}

\tableofcontents
\newpage


% ------------------------------------------------------------------------------
% 1. 개요 (Executive Summary)
% ------------------------------------------------------------------------------
\section{개요: AI의 진화와 LLM 개발의 현재}

이 문서는 대규모 언어 모델(LLM)과 에이전트(Agent)를 활용하여 실제 비즈니스 가치를 창출하는 방법을 다룹니다.
단순히 "신기한 기술"을 넘어, 데이터를 통합하고 배포하며 운영하는 전체 수명주기(LLMOps)를 이해하는 것이 목표입니다.

\begin{infobox}{학습 목표}
\begin{itemize}
    \item \textbf{이론:} AI/ML의 진화 과정과 생성형 AI(GenAI)의 차별점 이해
    \item \textbf{전략:} 조직의 성숙도에 따른 LLM 도입 단계 (Prompting $\to$ RAG $\to$ Fine-tuning)
    \item \textbf{실무:} Databricks의 도구(Genie, Agent Bricks, Unity Catalog)를 활용한 에이전트 구축
    \item \textbf{운영:} LLMOps, 환각(Hallucination) 관리, 윤리적 AI 사용
\end{itemize}
\end{infobox}

\vspace{1em}

% ------------------------------------------------------------------------------
% 2. 용어 정리 (Terminology)
% ------------------------------------------------------------------------------
\section{필수 용어 정리}

생성형 AI 분야는 새로운 용어가 많습니다. 아래 표는 이 문서를 이해하기 위한 핵심 단어장입니다.

\begin{table}[h]
\centering
\resizebox{\textwidth}{!}{%
\begin{tabular}{l l l}
\toprule
\textbf{용어 (약어)} & \textbf{원어} & \textbf{쉬운 설명 (직관적 비유)} \\
\midrule
\textbf{LLM} & Large Language Model & 인터넷의 모든 텍스트를 읽고 언어 패턴을 익힌 \textbf{'초거대 독서광 AI'} \\
\textbf{GenAI} & Generative AI & 데이터를 분류하는 것을 넘어, 새로운 텍스트/이미지를 \textbf{'창조'}하는 AI \\
\textbf{Hallucination} & Hallucination & AI가 사실이 아닌 내용을 마치 사실인 것처럼 \textbf{'뻔뻔하게 거짓말'}하는 현상 \\
\textbf{RAG} & Retrieval Augmented Generation & AI에게 \textbf{'오픈북 테스트'}를 치르게 하듯, 참고 자료를 검색해서 보여주고 답하게 하는 기술 \\
\textbf{Grounding} & Grounding & AI의 답변을 현실 세계의 사실이나 데이터에 \textbf{'단단히 묶어두는'} 과정 \\
\textbf{Fine-Tuning} & Fine-Tuning & 일반적인 지식을 가진 AI에게 특정 분야(예: 법률, 의학)를 \textbf{'심화 과외'} 시키는 것 \\
\textbf{Embeddings} & Embeddings & 텍스트의 의미를 컴퓨터가 이해할 수 있는 \textbf{'숫자 좌표(벡터)'}로 변환하는 기술 \\
\textbf{Vector DB} & Vector Database & 의미(Embedding)를 저장하여, 키워드가 달라도 \textbf{'문맥상 유사한'} 자료를 찾아주는 검색 엔진 \\
\textbf{Chain of Thought} & Chain of Thought & AI에게 "답만 말하지 말고 \textbf{'풀이 과정'}을 단계별로 써라"고 시키는 기법 \\
\bottomrule
\end{tabular}%
}
\caption{LLM 및 생성형 AI 핵심 용어 요약}
\label{tab:terminology}
\end{table}

\newpage

% ------------------------------------------------------------------------------
% 3. 핵심 개념: AI의 진화와 LLM 성숙도
% ------------------------------------------------------------------------------
\section{핵심 개념: AI의 진화와 도입 전략}

\subsection{1. AI의 포함 관계 (Venn Diagram)}
AI 기술은 갑자기 튀어나온 것이 아니라, 수십 년간 발전해 온 기술의 집약체입니다.

\begin{itemize}
    \item \textbf{AI (Artificial Intelligence):} 인간의 지능을 모방하는 모든 시스템 (가장 큰 범주)
    \item \textbf{ML (Machine Learning):} 데이터를 통해 학습하는 AI의 하위 분야
    \item \textbf{Deep Learning (Neural Networks):} 인간의 뇌 신경망을 모방한 복잡한 ML
    \item \textbf{GenAI (Generative AI):} 기존 데이터를 분석하는 것을 넘어, \textit{새로운 데이터}를 생성하는 딥러닝의 최신 분야
    \begin{itemize}
        \item \textbf{LLM:} 텍스트 생성 특화 (예: GPT, Gemini, Claude)
        \item \textbf{GAN:} 이미지/비디오 생성 특화 (예: Deepfake, StyleGAN)
    \end{itemize}
\end{itemize}

\subsection{2. LLM 성숙도 모델 (Maturity Curve)}
기업이나 개인이 LLM을 도입할 때, 무조건 처음부터 자체 모델을 만드는 것은 비효율적입니다. 비용과 복잡도를 고려하여 단계별로 접근해야 합니다.

\begin{tcolorbox}[colback=white, colframe=black, title=\textbf{LLM 도입의 4단계 (쉬움 $\to$ 어려움)}]
\begin{enumerate}
    \item \textbf{Prompt Engineering (프롬프트 엔지니어링)}
    \begin{itemize}
        \item \textbf{설명:} AI에게 일을 시키는 명령어를 정교하게 다듬는 단계.
        \item \textbf{비용/난이도:} 매우 낮음 / 데이터 필요 없음.
        \item \textbf{한계:} AI가 학습하지 않은 최신 정보나 사내 비공개 정보는 모름.
    \end{itemize}
    
    \item \textbf{RAG (검색 증강 생성) - \textit{*가장 권장되는 단계}}
    \begin{itemize}
        \item \textbf{설명:} 사내 문서나 데이터베이스를 검색(Retrieval)하여 그 내용을 AI에게 참고자료로 주고 답변(Generation)하게 함.
        \item \textbf{장점:} 최신 정보 반영 가능, 할루시네이션 감소, 가성비 최고.
        \item \textbf{비유:} 시험 보는 학생(AI)에게 교과서(사내 데이터)를 펼쳐놓고 답을 쓰게 하는 것.
    \end{itemize}
    
    \item \textbf{Fine-Tuning (파인 튜닝)}
    \begin{itemize}
        \item \textbf{설명:} 기존 모델에 우리만의 데이터로 추가 학습을 시키는 것.
        \item \textbf{용도:} 특수한 말투, 전문 용어, 특정 형식의 답변이 필요할 때.
        \item \textbf{비용:} 데이터 준비와 학습에 상당한 비용 발생.
    \end{itemize}
    
    \item \textbf{Pre-Training (사전 학습)}
    \begin{itemize}
        \item \textbf{설명:} 처음부터 모델을 바닥부터 만드는 것 (예: 자체 GPT 만들기).
        \item \textbf{현실:} 구글, 메타급 기업이 아니면 비용(수십~수백 억 원) 문제로 거의 불가능.
    \end{itemize}
\end{enumerate}
\end{tcolorbox}

\newpage

% ------------------------------------------------------------------------------
% 4. 기술 심화: RAG (Retrieval Augmented Generation)
% ------------------------------------------------------------------------------
\section{기술 심화: RAG (검색 증강 생성)}

LLM의 가장 큰 약점인 \textbf{"최신 정보 부재"}와 \textbf{"할루시네이션(거짓말)"}을 해결하는 가장 현실적인 기술입니다.

\subsection{1. 왜 RAG가 필요한가?}
LLM은 학습이 끝난 시점(예: 2023년) 이후의 정보는 모릅니다. 또한, 우리 회사의 비공개 문서는 당연히 본 적이 없습니다. 이를 해결하기 위해 모델을 재학습시키는 것은 너무 비쌉니다. RAG는 \textbf{"외부 지식 검색"}을 통해 이 문제를 해결합니다.

\subsection{2. RAG의 작동 원리 (Workflow)}
RAG 시스템은 크게 \textbf{데이터 준비(Ingestion)}와 \textbf{검색 및 답변(Retrieval \& Generation)} 두 단계로 나뉩니다.

\begin{itemize}
    \item \textbf{단계 1: 데이터 준비 (사전 작업)}
    \begin{enumerate}
        \item 문서를 잘게 쪼갭니다 (Chunking). (예: 800~1200자 단위)
        \item 쪼갠 문서를 AI가 이해하는 숫자 배열(Vector/Embedding)로 변환합니다.
        \item 이 숫자를 \textbf{Vector DB}에 저장합니다. (유사도 검색을 위해)
    \end{enumerate}
    
    \item \textbf{단계 2: 실제 질문 시 (실시간)}
    \begin{enumerate}
        \item 사용자가 질문을 합니다. ("우리 회사의 재택근무 규정이 뭐야?")
        \item 질문을 숫자(Vector)로 변환합니다.
        \item Vector DB에서 질문과 숫자가 가장 비슷한(의미가 유사한) 문서 조각을 찾아냅니다.
        \item 찾아낸 문서 조각을 \textbf{프롬프트에 붙여서} LLM에게 보냅니다.
        \item LLM은 "아래 문서를 참고해서 답변해"라는 지시를 받고 정확한 답을 생성합니다.
    \end{enumerate}
\end{itemize}

\begin{warnbox}{초심자가 자주 하는 오해}
\textbf{Q. RAG를 쓰면 LLM이 학습을 하나요?} \\
A. 아닙니다! RAG는 LLM에게 잠시 참고자료를 보여주는 것뿐입니다. LLM의 뇌(모델 파라미터) 자체는 변하지 않습니다. 마치 시험 때 오픈북을 한다고 해서 학생의 지능이 영구적으로 변하는 것은 아닌 것과 같습니다.
\end{warnbox}

% ------------------------------------------------------------------------------
% 5. 실무: Databricks 도구 활용 (Agents \& LLMOps)
% ------------------------------------------------------------------------------
\section{실무: Databricks 도구를 활용한 구현}

Databricks 플랫폼은 LLM 애플리케이션 개발을 위한 통합 도구를 제공합니다.

\subsection{1. Genie (지니)}
\begin{itemize}
    \item \textbf{정의:} 정형 데이터(SQL 테이블)와 대화하는 에이전트.
    \item \textbf{특징:} 텍스트로 질문하면 자동으로 SQL 쿼리를 생성하여 답을 줍니다.
    \item \textbf{장점:} 데이터의 메타데이터(스키마)만 보고 SQL을 짜기 때문에, 일반 LLM보다 할루시네이션이 적고 정확도가 높습니다.
    \item \textbf{사용 예:} "지난달 매출이 가장 높은 제품 5개 보여줘" $\to$ \texttt{SELECT name, sales FROM ...} 자동 실행
\end{itemize}

\subsection{2. Agent Bricks (에이전트 브릭스)}
\begin{itemize}
    \item \textbf{정의:} 코딩 없이 클릭 몇 번으로 RAG 에이전트를 만드는 'AutoML' 같은 도구.
    \item \textbf{기능:} PDF 파일이 있는 경로만 지정하면, 자동으로 벡터 DB를 구축하고 챗봇을 생성해줍니다.
    \item \textbf{활용:} 고객 응대 챗봇, 사내 규정 검색기 등을 10분 만에 프로토타이핑 가능.
\end{itemize}

\newpage

\subsection{3. AI Functions (SQL에서 AI 쓰기)}
복잡한 파이썬 코드 없이, SQL 쿼리 안에서 바로 LLM 기능을 호출할 수 있습니다.

\begin{lstlisting}[caption={AI Functions 활용 예시}, label={lst:aifunc}, breaklines=true]
-- 고객 리뷰의 감정을 분석하고 요약하는 SQL 쿼리
SELECT 
    review_text,
    ai_analyze_sentiment(review_text) AS sentiment, -- 긍정/부정 분석
    ai_summarize(review_text, 10) AS summary       -- 10단어 요약
FROM customer_reviews
WHERE date > '2025-01-01';
\end{lstlisting}

\subsection{4. LLMOps (LLM 운영)}
기존 MLOps(머신러닝 운영)에 \textbf{"언어적 특성"}이 추가된 개념입니다.
\begin{itemize}
    \item \textbf{모델 평가:} 답변이 정확한지, 독성이 없는지 평가 (LLM을 심판으로 사용하기도 함).
    \item \textbf{피드백 루프 (RLHF):} 사용자의 "좋아요/싫어요" 버튼 클릭을 모아 모델을 개선.
    \item \textbf{보안:} 프롬프트 인젝션(해킹) 방지, 개인정보 유출 방지.
\end{itemize}

% ------------------------------------------------------------------------------
% 6. 윤리 및 리스크 관리
% ------------------------------------------------------------------------------
\section{운영과 윤리: 리스크와 대응}

LLM은 강력하지만 위험 요소도 큽니다. 이를 관리하는 것이 엔지니어의 핵심 역량입니다.

\begin{table}[h]
\centering
\resizebox{\textwidth}{!}{%
\begin{tabular}{l l l}
\toprule
\textbf{리스크 유형} & \textbf{설명} & \textbf{대응 방안} \\
\midrule
\textbf{환각 (Hallucination)} & 없는 사실을 지어냄 & RAG 사용, Temperature(창의성 수치)를 0으로 설정 \\
\textbf{편향 (Bias)} & 특정 인종/성별에 차별적 발언 & 학습 데이터의 다양성 확보, 필터링 가드레일 설치 \\
\textbf{보안 위협} & 프롬프트 인젝션 (AI 속이기) & 입력값 검증, 관리자 권한 분리 \\
\textbf{개인정보 유출} & 학습된 민감 정보 노출 & 학습 데이터 정제(Masking), 사내 전용 모델 사용 \\
\bottomrule
\end{tabular}%
}
\caption{LLM 주요 리스크 및 대응 방안}
\end{table}

\begin{infobox}{ESG와 AI}
최근 기업은 \textbf{ESG(환경, 사회, 지배구조)} 관점에서 AI를 평가합니다. 
AI 모델 학습에 막대한 전기가 소모(환경)되거나, AI가 차별적 발언(사회)을 하는 것은 기업 가치에 치명적입니다. 따라서 "윤리적 AI"는 선택이 아닌 필수입니다.
\end{infobox}

% ------------------------------------------------------------------------------
% 7. 체크리스트 및 FAQ
% ------------------------------------------------------------------------------
\section{학습 체크리스트 및 공지사항}

\subsection{주요 일정 (강의 공지)}
\begin{itemize}
    \item \textbf{성적 공개:} Quiz-1, Assignment-3 채점 완료.
    \item \textbf{과제:} Use Case-1, Assignment-4 곧 채점 예정.
    \item \textbf{최종 프로젝트:} 
    \begin{itemize}
        \item 11월 18일: 프로젝트 주제 및 세부 내용 공개.
        \item 12월 16일: 최종 발표 (팀원 전원 참석 권장).
    \end{itemize}
    \item \textbf{기타:} 다음 주는 추수감사절(Thanksgiving)로 휴강. 12월 9일 산업계 전문가 초청 강연.
\end{itemize}

\subsection{FAQ: 자주 묻는 질문}
\textbf{Q. RAG를 쓸 때, 문서가 업데이트되면 어떻게 하나요?} \\
A. Vector DB를 업데이트해야 합니다. 문서를 통째로 교체하거나 변경된 부분만 다시 임베딩하여 덮어써야 합니다. 다만, 부분 업데이트 시 버전 관리가 복잡해질 수 있습니다.

\textbf{Q. LLM이 항상 최신 정보를 알 수 있나요?} \\
A. 순수 LLM은 불가능합니다. RAG나 웹 검색 에이전트를 붙여야만 실시간 정보를 반영할 수 있습니다.

\textbf{Q. 프롬프트 엔지니어링만으로 충분한가요?} \\
A. 간단한 업무는 충분합니다. 하지만 전문적인 지식이 필요하거나 사내 데이터를 써야 한다면 RAG가 필수적입니다.

\newpage

% ------------------------------------------------------------------------------
% 8. 1페이지 요약 (Quick Summary)
% ------------------------------------------------------------------------------
\section{빠르게 훑어보기: 1페이지 핵심 요약}

\begin{summarybox}
\begin{enumerate}
    \item \textbf{AI의 흐름:} Rule-based $\to$ ML $\to$ Deep Learning $\to$ \textbf{GenAI(생성형 AI)}.
    \item \textbf{LLM 활용 전략:} 무작정 모델을 만드는(Pre-train) 것이 아니라, \textbf{RAG(검색 증강)}를 통해 사내 데이터를 연결하는 것이 가장 효율적임.
    \item \textbf{RAG 핵심:} 문서 쪼개기(Chunking) $\to$ 숫자 변환(Embedding) $\to$ Vector DB 저장 $\to$ 유사도 검색 $\to$ 답변 생성.
    \item \textbf{Databricks 도구:}
    \begin{itemize}
        \item \textbf{Genie:} 정형 데이터(SQL) 담당, 정확도 높음.
        \item \textbf{Agent Bricks:} 비정형 데이터(PDF 등) 담당, 빠른 구축.
        \item \textbf{AI Gateway:} 모델 보안 및 관리의 관문.
    \end{itemize}
    \item \textbf{주의사항:} 할루시네이션(거짓말) 방지, 데이터 보안, 윤리적 책임이 기술 구현만큼 중요함.
    \item \textbf{미래:} 단순 챗봇을 넘어, 스스로 도구를 선택하고 행동하는 \textbf{에이전트(Agent)} 시대로 진입 중.
\end{enumerate}
\end{summarybox}

\vspace{2em}

% ------------------------------------------------------------------------------
% 9. 부록 (실습 가이드)
% ------------------------------------------------------------------------------
\section*{부록: 초심자를 위한 Agent 구축 단계 (Pseudo-Flow)}

Databricks에서 RAG 에이전트를 만드는 사고의 흐름입니다.

\begin{enumerate}
    \item \textbf{데이터 준비:} PDF 파일들을 Unity Catalog Volume에 업로드.
    \item \textbf{인덱싱 (Vector Search):} "이 파일들을 읽어서 검색 가능하게 만들어줘" (Agent Bricks 활용 시 클릭 몇 번으로 완료).
    \item \textbf{에이전트 생성:} "이 인덱스를 참고하는 챗봇을 만들어."
    \item \textbf{테스트 (Playground):} 채팅창에서 질문해보고 답변 확인.
    \begin{itemize}
        \item 답변이 이상하다? $\to$ 프롬프트 수정 ("전문가처럼 답변해", "모르면 모른다고 해").
        \item 근거가 없다? $\to$ 데이터(PDF) 보강.
    \end{itemize}
    \item \textbf{배포:} Review App을 통해 사용자들에게 링크 공유 및 피드백 수집.
\end{enumerate}

\end{document}
