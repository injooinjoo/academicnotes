%%%%%%%%%%%%%%%%%%%%%%%%%%%%%%%%%%%%%%%%%%%%%%%%%%%%%%%%%%%%%%%%%%%%%%%%%%%%%%%
% Harvard CSCI E-103: Reproducible Data Science and Machine Learning
% Lecture 10: Model Bias, Data Imbalance, and Real-World Fraud Detection
% English Version
%%%%%%%%%%%%%%%%%%%%%%%%%%%%%%%%%%%%%%%%%%%%%%%%%%%%%%%%%%%%%%%%%%%%%%%%%%%%%%%

\documentclass[11pt,a4paper]{article}

%========================================================================================
% Basic Packages
%========================================================================================

\usepackage[top=20mm, bottom=20mm, left=20mm, right=18mm]{geometry}
\usepackage{setspace}
\onehalfspacing
\setlength{\parskip}{0.5em}
\setlength{\parindent}{0pt}

\usepackage{booktabs}
\usepackage{tabularx}
\usepackage{array}
\usepackage{longtable}
\renewcommand{\arraystretch}{1.1}

%========================================================================================
% Header and Footer
%========================================================================================

\usepackage{fancyhdr}
\pagestyle{fancy}
\fancyhf{}
\fancyhead[L]{\small\textit{CSCI E-103: Reproducible Data Science}}
\fancyhead[R]{\small\textit{Lecture 10}}
\fancyfoot[C]{\thepage}
\renewcommand{\headrulewidth}{0.5pt}
\renewcommand{\footrulewidth}{0.3pt}

\fancypagestyle{firstpage}{
    \fancyhf{}
    \fancyfoot[C]{\thepage}
    \renewcommand{\headrulewidth}{0pt}
}

%========================================================================================
% Colors
%========================================================================================

\usepackage[dvipsnames]{xcolor}

\definecolor{lightblue}{RGB}{220, 235, 255}
\definecolor{lightgreen}{RGB}{220, 255, 235}
\definecolor{lightyellow}{RGB}{255, 250, 220}
\definecolor{lightpurple}{RGB}{240, 230, 255}
\definecolor{lightgray}{gray}{0.95}
\definecolor{lightpink}{RGB}{255, 235, 245}
\definecolor{boxgray}{gray}{0.95}
\definecolor{boxblue}{rgb}{0.9, 0.95, 1.0}
\definecolor{boxred}{rgb}{1.0, 0.95, 0.95}

\definecolor{darkblue}{RGB}{50, 80, 150}
\definecolor{darkgreen}{RGB}{40, 120, 70}
\definecolor{darkorange}{RGB}{200, 100, 30}
\definecolor{darkpurple}{RGB}{100, 60, 150}

%========================================================================================
% Box Environments (tcolorbox)
%========================================================================================

\usepackage[most]{tcolorbox}
\tcbuselibrary{skins, breakable}

\newtcolorbox{overviewbox}[1][]{
    enhanced,
    colback=lightpurple,
    colframe=darkpurple,
    fonttitle=\bfseries\large,
    title=Lecture Overview,
    arc=3mm,
    boxrule=1pt,
    left=8pt, right=8pt, top=8pt, bottom=8pt,
    breakable,
    #1
}

\newtcolorbox{summarybox}[1][]{
    enhanced,
    colback=lightblue,
    colframe=darkblue,
    fonttitle=\bfseries,
    title=Key Summary,
    arc=2mm,
    boxrule=0.7pt,
    left=6pt, right=6pt, top=6pt, bottom=6pt,
    breakable,
    #1
}

\newtcolorbox{infobox}[1][]{
    enhanced,
    colback=lightgreen,
    colframe=darkgreen,
    fonttitle=\bfseries,
    title=Key Information,
    arc=2mm,
    boxrule=0.7pt,
    left=6pt, right=6pt, top=6pt, bottom=6pt,
    breakable,
    #1
}

\newtcolorbox{warningbox}[1][]{
    enhanced,
    colback=lightyellow,
    colframe=darkorange,
    fonttitle=\bfseries,
    title=Warning,
    arc=2mm,
    boxrule=0.7pt,
    left=6pt, right=6pt, top=6pt, bottom=6pt,
    breakable,
    #1
}

\newtcolorbox{examplebox}[1][]{
    enhanced,
    colback=lightgray,
    colframe=black!60,
    fonttitle=\bfseries,
    title=Example: #1,
    arc=2mm,
    boxrule=0.7pt,
    left=6pt, right=6pt, top=6pt, bottom=6pt,
    breakable,
}

\newtcolorbox{definitionbox}[1][]{
    enhanced,
    colback=lightpink,
    colframe=purple!70!black,
    fonttitle=\bfseries,
    title=Definition: #1,
    arc=2mm,
    boxrule=0.7pt,
    left=6pt, right=6pt, top=6pt, bottom=6pt,
    breakable,
}

\newtcolorbox{importantbox}[1][]{
    enhanced,
    colback=boxred,
    colframe=red!70!black,
    fonttitle=\bfseries,
    title=Important: #1,
    arc=2mm,
    boxrule=0.7pt,
    left=6pt, right=6pt, top=6pt, bottom=6pt,
    breakable,
}

%========================================================================================
% Code Listings
%========================================================================================

\usepackage{listings}

\lstset{
    basicstyle=\ttfamily\small,
    backgroundcolor=\color{lightgray},
    keywordstyle=\color{darkblue}\bfseries,
    commentstyle=\color{darkgreen}\itshape,
    stringstyle=\color{purple!80!black},
    numberstyle=\tiny\color{black!60},
    numbers=left,
    numbersep=8pt,
    breaklines=true,
    breakatwhitespace=false,
    frame=single,
    frameround=tttt,
    rulecolor=\color{black!30},
    captionpos=b,
    showstringspaces=false,
    tabsize=2,
    xleftmargin=15pt,
    xrightmargin=5pt,
    escapeinside={\%*}{*)}
}

\lstdefinestyle{pythonstyle}{
    language=Python,
    morekeywords={self, True, False, None},
}

%========================================================================================
% Other Packages
%========================================================================================

\usepackage{tocloft}
\renewcommand{\cftsecleader}{\cftdotfill{\cftdotsep}}
\setlength{\cftbeforesecskip}{0.4em}
\renewcommand{\cftsecfont}{\bfseries}
\renewcommand{\cftsubsecfont}{\normalfont}

\usepackage{graphicx}
\usepackage{adjustbox}

\usepackage{caption}
\captionsetup[table]{labelfont=bf, textfont=it, skip=5pt}
\captionsetup[figure]{labelfont=bf, textfont=it, skip=5pt}

\usepackage{amsmath, amssymb, amsthm}

\theoremstyle{definition}
\newtheorem{theorem}{Theorem}[section]
\newtheorem{definition}{Definition}[section]
\newtheorem{example}{Example}[section]

\usepackage[
    colorlinks=true,
    linkcolor=blue!80!black,
    urlcolor=blue!80!black,
    citecolor=green!60!black,
    bookmarks=true,
    bookmarksnumbered=true,
    pdfborder={0 0 0}
]{hyperref}

\hypersetup{
    pdftitle={CSCI E-103: Model Bias and Fraud Detection - Lecture 10},
    pdfauthor={Lecture Notes},
    pdfsubject={Academic Notes}
}

\usepackage{enumitem}
\setlist{nosep, leftmargin=*, itemsep=0.3em}

\usepackage{microtype}
\usepackage{footnote}
\usepackage{url}
\urlstyle{same}

%========================================================================================
% Custom Commands
%========================================================================================

\newcommand{\important}[1]{\textbf{\textcolor{red!70!black}{#1}}}
\newcommand{\keyword}[1]{\textbf{#1}}
\newcommand{\term}[1]{\textit{#1}}
\newcommand{\code}[1]{\texttt{#1}}
\newcommand{\defterm}[2]{\textbf{#1}\footnote{#2}}
\newcommand{\newsection}[1]{\newpage\section{#1}}

%========================================================================================
% Title Style
%========================================================================================

\usepackage{titling}
\pretitle{\begin{center}\LARGE\bfseries}
\posttitle{\par\end{center}\vskip 0.5em}
\preauthor{\begin{center}\large}
\postauthor{\end{center}}
\predate{\begin{center}\large}
\postdate{\par\end{center}}

\usepackage{titlesec}
\titlespacing*{\section}{0pt}{1.5em}{0.8em}
\titlespacing*{\subsection}{0pt}{1.2em}{0.6em}
\titlespacing*{\subsubsection}{0pt}{1em}{0.5em}

%========================================================================================
% Meta Info Command
%========================================================================================

\newcommand{\metainfo}[4]{
\begin{tcolorbox}[
    colback=lightpurple,
    colframe=darkpurple,
    boxrule=1pt,
    arc=2mm,
    left=10pt, right=10pt, top=8pt, bottom=8pt
]
\begin{tabular}{@{}rl@{}}
$\blacksquare$ \textbf{Course:} & #1 \\[0.3em]
$\blacksquare$ \textbf{Week:} & #2 \\[0.3em]
$\blacksquare$ \textbf{Instructors:} & #3 \\[0.3em]
$\blacksquare$ \textbf{Objective:} & \begin{minipage}[t]{0.7\textwidth}#4\end{minipage}
\end{tabular}
\end{tcolorbox}
}

%========================================================================================
% Document
%========================================================================================

\title{Lecture 10: Model Bias, Data Imbalance, and Real-World Fraud Detection}
\author{CSCI E-103: Reproducible Data Science and Machine Learning}
\date{Harvard University}

\begin{document}

\maketitle
\thispagestyle{firstpage}

\metainfo{CSCI E-103: Reproducible Data Science}{Lecture 10}{Anindita Mahapatra \& Eric Gieseke}{Understand model errors, handle imbalanced data, and learn from a real-world fraud detection system at scale}

\tableofcontents

\newpage

%===================================================================================
\section{The Central Theme: Data Problems Lead to Model Problems}
%===================================================================================

\begin{overviewbox}
This lecture explores the critical relationship between data quality and model performance. Poor data doesn't just affect accuracy---it can create biased, unfair, and even harmful models.

\textbf{Key Topics:}
\begin{itemize}
    \item Machine Learning Errors: Bias, Variance, and Irreducible Error
    \item The Bias-Variance Tradeoff
    \item Model Bias: Why models can be unfair
    \item Class Imbalance: When your data is 99\% one thing
    \item SMOTE: Synthetic Minority Oversampling Technique
    \item AutoML: Blackbox vs. Glassbox approaches
    \item Data Classification for PII detection
    \item Real-World Case Study: Fraud Detection at 15ms latency
\end{itemize}
\end{overviewbox}

\newpage

%===================================================================================
\section{Machine Learning Errors: The Three Components}
%===================================================================================

Every ML model makes errors. Understanding \textbf{why} models err is crucial to improving them.

\begin{summarybox}[title={The Total Prediction Error}]
\textbf{Total Error = Bias$^2$ + Variance + Irreducible Error}
\end{summarybox}

\subsection{1. Bias (Underfitting)}

\begin{definitionbox}[Bias]
\textbf{Bias} is the error introduced by approximating a real-world problem (which may be extremely complicated) with a too-simple model.

\textbf{Symptom:} The model fails to capture the true relationship between features and target.

\textbf{Result:} Poor performance on \textbf{both} training AND test data.
\end{definitionbox}

\begin{examplebox}[High Bias Model]
Imagine you're trying to predict house prices, which depend on many factors in complex, non-linear ways.

If you use simple linear regression (a straight line), you might get:
\begin{itemize}
    \item Training accuracy: 60\%
    \item Test accuracy: 58\%
\end{itemize}

Both are bad! The model is too simple to capture reality.

\textbf{Analogy:} Trying to describe a curvy mountain road as ``mostly flat.''
\end{examplebox}

\subsection{2. Variance (Overfitting)}

\begin{definitionbox}[Variance]
\textbf{Variance} is the error introduced when a model is too sensitive to small fluctuations (noise) in the training data.

\textbf{Symptom:} The model memorizes the training data instead of learning general patterns.

\textbf{Result:} Excellent performance on training data, \textbf{terrible} performance on test data.
\end{definitionbox}

\begin{examplebox}[High Variance Model]
Using an extremely complex model (like a deep neural network with no regularization) on limited data:
\begin{itemize}
    \item Training accuracy: 99.5\%
    \item Test accuracy: 62\%
\end{itemize}

The model ``cheated'' by memorizing answers instead of learning rules.

\textbf{Analogy:} A student who memorizes all practice exam answers but can't solve new problems.
\end{examplebox}

\subsection{3. Irreducible Error}

\begin{definitionbox}[Irreducible Error]
\textbf{Irreducible Error} is the inherent noise in the data that cannot be reduced regardless of the model. It represents the randomness in real-world phenomena.

This is the floor---no model can be more accurate than the noise allows.
\end{definitionbox}

\subsection{The Bias-Variance Tradeoff}

\begin{importantbox}[The Fundamental Tradeoff]
Bias and variance are \textbf{inversely related}:

\begin{itemize}
    \item \textbf{Simple model} (fewer parameters) $\rightarrow$ High Bias, Low Variance
    \item \textbf{Complex model} (more parameters) $\rightarrow$ Low Bias, High Variance
\end{itemize}

The goal of ML is to find the \textbf{``sweet spot''}---a model complex enough to capture the true patterns (low bias) but not so complex that it fits noise (low variance).
\end{importantbox}

\begin{table}[htbp]
\centering
\caption{Bias vs. Variance Characteristics}
\begin{tabular}{@{}lll@{}}
\toprule
\textbf{Characteristic} & \textbf{High Bias (Underfitting)} & \textbf{High Variance (Overfitting)} \\
\midrule
Model complexity & Too simple & Too complex \\
Training error & High & Very low \\
Test error & High & High (much higher than training) \\
Cause & Not enough learning & Too much learning (noise included) \\
Fix & More complex model, more features & Regularization, more data, simpler model \\
\bottomrule
\end{tabular}
\end{table}

\newpage

%===================================================================================
\section{Model Bias: When AI Becomes Unfair}
%===================================================================================

This is a different kind of ``bias''---not underfitting, but \textbf{unfairness}. When models systematically disadvantage certain groups of people.

\subsection{Why Does Model Bias Happen?}

\subsubsection{1. Human Cognitive Bias in Data}

Models learn from data. If data reflects human prejudices, the model learns those prejudices.

\begin{examplebox}[Facial Recognition Accuracy Disparity]
A NIST study found that facial recognition systems from major companies (Microsoft, IBM, etc.) had dramatically different accuracy rates:

\begin{itemize}
    \item \textbf{Light-skinned males:} 90\%+ accuracy
    \item \textbf{Dark-skinned females:} 60-70\% accuracy
\end{itemize}

\textbf{Root Cause:} The training datasets were \textbf{under-represented} in dark-skinned faces, especially females. The model simply didn't have enough examples to learn from.

\textbf{Key Insight:} This isn't an algorithm problem---it's a \textbf{data collection} problem. The humans who created the dataset inadvertently (or sometimes systematically) included fewer examples of certain groups.
\end{examplebox}

\begin{examplebox}[Microsoft's Tay Chatbot]
In 2016, Microsoft released a chatbot named ``Tay'' that learned from Twitter conversations.

Within 24 hours, malicious users taught it to say racist, sexist, and Holocaust-denying statements.

\textbf{Lesson:} Models are only as good as their training data. Garbage in, garbage out---but magnified.
\end{examplebox}

\subsubsection{2. Poor Quality Training Data}

\begin{itemize}
    \item \textbf{Low resolution:} Images too blurry to distinguish (like the dog vs. arctic fox example)
    \item \textbf{Mislabeled data:} Human labelers may have their own biases (e.g., associating ``evil'' with dark colors)
    \item \textbf{Incomplete coverage:} Data missing important edge cases or scenarios
\end{itemize}

\subsection{How to Reduce Model Bias}

\begin{infobox}[title={Strategies for Reducing Bias}]
\begin{enumerate}
    \item \textbf{Ensure Representative Data (Most Important!):}
    \begin{itemize}
        \item Audit your training data for demographic representation
        \item Actively collect more data from underrepresented groups
        \item Consider the source of your data and its inherent biases
    \end{itemize}

    \item \textbf{Use Appropriate Models:}
    \begin{itemize}
        \item Don't use linear models for non-linear relationships
        \item Tree-based algorithms are often better at handling bias
    \end{itemize}

    \item \textbf{Apply Weighting or Penalized Models:}
    \begin{itemize}
        \item Give more weight to underrepresented groups
        \item Use class weights in your loss function
    \end{itemize}

    \item \textbf{Extensive Hyperparameter Tuning:}
    \begin{itemize}
        \item Don't just use defaults---optimize for fairness metrics
    \end{itemize}

    \item \textbf{Ensemble Methods (for reducing variance):}
    \begin{itemize}
        \item Combine weak and strong learners
        \item Random forests, boosting, etc.
    \end{itemize}
\end{enumerate}
\end{infobox}

\newpage

%===================================================================================
\section{Class Imbalance: The 99-1 Problem}
%===================================================================================

A special and extremely common case of data problems.

\begin{definitionbox}[Class Imbalance]
\textbf{Class Imbalance} occurs when one class (label) in your training data is much more frequent than another.

\textbf{Examples:}
\begin{itemize}
    \item Fraud detection: 99.9\% legitimate, 0.1\% fraud
    \item Cancer diagnosis: 98\% healthy, 2\% cancer
    \item Anomaly detection in manufacturing
    \item Network intrusion detection
\end{itemize}
\end{definitionbox}

\subsection{The Accuracy Trap}

Most ML algorithms optimize for \textbf{accuracy}---the percentage of correct predictions.

\begin{warningbox}[title={Why Accuracy is Misleading}]
Consider a fraud detection dataset with 99.9\% legitimate transactions.

A model that \textbf{always predicts ``legitimate''} will have:

\textbf{Accuracy = 99.9\%}

Sounds great, right? But this model catches \textbf{zero fraud}. It's completely useless for its intended purpose.

The lesson: \textbf{Accuracy is a terrible metric for imbalanced data.}
\end{warningbox}

\subsection{Better Metrics: Precision, Recall, and F1}

\begin{definitionbox}[Confusion Matrix Terminology]
\begin{itemize}
    \item \textbf{True Positive (TP):} Model predicted Fraud, actually was Fraud
    \item \textbf{True Negative (TN):} Model predicted Legitimate, actually was Legitimate
    \item \textbf{False Positive (FP):} Model predicted Fraud, but was actually Legitimate (Type I Error)
    \item \textbf{False Negative (FN):} Model predicted Legitimate, but was actually Fraud (Type II Error)
\end{itemize}
\end{definitionbox}

\begin{table}[htbp]
\centering
\caption{Key Metrics for Imbalanced Data}
\begin{tabular}{@{}p{0.15\textwidth}p{0.35\textwidth}p{0.45\textwidth}@{}}
\toprule
\textbf{Metric} & \textbf{Formula} & \textbf{When to Prioritize} \\
\midrule
\textbf{Precision}
& $\frac{TP}{TP + FP}$
& ``Of things I flagged as fraud, how many were actually fraud?''
\newline Prioritize when \textbf{false positives are costly} (blocking legitimate customers)
\\ \addlinespace

\textbf{Recall}
& $\frac{TP}{TP + FN}$
& ``Of all actual frauds, how many did I catch?''
\newline Prioritize when \textbf{false negatives are costly} (missing actual fraud/cancer)
\\ \addlinespace

\textbf{F1 Score}
& $2 \times \frac{Precision \times Recall}{Precision + Recall}$
& Harmonic mean of precision and recall.
\newline Use when \textbf{both matter}---the primary metric for imbalanced data
\\
\bottomrule
\end{tabular}
\end{table}

\begin{examplebox}[Cancer Screening]
In cancer diagnosis, \textbf{recall is critical}. Missing an actual cancer case (false negative) can be fatal.

A model with:
\begin{itemize}
    \item Precision: 90\% (some false alarms)
    \item Recall: 99\% (catches almost all cancers)
\end{itemize}

is far better than:
\begin{itemize}
    \item Precision: 99\% (rarely wrong when it says cancer)
    \item Recall: 70\% (misses 30\% of cancers!)
\end{itemize}
\end{examplebox}

\subsection{Resampling Techniques}

\begin{table}[htbp]
\centering
\caption{Resampling Strategies}
\begin{tabular}{@{}p{0.2\textwidth}p{0.4\textwidth}p{0.35\textwidth}@{}}
\toprule
\textbf{Technique} & \textbf{How It Works} & \textbf{Pros/Cons} \\
\midrule
\textbf{Undersampling}
& Remove samples from the majority class until balanced
& Simple, but \textbf{loses valuable data}
\\ \addlinespace

\textbf{Oversampling}
& Duplicate samples from the minority class
& Simple, but can cause \textbf{overfitting} (memorizing duplicates)
\\ \addlinespace

\textbf{SMOTE}
& Generate \textbf{synthetic} minority samples
& Best of both worlds---creates new data points
\\
\bottomrule
\end{tabular}
\end{table}

\subsection{SMOTE: Synthetic Minority Oversampling Technique}

\begin{infobox}[title={How SMOTE Works}]
SMOTE doesn't just copy existing minority samples---it \textbf{creates new synthetic ones}.

\textbf{Algorithm:}
\begin{enumerate}
    \item Pick a minority class sample $x_i$
    \item Find its $k$ nearest neighbors (also minority class)
    \item Randomly select one neighbor $x_j$
    \item Create a new synthetic point \textbf{along the line} between $x_i$ and $x_j$:
    \[
    x_{new} = x_i + \lambda \cdot (x_j - x_i), \quad \text{where } \lambda \in [0, 1]
    \]
\end{enumerate}

This effectively ``fills in'' the feature space around minority samples, giving the model a richer understanding of the minority class.
\end{infobox}

\begin{lstlisting}[style=pythonstyle, caption={Using SMOTE in Python}]
from imblearn.over_sampling import SMOTE
from sklearn.model_selection import train_test_split

# CRITICAL: Split BEFORE applying SMOTE!
X_train, X_test, y_train, y_test = train_test_split(X, y, test_size=0.2)

# Apply SMOTE only to training data
smote = SMOTE(random_state=42)
X_train_resampled, y_train_resampled = smote.fit_resample(X_train, y_train)

# Now train your model
model.fit(X_train_resampled, y_train_resampled)

# Evaluate on original (non-resampled) test data
model.score(X_test, y_test)
\end{lstlisting}

\begin{warningbox}[title={Critical SMOTE Warning}]
\textbf{NEVER apply SMOTE before train/test split!}

If you SMOTE the entire dataset first, then split, synthetic data may leak between train and test sets, causing artificially inflated test performance (data leakage).

\textbf{Correct order:}
\begin{enumerate}
    \item Split data into train/test
    \item Apply SMOTE to training set only
    \item Evaluate on original (un-SMOTEd) test set
\end{enumerate}
\end{warningbox}

\newpage

%===================================================================================
\section{AutoML: Blackbox vs. Glassbox}
%===================================================================================

\begin{definitionbox}[AutoML]
\textbf{AutoML (Automated Machine Learning)} automates the process of:
\begin{itemize}
    \item Feature engineering
    \item Model selection
    \item Hyperparameter tuning
    \item Model evaluation
\end{itemize}

You provide data, it provides a trained model.
\end{definitionbox}

\subsection{Blackbox AutoML}

\begin{itemize}
    \item \textbf{How it works:} Upload data, click ``train,'' get a model
    \item \textbf{Examples:} DataRobot, some commercial tools
    \item \textbf{Problem:} You can't see how the model was built. When things go wrong (and they will), you can't diagnose or fix them. For enterprise use cases requiring auditability, this is a dealbreaker.
\end{itemize}

\subsection{Glassbox AutoML (Databricks Approach)}

\begin{itemize}
    \item \textbf{How it works:} Same automation, but every step is exposed as a \textbf{notebook}
    \item \textbf{What you get:}
    \begin{enumerate}
        \item \textbf{Data Exploration Notebook:} Automatic EDA with profiling
        \item \textbf{Best Model Notebook:} Full source code for the winning model
        \item \textbf{MLflow Integration:} All experiments tracked
        \item \textbf{SHAP Explanations:} Feature importance built in
    \end{enumerate}
    \item \textbf{Advantage:} ``Citizen data scientists'' can start quickly, but experts can take over and customize
\end{itemize}

\begin{table}[htbp]
\centering
\caption{Models Available in Databricks AutoML}
\begin{tabular}{@{}lll@{}}
\toprule
\textbf{Classification} & \textbf{Regression} & \textbf{Forecasting} \\
\midrule
Decision Trees & Decision Trees & Prophet \\
Random Forests & Random Forests & Auto-ARIMA \\
Logistic Regression & Linear Regression & DeepAR \\
XGBoost & XGBoost & \\
LightGBM & LightGBM & \\
\bottomrule
\end{tabular}
\end{table}

\begin{examplebox}[AutoML Demo Walkthrough]
\textbf{Scenario:} Customer churn prediction

\textbf{Steps:}
\begin{enumerate}
    \item Select dataset (churn table from Unity Catalog)
    \item Choose target column (``Churn'')
    \item Set evaluation metric (F1 Score---because churn is imbalanced!)
    \item Set timeout (15 minutes)
    \item Click ``Start AutoML''
\end{enumerate}

\textbf{Results:}
\begin{itemize}
    \item AutoML runs Decision Trees, Random Forest, XGBoost, LightGBM in parallel
    \item \textbf{Best model:} LightGBM (based on F1 score)
    \item Click ``View notebook for best model'' to see full source code
    \item All experiments logged to MLflow
    \item SHAP analysis shows which features matter most
\end{itemize}
\end{examplebox}

\newpage

%===================================================================================
\section{Data Classification: Protecting Sensitive Information}
%===================================================================================

Models can inadvertently leak sensitive information (PII). If your training data contains social security numbers and someone asks the model about them...

\begin{definitionbox}[PII - Personally Identifiable Information]
Information that can identify an individual:
\begin{itemize}
    \item Name, phone number, email address
    \item Social Security Number, driver's license
    \item IP address, location data
    \item Bank account numbers
\end{itemize}

PII must be detected and removed before model training.
\end{definitionbox}

\subsection{Automated Data Classification in Databricks}

Unity Catalog can automatically scan tables and detect sensitive columns:

\begin{enumerate}
    \item Enable ``Data Classification'' at the catalog level
    \item System scans all tables (takes about 15 minutes)
    \item Results show which columns contain: Name, Phone, Email, SSN, IP Address, etc.
    \item Tags are automatically applied for governance
\end{enumerate}

This prevents accidentally feeding PII into models, which could create both legal liability and privacy leaks.

\newpage

%===================================================================================
\section{Real-World Case Study: Fraud Detection at Scale}
%===================================================================================

Eric Gieseke shares a real production fraud detection system he built. This is where all the theory meets hard engineering constraints.

\subsection{Business Requirements}

\begin{table}[htbp]
\centering
\caption{Fraud Detection System Requirements}
\begin{tabular}{@{}ll@{}}
\toprule
\textbf{Requirement} & \textbf{Target} \\
\midrule
\textbf{Accuracy} & High confidence---catch real fraud, don't block legitimate transactions \\
\textbf{Latency} & \textbf{15 milliseconds} (ms) response time \\
\textbf{Throughput} & 50,000+ transactions per second (TPS) \\
\textbf{Maintainability} & Easy to add/modify features and rules \\
\bottomrule
\end{tabular}
\end{table}

\begin{examplebox}[Why 15ms is Incredibly Hard]
For context:
\begin{itemize}
    \item A typical hard disk seek time: 10-15ms
    \item A network round-trip: 1-100ms depending on distance
    \item Human blink: 100-400ms
\end{itemize}

The entire fraud decision---load customer history, compute features, run ML model, apply rules, return decision---must happen in the time it takes a hard disk to \textbf{find} data, let alone read it.
\end{examplebox}

\subsection{Features for Fraud Detection}

The data science team developed hundreds of features:

\begin{table}[htbp]
\centering
\caption{Example Fraud Detection Features}
\begin{tabular}{@{}ll@{}}
\toprule
\textbf{Feature Type} & \textbf{Examples} \\
\midrule
\textbf{Transaction-based}
& Distance from customer's home address \\
& Difference from average transaction amount \\
& Number of transactions today \\
& Time since last transaction \\
\midrule
\textbf{Dimension-based}
& Merchant's average transaction amount \\
& Customer's transaction frequency \\
& Terminal's fraud history \\
& Geographic region risk score \\
\bottomrule
\end{tabular}
\end{table}

\subsection{Architecture: Lambda Architecture}

To meet both 15ms real-time AND large-scale batch requirements, they used \textbf{Lambda Architecture}.

\begin{infobox}[title={Lambda Architecture Overview}]
Lambda Architecture splits data processing into three layers:

\textbf{1. Batch Layer (Slow but Complete):}
\begin{itemize}
    \item Stores all historical data (immutable)
    \item Runs nightly/hourly batch jobs (Hadoop/Spark)
    \item Computes features on entire dataset (e.g., ``customer's average transaction over 1 year'')
    \item Results stored in Feature Store
\end{itemize}

\textbf{2. Speed Layer (Fast but Incremental):}
\begin{itemize}
    \item Processes real-time events as they arrive
    \item Uses Complex Event Processing (CEP)
    \item Computes real-time features (e.g., ``transactions in last 10 minutes'')
\end{itemize}

\textbf{3. Serving Layer (Query Layer):}
\begin{itemize}
    \item Combines batch and speed results
    \item Serves queries with low latency
    \item Cassandra used for this layer
\end{itemize}
\end{infobox}

\subsection{Key Innovation 1: Metadata-Driven Code Generation}

\begin{itemize}
    \item \textbf{Problem:} Hundreds of features need code for both batch (SQL) AND real-time (CEP language). Writing and maintaining both is error-prone.

    \item \textbf{Solution:} Define features as \textbf{metadata}, not code.
    \begin{lstlisting}[basicstyle=\ttfamily\small]
Feature: customer_avg_transaction_30d
Type: Average
Field: transaction_amount
Window: 30 days
Dimension: customer_id
    \end{lstlisting}

    \item \textbf{Code Generator} reads metadata and automatically produces:
    \begin{itemize}
        \item SQL for batch processing (Spark)
        \item EPL for real-time processing (CEP)
    \end{itemize}

    \item \textbf{Benefit:} Add a new feature by editing metadata, not writing two sets of code
\end{itemize}

\subsection{Key Innovation 2: Circular Buffer (Ring Buffer)}

\begin{examplebox}[The Problem]
To compute ``customer's average transaction amount over last 7 days,'' you'd normally:
\begin{enumerate}
    \item Query database for all transactions in last 7 days
    \item Sum them up
    \item Divide by count
\end{enumerate}

For millions of customers, this is \textbf{impossible in 15ms}.
\end{examplebox}

\begin{infobox}[title={Circular Buffer Solution}]
Instead of storing individual transactions, store \textbf{aggregates per time bucket}.

\textbf{Structure (7-day buffer):}
\begin{center}
\begin{tabular}{|c|c|c|c|c|c|c|}
\hline
Sun & Mon & Tue & Wed & Thu & Fri & Sat \\
\hline
(count, sum) & (count, sum) & (count, sum) & (count, sum) & (count, sum) & (count, sum) & (count, sum) \\
\hline
\end{tabular}
\end{center}

\textbf{Example:}
\begin{lstlisting}[basicstyle=\ttfamily\small]
Sun: (3, $150)   Mon: (5, $280)   Tue: (2, $90)   ...
\end{lstlisting}

\textbf{To compute 7-day average:}
\[
\text{Average} = \frac{\sum \text{sums}}{\sum \text{counts}} = \frac{150 + 280 + 90 + ...}{3 + 5 + 2 + ...}
\]

\textbf{Memory footprint:} Just 14 numbers per customer per feature (2 numbers × 7 days)

\textbf{When day changes:} Overwrite oldest bucket with zeros, start fresh

This was novel enough that the company \textbf{patented} it.
\end{infobox}

\subsection{Key Innovation 3: Cassandra as Feature Store}

\begin{itemize}
    \item \textbf{Why Cassandra?}
    \begin{itemize}
        \item Write speed: Extremely fast (writes are ``fire and forget'')
        \item \textbf{Read speed: 2.3ms achieved}---critical for 15ms budget
        \item Scalable to petabytes
        \item Supports wide rows (billions of columns per row)
    \end{itemize}

    \item \textbf{Data Model:}
    \begin{itemize}
        \item Only 2 tables: \textbf{Events} (fact table) and \textbf{Dimensions}
        \item New features = new columns (schema-less flexibility)
        \item Sparse data handled efficiently
    \end{itemize}
\end{itemize}

\subsection{System Flow Summary}

\begin{enumerate}
    \item \textbf{Fraud Analyst} defines new feature in Metadata Service
    \item \textbf{Batch Processing} (nightly) computes historical feature values $\rightarrow$ stores in Cassandra
    \item \textbf{Code Generator} creates real-time CEP code
    \item \textbf{Customer} swipes card at merchant
    \item \textbf{Payment Service} calls Fraud Detection Service
    \item \textbf{Fraud Detection} (within 15ms):
    \begin{enumerate}
        \item Retrieves pre-computed features from Cassandra (2.3ms)
        \item Computes real-time features using Circular Buffers (in-memory)
        \item Runs ML model + rules
        \item Returns ``Approve'' or ``Decline''
    \end{enumerate}
    \item \textbf{Customer} completes purchase (or doesn't)
\end{enumerate}

\subsection{Disaster Recovery}

For mission-critical payment systems:
\begin{itemize}
    \item Two data centers (US and Europe)
    \item If US fails, traffic automatically routes to Europe
    \item Slightly higher latency acceptable vs. complete outage
    \item Data replication between centers
\end{itemize}

\newpage

%===================================================================================
\section{Summary: One-Page Quick Reference}
%===================================================================================

\begin{tcolorbox}[
  title={ML Errors: Bias vs. Variance},
  colframe=blue!75!black, colback=blue!5!white, fonttitle=\bfseries
]
\begin{itemize}
    \item \textbf{Bias (Underfitting):} Model too simple. Both train and test error high.
    \item \textbf{Variance (Overfitting):} Model too complex. Train error low, test error high.
    \item \textbf{Tradeoff:} Simple $\leftrightarrow$ Complex. Find the sweet spot.
\end{itemize}
\end{tcolorbox}

\begin{tcolorbox}[
  title={Model Bias (Fairness)},
  colframe=red!60!black, colback=red!5!white, fonttitle=\bfseries
]
\begin{itemize}
    \item \textbf{Cause 1:} Human bias in data (under-representation of groups)
    \item \textbf{Cause 2:} Poor quality labels, mislabeled data
    \item \textbf{Fix:} Ensure representative, diverse training data (most important!)
\end{itemize}
\end{tcolorbox}

\begin{tcolorbox}[
  title={Class Imbalance (99-1 Problem)},
  colframe=purple!60!black, colback=purple!5!white, fonttitle=\bfseries
]
\begin{itemize}
    \item \textbf{Problem:} Accuracy is meaningless (99\% by always predicting majority)
    \item \textbf{Metrics:} Use \textbf{F1 Score} (harmonic mean of Precision and Recall)
    \item \textbf{Fix:} \textbf{SMOTE} (create synthetic minority samples)
    \item \textbf{Warning:} Apply SMOTE to training set \textbf{only}, after train/test split!
\end{itemize}
\end{tcolorbox}

\begin{tcolorbox}[
  title={AutoML: Blackbox vs. Glassbox},
  colframe=green!60!black, colback=green!5!white, fonttitle=\bfseries
]
\begin{itemize}
    \item \textbf{Blackbox:} Magic model, no visibility (enterprise unfriendly)
    \item \textbf{Glassbox:} Full notebooks, MLflow tracking, SHAP explanations
    \item \textbf{Use case:} Quick baseline, data validation, citizen data scientists
\end{itemize}
\end{tcolorbox}

\begin{tcolorbox}[
  title={Fraud Detection at 15ms},
  colframe=orange!60!black, colback=orange!5!white, fonttitle=\bfseries
]
\begin{itemize}
    \item \textbf{Architecture:} Lambda (Batch + Speed + Serving layers)
    \item \textbf{Key 1:} \textbf{Circular Buffer} - Store (count, sum) per time bucket, not individual records
    \item \textbf{Key 2:} \textbf{Cassandra} - 2.3ms reads for feature retrieval
    \item \textbf{Key 3:} \textbf{Code Generation} - Define features as metadata, auto-generate SQL/CEP
    \item \textbf{Lesson:} The difference between 15ms and 100ms is the difference between possible and impossible
\end{itemize}
\end{tcolorbox}

\end{document}
