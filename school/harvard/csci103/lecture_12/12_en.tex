%%%%%%%%%%%%%%%%%%%%%%%%%%%%%%%%%%%%%%%%%%%%%%%%%%%%%%%%%%%%%%%%%%%%%%%%%%%%%%%
% Harvard Academic Notes - English Master Template
% CSCI E-103: Reproducible Machine Learning - Lecture 12
% Topic: Data and AI Governance
% Version: 2.1 - English Edition
%%%%%%%%%%%%%%%%%%%%%%%%%%%%%%%%%%%%%%%%%%%%%%%%%%%%%%%%%%%%%%%%%%%%%%%%%%%%%%%

\documentclass[11pt,a4paper]{article}

%========================================================================================
% Basic Packages
%========================================================================================

% --- Page Layout ---
\usepackage[top=20mm, bottom=20mm, left=20mm, right=18mm]{geometry}
\usepackage{setspace}
\onehalfspacing
\setlength{\parskip}{0.5em}
\setlength{\parindent}{0pt}

% --- Table Packages ---
\usepackage{booktabs}
\usepackage{tabularx}
\usepackage{array}
\usepackage{longtable}
\renewcommand{\arraystretch}{1.1}

%========================================================================================
% Header and Footer
%========================================================================================

\usepackage{fancyhdr}
\pagestyle{fancy}
\fancyhf{}
\fancyhead[L]{\small\textit{CSCI E-103: Reproducible Machine Learning}}
\fancyhead[R]{\small\textit{Lecture 12}}
\fancyfoot[C]{\thepage}
\renewcommand{\headrulewidth}{0.5pt}
\renewcommand{\footrulewidth}{0.3pt}

\fancypagestyle{firstpage}{
    \fancyhf{}
    \fancyfoot[C]{\thepage}
    \renewcommand{\headrulewidth}{0pt}
}

%========================================================================================
% Color Definitions
%========================================================================================

\usepackage[dvipsnames]{xcolor}

\definecolor{lightblue}{RGB}{220, 235, 255}
\definecolor{lightgreen}{RGB}{220, 255, 235}
\definecolor{lightyellow}{RGB}{255, 250, 220}
\definecolor{lightpurple}{RGB}{240, 230, 255}
\definecolor{lightgray}{gray}{0.95}
\definecolor{lightpink}{RGB}{255, 235, 245}
\definecolor{boxgray}{gray}{0.95}
\definecolor{boxblue}{rgb}{0.9, 0.95, 1.0}
\definecolor{boxred}{rgb}{1.0, 0.95, 0.95}

\definecolor{darkblue}{RGB}{50, 80, 150}
\definecolor{darkgreen}{RGB}{40, 120, 70}
\definecolor{darkorange}{RGB}{200, 100, 30}
\definecolor{darkpurple}{RGB}{100, 60, 150}

%========================================================================================
% Box Environments (tcolorbox)
%========================================================================================

\usepackage[most]{tcolorbox}
\tcbuselibrary{skins, breakable}

\newtcolorbox{overviewbox}[1][]{
    enhanced,
    colback=lightpurple,
    colframe=darkpurple,
    fonttitle=\bfseries\large,
    title=Lecture Overview,
    arc=3mm,
    boxrule=1pt,
    left=8pt, right=8pt, top=8pt, bottom=8pt,
    breakable,
    #1
}

\newtcolorbox{summarybox}[1][]{
    enhanced,
    colback=lightblue,
    colframe=darkblue,
    fonttitle=\bfseries,
    title=Key Summary,
    arc=2mm,
    boxrule=0.7pt,
    left=6pt, right=6pt, top=6pt, bottom=6pt,
    breakable,
    #1
}

\newtcolorbox{infobox}[1][]{
    enhanced,
    colback=lightgreen,
    colframe=darkgreen,
    fonttitle=\bfseries,
    title=Key Information,
    arc=2mm,
    boxrule=0.7pt,
    left=6pt, right=6pt, top=6pt, bottom=6pt,
    breakable,
    #1
}

\newtcolorbox{warningbox}[1][]{
    enhanced,
    colback=lightyellow,
    colframe=darkorange,
    fonttitle=\bfseries,
    title=Warning,
    arc=2mm,
    boxrule=0.7pt,
    left=6pt, right=6pt, top=6pt, bottom=6pt,
    breakable,
    #1
}

\newtcolorbox{examplebox}[1][]{
    enhanced,
    colback=lightgray,
    colframe=black!60,
    fonttitle=\bfseries,
    title=Example: #1,
    arc=2mm,
    boxrule=0.7pt,
    left=6pt, right=6pt, top=6pt, bottom=6pt,
    breakable,
}

\newtcolorbox{definitionbox}[1][]{
    enhanced,
    colback=lightpink,
    colframe=purple!70!black,
    fonttitle=\bfseries,
    title=Definition: #1,
    arc=2mm,
    boxrule=0.7pt,
    left=6pt, right=6pt, top=6pt, bottom=6pt,
    breakable,
}

\newtcolorbox{importantbox}[1][]{
    enhanced,
    colback=boxred,
    colframe=red!70!black,
    fonttitle=\bfseries,
    title=Important: #1,
    arc=2mm,
    boxrule=0.7pt,
    left=6pt, right=6pt, top=6pt, bottom=6pt,
    breakable,
}

\let\cautionbox\warningbox
\let\endcautionbox\endwarningbox

%========================================================================================
% Code Block Settings
%========================================================================================

\usepackage{listings}

\definecolor{codegray}{rgb}{0.5,0.5,0.5}
\definecolor{codepurple}{rgb}{0.58,0,0.82}
\definecolor{backcolour}{rgb}{0.95,0.95,0.95}

\lstset{
    basicstyle=\ttfamily\small,
    backgroundcolor=\color{lightgray},
    keywordstyle=\color{darkblue}\bfseries,
    commentstyle=\color{darkgreen}\itshape,
    stringstyle=\color{purple!80!black},
    numberstyle=\tiny\color{black!60},
    numbers=left,
    numbersep=8pt,
    breaklines=true,
    breakatwhitespace=false,
    frame=single,
    frameround=tttt,
    rulecolor=\color{black!30},
    captionpos=b,
    showstringspaces=false,
    tabsize=2,
    xleftmargin=15pt,
    xrightmargin=5pt,
    escapeinside={\%*}{*)}
}

\lstdefinestyle{pythonstyle}{
    language=Python,
    morekeywords={self, True, False, None},
}

\lstdefinestyle{sqlstyle}{
    language=SQL,
    morekeywords={SELECT, FROM, WHERE, JOIN, GROUP, BY, ORDER, HAVING, CREATE, FUNCTION, RETURN, ALTER, SET, MASK, GRANT, REVOKE},
}

%========================================================================================
% Table of Contents Styling
%========================================================================================

\usepackage{tocloft}
\renewcommand{\cftsecleader}{\cftdotfill{\cftdotsep}}
\setlength{\cftbeforesecskip}{0.4em}
\renewcommand{\cftsecfont}{\bfseries}
\renewcommand{\cftsubsecfont}{\normalfont}

%========================================================================================
% Figures and Tables
%========================================================================================

\usepackage{graphicx}
\usepackage{adjustbox}

\usepackage{caption}
\captionsetup[table]{labelfont=bf, textfont=it, skip=5pt}
\captionsetup[figure]{labelfont=bf, textfont=it, skip=5pt}

%========================================================================================
% Mathematics
%========================================================================================

\usepackage{amsmath, amssymb, amsthm}

\theoremstyle{definition}
\newtheorem{theorem}{Theorem}[section]
\newtheorem{lemma}[theorem]{Lemma}
\newtheorem{proposition}[theorem]{Proposition}
\newtheorem{corollary}[theorem]{Corollary}
\newtheorem{definition}{Definition}[section]
\newtheorem{example}{Example}[section]

%========================================================================================
% Hyperlinks
%========================================================================================

\usepackage[
    colorlinks=true,
    linkcolor=blue!80!black,
    urlcolor=blue!80!black,
    citecolor=green!60!black,
    bookmarks=true,
    bookmarksnumbered=true,
    pdfborder={0 0 0}
]{hyperref}

\hypersetup{
    pdftitle={CSCI E-103: Reproducible ML - Lecture 12: Data and AI Governance},
    pdfauthor={Lecture Notes},
    pdfsubject={Academic Notes}
}

%========================================================================================
% Other Packages
%========================================================================================

\usepackage{enumitem}
\setlist{nosep, leftmargin=*, itemsep=0.3em}

\usepackage{microtype}
\usepackage{footnote}
\usepackage{url}
\urlstyle{same}

%========================================================================================
% Custom Commands
%========================================================================================

\newcommand{\important}[1]{\textbf{\textcolor{red!70!black}{#1}}}
\newcommand{\keyword}[1]{\textbf{#1}}
\newcommand{\term}[1]{\textit{#1}}
\newcommand{\code}[1]{\texttt{#1}}
\newcommand{\defterm}[2]{\textbf{#1}\footnote{#2}}
\newcommand{\newsection}[1]{\newpage\section{#1}}

%========================================================================================
% Title Styling
%========================================================================================

\usepackage{titling}
\pretitle{\begin{center}\LARGE\bfseries}
\posttitle{\par\end{center}\vskip 0.5em}
\preauthor{\begin{center}\large}
\postauthor{\end{center}}
\predate{\begin{center}\large}
\postdate{\par\end{center}}

%========================================================================================
% Section Spacing
%========================================================================================

\usepackage{titlesec}
\titlespacing*{\section}{0pt}{1.5em}{0.8em}
\titlespacing*{\subsection}{0pt}{1.2em}{0.6em}
\titlespacing*{\subsubsection}{0pt}{1em}{0.5em}

%========================================================================================
% Meta Information Box
%========================================================================================

\newcommand{\metainfo}[4]{
\begin{tcolorbox}[
    colback=lightpurple,
    colframe=darkpurple,
    boxrule=1pt,
    arc=2mm,
    left=10pt, right=10pt, top=8pt, bottom=8pt
]
\begin{tabular}{@{}rl@{}}
$\blacksquare$ \textbf{Course:} & #1 \\[0.3em]
$\blacksquare$ \textbf{Week:} & #2 \\[0.3em]
$\blacksquare$ \textbf{Instructors:} & #3 \\[0.3em]
$\blacksquare$ \textbf{Objective:} & \begin{minipage}[t]{0.72\textwidth}#4\end{minipage}
\end{tabular}
\end{tcolorbox}
}

%========================================================================================
% Document
%========================================================================================

\begin{document}

\metainfo{CSCI E-103: Reproducible Machine Learning}{Lecture 12}{Ram Sriharsha (Guest Speaker)}{Master the fundamentals of Data and AI Governance, including information governance frameworks, Unity Catalog implementation, and attribute-based access control (ABAC)}

\tableofcontents
\newpage

%===============================================================================
% SECTION 1: Introduction
%===============================================================================
\section{Introduction: Why Governance Matters}

This lecture covers \textbf{Data and AI Governance}---the frameworks, policies, and technologies that enable organizations to maximize the value of their data assets while minimizing security and compliance risks.

\begin{overviewbox}
\textbf{Key Learning Objectives:}
\begin{itemize}
    \item \textbf{Understand} the hierarchy: Information Governance $\supset$ Data Governance $\supset$ AI Governance
    \item \textbf{Learn} the four pillars: Policies, Procedures, Standards, and Controls
    \item \textbf{Implement} governance using Databricks Unity Catalog
    \item \textbf{Apply} Attribute-Based Access Control (ABAC) for fine-grained security
    \item \textbf{Monitor} data quality and track data lineage automatically
\end{itemize}
\end{overviewbox}

\begin{infobox}
\textbf{Why Should You Care?}

"Data is the new oil"---but unlike oil, data is \textbf{renewable and limitless}. Organizations constantly produce, collect, and process data. The challenge is:
\begin{itemize}
    \item \textbf{Maximize Value:} Make data accessible to drive business decisions
    \item \textbf{Minimize Risk:} Protect sensitive data from breaches and regulatory violations
\end{itemize}

These two goals are in constant tension. Governance is the art of balancing them.
\end{infobox}

%===============================================================================
% SECTION 2: Information Governance Overview
%===============================================================================
\section{Information Governance: The Big Picture}

\subsection{What Is Information Governance?}

\begin{definitionbox}{Information Governance}
Information Governance is the overarching framework that encompasses \textbf{all} organizational information---physical documents, digital files, knowledge assets, and AI models.

\textbf{Scope:} Anything that is created, collected, stored, processed, or shared.

\textbf{Examples:}
\begin{itemize}
    \item Locking your laptop when leaving your desk
    \item Shredding printed documents with sensitive information
    \item Policies about who can access which email folders
\end{itemize}
\end{definitionbox}

\subsection{The Governance Hierarchy}

Information Governance contains two major subsets:

\begin{center}
\begin{tabular}{|l|p{10cm}|}
\hline
\textbf{Type} & \textbf{Focus Area} \\
\hline
\textbf{Data Governance} & Managing digital data: quality, security, lifecycle, access control \\
\hline
\textbf{AI Governance} & Managing AI models: bias prevention, explainability, ethical use, training data lineage \\
\hline
\end{tabular}
\end{center}

\begin{warningbox}
\textbf{You Can't Walk Before You Run:}

AI Governance is \textbf{built on top of} Data Governance. If your data is messy, inconsistent, or poorly secured, you cannot build fair, transparent, and safe AI systems.

\textbf{Example:} If you don't know the lineage of your training data, how can you verify the model isn't trained on copyrighted or biased content?
\end{warningbox}

\subsection{Goals of Information Governance}

\begin{enumerate}
    \item \textbf{Maximize Information Value}
    \begin{itemize}
        \item Make data discoverable and accessible to authorized users
        \item Enable self-service analytics and AI development
    \end{itemize}

    \item \textbf{Mitigate Risk}
    \begin{itemize}
        \item Prevent data breaches (reputational and financial damage)
        \item Ensure regulatory compliance (GDPR, HIPAA, PCI-DSS, CCPA)
    \end{itemize}

    \item \textbf{Enhance Security}
    \begin{itemize}
        \item Implement multi-factor authentication
        \item Encrypt data at rest and in transit
    \end{itemize}

    \item \textbf{Optimize Lifecycle Management}
    \begin{itemize}
        \item Define data retention policies
        \item Archive or delete data when no longer needed
    \end{itemize}

    \item \textbf{Promote Transparency and Accountability}
    \begin{itemize}
        \item Clear roles: Data Stewards, Governance Officers, Catalog Owners
        \item Audit trails for all data access and modifications
    \end{itemize}
\end{enumerate}

%===============================================================================
% SECTION 3: The Four Pillars of Governance
%===============================================================================
\section{The Four Pillars: Policies, Procedures, Standards, Controls}

\subsection{Framework Overview}

\begin{table}[h!]
\centering
\begin{adjustbox}{max width=\textwidth}
\begin{tabular}{|l|p{5cm}|p{6cm}|}
\hline
\textbf{Pillar} & \textbf{Definition} & \textbf{Example (GDPR)} \\
\hline
\textbf{Policies} & High-level rules established by leadership or regulators & "Individuals have the right to request deletion of their personal data" \\
\hline
\textbf{Procedures} & Step-by-step instructions for implementing policies & "Process data deletion requests within 45 days" \\
\hline
\textbf{Standards} & Technical specifications and best practices & "Use AES-256 encryption; minimize data storage" \\
\hline
\textbf{Controls} & Mechanisms that enforce policies & "Multi-factor authentication; access logs; role-based permissions" \\
\hline
\end{tabular}
\end{adjustbox}
\caption{The Four Pillars of Governance}
\end{table}

\subsection{How They Work Together}

\begin{examplebox}{GDPR Compliance Flow}
\textbf{Policy:} GDPR mandates the "Right to Be Forgotten"

\textbf{Procedure:}
\begin{enumerate}
    \item User submits deletion request
    \item Request logged in ticketing system
    \item Data team identifies all user data across systems
    \item Data deleted or anonymized within 30 days
    \item Confirmation sent to user
\end{enumerate}

\textbf{Standards:}
\begin{itemize}
    \item Data must be encrypted at rest
    \item Access controlled via Role-Based Access Control (RBAC)
    \item Minimum data retention periods defined
\end{itemize}

\textbf{Controls:}
\begin{itemize}
    \item Technical: Encryption, access control lists
    \item Administrative: Background checks for data handlers
    \item Physical: Secure data centers
\end{itemize}
\end{examplebox}

%===============================================================================
% SECTION 4: Data Governance vs AI Governance
%===============================================================================
\section{Data Governance vs AI Governance}

\subsection{Data Governance}

\begin{definitionbox}{Data Governance}
Data Governance focuses on managing \textbf{digital data assets}---their quality, security, lifecycle, and accessibility.

\textbf{Key Questions:}
\begin{itemize}
    \item Who can access this dataset?
    \item Is this data encrypted?
    \item How long should we retain this data?
    \item Is the data accurate and up-to-date?
\end{itemize}
\end{definitionbox}

\subsection{AI Governance}

\begin{definitionbox}{AI Governance}
AI Governance extends data governance to \textbf{AI models and their outputs}.

\textbf{Key Questions:}
\begin{itemize}
    \item Is the training data free of bias?
    \item Can the model's decisions be explained?
    \item Was the training data legally obtained?
    \item Does the model produce fair outcomes across demographic groups?
\end{itemize}
\end{definitionbox}

\begin{importantbox}{AI Governance is Critical Now}
With the rise of LLMs trained on internet-scale data:
\begin{itemize}
    \item \textbf{Copyright concerns:} Was copyrighted material used for training?
    \item \textbf{Privacy violations:} Does the model memorize PII from training data?
    \item \textbf{Indemnification:} If you use an LLM to generate content that infringes IP, who is liable?
\end{itemize}

Organizations are increasingly asking: "What is our legal exposure when using third-party AI models?"
\end{importantbox}

%===============================================================================
% SECTION 5: Governance Maturity Model
%===============================================================================
\section{Governance Maturity Model}

Organizations progress through maturity levels. \textbf{You cannot skip levels}---each stage builds on the previous.

\begin{table}[h!]
\centering
\begin{adjustbox}{max width=\textwidth}
\begin{tabular}{|c|l|p{8cm}|}
\hline
\textbf{Level} & \textbf{Stage} & \textbf{Characteristics} \\
\hline
1 & \textbf{Initial/Aware} & No formal governance. Individuals manage data ad-hoc. \\
\hline
2 & \textbf{Reactive/Managed} & Problems trigger responses. Some documentation exists. \\
\hline
3 & \textbf{Defined/Proactive} & Enterprise-wide standards established. Most organizations target this. \\
\hline
4 & \textbf{Quantified} & Governance effectiveness measured with metrics. \\
\hline
5 & \textbf{Optimized} & Automated detection and remediation. Continuous improvement. \\
\hline
\end{tabular}
\end{adjustbox}
\caption{Data Governance Maturity Model}
\end{table}

\begin{infobox}
\textbf{The Goal for Most Organizations:} Level 3 (Defined/Proactive)

At this level, you have:
\begin{itemize}
    \item Documented policies and procedures
    \item Clear roles and responsibilities
    \item Automated access controls
    \item Regular audits and compliance checks
\end{itemize}
\end{infobox}

%===============================================================================
% SECTION 6: Governance Operating Models
%===============================================================================
\section{Governance Operating Models}

How governance is implemented depends on organizational culture and structure.

\begin{table}[h!]
\centering
\begin{adjustbox}{max width=\textwidth}
\begin{tabular}{|l|p{4cm}|p{4cm}|p{4cm}|}
\hline
\textbf{Model} & \textbf{Characteristics} & \textbf{Pros} & \textbf{Cons/Best For} \\
\hline
\textbf{Centralized} & Central team controls all governance & High consistency, strong security & Slow, bottlenecks (regulated industries) \\
\hline
\textbf{Decentralized} & Each department self-governs & Fast innovation, flexibility & No standards, duplication (startups) \\
\hline
\textbf{Federated} & Central guidelines + local execution & Balance of control and autonomy & Coordination challenges (enterprises) \\
\hline
\textbf{Hybrid} & Core data centralized, rest decentralized & Protects sensitive data + efficiency & Complex structure \\
\hline
\end{tabular}
\end{adjustbox}
\caption{Governance Operating Models}
\end{table}

\begin{examplebox}{Federated Model in Practice}
\textbf{Central Governance Office:}
\begin{itemize}
    \item Defines global policies (e.g., "All PII must be encrypted")
    \item Maintains the enterprise data catalog
    \item Conducts compliance audits
\end{itemize}

\textbf{Business Unit Data Stewards:}
\begin{itemize}
    \item Implement policies within their domain
    \item Define local schemas and data quality rules
    \item Grant access to their datasets
\end{itemize}
\end{examplebox}

%===============================================================================
% SECTION 7: Unity Catalog Deep Dive
%===============================================================================
\section{Databricks Unity Catalog: Implementation}

Unity Catalog (UC) is Databricks' unified governance solution for all data and AI assets.

\subsection{The Three-Level Hierarchy}

\begin{center}
\texttt{Metastore} $\rightarrow$ \texttt{Catalog} $\rightarrow$ \texttt{Schema} $\rightarrow$ \texttt{Table / Volume / Model / Function}
\end{center}

\begin{table}[h!]
\centering
\begin{tabular}{|l|p{9cm}|}
\hline
\textbf{Level} & \textbf{Description} \\
\hline
\textbf{Metastore} & Top-level container (typically one per cloud region). Stores all metadata. \\
\hline
\textbf{Catalog} & Largest grouping of data assets. Examples: \texttt{prod}, \texttt{dev}, \texttt{hr\_data} \\
\hline
\textbf{Schema} & Logical grouping within a catalog (equivalent to a database) \\
\hline
\textbf{Table/Volume} & Actual data. Tables = structured; Volumes = unstructured files \\
\hline
\end{tabular}
\caption{Unity Catalog Hierarchy}
\end{table}

\subsection{Managed vs External Tables}

\begin{table}[h!]
\centering
\begin{tabular}{|l|p{5cm}|p{5cm}|}
\hline
\textbf{Type} & \textbf{Managed Table} & \textbf{External Table} \\
\hline
\textbf{Storage} & Databricks manages location & You specify cloud storage path \\
\hline
\textbf{Lifecycle} & DROP TABLE deletes data files & DROP TABLE removes metadata only \\
\hline
\textbf{Use Case} & Recommended for most scenarios & Legacy data, shared storage \\
\hline
\end{tabular}
\caption{Managed vs External Tables}
\end{table}

\subsection{How Unity Catalog Security Works}

\begin{enumerate}
    \item \textbf{User submits query:} \texttt{SELECT * FROM catalog.schema.table}
    \item \textbf{Access Control check:} Does user have SELECT permission?
    \item \textbf{If authorized:} UC delegates to cloud IAM role to fetch data
    \item \textbf{Data returned:} User sees only what they're permitted to see
\end{enumerate}

\begin{infobox}
\textbf{Two-Layer Security:}
\begin{itemize}
    \item \textbf{Layer 1 (Cloud IAM):} Controls access to storage buckets (get, put, list)
    \item \textbf{Layer 2 (Unity Catalog):} Fine-grained control (SELECT, INSERT on specific tables/columns)
\end{itemize}

Advantage: You don't need hundreds of IAM roles for different access patterns. One IAM role with broad access + UC for fine-grained control.
\end{infobox}

%===============================================================================
% SECTION 8: ABAC - Attribute-Based Access Control
%===============================================================================
\section{ABAC: Attribute-Based Access Control}

\subsection{RBAC vs ABAC}

\begin{table}[h!]
\centering
\begin{tabular}{|l|p{5cm}|p{5cm}|}
\hline
\textbf{Approach} & \textbf{RBAC (Role-Based)} & \textbf{ABAC (Attribute-Based)} \\
\hline
\textbf{How it works} & Assign permissions to roles; assign roles to users & Define policies based on data attributes (tags) \\
\hline
\textbf{Example} & "Managers can see all HR data" & "Anyone querying PII-tagged columns sees masked data" \\
\hline
\textbf{Scalability} & Role explosion as permissions grow & Scales well with tags \\
\hline
\textbf{Flexibility} & Static; requires role changes & Dynamic; tag changes propagate automatically \\
\hline
\end{tabular}
\caption{RBAC vs ABAC Comparison}
\end{table}

\subsection{ABAC in Unity Catalog}

The ABAC workflow in Databricks:

\begin{enumerate}
    \item \textbf{Create Governance Tags:} Define tag names and allowed values
    \item \textbf{Tag Data Assets:} Apply tags to columns/tables (manually or via AI classification)
    \item \textbf{Create ABAC Policies:} Define rules based on tags
    \item \textbf{Automatic Enforcement:} Policies apply whenever tagged data is accessed
\end{enumerate}

\begin{lstlisting}[style=sqlstyle, caption={Creating a Column Masking Function}, breaklines=true]
-- Step 1: Create a masking function
CREATE FUNCTION ssn_mask(ssn STRING)
RETURNS STRING
RETURN
    CASE
        WHEN is_account_group_member('admin_group') THEN ssn
        WHEN is_account_group_member('analyst_group') THEN CONCAT('*\textbf{-}-', RIGHT(ssn, 4))
        ELSE '*\textbf{-}-**\textbf{'
    END;

-- Step 2: Apply mask to a column
ALTER TABLE employees
ALTER COLUMN social_security_number
SET MASK ssn_mask;
\end{lstlisting}

\subsection{ABAC Policy Types}

\begin{definitionbox}{Row-Level Filtering}
Restrict which \textbf{rows} a user can see based on a condition.

\textbf{Example:} Sales reps can only see customers in their assigned region.
\begin{lstlisting}[style=sqlstyle, breaklines=true]
-- Only show rows where region matches user's region
CREATE FUNCTION region_filter()
RETURNS BOOLEAN
RETURN region = current_user_region();
\end{lstlisting}
\end{definitionbox}

\begin{definitionbox}{Column-Level Masking}
Transform or hide \textbf{column values} based on user permissions.

\textbf{Example:} Non-admin users see email as \texttt{j}*@company.com}
\end{definitionbox}

\subsection{Automatic Data Classification}

Unity Catalog can automatically detect PII using AI models:

\begin{itemize}
    \item \textbf{Email addresses:} Detected and tagged automatically
    \item \textbf{Phone numbers:} Pattern recognition
    \item \textbf{Names, locations:} NER-based detection
    \item \textbf{Credit card numbers:} Regex + Luhn validation
\end{itemize}

\begin{infobox}
\textbf{Auto-Classification + ABAC = Powerful Automation}

When you combine:
\begin{enumerate}
    \item Auto-classification (AI detects email columns)
    \item Governance tags (email columns get "PII" tag)
    \item ABAC policy (PII-tagged columns are masked for analysts)
\end{enumerate}

New tables are automatically protected without manual intervention!
\end{infobox}

%===============================================================================
% SECTION 9: Data Quality Monitoring
%===============================================================================
\section{Data Quality Monitoring}

\subsection{Why Monitor Data Quality?}

"Garbage in, garbage out" applies doubly to AI. If your data quality degrades, your models and reports become unreliable.

\subsection{Key Metrics}

\begin{table}[h!]
\centering
\begin{tabular}{|l|p{8cm}|}
\hline
\textbf{Metric} & \textbf{Description} \\
\hline
\textbf{Freshness} & How recently was the data updated? \\
\hline
\textbf{Completeness} & Are all expected records present? (No sudden drops) \\
\hline
\textbf{Anomaly Detection} & Have data patterns changed unexpectedly? \\
\hline
\textbf{Data Profiling} & Statistics: min, max, nulls, distributions \\
\hline
\end{tabular}
\caption{Data Quality Metrics}
\end{table}

\begin{examplebox}{Completeness Monitoring}
\textbf{Normal Pattern:} 10,000 rows daily

\textbf{Day 1:} 10,200 rows \\
\textbf{Day 2:} 9,800 rows \\
\textbf{Day 3:} 10,100 rows \\
\textbf{Day 4:} \textcolor{red}{0 rows} $\leftarrow$ \textbf{ALERT!}

The monitoring system detects the anomaly and triggers an alert before downstream systems are affected.
\end{examplebox}

\subsection{Setting Up Monitoring in Databricks}

Data quality monitoring is enabled with a single click:
\begin{enumerate}
    \item Navigate to schema in Unity Catalog
    \item Click "Enable Quality Monitoring"
    \item System automatically tracks freshness, completeness, anomalies
    \item Alerts can be configured for threshold violations
\end{enumerate}

%===============================================================================
% SECTION 10: Lineage Tracking
%===============================================================================
\section{Lineage: Tracing Data Origins}

\subsection{What Is Data Lineage?}

\begin{definitionbox}{Data Lineage}
A visual representation of data's journey: where it came from, how it was transformed, and where it goes.

\textbf{Analogy:} A family tree (genealogy) for your data.
\end{definitionbox}

\subsection{Why Lineage Matters}

\begin{enumerate}
    \item \textbf{Debugging:} "This dashboard number looks wrong. Where did it come from?"
    \item \textbf{Impact Analysis:} "If I change this column, what downstream reports break?"
    \item \textbf{Compliance:} "Can we prove this data wasn't derived from restricted sources?"
    \item \textbf{AI Governance:} "What data was used to train this model?"
\end{enumerate}

\subsection{Lineage in Unity Catalog}

Unity Catalog automatically captures lineage:

\begin{itemize}
    \item \textbf{Table-level:} Which tables feed into which tables
    \item \textbf{Column-level:} How specific columns are derived (e.g., substring, join)
    \item \textbf{Custom lineage:} Connect external sources (Salesforce) and targets (PowerBI)
\end{itemize}

\begin{examplebox}{Lineage Use Case}
\textbf{Scenario:} A BI report shows incorrect revenue figures.

\textbf{With Lineage:}
\begin{enumerate}
    \item Navigate to the report's source table
    \item Click "View Lineage"
    \item Trace back through transformations
    \item Discover: A join condition was changed 3 days ago
    \item Fix the join and reprocess
\end{enumerate}
\end{examplebox}

%===============================================================================
% SECTION 11: Lakehouse Federation
%===============================================================================
\section{Lakehouse Federation: Unified Governance}

\subsection{The Problem}

Organizations have data in multiple systems:
\begin{itemize}
    \item Snowflake data warehouse
    \item PostgreSQL operational database
    \item MySQL legacy systems
    \item Cloud storage (S3, Azure Blob)
\end{itemize}

Managing governance separately in each system is impractical.

\subsection{The Solution: Federation}

\begin{definitionbox}{Lakehouse Federation}
Connect external databases to Unity Catalog \textbf{without copying data}. Query remote data as if it were local, with unified governance.

\textbf{Analogy:} An embassy on foreign soil---your laws (governance rules) apply, even though the data physically resides elsewhere.
\end{definitionbox}

\subsection{How Federation Works}

\begin{enumerate}
    \item \textbf{Create Connection:} Register external database (Snowflake, PostgreSQL, etc.)
    \item \textbf{Create Foreign Catalog:} Maps external schemas to UC
    \item \textbf{Query with Pushdown:} Queries are pushed to the source system for efficiency
    \item \textbf{Unified Governance:} Same grant statements, same ABAC policies
\end{enumerate}

\begin{lstlisting}[style=sqlstyle, caption={Creating a Federated Connection}, breaklines=true]
-- Create connection to external Snowflake
CREATE CONNECTION snowflake_conn
TYPE snowflake
OPTIONS (
    host = 'account.snowflakecomputing.com',
    warehouse = 'COMPUTE_WH'
);

-- Create foreign catalog
CREATE FOREIGN CATALOG snowflake_catalog
USING CONNECTION snowflake_conn;

-- Query as if local!
SELECT * FROM snowflake_catalog.schema.table;
\end{lstlisting}

%===============================================================================
% SECTION 12: Delta Sharing
%===============================================================================
\section{Delta Sharing: Secure Data Exchange}

\subsection{The Challenge of Data Sharing}

Traditional methods are insecure and inefficient:
\begin{itemize}
    \item Email CSV files (security nightmare)
    \item FTP transfers (no access control)
    \item Copy data to partner's system (data duplication)
\end{itemize}

\subsection{Delta Sharing Solution}

\begin{definitionbox}{Delta Sharing}
An open protocol for securely sharing data \textbf{without copying}. Recipients don't need Databricks---they can consume via Pandas, Tableau, PowerBI.
\end{definitionbox}

\textbf{What Can Be Shared:}
\begin{itemize}
    \item Tables (Delta format)
    \item Volumes (files)
    \item Notebooks
    \item AI Models
    \item Even federated tables from external sources!
\end{itemize}

\begin{infobox}
\textbf{Cross-Cloud Sharing:}

You can share a Snowflake table (connected via Federation) with a partner on AWS who uses Pandas. The data never leaves Snowflake, but governance is controlled through Unity Catalog.
\end{infobox}

%===============================================================================
% SECTION 13: The Competitive Landscape
%===============================================================================
\section{The Governance Platform Wars}

\subsection{Who's Competing?}

Every major data platform is building governance capabilities:

\begin{table}[h!]
\centering
\begin{tabular}{|l|l|l|}
\hline
\textbf{Platform} & \textbf{Governance Layer} & \textbf{Differentiator} \\
\hline
Databricks & Unity Catalog & ML/AI-first; models, volumes, functions \\
\hline
Snowflake & Polaris + Horizon & SQL warehouse heritage; Iceberg focus \\
\hline
Microsoft & Fabric & Office 365 integration; broad enterprise \\
\hline
AWS & Glue Catalog + Lake Formation & Native AWS integration \\
\hline
\end{tabular}
\caption{Governance Platform Comparison}
\end{table}

\begin{infobox}
\textbf{The Winner's Strategy:}

The platform that can govern \textbf{everyone's assets}---not just their own---will win. If Databricks can effectively govern Snowflake data and vice versa, the most interoperable platform gains the most "assets under management."
\end{infobox}

%===============================================================================
% SECTION 14: Real-World Considerations
%===============================================================================
\section{Real-World Governance Considerations}

\subsection{Data Breaches and Reputational Risk}

\begin{examplebox}{Case Study: Capital One Breach}
In 2019, Capital One suffered a major data breach affecting 100+ million customers.

\textbf{Consequences:}
\begin{itemize}
    \item \$80 million in regulatory fines
    \item Class action lawsuits
    \item Years of reputational damage
    \item Increased scrutiny from regulators
\end{itemize}

\textbf{Lesson:} The cost of poor governance far exceeds the cost of implementing it.
\end{examplebox}

\subsection{The Governance-Agility Tradeoff}

\begin{itemize}
    \item \textbf{Too Strict:} "I need 5 approvals to access any data"---innovation stalls
    \item \textbf{Too Loose:} "Everyone has access to everything"---breaches happen
\end{itemize}

\textbf{Solution: Risk-Based Governance}
\begin{itemize}
    \item \textbf{High-risk data (PII, financial):} Strict controls, approval workflows
    \item \textbf{Low-risk data (public datasets, experiments):} Permissive access
\end{itemize}

%===============================================================================
% SECTION 15: Quick Summary
%===============================================================================
\section{Quick Summary: One-Page Review}

\begin{summarybox}
\textbf{Key Takeaways from Lecture 12:}

\begin{enumerate}
    \item \textbf{Governance Hierarchy:} Information Governance $\supset$ Data Governance $\supset$ AI Governance

    \item \textbf{Four Pillars:} Policies (what) $\rightarrow$ Procedures (how) $\rightarrow$ Standards (specifications) $\rightarrow$ Controls (enforcement)

    \item \textbf{Unity Catalog Structure:} Metastore $\rightarrow$ Catalog $\rightarrow$ Schema $\rightarrow$ Table/Volume/Model

    \item \textbf{ABAC Advantages:}
    \begin{itemize}
        \item Tag-based policies scale better than per-table permissions
        \item Auto-classification discovers PII automatically
        \item Policies propagate to new data without manual intervention
    \end{itemize}

    \item \textbf{Data Quality:}
    \begin{itemize}
        \item Monitor: Freshness, Completeness, Anomalies
        \item Enable with one click in Unity Catalog
    \end{itemize}

    \item \textbf{Lineage:}
    \begin{itemize}
        \item Automatically captured for all Databricks operations
        \item Custom lineage for external sources/targets
    \end{itemize}

    \item \textbf{Federation:}
    \begin{itemize}
        \item Query external databases without copying data
        \item Unified governance across heterogeneous systems
    \end{itemize}

    \item \textbf{Delta Sharing:}
    \begin{itemize}
        \item Share data without copying
        \item Recipients don't need Databricks
    \end{itemize}
\end{enumerate}
\end{summarybox}

%===============================================================================
% SECTION 16: FAQ
%===============================================================================
\section{Frequently Asked Questions}

\textbf{Q: Why use Unity Catalog if we already have cloud IAM roles?}

A: Cloud IAM controls access at the file/folder level. Unity Catalog provides fine-grained control at the table, column, and row level. You can have one IAM role with broad storage access, and use UC for precise governance.

\vspace{1em}

\textbf{Q: Does governance slow down innovation?}

A: Initially, there's overhead in setting up policies. Long-term, governance \textbf{accelerates} work by:
\begin{itemize}
    \item Reducing time spent finding/validating data
    \item Avoiding security incidents that halt projects
    \item Enabling self-service access to pre-approved datasets
\end{itemize}

\vspace{1em}

\textbf{Q: What's the difference between RBAC and ABAC?}

A: RBAC assigns permissions to roles ("Managers can see HR data"). ABAC uses attributes/tags ("Anyone querying PII sees masked data"). ABAC is more flexible and scales better.

\vspace{1em}

\textbf{Q: How does federation handle performance?}

A: Queries are pushed down to the source system. If PostgreSQL is efficient at filtering, that filter runs on PostgreSQL---not in Spark. This minimizes data movement.

\end{document}
