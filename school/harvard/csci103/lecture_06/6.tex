%%%%%%%%%%%%%%%%%%%%%%%%%%%%%%%%%%%%%%%%%%%%%%%%%%%%%%%%%%%%%%%%%%%%%%%%%%%%%%%
% Harvard Academic Notes - 통합 마스터 템플릿
% 모든 강의 노트에 적용되는 통일된 스타일
% 버전: 2.1 - 가독성 개선 (선택적 최적화)
% 최종 수정일: 2025-11-17
%%%%%%%%%%%%%%%%%%%%%%%%%%%%%%%%%%%%%%%%%%%%%%%%%%%%%%%%%%%%%%%%%%%%%%%%%%%%%%%

\documentclass[11pt,a4paper]{article}

%========================================================================================
% 기본 패키지
%========================================================================================

% --- 한국어 지원 ---
\usepackage{kotex}

% --- 페이지 레이아웃 ---
\usepackage[top=20mm, bottom=20mm, left=20mm, right=18mm]{geometry}
\usepackage{setspace}
\onehalfspacing                      % 1.5배 줄간격
\setlength{\parskip}{0.5em}          % 문단 간격
\setlength{\parindent}{0pt}          % 들여쓰기 없음

% --- 표 관련 ---
\usepackage{booktabs}              % 고품질 표
\usepackage{tabularx}              % 자동 너비 조절 표
\usepackage{array}                 % 표 컬럼 확장
\usepackage{longtable}             % 여러 페이지 표
\renewcommand{\arraystretch}{1.1}  % 표 행간 조절

%========================================================================================
% 헤더 및 푸터
%========================================================================================

\usepackage{fancyhdr}
\pagestyle{fancy}
\fancyhf{}
\fancyhead[L]{\small\textit{CSCI E-103: 재현 가능한 머신러닝}}
\fancyhead[R]{\small\textit{Lecture 06}}
\fancyfoot[C]{\thepage}
\renewcommand{\headrulewidth}{0.5pt}
\renewcommand{\footrulewidth}{0.3pt}

% 첫 페이지는 헤더 없음
\fancypagestyle{firstpage}{
    \fancyhf{}
    \fancyfoot[C]{\thepage}
    \renewcommand{\headrulewidth}{0pt}
}

%========================================================================================
% 색상 정의 (파스텔 톤 + 다크모드 호환)
%========================================================================================

\usepackage[dvipsnames]{xcolor}

% 밝은 배경용 파스텔 색상
\definecolor{lightblue}{RGB}{220, 235, 255}      % 부드러운 파랑
\definecolor{lightgreen}{RGB}{220, 255, 235}     % 부드러운 초록
\definecolor{lightyellow}{RGB}{255, 250, 220}    % 부드러운 노랑
\definecolor{lightpurple}{RGB}{240, 230, 255}    % 부드러운 보라
\definecolor{lightgray}{gray}{0.95}              % 밝은 회색
\definecolor{lightpink}{RGB}{255, 235, 245}      % 부드러운 핑크
\definecolor{boxgray}{gray}{0.95}
\definecolor{boxblue}{rgb}{0.9, 0.95, 1.0}
\definecolor{boxred}{rgb}{1.0, 0.95, 0.95}

% 진한 색상 (테두리/제목용)
\definecolor{darkblue}{RGB}{50, 80, 150}
\definecolor{darkgreen}{RGB}{40, 120, 70}
\definecolor{darkorange}{RGB}{200, 100, 30}
\definecolor{darkpurple}{RGB}{100, 60, 150}

%========================================================================================
% 박스 환경 (tcolorbox) - 6가지 타입
%========================================================================================

\usepackage[most]{tcolorbox}
\tcbuselibrary{skins, breakable}

% 1. 개요 박스 (강의 시작 부분)
\newtcolorbox{overviewbox}[1][]{
    enhanced,
    colback=lightpurple,
    colframe=darkpurple,
    fonttitle=\bfseries\large,
    title=📚 강의 개요,
    arc=3mm,
    boxrule=1pt,
    left=8pt,
    right=8pt,
    top=8pt,
    bottom=8pt,
    breakable,
    #1
}

% 2. 요약 박스
\newtcolorbox{summarybox}[1][]{
    enhanced,
    colback=lightblue,
    colframe=darkblue,
    fonttitle=\bfseries,
    title=📝 핵심 요약,
    arc=2mm,
    boxrule=0.7pt,
    left=6pt,
    right=6pt,
    top=6pt,
    bottom=6pt,
    breakable,
    #1
}

% 3. 핵심 정보 박스
\newtcolorbox{infobox}[1][]{
    enhanced,
    colback=lightgreen,
    colframe=darkgreen,
    fonttitle=\bfseries,
    title=💡 핵심 정보,
    arc=2mm,
    boxrule=0.7pt,
    left=6pt,
    right=6pt,
    top=6pt,
    bottom=6pt,
    breakable,
    #1
}

% 4. 주의사항 박스
\newtcolorbox{warningbox}[1][]{
    enhanced,
    colback=lightyellow,
    colframe=darkorange,
    fonttitle=\bfseries,
    title=⚠️ 주의사항,
    arc=2mm,
    boxrule=0.7pt,
    left=6pt,
    right=6pt,
    top=6pt,
    bottom=6pt,
    breakable,
    #1
}

% 5. 예제 박스
\newtcolorbox{examplebox}[1][]{
    enhanced,
    colback=lightgray,
    colframe=black!60,
    fonttitle=\bfseries,
    title=📖 예제: #1,
    arc=2mm,
    boxrule=0.7pt,
    left=6pt,
    right=6pt,
    top=6pt,
    bottom=6pt,
    breakable,
}

% 6. 정의 박스
\newtcolorbox{definitionbox}[1][]{
    enhanced,
    colback=lightpink,
    colframe=purple!70!black,
    fonttitle=\bfseries,
    title=📌 정의: #1,
    arc=2mm,
    boxrule=0.7pt,
    left=6pt,
    right=6pt,
    top=6pt,
    bottom=6pt,
    breakable,
}

% 7. 중요 박스 (importantbox - warningbox와 유사)
\newtcolorbox{importantbox}[1][]{
    enhanced,
    colback=boxred,
    colframe=red!70!black,
    fonttitle=\bfseries,
    title=⚠️ 매우 중요: #1,
    arc=2mm,
    boxrule=0.7pt,
    left=6pt,
    right=6pt,
    top=6pt,
    bottom=6pt,
    breakable,
}

% 8. cautionbox (warningbox와 동일)
\let\cautionbox\warningbox
\let\endcautionbox\endwarningbox

%========================================================================================
% 코드 블록 설정 (밝은 배경)
%========================================================================================

\usepackage{listings}

\definecolor{codegray}{rgb}{0.5,0.5,0.5}
\definecolor{codepurple}{rgb}{0.58,0,0.82}
\definecolor{backcolour}{rgb}{0.95,0.95,0.95}

\lstset{
    basicstyle=\ttfamily\small,
    backgroundcolor=\color{lightgray},
    keywordstyle=\color{darkblue}\bfseries,
    commentstyle=\color{darkgreen}\itshape,
    stringstyle=\color{purple!80!black},
    numberstyle=\tiny\color{black!60},
    numbers=left,
    numbersep=8pt,
    breaklines=true,
    breakatwhitespace=false,
    frame=single,
    frameround=tttt,
    rulecolor=\color{black!30},
    captionpos=b,
    showstringspaces=false,
    tabsize=2,
    xleftmargin=15pt,
    xrightmargin=5pt,
    escapeinside={\%*}{*)}
}

% Python 코드 스타일
\lstdefinestyle{pythonstyle}{
    language=Python,
    morekeywords={self, True, False, None},
}

% SQL 코드 스타일
\lstdefinestyle{sqlstyle}{
    language=SQL,
    morekeywords={SELECT, FROM, WHERE, JOIN, GROUP, BY, ORDER, HAVING},
}

%========================================================================================
% 목차 스타일링
%========================================================================================

\usepackage{tocloft}
\renewcommand{\cftsecleader}{\cftdotfill{\cftdotsep}}
\setlength{\cftbeforesecskip}{0.4em}
\renewcommand{\cftsecfont}{\bfseries}
\renewcommand{\cftsubsecfont}{\normalfont}

%========================================================================================
% 표 및 그림
%========================================================================================

\usepackage{graphicx}              % 이미지
\usepackage{adjustbox}             % 표/박스 크기 조절

% 표 캡션 스타일
\usepackage{caption}
\captionsetup[table]{
    labelfont=bf,
    textfont=it,
    skip=5pt
}
\captionsetup[figure]{
    labelfont=bf,
    textfont=it,
    skip=5pt
}

%========================================================================================
% 수학
%========================================================================================

\usepackage{amsmath, amssymb, amsthm}

% 정리 환경
\theoremstyle{definition}
\newtheorem{theorem}{정리}[section]
\newtheorem{lemma}[theorem]{보조정리}
\newtheorem{proposition}[theorem]{명제}
\newtheorem{corollary}[theorem]{따름정리}
\newtheorem{definition}{정의}[section]
\newtheorem{example}{예제}[section]

%========================================================================================
% 하이퍼링크
%========================================================================================

\usepackage[
    colorlinks=true,
    linkcolor=blue!80!black,
    urlcolor=blue!80!black,
    citecolor=green!60!black,
    bookmarks=true,
    bookmarksnumbered=true,
    pdfborder={0 0 0}
]{hyperref}

% PDF 메타데이터는 각 문서에서 설정
\hypersetup{
    pdftitle={CSCI E-103: 재현 가능한 머신러닝 - Lecture 06},
    pdfauthor={강의 노트},
    pdfsubject={Academic Notes}
}

%========================================================================================
% 기타 유용한 패키지
%========================================================================================

\usepackage{enumitem}              % 리스트 커스터마이징
\setlist{nosep, leftmargin=*, itemsep=0.3em}

\usepackage{microtype}             % 타이포그래피 개선
\usepackage{footnote}              % 각주 개선
\usepackage{url}                   % URL 줄바꿈
\urlstyle{same}

%========================================================================================
% 사용자 정의 명령어
%========================================================================================

% 강조 텍스트
\newcommand{\important}[1]{\textbf{\textcolor{red!70!black}{#1}}}
\newcommand{\keyword}[1]{\textbf{#1}}
\newcommand{\term}[1]{\textit{#1}}
\newcommand{\code}[1]{\texttt{#1}}

% 용어 설명 (인라인)
\newcommand{\defterm}[2]{\textbf{#1}\footnote{#2}}

% 섹션 시작 전 페이지 분리
\newcommand{\newsection}[1]{\newpage\section{#1}}

%========================================================================================
% 문서 제목 스타일
%========================================================================================

\usepackage{titling}
\pretitle{\begin{center}\LARGE\bfseries}
\posttitle{\par\end{center}\vskip 0.5em}
\preauthor{\begin{center}\large}
\postauthor{\end{center}}
\predate{\begin{center}\large}
\postdate{\par\end{center}}

%========================================================================================
% 섹션 제목 간격
%========================================================================================

\usepackage{titlesec}
\titlespacing*{\section}{0pt}{1.5em}{0.8em}
\titlespacing*{\subsection}{0pt}{1.2em}{0.6em}
\titlespacing*{\subsubsection}{0pt}{1em}{0.5em}

%========================================================================================
% 메타 정보 박스 명령어
%========================================================================================

\newcommand{\metainfo}[4]{
\begin{tcolorbox}[
    colback=lightpurple,
    colframe=darkpurple,
    boxrule=1pt,
    arc=2mm,
    left=10pt,
    right=10pt,
    top=8pt,
    bottom=8pt
]
\begin{tabular}{@{}rl@{}}
▣ \textbf{강의명:} & #1 \\[0.3em]
▣ \textbf{주차:} & #2 \\[0.3em]
▣ \textbf{교수명:} & #3 \\[0.3em]
▣ \textbf{목적:} & \begin{minipage}[t]{0.75\textwidth}#4\end{minipage}
\end{tabular}
\end{tcolorbox}
}

%========================================================================================
% 끝
%========================================================================================


\begin{document}

\maketitle
\thispagestyle{firstpage}

\metainfo{CSCI E-103: 재현 가능한 머신러닝}{Lecture 06}{Anindita Mahapatra & Eric Gieseke}{Lecture 06의 핵심 개념 학습}


\tableofcontents

\newpage

% ----------------------------------------------------------------------
\section{강의 개요}
% ----------------------------------------------------------------------

\begin{boxSummary}
본 문서는 비즈니스 인텔리전스(BI) 분석 및 데이터 시각화의 핵심 개념을 다룹니다.
데이터 웨어하우스와 데이터 레이크의 전통적인 차이점에서 시작하여, 두 아키텍처의 장점을 결합한 \textbf{레이크하우스(Lakehouse)}의 필요성과 구조를 중점적으로 설명합니다.

또한 BI와 비즈니스 분석(BA)의 차이점을 명확히 하고, BI 분석가(BI Analyst)라는 핵심 페르소나와 이들의 주요 기술(SQL) 및 데이터 모델링(Star Schema 등) 방법을 배웁니다.

마지막으로 Databricks 플랫폼을 활용한 최신 BI 기능들, 특히 \textbf{Databricks SQL}, \textbf{Lakeview 대시보드}, 그리고 \textbf{Genie}와 같은 자연어 기반 AI 분석 도구의 작동 원리와 활용법을 실습 예제와 함께 살펴봅니다.
\end{boxSummary}

\subsection{이 강의의 주요 질문 (Key Questions)}

이 문서를 학습한 후, 다음 질문들에 답할 수 있어야 합니다.

\begin{itemize}
    \item 비즈니스 인텔리전스(BI)란 무엇이며, 비즈니스 분석(BA) 및 AI와 어떻게 다른가?
    \item 데이터 웨어하우스(Data Warehouse)와 데이터 레이크(Data Lake)의 근본적인 차이점은 무엇인가?
    \item 왜 \textbf{레이크하우스(Lakehouse)}라는 새로운 아키텍처가 필요하게 되었는가?
    \item BI 데이터를 소비하는 주요 사용자 페르소나는 누구이며, 그들의 핵심 기술은 무엇인가? (힌트: SQL)
    \item BI 성능을 측정하는 핵심성과지표(KPI)인 \textbf{동시성(Concurrency)}과 \textbf{지연 시간(Latency)}은 무엇을 의미하는가?
    \item Databricks SQL 환경에서 \textbf{AI 함수} (예: \texttt{ai\_query})를 어떻게 활용할 수 있는가?
    \item \textbf{Genie}가 일반 챗봇(ChatGPT 등)과 달리 "환각(Hallucination)"을 일으키지 않는 이유는 무엇인가?
\end{itemize}

\newpage

% ----------------------------------------------------------------------
\section{핵심 용어 정리}
% ----------------------------------------------------------------------

본격적인 학습에 앞서, 이번 강의에서 자주 등장하는 핵심 용어들을 정리합니다.

\begin{table}[h!]
    \centering
    \caption{BI 분석 및 데이터 저장소 핵심 용어}
    \label{tab:terminology}
    \begin{adjustbox}{width=\textwidth, center}
    \begin{tabular}{lp{6cm}ll}
        \toprule
        \textbf{용어 (Term)} & \textbf{쉬운 설명} & \textbf{원어 (English)} & \textbf{비고} \\
        \midrule
        \textbf{BI} & 기업이 더 나은 의사결정을 하도록 데이터를 수집, 분석, 시각화하는 기술. (과거~현재 "무엇"이 일어났는지) & Business Intelligence & 리포트, 대시보드 \\
        \textbf{BA} & 과거 데이터를 사용해 현재를 설명하고 미래를 예측하는 분석. ("왜" 일어났는지, "무엇"이 일어날지) & Business Analytics & 통계, 예측 모델링 \\
        \textbf{데이터 웨어하우스} & 정형화된(구조화된) 데이터만 저장하는, 빠르고 비싼 '데이터 도서관'. 스키마가 미리 정해짐. & Data Warehouse (DW) & Schema-on-Write \\
        \textbf{데이터 레이크} & 모든 종류(정형, 비정형)의 데이터를 원본 그대로 저장하는, 저렴하고 거대한 '데이터 차고'. & Data Lake (DL) & Schema-on-Read \\
        \textbf{레이크하우스} & 데이터 레이크의 저렴한 저장소 위에 데이터 웨어하우스의 성능/안정성(ACID)을 구현한 통합 아키텍처. & Lakehouse & DW + DL \\
        \textbf{메달리온 아키텍처} & 데이터를 3단계(Bronze: 원본, Silver: 정제, Gold: 집계/BI용)로 나누어 관리하는 파이프라인 구조. & Medallion Architecture & Bronze, Silver, Gold \\
        \textbf{데이터 사일로} & 데이터가 부서별, 시스템별로 고립되어 연결되지 않은 상태. & Data Silo & \\
        \textbf{데이터 스웜프} & 데이터 레이크가 거버넌스(관리) 없이 방치되어 쓸모없게 된 상태. '데이터 늪'. & Data Swamp & \\
        \textbf{데이터 연합} & 데이터를 물리적으로 이동(ETL)하지 않고, 원격 소스에서 직접 쿼리(읽기)하는 기술. & Data Federation & 소유권(Ownership)이 없음. \\
        \textbf{뷰 (View)} & 쿼리 자체를 가상 테이블처럼 저장한 것. 호출 시점에 쿼리가 실행됨. (물리적 저장 X) & View & \\
        \textbf{구체화된 뷰} & 쿼리 결과를 물리적으로 미리 계산하여 저장해 둔 테이블. (성능 향상 목적) & Materialized View (MV) & \\
        \textbf{동시성} & 시스템이 동시에 몇 개의 쿼리(작업)를 처리할 수 있는지 나타내는 KPI. & Concurrency & \\
        \textbf{지연 시간} & 쿼리를 요청한 시점부터 결과가 반환될 때까지 걸리는 시간 (딜레이). & Latency & \\
        \bottomrule
    \end{tabular}
    \end{adjustbox}
\end{table}

\newpage

% ----------------------------------------------------------------------
\section{핵심 개념 1: 데이터 저장소의 진화}
% ----------------------------------------------------------------------

BI를 이해하기 위해서는 먼저 BI가 사용하는 데이터가 어디에, 어떻게 저장되는지 알아야 합니다. 데이터 저장소는 크게 '데이터 웨어하우스'와 '데이터 레이크'로 나뉩니다.

\subsection{데이터 웨어하우스 (Data Warehouse) vs. 데이터 레이크 (Data Lake)}

두 개념은 데이터를 저장하는 목적과 방식에서 근본적인 차이가 있습니다.

\begin{boxExample}
    \textbf{비유로 이해하기: 도서관 vs. 차고}

    \begin{itemize}
        \item \textbf{데이터 웨어하우스(DW)는 '잘 정리된 도서관'입니다.}
        도서관에는 오직 '책'(정형 데이터)만 받습니다. 책을 받으면 사서가 즉시 분류하고(Schema-on-Write), 정해진 책장(테이블)에 꽂습니다. 덕분에 특정 책을 찾을 때(BI 쿼리) 매우 빠르고 정확합니다. 하지만 비디오테이프나 사진(비정형 데이터)은 보관할 수 없고, 도서관을 짓고 유지하는 데 비용이 많이 듭니다.

        \item \textbf{데이터 레이크(DL)는 '모든 것을 쌓아두는 차고'입니다.}
        차고에는 책, 사진, 비디오, 고장 난 자전거 등 모든 것(정형/비정형 데이터)을 원본 그대로 던져 넣을 수 있습니다. 저장 공간은 매우 저렴합니다. 나중에 무언가 필요할 때(Schema-on-Read) 차고를 뒤져서 찾아야 하므로 시간이 오래 걸립니다. 관리를 안 하면 쓰레기장, 즉 \textbf{데이터 스웜프(Data Swamp)}가 되기 쉽습니다.
    \end{itemize}
\end{boxExample}

다음은 데이터 레이크와 웨어하우스의 기술적 장단점을 비교한 표입니다.

\begin{table}[h!]
    \centering
    \caption{데이터 레이크와 데이터 웨어하우스 비교}
    \label{tab:lake_vs_warehouse}
    \begin{adjustbox}{width=\textwidth, center}
    \begin{tabular}{l|ll|ll}
        \toprule
        \multirow{2}{*}{\textbf{측면 (Dimension)}} & \multicolumn{2}{c|}{\textbf{데이터 레이크 (Data Lake)}} & \multicolumn{2}{c}{\textbf{데이터 웨어하우스 (Data Warehouse)}} \\
        & \textbf{장점 (Pro)} & \textbf{단점 (Con)} & \textbf{장점 (Pro)} & \textbf{단점 (Con)} \\
        \midrule
        \textbf{스토리지} &
        \begin{tabular}[t]{@{}l@{}}모든 파일 타입 지원 \\ (Open-format)\end{tabular} &
        \begin{tabular}[t]{@{}l@{}}데이터 품질이 낮을 수 있음 \\ 파일 수준의 접근 제어\end{tabular} &
        \begin{tabular}[t]{@{}l@{}}신뢰성 높은 데이터 \\ 세분화된 접근 제어\end{tabular} &
        \begin{tabular}[t]{@{}l@{}}주로 정형 데이터만 지원 \\ 특정 벤더 종속적 포맷\end{tabular} \\
        \midrule
        \textbf{컴퓨팅} &
        \begin{tabular}[t]{@{}l@{}}매우 경제적임 \\ (스토리지/컴퓨트 분리)\end{tabular} &
        \begin{tabular}[t]{@{}l@{}}운영 복잡성이 높음\end{tabular} &
        \begin{tabular}[t]{@{}l@{}}사용하기 쉬움 \\ 높은 동시성, 낮은 지연시간\end{tabular} &
        \begin{tabular}[t]{@{}l@{}}확장 시 비용이 많이 듬 \\ (스토리지/컴퓨트 결합)\end{tabular} \\
        \midrule
        \textbf{소비} &
        \begin{tabular}[t]{@{}l@{}}풍부한 도구 생태계 \\ (ML, AI, DS)\end{tabular} &
        \begin{tabular}[t]{@{}l@{}}BI 사용 사례에 최적화 안됨\end{tabular} &
        \begin{tabular}[t]{@{}l@{}}SQL에 최적화 (BI)\end{tabular} &
        \begin{tabular}[t]{@{}l@{}}ML, 스트리밍 사용 제한\end{tabular} \\
        \bottomrule
    \end{tabular}
    \end{adjustbox}
\end{table}

\subsection{아키텍처의 한계와 레이크하우스 (Lakehouse)의 등장}

전통적으로 기업들은 두 시스템을 함께 사용하려 했습니다.
\begin{enumerate}
    \item \textbf{1세대 (DW Only):} 정형 데이터만 ETL을 통해 DW에 적재. BI, 리포트에만 사용.
    \item \textbf{2세대 (Two-tier: Lake + DW):} 모든 데이터를 DL에 저장. 이 중 필요한 정형 데이터만 다시 ETL을 통해 DW로 복사하여 BI에 사용. ML/DS는 DL을 사용.
\end{enumerate}

\begin{boxWarning}
    \textbf{2세대 (Two-tier) 아키텍처의 문제점}

    모든 데이터를 DL에 저장하고 BI용 데이터만 DW로 복사하는 2세대 방식은 여러 문제를 야기했습니다.
    \begin{itemize}
        \item \textbf{데이터 중복:} 똑같은 데이터가 Lake와 Warehouse에 이중으로 저장되어 비용이 낭비됩니다.
        \item \textbf{복잡성 증가:} Lake와 DW라는 두 개의 시스템을 별도로 관리하고 동기화해야 합니다.
        \item \textbf{데이터 최신성 문제:} Lake의 원본 데이터가 DW로 복사(ETL)되는 데 시간이 걸려, BI 사용자는 항상 최신 데이터를 보지 못할 수 있습니다.
    \end{itemize}
\end{boxWarning}

이러한 문제를 해결하기 위해 \textbf{레이크하우스(Lakehouse)} 아키텍처가 등장했습니다.

\textbf{레이크하우스(Lakehouse)란?}
데이터 레이크의 저렴하고 유연한 개방형 스토리지(예: S3, ADLS) 위에, 데이터 웨어하우스의 핵심 기능(ACID 트랜잭션, 데이터 거버넌스, 빠른 쿼리 성능)을 제공하는 \textbf{단일 통합 아키텍처}입니다. (예: Databricks Delta Lake)

\textbf{레이크하우스의 장점:}
\begin{itemize}
    \item \textbf{단일 시스템 (Simplicity):} 데이터 복제나 별도 시스템 관리가 필요 없습니다.
    \item \textbf{모든 워크로드 지원:} BI, 리포팅, 데이터 사이언스(DS), 머신러닝(ML) 등 모든 작업을 \textbf{동일한 단일 데이터 소스}에서 직접 수행할 수 있습니다.
    \item \textbf{비용 효율성:} 웨어하우스의 성능을 레이크의 저렴한 비용으로 달성합니다.
    \item \textbf{최신성:} 데이터가 한 곳에만 있으므로(Single Source of Truth), BI 사용자도 항상 최신 데이터를 쿼리할 수 있습니다.
\end{itemize}

% ----------------------------------------------------------------------
\subsection{데이터 저장소 기술의 역사적 흐름}
% ----------------------------------------------------------------------

현재의 레이크하우스 개념은 다음과 같은 기술적 진화를 거쳐 등장했습니다.

\begin{enumerate}
    \item \textbf{스프레드시트 (Spreadsheets):} 가장 원시적인 데이터 저장소 (예: CSV)
    \item \textbf{데이터 웨어하우스 (Data Warehouse):}
    \begin{itemize}
        \item \textbf{Bill Inmon (인몬):} ER 모델 기반, 정규화(3NF)된 중앙 DW를 구축. (데이터 일관성 중시)
        \item \textbf{Ralph Kimball (킴볼):} 비즈니스 사용자에 초점, 비정규화된 차원 모델(Star/Snowflake Schema)을 제안. (BI 쿼리 속도 중시)
    \end{itemize}
    \item \textbf{MPP (대규모 병렬 처리):} Teradata, Greenplum 등. 데이터와 컴퓨팅을 여러 노드에 분산. 테라바이트(TB)급 처리가 가능해졌으나 매우 고가였습니다.
    \item \textbf{NoSQL / BigTable:} Google이 시작. 페타바이트(PB)급 대용량 테이블 데이터를 처리.
    \item \textbf{Hadoop / 데이터 레이크:} Doug Cutting. 저렴한 하드웨어(범용 서버)를 수평적으로 확장. 스토리지와 컴퓨팅을 분리하는 개념을 도입하며 '데이터 레이크' 시대를 열었습니다.
    \item \textbf{데이터 메시 / 패브릭 (Data Mesh / Fabric):} 데이터를 '제품'으로 취급하는 분산형 아키텍처(Mesh)와 데이터 위치를 추상화하는 기술(Fabric)이 등장.
    \item \textbf{레이크하우스 (Lakehouse):} 데이터 레이크 위에 DW 기능을 결합한 현재의 아키텍처.
\end{enumerate}

\newpage

% ----------------------------------------------------------------------
\section{핵심 개념 2: 비즈니스 인텔리전스 (BI) 란?}
% ----------------------------------------------------------------------

\subsection{BI의 정의와 목적}

\begin{boxExample}
    \textbf{비유: "오즈의 마법사"의 수정 구슬}

    BI는 기업의 경영진에게 마치 '수정 구슬'과 같습니다. 비즈니스는 데이터를 통해 현재 무슨 일이 일어나고 있는지 명확히 보고(Insight), 특히 미래에 어떤 일이 일어날지 예측(Foresight)하여 경쟁 우위(Competitive Advantage)를 점하고 싶어 합니다.
\end{boxExample}

\textbf{비즈니스 인텔리전스 (BI)}란, 기업의 데이터를 수집, 통합, 분석, 시각화하여 \textbf{더 나은 비즈니스 의사결정을 지원}하는 모든 기술, 애플리케이션, 프로세스를 의미합니다.

\begin{tcolorbox}[title=BI의 핵심 철학, colback=gray!5, colframe=gray!50!black, sharp corners]
    "데이터(Data)는 분석(Analytics)을 위해 필요한 것이다. \\
    정보(Information)는 비즈니스(Business)를 위해 필요한 것이다."
\end{tcolorbox}

BI는 원본 데이터(Data)를 가공하여 비즈니스에 유용한 '정보(Information)'로 만드는 과정입니다.

\textbf{BI의 주요 구성 요소:}
\begin{itemize}
    \item \textbf{데이터 분석 (Data Analysis):} 데이터 탐색 및 쿼리
    \item \textbf{시각적 분석 (Visual Analytics):} 차트, 그래프, 대시보드
    \item \textbf{고급 분석 (Advanced Analytics):} 예측, 통계 (BA 영역과 겹침)
    \item \textbf{데이터 거버넌스 (Data Governance):} 데이터 품질, 보안, 접근 제어
    \item \textbf{전략 문서화 (Strategy Documentation):} 비즈니스 미션과 전략
\end{itemize}

\subsection{BI vs. BA: 무엇이 다른가?}

BI와 BA(Business Analytics)는 자주 혼용되지만, 초점과 질문이 다릅니다.

\begin{table}[h!]
    \centering
    \caption{Business Intelligence (BI) vs. Business Analytics (BA)}
    \label{tab:bi_vs_ba}
    \begin{tabular}{p{0.45\textwidth}|p{0.45\textwidth}}
        \toprule
        \textbf{비즈니스 인텔리전스 (BI)} & \textbf{비즈니스 분석 (BA)} \\
        \midrule
        \textbf{과거와 현재}의 데이터를 사용합니다. & \textbf{과거} 데이터를 사용합니다. \\
        \midrule
        \textbf{"무엇"이, "어떻게"} 일어났는지에 집중합니다. (Descriptive) & \textbf{"왜"} 일어났는지 설명하고, \textbf{"무엇이 일어날지"} 예측합니다. (Explanatory, Predictive) \\
        \midrule
        \textbf{주요 질문 예시:}
        \begin{itemize}
            \item "지난 분기 매출은 얼마인가?" (What)
            \item "가장 많이 팔린 제품은?" (Who)
            \item "언제 가장 많이 팔렸나?" (When)
        \end{itemize} &
        \textbf{주요 질문 예시:}
        \begin{itemize}
            \item "왜 그 제품이 많이 팔렸나?" (Why)
            \item "이 추세가 계속될까?" (Will it happen again?)
            \item "가격을 10\% 올리면 어떻게 될까?" (What if?)
        \end{itemize} \\
        \midrule
        \textbf{주요 기술:} 리포팅, 대시보드, OLAP, Ad-hoc 쿼리 & \textbf{주요 기술:} 통계 분석, 데이터 마이닝, 예측 모델링, A/B 테스트 \\
        \bottomrule
    \end{tabular}
\end{table}

최근 BI의 트렌드는 단순한 '서술적(Descriptive)' 분석(과거 리포팅)에서 '처방적(Prescriptive)' 분석(미래에 무엇을 해야 하는지 제안)으로 나아가고 있습니다.

\subsection{BI 프로세스 (5단계)}

BI는 다음과 같은 5단계를 거쳐 비즈니스 가치를 창출합니다.

\begin{enumerate}
    \item \textbf{데이터 수집 (Collect):} 여러 소스 시스템(CRM, ERP 등)의 데이터를 데이터 웨어하우스(또는 레이크하우스)로 통합(ETL)합니다.
    \item \textbf{데이터 조직 (Organize):} 수집된 데이터를 분석하기 좋은 모델(예: OLAP 큐브, Star Schema)로 구성합니다.
    \item \textbf{데이터 분석 (Analyze):} BI 분석가나 사용자가 \textbf{SQL}을 사용해 데이터를 쿼리합니다.
    \item \textbf{데이터 시각화 (Visualize):} 쿼리 결과를 차트, 대시보드, 리포트 등 이해하기 쉬운 형태로 만듭니다.
    \item \textbf{의사 결정 (Decide):} 경영진과 실무자가 이 시각화된 '정보'를 보고 전략적 의사결정을 내립니다. (예: 어떤 신제품을 개발할지, 어떤 시장에 진출할지)
\end{enumerate}

\newpage

% ----------------------------------------------------------------------
\section{핵심 개념 3: BI 페르소나와 데이터 모델링}
% ----------------------------------------------------------------------

\subsection{BI 분석가 (BI Analyst) 페르소나}

BI 워크플로우에는 여러 역할(데이터 엔지니어, 데이터 과학자 등)이 있지만, BI의 핵심 소비자는 \textbf{BI 분석가}입니다.

\begin{itemize}
    \item \textbf{주요 기술: SQL}
    BI 분석가의 주무기는 \textbf{SQL}입니다. 이들은 SQL을 사용해 데이터를 탐색하고, 비즈니스 질문에 답하며, 대시보드에 필요한 데이터를 추출합니다.
    \item \textbf{소비 데이터: 정제된 데이터 (Curated Data)}
    BI 분석가는 원본(Bronze) 데이터가 아닌, 데이터 엔지니어가 1차 정제(Silver)하고 비즈니스 용도에 맞게 집계/가공한(Gold) 데이터를 주로 사용합니다.
    \item \textbf{역할: 분석 엔지니어링 (Analytics Engineering)}
    최근에는 BI 분석가가 SQL을 사용해 데이터를 모델링하고 골드 테이블을 직접 큐레이션하는 역할까지 맡기도 하며, 이를 '분석 엔지니어링'이라고 부릅니다.
\end{itemize}

\subsection{BI를 위한 데이터 모델링 (Data Modeling for BI)}

BI 쿼리는 매우 빨라야 하므로(Low Latency), 데이터를 BI에 최적화된 구조로 모델링해야 합니다. 이때 가장 널리 쓰이는 방식이 \textbf{차원 모델링(Dimensional Modeling)}, 특히 \textbf{스타 스키마(Star Schema)}입니다.

\begin{itemize}
    \item \textbf{스타 스키마 (Star Schema):} 이름처럼 '별' 모양의 구조입니다.
    \begin{itemize}
        \item \textbf{팩트 테이블 (Fact Table):} 중앙에 위치. 비즈니스 이벤트의 측정값(숫자 데이터)을 담습니다. (예: `sales_amount`, `quantity_sold`)
        \item \textbf{디멘션 테이블 (Dimension Tables):} 팩트 테이블을 둘러싼 '별'의 꼭짓점. 이벤트가 일어난 맥락(Context)을 설명합니다. (예: \texttt{dim\_customer}, \texttt{dim\_product}, \texttt{dim\_time})
    \end{itemize}
    \item \textbf{스노우플레이크 스키마 (Snowflake Schema):} 스타 스키마의 변형. 디멘션 테이블이 추가로 정규화되어 또 다른 테이블에 연결된 구조. (예: `dim_product`가 `dim_category`에 연결됨)
\end{itemize}

\begin{boxExample}
    \textbf{스타 스키마 예시: 온라인 상점 매출}

    \begin{itemize}
        \item \textbf{Fact\_Sales (팩트 테이블):}
        \texttt{\{date\_key, product\_key, customer\_key, sales\_amount, quantity\}}
        \item \textbf{Dim\_Time (시간 차원):}
        \texttt{\{date\_key, date, month, year, quarter, day\_of\_week\}}
        \item \textbf{Dim\_Product (제품 차원):}
        \texttt{\{product\_key, product\_name, category, brand\}}
        \item \textbf{Dim\_Customer (고객 차원):}
        \texttt{\{customer\_key, customer\_name, city, country\}}
    \end{itemize}

    "2025년 1분기, 서울에 거주하는 고객들의 카테고리별 매출액은?" 같은 BI 쿼리를 매우 빠르고 단순한 Join으로 처리할 수 있습니다.
\end{boxExample}

\begin{boxWarning}
    \textbf{Data Vault 모델}

    또 다른 모델로 \textbf{Data Vault}가 있습니다. 이는 허브(Hubs: 핵심 비즈니스 키), 링크(Links: 관계), 새틀라이트(Satellites: 설명 속성)로 구성되며, 변화에 유연하게 대응할 수 있습니다.

    하지만 Data Vault가 Silver 레이어에서 데이터를 유연하게 통합하는 데 쓰이더라도, 최종적으로 BI 사용자가 쿼리하는 \textbf{Gold 레이어는 여전히 Star Schema 같은 차원 모델로 변환}되는 경우가 많습니다.
\end{boxWarning}

\newpage
% ----------------------------------------------------------------------
\section{Databricks를 활용한 BI 및 데이터 웨어하우징}
% ----------------------------------------------------------------------

Databricks는 '레이크하우스' 아키텍처를 기반으로 BI 및 데이터 웨어하우징 기능을 제공합니다.

\subsection{Databricks 데이터 인텔리전스 플랫폼}

Databricks 플랫폼은 데이터 흐름에 따라 여러 구성요소로 나뉩니다.
\begin{description}
    \item[Source (소스):] 모든 종류의 데이터 (정형, 비정형, 스트리밍)
    \item[Ingest (수집):] 데이터를 레이크하우스로 가져옵니다.
    \begin{itemize}
        \item \textbf{ETL:} 전통적인 방식. 데이터를 복사하여 레이크하우스가 '소유'합니다.
        \item \textbf{Data Federation (데이터 연합):} 데이터를 복사하지 않고, 외부 시스템(예: Oracle, Redshift)에 \textbf{읽기 전용(read-only) 쿼리}를 날려 가상으로 데이터를 가져옵니다. \textbf{소유권(Ownership)이 없는 것}이 ETL과의 핵심 차이입니다.
    \end{itemize}
    \item[Transform (변환):] Medallion Architecture (Bronze $\to$ Silver $\to$ Gold)에 따라 데이터를 정제하고 가공합니다.
    \item[Query and Process (쿼리/처리):]
    \begin{itemize}
        \item \textbf{Databricks SQL (DBSQL):} BI 및 데이터 웨어하우징 워크로드를 위한 SQL 엔진입니다. \textbf{ANSI SQL} 표준을 따릅니다.
        \item \textbf{Data Science \& ML:} 데이터 과학 및 머신러닝 워크로드.
    \end{itemize}
    \item[Governance (거버넌스):] \textbf{Unity Catalog}가 모든 데이터 자산(테이블, 파일, 모델)의 접근 제어, 데이터 계보(Lineage), 감사를 중앙에서 관리합니다.
    \item[Core Engine (엔진):] \textbf{Photon} (Spark을 C++로 재작성한 차세대 벡터화 엔진)이 쿼리 성능을 높여줍니다.
    \item[Serve/Analysis (제공/분석):] \textbf{Lakeview Dashboards} (내장 대시보드), \textbf{BI Tools} (Tableau, PowerBI 연동), Lakehouse Apps 등을 통해 최종 사용자에게 데이터를 제공합니다.
\end{description}

\begin{boxWarning}
    \textbf{ETL vs. Federation 선택 기준}

    Federation은 데이터를 이동할 필요가 없어 편리해 보이지만 만능이 아닙니다.
    \begin{itemize}
        \item \textbf{Federation이 적합할 때:} 외부 시스템의 데이터 용량이 작거나(Modest data), 단순 참조/조회용(Lookup)일 때 사용합니다.
        \item \textbf{ETL이 적합할 때:} 대용량 데이터를 처리해야 하거나, 낮은 지연 시간(Low Latency)의 빠른 쿼리 성능이 반드시 필요할 때는 ETL을 통해 데이터를 레이크하우스로 물리적으로 가져와야 합니다.
    \end{itemize}
\end{boxWarning}


\subsection{Databricks SQL (DBSQL)의 주요 BI 기능}

DBSQL은 BI 분석가를 위해 다음과 같은 강력한 기능들을 제공합니다.

\begin{itemize}
    \item \textbf{서버리스 컴퓨팅 (Serverless Compute):}
    쿼리 요청 시 즉시 컴퓨트 자원을 할당받습니다. 기존 방식처럼 VM(클러스터)이 켜지는 데 3~6분씩 기다릴 필요가 없습니다.
    \item \textbf{스트리밍 테이블 (Streaming Tables):}
    복잡한 코드 없이, SQL만으로 스트리밍 데이터 소스(예: Kafka, 클라우드 파일)를 읽어 자동으로 업데이트되는 테이블을 정의할 수 있습니다.
    \item \textbf{구체화된 뷰 (Materialized Views - MV):}
    복잡하고 오래 걸리는 쿼리 결과를 미리 계산하여 물리적 테이블로 저장합니다. BI 대시보드가 이 MV를 조회하면 매우 빠른 응답을 얻을 수 있습니다. 데이터 원본이 변경되면 MV는 증분(incrementally) 업데이트됩니다.
    \item \textbf{지오스페이셜 지원 (Geospatial Support):}
    위치/지리 정보(예: H3)를 처리하는 함수를 SQL에서 바로 지원합니다.
\end{itemize}

\subsection{SQL의 AI 기능 (AI Functions in SQL)}

Databricks SQL은 LLM(거대 언어 모델)을 SQL 쿼리 내에서 직접 호출하는 혁신적인 기능을 제공합니다.

\subsubsection{1. ai\_query(): 외부 LLM 호출}
\texttt{ai\_query()} 함수를 사용하면 SQL 문 내에서 OpenAI의 GPT나 Anthropic의 Claude 같은 외부 모델을 직접 호출하여 데이터를 보강(enrich)할 수 있습니다.

\begin{lstlisting}[language=SQL, caption={ai\_query()를 사용한 제품 홍보 문구 생성 예시}, label={lst:ai_query}]
-- 'my-openai-chat'이라는 모델 엔드포인트를 호출
SELECT
  sku_id,
  product_name,
  ai_query(
    "my-openai-chat",  -- 미리 등록한 모델 엔드포인트
    -- 프롬프트: 제품 이름을 포함하여 30단어 홍보 문구 생성
    "You are a marketing expert. Generate a promotional text 
     in 30 words for product: " || product_name
  ) AS promotional_text
FROM
  retail_products;
\end{lstlisting}

\subsubsection{2. 내장 AI 함수 (Built-in AI Functions)}
자주 사용되는 AI 작업을 위해 미리 구축된 모델을 제공합니다. 외부 모델을 설정할 필요 없이 즉시 사용 가능합니다.

\begin{lstlisting}[language=SQL, caption={내장 AI 함수 사용 예시}, label={lst:ai_functions}]
-- 감성 분석 (Sentiment Analysis)
-- 'positive', 'negative', 'neutral' 반환
SELECT ai_analyze_sentiment('I am happy');
-- 결과: positive

-- 텍스트 분류 (Classification)
-- 주어진 레이블 중 하나로 분류
SELECT ai_classify('My password is leaked.', ARRAY('urgent', 'not urgent'));
-- 결과: urgent

-- 정보 추출 (Extraction)
-- 텍스트에서 원하는 정보(이름, 이메일 등) 추출
SELECT ai_extract('John Doe lives in New York', ARRAY('person', 'location'));
-- 결과: {"person": "John Doe", "location": "New York"}

-- 문법 교정 (Grammar Fix)
SELECT ai_fix_grammar('This sentence have some mistake');
-- 결과: 'This sentence has some mistakes'

-- 민감 정보 마스킹 (Masking)
SELECT ai_mask('My email is john.doe@example.com', ARRAY('email'));
-- 결과: 'My email is [MASKED]'
\end{lstlisting}

\newpage
% ----------------------------------------------------------------------
\section{실습: Lakeview 대시보드와 Genie}
% ----------------------------------------------------------------------

Databricks는 내장 대시보드 도구인 \textbf{Lakeview}와 대화형 AI 분석 도구인 \textbf{Genie}를 제공합니다.

\subsection{Lakeview 대시보드 (Lakeview Dashboards)}

\begin{itemize}
    \item \textbf{정의:} Databricks 플랫폼 내에서 직접 데이터를 시각화하고 대시보드를 구축하는 도구입니다.
    \item \textbf{구성:} '데이터' 탭에서 대시보드에 사용할 데이터를 정의하고, 캔버스에 '시각화 위젯', '텍스트 상자', '필터' 등을 추가하여 대시보드를 만듭니다.
    \item \textbf{자연어 생성:} 차트를 만들 때, 내장된 \textbf{Assistant}에게 자연어로 요청할 수 있습니다. (예: "show me total revenue by zip code")
\end{itemize}

\begin{boxWarning}
    \textbf{Lakeview vs. PowerBI/Tableau: 언제 무엇을 쓸까?}

    \begin{itemize}
        \item \textbf{Lakeview (Databricks 내장):}
        추가 라이선스 비용이 없습니다. 데이터 엔지니어, 분석가가 빠르게 데이터를 확인하고 공유하는 \textbf{'운영용' 대시보드}에 적합합니다.
        \item \textbf{PowerBI / Tableau (외부 BI 도구):}
        경영진(CEO 등)에게 보고하기 위한 매우 정교하고 복잡한(예: 3D 효과, 특정 브랜드 색상) \textbf{'전략적' 대시보드}에 적합합니다.
    \end{itemize}
\end{boxWarning}

\subsection{게시 (Publish) 및 공유 (Share)}

\begin{itemize}
    \item \textbf{공유 (Share):} 조직 내부의 다른 Databricks 사용자에게 대시보드 접근 권한(보기/편집)을 부여합니다.
    \item \textbf{게시 (Publish):} Databricks 계정이 \textbf{없는} 외부 사용자에게 대시보드를 공유하는 기능입니다.
    \begin{itemize}
        \item \textbf{'자격 증명 포함(Embed credentials)'} 옵션으로 게시합니다.
        \item \textbf{비용 주의:} 게시된 대시보드의 쿼리가 캐시(Cache)되지 않은 상태에서 외부 사용자가 대시보드를 조회하면, \textbf{대시보드를 게시한 사람(Publisher)의 계정에서 컴퓨트 비용이 발생}합니다.
    \end{itemize}
\end{itemize}

\subsection{Databricks Genie: 대화형 AI 분석}

\textbf{Genie}는 \textit{게시된} 대시보드에서 사용할 수 있는 대화형 AI 비서입니다.

\begin{itemize}
    \item \textbf{사용 시나리오:}
    BI 분석가가 만든 대시보드를 현업 사용자(예: 마케팅 매니저)가 보다가 추가적인 궁금증이 생겼습니다. (예: "이 데이터에서 20대 남성 고객만 보면 어떨까?")
    
    과거에는 이 요청을 BI팀에 보내고 답변까지 몇 주를 기다려야 했지만, 이제는 Genie에게 \textbf{자연어로 직접 질문}하여 즉시 답을 얻을 수 있습니다.

    \item \textbf{작동 방식 (예시):}
    \begin{enumerate}
        \item \textbf{사용자 질문:} "평균 여행 시간은 얼마야?"
        \item \textbf{Genie 답변:} "평균 13.7분입니다."
        \item \textbf{투명성 (Transparency):} Genie는 이 답을 얻기 위해 실행한 \textbf{SQL 쿼리를 함께 보여줍니다.} (예: \texttt{SELECT avg(dropoff\_time - pickup\_time) ...})
    \end{enumerate}

    \item \textbf{환각(Hallucination)이 없는 이유:}
    Genie는 ChatGPT 같은 범용 챗봇이 아닙니다. Genie의 답변 범위는 대시보드를 생성할 때 사용된 \textbf{특정 테이블의 메타데이터(컬럼명, 타입 등)로 엄격하게 제한}됩니다.
    
    만약 사용자가 "날씨가 어때?"처럼 데이터와 무관한 질문을 하면, Genie는 "죄송합니다. 저는 해당 데이터에 대해서만 답변할 수 있습니다"라고 답하며 환각을 일으키지 않습니다.
\end{itemize}

\newpage
% ----------------------------------------------------------------------
\section{학습 점검 체크리스트}
% ----------------------------------------------------------------------

이 강의의 핵심 내용을 잘 이해했는지 다음 항목들로 점검해 보세요.

\begin{itemize}
    \item [ ] 데이터 웨어하우스(DW)와 데이터 레이크(DL)의 4가지 주요 차이점(데이터, 스키마, 비용, 용도)을 설명할 수 있는가?
    \item [ ] "레이크하우스" 아키텍처가 왜 등장했으며, 기존 2-tier (Lake + DW) 아키텍처의 어떤 문제를 해결하는가? (힌트: 데이터 중복, 복잡성)
    \item [ ] BI와 BA의 차이점을 "질문"과 "목적" 관점에서 비교 설명할 수 있는가?
    \item [ ] BI 분석가 페르소나의 핵심 기술은 무엇인가? (정답: SQL)
    \item [ ] 데이터 모델링에서 킴볼(Kimball)의 "스타 스키마"가 무엇인지 팩트/디멘션 테이블로 설명할 수 있는가?
    \item [ ] "데이터 연합(Data Federation)"과 "ETL"의 가장 큰 차이점은 무엇인가? (정답: 데이터 소유권/이동 여부)
    \item [ ] "뷰(View)"와 "구체화된 뷰(Materialized View)"의 차이점(저장 공간, 성능)을 설명할 수 있는가?
    \item [ ] Databricks SQL의 \texttt{ai\_query()}와 \texttt{ai\_analyze\_sentiment()}의 차이점과 용도를 아는가?
    \item [ ] Lakeview 대시보드를 PowerBI 대신 사용하는 이유는 무엇인가? (운영용, 라이선스 비용 없음)
    \item [ ] Databricks Genie가 챗GPT와 달리 "환각(Hallucination)"을 일으키지 않는 이유는 무엇인가? (정답: 지정된 테이블 메타데이터만 참조)
    \item [ ] 외부 사용자와 대시보드를 "게시(Publish)" 기능으로 공유할 때 발생할 수 있는 비용 문제는 무엇인가?
\end{itemize}

% ----------------------------------------------------------------------
\section{FAQ (자주 묻는 질문)}
% ----------------------------------------------------------------------

\begin{boxExample}
    \textbf{Q: 데이터 레이크가 있는데 왜 굳이 데이터 웨어하우스가 필요한가요?}

    \textbf{A:} 데이터 레이크는 모든 데이터를 *저장*하는 데는 뛰어나지만, 정제되지 않아 *분석*하기에는 느리고 복잡합니다. 데이터 웨어하우스는 BI 리포팅 및 대시보드처럼 *매우 빠른 응답 속도*(Low Latency)와 *높은 동시성*(High Concurrency)이 필요한 BI 워크로드에 특화되어 있습니다. 레이크하우스는 이 두 장점을 결합하려는 시도입니다.
\end{boxExample}

\begin{boxExample}
    \textbf{Q: 레이크하우스는 그냥 마케팅 용어 아닌가요? 데이터 레이크와 뭐가 다른가요?}

    \textbf{A:} 레이크하우스는 데이터 레이크의 저렴한 저장소(S3, ADLS 등) 위에 ACID 트랜잭션, 데이터 버전 관리, 인덱싱 등 웨어하우스의 핵심 기능을 제공하는 *기술적 아키텍처*(예: Delta Lake)입니다. 덕분에 별도의 DW 없이 레이크에서 직접 BI와 ML을 모두 수행할 수 있습니다.
\end{boxExample}

\begin{boxExample}
    \textbf{Q: '데이터 연합(Federation)'은 항상 ETL보다 좋은 것 아닌가요?}

    \textbf{A:} 아닙니다. 연합은 데이터를 복제/이동하지 않아 편리하지만, 실시간으로 원격 시스템에 쿼리를 날립니다. 이는 *소량의 데이터*나 *참조용 데이터*(lookup)에는 좋지만, 대용량 데이터를 조인하거나 빠른 성능이 필요할 때는 매우 비효율적입니다. 이 경우 ETL을 통해 데이터를 레이크하우스로 가져오는 것이 성능상 유리합니다.
\end{boxExample}

\begin{boxExample}
    \textbf{Q: Genie가 SQL을 생성해준다면, 이제 SQL을 배울 필요가 없나요?}

    \textbf{A:} 아닙니다. Genie는 *보조* 도구입니다. Genie가 생성한 SQL이 100\% 정확하지 않을 수 있으며, 복잡한 비즈니스 로직을 구현하려면 여전히 SQL 지식이 필수입니다. Genie가 생성한 SQL을 *검증하고 수정*할 수 있어야 합니다.
\end{boxExample}

\newpage
% ----------------------------------------------------------------------
\section{빠르게 훑어보기 (1-Page Summary)}
% ----------------------------------------------------------------------

\begin{tcolorbox}[
    title=저장소 비교: 웨어하우스 vs. 레이크,
    colback=blue!5, colframe=blue!60, breakable,
    ]
    \begin{itemize}
        \item \textbf{웨어하우스(DW):} 깐깐한 도서관 (정형 데이터, Schema-on-Write, BI/SQL 최적화, 고비용, 고성능)
        \item \textbf{레이크(Lake):} 막 쌓는 차고 (모든 데이터, Schema-on-Read, ML/DS 최적화, 저비용, 저성능)
    \end{itemize}
\end{tcolorbox}

\begin{tcolorbox}[
    title=레이크하우스 (Lakehouse): 두 세계의 통합,
    colback=green!5, colframe=green!60, breakable,
    ]
    \textbf{레이크의 저렴한 스토리지 + 웨어하우스의 성능/안정성/거버넌스}
    \begin{itemize}
        \item 단일 시스템에서 BI와 ML/DS 워크로드를 모두 지원.
        \item 데이터 중복 및 ETL 파이프라인 복잡성 해소.
    \end{itemize}
\end{tcolorbox}

\begin{tcolorbox}[
    title=질문의 차이: BI vs. BA,
    colback=orange!5, colframe=orange!60, breakable,
    ]
    \begin{itemize}
        \item \textbf{BI (Business Intelligence):} "무엇이 일어났나?" (과거/현재 리포팅, 대시보드)
        \item \textbf{BA (Business Analytics):} "왜 일어났나? 무엇이 일어날까?" (통계, 예측 모델링)
    \end{itemize}
\end{tcolorbox}

\begin{tcolorbox}[
    title=BI 분석가와 데이터 모델링,
    colback=purple!5, colframe=purple!60, breakable,
    ]
    \begin{itemize}
        \item \textbf{BI 페르소나:} 핵심 기술은 \textbf{SQL}.
        \item \textbf{데이터 모델링:} \textbf{Star Schema} (중앙의 팩트 테이블 + 주변의 디멘션 테이블)가 BI 쿼리 성능에 가장 효율적임. (킴볼 방식)
    \end{itemize}
\end{tcolorbox}

\begin{tcolorbox}[
    title=Databricks SQL 핵심 기능,
    colback=gray!5, colframe=gray!60, breakable,
    ]
    \begin{itemize}
        \item \textbf{서버리스 컴퓨트:} 클러스터 부팅 대기 시간 없음 (Instant Compute).
        \item \textbf{Materialized Views (MV):} 복잡한 쿼리 결과를 미리 계산/저장하여 대시보드 속도 향상.
        \item \textbf{Streaming Tables:} SQL만으로 스트리밍 데이터 처리.
    \end{itemize}
\end{tcolorbox}

\begin{tcolorbox}[
    title=SQL + AI: 지능형 쿼리,
    colback=red!5, colframe=red!60, breakable,
    ]
    \begin{itemize}
        \item \textbf{ai\_query():} SQL 내에서 외부 LLM (GPT 등) 호출.
        \item \textbf{ai\_analyze\_sentiment(), ai\_classify()...:} 내장 AI 함수로 감성분석, 분류 등을 SQL로 즉시 수행.
    \end{itemize}
\end{tcolorbox}

\begin{tcolorbox}[
    title=대시보드와 AI 비서: Lakeview & Genie,
    colback=teal!5, colframe=teal!60, breakable,
    ]
    \begin{itemize}
        \item \textbf{Lakeview:} Databricks 내장 대시보드 (운영용, 라이선스 무료).
        \item \textbf{Genie:} 게시된 대시보드에서 사용하는 \textbf{대화형 AI}. 자연어 질문을 SQL로 변환.
        \item \textbf{Genie의 특징:} 환각(Hallucination) 없음. 지정된 테이블의 메타데이터만 참조.
    \end{itemize}
\end{tcolorbox}

\end{document}
