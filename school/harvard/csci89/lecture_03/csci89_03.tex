%%%%%%%%%%%%%%%%%%%%%%%%%%%%%%%%%%%%%%%%%%%%%%%%%%%%%%%%%%%%%%%%%%%%%%%%%%%%%%%
% Harvard Academic Notes - 통합 마스터 템플릿
% 모든 강의 노트에 적용되는 통일된 스타일
% 버전: 2.1 - 가독성 개선 (선택적 최적화)
% 최종 수정일: 2025-11-17
%%%%%%%%%%%%%%%%%%%%%%%%%%%%%%%%%%%%%%%%%%%%%%%%%%%%%%%%%%%%%%%%%%%%%%%%%%%%%%%

\documentclass[11pt,a4paper]{article}

%========================================================================================
% 기본 패키지
%========================================================================================

% --- 한국어 지원 ---
\usepackage{kotex}

% --- 페이지 레이아웃 ---
\usepackage[top=20mm, bottom=20mm, left=20mm, right=18mm]{geometry}
\usepackage{setspace}
\onehalfspacing                      % 1.5배 줄간격
\setlength{\parskip}{0.5em}          % 문단 간격
\setlength{\parindent}{0pt}          % 들여쓰기 없음

% --- 표 관련 ---
\usepackage{booktabs}              % 고품질 표
\usepackage{tabularx}              % 자동 너비 조절 표
\usepackage{array}                 % 표 컬럼 확장
\usepackage{longtable}             % 여러 페이지 표
\renewcommand{\arraystretch}{1.1}  % 표 행간 조절

%========================================================================================
% 헤더 및 푸터
%========================================================================================

\usepackage{fancyhdr}
\pagestyle{fancy}
\fancyhf{}
\fancyhead[L]{\small\textit{CSCI E-89B: 자연어 처리 입문}}
\fancyhead[R]{\small\textit{Lecture 03}}
\fancyfoot[C]{\thepage}
\renewcommand{\headrulewidth}{0.5pt}
\renewcommand{\footrulewidth}{0.3pt}

% 첫 페이지는 헤더 없음
\fancypagestyle{firstpage}{
    \fancyhf{}
    \fancyfoot[C]{\thepage}
    \renewcommand{\headrulewidth}{0pt}
}

%========================================================================================
% 색상 정의 (파스텔 톤 + 다크모드 호환)
%========================================================================================

\usepackage[dvipsnames]{xcolor}

% 밝은 배경용 파스텔 색상
\definecolor{lightblue}{RGB}{220, 235, 255}      % 부드러운 파랑
\definecolor{lightgreen}{RGB}{220, 255, 235}     % 부드러운 초록
\definecolor{lightyellow}{RGB}{255, 250, 220}    % 부드러운 노랑
\definecolor{lightpurple}{RGB}{240, 230, 255}    % 부드러운 보라
\definecolor{lightgray}{gray}{0.95}              % 밝은 회색
\definecolor{lightpink}{RGB}{255, 235, 245}      % 부드러운 핑크
\definecolor{boxgray}{gray}{0.95}
\definecolor{boxblue}{rgb}{0.9, 0.95, 1.0}
\definecolor{boxred}{rgb}{1.0, 0.95, 0.95}

% 진한 색상 (테두리/제목용)
\definecolor{darkblue}{RGB}{50, 80, 150}
\definecolor{darkgreen}{RGB}{40, 120, 70}
\definecolor{darkorange}{RGB}{200, 100, 30}
\definecolor{darkpurple}{RGB}{100, 60, 150}

%========================================================================================
% 박스 환경 (tcolorbox) - 6가지 타입
%========================================================================================

\usepackage[most]{tcolorbox}
\tcbuselibrary{skins, breakable}

% 1. 개요 박스 (강의 시작 부분)
\newtcolorbox{overviewbox}[1][]{
    enhanced,
    colback=lightpurple,
    colframe=darkpurple,
    fonttitle=\bfseries\large,
    title=📚 강의 개요,
    arc=3mm,
    boxrule=1pt,
    left=8pt,
    right=8pt,
    top=8pt,
    bottom=8pt,
    breakable,
    #1
}

% 2. 요약 박스
\newtcolorbox{summarybox}[1][]{
    enhanced,
    colback=lightblue,
    colframe=darkblue,
    fonttitle=\bfseries,
    title=📝 핵심 요약,
    arc=2mm,
    boxrule=0.7pt,
    left=6pt,
    right=6pt,
    top=6pt,
    bottom=6pt,
    breakable,
    #1
}

% 3. 핵심 정보 박스
\newtcolorbox{infobox}[1][]{
    enhanced,
    colback=lightgreen,
    colframe=darkgreen,
    fonttitle=\bfseries,
    title=💡 핵심 정보,
    arc=2mm,
    boxrule=0.7pt,
    left=6pt,
    right=6pt,
    top=6pt,
    bottom=6pt,
    breakable,
    #1
}

% 4. 주의사항 박스
\newtcolorbox{warningbox}[1][]{
    enhanced,
    colback=lightyellow,
    colframe=darkorange,
    fonttitle=\bfseries,
    title=⚠️ 주의사항,
    arc=2mm,
    boxrule=0.7pt,
    left=6pt,
    right=6pt,
    top=6pt,
    bottom=6pt,
    breakable,
    #1
}

% 5. 예제 박스
\newtcolorbox{examplebox}[1][]{
    enhanced,
    colback=lightgray,
    colframe=black!60,
    fonttitle=\bfseries,
    title=📖 예제: #1,
    arc=2mm,
    boxrule=0.7pt,
    left=6pt,
    right=6pt,
    top=6pt,
    bottom=6pt,
    breakable,
}

% 6. 정의 박스
\newtcolorbox{definitionbox}[1][]{
    enhanced,
    colback=lightpink,
    colframe=purple!70!black,
    fonttitle=\bfseries,
    title=📌 정의: #1,
    arc=2mm,
    boxrule=0.7pt,
    left=6pt,
    right=6pt,
    top=6pt,
    bottom=6pt,
    breakable,
}

% 7. 중요 박스 (importantbox - warningbox와 유사)
\newtcolorbox{importantbox}[1][]{
    enhanced,
    colback=boxred,
    colframe=red!70!black,
    fonttitle=\bfseries,
    title=⚠️ 매우 중요: #1,
    arc=2mm,
    boxrule=0.7pt,
    left=6pt,
    right=6pt,
    top=6pt,
    bottom=6pt,
    breakable,
}

% 8. cautionbox (warningbox와 동일)
\let\cautionbox\warningbox
\let\endcautionbox\endwarningbox

%========================================================================================
% 코드 블록 설정 (밝은 배경)
%========================================================================================

\usepackage{listings}

\definecolor{codegray}{rgb}{0.5,0.5,0.5}
\definecolor{codepurple}{rgb}{0.58,0,0.82}
\definecolor{backcolour}{rgb}{0.95,0.95,0.95}

\lstset{
    basicstyle=\ttfamily\small,
    backgroundcolor=\color{lightgray},
    keywordstyle=\color{darkblue}\bfseries,
    commentstyle=\color{darkgreen}\itshape,
    stringstyle=\color{purple!80!black},
    numberstyle=\tiny\color{black!60},
    numbers=left,
    numbersep=8pt,
    breaklines=true,
    breakatwhitespace=false,
    frame=single,
    frameround=tttt,
    rulecolor=\color{black!30},
    captionpos=b,
    showstringspaces=false,
    tabsize=2,
    xleftmargin=15pt,
    xrightmargin=5pt,
    escapeinside={\%*}{*)}
}

% Python 코드 스타일
\lstdefinestyle{pythonstyle}{
    language=Python,
    morekeywords={self, True, False, None},
}

% SQL 코드 스타일
\lstdefinestyle{sqlstyle}{
    language=SQL,
    morekeywords={SELECT, FROM, WHERE, JOIN, GROUP, BY, ORDER, HAVING},
}

%========================================================================================
% 목차 스타일링
%========================================================================================

\usepackage{tocloft}
\renewcommand{\cftsecleader}{\cftdotfill{\cftdotsep}}
\setlength{\cftbeforesecskip}{0.4em}
\renewcommand{\cftsecfont}{\bfseries}
\renewcommand{\cftsubsecfont}{\normalfont}

%========================================================================================
% 표 및 그림
%========================================================================================

\usepackage{graphicx}              % 이미지
\usepackage{adjustbox}             % 표/박스 크기 조절

% 표 캡션 스타일
\usepackage{caption}
\captionsetup[table]{
    labelfont=bf,
    textfont=it,
    skip=5pt
}
\captionsetup[figure]{
    labelfont=bf,
    textfont=it,
    skip=5pt
}

%========================================================================================
% 수학
%========================================================================================

\usepackage{amsmath, amssymb, amsthm}

% 정리 환경
\theoremstyle{definition}
\newtheorem{theorem}{정리}[section]
\newtheorem{lemma}[theorem]{보조정리}
\newtheorem{proposition}[theorem]{명제}
\newtheorem{corollary}[theorem]{따름정리}
\newtheorem{definition}{정의}[section]
\newtheorem{example}{예제}[section]

%========================================================================================
% 하이퍼링크
%========================================================================================

\usepackage[
    colorlinks=true,
    linkcolor=blue!80!black,
    urlcolor=blue!80!black,
    citecolor=green!60!black,
    bookmarks=true,
    bookmarksnumbered=true,
    pdfborder={0 0 0}
]{hyperref}

% PDF 메타데이터는 각 문서에서 설정
\hypersetup{
    pdftitle={CSCI E-89B: 자연어 처리 입문 - Lecture 03},
    pdfauthor={강의 노트},
    pdfsubject={Academic Notes}
}

%========================================================================================
% 기타 유용한 패키지
%========================================================================================

\usepackage{enumitem}              % 리스트 커스터마이징
\setlist{nosep, leftmargin=*, itemsep=0.3em}

\usepackage{microtype}             % 타이포그래피 개선
\usepackage{footnote}              % 각주 개선
\usepackage{url}                   % URL 줄바꿈
\urlstyle{same}

%========================================================================================
% 사용자 정의 명령어
%========================================================================================

% 강조 텍스트
\newcommand{\important}[1]{\textbf{\textcolor{red!70!black}{#1}}}
\newcommand{\keyword}[1]{\textbf{#1}}
\newcommand{\term}[1]{\textit{#1}}
\newcommand{\code}[1]{\texttt{#1}}

% 용어 설명 (인라인)
\newcommand{\defterm}[2]{\textbf{#1}\footnote{#2}}

% 섹션 시작 전 페이지 분리
\newcommand{\newsection}[1]{\newpage\section{#1}}

%========================================================================================
% 문서 제목 스타일
%========================================================================================

\usepackage{titling}
\pretitle{\begin{center}\LARGE\bfseries}
\posttitle{\par\end{center}\vskip 0.5em}
\preauthor{\begin{center}\large}
\postauthor{\end{center}}
\predate{\begin{center}\large}
\postdate{\par\end{center}}

%========================================================================================
% 섹션 제목 간격
%========================================================================================

\usepackage{titlesec}
\titlespacing*{\section}{0pt}{1.5em}{0.8em}
\titlespacing*{\subsection}{0pt}{1.2em}{0.6em}
\titlespacing*{\subsubsection}{0pt}{1em}{0.5em}

%========================================================================================
% 메타 정보 박스 명령어
%========================================================================================

\newcommand{\metainfo}[4]{
\begin{tcolorbox}[
    colback=lightpurple,
    colframe=darkpurple,
    boxrule=1pt,
    arc=2mm,
    left=10pt,
    right=10pt,
    top=8pt,
    bottom=8pt
]
\begin{tabular}{@{}rl@{}}
▣ \textbf{강의명:} & #1 \\[0.3em]
▣ \textbf{주차:} & #2 \\[0.3em]
▣ \textbf{교수명:} & #3 \\[0.3em]
▣ \textbf{목적:} & \begin{minipage}[t]{0.75\textwidth}#4\end{minipage}
\end{tabular}
\end{tcolorbox}
}

%========================================================================================
% 끝
%========================================================================================


\begin{document}

\maketitle
\thispagestyle{firstpage}

\metainfo{CSCI E-89B: 자연어 처리 입문}{Lecture 03}{Dmitry Kurochkin}{Lecture 03의 핵심 개념 학습}


\tableofcontents

\newpage
\section{개요: 자연어 처리란 무엇인가?}

\begin{summarybox}
자연어 처리(Natural Language Processing, NLP)는 컴퓨터가 인간의 언어를 이해하고, 해석하며, 생성하게 만드는 기술 분야입니다.
이 분야는 언어학(Linguistics)과 인공지능(Artificial Intelligence, AI)이 만나는 지점에 있습니다.
궁극적인 목표는 인간과 컴퓨터가 보다 자연스럽게 소통하는 것입니다.
NLP는 텍스트 분류, 기계 번역, 챗봇 등 다양한 응용 분야의 핵심 기술입니다.
성공적인 NLP 모델을 구축하려면 텍스트를 컴퓨터가 이해할 수 있는 형태로 가공하는 전처리 과정이 매우 중요합니다.
\end{summarybox}

\subsection{인공지능, 머신러닝, 딥러닝, NLP의 관계}

이 네 가지 개념은 종종 혼용되지만, 포함 관계를 이해하는 것이 중요합니다.

\begin{itemize}
    \item \textbf{인공지능(AI)}: 가장 넓은 개념으로, 인간의 지능을 모방하는 모든 기술을 포함합니다. 여기에는 규칙 기반 시스템(rule-based system)처럼 전문가가 직접 규칙을 코딩하는 방식도 포함됩니다.
    \item \textbf{머신러닝(ML)}: AI의 한 분야로, 데이터로부터 컴퓨터가 스스로 '규칙'을 학습하게 하는 접근 방식입니다. 데이터를 입력하면 모델이 패턴을 찾아내 예측이나 분류를 수행합니다.
    \item \textbf{딥러닝(DL)}: 머신러닝의 한 분야로, 인간의 뇌 신경망을 모방한 심층 신경망(Deep Neural Network)을 사용합니다. 특히 이미지, 음성, 텍스트와 같은 비정형 데이터 처리에서 뛰어난 성능을 보입니다.
    \item \textbf{자연어 처리(NLP)}: AI의 한 분야로, 인간 언어에 특화된 기술입니다. NLP 문제를 해결하기 위해 규칙 기반 접근법, 전통적인 머신러닝, 그리고 최근에는 딥러닝 방법론이 활발히 사용됩니다.
\end{itemize}

% 아래는 슬라이드의 벤 다이어그램을 텍스트로 설명한 부분입니다.
\begin{tcolorbox}[title=개념 간의 관계 (벤 다이어그램 설명)]
    가장 큰 원인 \textbf{인공지능(AI)} 안에 \textbf{머신러닝(ML)}이 포함됩니다. \\
    머신러닝 원 안에 \textbf{딥러닝(DL)}이 더 작은 원으로 포함됩니다. \\
    \textbf{자연어 처리(NLP)}는 AI의 큰 원 안에 있으면서, 머신러닝과 딥러닝 영역과 상당 부분 겹칩니다. 이는 NLP가 AI의 다양한 기술을 활용한다는 것을 의미합니다.
\end{tcolorbox}


%========================================================================================
% SECTION 2: 용어 정리
%========================================================================================
\newpage
\section{핵심 용어 정리}

\begin{tabular}{@{}lp{6cm}p{3cm}p{3cm}@{}}
\toprule
\textbf{용어} & \textbf{쉬운 설명} & \textbf{원어} & \textbf{비고} \\
\midrule
\textbf{토큰화} & 문장을 단어나 문장 같은 의미 있는 작은 단위(토큰)로 쪼개는 과정 & Tokenization & NLP 전처리의 가장 첫 단계 \\
\textbf{어간 추출} & 단어의 접사(접두사, 접미사)를 제거하여 어간(stem)을 추출하는 과정 & Stemming & 빠르지만, 결과가 사전에 없는 단어일 수 있음 ('running' $\to$ 'run') \\
\textbf{표제어 추출} & 단어의 문법적 형태를 고려하여 기본형, 즉 표제어(lemma)를 찾는 과정 & Lemmatization & 문맥을 파악해야 하므로 느리지만, 결과가 사전에 있는 단어임 ('driving' $\to$ 'drive') \\
\textbf{임베딩} & 단어를 저차원의 밀집된 숫자 벡터(dense vector)로 변환하는 기술 & Embedding & 단어 간의 의미적 관계를 벡터 공간에 표현. 원-핫 인코딩의 단점을 보완. \\
\textbf{원-핫 인코딩} & 단어 사전에 있는 단어 중 하나만 1이고 나머지는 모두 0인 희소 벡터(sparse vector)로 표현하는 방식 & One-Hot Encoding & 단어 수가 많아지면 벡터 차원이 커지고, 단어 간 유사성을 표현하지 못함. \\
\textbf{불용어} & 문장에서 큰 의미를 갖지 않아 분석에 불필요한 단어 (조사, 관사 등) & Stop Words & 전처리 과정에서 제거하여 계산 효율성을 높임. (예: a, the, is, 의, 는) \\
\textbf{OOV} & 훈련 데이터의 단어 사전에 없어 알 수 없는 단어 & Out-of-Vocabulary & 테스트 시 새로운 단어가 나타날 때를 대비해 특별 토큰으로 처리. \\
\bottomrule
\end{tabular}

%========================================================================================
% SECTION 3: 핵심 개념/원리
%========================================================================================
\newpage
\section{핵심 개념 및 원리}

\subsection{텍스트 전처리(Text Preprocessing)의 중요성}
컴퓨터는 텍스트를 그대로 이해할 수 없습니다. 따라서 모델에 입력하기 전에 텍스트를 숫자 형태의 데이터로 변환하고 정제하는 과정이 필수적입니다. 이 과정을 '전처리'라고 부릅니다.

\begin{itemize}
    \item \textbf{목표}: 분석에 불필요한 노이즈를 제거하고, 텍스트의 핵심 정보를 보존하며, 모델이 학습하기 좋은 형태로 데이터를 가공하는 것.
    \item \textbf{주요 단계}: 토큰화(Tokenization) $\to$ 정제(Cleaning, 예: 불용어 제거) $\to$ 정규화(Normalization, 예: 어간/표제어 추출) $\to$ 벡터화(Vectorization).
\end{itemize}

\subsubsection{1. 토큰화 (Tokenization)}
텍스트를 최소 단위인 '토큰(token)'으로 분할하는 과정입니다. 어떤 단위를 토큰으로 삼을지에 따라 여러 기법이 있습니다.
\begin{itemize}
    \item \textbf{단어 토큰화 (Word Tokenization)}: 띄어쓰기나 구두점을 기준으로 텍스트를 단어 단위로 나눕니다.
    \begin{examplebox}
    \textbf{입력}: "Henry Ford's innovation, the assembly line." \\
    \textbf{결과}: \texttt{['Henry', 'Ford', "'s", 'innovation', ',', 'the', 'assembly', 'line', '.']}
    \end{examplebox}

    \item \textbf{문장 토큰화 (Sentence Tokenization)}: 마침표(.), 물음표(?) 등 문장 끝을 나타내는 기호를 기준으로 텍스트를 문장 단위로 나눕니다.
    \begin{examplebox}
    \textbf{입력}: "He created assembly lines. This revolutionized production." \\
    \textbf{결과}: \texttt{['He created assembly lines.', 'This revolutionized production.']}
    \end{examplebox}

    \item \textbf{서브워드 토큰화 (Subword Tokenization)}: 단어를 의미 있는 더 작은 단위(subword)로 나눕니다. OOV(Out-of-Vocabulary) 문제를 완화하는 데 효과적입니다.
\end{itemize}

\subsubsection{2. 어간 추출 (Stemming)과 표제어 추출 (Lemmatization)}
'running', 'runs', 'ran'과 같은 단어들은 형태는 다르지만 '달리다'라는 동일한 의미를 갖습니다. 이렇게 다양한 형태의 단어들을 하나의 기본 형태로 통일하여 단어 사전의 크기를 줄이고 모델의 일반화 성능을 높일 수 있습니다.

\begin{tcolorbox}[title=어간 추출 vs. 표제어 추출 비교]
    \begin{tabular}{@{}lll@{}}
    \toprule
    \textbf{특징} & \textbf{어간 추출 (Stemming)} & \textbf{표제어 추출 (Lemmatization)} \\
    \midrule
    \textbf{목표} & 단어의 어미를 잘라내어 어간(기본 형태)을 찾음 & 단어의 사전적 기본형(표제어)을 찾음 \\
    \textbf{방식} & 규칙 기반으로 접사를 기계적으로 제거 & 품사 등 문맥 정보를 활용하여 사전을 참조 \\
    \textbf{속도} & 빠름 & 느림 \\
    \textbf{결과} & 사전에 없는 단어가 될 수 있음 & 항상 사전에 있는 단어 \\
    \textbf{예시} & "driving" $\to$ "driv" & "driving" $\to$ "drive" \\
     & "transportation" $\to$ "transport" & "was" $\to$ "be" \\
     & "electric" $\to$ "electr" & "cars" $\to$ "car" \\
    \bottomrule
    \end{tabular}
    
    \vspace{0.5em}
    \textbf{언제 무엇을 쓸까?}
    \begin{itemize}
        \item \textbf{어간 추출}: 검색 엔진의 인덱싱처럼 속도가 중요하고, 결과 단어의 언어적 정확성이 덜 중요한 경우.
        \item \textbf{표제어 추출}: 챗봇이나 기계 번역처럼 결과의 의미적 정확성이 매우 중요한 경우.
    \end{itemize}
\end{tcolorbox}

\subsubsection{3. 임베딩 (Embedding)}
텍스트를 벡터로 만드는 과정에서, 원-핫 인코딩은 단어 수만큼 차원이 커지고 단어 간 유사성을 표현하지 못하는 한계가 있습니다. 임베딩은 이러한 문제를 해결하기 위해 등장했습니다.

\begin{itemize}
    \item \textbf{핵심 아이디어}: 단어를 고차원의 희소 벡터(sparse vector)에서 저차원의 밀집 벡터(dense vector)로 변환하는 것.
    \item \textbf{과정}:
    \begin{enumerate}
        \item 단어 사전에 있는 각 단어에 대해 고유한 인덱스(정수)를 부여합니다.
        \item 신경망에 '임베딩 레이어'를 추가합니다. 이 레이어는 각 인덱스를 입력받아 특정 차원(예: 128차원)의 벡터로 매핑합니다.
        \item 이 매핑 가중치(lookup table)는 모델이 훈련하는 과정에서 다른 가중치들과 함께 학습됩니다. 결과적으로 의미가 비슷한 단어들은 벡터 공간에서 서로 가까운 위치에 배치됩니다.
    \end{enumerate}
    \item \textbf{장점}:
    \begin{itemize}
        \item \textbf{차원 축소}: 수만 개의 차원을 수백 개의 차원으로 줄여 계산 효율성을 높입니다.
        \item \textbf{의미 학습}: 단어의 문맥적 의미를 벡터에 담을 수 있습니다. (예: \texttt{king} - \texttt{man} + \texttt{woman} $\approx$ \texttt{queen})
        \item \textbf{훈련 가능}: 임베딩 벡터 자체가 모델의 파라미터로서 특정 과제에 맞게 최적화됩니다.
    \end{itemize}
\end{itemize}

\begin{examplebox}[title=임베딩의 직관적 예시]
    'female'과 'male'이라는 두 단어가 있다고 가정해봅시다.
    \begin{itemize}
        \item \textbf{원-핫 인코딩}: female: \texttt{[1, 0]}, male: \texttt{[0, 1]} (2차원)
        \item \textbf{임베딩 (1차원으로 축소)}: female: \texttt{[1]}, male: \texttt{[0]}
    \end{itemize}
    수만 개의 단어가 있는 실제 문제에서는, 10000차원의 원-핫 벡터를 128차원의 임베딩 벡터로 변환하는 것과 같습니다. 이 변환 과정은 훈련을 통해 최적의 값을 찾습니다.
\end{examplebox}

%========================================================================================
% SECTION 4: 절차/방법
%========================================================================================
\newpage
\section{NLP 텍스트 분류 절차}

다음은 \texttt{20 newsgroups} 데이터셋을 사용하여 '하키(hockey)'와 '판매(for sale)' 관련 게시물을 분류하는 모델을 구축하는 일반적인 절차입니다.

\begin{enumerate}
    \item \textbf{데이터 로드 및 준비}
    \begin{itemize}
        \item 훈련(train) 데이터와 테스트(test) 데이터를 로드합니다.
        \item 분류할 카테고리(예: \texttt{rec.sport.hockey}, \texttt{misc.forsale})를 지정합니다.
    \end{itemize}
    
    \item \textbf{텍스트 전처리 및 단어 사전 구축}
    \begin{itemize}
        \item \textbf{토큰화}: 모든 훈련 텍스트를 단어 토큰으로 분할합니다.
        \item \textbf{정규화 (선택)}: 어간 추출(stemming)이나 표제어 추출(lemmatization)을 적용하여 단어를 기본 형태로 통일합니다.
        \item \textbf{빈도 계산}: 각 단어의 등장 빈도를 계산합니다.
        \item \textbf{단어 사전 생성}: 가장 빈도가 높은 N개(예: 10,000개)의 단어만 선택하여 단어 사전을 만듭니다.
        \item \textbf{OOV 처리}: 사전에 없는 단어(Out-of-Vocabulary)를 처리하기 위해 특별 토큰(예: 인덱스 0)을 예약합니다.
    \end{itemize}
    
    \item \textbf{텍스트를 시퀀스로 변환}
    \begin{itemize}
        \item 각 텍스트(게시물)를 단어 사전에 따라 정수 인덱스의 시퀀스로 변환합니다.
        \item 예: "the game was fun" $\to$ \texttt{[5, 120, 25, 8]}
        \item OOV 단어는 예약된 인덱스(예: 0)로 변환합니다.
    \end{itemize}
    
    \item \textbf{패딩 (Padding)}
    \begin{itemize}
        \item 신경망 모델에 입력하려면 모든 시퀀스의 길이가 동일해야 합니다.
        \item 최대 길이(예: 500)를 정하고, 이보다 짧은 시퀀스는 뒤쪽에 특정 값(보통 0)을 채워 길이를 맞춥니다.
    \end{itemize}
    
    \item \textbf{신경망 모델 구축}
    \begin{itemize}
        \item \textbf{임베딩 레이어}: 정수 인덱스를 입력받아 밀집 벡터로 변환합니다. (입력 차원: 단어 사전 크기, 출력 차원: 임베딩 차원)
        \item \textbf{순환 신경망 (RNN/LSTM) 레이어}: 시퀀스 데이터의 시간적 패턴을 학습합니다.
        \item \textbf{드롭아웃 (Dropout) 레이어}: 과적합(overfitting)을 방지하기 위해 훈련 중 일부 뉴런을 무작위로 비활성화합니다.
        \item \textbf{출력 레이어}: 최종적으로 분류 결과를 출력합니다. (이진 분류의 경우, Sigmoid 활성화 함수를 사용하는 하나의 뉴런)
    \end{itemize}
    
    \item \textbf{모델 훈련 및 평가}
    \begin{itemize}
        \item 모델을 컴파일합니다. (손실 함수: \texttt{binary_crossentropy}, 옵티마이저: \texttt{adam})
        \item 준비된 훈련 데이터로 모델을 훈련시킵니다.
        \item 훈련이 끝난 후, 테스트 데이터로 모델의 성능(예: 정확도)을 평가합니다.
    \end{itemize}
\end{enumerate}

%========================================================================================
% SECTION 5: 실습/코드
%========================================================================================
\newpage
\section{실습 코드 및 결과 분석}

\subsection{1. NLTK를 이용한 토큰화 및 어간 추출}

\subsubsection{토큰화 (Tokenization)}
NLTK(Natural Language Toolkit) 라이브러리를 사용하면 단어 및 문장 토큰화를 쉽게 수행할 수 있습니다.
\begin{codeexamplebox}[title=NLTK 단어 토큰화 코드]
\begin{lstlisting}[language=Python, caption=NLTK를 사용한 단어 토큰화, breaklines=true]
import nltk
from nltk.tokenize import word_tokenize

# NLTK 데이터 다운로드 (최초 1회 필요)
nltk.download('punkt')

text = "Henry Ford's innovation, the assembly line process, changed the car industry's dynamics profoundly."

# 단어로 토큰화
word_tokens = word_tokenize(text)
print(word_tokens)

# 결과:
# ['Henry', 'Ford', "'s", 'innovation', ',', 'the', 'assembly', 'line', 'process', ',', 'changed', 'the', 'car', 'industry', "'s", 'dynamics', 'profoundly', '.']
\end{lstlisting}
\end{codeexamplebox}

\subsubsection{어간 추출 (Stemming)}
가장 널리 쓰이는 Porter Stemmer와 Snowball Stemmer를 사용하여 토큰화된 단어들에 어간 추출을 적용할 수 있습니다.
\begin{codeexamplebox}[title=NLTK 어간 추출 코드]
\begin{lstlisting}[language=Python, caption=Porter 및 Snowball Stemmer 적용, breaklines=true]
from nltk.stem import PorterStemmer, SnowballStemmer

words = ['Henry', 'Ford', "'s", 'innovation', 'changed', 'industry', 'dynamics', 'profoundly']

# Porter Stemmer 적용
porter = PorterStemmer()
porter_stems = [porter.stem(word) for word in words]
print("Porter Stemmer:", porter_stems)
# 결과: ['henri', 'ford', "'s", 'innov', 'chang', 'industri', 'dynam', 'profoundli']

# Snowball Stemmer 적용 (Porter보다 개선됨)
snowball = SnowballStemmer("english")
snowball_stems = [snowball.stem(word) for word in words]
print("Snowball Stemmer:", snowball_stems)
# 결과: ['henri', 'ford', "'s", 'innov', 'chang', 'industri', 'dynam', 'profound']
\end{lstlisting}
\end{codeexamplebox}
\begin{warningbox}
    결과에서 볼 수 있듯이, 어간 추출은 단어를 \texttt{industri}나 \texttt{profoundli}처럼 사전에 존재하지 않는 형태로 만들 수 있습니다. 이는 단순히 규칙에 따라 접미사를 제거하기 때문입니다. \texttt{Henry}가 \texttt{Henri}로 변하는 등 고유명사에도 적용될 수 있습니다.
\end{warningbox}

\subsection{2. SpaCy를 이용한 표제어 추출}
SpaCy는 산업 수준의 성능을 제공하는 NLP 라이브러리로, 정확한 표제어 추출 기능을 제공합니다.
\begin{codeexamplebox}[title=SpaCy 표제어 추출 코드]
\begin{lstlisting}[language=Python, caption=SpaCy를 사용한 표제어 추출, breaklines=true]
import spacy

# SpaCy 모델 로드 (최초 1회 다운로드 필요)
# python -m spacy download en_core_web_sm
nlp = spacy.load("en_core_web_sm")

text = "Henry Ford's innovation changed the car industry's dynamics profoundly."
doc = nlp(text)

# 표제어 추출
spacy_lemmas = [token.lemma_ for token in doc]
print(spacy_lemmas)

# 결과:
# ['Henry', 'Ford', "'s", 'innovation', 'change', 'the', 'car', 'industry', "'s", 'dynamic', 'profoundly', '.']
\end{lstlisting}
\end{codeexamplebox}
\begin{tcolorbox}[title=결과 비교: Stemming vs. Lemmatization]
- \texttt{changed} $\to$ \texttt{chang} (Porter Stemmer) vs. \texttt{change} (SpaCy Lemmatizer)
- \texttt{dynamics} $\to$ \texttt{dynam} (Porter Stemmer) vs. \texttt{dynamic} (SpaCy Lemmatizer)

표제어 추출이 문법적으로 더 올바르고 해석 가능한 결과를 제공함을 알 수 있습니다.
\end{tcolorbox}


\subsection{3. 텍스트 분류 모델 결과 분석}
\texttt{20 newsgroups} 데이터셋으로 훈련한 LSTM 모델의 테스트 정확도는 전처리 방식에 따라 미세한 차이를 보였습니다.

\begin{itemize}
    \item \textbf{단순 토큰화만 적용}: 약 93.5\%의 테스트 정확도
    \item \textbf{어간 추출(Stemming) 적용}: 약 97\%의 테스트 정확도
    \item \textbf{표제어 추출(Lemmatization) 적용}: 약 95-96\%의 테스트 정확도
\end{itemize}

\begin{summarybox}[title=결과 해석]
이 특정 과제에서는 어간 추출을 적용했을 때 성능이 가장 좋았습니다. 이는 단어의 다양한 변형을 하나의 형태로 통일함으로써 모델이 더 적은 수의 특징으로 핵심 패턴을 학습할 수 있었기 때문일 수 있습니다. 표제어 추출은 어간 추출보다 더 정교하지만, 이로 인한 계산 비용 증가나 미미한 성능 차이로 인해 특정 상황에서는 어간 추출이 더 효율적일 수 있습니다.

\textbf{결론}: 어떤 전처리 기법이 최적인지는 데이터와 과제에 따라 다르므로, 여러 방법을 실험하고 검증 데이터셋에서의 성능을 비교하여 결정하는 것이 좋습니다.
\end{summarybox}

%========================================================================================
% SECTION 6: FAQ
%========================================================================================
\newpage
\section{자주 묻는 질문 (FAQ)}

\begin{tcolorbox}[title=Q1: 어간 추출과 표제어 추출 중 항상 더 좋은 방법이 있나요?]
\textbf{A:} 아니요, 항상 더 좋은 방법은 없습니다. 작업의 목표에 따라 선택이 달라집니다.
\begin{itemize}
    \item \textbf{속도가 중요하다면} (예: 대규모 문서 인덱싱) 어간 추출이 더 나은 선택일 수 있습니다.
    \item \textbf{의미의 정확성이 중요하다면} (예: 챗봇, 기계 번역) 표제어 추출이 필수적입니다.
\end{itemize}
실제 프로젝트에서는 두 가지 방법을 모두 시도해보고 성능이 더 잘 나오는 쪽을 선택하는 경우가 많습니다.
\end{tcolorbox}

\begin{tcolorbox}[title=Q2: 왜 원-핫 인코딩 대신 임베딩을 사용해야 하나요?]
\textbf{A:} 두 가지 주된 이유가 있습니다.
\begin{enumerate}
    \item \textbf{차원의 저주 회피}: 원-핫 인코딩은 단어 수가 많아지면 벡터의 차원이 수만, 수십만으로 커져 계산이 비효율적입니다. 임베딩은 이를 수백 차원의 밀집 벡터로 압축합니다.
    \item \textbf{의미 관계 학습}: 원-핫 벡터들은 모두 서로 직교하므로 단어 간 유사도를 계산할 수 없습니다. 임베딩은 훈련 과정에서 비슷한 의미의 단어들을 벡터 공간상에 가깝게 배치하여 의미적 관계를 학습합니다.
\end{enumerate}
\end{tcolorbox}

\begin{tcolorbox}[title=Q3: OOV(사전에 없는 단어)는 왜 중요하고 어떻게 처리하나요?]
\textbf{A:} 훈련 데이터에 없던 단어가 테스트 데이터에 나타나면 모델은 이를 처리할 수 없어 오류가 발생하거나 성능이 저하됩니다. 이것이 OOV 문제입니다.
\begin{itemize}
    \item \textbf{처리 방법}: 단어 사전을 만들 때 '알 수 없는 단어'를 의미하는 특별 토큰(예: \texttt{<UNK>} 또는 \texttt{<OOV>})을 추가합니다. 그리고 훈련 시에도 일부러 드물게 나타나는 단어들을 이 토큰으로 대체하여 모델이 OOV 상황에 대처하도록 학습시킬 수 있습니다. 테스트 시 사전에 없는 단어가 나오면 이 특별 토큰으로 매핑하여 처리합니다.
\end{itemize}
\end{tcolorbox}

\begin{tcolorbox}[title=Q4: 드롭아웃(Dropout)은 왜 사용하나요?]
\textbf{A:} 드롭아웃은 모델의 \textbf{과적합(Overfitting)}을 방지하기 위한 정규화(regularization) 기법입니다. 과적합은 모델이 훈련 데이터에만 너무 과도하게 맞춰져서, 새로운 데이터(테스트 데이터)에 대해서는 성능이 떨어지는 현상을 말합니다.
\begin{itemize}
    \item \textbf{작동 원리}: 훈련 과정에서 각 미니배치마다 신경망의 뉴런 중 일부를 확률적으로 비활성화(출력을 0으로 만듦)합니다. 이를 통해 모델이 특정 뉴런에 과도하게 의존하는 것을 막고, 여러 뉴런이 협력하여 더 강건한(robust) 특징을 학습하도록 유도합니다.
\end{itemize}
\end{tcolorbox}


%========================================================================================
% SECTION 7: 빠르게 훑어보기 (1페이지 요약)
%========================================================================================
\newpage
\section{한눈에 보는 핵심 요약}

\begin{tcolorbox}[title=NLP 기본 개념, enhanced, sharp corners, colback=yellow!10!white, colframe=yellow!50!black]
    \textbf{자연어 처리 (NLP)} \\
    컴퓨터가 인간의 언어를 다루게 하는 AI의 한 분야. 언어학과 컴퓨터 과학의 교차점.
\end{tcolorbox}

\begin{tcolorbox}[title=텍스트 전처리 3대장, enhanced, sharp corners, colback=blue!10!white, colframe=blue!50!black]
    \begin{tabular}{@{}ll@{}}
    \textbf{1. 토큰화 (Tokenization)} & 문장을 단어/문장 등 작은 단위로 쪼개기. \\
    \textbf{2. 어간 추출 (Stemming)} & 단어 끝을 잘라 기본형 찾기. 빠르지만 부정확할 수 있음. \\
    \textbf{3. 표제어 추출 (Lemmatization)} & 사전을 이용해 진짜 기본형(표제어) 찾기. 정확하지만 느림.
    \end{tabular}
\end{tcolorbox}

\begin{tcolorbox}[title=단어의 벡터화: 원-핫 인코딩 vs. 임베딩, enhanced, sharp corners, colback=green!10!white, colframe=green!50!black]
    \begin{tabular}{@{}p{3cm}p{6cm}p{5.5cm}@{}}
    & \textbf{원-핫 인코딩} & \textbf{임베딩} \\ \midrule
    \textbf{형태} & 희소 벡터 (대부분 0) & 밀집 벡터 (의미 있는 실수값) \\
    \textbf{차원} & 고차원 (단어 수만큼) & 저차원 (사용자 지정, 예: 128) \\
    \textbf{의미 표현} & 불가능 (모든 단어 독립적) & 가능 (비슷한 단어는 벡터 공간에서 가까움) \\
    \textbf{결론} & 간단하지만 한계 명확 & 차원 축소 및 의미 학습에 효과적
    \end{tabular}
\end{tcolorbox}

\begin{tcolorbox}[title=NLP 텍스트 분류 파이프라인, enhanced, sharp corners, colback=purple!10!white, colframe=purple!50!black]
    \textbf{데이터 로드} $\to$ \textbf{토큰화} $\to$ \textbf{정규화 (Stem/Lemma)} $\to$ \textbf{정수 인코딩} $\to$ \textbf{패딩} $\to$ \textbf{모델 훈련 (Embedding+RNN)} $\to$ \textbf{평가}
\end{tcolorbox}

\begin{tcolorbox}[title=핵심 모델 구성 요소, enhanced, sharp corners, colback=orange!10!white, colframe=orange!50!black]
    \begin{itemize}
        \item \textbf{LSTM (Long Short-Term Memory)}: 순서가 중요한 시퀀스 데이터(문장 등)를 잘 처리하는 RNN의 한 종류.
        \item \textbf{Dropout}: 훈련 데이터에 모델이 과하게 맞춰지는 과적합을 방지하는 기술.
        \item \textbf{이진 교차 엔트로피 (Binary Cross-Entropy)}: 두 개 중 하나를 맞추는 이진 분류 문제에서 주로 사용하는 손실 함수.
    \end{itemize}
\end{tcolorbox}

\end{document}
