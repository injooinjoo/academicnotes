%%%%%%%%%%%%%%%%%%%%%%%%%%%%%%%%%%%%%%%%%%%%%%%%%%%%%%%%%%%%%%%%%%%%%%%%%%%%%%%
% Harvard Academic Notes - 통합 마스터 템플릿
% 모든 강의 노트에 적용되는 통일된 스타일
% 버전: 2.1 - 가독성 개선 (선택적 최적화)
% 최종 수정일: 2025-11-17
%%%%%%%%%%%%%%%%%%%%%%%%%%%%%%%%%%%%%%%%%%%%%%%%%%%%%%%%%%%%%%%%%%%%%%%%%%%%%%%

\documentclass[11pt,a4paper]{article}

%========================================================================================
% 기본 패키지
%========================================================================================

% --- 한국어 지원 ---
\usepackage{kotex}

% --- 페이지 레이아웃 ---
\usepackage[top=20mm, bottom=20mm, left=20mm, right=18mm]{geometry}
\usepackage{setspace}
\onehalfspacing                      % 1.5배 줄간격
\setlength{\parskip}{0.5em}          % 문단 간격
\setlength{\parindent}{0pt}          % 들여쓰기 없음

% --- 표 관련 ---
\usepackage{booktabs}              % 고품질 표
\usepackage{tabularx}              % 자동 너비 조절 표
\usepackage{array}                 % 표 컬럼 확장
\usepackage{longtable}             % 여러 페이지 표
\renewcommand{\arraystretch}{1.1}  % 표 행간 조절

%========================================================================================
% 헤더 및 푸터
%========================================================================================

\usepackage{fancyhdr}
\pagestyle{fancy}
\fancyhf{}
\fancyhead[L]{\small\textit{CSCI E-89B: 자연어 처리 입문}}
\fancyhead[R]{\small\textit{Lecture 10}}
\fancyfoot[C]{\thepage}
\renewcommand{\headrulewidth}{0.5pt}
\renewcommand{\footrulewidth}{0.3pt}

% 첫 페이지는 헤더 없음
\fancypagestyle{firstpage}{
    \fancyhf{}
    \fancyfoot[C]{\thepage}
    \renewcommand{\headrulewidth}{0pt}
}

%========================================================================================
% 색상 정의 (파스텔 톤 + 다크모드 호환)
%========================================================================================

\usepackage[dvipsnames]{xcolor}

% 밝은 배경용 파스텔 색상
\definecolor{lightblue}{RGB}{220, 235, 255}      % 부드러운 파랑
\definecolor{lightgreen}{RGB}{220, 255, 235}     % 부드러운 초록
\definecolor{lightyellow}{RGB}{255, 250, 220}    % 부드러운 노랑
\definecolor{lightpurple}{RGB}{240, 230, 255}    % 부드러운 보라
\definecolor{lightgray}{gray}{0.95}              % 밝은 회색
\definecolor{lightpink}{RGB}{255, 235, 245}      % 부드러운 핑크
\definecolor{boxgray}{gray}{0.95}
\definecolor{boxblue}{rgb}{0.9, 0.95, 1.0}
\definecolor{boxred}{rgb}{1.0, 0.95, 0.95}

% 진한 색상 (테두리/제목용)
\definecolor{darkblue}{RGB}{50, 80, 150}
\definecolor{darkgreen}{RGB}{40, 120, 70}
\definecolor{darkorange}{RGB}{200, 100, 30}
\definecolor{darkpurple}{RGB}{100, 60, 150}

%========================================================================================
% 박스 환경 (tcolorbox) - 6가지 타입
%========================================================================================

\usepackage[most]{tcolorbox}
\tcbuselibrary{skins, breakable}

% 1. 개요 박스 (강의 시작 부분)
\newtcolorbox{overviewbox}[1][]{
    enhanced,
    colback=lightpurple,
    colframe=darkpurple,
    fonttitle=\bfseries\large,
    title=📚 강의 개요,
    arc=3mm,
    boxrule=1pt,
    left=8pt,
    right=8pt,
    top=8pt,
    bottom=8pt,
    breakable,
    #1
}

% 2. 요약 박스
\newtcolorbox{summarybox}[1][]{
    enhanced,
    colback=lightblue,
    colframe=darkblue,
    fonttitle=\bfseries,
    title=📝 핵심 요약,
    arc=2mm,
    boxrule=0.7pt,
    left=6pt,
    right=6pt,
    top=6pt,
    bottom=6pt,
    breakable,
    #1
}

% 3. 핵심 정보 박스
\newtcolorbox{infobox}[1][]{
    enhanced,
    colback=lightgreen,
    colframe=darkgreen,
    fonttitle=\bfseries,
    title=💡 핵심 정보,
    arc=2mm,
    boxrule=0.7pt,
    left=6pt,
    right=6pt,
    top=6pt,
    bottom=6pt,
    breakable,
    #1
}

% 4. 주의사항 박스
\newtcolorbox{warningbox}[1][]{
    enhanced,
    colback=lightyellow,
    colframe=darkorange,
    fonttitle=\bfseries,
    title=⚠️ 주의사항,
    arc=2mm,
    boxrule=0.7pt,
    left=6pt,
    right=6pt,
    top=6pt,
    bottom=6pt,
    breakable,
    #1
}

% 5. 예제 박스
\newtcolorbox{examplebox}[1][]{
    enhanced,
    colback=lightgray,
    colframe=black!60,
    fonttitle=\bfseries,
    title=📖 예제: #1,
    arc=2mm,
    boxrule=0.7pt,
    left=6pt,
    right=6pt,
    top=6pt,
    bottom=6pt,
    breakable,
}

% 6. 정의 박스
\newtcolorbox{definitionbox}[1][]{
    enhanced,
    colback=lightpink,
    colframe=purple!70!black,
    fonttitle=\bfseries,
    title=📌 정의: #1,
    arc=2mm,
    boxrule=0.7pt,
    left=6pt,
    right=6pt,
    top=6pt,
    bottom=6pt,
    breakable,
}

% 7. 중요 박스 (importantbox - warningbox와 유사)
\newtcolorbox{importantbox}[1][]{
    enhanced,
    colback=boxred,
    colframe=red!70!black,
    fonttitle=\bfseries,
    title=⚠️ 매우 중요: #1,
    arc=2mm,
    boxrule=0.7pt,
    left=6pt,
    right=6pt,
    top=6pt,
    bottom=6pt,
    breakable,
}

% 8. cautionbox (warningbox와 동일)
\let\cautionbox\warningbox
\let\endcautionbox\endwarningbox

%========================================================================================
% 코드 블록 설정 (밝은 배경)
%========================================================================================

\usepackage{listings}

\definecolor{codegray}{rgb}{0.5,0.5,0.5}
\definecolor{codepurple}{rgb}{0.58,0,0.82}
\definecolor{backcolour}{rgb}{0.95,0.95,0.95}

\lstset{
    basicstyle=\ttfamily\small,
    backgroundcolor=\color{lightgray},
    keywordstyle=\color{darkblue}\bfseries,
    commentstyle=\color{darkgreen}\itshape,
    stringstyle=\color{purple!80!black},
    numberstyle=\tiny\color{black!60},
    numbers=left,
    numbersep=8pt,
    breaklines=true,
    breakatwhitespace=false,
    frame=single,
    frameround=tttt,
    rulecolor=\color{black!30},
    captionpos=b,
    showstringspaces=false,
    tabsize=2,
    xleftmargin=15pt,
    xrightmargin=5pt,
    escapeinside={\%*}{*)}
}

% Python 코드 스타일
\lstdefinestyle{pythonstyle}{
    language=Python,
    morekeywords={self, True, False, None},
}

% SQL 코드 스타일
\lstdefinestyle{sqlstyle}{
    language=SQL,
    morekeywords={SELECT, FROM, WHERE, JOIN, GROUP, BY, ORDER, HAVING},
}

%========================================================================================
% 목차 스타일링
%========================================================================================

\usepackage{tocloft}
\renewcommand{\cftsecleader}{\cftdotfill{\cftdotsep}}
\setlength{\cftbeforesecskip}{0.4em}
\renewcommand{\cftsecfont}{\bfseries}
\renewcommand{\cftsubsecfont}{\normalfont}

%========================================================================================
% 표 및 그림
%========================================================================================

\usepackage{graphicx}              % 이미지
\usepackage{adjustbox}             % 표/박스 크기 조절

% 표 캡션 스타일
\usepackage{caption}
\captionsetup[table]{
    labelfont=bf,
    textfont=it,
    skip=5pt
}
\captionsetup[figure]{
    labelfont=bf,
    textfont=it,
    skip=5pt
}

%========================================================================================
% 수학
%========================================================================================

\usepackage{amsmath, amssymb, amsthm}

% 정리 환경
\theoremstyle{definition}
\newtheorem{theorem}{정리}[section]
\newtheorem{lemma}[theorem]{보조정리}
\newtheorem{proposition}[theorem]{명제}
\newtheorem{corollary}[theorem]{따름정리}
\newtheorem{definition}{정의}[section]
\newtheorem{example}{예제}[section]

%========================================================================================
% 하이퍼링크
%========================================================================================

\usepackage[
    colorlinks=true,
    linkcolor=blue!80!black,
    urlcolor=blue!80!black,
    citecolor=green!60!black,
    bookmarks=true,
    bookmarksnumbered=true,
    pdfborder={0 0 0}
]{hyperref}

% PDF 메타데이터는 각 문서에서 설정
\hypersetup{
    pdftitle={CSCI E-89B: 자연어 처리 입문 - Lecture 10},
    pdfauthor={강의 노트},
    pdfsubject={Academic Notes}
}

%========================================================================================
% 기타 유용한 패키지
%========================================================================================

\usepackage{enumitem}              % 리스트 커스터마이징
\setlist{nosep, leftmargin=*, itemsep=0.3em}

\usepackage{microtype}             % 타이포그래피 개선
\usepackage{footnote}              % 각주 개선
\usepackage{url}                   % URL 줄바꿈
\urlstyle{same}

%========================================================================================
% 사용자 정의 명령어
%========================================================================================

% 강조 텍스트
\newcommand{\important}[1]{\textbf{\textcolor{red!70!black}{#1}}}
\newcommand{\keyword}[1]{\textbf{#1}}
\newcommand{\term}[1]{\textit{#1}}
\newcommand{\code}[1]{\texttt{#1}}

% 용어 설명 (인라인)
\newcommand{\defterm}[2]{\textbf{#1}\footnote{#2}}

% 섹션 시작 전 페이지 분리
\newcommand{\newsection}[1]{\newpage\section{#1}}

%========================================================================================
% 문서 제목 스타일
%========================================================================================

\usepackage{titling}
\pretitle{\begin{center}\LARGE\bfseries}
\posttitle{\par\end{center}\vskip 0.5em}
\preauthor{\begin{center}\large}
\postauthor{\end{center}}
\predate{\begin{center}\large}
\postdate{\par\end{center}}

%========================================================================================
% 섹션 제목 간격
%========================================================================================

\usepackage{titlesec}
\titlespacing*{\section}{0pt}{1.5em}{0.8em}
\titlespacing*{\subsection}{0pt}{1.2em}{0.6em}
\titlespacing*{\subsubsection}{0pt}{1em}{0.5em}

%========================================================================================
% 메타 정보 박스 명령어
%========================================================================================

\newcommand{\metainfo}[4]{
\begin{tcolorbox}[
    colback=lightpurple,
    colframe=darkpurple,
    boxrule=1pt,
    arc=2mm,
    left=10pt,
    right=10pt,
    top=8pt,
    bottom=8pt
]
\begin{tabular}{@{}rl@{}}
▣ \textbf{강의명:} & #1 \\[0.3em]
▣ \textbf{주차:} & #2 \\[0.3em]
▣ \textbf{교수명:} & #3 \\[0.3em]
▣ \textbf{목적:} & \begin{minipage}[t]{0.75\textwidth}#4\end{minipage}
\end{tabular}
\end{tcolorbox}
}

%========================================================================================
% 끝
%========================================================================================


\begin{document}

\maketitle
\thispagestyle{firstpage}

\metainfo{CSCI E-89B: 자연어 처리 입문}{Lecture 10}{Dmitry Kurochkin}{Lecture 10의 핵심 개념 학습}


\tableofcontents

\newpage

% ==========
% 개요
% ==========
\begin{summarybox}
    \textbf{문서 개요:}
    이 문서는 자연어 처리(NLP)의 핵심 기술인 \textbf{개체명 인식(Named Entity Recognition, NER)}에 대해 다룹니다. NER이 무엇인지, 왜 중요한지, 그리고 어떻게 작동하는지 설명합니다.

    \textbf{핵심 내용:}
    \begin{itemize}
        \item \textbf{NER 정의:} 텍스트에서 '이름'을 가진 개체(예: 사람, 기관, 장소)를 찾아내고 분류하는 기술입니다.
        \item \textbf{두 가지 접근법:}
            \begin{enumerate}
                \item \textbf{규칙 기반(Rule-based):} 정규표현식(Regex)처럼 사전에 정의된 규칙으로 개체를 찾습니다. (예: "Inc."로 끝나면 '기관')
                \item \textbf{통계 기반(Statistical):} 대규모 데이터를 학습한 모델(예: 신경망)이 문맥을 파악하여 개체를 찾습니다.
            \end{enumerate}
        \item \textbf{구현:} `NLTK`와 `spaCy` 라이브러리를 사용한 NER 구현 방법을 배웁니다.
        \item \textbf{심화:} 통계 기반 NER(특히 CNN)의 내부 작동 원리와 기존 모델을 특정 데이터에 맞게 추가 학습시키는 \textbf{미세조정(Fine-tuning)} 개념을 살펴봅니다.
    \end{itemize}

    \textbf{학습 목표:}
    이 문서를 통해 수강생은 NER의 기본 원리를 이해하고, 파이썬으로 간단한 NER 시스템을 구현하며, NER을 다른 NLP 작업(예: 텍스트 분류)에 응용하는 방법을 설명할 수 있습니다.
\end{summarybox}

\newpage

% ==========
% 1. NER이란 무엇인가?
% ==========
\section{개체명 인식(NER)이란 무엇인가?}

\begin{keyconceptbox}
    \textbf{한 줄 정의: NER (Named Entity Recognition)}

    텍스트에서 \textbf{'이름'이 붙은 고유한 대상}을 찾아내고, 그것이 어떤 유형(예: 사람, 기관, 장소)인지 \textbf{분류(Labeling)}하는 작업입니다.

    \textbf{직관적 비유: '형광펜 분류기'}

    NER을 "여러 색상의 형광펜을 가진 조교"라고 생각할 수 있습니다.
    \begin{itemize}
        \item \textbf{노란색:} 사람 이름 (예: "Donald Trump", "Elon Musk")
        \item \textbf{파란색:} 기관/조직 (예: "Tesla", "Federal Reserve", "Apple Inc.")
        \item \textbf{녹색:} 장소 (예: "America", "China")
        \item \textbf{주황색:} 날짜/시간 (예: "this year", "March")
        \item \textbf{분홍색:} 돈/수량 (예: "\$1trn", "40\%")
    \end{itemize}
    NER의 임무는 긴 문서(텍스트)를 읽으며 이 형광펜으로 정확하게 밑줄을 긋고 분류하는 것입니다.
\end{keyconceptbox}

\subsection{NER의 주요 개체 유형}
NER이 인식하는 '개체'는 표준화된 범주를 따르는 경우가 많습니다.

\begin{table}[h!]
    \centering
    \caption{NER의 일반적인 개체 유형}
    \label{tab:ner_types}
    \begin{tabular}{lll}
        \toprule
        \textbf{태그} & \textbf{원어 (Type)} & \textbf{설명 및 예시} \\
        \midrule
        \texttt{PERSON} & Person & 사람 이름 (예: "Donald Trump", "Dr. John Smith") \\
        \texttt{ORG} & Organization & 기관, 회사, 정부 (예: "Tesla", "Bank of England") \\
        \texttt{GPE} & Geopolitical Entity & 지정학적 개체 (국가, 도시, 주) (예: "America", "New York City") \\
        \texttt{DATE} & Date & 날짜 (예: "November 18, 2024", "this year") \\
        \texttt{TIME} & Time & 시간 (예: "10:00 AM") \\
        \texttt{MONEY} & Money & 화폐 단위 (예: "\$1trn", "\$90,000") \\
        \texttt{PERCENT} & Percent & 백분율 (예: "40\%", "2.6\%") \\
        \texttt{CARDINAL} & Cardinal Number & 기수 (일반 숫자) (예: "50", "44,000") \\
        \texttt{ORDINAL} & Ordinal Number & 서수 (순서) (예: "first") \\
        \texttt{NORP} & Nationalities or \dots & 국적, 종교, 정치 단체 (예: "Chinese") \\
        \bottomrule
    \end{tabular}
\end{table}

\subsection{NER 적용 예시}
다음은 특정 경제 기사(The Economist)에 NER을 적용한 예시입니다.

\begin{examplebox}
    \textbf{원본 텍스트:}
    "Markets continued to rally in response to \textbf{Donald Trump}'s election victory. The \textbf{S\&P 500} hit another high (it has broken \textbf{more than 50} records so far \textbf{this year}) \dots The rise in \textbf{Tesla}'s stock pushed the carmaker above a valuation of \textbf{\$1trn}, which it last achieved in \textbf{early 2022}. \dots Traders still expect the \textbf{Federal Reserve} to cut interest rates \dots"

    \textbf{NER 적용 결과:}
    \begin{itemize}
        \item "Donald Trump": \textbf{\texttt{PERSON}}
        \item "S\&P 500": \textbf{\texttt{ORG}} (기관으로 분류될 수 있음)
        \item "more than 50": \textbf{\texttt{CARDINAL}}
        \item "this year": \textbf{\texttt{DATE}}
        \item "Tesla": \textbf{\texttt{ORG}}
        \item "\$1trn": \textbf{\texttt{MONEY}}
        \item "early 2022": \textbf{\texttt{DATE}}
        \item "Federal Reserve": \textbf{\texttt{ORG}}
    \end{itemize}
    \textit{참고: "Bitcoin"의 경우, 문맥에 따라 \texttt{PERSON}으로 잘못 분류될 수도 있습니다. (예: "Bitcoin surged \dots" 이 구문만 보면 사람 이름처럼 보일 수 있음) 이는 모델이 문맥을 어떻게 학습했는지에 따라 달라집니다.}
\end{examplebox}

\newpage

% ==========
% 2. NER의 활용
% ==========
\section{NER은 왜 중요한가? (주요 활용 분야)}
NER은 단순히 텍스트에 밑줄을 긋는 것을 넘어, 다른 NLP 작업들의 성능을 비약적으로 향상시키는 \textbf{전처리(preprocessing)} 또는 \textbf{특성 공학(feature engineering)}의 핵심 단계입니다.

\begin{itemize}
    \item \textbf{정보 추출 (Information Extraction):}
    비정형 텍스트(예: 뉴스 기사, 이메일)에서 핵심 정보를 뽑아내어 정형 데이터(예: 데이터베이스, 엑셀 시트)를 만드는 데 사용됩니다.
    \begin{itemize}
        \item \textit{예시:} 수천 개의 이력서에서 '사람 이름', '회사명', '직무'를 자동으로 추출하여 표로 정리합니다.
    \end{itemize}

    \item \textbf{텍스트 분류 (Text Classification) 성능 향상:}
    문서에 어떤 '유형'의 개체가 포함되어 있는지를 모델에 알려주어 분류 정확도를 높입니다.
    \begin{itemize}
        \item \textit{예시:} 문서에 \texttt{ORG} (기관) 태그가 많이 등장하면 '비즈니스' 또는 '정치' 기사일 확률이 높습니다. (자세한 내용은 \ref{sec:app_ner_classification} 참조)
    \end{itemize}

    \item \textbf{질의응답 (Question Answering) 시스템:}
    사용자의 질문(Query)과 문서 내의 잠재적 답변(Answer)에서 개체 유형을 일치시켜 정확한 답을 찾습니다.
    \begin{itemize}
        \item \textit{질문:} "Who is the CEO of Tesla?" (질문 유형: \texttt{PERSON})
        \item \textit{문서 검색:} "Tesla (\texttt{ORG}) \dots Elon Musk (\texttt{PERSON}) \dots"
        \item \textit{답변:} 질문이 \texttt{PERSON}을 물었으므로, 문서에서 찾은 \texttt{PERSON}인 "Elon Musk"를 답변으로 반환합니다.
    \end{itemize}

    \item \textbf{기계 번역 (Machine Translation):}
    '이름'을 일반 명사로 오인하여 잘못 번역하는 것을 방지합니다.
    \begin{itemize}
        \item \textit{오번역 예시:} "Apple (회사) is hiring." $\rightarrow$ "사과가 고용 중이다." (X)
        \item \textit{NER 적용:} "Apple (\texttt{ORG}) is hiring." $\rightarrow$ "애플(사)이 고용 중이다." (O)
    \end{itemize}

    \item \textbf{의미 검색 (Semantic Search):}
    단순 키워드 검색이 아닌 '의미' 기반 검색을 가능하게 합니다.
    \begin{itemize}
        \item \textit{검색어:} "Apple"
        \item \textit{결과:} '사과(과일)'에 대한 문서와 '애플(회사)'에 대한 문서를 NER로 구분하여 사용자에게 제시합니다.
    \end{itemize}
\end{itemize}

\newpage

% ==========
% 3. 규칙 기반 NER
% ==========
\section{접근법 1: 규칙 기반 NER (Rule-based NER)}

\begin{keyconceptbox}
    \textbf{한 줄 정의:}
    개발자가 사전에 정의한 \textbf{명시적인 규칙(Rule)의 목록}을 사용하여 텍스트에서 개체명을 찾는 방식입니다.

    \textbf{직관적 비유: '엄격한 체크리스트 검사원'}
    규칙 기반 NER은 "체크리스트만 보고 판단하는 검사원"과 같습니다.
    \begin{itemize}
        \item "텍스트에 'Mr.', 'Mrs.', 'Dr.'가 있으면 그 뒤 단어는 \texttt{PERSON}이다."
        \item "텍스트가 '000-000-0000' 형식이면 \texttt{PHONE\_NUMBER}이다."
        \item "텍스트가 'Inc.', 'Ltd.', 'Corp.'로 끝나면 \texttt{ORG}이다."
    \end{itemize}
    이 검사원은 빠르고 정확하지만, 체크리스트에 없는 새로운 패턴(예: "Tesla"처럼 'Inc.'가 없는 회사명)은 절대로 인식하지 못합니다.
\end{keyconceptbox}

\subsection{주요 기술: 정규표현식 (Regular Expressions)}
규칙 기반 NER에서 가장 많이 사용되는 도구는 \textbf{정규표현식(Regex)}입니다. Regex는 특정 문자열 패턴을 정의하는 문법입니다.

\begin{examplebox}
    \textbf{사례 1: 날짜, 이메일, 조직명(Inc) 인식하기}

    다음은 파이썬의 `re` 라이브러리를 사용한 예시입니다.
    \begin{lstlisting}[language=Python, caption={규칙 기반 NER을 위한 정규표현식 예시 (날짜, 이메일 등)}, label={lst:regex_ner1}, breaklines=true]
    import re

    text = """
    Dr. John Smith met with Apple Inc. on November 18, 2024.
    His email is john.smith@email.com, and the meeting was at 10:00 AM.
    """

    # 1. 날짜 패턴 (예: MM/DD/YYYY 또는 Month D, YYYY)
    date_pattern = r'\b(?:January|February|...|November|December)\s+\d{1,2},\s*\d{4}\b'
    # 2. 이메일 패턴
    email_pattern = r'\b[A-Za-z0-9._%+-]+@[A-Za-z0-9.-]+\.[A-Z|a-z]{2,}\b'
    # 3. 시간 패턴 (AM/PM)
    time_pattern = r'\b\d{1,2}:\d{2}\s*(?:AM|PM)\b'
    # 4. 조직명 패턴 (Inc, Ltd, Corp로 끝나는 경우)
    org_pattern = r'\b[A-Z][a-zA-Z]+\s+(?:Inc|Ltd|Corp)\b'

    print(f"Dates: {re.findall(date_pattern, text)}")
    print(f"Emails: {re.findall(email_pattern, text)}")
    print(f"Times: {re.findall(time_pattern, text, flags=re.IGNORECASE)}")
    print(f"Orgs: {re.findall(org_pattern, text)}")

    # --- 출력 결과 ---
    # Dates: ['November 18, 2024']
    # Emails: ['john.smith@email.com']
    # Times: ['10:00 AM']
    # Orgs: ['Apple Inc']
    \end{lstlisting}
\end{examplebox}

\begin{examplebox}
    \textbf{사례 2: 호칭(Title)을 이용한 사람 이름 인식하기}

    'Mr.', 'Dr.' 등의 호칭(Title)을 기준으로 사람을 찾는 더 간단한 규칙입니다.
    \begin{lstlisting}[language=Python, caption={호칭(Title) 기반 정규표현식 예시}, label={lst:regex_ner2}, breaklines=true]
    text = "Mr. Musk and Mr. Trump met with Dr. Emily White."

    # 1. 호칭 패턴 (Mr, Mrs, Ms, Dr, Professor)
    # \s+ : 하나 이상의 공백
    # [A-Z][a-z]+ : 대문자로 시작하고 소문자가 이어지는 단어 (이름)
    person_pattern = r'\b(Mr|Mrs|Ms|Dr|Professor)\.?\s+[A-Z][a-z]+(?:\s+[A-Z][a-z]+)?'
    # (?:\s+[A-Z][a-z]+)? : (선택적) 공백과 성(Last Name)이 나올 수도 있음

    print(f"Persons: {re.findall(person_pattern, text)}")

    # --- 출력 결과 ---
    # Persons: [('Mr', ' Musk'), ('Mr', ' Trump'), ('Dr', ' Emily White')]
    # (참고: 정규식의 캡처 그룹 설정에 따라 출력 형태가 다를 수 있습니다.)
    \end{lstlisting}
\end{examplebox}

\subsection{규칙 기반 NER의 장단점}
\begin{table}[h!]
    \centering
    \caption{규칙 기반 NER의 장점과 단점}
    \label{tab:rule_based_pros_cons}
    \begin{tabular}{p{0.45\textwidth} p{0.45\textwidth}}
        \toprule
        \textbf{👍 장점 (Pros)} & \textbf{👎 단점 (Cons)} \\
        \midrule
        \textbf{높은 정밀도(Precision):} 규칙이 명확한 도메인(예: 이메일, 주민번호)에서는 거의 100\% 정확하게 작동합니다. & \textbf{낮은 재현율(Recall) / 유연성 부족:} 규칙에 없는 새로운 패턴(예: "Apple")을 놓치기 쉽습니다. \\
        \textbf{해석 가능성 (Interpretability):} 왜 해당 개체를 인식했는지 규칙을 통해 100\% 설명 가능합니다. & \textbf{유지보수 부담:} 언어 사용이 변하거나 새로운 유형의 개체가 생기면(예: "COVID-19") 매번 규칙을 수동으로 업데이트해야 합니다. \\
        \textbf{데이터 불필요:} 모델을 학습시키기 위한 대규모 라벨링 데이터가 필요 없습니다. & \textbf{도메인 전문성 요구:} 효과적인 규칙을 작성하려면 해당 분야(예: 법률, 의료)의 도메인 지식이 필요합니다. \\
        \bottomrule
    \end{tabular}
\end{table}

\begin{cautionbox}
    \textbf{규칙 기반의 한계: 문맥(Context)의 부재}

    규칙 기반 방식의 가장 큰 한계는 \textbf{문맥을 이해하지 못한다}는 것입니다.
    \begin{itemize}
        \item \textit{예시:} "I bought an \textbf{Apple}." vs "I work at \textbf{Apple}."
    \end{itemize}
    규칙 기반 시스템은 이 두 "Apple"을 구분할 수 없습니다. 반면, 통계 기반 모델은 "bought"라는 단어(과일)와 "work at"라는 단어(회사)라는 문맥을 통해 이를 구분할 수 있습니다.
\end{cautionbox}

\newpage

% ==========
% 4. 통계 기반 NER
% ==========
\section{접근법 2: 통계 기반 NER (Statistical NER)}

\begin{keyconceptbox}
    \textbf{한 줄 정의:}
    대규모의 \textbf{정답(라벨링) 데이터}를 학습하여, 특정 단어 시퀀스가 개체명일 \textbf{확률(Probability)}을 계산하는 방식입니다.

    \textbf{직관적 비유: '수천 건의 사례를 학습한 탐정'}
    통계 기반 NER은 "수천 건의 사건 파일을 학습한 탐정"과 같습니다.
    \begin{itemize}
        \item 이 탐정은 명시적인 '규칙'에만 의존하지 않습니다.
        \item 대신, \textbf{문맥(Context)}과 \textbf{패턴(Pattern)}을 학습합니다.
        \item \textit{"'Tesla'라는 단어가 'electric car', 'stock', 'CEO' 같은 단어 근처에 나오면, 99\% 확률로 \texttt{ORG}(기관)이다."}
        \item \textit{"'Tesla'라는 단어가 'physicist', 'invention', 'Wardenclyffe' 같은 단어 근처에 나오면, 98\% 확률로 \texttt{PERSON}(사람)이다."}
    \end{itemize}
    이 방식은 새로운 데이터에도 유연하게 적응할 수 있지만, 왜 그렇게 판단했는지 정확히 설명하기는 어렵습니다. (블랙박스)
\end{keyconceptbox}

\subsection{주요 모델 및 기술}
통계 기반 NER은 전통적인 머신러닝 모델에서 딥러닝 모델로 발전해 왔습니다.

\begin{itemize}
    \item \textbf{은닉 마르코프 모델 (Hidden Markov Models, HMMs):}
    관찰된 단어(Observable) 뒤에 숨겨진(Hidden) 개체 태그(State)를 예측하는 초기 확률 모델입니다.
    
    \item \textbf{조건부 무작위장 (Conditional Random Fields, CRFs):}
    HMM보다 발전된 모델로, 단어 시퀀스 전체의 문맥을 고려하여 가장 확률이 높은 태그 시퀀스를 예측합니다. (오랫동안 NER의 표준으로 사용됨)
    
    \item \textbf{신경망 (Neural Networks, NN):}
    최근의 NER 시스템은 대부분 딥러닝, 특히 순환 신경망(RNN, LSTM)이나 트랜스포머(BERT)를 사용합니다. `spaCy`의 기본 모델 중 하나는 \textbf{CNN (Convolutional Neural Networks)}을 사용합니다.
\end{itemize}

\subsection{통계 기반 NER의 장단점}
\begin{table}[h!]
    \centering
    \caption{통계 기반 NER의 장점과 단점}
    \label{tab:stat_based_pros_cons}
    \begin{tabular}{p{0.45\textwidth} p{0.45\textwidth}}
        \toprule
        \textbf{👍 장점 (Pros)} & \textbf{👎 단점 (Cons)} \\
        \midrule
        \textbf{데이터 기반 학습 (유연성):} 새로운 패턴이나 문맥을 데이터로부터 스스로 학습하여 적응합니다. (예: "Apple"을 문맥으로 구분) & \textbf{대규모 데이터 요구:} 높은 성능을 내기 위해 \textbf{잘 라벨링된(annotated)} 대량의 데이터가 필수적입니다. \\
        \textbf{도메인 적응성:} 새로운 도메인(예: 의료)의 라벨링된 데이터를 제공하면 해당 도메인에 맞게 모델을 '학습' 또는 '미세조정'할 수 있습니다. & \textbf{블랙박스 (해석 불가):} 왜 모델이 "Tesla"를 \texttt{PERSON}으로 분류했는지 명확히 설명하기 어렵습니다. \\
        \textbf{높은 재현율(Recall):} 규칙에 없는 새로운 개체명이라도 문맥이 비슷하다면 인식할 가능성이 높습니다. & \textbf{높은 컴퓨팅 비용:} 딥러닝 모델(예: BERT)을 학습시키려면 고사양의 GPU와 많은 시간이 필요합니다. \\
        \bottomrule
    \end{tabular}
\end{table}

\newpage

% ==========
% 5. 심층 분석: CNN 기반 NER
% ==========
\section{심층 분석: spaCy는 NER을 어떻게 수행하는가? (CNN 기반)}
`spaCy`와 같은 최신 라이브러리는 어떻게 문맥을 파악하여 NER을 수행할까요? 강의에서는 \textbf{CNN(합성곱 신경망)}을 이용한 NER의 작동 원리를 설명합니다.

\begin{keyconceptbox}
    \textbf{핵심 아이디어: 텍스트를 '이미지'로 바라보기}

    우리는 보통 텍스트를 '단어의 1차원 배열'로 생각합니다. 하지만 딥러닝 기반 NER은 텍스트를 \textbf{'2차원 이미지'}처럼 처리합니다.

    \begin{enumerate}
        \item \textbf{임베딩 (Embedding):} 각 단어를 고유한 숫자 벡터(예: 300차원)로 변환합니다.
        \item \textbf{2D 변환:} "The cat sat on the mat" (5개 단어)라는 문장은 5 $\times$ 300 크기의 \textbf{행렬(Matrix)}이 됩니다.
        \item \textbf{이미지 비유:} 이 5 $\times$ 300 행렬을 5픽셀(세로) $\times$ 300픽셀(가로) 크기의 흑백 '이미지'라고 상상할 수 있습니다.
    \end{enumerate}
\end{keyconceptbox}

\subsection{CNN을 이용한 NER의 3단계}

\textbf{1단계: 입력 (텍스트 $\rightarrow$ 임베딩 행렬)}
문장 "The cat sat on the mat"은 각 단어의 임베딩 벡터로 구성된 행렬(이미지)이 됩니다.

\begin{center}
    [Vector for "The"] \\
    [Vector for "cat"] \\
    [Vector for "sat"] \\
    [Vector for "on"] \\
    [Vector for "mat"]
\end{center}

\textbf{2단계: 합성곱 (CNN 필터 적용)}
이미지 처리에서 CNN 필터(커널)가 이미지의 '특징'(예: 선, 모서리)을 감지하듯, NER에서 CNN 필터는 \textbf{'문맥적 특징'}을 감지합니다.

\begin{itemize}
    \item 3 $\times$ 300 크기의 필터(창문)가 이 '이미지'를 위에서 아래로 훑고 지나갑니다.
    \item 첫 번째 위치에서 "The", "cat", "sat"의 임베딩을 \textbf{동시에} 봅니다.
    \item 다음 위치에서 "cat", "sat", "on"의 임베딩을 \textbf{동시에} 봅니다.
    \item \textbf{이유:} 이 필터는 "sat"이라는 단어 하나만 보는 것이 아니라, 그 주변 단어("cat", "on")를 함께 봄으로써 \textbf{지역적 문맥(Local Context)}을 포착합니다.
\end{itemize}

\textbf{3단계: 출력 (Softmax $\rightarrow$ 태그 예측)}
CNN을 통과한 결과는 \textbf{각 단어(토큰)마다} 모든 개체 유형에 대한 확률 벡터를 출력합니다.

\begin{examplebox}
    \textbf{단어 "Tesla"에 대한 가상의 출력 확률:}

    \textbf{문맥 1:} "...physicist \textbf{Tesla} invented..."
    \begin{itemize}
        \item \texttt{PERSON}: 98\%
        \item \texttt{ORG}: 1\%
        \item \texttt{GPE}: 0.5\%
        \item \texttt{O} (Other/없음): 0.5\%
        \item \textbf{$\rightarrow$ 최종 예측: PERSON}
    \end{itemize}

    \textbf{문맥 2:} "...electric car \textbf{Tesla} announced..."
    \begin{itemize}
        \item \texttt{PERSON}: 1\%
        \item \texttt{ORG}: 97\%
        \item \texttt{GPE}: 0.5\%
        \item \texttt{O} (Other/없음): 1.5\%
        \item \textbf{$\rightarrow$ 최종 예측: ORG}
    \end{itemize}
\end{examplebox}

\subsection{자주 묻는 질문 (FAQ)}

\textbf{Q: "Bank of England"처럼 여러 단어로 된 개체는 어떻게 인식하나요?}

A: 훌륭한 질문입니다. 딥러닝 모델은 단순히 단어마다 태그를 붙이는 것이 아니라, \textbf{BIO 태깅 스킴(Tagging Scheme)}을 사용합니다.
\begin{itemize}
    \item \textbf{B} (Beginning): 개체가 시작되는 단어
    \item \textbf{I} (Inside): 개체 내부에 속하는 단어 (시작 아님)
    \item \textbf{O} (Outside): 개체에 속하지 않는 단어
\end{itemize}
"Bank of England"는 다음과 같이 태깅됩니다.
\begin{itemize}
    \item Bank: \textbf{B-ORG} (기관 개체의 시작)
    \item of: \textbf{I-ORG} (기관 개체의 내부)
    \item England: \textbf{I-ORG} (기관 개체의 내부)
\end{itemize}
모델은 "B-ORG" 뒤에 "I-ORG"가 나올 확률을 학습하여 여러 단어로 구성된 개체를 하나의 덩어리(Chunk)로 묶습니다.

\textbf{Q: "Tesla"라는 단어가 한 문서에 여러 번 나오면 어떻게 되나요?}

A: NER은 문서 전체가 아닌, \textbf{각 토큰(단어)의 위치마다} 문맥을 평가합니다. 한 문서의 앞부분에서 "Tesla (\texttt{ORG})"가 나왔더라도, 뒷부분에서 "Tesla (\texttt{PERSON})"가 다른 문맥으로 나오면 각각 다르게 태깅할 수 있습니다. CNN 필터는 \textbf{지역 문맥(Local Context)}에 집중하기 때문입니다.

\newpage

% ==========
% 6. 파이썬 구현
% ==========
\section{파이썬을 이용한 NER 구현}
파이썬에서는 `NLTK`와 `spaCy`가 NER 및 관련 작업을 위해 널리 사용됩니다.

\subsection{NLTK를 이용한 품사 판별 (POS Tagging)}
NER에 앞서, 각 단어의 \textbf{품사(Part-of-Speech, POS)}를 판별하는 것은 문장의 구조를 이해하는 데 도움이 됩니다. NER이 '고유명사'를 찾는 작업이라면, POS는 '명사', '동사', '형용사' 등 일반적인 문법 요소를 찾습니다.

\begin{lstlisting}[language=Python, caption={NLTK를 사용한 문장 토큰화 및 POS Tagging}, label={lst:nltk_pos}, breaklines=true]
import nltk
# NLTK의 필요 리소스 다운로드 (최초 1회)
# nltk.download('punkt')
# nltk.download('averaged_perceptron_tagger')

text = "From electric cars to solar panels, Mr. Musk is busy."

# 1. 문장을 단어(토큰)로 분리 (Word Tokenization)
tokens = nltk.word_tokenize(text)
# ['From', 'electric', 'cars', 'to', 'solar', 'panels', ',', 
#  'Mr.', 'Musk', 'is', 'busy', '.']

# 2. 토큰에 대해 POS Tagging 수행
pos_tags = nltk.pos_tag(tokens)

print(pos_tags)

# --- 출력 결과 (튜플의 리스트) ---
# [('From', 'IN'),         # IN: 전치사 (Preposition)
#  ('electric', 'JJ'),     # JJ: 형용사 (Adjective)
#  ('cars', 'NNS'),        # NNS: 명사, 복수형 (Noun, plural)
#  ('to', 'TO'),           # TO: 'to'
#  ('solar', 'JJ'),
#  ('panels', 'NNS'),
#  (',', ','),
#  ('Mr.', 'NNP'),        # NNP: 고유명사, 단수 (Proper noun, singular)
#  ('Musk', 'NNP'),
#  ('is', 'VBZ'),          # VBZ: 동사, 3인칭 단수 현재 (Verb)
#  ('busy', 'JJ'),
#  ('.', '.')]
\end{lstlisting}

\subsection{spaCy를 이용한 개체명 인식 (NER)}
`spaCy`는 현대적인 NLP 라이브러리로, 고도로 최적화된 통계 기반 NER 모델을 기본 제공합니다.

\begin{lstlisting}[language=Python, caption={spaCy를 사용한 NER 수행}, label={lst:spacy_ner}, breaklines=true]
import spacy
from spacy import displacy # 시각화 도구

# 1. 사전 학습된 spaCy 모델 로드 (영어, 스몰 버전)
#    (설치: python -m spacy download en_core_web_sm)
nlp = spacy.load("en_core_web_sm")

# 2. 텍스트 처리
#    spaCy의 nlp 객체는 토큰화, POS, NER 등 전체 파이프라인을 실행합니다.
text = """
    From electric cars to solar panels, Mr. Musk is busy.
    Tesla sells electric vehicles.
    Donald J. Trump is a person.
"""
doc = nlp(text)

# 3. 인식된 개체(Entities) 순회 및 출력
print("--- 인식된 개체 목록 ---")
for ent in doc.ents:
    # ent.text: 개체 텍스트
    # ent.label_: 개체 유형 (태그)
    print(f"Text: {ent.text}, Label: {ent.label_}")

# 4. (Jupyter/웹) 시각화
# displacy.render(doc, style="ent", jupyter=True)

# --- 출력 결과 ---
# --- 인식된 개체 목록 ---
# Text: Musk, Label: PERSON
# Text: Tesla, Label: ORG
# Text: Donald J. Trump, Label: PERSON
\end{lstlisting}

\begin{cautionbox}
    \textbf{spaCy의 NLP 파이프라인}

    `doc = nlp(text)`라는 한 줄의 코드는 `spaCy` 내부에서 복잡한 \textbf{파이프라인(Pipeline)}을 실행합니다.
    \begin{enumerate}
        \item \textbf{Tokenizer:} 텍스트를 토큰으로 분리
        \item \textbf{Tagger:} POS 태깅
        \item \textbf{Parser:} 문장 구조 분석 (의존성 파싱)
        \item \textbf{NER:} 개체명 인식
        \item \dots (기타)
    \end{enumerate}
    `doc` 객체에는 이 모든 분석 결과가 포함되어 있습니다.
\end{cautionbox}

\newpage

% ==========
% 7. NER 모델 미세조정
% ==========
\section{NER 모델 미세조정 (Fine-Tuning)}
사전 학습된 `spaCy` 모델이 내가 가진 특정 도메인(예: 자동차 산업)의 용어를 잘 인식하지 못할 수 있습니다. 예를 들어, "Honda Civic"을 \texttt{ORG}(기관)가 아닌 \texttt{VEHICLE}(차량)이라는 새로운 유형으로 인식하게 하고 싶을 수 있습니다.

이때, 새로운 데이터를 추가하여 기존 모델을 \textbf{추가 학습(Fine-Tuning)}시킬 수 있습니다.

\begin{keyconceptbox}
    \textbf{미세조정 (Fine-Tuning)이란?}
    
    이미 수많은 데이터로 학습된 똑똑한 모델(Pre-trained model)을 가져와서, 내가 가진 \textbf{소량의 특정 데이터}를 \textbf{추가로 학습}시켜 내 입맛에 맞게 '조율'하는 과정입니다.
    
    처음부터 모든 것을 학습(Training from scratch)하는 것보다 훨씬 효율적이고 적은 데이터로도 좋은 성능을 낼 수 있습니다.
\end{keyconceptbox}

\begin{examplebox}
    \textbf{spaCy NER 모델에 'VEHICLE' 유형 추가 학습 시도} (개념적 예시)

    다음 코드는 `spaCy` 모델의 `ner` 파이프라인만 선택적으로 비활성화하고, 새로운 'VEHICLE' 라벨이 포함된 데이터로 추가 학습(update)을 시도하는 과정을 보여줍니다.

    \begin{lstlisting}[language=Python, caption={spaCy NER 모델 미세조정(Fine-Tuning) 시도}, label={lst:spacy_finetune}, breaklines=true]
    import spacy
    import random

    # 1. 기존 모델 로드
    nlp = spacy.load("en_core_web_sm")

    # 2. 새로운 학습 데이터 (VEHICLE 유형 추가)
    #    (텍스트, {개체 목록: [(시작 인덱스, 끝 인덱스, 라벨)]})
    TRAIN_DATA = [
        ("Cars in China", {"entities": [(0, 4, "VEHICLE")]}),
        ("My family loves our Honda Civic", {"entities": [(20, 31, "VEHICLE")]}),
        ("This car is the last one", {"entities": [(5, 8, "VEHICLE")]})
    ]

    # 3. 파이프라인에서 'ner' 컴포넌트 가져오기
    ner = nlp.get_pipe("ner")

    # 4. 새로운 라벨(VEHICLE)을 ner 컴포넌트에 추가
    ner.add_label("VEHICLE")

    # 5. 다른 파이프라인은 비활성화하고 'ner'만 학습
    other_pipes = [pipe for pipe in nlp.pipe_names if pipe != "ner"]
    with nlp.disable_pipes(*other_pipes):
        optimizer = nlp.begin_training()
        for itn in range(20): # 20회 반복 학습
            random.shuffle(TRAIN_DATA)
            losses = {}
            for text, annotations in TRAIN_DATA:
                doc = nlp.make_doc(text)
                example = spacy.training.Example.from_dict(doc, annotations)
                nlp.update([example], sgd=optimizer, losses=losses)
            # print(f"Iteration {itn}, Losses: {losses}")

    # 6. 학습된 모델로 테스트
    doc = nlp("My family loves our Honda Civic. Tesla is a company.")
    for ent in doc.ents:
        print(f"Text: {ent.text}, Label: {ent.label_}")
        
    # --- 기대하는 출력 (성공 시) ---
    # Text: Honda Civic, Label: VEHICLE
    # Text: Tesla, Label: ORG
    \end{lstlisting}
\end{examplebox}

\begin{cautionbox}
    \textbf{미세조정의 위험: 파국적 망각 (Catastrophic Forgetting)}

    미세조정은 매우 적은 수의 예시(위 예제에서는 단 3개)로 시도할 경우, 모델이 새로운 데이터에 \textbf{과적합(overfitting)}되어 기존에 잘하던 것까지 잊어버리는 \textbf{'파국적 망각'} 현상이 발생할 수 있습니다.

    실제 강의의 시연에서도, 위와 같이 'VEHICLE'을 학습시킨 후 \textbf{"Tesla"를 \texttt{ORG}가 아닌 \texttt{PERSON}으로 잘못 분류하는 오류}가 발생했습니다. 이는 모델이 "Tesla"를 학습 데이터에서 본 적이 없기 때문에, 새로운 학습 과정에서 가중치가 망가져 기존의 지식을 잃어버렸기 때문입니다.

    \textbf{해결책:} 미세조정을 할 때는 새로운 데이터뿐만 아니라, \textbf{기존의 정답 예시(Original Examples)}도 함께 학습시켜야 합니다.
\end{cautionbox}

\newpage

% ==========
% 8. 실전 응용 (과제)
% ==========
\section{실전 응용: 텍스트 분류 모델에 NER 활용하기} \label{sec:app_ner_classification}

NER은 그 자체로도 유용하지만, 다른 NLP 모델의 성능을 높이는 \textbf{'특성(Feature)'}으로 사용될 때 더욱 강력합니다.

\textbf{목표:} 뉴스 기사를 7개의 카테고리(예: 스포츠, 정치, 기술)로 분류하는 모델을 만든다고 가정합니다.

\subsection{접근법 1: TF-IDF만 사용 (Baseline)}
전통적인 방식은 TF-IDF 벡터를 만들어 모델(예: 로지스틱 회귀, 신경망)을 학습시키는 것입니다.

\begin{itemize}
    \item \textbf{입력 데이터:} (기사 100개 $\times$ 단어 5000개) 크기의 TF-IDF 행렬
    \item \textbf{문제점:} 이 방식은 문맥을 잃어버립니다. "Apple"이라는 단어가 '기술' 기사에 중요한지, '요리' 기사에 중요한지 TF-IDF 값만으로는 알기 어렵습니다.
\end{itemize}

\subsection{접근법 2: TF-IDF + NER 특성 (Enhanced)}
기존 TF-IDF 특성에 \textbf{NER로 추출한 개체 정보를 추가}하여 모델을 더 '똑똑하게' 만들 수 있습니다.

\textbf{작업 순서:}
\begin{enumerate}
    \item 모든 기사(예: 100개)에 대해 `spaCy`로 NER을 실행합니다.
    \item 각 기사에 어떤 유형의 개체(\texttt{PERSON}, \texttt{ORG}, \texttt{GPE} 등)가 \textbf{존재하는지 여부}를 0 또는 1의 \textbf{더미 변수(Dummy Variable)}로 만듭니다.
    \item 이 더미 변수들을 기존 TF-IDF 행렬에 \textbf{열(Column)로 추가}합니다.
\end{enumerate}

\begin{table}[h!]
    \centering
    \caption{NER 특성 추가 전후의 특성 행렬(Feature Matrix) 비교}
    \label{tab:feature_matrix}
    \begin{adjustbox}{width=\textwidth,center}
    \begin{tabular}{l|ccc|cccc}
        \toprule
        & \multicolumn{3}{c|}{\textbf{접근법 1: TF-IDF만 사용}} & \multicolumn{4}{c}{\textbf{접근법 2: TF-IDF + NER 특성}} \\
        \textbf{문서 (기사)} & \textbf{TF-IDF("Apple")} & \textbf{TF-IDF("Musk")} & \textbf{...} & \textbf{TF-IDF("Apple")} & \textbf{TF-IDF("Musk")} & \textbf{...} & \textbf{Has\_ORG (NER)} & \textbf{Has\_PERSON (NER)} \\
        \midrule
        기사 1: "Apple...CEO..." & 0.35 & 0.0 & \dots & 0.35 & 0.0 & \dots & \textbf{1} & \textbf{0} \\
        기사 2: "Musk...Tesla..." & 0.0 & 0.41 & \dots & 0.0 & 0.41 & \dots & \textbf{1} & \textbf{1} \\
        기사 3: "recipe...apple..." & 0.28 & 0.0 & \dots & 0.28 & 0.0 & \dots & \textbf{0} & \textbf{0} \\
        \bottomrule
    \end{tabular}
    \end{adjustbox}
\end{table}

\textbf{결과:}
\begin{itemize}
    \item 이제 모델은 TF-IDF 값(\texttt{"Apple"} = 0.35)뿐만 아니라, \textbf{문맥 정보(\texttt{Has\_ORG} = 1)}를 함께 볼 수 있습니다.
    \item 모델은 '기사 1'과 '기사 3'이 모두 "Apple"이라는 단어를 포함하지만, '기사 1'은 \texttt{ORG}(기관)와 관련이 있으므로 '기술' 또는 '비즈니스' 카테고리일 것이라고 학습할 수 있습니다.
    \item 이처럼 NER 특성을 추가하는 것은 모델에게 강력한 \textbf{'힌트'}를 제공하여, 특히 적은 데이터셋(예: 124개 기사)에서도 분류 성능을 향상시키는 데 도움을 줄 수 있습니다.
\end{itemize}

\newpage

% ==========
% 9. 학습 체크리스트 및 FAQ
% ==========
\section{학습 체크리스트 및 FAQ}

\subsection{학습 내용 점검 체크리스트}
\begin{itemize}
    \item [ ] NER의 정의를 "형광펜 비유"를 들어 설명할 수 있는가?
    \item [ ] NER이 사용되는 3가지 주요 응용 분야(예: Q\&A, 분류)를 말할 수 있는가?
    \item [ ] 규칙 기반 NER과 통계 기반 NER의 핵심 차이점(규칙 vs 문맥)을 설명할 수 있는가?
    \item [ ] 규칙 기반 NER의 장점(해석 가능)과 단점(유연성 부족)을 아는가?
    \item [ ] 통계 기반 NER의 장점(문맥 이해)과 단점(데이터 요구, 블랙박스)을 아는가?
    \item [ ] `spaCy`의 `nlp(text)` 코드가 단순한 작업이 아닌 '파이프라인'임을 이해하는가?
    \item [ ] 텍스트 분류 모델의 성능을 향상시키기 위해 NER 결과를 어떻게 활용할 수 있는지(특성 행렬) 설명할 수 있는가?
    \item [ ] NER 모델을 '미세조정'한다는 것의 의미와 그 위험성(파국적 망각)을 이해하는가?
\end{itemize}

\subsection{초심자 FAQ}

\textbf{Q: NER과 POS Tagging은 같은 것 아닌가요?}
A: 다릅니다. \textbf{POS Tagging}은 \textit{일반 문법}(명사, 동사, 형용사)을 찾는 것이고, \textbf{NER}은 \textit{고유한 이름}(사람, 기관, 장소)을 찾는 것입니다.
\begin{itemize}
    \item "Musk (\texttt{NNP}, 고유명사)" $\rightarrow$ \textit{(POS Tagging)}
    \item "Musk (\texttt{PERSON}, 사람)" $\rightarrow$ \textit{(NER)}
\end{itemize}
모든 \texttt{PERSON}은 \texttt{NNP}(고유명사)일 수 있지만, 모든 \texttt{NNP}가 \texttt{PERSON}은 아닙니다. (예: "Tesla"는 \texttt{NNP}이지만 \texttt{ORG}입니다.)

\textbf{Q: "Tesla"가 사람 이름으로도, 회사 이름으로도 쓰이는데, 모델은 어떻게 구분하나요?}
A: \textbf{문맥(Context)}입니다. 통계 기반 모델은 "Tesla"라는 단어 자체보다 \textit{주변 단어}를 더 중요하게 봅니다.
\begin{itemize}
    \item "Tesla \textbf{invented} \dots" $\rightarrow$ '발명하다'와 어울리는 것은 \texttt{PERSON}입니다.
    \item "Tesla \textbf{sells} \dots" $\rightarrow$ '판매하다'와 어울리는 것은 \texttt{ORG}입니다.
\end{itemize}

\textbf{Q: 제 데이터에는 `spaCy`가 잘 작동하지 않습니다. 어떻게 해야 하나요?}
A: 두 가지 방법이 있습니다.
\begin{enumerate}
    \item \textbf{규칙 기반 추가:} `spaCy`의 통계 모델이 놓친 부분을 정규표현식(Regex)을 사용한 규칙 기반으로 보완합니다. (하이브리드 접근)
    \item \textbf{미세조정 (Fine-Tuning):} 내 도메인에 맞는 정답 데이터를 수백~수천 건 만들어서 `spaCy` 모델을 추가 학습시킵니다. (위험성 인지!)
\end{enumerate}

\newpage

% ==========
% 부록: 퀴즈 리뷰
% ==========
\appendix
\section{부록: 강의 10 이전 퀴즈 리뷰 (핵심 개념 복습)}
강의 10 본편(NER)에 앞서, 이전 강의들의 핵심 개념들에 대한 퀴즈 리뷰가 진행되었습니다. 다음은 복습을 위한 요약입니다.

\subsection{A1: 나이브 베이즈 (Naive Bayes)}
\begin{itemize}
    \item \textbf{질문:} 나이브 베이즈 분류기의 '나이브(Naive, 순진한)'한 기본 가정은 무엇인가?
    \item \textbf{핵심:} \textbf{특성(Feature) 간의 조건부 독립(Conditional Independence)}을 가정합니다.
    \item \textbf{쉬운 설명:} 스팸 메일을 분류할 때, "viagra"라는 단어의 등장과 "free"라는 단어의 등장이 (스팸이라는 조건 하에서) \textbf{서로 아무런 영향을 주지 않는다}고 '순진하게' 가정하는 것입니다.
    \item \textbf{현실:} 실제로는 "viagra"와 "free"는 함께 등장할 확률이 높습니다. (즉, 독립이 아닙니다.)
    \item \textbf{결론:} 이 가정은 비현실적이지만(Naive), 그럼에도 불구하고 나이브 베이즈는 종종 빠르고 준수한 성능을 보여줍니다.
\end{itemize}

\subsection{A2: K-최근접 이웃 (K-Nearest Neighbors, KNN)}
\begin{itemize}
    \item \textbf{질문:} 키(cm)와 몸무게(kg)처럼 단위(Scale)가 다른 특성들을 KNN에 사용할 때 왜 문제가 되는가?
    \item \textbf{핵심:} KNN은 \textbf{유클리드 거리(Euclidean Distance)}를 기반으로 작동하기 때문입니다.
    \item \textbf{쉬운 설명:}
        \item 두 사람의 키 차이: 170cm vs 180cm $\rightarrow$ 차이 10 $\rightarrow$ 거리 계산 시 $10^2 = 100$
        \item 두 사람의 몸무게 차이: 70kg vs 71kg $\rightarrow$ 차이 1 $\rightarrow$ 거리 계산 시 $1^2 = 1$
    \item \textbf{문제점:} 키(cm)처럼 \textbf{값의 범위가 큰 특성}이 몸무게(kg) 같은 작은 특성보다 거리 계산에 \textbf{훨씬 더 큰 영향}을 미칩니다. 모델이 사실상 '키'만 보고 판단하게 됩니다.
    \item \textbf{해결책:} 모든 특성을 동일한 범위(예: 0~1)로 만드는 \textbf{정규화(Normalization)} 또는 \textbf{표준화(Standardization)} (피처 스케일링)가 필수적입니다.
\end{itemize}

\subsection{A3: 로지스틱 회귀 (Logistic Regression)}
\begin{itemize}
    \item \textbf{질문:} 로지스틱 회귀에서 \textbf{최대가능도추정(Maximum Likelihood Estimation, MLE)}은 어떤 역할을 하는가?
    \item \textbf{핵심:} 우리가 가진 \textbf{관측 데이터(Observations)가 나타날 확률을 최대화}하는 파라미터(가중치)를 찾는 방법입니다.
    \item \textbf{쉬운 설명 (동전 던지기):}
        \item \textbf{데이터:} 동전을 10번 던졌더니 앞면(H) 7번, 뒷면(T) 3번이 나왔다.
        \item \textbf{질문:} 이 동전의 앞면이 나올 확률(P)은 얼마일까?
        \item \textbf{MLE 접근:}
            \item P=0.5라고 가정: (0.5)$^7 \times$ (0.5)$^3$ $\rightarrow$ 매우 낮은 확률
            \item P=0.9라고 가정: (0.9)$^7 \times$ (0.1)$^3$ $\rightarrow$ 낮은 확률
            \item P=0.7이라고 가정: (0.7)$^7 \times$ (0.3)$^3$ $\rightarrow$ \textbf{가장 높은 확률 (Likelihood)}
    \item \textbf{결론:} "7번의 성공과 3번의 실패"라는 데이터를 관찰할 확률(Likelihood)을 \textbf{최대(Maximum)}로 만드는 P는 0.7입니다. 로지스틱 회귀도 이와 같이, 주어진 데이터(X)와 라벨(Y=0 또는 1)이 관측될 확률을 최대로 만드는 최적의 가중치(W)를 찾습니다.
\end{itemize}

\subsection{A4: 서포트 벡터 머신 (SVM)}
\begin{itemize}
    \item \textbf{질문:} SVM에서 \textbf{서포트 벡터(Support Vectors)}란 무엇인가?
    \item \textbf{핵심:} 두 클래스 간의 \textbf{경계(Decision Boundary)를 결정하는 데 직접적인 영향을 미치는} 데이터 포인트들입니다.
    \item \textbf{쉬운 설명:} 서포트 벡터는 두 나라 사이의 '국경선'에 가장 가까이 있는 '최전방 초소'들입니다.
    \item \textbf{특징:} 경계(마진)에서 멀리 떨어진 '후방'의 데이터 포인트들은 아무리 많이 추가되거나 이동해도 경계선에 영향을 주지 않습니다. 오직 이 '최전방 초소'들(서포트 벡터)만이 경계선을 결정합니다.
    \item \textbf{허용치(Allowance, C 파라미터):}
        \item \textbf{Hard Margin (C=무한대):} 단 하나의 오류도 허용하지 않음. 국경선(마진)이 매우 좁고, 아웃라이어에 민감함. (과적합 위험)
        \item \textbf{Soft Margin (C=작은 값):} 일부 데이터가 마진을 넘거나 잘못 분류되는 것을 '허용'함. 마진이 넓어지고, 아웃라이어에 덜 민감함. (일반화 성능 향상)
\end{itemize}

\subsection{A5: 랜덤 포레스트 (Random Forest)}
\begin{itemize}
    \item \textbf{질문:} 단일 결정 트리(Decision Tree)에 비해 랜덤 포레스트가 갖는 주요 이점은 무엇인가?
    \item \textbf{핵심:} \textbf{분산(Variance)을 줄여 일반화 성능을 높입니다.}
    \item \textbf{쉬운 설명:}
        \item \textbf{단일 트리:} 한 명의 '편협한' 전문가. 특정 데이터셋에 과적합(Overfitting)되기 쉬움. (분산이 높다)
        \item \textbf{랜덤 포레스트:} 수백 명의 '다양한' 전문가 집단. 각 전문가(트리)는 데이터를 무작위로 샘플링(배깅)하고, 질문(특성)도 무작위로 선택하여 학습합니다.
    \item \textbf{결론:} 각 트리는 조금씩 편향(Bias)되어 있지만, 이들의 예측을 \textbf{평균(Averaging) 또는 투표(Voting)}함으로써 개별 트리의 오류(분산)가 상쇄됩니다. 이는 모델 전체의 안정성과 일반화 성능을 크게 향상시킵니다.
\end{itemize}

\end{document}
