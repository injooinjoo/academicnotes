%%%%%%%%%%%%%%%%%%%%%%%%%%%%%%%%%%%%%%%%%%%%%%%%%%%%%%%%%%%%%%%%%%%%%%%%%%%%%%%
% Harvard Academic Notes - 통합 마스터 템플릿
% 모든 강의 노트에 적용되는 통일된 스타일
% 버전: 2.1 - 가독성 개선 (선택적 최적화)
% 최종 수정일: 2025-11-17
%%%%%%%%%%%%%%%%%%%%%%%%%%%%%%%%%%%%%%%%%%%%%%%%%%%%%%%%%%%%%%%%%%%%%%%%%%%%%%%

\documentclass[11pt,a4paper]{article}

%========================================================================================
% 기본 패키지
%========================================================================================

% --- 한국어 지원 ---
\usepackage{kotex}

% --- 페이지 레이아웃 ---
\usepackage[top=20mm, bottom=20mm, left=20mm, right=18mm]{geometry}
\usepackage{setspace}
\onehalfspacing                      % 1.5배 줄간격
\setlength{\parskip}{0.5em}          % 문단 간격
\setlength{\parindent}{0pt}          % 들여쓰기 없음

% --- 표 관련 ---
\usepackage{booktabs}              % 고품질 표
\usepackage{tabularx}              % 자동 너비 조절 표
\usepackage{array}                 % 표 컬럼 확장
\usepackage{longtable}             % 여러 페이지 표
\renewcommand{\arraystretch}{1.1}  % 표 행간 조절

%========================================================================================
% 헤더 및 푸터
%========================================================================================

\usepackage{fancyhdr}
\pagestyle{fancy}
\fancyhf{}
\fancyhead[L]{\small\textit{CSCI E-89B: 자연어 처리 입문}}
\fancyhead[R]{\small\textit{Lecture 13}}
\fancyfoot[C]{\thepage}
\renewcommand{\headrulewidth}{0.5pt}
\renewcommand{\footrulewidth}{0.3pt}

% 첫 페이지는 헤더 없음
\fancypagestyle{firstpage}{
    \fancyhf{}
    \fancyfoot[C]{\thepage}
    \renewcommand{\headrulewidth}{0pt}
}

%========================================================================================
% 색상 정의 (파스텔 톤 + 다크모드 호환)
%========================================================================================

\usepackage[dvipsnames]{xcolor}

% 밝은 배경용 파스텔 색상
\definecolor{lightblue}{RGB}{220, 235, 255}      % 부드러운 파랑
\definecolor{lightgreen}{RGB}{220, 255, 235}     % 부드러운 초록
\definecolor{lightyellow}{RGB}{255, 250, 220}    % 부드러운 노랑
\definecolor{lightpurple}{RGB}{240, 230, 255}    % 부드러운 보라
\definecolor{lightgray}{gray}{0.95}              % 밝은 회색
\definecolor{lightpink}{RGB}{255, 235, 245}      % 부드러운 핑크
\definecolor{boxgray}{gray}{0.95}
\definecolor{boxblue}{rgb}{0.9, 0.95, 1.0}
\definecolor{boxred}{rgb}{1.0, 0.95, 0.95}

% 진한 색상 (테두리/제목용)
\definecolor{darkblue}{RGB}{50, 80, 150}
\definecolor{darkgreen}{RGB}{40, 120, 70}
\definecolor{darkorange}{RGB}{200, 100, 30}
\definecolor{darkpurple}{RGB}{100, 60, 150}

%========================================================================================
% 박스 환경 (tcolorbox) - 6가지 타입
%========================================================================================

\usepackage[most]{tcolorbox}
\tcbuselibrary{skins, breakable}

% 1. 개요 박스 (강의 시작 부분)
\newtcolorbox{overviewbox}[1][]{
    enhanced,
    colback=lightpurple,
    colframe=darkpurple,
    fonttitle=\bfseries\large,
    title=📚 강의 개요,
    arc=3mm,
    boxrule=1pt,
    left=8pt,
    right=8pt,
    top=8pt,
    bottom=8pt,
    breakable,
    #1
}

% 2. 요약 박스
\newtcolorbox{summarybox}[1][]{
    enhanced,
    colback=lightblue,
    colframe=darkblue,
    fonttitle=\bfseries,
    title=📝 핵심 요약,
    arc=2mm,
    boxrule=0.7pt,
    left=6pt,
    right=6pt,
    top=6pt,
    bottom=6pt,
    breakable,
    #1
}

% 3. 핵심 정보 박스
\newtcolorbox{infobox}[1][]{
    enhanced,
    colback=lightgreen,
    colframe=darkgreen,
    fonttitle=\bfseries,
    title=💡 핵심 정보,
    arc=2mm,
    boxrule=0.7pt,
    left=6pt,
    right=6pt,
    top=6pt,
    bottom=6pt,
    breakable,
    #1
}

% 4. 주의사항 박스
\newtcolorbox{warningbox}[1][]{
    enhanced,
    colback=lightyellow,
    colframe=darkorange,
    fonttitle=\bfseries,
    title=⚠️ 주의사항,
    arc=2mm,
    boxrule=0.7pt,
    left=6pt,
    right=6pt,
    top=6pt,
    bottom=6pt,
    breakable,
    #1
}

% 5. 예제 박스
\newtcolorbox{examplebox}[1][]{
    enhanced,
    colback=lightgray,
    colframe=black!60,
    fonttitle=\bfseries,
    title=📖 예제: #1,
    arc=2mm,
    boxrule=0.7pt,
    left=6pt,
    right=6pt,
    top=6pt,
    bottom=6pt,
    breakable,
}

% 6. 정의 박스
\newtcolorbox{definitionbox}[1][]{
    enhanced,
    colback=lightpink,
    colframe=purple!70!black,
    fonttitle=\bfseries,
    title=📌 정의: #1,
    arc=2mm,
    boxrule=0.7pt,
    left=6pt,
    right=6pt,
    top=6pt,
    bottom=6pt,
    breakable,
}

% 7. 중요 박스 (importantbox - warningbox와 유사)
\newtcolorbox{importantbox}[1][]{
    enhanced,
    colback=boxred,
    colframe=red!70!black,
    fonttitle=\bfseries,
    title=⚠️ 매우 중요: #1,
    arc=2mm,
    boxrule=0.7pt,
    left=6pt,
    right=6pt,
    top=6pt,
    bottom=6pt,
    breakable,
}

% 8. cautionbox (warningbox와 동일)
\let\cautionbox\warningbox
\let\endcautionbox\endwarningbox

%========================================================================================
% 코드 블록 설정 (밝은 배경)
%========================================================================================

\usepackage{listings}

\definecolor{codegray}{rgb}{0.5,0.5,0.5}
\definecolor{codepurple}{rgb}{0.58,0,0.82}
\definecolor{backcolour}{rgb}{0.95,0.95,0.95}

\lstset{
    basicstyle=\ttfamily\small,
    backgroundcolor=\color{lightgray},
    keywordstyle=\color{darkblue}\bfseries,
    commentstyle=\color{darkgreen}\itshape,
    stringstyle=\color{purple!80!black},
    numberstyle=\tiny\color{black!60},
    numbers=left,
    numbersep=8pt,
    breaklines=true,
    breakatwhitespace=false,
    frame=single,
    frameround=tttt,
    rulecolor=\color{black!30},
    captionpos=b,
    showstringspaces=false,
    tabsize=2,
    xleftmargin=15pt,
    xrightmargin=5pt,
    escapeinside={\%*}{*)}
}

% Python 코드 스타일
\lstdefinestyle{pythonstyle}{
    language=Python,
    morekeywords={self, True, False, None},
}

% SQL 코드 스타일
\lstdefinestyle{sqlstyle}{
    language=SQL,
    morekeywords={SELECT, FROM, WHERE, JOIN, GROUP, BY, ORDER, HAVING},
}

%========================================================================================
% 목차 스타일링
%========================================================================================

\usepackage{tocloft}
\renewcommand{\cftsecleader}{\cftdotfill{\cftdotsep}}
\setlength{\cftbeforesecskip}{0.4em}
\renewcommand{\cftsecfont}{\bfseries}
\renewcommand{\cftsubsecfont}{\normalfont}

%========================================================================================
% 표 및 그림
%========================================================================================

\usepackage{graphicx}              % 이미지
\usepackage{adjustbox}             % 표/박스 크기 조절

% 표 캡션 스타일
\usepackage{caption}
\captionsetup[table]{
    labelfont=bf,
    textfont=it,
    skip=5pt
}
\captionsetup[figure]{
    labelfont=bf,
    textfont=it,
    skip=5pt
}

%========================================================================================
% 수학
%========================================================================================

\usepackage{amsmath, amssymb, amsthm}

% 정리 환경
\theoremstyle{definition}
\newtheorem{theorem}{정리}[section]
\newtheorem{lemma}[theorem]{보조정리}
\newtheorem{proposition}[theorem]{명제}
\newtheorem{corollary}[theorem]{따름정리}
\newtheorem{definition}{정의}[section]
\newtheorem{example}{예제}[section]

%========================================================================================
% 하이퍼링크
%========================================================================================

\usepackage[
    colorlinks=true,
    linkcolor=blue!80!black,
    urlcolor=blue!80!black,
    citecolor=green!60!black,
    bookmarks=true,
    bookmarksnumbered=true,
    pdfborder={0 0 0}
]{hyperref}

% PDF 메타데이터는 각 문서에서 설정
\hypersetup{
    pdftitle={CSCI E-89B: 자연어 처리 입문 - Lecture 13},
    pdfauthor={강의 노트},
    pdfsubject={Academic Notes}
}

%========================================================================================
% 기타 유용한 패키지
%========================================================================================

\usepackage{enumitem}              % 리스트 커스터마이징
\setlist{nosep, leftmargin=*, itemsep=0.3em}

\usepackage{microtype}             % 타이포그래피 개선
\usepackage{footnote}              % 각주 개선
\usepackage{url}                   % URL 줄바꿈
\urlstyle{same}

%========================================================================================
% 사용자 정의 명령어
%========================================================================================

% 강조 텍스트
\newcommand{\important}[1]{\textbf{\textcolor{red!70!black}{#1}}}
\newcommand{\keyword}[1]{\textbf{#1}}
\newcommand{\term}[1]{\textit{#1}}
\newcommand{\code}[1]{\texttt{#1}}

% 용어 설명 (인라인)
\newcommand{\defterm}[2]{\textbf{#1}\footnote{#2}}

% 섹션 시작 전 페이지 분리
\newcommand{\newsection}[1]{\newpage\section{#1}}

%========================================================================================
% 문서 제목 스타일
%========================================================================================

\usepackage{titling}
\pretitle{\begin{center}\LARGE\bfseries}
\posttitle{\par\end{center}\vskip 0.5em}
\preauthor{\begin{center}\large}
\postauthor{\end{center}}
\predate{\begin{center}\large}
\postdate{\par\end{center}}

%========================================================================================
% 섹션 제목 간격
%========================================================================================

\usepackage{titlesec}
\titlespacing*{\section}{0pt}{1.5em}{0.8em}
\titlespacing*{\subsection}{0pt}{1.2em}{0.6em}
\titlespacing*{\subsubsection}{0pt}{1em}{0.5em}

%========================================================================================
% 메타 정보 박스 명령어
%========================================================================================

\newcommand{\metainfo}[4]{
\begin{tcolorbox}[
    colback=lightpurple,
    colframe=darkpurple,
    boxrule=1pt,
    arc=2mm,
    left=10pt,
    right=10pt,
    top=8pt,
    bottom=8pt
]
\begin{tabular}{@{}rl@{}}
▣ \textbf{강의명:} & #1 \\[0.3em]
▣ \textbf{주차:} & #2 \\[0.3em]
▣ \textbf{교수명:} & #3 \\[0.3em]
▣ \textbf{목적:} & \begin{minipage}[t]{0.75\textwidth}#4\end{minipage}
\end{tabular}
\end{tcolorbox}
}

%========================================================================================
% 끝
%========================================================================================


\begin{document}

\metainfo{CSCI E-89B: 자연어 처리 입문}{Lecture 13}{Dmitry Kurochkin}{Lecture 13의 핵심 개념 학습}

% 제목 섹션
\begin{center}
    \vspace*{1cm}
    {\Huge \textbf{기계 번역(Machine Translation) 심층 분석}} \\[0.5cm]
    {\Large 자연어 처리(NLP) 입문 - 13주차 강의 통합 요약} \\[0.5cm]
    \today
    \vspace*{1cm}
\end{center}

\begin{summarybox}{문서 개요 (Executive Summary)}
    \begin{itemize}
        \item \textbf{기계 번역(MT)}은 소프트웨어를 통해 인간 수준의 번역 품질을 목표로 하는 기술입니다.
        \item 초기 \textbf{규칙 기반(RBMT)}에서 \textbf{통계 기반(SMT)}을 거쳐, 현재는 \textbf{신경망 기반(NMT)}이 주류입니다.
        \item \textbf{Seq2Seq} 모델은 입력과 출력을 연결하는 혁신을 가져왔으나 병목 현상이 있었고, 이를 \textbf{어텐션(Attention)}과 \textbf{트랜스포머(Transformer)}가 해결했습니다.
        \item 번역 품질 평가는 주로 \textbf{BLEU 점수}를 사용하며, 최근에는 규칙과 신경망을 합친 \textbf{하이브리드 모델}도 특정 분야에서 강세를 보입니다.
    \end{itemize}
\end{summarybox}

\newpage
\tableofcontents
\newpage

% -----------------------------------------------------------------------------
\section{기본 개념 및 용어 정리}
% -----------------------------------------------------------------------------

기계 번역을 처음 접할 때 반드시 알아야 할 핵심 용어들입니다. 이 용어들은 문서 전체에서 반복되므로 미리 숙지하면 이해가 빠릅니다.

\begin{table}[h]
    \centering
    \caption{기계 번역 핵심 용어 정의}
    \label{tab:terminology}
    \renewcommand{\arraystretch}{1.5}
    \resizebox{\textwidth}{!}{
    \begin{tabular}{l l p{8cm}}
        \toprule
        \textbf{용어 (약어)} & \textbf{원어} & \textbf{쉬운 설명 및 비고} \\ 
        \midrule
        \textbf{MT} & Machine Translation & 컴퓨터가 한 언어를 다른 언어로 번역하는 기술 전반. \\ 
        \textbf{RBMT} & Rule-Based MT & 1950~80년대 방식. 사람이 직접 문법 규칙과 사전을 입력해줌. (예: "I am" $\to$ "나는 이다") \\ 
        \textbf{SMT} & Statistical MT & 1990~2010년대 방식. 수많은 번역 데이터에서 '확률'을 계산해 번역. (예: 이 단어 뒤엔 저 단어가 올 확률이 80\%) \\ 
        \textbf{NMT} & Neural MT & 2010년 이후 방식. 딥러닝(인공신경망)을 사용해 문장 전체의 맥락을 이해하고 번역. 현재의 주류. \\ 
        \textbf{Seq2Seq} & Sequence-to-Sequence & 입력 문장(Sequence)을 받아 출력 문장(Sequence)을 만들어내는 딥러닝 모델 구조. \\ 
        \textbf{Attention} & Attention Mechanism & 긴 문장을 번역할 때, "지금 번역하는 단어와 관련된 입력 단어"에 집중(Attention)하게 만드는 기술. \\ 
        \textbf{Transformer} & Transformer & RNN을 버리고 'Attention'만으로 구성하여 속도와 성능을 획기적으로 높인 현대 AI의 기반 모델 (BERT, GPT의 조상). \\ 
        \textbf{BLEU} & Bilingual Evaluation Understudy & 기계 번역이 얼마나 잘 됐는지 채점하는 점수. 인간 번역과 얼마나 겹치는지 확인. \\ 
        \bottomrule
    \end{tabular}
    }
\end{table}

% -----------------------------------------------------------------------------
\section{기계 번역(MT)의 개요와 발전사}
% -----------------------------------------------------------------------------

\subsection{기계 번역이란 무엇인가?}
기계 번역은 단순히 단어를 1:1로 바꾸는 것이 아닙니다. 목표는 **"의미와 문화적 뉘앙스를 보존하면서 문법적으로 정확한 문장을 만드는 것"**입니다.
\begin{itemize}
    \item \textbf{목표:} 인간 번역가의 품질을 복제하는 것.
    \item \textbf{용도:} 국제 비즈니스, 외교 문서, 실시간 회의 통역, 관용구(예: "hit the nail on the head" $\to$ "정곡을 찌르다")의 적절한 의역 등.
\end{itemize}

\subsection{발전의 3단계 흐름}
기계 번역은 크게 세 가지 시대를 거쳐 발전했습니다. "왜 이전 기술이 도태되었는가?"를 이해하는 것이 중요합니다.

\begin{conceptbox}{1. 규칙 기반 (RBMT, 1950s--1980s)}
    \begin{itemize}
        \item \textbf{방식:} 언어학자가 문법 규칙과 사전을 일일이 코딩했습니다.
        \item \textbf{장점:} 날씨 예보나 법률 문서처럼 형식이 정해진 문서는 아주 정확합니다.
        \item \textbf{단점:} 예외가 너무 많아 확장이 어렵고, 자연스러운 구어체 번역에 실패합니다.
        \item \textbf{사례:} 1954년 조지타운-IBM 실험 (러시아어 $\to$ 영어).
    \end{itemize}
\end{conceptbox}

\begin{conceptbox}{2. 통계 기반 (SMT, 1990s--2010s)}
    \begin{itemize}
        \item \textbf{방식:} 수많은 이중 언어 데이터(예: UN 회의록)를 통계적으로 분석해 "이 단어 옆엔 저 단어가 온다"는 확률을 학습합니다.
        \item \textbf{장점:} 규칙을 일일이 짤 필요가 없고, 데이터가 많으면 성능이 좋아집니다.
        \item \textbf{단점:} 문맥보다는 '구(Phrase)' 단위로 쪼개서 번역하므로 문장이 뚝뚝 끊기는 느낌이 듭니다.
    \end{itemize}
\end{conceptbox}

\begin{conceptbox}{3. 신경망 기반 (NMT, 2010s--Present)}
    \begin{itemize}
        \item \textbf{방식:} 딥러닝(Deep Learning)을 사용해 문장 전체를 하나의 벡터(숫자 덩어리)로 이해하고 번역합니다.
        \item \textbf{장점:} **"End-to-End" 학습** (입력에서 출력까지 한 번에 학습). 문맥을 파악하여 훨씬 유창하고 자연스럽습니다.
        \item \textbf{현재:} Seq2Seq와 Transformer 모델이 이 시대를 이끌고 있습니다.
    \end{itemize}
\end{conceptbox}

% -----------------------------------------------------------------------------
\section{기계 번역의 난제 (Challenges)}
% -----------------------------------------------------------------------------

번역은 왜 어려울까요? 컴퓨터 입장에서 가장 헷갈리는 문제들입니다.

\begin{enumerate}
    \item \textbf{언어의 모호성 (Ambiguity)}
    \begin{itemize}
        \item 단어 하나가 여러 뜻을 가집니다.
        \item \textit{예시:} 영어 "Bank"는 '은행'일 수도, '강둑'일 수도 있습니다. 문맥 없이 "I went to the bank"만 보고는 알 수 없습니다.
    \end{itemize}
    
    \item \textbf{구조적 차이 (Structural Differences)}
    \begin{itemize}
        \item 어순이 다릅니다.
        \item \textit{예시:} 영어는 SVO (주어-동사-목적어)인 반면, 한국어는 SOV (주어-목적어-동사)입니다. 이를 맞추려면 문장 전체를 다 듣고 재배열해야 합니다.
    \end{itemize}
    
    \item \textbf{관용구 (Idiomatic Expressions)}
    \begin{itemize}
        \item 직역하면 의미가 통하지 않습니다.
        \item \textit{예시:} "Shoot the breeze" $\to$ (직역) 산들바람을 쏘다? $\to$ (의역) 수다를 떨다.
    \end{itemize}
    
    \item \textbf{문맥 보존 (Context Preservation)}
    \begin{itemize}
        \item 긴 문단에서 앞 내용을 기억해야 합니다. 대명사(그것, 그 사람)가 무엇을 가리키는지 파악하는 것은 매우 어렵습니다.
    \end{itemize}
\end{enumerate}

% -----------------------------------------------------------------------------
\section{핵심 기술 1: Seq2Seq 모델}
% -----------------------------------------------------------------------------

\subsection{Seq2Seq의 구조}
NMT(신경망 번역)의 시초가 된 모델입니다. 크게 두 부분으로 나뉩니다.

\begin{itemize}
    \item \textbf{인코더 (Encoder):} 원문을 읽고 그 의미를 압축하여 하나의 **'컨텍스트 벡터(Context Vector)'**로 만듭니다. (독해 담당)
    \item \textbf{디코더 (Decoder):} 컨텍스트 벡터를 받아서 도착 언어로 문장을 생성합니다. (작문 담당)
\end{itemize}

\begin{warningbox}{Seq2Seq의 치명적 단점: 병목 현상 (Bottleneck)}
    초기 Seq2Seq는 긴 문장을 아주 작은 벡터 하나에 억지로 구겨 넣어야 했습니다.
    
    \textbf{비유:} 책 한 권을 읽고 내용을 단 한 문장으로 요약한 뒤, 그 요약본만 보고 다시 책 전체를 다른 언어로 써내라는 것과 같습니다. 정보 손실이 발생하여 긴 문장 번역 품질이 떨어집니다.
\end{warningbox}

\subsection{해결책: 어텐션 메커니즘 (Attention Mechanism)}
어텐션은 **"다 기억하려 하지 말고, 필요할 때 원문을 다시 컨닝하자!"**는 아이디어입니다.
\begin{itemize}
    \item 디코더가 번역할 때, 문장의 모든 부분을 동일하게 보는 게 아니라 **관련된 단어에 가중치(집중)**를 둡니다.
    \item 병목 현상을 해결하고 긴 문장 번역 성능을 획기적으로 높였습니다.
\end{itemize}

% -----------------------------------------------------------------------------
\section{핵심 기술 2: 트랜스포머 (Transformers)}
% -----------------------------------------------------------------------------

2017년 논문 \textit{"Attention is All You Need"}에서 발표된 모델로, 현대 AI(GPT, BERT)의 조상입니다.

\subsection{기존 모델(RNN/LSTM)과의 차이}
\begin{itemize}
    \item \textbf{기존:} 단어를 순서대로 하나씩 처리(순차적). 시간이 오래 걸리고 앞부분 내용을 까먹기 쉽습니다.
    \item \textbf{트랜스포머:} 문장 전체를 한 번에 처리(\textbf{병렬 처리}). 속도가 매우 빠르고 문장 내 단어 간의 관계를 한눈에 파악합니다.
\end{itemize}

\subsection{핵심 구성 요소}
\begin{enumerate}
    \item \textbf{Self-Attention (셀프 어텐션):} 문장 내의 단어들이 서로 어떤 관계가 있는지 계산합니다. (예: "그것"이 앞의 "사과"를 가리키는지, "바나나"를 가리키는지 파악)
    \item \textbf{Positional Encoding (위치 인코딩):} 한꺼번에 입력받으므로 단어 순서 정보를 인위적으로 추가해줍니다.
    \item \textbf{Encoder-Decoder 구조:} 여전히 번역을 위해 인코더와 디코더 구조를 유지하지만, 내부는 전부 어텐션으로 채워져 있습니다.
\end{enumerate}

\begin{summarybox}{트랜스포머의 충격적인 효과}
    \begin{itemize}
        \item \textbf{확장성:} 병렬 처리가 가능해져서 엄청나게 큰 데이터로 학습이 가능해졌습니다.
        \item \textbf{범용성:} 번역뿐만 아니라 요약, 챗봇, 심지어 이미지 분석(Vision Transformer)이나 단백질 구조 예측에도 쓰입니다.
    \end{itemize}
\end{summarybox}

% -----------------------------------------------------------------------------
\section{평가 지표: BLEU Score}
% -----------------------------------------------------------------------------

기계 번역이 잘 됐는지 사람이 일일이 채점할 수 없으므로, **자동화된 채점 방식**이 필요합니다.

\subsection{BLEU (Bilingual Evaluation Understudy)의 개념}
기계가 번역한 문장이 **사람(참조, Reference)이 번역한 문장과 얼마나 단어가 많이 겹치는가**를 측정합니다.

\subsection{계산 방법 (직관적 이해)}
\begin{enumerate}
    \item \textbf{N-gram 정밀도 (Precision):} 단어(1-gram), 두 단어 짝(2-gram), 세 단어 짝(3-gram)... 이 얼마나 일치하는지 봅니다.
    \begin{equation}
        \text{Precision}_n = \frac{\sum \text{Matches (일치하는 n-gram 수)}}{\sum \text{Total (기계 번역의 총 n-gram 수)}}
    \end{equation}
    \item \textbf{간결성 페널티 (Brevity Penalty, BP):} 
    기계가 틀릴까 봐 단어 하나만 내뱉는 꼼수를 부리지 못하게, \textbf{길이가 참조 문장보다 짧으면 점수를 깎습니다.}
    \item \textbf{최종 점수:} N-gram 정밀도의 평균에 BP를 곱해서 계산합니다. (1에 가까울수록 완벽, 0에 가까울수록 엉망)
\end{enumerate}

\begin{warningbox}{BLEU 점수의 한계}
    \begin{itemize}
        \item \textbf{의미 파악 불가:} "아름답다"를 "예쁘다"로 번역하면, 뜻은 통하지만 단어가 달라서 점수가 깎일 수 있습니다. (동의어 처리 미흡)
        \item \textbf{어순 유연성 부족:} 한국어처럼 어순이 자유로운 언어에서는 점수가 부정확할 수 있습니다.
    \end{itemize}
\end{warningbox}

% -----------------------------------------------------------------------------
\section{하이브리드 모델 (Hybrid Models)}
% -----------------------------------------------------------------------------

최근에는 하나의 방식만 고집하지 않고 섞어서 씁니다.

\subsection{왜 섞어 쓰는가?}
\begin{itemize}
    \item \textbf{NMT(신경망)의 약점:} 가끔 엉뚱한 번역을 하거나 전문 용어를 틀립니다.
    \item \textbf{RBMT/SMT(규칙/통계)의 강점:} 문법이 정확하고 특정 분야 용어 사전을 강제로 적용하기 좋습니다.
\end{itemize}

\subsection{구성 요소 및 장점}
\begin{itemize}
    \item \textbf{신경망 + 규칙:} 전체적인 번역은 신경망이 유창하게 하고, 법률 용어나 문법 교정은 규칙 기반 시스템이 마무리합니다.
    \item \textbf{장점:} 유창함(Fluency)과 정확성(Accuracy)을 동시에 잡을 수 있습니다.
    \item \textbf{사례:} SYSTRAN, Microsoft Translator, IBM Watson (의료, 법률 등 특수 분야용).
\end{itemize}

% -----------------------------------------------------------------------------
\section{학습 체크리스트 및 FAQ}
% -----------------------------------------------------------------------------

\subsection{학습 완료 체크리스트}
\begin{itemize}
    \item [ ] RBMT, SMT, NMT의 차이점과 발전 순서를 설명할 수 있는가?
    \item [ ] 기계 번역의 4가지 난제(모호성, 구조, 관용구, 문맥)를 예시와 함께 말할 수 있는가?
    \item [ ] Seq2Seq의 '병목 현상'이 무엇이며, '어텐션'이 이를 어떻게 해결했는지 이해했는가?
    \item [ ] 트랜스포머가 RNN보다 빠른 이유(병렬 처리)를 아는가?
    \item [ ] BLEU 점수의 계산 원리(N-gram 일치)와 한계점(의미 파악 불가)을 아는가?
\end{itemize}

\subsection{FAQ: 자주 묻는 질문}

\begin{description}
    \item[Q. 트랜스포머는 번역에만 쓰이나요?] 
    \textbf{A.} 아니요. 트랜스포머는 현재 AI의 표준입니다. 챗GPT 같은 대화형 AI, 요약, 문장 생성, 심지어 코딩 작성까지 모든 자연어 처리에 쓰입니다.
    
    \item[Q. BLEU 점수가 높으면 무조건 좋은 번역인가요?] 
    \textbf{A.} 꼭 그렇지는 않습니다. 점수는 높아도 사람이 읽기에 어색할 수 있고, 점수는 낮아도 의미가 완벽하게 통할 수 있습니다. 그래서 사람에 의한 평가도 여전히 중요합니다.
    
    \item[Q. 요즘도 규칙 기반(RBMT)을 쓰나요?] 
    \textbf{A.} 순수 RBMT는 거의 안 쓰지만, 하이브리드 모델의 일부로써 문법 교정이나 특정 용어 강제 적용을 위해 여전히 중요한 역할을 합니다.
\end{description}

\newpage

% -----------------------------------------------------------------------------
\section{부록: 1페이지 요약 (Cheat Sheet)}
% -----------------------------------------------------------------------------

\begin{tcolorbox}[colback=white, colframe=black, title=\textbf{Machine Translation 한 장 정리}]
    \begin{center}
        \textbf{1. 역사적 흐름 (Evolution)}
    \end{center}
    \begin{itemize}
        \item \textbf{Rule-Based (50s-80s):} 규칙+사전. 정확하지만 확장 불가. (비유: 문법책 보고 작문)
        \item \textbf{Statistical (90s-10s):} 데이터 확률 기반. 유연하지만 문맥 부족. (비유: 단어 맞추기 퍼즐)
        \item \textbf{Neural (10s-Present):} 딥러닝(Seq2Seq, Transformer). 유창하고 문맥 파악. (비유: 전문 통역사)
    \end{itemize}

    \begin{center}
        \textbf{2. 핵심 모델 (Models)}
    \end{center}
    \begin{itemize}
        \item \textbf{Seq2Seq:} 인코더 $\to$ 컨텍스트 벡터 $\to$ 디코더. (단점: 정보 병목)
        \item \textbf{Attention:} "필요한 부분만 집중해서 보자". 병목 해결, 긴 문장 OK.
        \item \textbf{Transformer:} "Attention is All You Need". 병렬 처리로 초고속 학습, 현대 AI의 기반.
    \end{itemize}

    \begin{center}
        \textbf{3. 평가 (Evaluation - BLEU)}
    \end{center}
    \begin{itemize}
        \item \textbf{원리:} 기계 번역과 사람 번역의 N-gram(단어 뭉치) 일치도 측정.
        \item \textbf{특징:} 1.0 만점. 짧으면 페널티(BP). 동의어/의미 파악은 못함.
    \end{itemize}
    
    \begin{center}
        \textbf{4. 하이브리드 (Hybrid)}
    \end{center}
    \begin{itemize}
        \item 신경망의 유창함 + 통계/규칙의 정확성 결합.
        \item 법률, 의료 등 전문 분야에서 선호됨.
    \end{itemize}
\end{tcolorbox}

\end{document}
