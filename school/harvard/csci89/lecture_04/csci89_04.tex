%%%%%%%%%%%%%%%%%%%%%%%%%%%%%%%%%%%%%%%%%%%%%%%%%%%%%%%%%%%%%%%%%%%%%%%%%%%%%%%
% Harvard Academic Notes - 통합 마스터 템플릿
% 모든 강의 노트에 적용되는 통일된 스타일
% 버전: 2.0
% 최종 수정일: 2025-10-26
%%%%%%%%%%%%%%%%%%%%%%%%%%%%%%%%%%%%%%%%%%%%%%%%%%%%%%%%%%%%%%%%%%%%%%%%%%%%%%%

\documentclass[11pt,a4paper]{article}

%========================================================================================
% 기본 패키지
%========================================================================================

% --- 한국어 지원 ---
\usepackage{kotex}

% --- 페이지 레이아웃 ---
\usepackage[margin=25mm]{geometry}
\usepackage{setspace}
\onehalfspacing                      % 1.5배 줄간격
\setlength{\parskip}{0.6em}          % 문단 간격
\setlength{\parindent}{0pt}          % 들여쓰기 없음

% --- 표 관련 ---
\usepackage{booktabs}              % 고품질 표
\usepackage{tabularx}              % 자동 너비 조절 표
\usepackage{array}                 % 표 컬럼 확장
\usepackage{longtable}             % 여러 페이지 표
\renewcommand{\arraystretch}{1.2}  % 표 행간 조절

%========================================================================================
% 헤더 및 푸터
%========================================================================================

\usepackage{fancyhdr}
\pagestyle{fancy}
\fancyhf{}
\fancyhead[L]{\small\textit{CSCI E-89B: 자연어 처리 입문}}
\fancyhead[R]{\small\textit{Lecture 04}}
\fancyfoot[C]{\thepage}
\renewcommand{\headrulewidth}{0.5pt}
\renewcommand{\footrulewidth}{0.3pt}

% 첫 페이지는 헤더 없음
\fancypagestyle{firstpage}{
    \fancyhf{}
    \fancyfoot[C]{\thepage}
    \renewcommand{\headrulewidth}{0pt}
}

%========================================================================================
% 색상 정의 (파스텔 톤 + 다크모드 호환)
%========================================================================================

\usepackage[dvipsnames]{xcolor}

% 밝은 배경용 파스텔 색상
\definecolor{lightblue}{RGB}{220, 235, 255}      % 부드러운 파랑
\definecolor{lightgreen}{RGB}{220, 255, 235}     % 부드러운 초록
\definecolor{lightyellow}{RGB}{255, 250, 220}    % 부드러운 노랑
\definecolor{lightpurple}{RGB}{240, 230, 255}    % 부드러운 보라
\definecolor{lightgray}{gray}{0.95}              % 밝은 회색
\definecolor{lightpink}{RGB}{255, 235, 245}      % 부드러운 핑크
\definecolor{boxgray}{gray}{0.95}
\definecolor{boxblue}{rgb}{0.9, 0.95, 1.0}
\definecolor{boxred}{rgb}{1.0, 0.95, 0.95}

% 진한 색상 (테두리/제목용)
\definecolor{darkblue}{RGB}{50, 80, 150}
\definecolor{darkgreen}{RGB}{40, 120, 70}
\definecolor{darkorange}{RGB}{200, 100, 30}
\definecolor{darkpurple}{RGB}{100, 60, 150}

%========================================================================================
% 박스 환경 (tcolorbox) - 6가지 타입
%========================================================================================

\usepackage[most]{tcolorbox}
\tcbuselibrary{skins, breakable}

% 1. 개요 박스 (강의 시작 부분)
\newtcolorbox{overviewbox}[1][]{
    enhanced,
    colback=lightpurple,
    colframe=darkpurple,
    fonttitle=\bfseries\large,
    title=📚 강의 개요,
    arc=3mm,
    boxrule=1pt,
    left=8pt,
    right=8pt,
    top=8pt,
    bottom=8pt,
    breakable,
    #1
}

% 2. 요약 박스
\newtcolorbox{summarybox}[1][]{
    enhanced,
    colback=lightblue,
    colframe=darkblue,
    fonttitle=\bfseries,
    title=📝 핵심 요약,
    arc=2mm,
    boxrule=0.7pt,
    left=6pt,
    right=6pt,
    top=6pt,
    bottom=6pt,
    breakable,
    #1
}

% 3. 핵심 정보 박스
\newtcolorbox{infobox}[1][]{
    enhanced,
    colback=lightgreen,
    colframe=darkgreen,
    fonttitle=\bfseries,
    title=💡 핵심 정보,
    arc=2mm,
    boxrule=0.7pt,
    left=6pt,
    right=6pt,
    top=6pt,
    bottom=6pt,
    breakable,
    #1
}

% 4. 주의사항 박스
\newtcolorbox{warningbox}[1][]{
    enhanced,
    colback=lightyellow,
    colframe=darkorange,
    fonttitle=\bfseries,
    title=⚠️ 주의사항,
    arc=2mm,
    boxrule=0.7pt,
    left=6pt,
    right=6pt,
    top=6pt,
    bottom=6pt,
    breakable,
    #1
}

% 5. 예제 박스
\newtcolorbox{examplebox}[1][]{
    enhanced,
    colback=lightgray,
    colframe=black!60,
    fonttitle=\bfseries,
    title=📖 예제: #1,
    arc=2mm,
    boxrule=0.7pt,
    left=6pt,
    right=6pt,
    top=6pt,
    bottom=6pt,
    breakable,
}

% 6. 정의 박스
\newtcolorbox{definitionbox}[1][]{
    enhanced,
    colback=lightpink,
    colframe=purple!70!black,
    fonttitle=\bfseries,
    title=📌 정의: #1,
    arc=2mm,
    boxrule=0.7pt,
    left=6pt,
    right=6pt,
    top=6pt,
    bottom=6pt,
    breakable,
}

% 7. 중요 박스 (importantbox - warningbox와 유사)
\newtcolorbox{importantbox}[1][]{
    enhanced,
    colback=boxred,
    colframe=red!70!black,
    fonttitle=\bfseries,
    title=⚠️ 매우 중요: #1,
    arc=2mm,
    boxrule=0.7pt,
    left=6pt,
    right=6pt,
    top=6pt,
    bottom=6pt,
    breakable,
}

% 8. cautionbox (warningbox와 동일)
\let\cautionbox\warningbox
\let\endcautionbox\endwarningbox

%========================================================================================
% 코드 블록 설정 (밝은 배경)
%========================================================================================

\usepackage{listings}

\definecolor{codegray}{rgb}{0.5,0.5,0.5}
\definecolor{codepurple}{rgb}{0.58,0,0.82}
\definecolor{backcolour}{rgb}{0.95,0.95,0.95}

\lstset{
    basicstyle=\ttfamily\small,
    backgroundcolor=\color{lightgray},
    keywordstyle=\color{darkblue}\bfseries,
    commentstyle=\color{darkgreen}\itshape,
    stringstyle=\color{purple!80!black},
    numberstyle=\tiny\color{black!60},
    numbers=left,
    numbersep=8pt,
    breaklines=true,
    breakatwhitespace=false,
    frame=single,
    frameround=tttt,
    rulecolor=\color{black!30},
    captionpos=b,
    showstringspaces=false,
    tabsize=2,
    xleftmargin=15pt,
    xrightmargin=5pt,
    escapeinside={\%*}{*)}
}

% Python 코드 스타일
\lstdefinestyle{pythonstyle}{
    language=Python,
    morekeywords={self, True, False, None},
}

% SQL 코드 스타일
\lstdefinestyle{sqlstyle}{
    language=SQL,
    morekeywords={SELECT, FROM, WHERE, JOIN, GROUP, BY, ORDER, HAVING},
}

%========================================================================================
% 목차 스타일링
%========================================================================================

\usepackage{tocloft}
\renewcommand{\cftsecleader}{\cftdotfill{\cftdotsep}}
\setlength{\cftbeforesecskip}{0.4em}
\renewcommand{\cftsecfont}{\bfseries}
\renewcommand{\cftsubsecfont}{\normalfont}

%========================================================================================
% 표 및 그림
%========================================================================================

\usepackage{graphicx}              % 이미지
\usepackage{adjustbox}             % 표/박스 크기 조절

% 표 캡션 스타일
\usepackage{caption}
\captionsetup[table]{
    labelfont=bf,
    textfont=it,
    skip=5pt
}
\captionsetup[figure]{
    labelfont=bf,
    textfont=it,
    skip=5pt
}

%========================================================================================
% 수학
%========================================================================================

\usepackage{amsmath, amssymb, amsthm}

% 정리 환경
\theoremstyle{definition}
\newtheorem{theorem}{정리}[section]
\newtheorem{lemma}[theorem]{보조정리}
\newtheorem{proposition}[theorem]{명제}
\newtheorem{corollary}[theorem]{따름정리}
\newtheorem{definition}{정의}[section]
\newtheorem{example}{예제}[section]

%========================================================================================
% 하이퍼링크
%========================================================================================

\usepackage[
    colorlinks=true,
    linkcolor=blue!80!black,
    urlcolor=blue!80!black,
    citecolor=green!60!black,
    bookmarks=true,
    bookmarksnumbered=true,
    pdfborder={0 0 0}
]{hyperref}

% PDF 메타데이터는 각 문서에서 설정
\hypersetup{
    pdftitle={CSCI E-89B: 자연어 처리 입문 - Lecture 04},
    pdfauthor={강의 노트},
    pdfsubject={Academic Notes}
}

%========================================================================================
% 기타 유용한 패키지
%========================================================================================

\usepackage{enumitem}              % 리스트 커스터마이징
\setlist{nosep, leftmargin=*, itemsep=0.3em}

\usepackage{microtype}             % 타이포그래피 개선
\usepackage{footnote}              % 각주 개선
\usepackage{url}                   % URL 줄바꿈
\urlstyle{same}

%========================================================================================
% 사용자 정의 명령어
%========================================================================================

% 강조 텍스트
\newcommand{\important}[1]{\textbf{\textcolor{red!70!black}{#1}}}
\newcommand{\keyword}[1]{\textbf{#1}}
\newcommand{\term}[1]{\textit{#1}}
\newcommand{\code}[1]{\texttt{#1}}

% 용어 설명 (인라인)
\newcommand{\defterm}[2]{\textbf{#1}\footnote{#2}}

% 섹션 시작 전 페이지 분리
\newcommand{\newsection}[1]{\newpage\section{#1}}

%========================================================================================
% 문서 제목 스타일
%========================================================================================

\usepackage{titling}
\pretitle{\begin{center}\LARGE\bfseries}
\posttitle{\par\end{center}\vskip 0.5em}
\preauthor{\begin{center}\large}
\postauthor{\end{center}}
\predate{\begin{center}\large}
\postdate{\par\end{center}}

%========================================================================================
% 섹션 제목 간격
%========================================================================================

\usepackage{titlesec}
\titlespacing*{\section}{0pt}{1.5em}{0.8em}
\titlespacing*{\subsection}{0pt}{1.2em}{0.6em}
\titlespacing*{\subsubsection}{0pt}{1em}{0.5em}

%========================================================================================
% 메타 정보 박스 명령어
%========================================================================================

\newcommand{\metainfo}[4]{
\begin{tcolorbox}[
    colback=lightpurple,
    colframe=darkpurple,
    boxrule=1pt,
    arc=2mm,
    left=10pt,
    right=10pt,
    top=8pt,
    bottom=8pt
]
\begin{tabular}{@{}rl@{}}
▣ \textbf{강의명:} & #1 \\[0.3em]
▣ \textbf{주차:} & #2 \\[0.3em]
▣ \textbf{교수명:} & #3 \\[0.3em]
▣ \textbf{목적:} & \begin{minipage}[t]{0.75\textwidth}#4\end{minipage}
\end{tabular}
\end{tcolorbox}
}

%========================================================================================
% 끝
%========================================================================================


\begin{document}

\metainfo{CSCI E-89B: 자연어 처리 입문}{Lecture 04}{Dmitry Kurochkin}{Lecture 04의 핵심 개념 학습}

\metainfo{CSCI E-89B: 자연어 처리 입문}{Lecture 04}{Dmitry Kurochkin}{Lecture 04의 핵심 개념 학습}

\metainfo{CSCI E-89B: 자연어 처리 입문}{Lecture 04}{Dmitry Kurochkin}{Lecture 04의 핵심 개념 학습}

\metainfo{CSCI E-89B: 자연어 처리 입문}{Lecture 04}{Dmitry Kurochkin}{Lecture 04의 핵심 개념 학습}

\metainfo{CSCI E-89B: 자연어 처리 입문}{Lecture 04}{Dmitry Kurochkin}{Lecture 04의 핵심 개념 학습}

% 개요 섹션
\begin{overviewbox}
\textbf{학습 목표:}
\begin{itemize}
    \item 이 강의의 핵심 개념을 이해합니다
    \item 실전에 적용할 수 있는 지식을 습득합니다
\end{itemize}

\textbf{주요 키워드:} [자동으로 채워질 예정]

\textbf{선행 지식:} 기본적인 컴퓨터 사용 능력
\end{overviewbox}



\begin{center}
    \huge\bfseries CSCI E-89B 자연어 처리 4주차 노트 \\[0.5em]
    \large Bag of Words, n-grams, and CNN
\end{center}

\tableofcontents

\newpage

%====================================================================
% 1. 개요
%====================================================================
\section{개요: 텍스트를 숫자로 바꾸는 여정}

\begin{summarybox}{핵심 요약}
이번 강의에서는 텍스트를 컴퓨터가 이해할 수 있는 숫자 벡터로 변환하는 방법을 다룹니다.
가장 기본적인 \textbf{Bag of Words}부터 시작하여, 단어의 순서를 일부 고려하는 \textbf{n-grams}를 배웁니다.
마지막으로, 이미지 처리에 주로 쓰이는 \textbf{Convolutional Neural Networks (CNN)}를 텍스트에 적용하는 원리를 탐구합니다.
이러한 기법들은 텍스트 분류, 감성 분석 등 다양한 자연어 처리(NLP) 문제의 기초가 됩니다.
궁극적으로는 텍스트의 의미적, 구조적 정보를 어떻게 효과적으로 포착할 것인가에 대한 고민이 담겨 있습니다.
\end{summarybox}

\begin{examplebox}{학습 로드맵}
\begin{enumerate}
    \item \textbf{기초 다지기}: Bag of Words (BoW)의 개념과 한계를 명확히 이해합니다.
    \item \textbf{문맥 추가하기}: n-grams가 BoW의 어떤 단점을 보완하는지 파악합니다.
    \item \textbf{고급 모델 맛보기}: 텍스트를 이미지처럼 다루는 CNN의 아이디어를 이해합니다.
    \item \textbf{실습으로 체득하기}: Python 라이브러리(\texttt{sklearn}, \texttt{NLTK}, \texttt{spaCy})를 사용해 BoW와 n-grams를 직접 구현해봅니다.
    \item \textbf{개념 연결하기}: BoW의 희소성(sparsity) 문제를 해결하기 위한 대안으로 임베딩(embedding)의 필요성을 인식합니다.
\end{enumerate}
\end{examplebox}


\newpage

%====================================================================
% 2. 용어 정리
%====================================================================
\section{핵심 용어 정리}

자주 사용되는 전문 용어를 미리 익혀두면 학습에 도움이 됩니다.

\begin{table}[h!]
\centering
\caption{4주차 핵심 용어}
\label{tab:terms}
\begin{tabular}{lp{5cm}p{3.5cm}p{3cm}}
\toprule
\textbf{용어} & \textbf{쉬운 설명} & \textbf{원어} & \textbf{비고} \\
\midrule
\textbf{Bag of Words} & 문장의 단어 순서를 무시하고, 단어의 출현 빈도수만 가방에 담듯이 세는 방법 & Bag of Words (BoW) & 간단하지만 문맥 정보를 잃어버림 \\
\textbf{토큰화} & 문장을 의미 있는 단위(토큰)로 쪼개는 과정 & Tokenization & 단어, 글자, 서브워드(subword) 등이 토큰이 될 수 있음 \\
\textbf{n-gram} & 텍스트에서 연속적으로 나타나는 n개의 단어 묶음 & n-gram & 2-gram(bigram), 3-gram(trigram) 등이 있음. 지역적 문맥 포착 \\
\textbf{어간 추출} & 단어에서 접사(prefix, suffix)를 제거하여 기본형(어간)을 찾는 과정 & Stemming & 빠르지만, 결과가 실제 단어가 아닐 수 있음 (예: octopi $\rightarrow$ octop) \\
\textbf{표제어 추출} & 단어의 사전적 기본형(표제어)을 찾는 과정 & Lemmatization & 문법적 품사를 고려하여 더 정확하지만, 어간 추출보다 느림 \\
\textbf{임베딩} & 단어를 의미를 담은 저차원의 조밀한(dense) 벡터로 표현하는 기법 & Embedding & 단어 간의 의미적 유사성을 벡터 공간의 거리로 표현 가능 \\
\textbf{CNN} & 이미지의 지역적 패턴을 추출하는 데 특화된 딥러닝 모델 & Convolutional Neural Network & 텍스트에 적용 시, n-gram처럼 지역적 단어 패턴을 학습 \\
\textbf{필터 (커널)} & CNN에서 특정 특징(예: 수직선, 특정 단어 조합)을 감지하는 가중치 행렬 & Filter (Kernel) & 필터를 입력 데이터에 슬라이딩하며 특징 맵(feature map)을 생성 \\
\textbf{패딩} & 필터 연산 시 출력 크기가 줄어드는 것을 막기 위해 입력 데이터 주변을 특정 값(주로 0)으로 채우는 것 & Padding & \texttt{VALID}(패딩 없음) vs \texttt{SAME}(출력 크기 유지) \\
\textbf{스트라이드} & 필터가 입력 데이터 위를 한 번에 이동하는 칸의 수 & Strides & 스트라이드가 크면 출력 크기가 더 많이 줄어듦 \\
\textbf{풀링} & 특징 맵의 크기를 줄여(down-sampling) 계산량을 감소시키고, 주요 특징을 강조하는 과정 & Pooling & Max Pooling은 특정 영역에서 가장 큰 값만 남김 \\
\bottomrule
\end{tabular}
\end{table}

\newpage

%====================================================================
% 3. 핵심 개념/원리
%====================================================================
\section{Bag of Words (BoW)}

\subsection{BoW의 정의와 직관}

\textbf{Bag of Words (BoW)}는 텍스트를 숫자 벡터로 표현하는 가장 직관적이고 기본적인 방법입니다.  이름 그대로, 단어들을 순서 없이 '가방'에 넣고 각 단어가 몇 번 등장했는지만 세는 방식입니다. 

\begin{summarybox}{핵심 아이디어}
문법이나 단어의 순서는 완전히 무시하고, 오직 문서 내 각 단어의 출현 빈도에만 집중합니다. 
"John likes to watch movies. Mary likes movies too." 라는 문장이 있다면, BoW는 단순히 'John:1, likes:2, to:1, watch:1, movies:2, Mary:1, too:1' 와 같이 빈도를 기록합니다.
\end{summarybox}

이 방법은 텍스트 분류, 감성 분석, 주제 모델링 등에서 간단하면서도 강력한 특징 벡터(feature vector)를 생성하는 데 사용됩니다. 

\subsection{BoW 생성 절차}

BoW 표현을 만드는 과정은 다음 3단계로 나눌 수 있습니다. 

\begin{enumerate}
    \item \textbf{1단계: 토큰화 (Tokenization)}
    
    분석할 전체 텍스트(말뭉치, Corpus)를 의미 있는 최소 단위인 \textbf{토큰(token)}으로 분리합니다.  보통 단어 단위로 쪼갭니다.
    
    \textbf{예시 문장}: "Henry Ford introduced the Model T. Ford Model T was revolutionary."
    
    \textbf{토큰화 결과}: \texttt{['Henry', 'Ford', 'introduced', 'the', 'Model', 'T', '.', 'Ford', 'Model', 'T', 'was', 'revolutionary']} 

    \item \textbf{2단계: 어휘 집합(Vocabulary) 생성}
    
    전체 말뭉치에서 등장한 모든 고유한 토큰들을 모아 하나의 집합, 즉 어휘 집합을 만듭니다.  이 어휘 집합이 벡터의 차원이자 기준이 됩니다. 보통 텍스트에 등장하는 순서대로 추가합니다. 
    
    \textbf{예시 어휘 집합}: \texttt{\{"Henry", "Ford", "introduced", "the", "Model", "T", "was", "revolutionary"\}} 
    
    \item \textbf{3단계: 벡터화 (Vectorization)}
    
    각 문장(또는 문서)을 어휘 집합을 기준으로 벡터로 변환합니다. 벡터의 각 차원은 어휘 집합의 특정 단어에 해당하며, 값은 해당 단어가 문장에서 나타난 빈도수입니다. 
    
    \textbf{예시 문장 벡터}: 어휘 집합 순서에 따라 빈도를 세면 다음과 같습니다.
    \begin{itemize}
        \item Henry: 1
        \item Ford: 2
        \item introduced: 1
        \item the: 1
        \item Model: 2
        \item T: 2
        \item was: 1
        \item revolutionary: 1
    \end{itemize}
    \textbf{최종 벡터}: \texttt{[1, 2, 1, 1, 2, 2, 1, 1]}
\end{enumerate}

\cautionbox{빈도수 vs 이진(Binary) 표현}
BoW 벡터를 만들 때, 단어의 빈도수를 그대로 사용할 수도 있고(\texttt{[1, 2, 1, ...]}) 단순히 단어의 출현 여부만 0과 1로 표시할 수도 있습니다. (\texttt{[1, 1, 1, ...]}) 후자를 이진 BoW라고 하며, 어떤 방식을 택할지는 문제의 특성에 따라 달라집니다. 

\subsection{BoW의 한계와 대안}

BoW는 간단하고 효과적이지만, 명확한 한계점을 가집니다.

\begin{itemize}
    \item \textbf{문맥 정보 상실}: 단어의 순서를 무시하므로, "not good"과 "good, not bad"를 비슷하게 취급할 수 있습니다. 의미적 뉘앙스를 파악하기 어렵습니다. 
    \item \textbf{희소성 (Sparsity) 문제}: 어휘 집합의 크기가 수만 개에 달하면, 각 문장 벡터는 대부분의 값이 0인 거대한 희소 벡터(sparse vector)가 됩니다. 이는 저장 공간과 계산 비효율을 초래합니다. 
    \item \textbf{의미 관계 표현 불가}: 'car'와 'automobile'이 의미적으로 유사하다는 점을 BoW는 전혀 반영하지 못합니다. 모든 단어를 독립적인 개체로 취급합니다. 
\end{itemize}

이러한 한계를 극복하기 위해 다음과 같은 대안들이 제시됩니다.

\begin{itemize}
    \item \textbf{n-grams}: 연속된 단어 묶음을 사용해 지역적인 문맥을 포착합니다.
    \item \textbf{단어 임베딩 (Word Embeddings)}: Word2Vec, GloVe 등 단어를 의미 공간의 조밀한 벡터로 표현하여 의미적 유사성을 학습합니다. 
    \item \textbf{순환 신경망 (RNNs)}: 순서 정보를 모델링하여 장기 의존성을 학습합니다. 
    \item \textbf{컨볼루션 신경망 (CNNs)}: 텍스트의 지역적 패턴(n-gram과 유사)을 효과적으로 포착합니다. 
\end{itemize}

\newpage

%====================================================================
% 4. n-grams
%====================================================================
\section{n-grams: 지역적 문맥 포착하기}

\subsection{n-gram의 정의와 종류}

\textbf{n-gram}은 텍스트에서 연속적으로 나타나는 n개의 아이템(주로 단어) 시퀀스를 의미합니다.  BoW가 잃어버리는 단어의 순서 정보를 일부 보완하여, 지역적인 문맥을 포착하는 데 도움을 줍니다. 

\begin{itemize}
    \item \textbf{1-gram (unigram)}: 단어 하나. BoW와 동일합니다. (예: 'this', 'is', 'a', 'movie') 
    \item \textbf{2-gram (bigram)}: 연속된 두 단어. (예: 'this is', 'is a', 'a movie') 
    \item \textbf{3-gram (trigram)}: 연속된 세 단어. (예: 'this is a', 'is a movie') 
\end{itemize}

\begin{examplebox}{n-gram의 효과}
감성 분석에서 "This movie was \textbf{not funny}"라는 문장을 생각해 봅시다.
\begin{itemize}
    \item \textbf{BoW (unigram)}: 'not'과 'funny'를 별개의 긍정/부정 단어로 취급하여 문장의 전체적인 부정적 뉘앙스를 놓칠 수 있습니다.
    \item \textbf{2-gram (bigram)}: '\textbf{not funny}'라는 토큰을 생성하여, 이 조합이 강한 부정적 의미를 지닌다는 것을 학습할 수 있습니다. 
\end{itemize}
따라서 2-gram이나 3-gram은 특히 감성 분석이나 기계 번역처럼 단어 조합이 중요한 작업에서 성능을 향상시킬 수 있습니다. 
\end{examplebox}

\subsection{n-gram의 한계}

n-gram 역시 완벽한 해결책은 아니며 다음과 같은 한계를 가집니다.

\begin{itemize}
    \item \textbf{데이터 희소성 (Sparsity)}: n이 커질수록 가능한 n-gram의 조합은 기하급수적으로 늘어납니다. 훈련 데이터에 등장하지 않은 n-gram이 많아져 희소성 문제가 BoW보다 심각해집니다. 
    \item \textbf{제한된 문맥 범위}: n-gram은 n개의 단어라는 고정된 창(window) 내의 지역적 문맥만 포착할 수 있습니다. 문장 전체에 걸친 장거리 의존성(long-range dependency)은 파악하기 어렵습니다. 
    \item \textbf{저장 및 계산 비용}: 어휘 집합의 크기가 매우 커져 많은 저장 공간과 계산 자원을 필요로 합니다. 
\end{itemize}

\newpage

%====================================================================
% 5. Convolutional Neural Networks (CNN)
%====================================================================
\section{Convolutional Neural Networks (CNN) for Text}

\subsection{CNN의 기본 아이디어}

\textbf{CNN}은 본래 이미지 처리를 위해 개발된 딥러닝 모델입니다.  이미지의 픽셀처럼, 텍스트의 단어들도 순서대로 배열되어 있다는 점에서 착안하여 NLP에 적용할 수 있습니다. 

\begin{summarybox}{핵심 아이디어: 텍스트를 1D 이미지로 보기}
문장을 단어 임베딩 벡터의 시퀀스로 변환하면, 이는 \texttt{(문장 길이) x (임베딩 차원)} 크기의 2D 행렬이 됩니다. 이 행렬을 흑백 이미지처럼 간주하고, CNN의 \textbf{필터(filter)}를 적용하여 지역적인 특징을 추출할 수 있습니다. 

여기서 CNN 필터는 마치 n-gram처럼, 연속된 몇 개의 단어(예: 2~3개)를 한 번에 훑으며 의미 있는 패턴을 감지하는 역할을 합니다. 
\end{summarybox}

% 이미지 설명을 텍스트로 대체
\begin{tcolorbox}[title={시각적 비유: 컨볼루션 연산}, colback=gray!5, colframe=gray!50]
    \textbf{컨볼루션(Convolution)}이란, 신호 처리에서 유래한 수학적 연산입니다. \texttt{f(s)}라는 신호가 있고, \texttt{h(x-s)}라는 감지기(필터)가 있다고 상상해 보세요. 감지기가 신호 위를 지나가면서 두 함수가 겹치는 영역의 값을 곱하고 모두 더하는 것이 컨볼루션입니다.
    
    CNN에서는 이미지(또는 텍스트 행렬)가 신호, 필터가 감지기 역할을 합니다. 필터가 이미지 위를 이동하면서, 필터의 가중치와 겹치는 이미지 픽셀 값을 곱해 더함으로써 특정 패턴(예: 수직선, 특정 단어 조합)이 얼마나 강하게 나타나는지를 측정합니다. 
\end{tcolorbox}


\subsection{CNN의 주요 구성 요소}

텍스트 처리를 위한 CNN은 주로 다음과 같은 레이어들로 구성됩니다.

\begin{enumerate}
    \item \textbf{임베딩 레이어 (Embedding Layer)}
    
    입력된 텍스트의 각 단어(정수 인덱스)를 고정된 크기의 조밀한 벡터로 변환합니다. 이는 CNN이 처리할 수 있는 입력 형태를 만듭니다.

    \item \textbf{컨볼루션 레이어 (Convolutional Layer)}
    
    이 레이어의 핵심은 \textbf{필터(Filter) 또는 커널(Kernel)}입니다. 필터는 작은 크기(예: 3x5, 3개의 단어, 5차원 임베딩)의 가중치 행렬입니다. 
    
    필터는 입력된 임베딩 행렬 위를 위에서 아래로 \textbf{스트라이드(stride)}만큼 이동하면서, 겹치는 부분과 요소별 곱(element-wise product)을 수행한 뒤 모두 더해 하나의 스칼라 값을 계산합니다. 
    
    이 과정을 통해 입력 데이터에서 필터가 감지하려는 특정 패턴(예: '매우 재미있는')이 얼마나 활성화되는지를 나타내는 \textbf{특징 맵(feature map)}이 생성됩니다.
    
    \begin{cautionbox}{가중치 공유 (Weight Sharing)}
    하나의 필터는 입력 데이터의 모든 위치에서 동일한 가중치를 사용합니다. 이를 가중치 공유라고 합니다.  덕분에 "not good"이라는 패턴이 문장 처음에 나오든 끝에 나오든 동일한 필터로 감지할 수 있으며, 학습할 파라미터 수를 크게 줄여줍니다. 
    \end{cautionbox}
    
    \item \textbf{풀링 레이어 (Pooling Layer)}
    
    컨볼루션 레이어를 통과한 특징 맵의 차원을 줄이는(down-sampling) 역할을 합니다. 
    
    \textbf{Max Pooling}이 가장 널리 쓰이며, 특징 맵을 일정 구역으로 나눈 뒤 각 구역에서 가장 큰 값(가장 활성화된 특징)만 남깁니다.  이를 통해 중요한 특징을 유지하면서 계산량을 줄이고, 특징의 위치 변화에 조금 더 강인한 모델을 만들 수 있습니다.

    \item \textbf{완전 연결 레이어 (Fully-Connected Layer) 및 출력}
    
    여러 컨볼루션과 풀링 레이어를 거쳐 추출된 특징들은 \textbf{Flatten} 레이어를 통해 1차원 벡터로 펼쳐집니다.  이 벡터가 일반적인 신경망(Dense Layer)에 입력되어 최종적으로 텍스트 분류나 감성 분석 등의 결과를 출력합니다.
\end{enumerate}

\begin{examplebox}{CNN 아키텍처 예시 (Python Keras 코드 기반)}
\begin{lstlisting}[caption={텍스트 분류를 위한 간단한 CNN 모델 구조}, label={lst:cnn_model}]
import tensorflow as tf
from tensorflow.keras.models import Sequential
from tensorflow.keras.layers import Embedding, Conv1D, GlobalMaxPooling1D, Dense

# 모델 정의
model = Sequential([
    # 1. 임베딩 레이어: 10000개의 단어를 128차원 벡터로
    Embedding(input_dim=10000, output_dim=128, input_length=50),
    
    # 2. 컨볼루션 레이어: 3개 단어 크기의 필터 64개 적용
    Conv1D(filters=64, kernel_size=3, activation='relu'),
    
    # 3. 풀링 레이어: 특징 맵 전체에서 가장 큰 값 하나를 선택
    GlobalMaxPooling1D(),
    
    # 4. 출력 레이어: 긍정/부정(1개 유닛) 분류를 위한 Dense 레이어
    Dense(1, activation='sigmoid')
])

model.summary()
\end{lstlisting}
\end{examplebox}

\newpage

%====================================================================
% 6. 실습/코드
%====================================================================
\section{Python을 이용한 실습}

\subsection{Scikit-learn을 이용한 BoW 및 n-gram 생성}

\texttt{sklearn}의 \texttt{CountVectorizer}는 BoW와 n-gram을 손쉽게 생성할 수 있는 강력한 도구입니다.

\begin{lstlisting}[caption={Scikit-learn으로 BoW 벡터 생성하기}, label={lst:sklearn_bow}]
from sklearn.feature_extraction.text import CountVectorizer

# 예시 말뭉치 (리스트의 각 요소가 하나의 문서)
corpus = [
    "Henry Ford introduced the Model T.",
    "Ford Model T was revolutionary."
]

# 1. BoW (unigram)
vectorizer_bow = CountVectorizer()
X_bow = vectorizer_bow.fit_transform(corpus)

print("BoW 어휘 집합:", vectorizer_bow.get_feature_names_out())
print("BoW 벡터:\n", X_bow.toarray())

# 2. 2-gram (bigram)
# ngram_range=(2, 2)는 bigram만 사용하겠다는 의미
vectorizer_ngram = CountVectorizer(ngram_range=(2, 2))
X_ngram = vectorizer_ngram.fit_transform(corpus)

print("\nn-gram 어휘 집합:", vectorizer_ngram.get_feature_names_out())
print("n-gram 벡터:\n", X_ngram.toarray())
\end{lstlisting}

\subsection{NLTK를 이용한 BoW 및 n-gram 생성}
\texttt{NLTK}는 더 세밀한 제어가 가능하며, 학술 연구에서 많이 사용됩니다. 

\begin{lstlisting}[caption={NLTK로 BoW 및 n-gram 생성하기}, label={lst:nltk_bow}]
import nltk
from nltk.tokenize import word_tokenize
from nltk.util import ngrams
from collections import Counter

# nltk.download('punkt') # 최초 실행 시 필요

text = "Henry Ford introduced the Model T was revolutionary."
tokens = word_tokenize(text)

# 1. BoW (unigram)
# NLTK에는 BoW를 위한 직접적인 클래스가 없으므로 수동으로 구현
vocab_bow = sorted(list(set(tokens)))
bow_vector = [tokens.count(word) for word in vocab_bow]
print("BoW 어휘 집합:", vocab_bow)
print("BoW 벡터:", bow_vector)

# 2. n-gram (bigram)
bigrams = list(ngrams(tokens, 2))
vocab_ngram = sorted(list(set(bigrams)))
bigram_counts = Counter(bigrams)
ngram_vector = [bigram_counts[bigram] for bigram in vocab_ngram]

print("\nn-gram (튜플 형태):", bigrams)
print("n-gram 벡터:", ngram_vector)
\end{lstlisting}

\subsection{spaCy를 이용한 BoW 생성}
\texttt{spaCy}는 상용 수준의 성능과 편의성을 제공하는 최신 라이브러리입니다. 

\begin{lstlisting}[caption={spaCy로 BoW 생성하기}, label={lst:spacy_bow}]
import spacy
from collections import Counter

# python -m spacy download en_core_web_sm # 최초 실행 시 모델 다운로드 필요
nlp = spacy.load("en_core_web_sm")

text = "Henry Ford introduced the Model T. Ford Model T was revolutionary."
doc = nlp(text)

# spaCy는 구두점 등을 자동으로 처리해줌
tokens = [token.text for token in doc if not token.is_punct]

vocab = sorted(list(set(tokens)))
token_counts = Counter(tokens)
bow_vector = [token_counts[word] for word in vocab]

print("어휘 집합:", vocab)
print("BoW 벡터:", bow_vector)
\end{lstlisting}

\cautionbox{신규 문서 처리 시 주의사항}
모델을 훈련시킬 때 사용한 어휘 집합(vocabulary)은 고정되어야 합니다.  새로운 테스트 문서에 어휘 집합에 없는 단어(Out-of-Vocabulary, OOV)가 나타나면, 이 단어들은 무시되거나 특별한 'UNK' (Unknown) 토큰으로 처리됩니다. 훈련 시에 어휘 집합을 만들고, 테스트 시에는 이 어휘 집합을 그대로 사용하여 변환(\texttt{transform})만 수행해야 합니다. 

\newpage

%====================================================================
% 7. FAQ
%====================================================================
\section{자주 묻는 질문 (FAQ)}

\begin{tcolorbox}[title={Q1: BoW와 단어 임베딩의 근본적인 차이는 무엇인가요?}]
\textbf{A}: 가장 큰 차이는 \textbf{표현 방식}과 \textbf{정보 보존}에 있습니다.
\begin{itemize}
    \item \textbf{BoW}: 각 단어를 거대한 어휘 집합 크기의 벡터에서 하나의 차원으로 표현하는 \textbf{희소(sparse) 벡터}입니다. 단어 간 의미 관계나 순서 정보가 없습니다.
    \item \textbf{임베딩}: 각 단어를 저차원(예: 100~300차원)의 \textbf{조밀한(dense) 벡터}로 표현합니다. 벡터 공간에서 비슷한 단어는 가깝게 위치하도록 학습되어 \textbf{의미 관계}를 내포하며, 모델에 따라 \textbf{순서 정보}를 활용할 수 있습니다. 
\end{itemize}
\end{tcolorbox}

\begin{tcolorbox}[title={Q2: 어간 추출(Stemming)과 표제어 추출(Lemmatization) 중 무엇을 써야 하나요?}]
\textbf{A}: 목적과 자원에 따라 다릅니다.
\begin{itemize}
    \item \textbf{어간 추출}: 속도가 매우 빠르지만, 결과물이 사전에 없는 단어일 수 있습니다. (예: 'studies' $\rightarrow$ 'studi') 정보 검색처럼 속도가 중요하고 단어의 정확한 형태보다 핵심 의미만 필요할 때 유용합니다. 
    \item \textbf{표제어 추출}: 품사 등 문법적 정보를 고려하여 사전적 원형을 찾아주므로 더 정확합니다. (예: 'studies' $\rightarrow$ 'study') 챗봇이나 기계 번역처럼 생성되는 단어의 문법적 정확성이 중요할 때 사용됩니다. 계산 비용은 더 높습니다. 
\end{itemize}
\end{tcolorbox}

\begin{tcolorbox}[title={Q3: CNN에서 필터 크기는 어떻게 정하나요?}]
\textbf{A}: 경험적으로 정해지며, 문제에 따라 여러 크기를 함께 사용하기도 합니다. 텍스트 처리에서는 보통 2, 3, 4, 5개 단어에 해당하는 필터 크기를 많이 사용합니다. 이는 각각 bigram, trigram, 4-gram, 5-gram과 유사한 패턴을 감지하는 효과를 냅니다. 하나의 컨볼루션 레이어에 다양한 크기의 필터를 병렬로 적용하여 여러 수준의 지역적 패턴을 동시에 학습하는 구조도 널리 쓰입니다. (예: Inception 모델) 
\end{tcolorbox}

\begin{tcolorbox}[title={Q4: 불용어(Stop words)는 항상 제거해야 하나요?}]
\textbf{A}: 꼭 그렇지는 않습니다. 'the', 'a', 'is'와 같은 불용어는 일반적으로 문장의 핵심 의미에 큰 영향을 주지 않아, BoW 모델의 차원을 줄이고 효율성을 높이기 위해 제거하는 경우가 많습니다.  하지만, 문맥이 중요한 최신 딥러닝 모델(예: BERT)에서는 불용어도 문법적 구조를 파악하는 데 중요한 단서가 될 수 있으므로 제거하지 않는 것이 일반적입니다. 
\end{tcolorbox}


\newpage

%====================================================================
% 8. 빠르게 훑어보기 (1페이지 요약)
%====================================================================
\section{빠르게 훑어보기 (1페이지 요약)}

\begin{summarybox}{Bag of Words (BoW)}
\begin{itemize}
    \item \textbf{개념}: 단어 순서 무시, 출현 빈도만 세는 텍스트 표현.
    \item \textbf{장점}: 구현이 간단하고, 많은 경우 준수한 성능을 보임.
    \item \textbf{단점}: 문맥 정보 상실, 희소성 문제, 유의어 처리 불가.
    \item \textbf{프로세스}: 토큰화 $\rightarrow$ 어휘 집합 구축 $\rightarrow$ 빈도수 기반 벡터화.
\end{itemize}
\end{summarybox}

\begin{summarybox}{n-grams}
\begin{itemize}
    \item \textbf{개념}: 연속된 n개의 단어 묶음. (예: 2-gram \texttt{not good})
    \item \textbf{장점}: BoW가 놓치는 지역적 문맥과 단어 순서를 일부 포착.
    \item \textbf{단점}: n이 커지면 희소성 문제 심화, 장거리 의존성 포착 불가.
    \item \textbf{활용}: 감성 분석, 텍스트 예측 등에서 unigram의 한계 보완.
\end{itemize}
\end{summarybox}

\begin{summarybox}{Convolutional Neural Networks (CNN) for Text}
\begin{itemize}
    \item \textbf{개념}: 텍스트를 이미지처럼 취급, 필터를 이용해 지역적 패턴 추출.
    \item \textbf{핵심 요소}:
    \begin{itemize}
        \item \textbf{Filter}: n-gram처럼 여러 단어에 걸친 패턴 감지기.
        \item \textbf{Pooling}: 중요한 특징을 강조하며 데이터 차원 축소.
        \item \textbf{Weight Sharing}: 파라미터 수를 줄이고 위치 불변성 확보.
    \end{itemize}
    \item \textbf{장점}: n-gram과 유사한 패턴을 계층적으로 학습 가능, 효율적.
    \item \textbf{프로세스}: 임베딩 $\rightarrow$ 컨볼루션 + 풀링 (반복) $\rightarrow$ 분류기.
\end{itemize}
\end{summarybox}

\begin{cautionbox}{학습 체크리스트}
\begin{itemize}
    \item[\_] BoW가 문맥 정보를 잃는다는 것의 의미를 설명할 수 있는가?
    \item[\_] "not good" 예시를 통해 n-gram의 필요성을 설명할 수 있는가?
    \item[\_] CNN의 필터가 텍스트에서 어떤 역할을 하는지 비유적으로 설명할 수 있는가?
    \item[\_] \texttt{CountVectorizer}의 \texttt{ngram_range} 파라미터를 사용하여 원하는 n-gram을 생성할 수 있는가?
    \item[\_] 희소 벡터(sparse vector)가 왜 문제가 되며, 임베딩이 어떻게 이를 해결하는지 이해했는가?
\end{itemize}
\end{cautionbox}

\end{document}
