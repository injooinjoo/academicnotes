%%%%%%%%%%%%%%%%%%%%%%%%%%%%%%%%%%%%%%%%%%%%%%%%%%%%%%%%%%%%%%%%%%%%%%%%%%%%%%%
% Harvard Academic Notes - 통합 마스터 템플릿
% 모든 강의 노트에 적용되는 통일된 스타일
% 버전: 2.1 - 가독성 개선 (선택적 최적화)
% 최종 수정일: 2025-11-17
%%%%%%%%%%%%%%%%%%%%%%%%%%%%%%%%%%%%%%%%%%%%%%%%%%%%%%%%%%%%%%%%%%%%%%%%%%%%%%%

\documentclass[11pt,a4paper]{article}

%========================================================================================
% 기본 패키지
%========================================================================================

% --- 한국어 지원 ---
\usepackage{kotex}

% --- 페이지 레이아웃 ---
\usepackage[top=20mm, bottom=20mm, left=20mm, right=18mm]{geometry}
\usepackage{setspace}
\onehalfspacing                      % 1.5배 줄간격
\setlength{\parskip}{0.5em}          % 문단 간격
\setlength{\parindent}{0pt}          % 들여쓰기 없음

% --- 표 관련 ---
\usepackage{booktabs}              % 고품질 표
\usepackage{tabularx}              % 자동 너비 조절 표
\usepackage{array}                 % 표 컬럼 확장
\usepackage{longtable}             % 여러 페이지 표
\renewcommand{\arraystretch}{1.1}  % 표 행간 조절

%========================================================================================
% 헤더 및 푸터
%========================================================================================

\usepackage{fancyhdr}
\pagestyle{fancy}
\fancyhf{}
\fancyhead[L]{\small\textit{CS109A: 데이터 과학 입문}}
\fancyhead[R]{\small\textit{Lecture 23}}
\fancyfoot[C]{\thepage}
\renewcommand{\headrulewidth}{0.5pt}
\renewcommand{\footrulewidth}{0.3pt}

% 첫 페이지는 헤더 없음
\fancypagestyle{firstpage}{
    \fancyhf{}
    \fancyfoot[C]{\thepage}
    \renewcommand{\headrulewidth}{0pt}
}

%========================================================================================
% 색상 정의 (파스텔 톤 + 다크모드 호환)
%========================================================================================

\usepackage[dvipsnames]{xcolor}

% 밝은 배경용 파스텔 색상
\definecolor{lightblue}{RGB}{220, 235, 255}      % 부드러운 파랑
\definecolor{lightgreen}{RGB}{220, 255, 235}     % 부드러운 초록
\definecolor{lightyellow}{RGB}{255, 250, 220}    % 부드러운 노랑
\definecolor{lightpurple}{RGB}{240, 230, 255}    % 부드러운 보라
\definecolor{lightgray}{gray}{0.95}              % 밝은 회색
\definecolor{lightpink}{RGB}{255, 235, 245}      % 부드러운 핑크
\definecolor{boxgray}{gray}{0.95}
\definecolor{boxblue}{rgb}{0.9, 0.95, 1.0}
\definecolor{boxred}{rgb}{1.0, 0.95, 0.95}

% 진한 색상 (테두리/제목용)
\definecolor{darkblue}{RGB}{50, 80, 150}
\definecolor{darkgreen}{RGB}{40, 120, 70}
\definecolor{darkorange}{RGB}{200, 100, 30}
\definecolor{darkpurple}{RGB}{100, 60, 150}

%========================================================================================
% 박스 환경 (tcolorbox) - 6가지 타입
%========================================================================================

\usepackage[most]{tcolorbox}
\tcbuselibrary{skins, breakable}

% 1. 개요 박스 (강의 시작 부분)
\newtcolorbox{overviewbox}[1][]{
    enhanced,
    colback=lightpurple,
    colframe=darkpurple,
    fonttitle=\bfseries\large,
    title=📚 강의 개요,
    arc=3mm,
    boxrule=1pt,
    left=8pt,
    right=8pt,
    top=8pt,
    bottom=8pt,
    breakable,
    #1
}

% 2. 요약 박스
\newtcolorbox{summarybox}[1][]{
    enhanced,
    colback=lightblue,
    colframe=darkblue,
    fonttitle=\bfseries,
    title=📝 핵심 요약,
    arc=2mm,
    boxrule=0.7pt,
    left=6pt,
    right=6pt,
    top=6pt,
    bottom=6pt,
    breakable,
    #1
}

% 3. 핵심 정보 박스
\newtcolorbox{infobox}[1][]{
    enhanced,
    colback=lightgreen,
    colframe=darkgreen,
    fonttitle=\bfseries,
    title=💡 핵심 정보,
    arc=2mm,
    boxrule=0.7pt,
    left=6pt,
    right=6pt,
    top=6pt,
    bottom=6pt,
    breakable,
    #1
}

% 4. 주의사항 박스
\newtcolorbox{warningbox}[1][]{
    enhanced,
    colback=lightyellow,
    colframe=darkorange,
    fonttitle=\bfseries,
    title=⚠️ 주의사항,
    arc=2mm,
    boxrule=0.7pt,
    left=6pt,
    right=6pt,
    top=6pt,
    bottom=6pt,
    breakable,
    #1
}

% 5. 예제 박스
\newtcolorbox{examplebox}[1][]{
    enhanced,
    colback=lightgray,
    colframe=black!60,
    fonttitle=\bfseries,
    title=📖 예제: #1,
    arc=2mm,
    boxrule=0.7pt,
    left=6pt,
    right=6pt,
    top=6pt,
    bottom=6pt,
    breakable,
}

% 6. 정의 박스
\newtcolorbox{definitionbox}[1][]{
    enhanced,
    colback=lightpink,
    colframe=purple!70!black,
    fonttitle=\bfseries,
    title=📌 정의: #1,
    arc=2mm,
    boxrule=0.7pt,
    left=6pt,
    right=6pt,
    top=6pt,
    bottom=6pt,
    breakable,
}

% 7. 중요 박스 (importantbox - warningbox와 유사)
\newtcolorbox{importantbox}[1][]{
    enhanced,
    colback=boxred,
    colframe=red!70!black,
    fonttitle=\bfseries,
    title=⚠️ 매우 중요: #1,
    arc=2mm,
    boxrule=0.7pt,
    left=6pt,
    right=6pt,
    top=6pt,
    bottom=6pt,
    breakable,
}

% 8. cautionbox (warningbox와 동일)
\let\cautionbox\warningbox
\let\endcautionbox\endwarningbox

%========================================================================================
% 코드 블록 설정 (밝은 배경)
%========================================================================================

\usepackage{listings}

\definecolor{codegray}{rgb}{0.5,0.5,0.5}
\definecolor{codepurple}{rgb}{0.58,0,0.82}
\definecolor{backcolour}{rgb}{0.95,0.95,0.95}

\lstset{
    basicstyle=\ttfamily\small,
    backgroundcolor=\color{lightgray},
    keywordstyle=\color{darkblue}\bfseries,
    commentstyle=\color{darkgreen}\itshape,
    stringstyle=\color{purple!80!black},
    numberstyle=\tiny\color{black!60},
    numbers=left,
    numbersep=8pt,
    breaklines=true,
    breakatwhitespace=false,
    frame=single,
    frameround=tttt,
    rulecolor=\color{black!30},
    captionpos=b,
    showstringspaces=false,
    tabsize=2,
    xleftmargin=15pt,
    xrightmargin=5pt,
    escapeinside={\%*}{*)}
}

% Python 코드 스타일
\lstdefinestyle{pythonstyle}{
    language=Python,
    morekeywords={self, True, False, None},
}

% SQL 코드 스타일
\lstdefinestyle{sqlstyle}{
    language=SQL,
    morekeywords={SELECT, FROM, WHERE, JOIN, GROUP, BY, ORDER, HAVING},
}

%========================================================================================
% 목차 스타일링
%========================================================================================

\usepackage{tocloft}
\renewcommand{\cftsecleader}{\cftdotfill{\cftdotsep}}
\setlength{\cftbeforesecskip}{0.4em}
\renewcommand{\cftsecfont}{\bfseries}
\renewcommand{\cftsubsecfont}{\normalfont}

%========================================================================================
% 표 및 그림
%========================================================================================

\usepackage{graphicx}              % 이미지
\usepackage{adjustbox}             % 표/박스 크기 조절

% 표 캡션 스타일
\usepackage{caption}
\captionsetup[table]{
    labelfont=bf,
    textfont=it,
    skip=5pt
}
\captionsetup[figure]{
    labelfont=bf,
    textfont=it,
    skip=5pt
}

%========================================================================================
% 수학
%========================================================================================

\usepackage{amsmath, amssymb, amsthm}

% 정리 환경
\theoremstyle{definition}
\newtheorem{theorem}{정리}[section]
\newtheorem{lemma}[theorem]{보조정리}
\newtheorem{proposition}[theorem]{명제}
\newtheorem{corollary}[theorem]{따름정리}
\newtheorem{definition}{정의}[section]
\newtheorem{example}{예제}[section]

%========================================================================================
% 하이퍼링크
%========================================================================================

\usepackage[
    colorlinks=true,
    linkcolor=blue!80!black,
    urlcolor=blue!80!black,
    citecolor=green!60!black,
    bookmarks=true,
    bookmarksnumbered=true,
    pdfborder={0 0 0}
]{hyperref}

% PDF 메타데이터는 각 문서에서 설정
\hypersetup{
    pdftitle={CS109A: 데이터 과학 입문 - Lecture 23},
    pdfauthor={강의 노트},
    pdfsubject={Academic Notes}
}

%========================================================================================
% 기타 유용한 패키지
%========================================================================================

\usepackage{enumitem}              % 리스트 커스터마이징
\setlist{nosep, leftmargin=*, itemsep=0.3em}

\usepackage{microtype}             % 타이포그래피 개선
\usepackage{footnote}              % 각주 개선
\usepackage{url}                   % URL 줄바꿈
\urlstyle{same}

%========================================================================================
% 사용자 정의 명령어
%========================================================================================

% 강조 텍스트
\newcommand{\important}[1]{\textbf{\textcolor{red!70!black}{#1}}}
\newcommand{\keyword}[1]{\textbf{#1}}
\newcommand{\term}[1]{\textit{#1}}
\newcommand{\code}[1]{\texttt{#1}}

% 용어 설명 (인라인)
\newcommand{\defterm}[2]{\textbf{#1}\footnote{#2}}

% 섹션 시작 전 페이지 분리
\newcommand{\newsection}[1]{\newpage\section{#1}}

%========================================================================================
% 문서 제목 스타일
%========================================================================================

\usepackage{titling}
\pretitle{\begin{center}\LARGE\bfseries}
\posttitle{\par\end{center}\vskip 0.5em}
\preauthor{\begin{center}\large}
\postauthor{\end{center}}
\predate{\begin{center}\large}
\postdate{\par\end{center}}

%========================================================================================
% 섹션 제목 간격
%========================================================================================

\usepackage{titlesec}
\titlespacing*{\section}{0pt}{1.5em}{0.8em}
\titlespacing*{\subsection}{0pt}{1.2em}{0.6em}
\titlespacing*{\subsubsection}{0pt}{1em}{0.5em}

%========================================================================================
% 메타 정보 박스 명령어
%========================================================================================

\newcommand{\metainfo}[4]{
\begin{tcolorbox}[
    colback=lightpurple,
    colframe=darkpurple,
    boxrule=1pt,
    arc=2mm,
    left=10pt,
    right=10pt,
    top=8pt,
    bottom=8pt
]
\begin{tabular}{@{}rl@{}}
▣ \textbf{강의명:} & #1 \\[0.3em]
▣ \textbf{주차:} & #2 \\[0.3em]
▣ \textbf{교수명:} & #3 \\[0.3em]
▣ \textbf{목적:} & \begin{minipage}[t]{0.75\textwidth}#4\end{minipage}
\end{tabular}
\end{tcolorbox}
}

%========================================================================================
% 끝
%========================================================================================


\begin{document}

\metainfo{CS109A: 데이터 과학 입문}{Lecture 23}{Pavlos Protopapas, Kevin Rader, Chris Gumb}{Lecture 23의 핵심 개념 학습}

\tableofcontents
\newpage


% 제목 섹션
\begin{center}
    \LARGE \textbf{부스팅(Boosting)과 그레디언트 부스팅(Gradient Boosting)} \\
    \large \textit{약한 모델들을 모아 강력한 모델을 만드는 앙상블 기법 완전 정복}
\end{center}

\vspace{0.5cm}

% ------------------------------------------------------------------------
% 0. 개요 (Abstract)
% ------------------------------------------------------------------------
\begin{summarybox}{문서 핵심 요약}
이 문서는 머신러닝의 강력한 앙상블 기법인 '부스팅(Boosting)'의 개념부터 '그레디언트 부스팅(Gradient Boosting)'의 수학적 원리까지 다룹니다.
\begin{itemize}
    \item \textbf{핵심 아이디어:} 단순한 모델(Weak Learner)을 순차적으로 연결하여, 앞선 모델의 실수를 뒤따르는 모델이 바로잡습니다.
    \item \textbf{목표:} 편향(Bias)을 줄여 예측 성능을 극대화합니다. (랜덤 포레스트가 분산(Variance)을 줄이는 것과 대조적)
    \item \textbf{원리:} 잔차(Residual)를 새로운 모델의 목표값(Target)으로 학습하며, 이는 수학적으로 경사 하강법(Gradient Descent)과 동일합니다.
    \item \textbf{실전:} 학습률(Learning Rate) 조정을 통해 과적합(Overfitting)을 방지하는 것이 중요합니다.
\end{itemize}
\end{summarybox}

\vspace{1cm}

% ------------------------------------------------------------------------
% 1. 용어 정리
% ------------------------------------------------------------------------
\section{필수 용어 정리}
본격적인 학습에 앞서, 아래 용어들을 먼저 숙지하면 이해가 훨씬 수월합니다.

\begin{table}[h]
\centering
\begin{adjustbox}{width=\textwidth}
\begin{tabular}{l l l l}
\toprule
\textbf{용어 (한글)} & \textbf{원어 (English)} & \textbf{쉬운 설명} & \textbf{비고} \\ 
\midrule
약한 학습기 & Weak Learner & 성능이 턱걸이 수준인 단순한 모델 (예: 깊이가 1인 트리) & 부스팅의 재료 \\ 
\midrule
강한 학습기 & Strong Learner & 약한 학습기를 모아 만든 고성능 모델 & 최종 결과물 \\ 
\midrule
잔차 & Residual & 정답과 내 예측값의 차이 ($y - \hat{y}$), 즉 '오차' & 다음 모델의 목표 \\ 
\midrule
앙상블 & Ensemble & 여러 모델을 조합하여 성능을 높이는 기법 & 협동 작업 \\ 
\midrule
스텀프 & Stump & 가지가 딱 한 번만 뻗은 아주 얕은 결정 트리 & 깊이(Depth)=1 \\ 
\midrule
학습률 & Learning Rate & 모델을 업데이트할 때 얼마나 반영할지 결정하는 보폭 ($\lambda$) & 과적합 방지용 \\ 
\bottomrule
\end{tabular}
\end{adjustbox}
\caption{부스팅 학습을 위한 필수 용어 사전}
\label{tab:terms}
\end{table}

\newpage

% ------------------------------------------------------------------------
% 2. 부스팅(Boosting)의 기초와 직관
% ------------------------------------------------------------------------
\section{부스팅(Boosting)이란 무엇인가?}

\subsection{왜 필요한가? (배경)}
우리는 이전에 **랜덤 포레스트(Random Forest)**와 **배깅(Bagging)**을 배웠습니다. 이들은 깊고 복잡한 트리(Overfitting 되기 쉬움)를 여러 개 만들어 평균을 냄으로써 **분산(Variance)**을 줄이는 전략을 썼습니다.

하지만 **부스팅**은 정반대의 접근을 취합니다.
\begin{itemize}
    \item 아주 단순하고 멍청한 모델(High Bias, Low Variance)에서 시작합니다.
    \item 이 모델의 **실수(편향, Bias)**를 줄이는 방향으로 모델을 계속 추가합니다.
    \item 결과적으로 **편향을 획기적으로 줄여** 강력한 예측 모델을 만듭니다.
\end{itemize}

\begin{tipbox}{직관적 비유: 어려운 시험 통과하기}
여러분이 통과하기 매우 어려운 시험을 앞두고 있다고 상상해 봅시다. 한 명의 천재가 문제를 다 푸는 것이 아니라, 평범한 학생들의 지혜를 모으는 방식입니다.

\begin{enumerate}
    \item \textbf{학생 1 (단순한 규칙):} "일단 기출문제를 보니, 4번은 정답이 아니야." $\rightarrow$ 정확도 60\%
    \item \textbf{학생 2 (실수 보완):} "학생 1이 틀린 문제들을 보니, 보기에 '과적합'이라는 단어가 있으면 그게 정답이더라." $\rightarrow$ 정확도 65\%
    \item \textbf{학생 3 (추가 보완):} "학생 1, 2가 틀린 걸 보니, '교차 검증'이 답인 경우가 많아." $\rightarrow$ 정확도 70\%
\end{enumerate}

이 세 학생의 의견을 적절히 합치면(가중치 부여), 혼자서는 풀 수 없던 문제를 90점 이상으로 통과할 수 있습니다. 이것이 바로 **부스팅**입니다.
\end{tipbox}

\subsection{부스팅의 핵심 메커니즘}
부스팅은 **순차적(Iterative)**이고 **덧셈적(Additive)**인 과정입니다.

\[
T_{final}(x) = \sum_{h \in H} \lambda_h T_h(x)
\]

\begin{itemize}
    \item $T_h$: 약한 학습기 (개별 학생의 의견)
    \item $\lambda_h$: 가중치 (그 학생의 의견을 얼마나 신뢰할 것인가)
    \item $T_{final}$: 최종 모델 (강한 학습기)
\end{itemize}

**중요한 점:** 한 번에 모든 모델을 만드는 것이 아니라, **앞선 모델이 저지른 실수를 다음 모델이 해결**하도록 순서대로 만듭니다.

\newpage

% ------------------------------------------------------------------------
% 3. 그레디언트 부스팅(Gradient Boosting)
% ------------------------------------------------------------------------
\section{그레디언트 부스팅 (Gradient Boosting)}

부스팅 중에서도 가장 널리 쓰이고 강력한 방법이 **그레디언트 부스팅**입니다. 이름이 어렵게 들리지만, 원리는 간단합니다. **"이전 모델의 오차(잔차)를 새로운 모델이 학습한다"**는 것입니다.

\subsection{핵심 원리: 잔차(Residual) 학습하기}

\begin{tipbox}{비유: 골프 퍼팅}
목표 지점(홀컵)까지 공을 보내야 합니다.
\begin{enumerate}
    \item \textbf{첫 번째 샷 ($T_0$):} 공을 쳤는데 홀컵까지 10m가 남았습니다. (잔차 = 10m)
    \item \textbf{두 번째 샷 ($T_1$):} 이제 목표는 홀컵이 아니라, '남은 거리 10m'를 보내는 것입니다. 쳤는데 2m가 남았습니다. (잔차 = 2m)
    \item \textbf{세 번째 샷 ($T_2$):} 이제 목표는 '남은 거리 2m'입니다.
\end{enumerate}
이렇게 계속 남은 거리(잔차)를 메우는 샷을 더해나가는 것이 그레디언트 부스팅입니다.
\end{tipbox}

\subsection{동작 알고리즘 (단계별 설명)}

데이터가 입력값 $x$와 정답 $y$로 구성되어 있다고 합시다.

\textbf{Step 1: 아주 단순한 첫 번째 모델($T_0$)을 만듭니다.}
\begin{itemize}
    \item 예: 그냥 전체 데이터의 평균값으로 예측합니다. 당연히 많이 틀립니다.
\end{itemize}

\textbf{Step 2: 잔차(Residual)를 계산합니다.}
\begin{itemize}
    \item $r_0 = y - T_0(x)$ 
    \item 즉, (실제 정답) - (첫 번째 모델의 예측값) = (아직 맞추지 못한 남은 오차)입니다.
\end{itemize}

\textbf{Step 3: 잔차를 예측하는 두 번째 모델($T_1$)을 만듭니다.}
\begin{itemize}
    \item 이번에는 $y$를 맞추는 게 아니라, $r_0$(오차)를 맞추도록 학습합니다.
    \item 즉, "얼마나 더 더해야 정답에 가까워지는가?"를 배웁니다.
\end{itemize}

\textbf{Step 4: 모델을 업데이트합니다.}
\begin{itemize}
    \item $T_{new} = T_0 + \lambda \times T_1$
    \item 여기서 $\lambda$(람다)는 **학습률(Learning Rate)**입니다. 두 번째 모델의 의견을 100\% 반영하지 않고, 조금만 반영하여 과적합을 막습니다.
\end{itemize}

\textbf{Step 5: 반복합니다.}
\begin{itemize}
    \item 다시 새로운 잔차를 계산하고, 그 잔차를 맞추는 모델 $T_2, T_3, \dots$를 계속 더해 나갑니다.
\end{itemize}

\begin{alertbox}{주의: 실제로는 거대한 나무 하나를 만드는 게 아니다}
"트리를 합친다"고 해서 실제로 노드와 가지를 물리적으로 합쳐서 거대한 트리 하나를 만드는 것이 아닙니다.
추론(예측) 시에는 $T_0$의 출력값, $T_1$의 출력값, ... $T_n$의 출력값을 모두 계산한 뒤 **숫자들을 더해서** 최종 값을 냅니다.
\end{alertbox}

\newpage

% ------------------------------------------------------------------------
% 4. 시각적 예시 (Pseudo-Code & Flow)
% ------------------------------------------------------------------------
\section{시각적 예시와 흐름}

그레디언트 부스팅이 데이터를 어떻게 학습하는지 1차원 데이터 예시로 살펴봅니다.

\subsection{데이터 상황}
\begin{itemize}
    \item $x$: 입력 변수 (0~8 사이의 값)
    \item $y$: 출력 변수 (구불구불한 곡선 형태의 데이터)
\end{itemize}

\subsection{학습 과정 시각화}

\begin{table}[h]
\centering
\begin{adjustbox}{width=\textwidth}
\begin{tabular}{c l l}
\toprule
\textbf{단계} & \textbf{모델의 행동} & \textbf{결과 해석} \\ 
\midrule
\textbf{Round 1} & \begin{tabular}[c]{@{}l@{}}단순한 스텀프(Stump) 하나로\\ 전체 평균을 예측 ($T_0$)\end{tabular} & \begin{tabular}[c]{@{}l@{}}데이터의 복잡한 패턴을 전혀 못 잡음.\\ 잔차(오차)가 매우 큼.\end{tabular} \\ 
\midrule
\textbf{Round 2} & \begin{tabular}[c]{@{}l@{}}Round 1의 잔차(실제값 - 예측값)를\\ 목표로 하는 새 스텀프 학습 ($T_1$)\end{tabular} & \begin{tabular}[c]{@{}l@{}}큰 오차가 있던 부분을 보정함.\\ 전체 모델 모양이 데이터에 약간 가까워짐.\end{tabular} \\ 
\midrule
\textbf{Round 3} & \begin{tabular}[c]{@{}l@{}}현재까지 모델($T_0 + \lambda T_1$)의\\ 잔차를 다시 계산해 $T_2$ 학습\end{tabular} & \begin{tabular}[c]{@{}l@{}}세밀한 굴곡을 맞추기 시작함.\\ 오차가 점점 0에 가까워짐.\end{tabular} \\ 
\midrule
\textbf{...} & \textbf{반복 (Iteration)} & \textbf{점점 정밀한 모델 완성} \\ 
\bottomrule
\end{tabular}
\end{adjustbox}
\caption{그레디언트 부스팅의 단계별 학습 흐름}
\label{tab:gb_flow}
\end{table}

\subsection{수식 요약}
\[
\text{최종 예측 } \hat{y} = T_0(x) + \lambda T_1(x) + \lambda T_2(x) + \dots + \lambda T_N(x)
\]
\begin{itemize}
    \item 각 $T_i$는 이전 단계까지 해결하지 못한 '나머지 오차'를 해결하는 전문가입니다.
    \item $\lambda$(학습률)가 작을수록, 더 많은 트리가 필요하지만 더 정교하고 안정적인 모델이 됩니다.
\end{itemize}

\newpage

% ------------------------------------------------------------------------
% 5. 왜 '그레디언트'인가? (수학적 원리)
% ------------------------------------------------------------------------
\section{심화: 왜 '그레디언트(Gradient)' 부스팅인가?}

"잔차를 학습하는 것"이 왜 "경사 하강법(Gradient Descent)"과 같은지 이해하면, 이 알고리즘의 본질을 깨닫게 됩니다.

\subsection{경사 하강법 (Gradient Descent) 복습}
우리가 산 정상에서 가장 낮은 계곡(손실 함수 $L$의 최소값)으로 내려가려 합니다.
\begin{itemize}
    \item 눈을 가리고 있다면, 발끝으로 경사를 느낍니다.
    \item 경사가 가장 가파르게 올라가는 방향(Gradient, $\nabla L$)의 **반대 방향**으로 발을 내디뎌야 내려갈 수 있습니다.
    \item 공식: $w_{new} = w_{old} - \lambda \times \nabla L$ (파라미터 업데이트)
\end{itemize}

\subsection{함수 공간(Function Space)에서의 하강}
일반적인 머신러닝은 파라미터($w$, 가중치)를 수정합니다. 하지만 부스팅은 **함수(모델의 예측값 $\hat{y}$) 자체**를 수정하여 정답에 다가갑니다.

우리가 최소화하고 싶은 손실 함수가 평균 제곱 오차(MSE)라고 가정해 봅시다.
\[
L(y, \hat{y}) = \frac{1}{2}(y - \hat{y})^2
\]

이 손실 함수를 예측값 $\hat{y}$에 대해 미분해 봅니다(경사를 구합니다).
\[
\frac{\partial L}{\partial \hat{y}} = -(y - \hat{y})
\]

어라? 결과가 익숙합니다.
\begin{itemize}
    \item $y - \hat{y}$는 바로 **잔차(Residual)**입니다.
    \item 즉, **-(잔차)**가 곧 **기울기(Gradient)**입니다.
    \item 경사 하강법은 기울기의 '반대 방향'으로 가는 것이므로, **-(-잔차) = 잔차 방향**으로 이동하면 됩니다.
\end{itemize}

\begin{summarybox}{핵심 결론}
\textbf{"잔차를 더해주는 것"}은 수학적으로 \textbf{"손실 함수의 기울기(Gradient) 반대 방향으로 이동하여 에러를 줄이는 것"}과 완벽하게 동일합니다.
따라서 이 방법을 \textbf{그레디언트 부스팅}이라고 부릅니다.
\end{summarybox}

\newpage

% ------------------------------------------------------------------------
% 6. 실전: 하이퍼파라미터와 튜닝
% ------------------------------------------------------------------------
\section{실전 가이드: 모델 튜닝하기}

그레디언트 부스팅을 실제로 사용할 때 가장 중요한 두 가지 설정(Hyperparameter)이 있습니다.

\subsection{1. 학습률 (Learning Rate, $\lambda$)}
\begin{itemize}
    \item **정의:** 새로 추가되는 트리의 의견을 얼마나 반영할지 결정하는 비율 (보통 0.01 ~ 0.1 사이).
    \item **$\lambda$가 너무 클 때:** 학습이 빠르지만, 최적점을 지나치거나(Overshooting) 과적합될 위험이 큽니다. (성큼성큼 걷다가 구덩이에 빠짐)
    \item **$\lambda$가 너무 작을 때:** 학습이 매우 느리고 많은 트리가 필요하지만, 일반화 성능이 좋습니다. (아주 조심스럽게 발을 내딛음)
\end{itemize}

\subsection{2. 반복 횟수 (Number of Iterations)}
\begin{itemize}
    \item 트리를 몇 개나 더할 것인가를 결정합니다.
    \item 너무 많으면 훈련 데이터에 과도하게 맞춰져(Overfitting) 테스트 성능이 떨어질 수 있습니다.
    \item **조기 종료(Early Stopping):** 검증 데이터(Validation Set)의 성능이 더 이상 좋아지지 않으면 학습을 멈추는 기법을 주로 사용합니다.
\end{itemize}

\subsection{튜닝 전략}
보통 **학습률을 낮게($\lambda \downarrow$)** 설정하고, **트리 개수를 늘리는($N \uparrow$)** 방식이 성능이 가장 좋습니다. 다만 계산 시간이 오래 걸린다는 단점이 있습니다.

% ------------------------------------------------------------------------
% 7. FAQ 및 오해 바로잡기
% ------------------------------------------------------------------------
\section{자주 묻는 질문 (FAQ) 및 오해 바로잡기}

\begin{tcolorbox}[colback=white, colframe=mainblue, title=Q1. 랜덤 포레스트와 부스팅 중 뭐가 더 좋나요?]
\textbf{A.} 정답은 "데이터에 따라 다르다"입니다.
\begin{itemize}
    \item \textbf{랜덤 포레스트:} 병렬 학습이 가능해 빠르고, 튜닝 없이도 무난하게 좋은 성능을 냅니다. (Overfitting에 강함)
    \item \textbf{그레디언트 부스팅:} 순차 학습이라 느리고 튜닝이 까다롭지만, 잘 튜닝하면 일반적으로 랜덤 포레스트보다 \textbf{더 높은 정확도}를 냅니다. (Kaggle 대회 우승 알고리즘의 주역)
\end{itemize}
테이블 형태(Tabular) 데이터에서는 두 모델 모두 딥러닝보다 더 좋은 성능을 낼 때가 많습니다.
\end{tcolorbox}

\begin{tcolorbox}[colback=white, colframe=mainblue, title=Q2. 'Inference'란 무슨 뜻인가요?]
\textbf{A.} 문맥에 따라 다릅니다.
\begin{itemize}
    \item \textbf{통계학:} 데이터로부터 모집단의 특성을 추론하고 가설을 검정하는 것 (예: p-value 구하기).
    \item \textbf{머신러닝/공학:} 학습된 모델에 새로운 데이터를 넣어 \textbf{예측(Prediction)}값을 뽑아내는 과정.
\end{itemize}
부스팅 강의나 자료에서 "Inference가 느리다"고 하면 "예측값을 계산하는 데 시간이 걸린다"는 뜻입니다.
\end{tcolorbox}

\newpage

% ------------------------------------------------------------------------
% 8. 최종 체크리스트
% ------------------------------------------------------------------------
\section{학습 마무리 체크리스트}

이 문서를 다 읽은 후, 아래 질문에 스스로 답할 수 있다면 완벽하게 이해한 것입니다.

\begin{itemize}[label=$\square$]
    \item \textbf{개념:} 부스팅이 배깅(랜덤 포레스트)과 근본적으로 어떻게 다른지 설명할 수 있는가? (힌트: 병렬 vs 순차, 분산 감소 vs 편향 감소)
    \item \textbf{직관:} "약한 학습기"와 "강한 학습기"의 관계를 비유를 들어 설명할 수 있는가?
    \item \textbf{알고리즘:} 그레디언트 부스팅이 다음 트리를 만들 때 사용하는 '타겟 값'이 무엇인가? (힌트: 잔차)
    \item \textbf{수학:} 잔차(Residual)가 왜 손실 함수의 기울기(Gradient)와 관련이 있는지 설명할 수 있는가?
    \item \textbf{실전:} 학습률($\lambda$)이 높을 때와 낮을 때의 장단점을 알고 있는가?
\end{itemize}

\vspace{2cm}

\begin{center}
    \textit{"여러 개의 멍청한 모델이 힘을 합치면, 하나의 천재 모델보다 똑똑할 수 있다."} \\
    \textbf{-- 부스팅의 철학}
\end{center}

\end{document}
