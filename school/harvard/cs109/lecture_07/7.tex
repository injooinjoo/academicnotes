%%%%%%%%%%%%%%%%%%%%%%%%%%%%%%%%%%%%%%%%%%%%%%%%%%%%%%%%%%%%%%%%%%%%%%%%%%%%%%%
% Harvard Academic Notes - 통합 마스터 템플릿
% 모든 강의 노트에 적용되는 통일된 스타일
% 버전: 2.0
% 최종 수정일: 2025-10-26
%%%%%%%%%%%%%%%%%%%%%%%%%%%%%%%%%%%%%%%%%%%%%%%%%%%%%%%%%%%%%%%%%%%%%%%%%%%%%%%

\documentclass[11pt,a4paper]{article}

%========================================================================================
% 기본 패키지
%========================================================================================

% --- 한국어 지원 ---
\usepackage{kotex}

% --- 페이지 레이아웃 ---
\usepackage[margin=25mm]{geometry}
\usepackage{setspace}
\onehalfspacing                      % 1.5배 줄간격
\setlength{\parskip}{0.6em}          % 문단 간격
\setlength{\parindent}{0pt}          % 들여쓰기 없음

% --- 표 관련 ---
\usepackage{booktabs}              % 고품질 표
\usepackage{tabularx}              % 자동 너비 조절 표
\usepackage{array}                 % 표 컬럼 확장
\usepackage{longtable}             % 여러 페이지 표
\renewcommand{\arraystretch}{1.2}  % 표 행간 조절

%========================================================================================
% 헤더 및 푸터
%========================================================================================

\usepackage{fancyhdr}
\pagestyle{fancy}
\fancyhf{}
\fancyhead[L]{\small\textit{CS109A: 데이터 과학 입문}}
\fancyhead[R]{\small\textit{Lecture 07}}
\fancyfoot[C]{\thepage}
\renewcommand{\headrulewidth}{0.5pt}
\renewcommand{\footrulewidth}{0.3pt}

% 첫 페이지는 헤더 없음
\fancypagestyle{firstpage}{
    \fancyhf{}
    \fancyfoot[C]{\thepage}
    \renewcommand{\headrulewidth}{0pt}
}

%========================================================================================
% 색상 정의 (파스텔 톤 + 다크모드 호환)
%========================================================================================

\usepackage[dvipsnames]{xcolor}

% 밝은 배경용 파스텔 색상
\definecolor{lightblue}{RGB}{220, 235, 255}      % 부드러운 파랑
\definecolor{lightgreen}{RGB}{220, 255, 235}     % 부드러운 초록
\definecolor{lightyellow}{RGB}{255, 250, 220}    % 부드러운 노랑
\definecolor{lightpurple}{RGB}{240, 230, 255}    % 부드러운 보라
\definecolor{lightgray}{gray}{0.95}              % 밝은 회색
\definecolor{lightpink}{RGB}{255, 235, 245}      % 부드러운 핑크
\definecolor{boxgray}{gray}{0.95}
\definecolor{boxblue}{rgb}{0.9, 0.95, 1.0}
\definecolor{boxred}{rgb}{1.0, 0.95, 0.95}

% 진한 색상 (테두리/제목용)
\definecolor{darkblue}{RGB}{50, 80, 150}
\definecolor{darkgreen}{RGB}{40, 120, 70}
\definecolor{darkorange}{RGB}{200, 100, 30}
\definecolor{darkpurple}{RGB}{100, 60, 150}

%========================================================================================
% 박스 환경 (tcolorbox) - 6가지 타입
%========================================================================================

\usepackage[most]{tcolorbox}
\tcbuselibrary{skins, breakable}

% 1. 개요 박스 (강의 시작 부분)
\newtcolorbox{overviewbox}[1][]{
    enhanced,
    colback=lightpurple,
    colframe=darkpurple,
    fonttitle=\bfseries\large,
    title=📚 강의 개요,
    arc=3mm,
    boxrule=1pt,
    left=8pt,
    right=8pt,
    top=8pt,
    bottom=8pt,
    breakable,
    #1
}

% 2. 요약 박스
\newtcolorbox{summarybox}[1][]{
    enhanced,
    colback=lightblue,
    colframe=darkblue,
    fonttitle=\bfseries,
    title=📝 핵심 요약,
    arc=2mm,
    boxrule=0.7pt,
    left=6pt,
    right=6pt,
    top=6pt,
    bottom=6pt,
    breakable,
    #1
}

% 3. 핵심 정보 박스
\newtcolorbox{infobox}[1][]{
    enhanced,
    colback=lightgreen,
    colframe=darkgreen,
    fonttitle=\bfseries,
    title=💡 핵심 정보,
    arc=2mm,
    boxrule=0.7pt,
    left=6pt,
    right=6pt,
    top=6pt,
    bottom=6pt,
    breakable,
    #1
}

% 4. 주의사항 박스
\newtcolorbox{warningbox}[1][]{
    enhanced,
    colback=lightyellow,
    colframe=darkorange,
    fonttitle=\bfseries,
    title=⚠️ 주의사항,
    arc=2mm,
    boxrule=0.7pt,
    left=6pt,
    right=6pt,
    top=6pt,
    bottom=6pt,
    breakable,
    #1
}

% 5. 예제 박스
\newtcolorbox{examplebox}[1][]{
    enhanced,
    colback=lightgray,
    colframe=black!60,
    fonttitle=\bfseries,
    title=📖 예제: #1,
    arc=2mm,
    boxrule=0.7pt,
    left=6pt,
    right=6pt,
    top=6pt,
    bottom=6pt,
    breakable,
}

% 6. 정의 박스
\newtcolorbox{definitionbox}[1][]{
    enhanced,
    colback=lightpink,
    colframe=purple!70!black,
    fonttitle=\bfseries,
    title=📌 정의: #1,
    arc=2mm,
    boxrule=0.7pt,
    left=6pt,
    right=6pt,
    top=6pt,
    bottom=6pt,
    breakable,
}

% 7. 중요 박스 (importantbox - warningbox와 유사)
\newtcolorbox{importantbox}[1][]{
    enhanced,
    colback=boxred,
    colframe=red!70!black,
    fonttitle=\bfseries,
    title=⚠️ 매우 중요: #1,
    arc=2mm,
    boxrule=0.7pt,
    left=6pt,
    right=6pt,
    top=6pt,
    bottom=6pt,
    breakable,
}

% 8. cautionbox (warningbox와 동일)
\let\cautionbox\warningbox
\let\endcautionbox\endwarningbox

%========================================================================================
% 코드 블록 설정 (밝은 배경)
%========================================================================================

\usepackage{listings}

\definecolor{codegray}{rgb}{0.5,0.5,0.5}
\definecolor{codepurple}{rgb}{0.58,0,0.82}
\definecolor{backcolour}{rgb}{0.95,0.95,0.95}

\lstset{
    basicstyle=\ttfamily\small,
    backgroundcolor=\color{lightgray},
    keywordstyle=\color{darkblue}\bfseries,
    commentstyle=\color{darkgreen}\itshape,
    stringstyle=\color{purple!80!black},
    numberstyle=\tiny\color{black!60},
    numbers=left,
    numbersep=8pt,
    breaklines=true,
    breakatwhitespace=false,
    frame=single,
    frameround=tttt,
    rulecolor=\color{black!30},
    captionpos=b,
    showstringspaces=false,
    tabsize=2,
    xleftmargin=15pt,
    xrightmargin=5pt,
    escapeinside={\%*}{*)}
}

% Python 코드 스타일
\lstdefinestyle{pythonstyle}{
    language=Python,
    morekeywords={self, True, False, None},
}

% SQL 코드 스타일
\lstdefinestyle{sqlstyle}{
    language=SQL,
    morekeywords={SELECT, FROM, WHERE, JOIN, GROUP, BY, ORDER, HAVING},
}

%========================================================================================
% 목차 스타일링
%========================================================================================

\usepackage{tocloft}
\renewcommand{\cftsecleader}{\cftdotfill{\cftdotsep}}
\setlength{\cftbeforesecskip}{0.4em}
\renewcommand{\cftsecfont}{\bfseries}
\renewcommand{\cftsubsecfont}{\normalfont}

%========================================================================================
% 표 및 그림
%========================================================================================

\usepackage{graphicx}              % 이미지
\usepackage{adjustbox}             % 표/박스 크기 조절

% 표 캡션 스타일
\usepackage{caption}
\captionsetup[table]{
    labelfont=bf,
    textfont=it,
    skip=5pt
}
\captionsetup[figure]{
    labelfont=bf,
    textfont=it,
    skip=5pt
}

%========================================================================================
% 수학
%========================================================================================

\usepackage{amsmath, amssymb, amsthm}

% 정리 환경
\theoremstyle{definition}
\newtheorem{theorem}{정리}[section]
\newtheorem{lemma}[theorem]{보조정리}
\newtheorem{proposition}[theorem]{명제}
\newtheorem{corollary}[theorem]{따름정리}
\newtheorem{definition}{정의}[section]
\newtheorem{example}{예제}[section]

%========================================================================================
% 하이퍼링크
%========================================================================================

\usepackage[
    colorlinks=true,
    linkcolor=blue!80!black,
    urlcolor=blue!80!black,
    citecolor=green!60!black,
    bookmarks=true,
    bookmarksnumbered=true,
    pdfborder={0 0 0}
]{hyperref}

% PDF 메타데이터는 각 문서에서 설정
\hypersetup{
    pdftitle={CS109A: 데이터 과학 입문 - Lecture 07},
    pdfauthor={강의 노트},
    pdfsubject={Academic Notes}
}

%========================================================================================
% 기타 유용한 패키지
%========================================================================================

\usepackage{enumitem}              % 리스트 커스터마이징
\setlist{nosep, leftmargin=*, itemsep=0.3em}

\usepackage{microtype}             % 타이포그래피 개선
\usepackage{footnote}              % 각주 개선
\usepackage{url}                   % URL 줄바꿈
\urlstyle{same}

%========================================================================================
% 사용자 정의 명령어
%========================================================================================

% 강조 텍스트
\newcommand{\important}[1]{\textbf{\textcolor{red!70!black}{#1}}}
\newcommand{\keyword}[1]{\textbf{#1}}
\newcommand{\term}[1]{\textit{#1}}
\newcommand{\code}[1]{\texttt{#1}}

% 용어 설명 (인라인)
\newcommand{\defterm}[2]{\textbf{#1}\footnote{#2}}

% 섹션 시작 전 페이지 분리
\newcommand{\newsection}[1]{\newpage\section{#1}}

%========================================================================================
% 문서 제목 스타일
%========================================================================================

\usepackage{titling}
\pretitle{\begin{center}\LARGE\bfseries}
\posttitle{\par\end{center}\vskip 0.5em}
\preauthor{\begin{center}\large}
\postauthor{\end{center}}
\predate{\begin{center}\large}
\postdate{\par\end{center}}

%========================================================================================
% 섹션 제목 간격
%========================================================================================

\usepackage{titlesec}
\titlespacing*{\section}{0pt}{1.5em}{0.8em}
\titlespacing*{\subsection}{0pt}{1.2em}{0.6em}
\titlespacing*{\subsubsection}{0pt}{1em}{0.5em}

%========================================================================================
% 메타 정보 박스 명령어
%========================================================================================

\newcommand{\metainfo}[4]{
\begin{tcolorbox}[
    colback=lightpurple,
    colframe=darkpurple,
    boxrule=1pt,
    arc=2mm,
    left=10pt,
    right=10pt,
    top=8pt,
    bottom=8pt
]
\begin{tabular}{@{}rl@{}}
▣ \textbf{강의명:} & #1 \\[0.3em]
▣ \textbf{주차:} & #2 \\[0.3em]
▣ \textbf{교수명:} & #3 \\[0.3em]
▣ \textbf{목적:} & \begin{minipage}[t]{0.75\textwidth}#4\end{minipage}
\end{tabular}
\end{tcolorbox}
}

%========================================================================================
% 끝
%========================================================================================


\begin{document}

% 제목/저자/날짜
}
\maketitle
\thispagestyle{firstpage}

\metainfo{CS109A: 데이터 과학 입문}{Lecture 07}{Pavlos Protopapas, Kevin Rader, Chris Gumb}{Lecture 07의 핵심 개념 학습}


\thispagestyle{fancy} % 제목 페이지에도 헤더/푸터 적용

\tableofcontents % 목차 생성

\newpage

%====================
% 1. 개요
%====================
\section{개요}

이 문서는 머신러닝 모델의 성능을 평가하고 개선하는 핵심 원리인 **일반화(Generalization)**, **편향-분산 트레이드오프(Bias-Variance Tradeoff)**, 그리고 **정규화(Regularization)** 기법에 대해 다룹니다.

우리의 최종 목표는 훈련(training) 데이터에만 잘 맞는 모델이 아니라, 한 번도 본 적 없는 새로운 데이터(test data)에서도 좋은 성능을 내는 **'일반화 성능이 뛰어난'** 모델을 만드는 것입니다.

모델이 너무 단순하면 훈련 데이터조차 제대로 학습하지 못하며(과소적합, High Bias),
모델이 너무 복잡하면 훈련 데이터의 노이즈까지 암기해버려 새로운 데이터에서 형편없는 성능을 보입니다(과대적합, High Variance).

이 문서는 과대적합의 주된 증상(모델 계수의 폭주)을 진단하고, 이를 해결하기 위해 손실 함수(Loss Function)에 **'패널티'**를 부과하는 정규화 기법, 특히 **릿지(Ridge, L2)**와 **라쏘(Lasso, L1)**를 중점적으로 설명합니다.

마지막으로, 이 패널티의 강도를 조절하는 하이퍼파라미터 **$\lambda$ (람다)**를 찾기 위한 체계적인 절차로 **검증 세트(Validation Set)**와 **교차 검증(Cross-Validation)** 방법을 단계별로 학습합니다.

\newpage

%====================
% 2. 용어 정리
%====================
\section{용어 정리}

핵심 용어들을 미리 이해하면 학습에 도움이 됩니다.

\begin{table}[h!]
\caption{핵심 용어 정리표}
\label{tab:terms}
\begin{adjustbox}{width=\textwidth, center}
\begin{tabular}{@{}llll@{}}
\toprule
\textbf{용어} & \textbf{원어 (Full Term)} & \textbf{쉬운 설명 (초심자용)} & \textbf{비고} \\
\midrule
일반화 & Generalization & 모델이 훈련 데이터가 아닌 '새로운 데이터'를 얼마나 잘 맞추는지의 능력. & 높을수록 좋은 모델입니다. \\
\addlinespace
과소적합 & Underfitting & 모델이 너무 단순해서 훈련 데이터조차 제대로 학습하지 못한 상태. & 편향(Bias)이 높은 상태입니다. \\
\addlinespace
과대적합 & Overfitting & 모델이 너무 복잡해서 훈련 데이터의 '노이즈'까지 암기해버린 상태. & 새로운 데이터에서 성능이 급격히 저하됩니다. 분산(Variance)이 높은 상태입니다. \\
\addlinespace
편향 & Bias & 모델의 예측이 '실제 정답'과 평균적으로 얼마나 멀리 떨어져 있는가. & \textbf{'부정확성'.} 과녁의 중심을 못 맞춤. \\
\addlinespace
분산 & Variance & 훈련 데이터가 조금 바뀔 때 모델의 예측이 얼마나 크게 출렁이는가. & \textbf{'비일관성'.} 쏠 때마다 탄착군이 흩어짐. \\
\addlinespace
정규화 & Regularization & 모델의 복잡도(주로 계수 값)에 패널티를 부과하여 과대적합을 막는 기법. & "모델이 너무 복잡해지지 마!" \\
\addlinespace
릿지 (L2) & Ridge Regression & 계수의 '제곱의 합'에 패널티를 주는 정규화. ($L_2$ Norm) & 계수를 0에 가깝게 줄이지만 0으로 만들진 않음. \\
\addlinespace
라쏘 (L1) & Lasso Regression & 계수의 '절대값의 합'에 패널티를 주는 정규화. ($L_1$ Norm) & 불필요한 계수를 아예 0으로 만들어 '변수 선택' 효과. \\
\addlinespace
$\lambda$ (람다) & Lambda & 정규화의 '강도'를 조절하는 하이퍼파라미터. & 0이면 정규화 안함. $\infty$면 모든 계수가 0이 됨. \\
\addlinespace
하이퍼파라미터 & Hyperparameter & 모델이 스스로 학습하는 값($\beta$)이 아니라, '사람이 직접 설정'해줘야 하는 값. & $\lambda$나 K-Fold의 $K$ 값 등. \\
\bottomrule
\end{tabular}
\end{adjustbox}
\end{table}

\newpage

%====================
% 3. 핵심 개념: 편향-분산 트레이드오프
%====================
\section{핵심 개념: 편향-분산 트레이드오프 (Bias-Variance Tradeoff)}

모델의 예측 오차(Error)는 우리가 어떻게 할 수 없는 부분과, 우리가 개선할 수 있는 부분으로 나뉩니다.

\subsection{모델 오차의 두 가지 근원}

\begin{tcolorbox}[title=1. 환원 불가능한 오차 (Irreducible Error)]
이 오차는 데이터 자체에 내재된 **'무작위 노이즈'** 때문에 발생합니다.
아무리 완벽한 모델을 만들어도 이 오차는 절대 0이 될 수 없습니다.

\begin{examplebox}
**비유 (Aleatoric Error):**
아무리 비싼 최고급 마이크($\approx$모델)를 사용해도, 녹음실 주변의 공사장 소음($\approx$노이즈)까지 함께 녹음되는 것과 같습니다. 이 소음은 마이크 성능으로 제거할 수 없습니다.

우리는 이 오차의 존재를 인정하고, 우리가 줄일 수 있는 오차에 집중해야 합니다.
\end{examplebox}
\end{tcolorbox}

\begin{tcolorbox}[title=2. 환원 가능한 오차 (Reducible Error)]
이 오차는 우리가 **'모델을 잘못 선택'**했기 때문에 발생합니다.
우리의 임무는 이 오차를 최소화하는 것이며, 이 오차는 다시 '편향'과 '분산'이라는 두 가지 요소로 분해됩니다.
\end{tcolorbox}

\subsection{편향(Bias)과 분산(Variance)의 정의}

모델의 성능을 사격에 비유하여 편향과 분산을 이해할 수 있습니다.

\begin{tcolorbox}[title=편향 (Bias): 과녁을 놓치다 (부정확성)]
**편향**은 모델의 예측값이 실제 정답(과녁의 중심)과 평균적으로 얼마나 멀리 떨어져 있는지를 나타냅니다.

\begin{itemize}
    \item \textbf{High Bias (높은 편향):} 모델이 너무 단순하여 데이터의 복잡한 패턴을 전혀 학습하지 못합니다. (예: S자 곡선 데이터를 직선으로 예측하려는 시도)
    \item \textbf{결과:} \textbf{과소적합 (Underfitting)}. 훈련 데이터에서도, 테스트 데이터에서도 모두 성능이 나쁩니다.
    \item \textbf{특징:} 훈련 데이터를 바꿔가며 여러 번 학습해도 예측 결과가 거의 변하지 않습니다 (낮은 분산).
\end{itemize}
\end{tcolorbox}

\begin{tcolorbox}[title=분산 (Variance): 탄착군이 흩어지다 (비일관성)]
**분산**은 훈련 데이터가 조금만 바뀌어도 모델의 예측이 얼마나 크게 변동하는지를 나타냅니다.

\begin{itemize}
    \item \textbf{High Variance (높은 분산):} 모델이 너무 복잡하여 훈련 데이터의 사소한 노이즈까지 '암기'해버립니다.
    \item \textbf{결과:} \textbf{과대적합 (Overfitting)}. 훈련 데이터에서는 완벽(0에 가까운 오차)하지만, 새로운 테스트 데이터에서는 성능이 재앙 수준입니다.
    \item \textbf{특징:} 훈련 데이터 샘플이 조금만 달라져도 모델의 형태가 스파게티 면발처럼 마구 요동칩니다.
\end{itemize}
\end{tcolorbox}

\begin{examplebox}
**시뮬레이션 예시 (스파게티 면발):**
서로 다른 2,000개의 샘플 데이터셋을 뽑아서 모델을 2,000번 학습시켰다고 가정해봅시다.

\begin{itemize}
    \item \textbf{단순한 선형 모델 (Low Variance):} 2,000개의 예측선이 모두 비슷하게 그려집니다. (안정적)
    \item \textbf{복잡한 10차 다항식 모델 (High Variance):} 2,000개의 예측선이 샘플 데이터의 노이즈에 민감하게 반응하여, 마치 스파게티 면발처럼 서로 얽히고설켜 그려집니다. (불안정)
\end{itemize}
\end{examplebox}


\subsection{트레이드오프(Tradeoff) 관계}

편향과 분산은 한쪽이 줄어들면 다른 한쪽이 늘어나는 **'시소 관계'**에 있습니다.

\begin{tcolorbox}[title=⭐ 모델 복잡도에 따른 오차의 변화]
\textbf{모델 복잡도(X축) vs. 총 오차(Y축)}

\begin{itemize}
    \item \textbf{모델이 단순할수록 (좌측):}
        \begin{itemize}
            \item 편향(Bias)은 \textbf{높고} (데이터를 못 맞춤)
            \item 분산(Variance)은 \textbf{낮습니다}. (예측이 안정적)
        \end{itemize}
    \item \textbf{모델이 복잡해질수록 (우측):}
        \begin{itemize}
            \item 편향(Bias)은 \textbf{낮아지고} (훈련 데이터를 완벽히 맞춤)
            \item 분산(Variance)은 \textbf{급격히 높아집니다}. (데이터에 과민 반응)
        \end{itemize}
\end{itemize}
총 오차(Total Error = Bias$^2$ + Variance + Irreducible Error)는 U자 형태의 곡선을 그립니다.
우리의 목표는 이 U자 곡선의 가장 낮은 지점, 즉 \textbf{총 오차를 최소화하는 최적의 복잡도}를 찾는 것입니다.
\end{tcolorbox}

\textbf{사격 과녁 비유 요약표}

\begin{table}[h!]
\caption{편향과 분산의 4가지 시나리오 (사격 비유)}
\label{tab:target}
\begin{adjustbox}{width=\textwidth, center}
\begin{tabular}{@{}lll@{}}
\toprule
& \textbf{Low Variance (낮은 분산 / 일관성 $\uparrow$)} & \textbf{High Variance (높은 분산 / 일관성 $\downarrow$)} \\
\midrule
\textbf{High Bias (높은 편향 / 정확도 $\downarrow$)} &
  \begin{minipage}{0.45\textwidth}
    \textbf{과소적합 (Underfitting)}
    \begin{itemize}
        \item 과녁의 중심은 못 맞추지만, 쏜 지점에 계속 일관되게 쏨.
        \item (예: 단순 선형 모델)
    \end{itemize}
  \end{minipage} &
  \begin{minipage}{0.45\textwidth}
    \textbf{최악의 모델 (Worst)}
    \begin{itemize}
        \item 과녁의 중심도 못 맞추고, 쏠 때마다 아무데나 흩어짐.
    \end{itemize}
  \end{minipage} \\
\addlinespace
\textbf{Low Bias (낮은 편향 / 정확도 $\uparrow$)} &
  \begin{minipage}{0.45\textwidth}
    \textbf{이상적인 모델 (Ideal)}
    \begin{itemize}
        \item 과녁의 중심(정답)에 정확하고 일관되게 쏨.
        \item \textbf{우리의 목표!}
    \end{itemize}
  \end{minipage} &
  \begin{minipage}{0.45\textwidth}
    \textbf{과대적합 (Overfitting)}
    \begin{itemize}
        \item 평균적으로 과녁 중심 근처에 맞지만(훈련 데이터), 쏠 때마다 탄착군이 너무 흩어짐.
        \item (예: 고차 다항식 모델)
    \end{itemize}
  \end{minipage} \\
\bottomrule
\end{tabular}
\end{adjustbox}
\end{table}

\newpage

%====================
% 4. 문제 진단
%====================
\section{문제 진단: 과대적합과 모델 계수}

\textbf{질문: "모델이 과대적합(High Variance)되었는지 어떻게 알 수 있습니까?"}

\textbf{답변: "모델의 계수(coefficients, $\beta$) 값을 확인하면 됩니다."}

모델이 과대적합 상태일 때, 즉 훈련 데이터의 노이즈에 과민하게 반응할 때, 모델의 **계수($\beta_j$) 값들은 비정상적으로 커지거나 극단적으로 불안정한 값**을 갖게 됩니다.

\begin{examplebox}
**계수 값 분포 비교 (Violin Plots):**
서로 다른 2,000개의 샘플 데이터로 2,000개의 모델을 학습시킨 경우의 계수 분포입니다.

\begin{itemize}
    \item \textbf{단순 선형 모델 (Low Variance):}
        $\beta_0$, $\beta_1$ 계수 값들이 0.0에서 1.25 사이의 좁은 범위에서 안정적으로 분포합니다.
    \item \textbf{복잡한 10차 다항식 모델 (High Variance):}
        $\beta_5, \beta_8, \beta_9$ 등의 고차항 계수 값들이 $1e9$ (즉, 10억) 스케일로 폭주하며, 매우 넓은 범위(큰 분산)를 갖습니다.
\end{itemize}
\end{examplebox}

\begin{warningbox}
\textbf{과대적합의 핵심 증상:}
높은 분산(High Variance) $\rightarrow$ 불안정하고 극단적인 계수(Large $\beta$) 값

따라서, 과대적합을 막기 위한 해결책은 \textbf{"모델의 계수 값이 너무 커지지 않도록 억제하는 것"}입니다.
이것이 바로 '정규화(Regularization)'의 핵심 아이디어입니다.
\end{warningbox}

\newpage

%====================
% 5. 해결책: 정규화
%====================
\section{해결책: 정규화 (Regularization)}

\subsection{정규화의 핵심 아이디어}

정규화는 모델의 손실 함수(Loss Function)를 수정하여, 모델이 두 가지 목표를 동시에 달성하도록 강제합니다.

\begin{enumerate}
    \item \textbf{목표 1: 데이터를 잘 맞춰라.} (기존의 목표)
        $\rightarrow$ 손실 함수(MSE)를 최소화. $\frac{1}{n}\sum_{i=1}^{n}(y_i - \hat{y}_i)^2$
    \item \textbf{목표 2: 계수 값을 작게 유지해라.} (새로운 목표: 과대적합 방지)
        $\rightarrow$ 계수의 크기에 대한 '패널티 항'을 최소화.
\end{enumerate}

\textbf{새로운 정규화 손실 함수:}
$$
\mathcal{L}_{\text{REG}} = (\text{기존 MSE}) + \lambda \times (\text{패널티 항})
$$
$$
\mathcal{L}_{\text{REG}} = \frac{1}{n}\sum_{i=1}^{n}(y_i - \beta^{\top}x_i)^2 + \lambda L_{\text{reg}}
$$

모델은 이제 MSE만 줄이는 것이 아니라, (MSE + $\lambda \times$ 패널티)의 \textbf{총합}을 최소화해야 합니다.

\subsection{$\lambda$ (람다)의 역할: 패널티의 강도}

$\lambda$ (람다)는 패널티의 '강도'를 조절하는 하이퍼파라미터입니다.

\begin{itemize}
    \item \textbf{만약 $\lambda = 0$ 이라면:}
        패널티가 0이 됩니다. 이는 일반적인 선형 회귀와 같으며 과대적합의 위험이 있습니다.
    \item \textbf{만약 $\lambda \rightarrow \infty$ (무한대)라면:}
        패널티가 너무 강력해집니다. 모델은 MSE를 무시하고 오직 패널티($\sum \beta_j^2$ 또는 $\sum |\beta_j|$)를 0으로 만드는 데만 집중합니다. 그 결과 모든 계수 $\beta_j$가 0이 되어, 모델은 단순한 수평선(평균값)이 됩니다 (과소적합, High Bias).
\end{itemize}

\begin{summarybox}
우리의 목표는 훈련 데이터와 검증 데이터를 사용하여 편향과 분산 사이의 균형을 잡는 \textbf{'최적의 $\lambda$'}를 찾는 것입니다.
\end{summarybox}

\subsection{두 가지 정규화 기법: L2 (Ridge) vs. L1 (Lasso)}

패널티 항($L_{\text{reg}}$)을 어떻게 정의하느냐에 따라 릿지(Ridge)와 라쏘(Lasso)로 나뉩니다.

\begin{tcolorbox}[title=1. L2 정규화: 릿지 회귀 (Ridge Regression)]
\textbf{패널티 항:} 계수의 **제곱의 합** ($L_2$ Norm) $\rightarrow L_{\text{reg}} = \sum_{j=1}^{J} \beta_j^2$

\textbf{최종 손실 함수:}
$$
\mathcal{L}_{\text{RIDGE}} = \frac{1}{n}\sum_{i=1}^{n}(y_i - \beta^{\top}x_i)^2 + \lambda \sum_{j=1}^{J} \beta_j^2
$$

\begin{itemize}
    \item \textbf{특징:} 
        계수 값이 커질수록 패널티가 '제곱'으로 증가하므로, 매우 큰(튀는) 계수 값을 강력하게 억제합니다. 모든 계수를 0에 '가깝게' 줄이지만, 정확히 0으로 만들지는 않습니다.
    \item \textbf{장점:} 
        계산이 빠릅니다 (수학적인 공식, 즉 'Analytical Solution'이 존재함). 다중공선성(Multicollinearity: 예측 변수 간 강한 상관관계)이 있을 때 모델을 안정화시키는 데 매우 효과적입니다.
    \item \textbf{적합한 상황:} 
        모든 변수(feature)가 예측에 어느 정도 기여한다고 판단될 때 사용합니다.
\end{itemize}
\end{tcolorbox}

\begin{tcolorbox}[title=2. L1 정규화: 라쏘 회귀 (Lasso Regression)]
\textbf{패널티 항:} 계수의 **절대값의 합** ($L_1$ Norm) $\rightarrow L_{\text{reg}} = \sum_{j=1}^{J} |\beta_j|$

\textbf{최종 손실 함수:}
$$
\mathcal{L}_{\text{LASSO}} = \frac{1}{n}\sum_{i=1}^{n}(y_i - \beta^{\top}x_i)^2 + \lambda \sum_{j=1}^{J} |\beta_j|
$$

\begin{itemize}
    \item \textbf{특징:} 
        중요하지 않은 변수의 계수는 '정확히 0'으로 만들어 버립니다. 이는 모델에서 해당 변수를 아예 '제거'하는 것과 같은 효과를 줍니다.
    \item \textbf{장점:} 
        모델의 복잡도를 근본적으로 낮추는 **자동 변수 선택(Feature Selection)** 기능이 있습니다. 해석하기 쉬운 단순한 모델을 만듭니다.
    \item \textbf{적합한 상황:} 
        수백, 수천 개의 변수 중 실제 중요한 변수는 몇 개 안 된다고 의심될 때 매우 유용합니다.
\end{itemize}
\end{tcolorbox}

\begin{warningbox}
\textbf{주의: 절편($\beta_0$)은 정규화하지 않습니다.}
$L_1, L_2$ 패널티 항은 $\beta_1$부터 $\beta_J$까지만 적용됩니다.
절편(intercept) $\beta_0$는 특정 변수와 연결된 민감도가 아니라, 모델 전체의 기본 '높낮이(offset)'를 조절할 뿐이므로 패널티 대상에서 제외합니다.
\end{warningbox}

\subsection{계산적 차이 및 비교 요약}

\begin{itemize}
    \item \textbf{릿지(Ridge):} 행렬을 이용한 명확한 수학 공식(Analytical Solution)으로 $\beta$ 값을 한 번에 계산할 수 있습니다.
    $$
    \hat{\beta}_{\text{Ridge}} = (X^{\top}X + \lambda I)^{-1} X^{\top}Y
    $$
    \item \textbf{라쏘(Lasso):} 절대값 함수는 $0$에서 미분이 불가능하므로, 이런 수학 공식이 없습니다. 대신 '수치적 최적화(Numerical Optimization)' 기법 (예: Solver, Coordinate Descent)을 사용하여 반복적으로 $\beta$ 값을 찾아가야 하므로, 릿지보다 계산 속도가 느릴 수 있습니다.
\end{itemize}

\begin{table}[h!]
\caption{Ridge (L2) vs. Lasso (L1) 비교 요약}
\label{tab:compare}
\begin{adjustbox}{width=\textwidth, center}
\begin{tabular}{@{}lll@{}}
\toprule
\textbf{구분} & \textbf{릿지 회귀 (Ridge, L2)} & \textbf{라쏘 회귀 (Lasso, L1)} \\
\midrule
\textbf{패널티 항} & 계수의 제곱의 합 ($\sum \beta_j^2$) & 계수의 절대값의 합 ($\sum |\beta_j|$) \\
\addlinespace
\textbf{계수 축소} & 계수를 0에 \textbf{가깝게} 줄임 (0은 안 됨) & 불필요한 계수를 정확히 \textbf{0}으로 만듦 \\
\addlinespace
\textbf{핵심 기능} & 모델 안정화, 다중공선성 제어 & \textbf{변수 선택 (Feature Selection)} \\
\addlinespace
\textbf{계산 방식} & \textbf{빠름} (Analytical Solution 존재) & \textbf{느림} (Numerical Solver 필요) \\
\addlinespace
\textbf{도형적 해석} & 패널티 영역이 '원' (Circle) 형태 & 패널티 영역이 '다이아몬드' (Diamond) 형태 \\
\bottomrule
\end{tabular}
\end{adjustbox}
\end{table}

\newpage

%====================
% 6. 절차: 최적의 람다 찾기
%====================
\section{절차: 최적의 $\lambda$ (람다) 찾기 (Hyperparameter Tuning)}

$\lambda$는 하이퍼파라미터입니다. 즉, 모델이 스스로 학습하는 값이 아니라 우리가 정해줘야 하는 값입니다. 최적의 $\lambda$를 찾는 과정을 '튜닝(Tuning)'이라고 합니다.

\begin{warningbox}
\textbf{하이퍼파라미터 튜닝의 철칙:}
\begin{itemize}
    \item \textbf{훈련(Train) 데이터}로 튜닝하면 안 됩니다. (모델이 $\lambda=0$을 선호하게 됨)
    \item \textbf{테스트(Test) 데이터}로 튜닝하면 절대 안 됩니다. (정보 유출, 즉 'Cheating'임)
    \item 오직 \textbf{검증(Validation) 데이터} 또는 \textbf{교차 검증(Cross-Validation)}을 사용해야 합니다.
\end{itemize}
\end{warningbox}

\subsection{방법 1: 단일 검증 세트 (Single Validation Set) 사용}

가장 기본적인 방법으로, 데이터를 Train / Validation / Test 세 부분으로 나누어 진행합니다.

\begin{enumerate}
    \item \textbf{Step 1: 데이터 분할 (Split Data)}
    데이터를 훈련(Train), 검증(Validation), 테스트(Test)용으로 3-way 분할합니다.

    \item \textbf{Step 2: $\lambda$ 후보군 선정 (Select $\lambda$ range)}
    테스트할 $\lambda$ 값들의 목록을 만듭니다. (예: `[0.0001, 0.001, 0.01, 0.1, 1, 10, 100]`)

    \item \textbf{Step 3: 모델 훈련 (Train Models)}
    \textbf{훈련(Train) 데이터}를 사용하여, \textbf{각 $\lambda$ 후보마다} 정규화(Ridge/Lasso) 모델을 학습시키고 계수($\beta_{\lambda}$)를 얻습니다.

    \item \textbf{Step 4: 검증 및 MSE 기록 (Validate)}
    \textbf{검증(Validation) 데이터}를 사용하여, 3단계에서 얻은 각 모델($\beta_{\lambda}$)의 성능(MSE)을 측정합니다.

    \begin{warningbox}
    \textbf{매우 중요!} 이 단계에서 성능을 측정할 때는 패널티 항($\lambda L_{reg}$)을 \textbf{제외}하고, \textbf{순수한 MSE} 값만 계산합니다.
    
    \textbf{이유:} $\lambda$는 모델을 '훈련'시킬 때 과대적합을 막기 위한 도구일 뿐, 모델의 '순수한 예측 성능'을 평가할 때는 MSE(실제 정답과의 차이)만 보는 것이 타당합니다.
    \end{warningbox}

    \item \textbf{Step 5: 최적 $\lambda^*$ 선정 (Select Best $\lambda$)}
    4단계에서 기록한 MSE 값들 중, \textbf{가장 낮은 MSE}를 기록한 $\lambda$를 최종 $\lambda^*$ (람다-스타)로 선정합니다.

    \item \textbf{Step 6: (권장) 모델 재훈련 (Refit Model)}
    최적의 하이퍼파라미터 $\lambda^*$를 찾았으므로, 이제 '검증 세트'의 임무는 끝났습니다.
    훈련(Train) 데이터와 검증(Validation) 데이터를 \textbf{다시 하나로 합친} 더 큰 데이터셋을 사용하여, $\lambda^*$ 값으로 모델을 \textbf{단 한 번} 재훈련합니다.
    
    \textbf{이유:} 모델 선택이 끝났으니, 검증에 썼던 데이터도 훈련에 사용하여 최종 모델이 조금이라도 더 많은 정보를 학습하게 합니다.

    \item \textbf{Step 7: 최종 평가 (Final Report)}
    지금까지 단 한 번도 사용하지 않은 \textbf{테스트(Test) 데이터}를 사용하여, 6단계에서 얻은 최종 모델의 성능(MSE)을 평가하고 이 점수를 보고합니다.
\end{enumerate}

\subsection{방법 2: K-Fold 교차 검증 (Cross-Validation)}

단일 검증 세트는 데이터가 어떻게 분할되었느냐에 따라 '운' 좋게 특정 $\lambda$에 유리한 결과가 나올 수 있습니다. (e.g., 검증 데이터가 우연히 직선 형태)

**K-Fold 교차 검증(CV)**은 이 '운'의 요소를 제거하여 더 안정적이고 신뢰할 수 있는 $\lambda$를 찾는, 더 강력한 방법입니다.

\begin{enumerate}
    \item \textbf{Step 1: 데이터 분할 (Split Data)}
    데이터를 (훈련+검증)용 \textbf{훈련 풀(Training Pool)}과 \textbf{테스트(Test)}용으로 2-way 분할합니다.

    \item \textbf{Step 2: $\lambda$ 후보군 선정 (Select $\lambda$ range)}
    (이전과 동일. 예: `[0.001, 0.1, 1, 10, 100]`)

    \item \textbf{Step 3: K-Fold 분할 (Split K-Folds)}
    \textbf{훈련 풀(Training Pool)}을 K개 (예: 5개)의 '폴드(fold)'로 균등하게 나눕니다.

    \item \textbf{Step 4: K-Fold CV 루프 실행 (Run CV Loop)}
    K번 반복합니다. (예: $k=1$부터 $5$까지)
    \begin{itemize}
        \item \textbf{For $k=1$:} Fold 1을 '검증용', Fold 2~5를 '훈련용'으로 사용.
        \item \textbf{For $k=2$:} Fold 2를 '검증용', Fold 1, 3~5를 '훈련용'으로 사용.
        \item \dots (K번 반복) \dots
    \end{itemize}
    각 $k$번째 반복마다, \textbf{모든 $\lambda$ 후보}에 대해 모델을 훈련하고 검증용 폴드의 MSE를 기록합니다.

    \item \textbf{Step 5: 평균 MSE 계산 (Average MSEs)}
    4단계가 끝나면, 각 $\lambda$ 후보마다 K개의 MSE 값 (예: $\lambda=0.1$일 때 5개의 MSE)이 쌓입니다.
    각 $\lambda$별로 K개의 MSE 값의 \textbf{평균}을 계산합니다.
    
    (예: `Avg_MSE[$\lambda=0.1$] = (MSE_k1 + ... + MSE_k5) / 5`)

    \item \textbf{Step 6: 최적 $\lambda^*$ 선정 (Select Best $\lambda$)}
    5단계에서 계산한 \textbf{평균 MSE} 값들 중, \textbf{가장 낮은 평균 MSE}를 기록한 $\lambda$를 최종 $\lambda^*$로 선정합니다.

    \item \textbf{Step 7: 모델 재훈련 (Refit Model)}
    K-Fold CV는 오직 $\lambda^*$를 찾기 위한 과정이었습니다.
    이제 \textbf{훈련 풀 전체(1~5 Fold 모두)}를 사용하여, $\lambda^*$ 값으로 모델을 \textbf{단 한 번} 재훈련합니다.
    
    \textbf{이유:} K개의 모델 중 하나를 고르는 것이 아니라, 찾은 최적의 $\lambda$를 사용하여 *모든* 훈련 데이터를 학습한 최종 모델을 얻기 위함입니다.

    \item \textbf{Step 8: 최종 평가 (Final Report)}
    지금까지 단 한 번도 사용하지 않은 \textbf{테스트(Test) 데이터}로 7단계의 최종 모델 성능을 평가하고 보고합니다.
\end{enumerate}

\newpage

%====================
% 7. FAQ
%====================
\section{FAQ 및 주요 질문}

\begin{tcolorbox}[title={Q: 릿지(Ridge)와 라쏘(Lasso) 중 무엇을 써야 하나요?}]
\textbf{A:} 정답은 없습니다. 상황에 따라 다르며, 둘 다 시도하고 교차 검증(CV) 점수를 비교하는 것이 가장 좋습니다.

\begin{itemize}
    \item \textbf{라쏘(Lasso)가 유리할 때:} 
        변수가 수백~수천 개로 매우 많고, 그 중 '소수의 핵심 변수'만 예측에 중요하다고 의심될 때. 라쏘가 불필요한 변수들을 0으로 만들어 \textbf{변수 선택(Feature Selection)}을 자동으로 해줍니다.
    \item \textbf{릿지(Ridge)가 유리할 때:} 
        모든 변수가 예측에 조금씩이라도 기여한다고 생각될 때. 특히 변수들 간에 강한 상관관계(다중공선성)가 있을 때, 라쏘보다 더 안정적인 성능을 보입니다.
\end{itemize}
\end{tcolorbox}

\begin{tcolorbox}[title={Q: $\lambda$ 탐색 범위를 정했는데, 최적값이 범위의 경계(예: 0.001 또는 100)에서 나왔습니다.}]
\textbf{A:} \textbf{탐색 범위를 더 넓혀야 합니다.}
만약 $\lambda=100$에서 MSE가 최소였다면, 이는 $\lambda=1000, 10000$일 때 MSE가 더 낮아질 가능성이 있다는 신호입니다.
최적의 $\lambda$는 U자형 MSE 곡선의 '바닥'에 있어야 합니다. 탐색 범위의 경계에서 최적값이 나왔다면, 아직 U자 곡선의 바닥을 찾지 못했다는 뜻이므로 범위를 확장하여 다시 시도해야 합니다.
\end{tcolorbox}

\begin{tcolorbox}[title={Q: (단일 검증) Step 6에서 왜 검증(Validation) 세트를 다시 훈련(Train) 세트에 합쳐서 재훈련하나요?}]
\textbf{A:} 검증 세트의 유일한 임무는 '최적의 하이퍼파라미터($\lambda^*$)를 찾는 것'이었습니다.
일단 $\lambda^*$를 찾았다면, 검증 세트는 더 이상 필요 없습니다. 이 데이터를 '버리기'보다는, 최종 모델을 훈련시킬 때 \textbf{훈련 데이터에 다시 합쳐서} 모델이 조금이라도 더 많은 데이터를 학습하게 하는 것이 성능에 유리합니다.

단, \textbf{테스트(Test) 세트는 절대 훈련에 사용해서는 안 됩니다.}
\end{tcolorbox}

\begin{tcolorbox}[title={Q: K-Fold CV에서 K는 몇으로 정해야 하나요?}]
\textbf{A:} \textbf{일반적으로 K=5 또는 K=10을 가장 많이 씁니다.}
$K$ 역시 하이퍼파라미터지만, $K$ 값을 튜닝하기 위해 또 CV를 하는 것은 비효율적일 수 있습니다. (예: $K=5$일 때와 $K=10$일 때의 최종 성능 차이가 미미한 경우가 많음)

$K=5$ 또는 $K=10$ 정도로도 충분히 안정적인 $\lambda$ 값을 찾을 수 있습니다.
\end{tcolorbox}

\newpage

%====================
% 8. 1페이지 요약
%====================
\section{1페이지 요약: 빠른 훑어보기}

\begin{summarybox}
\textbf{핵심 목표: 일반화 (Generalization)}
훈련 데이터가 아닌, '새로운 데이터'를 잘 맞추는 모델을 원한다.
\end{summarybox}

\begin{tcolorbox}[title=1. 문제: 편향 vs. 분산 (Bias vs. Variance)]
\begin{itemize}
    \item \textbf{High Bias (과소적합):} 모델이 너무 단순함. (예측이 정답과 멂)
    \item \textbf{High Variance (과대적합):} 모델이 너무 복잡함. (예측이 데이터마다 널뛰기함)
\end{itemize}
\end{tcolorbox}

\begin{tcolorbox}[title=2. 진단: 과대적합의 징후]
모델의 \textbf{계수($\beta$) 값}이 비정상적으로 \textbf{크고 불안정}해진다.
($\rightarrow$ "계수 값을 줄여야 한다!")
\end{tcolorbox}

\begin{tcolorbox}[title=3. 해결: 정규화 (Regularization)]
손실 함수에 '패널티 항'을 추가하여 계수 값이 커지는 것을 억제한다.
$$
\text{New Loss} = \text{MSE} + \lambda \times (\text{Penalty})
$$
\end{tcolorbox}

\begin{tcolorbox}[title=4. 방법: Ridge (L2) vs. Lasso (L1)]
\begin{itemize}
    \item \textbf{Ridge (L2) $\rightarrow \sum \beta_j^2$:} 
        계수를 0에 \textbf{가깝게} 줄인다. (안정성 $\uparrow$)
    \item \textbf{Lasso (L1) $\rightarrow \sum |\beta_j|$:} 
        계수를 정확히 \textbf{0}으로 만든다. (\textbf{변수 선택} 기능)
\end{itemize}
\end{tcolorbox}

\begin{tcolorbox}[title=5. 튜닝: 최적의 $\lambda$ 찾기 (by CV)]
\begin{enumerate}
    \item $\lambda$ 후보 목록을 만든다. (e.g., [0.01, 0.1, 1, 10])
    \item 각 $\lambda$마다 K-Fold CV를 실행하여 \textbf{평균 검증 MSE}를 계산한다.
    \item \textbf{평균 검증 MSE가 가장 낮은 $\lambda^*$}를 선택한다.
    \item (훈련+검증) \textbf{전체 데이터}로 $\lambda^*$를 적용하여 최종 모델을 재훈련한다.
    \item \textbf{테스트(Test) 세트}로 최종 성능을 딱 1번 보고한다.
\end{enumerate}
\end{tcolorbox}

\end{document}
