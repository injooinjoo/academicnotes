%%%%%%%%%%%%%%%%%%%%%%%%%%%%%%%%%%%%%%%%%%%%%%%%%%%%%%%%%%%%%%%%%%%%%%%%%%%%%%%
% Harvard Academic Notes - 통합 마스터 템플릿
% 모든 강의 노트에 적용되는 통일된 스타일
% 버전: 2.1 - 가독성 개선 (선택적 최적화)
% 최종 수정일: 2025-11-17
%%%%%%%%%%%%%%%%%%%%%%%%%%%%%%%%%%%%%%%%%%%%%%%%%%%%%%%%%%%%%%%%%%%%%%%%%%%%%%%

\documentclass[11pt,a4paper]{article}

%========================================================================================
% 기본 패키지
%========================================================================================

% --- 한국어 지원 ---
\usepackage{kotex}

% --- 페이지 레이아웃 ---
\usepackage[top=20mm, bottom=20mm, left=20mm, right=18mm]{geometry}
\usepackage{setspace}
\onehalfspacing                      % 1.5배 줄간격
\setlength{\parskip}{0.5em}          % 문단 간격
\setlength{\parindent}{0pt}          % 들여쓰기 없음

% --- 표 관련 ---
\usepackage{booktabs}              % 고품질 표
\usepackage{tabularx}              % 자동 너비 조절 표
\usepackage{array}                 % 표 컬럼 확장
\usepackage{longtable}             % 여러 페이지 표
\renewcommand{\arraystretch}{1.1}  % 표 행간 조절

%========================================================================================
% 헤더 및 푸터
%========================================================================================

\usepackage{fancyhdr}
\pagestyle{fancy}
\fancyhf{}
\fancyhead[L]{\small\textit{CS109A: 데이터 과학 입문}}
\fancyhead[R]{\small\textit{Lecture 22}}
\fancyfoot[C]{\thepage}
\renewcommand{\headrulewidth}{0.5pt}
\renewcommand{\footrulewidth}{0.3pt}

% 첫 페이지는 헤더 없음
\fancypagestyle{firstpage}{
    \fancyhf{}
    \fancyfoot[C]{\thepage}
    \renewcommand{\headrulewidth}{0pt}
}

%========================================================================================
% 색상 정의 (파스텔 톤 + 다크모드 호환)
%========================================================================================

\usepackage[dvipsnames]{xcolor}

% 밝은 배경용 파스텔 색상
\definecolor{lightblue}{RGB}{220, 235, 255}      % 부드러운 파랑
\definecolor{lightgreen}{RGB}{220, 255, 235}     % 부드러운 초록
\definecolor{lightyellow}{RGB}{255, 250, 220}    % 부드러운 노랑
\definecolor{lightpurple}{RGB}{240, 230, 255}    % 부드러운 보라
\definecolor{lightgray}{gray}{0.95}              % 밝은 회색
\definecolor{lightpink}{RGB}{255, 235, 245}      % 부드러운 핑크
\definecolor{boxgray}{gray}{0.95}
\definecolor{boxblue}{rgb}{0.9, 0.95, 1.0}
\definecolor{boxred}{rgb}{1.0, 0.95, 0.95}

% 진한 색상 (테두리/제목용)
\definecolor{darkblue}{RGB}{50, 80, 150}
\definecolor{darkgreen}{RGB}{40, 120, 70}
\definecolor{darkorange}{RGB}{200, 100, 30}
\definecolor{darkpurple}{RGB}{100, 60, 150}

%========================================================================================
% 박스 환경 (tcolorbox) - 6가지 타입
%========================================================================================

\usepackage[most]{tcolorbox}
\tcbuselibrary{skins, breakable}

% 1. 개요 박스 (강의 시작 부분)
\newtcolorbox{overviewbox}[1][]{
    enhanced,
    colback=lightpurple,
    colframe=darkpurple,
    fonttitle=\bfseries\large,
    title=📚 강의 개요,
    arc=3mm,
    boxrule=1pt,
    left=8pt,
    right=8pt,
    top=8pt,
    bottom=8pt,
    breakable,
    #1
}

% 2. 요약 박스
\newtcolorbox{summarybox}[1][]{
    enhanced,
    colback=lightblue,
    colframe=darkblue,
    fonttitle=\bfseries,
    title=📝 핵심 요약,
    arc=2mm,
    boxrule=0.7pt,
    left=6pt,
    right=6pt,
    top=6pt,
    bottom=6pt,
    breakable,
    #1
}

% 3. 핵심 정보 박스
\newtcolorbox{infobox}[1][]{
    enhanced,
    colback=lightgreen,
    colframe=darkgreen,
    fonttitle=\bfseries,
    title=💡 핵심 정보,
    arc=2mm,
    boxrule=0.7pt,
    left=6pt,
    right=6pt,
    top=6pt,
    bottom=6pt,
    breakable,
    #1
}

% 4. 주의사항 박스
\newtcolorbox{warningbox}[1][]{
    enhanced,
    colback=lightyellow,
    colframe=darkorange,
    fonttitle=\bfseries,
    title=⚠️ 주의사항,
    arc=2mm,
    boxrule=0.7pt,
    left=6pt,
    right=6pt,
    top=6pt,
    bottom=6pt,
    breakable,
    #1
}

% 5. 예제 박스
\newtcolorbox{examplebox}[1][]{
    enhanced,
    colback=lightgray,
    colframe=black!60,
    fonttitle=\bfseries,
    title=📖 예제: #1,
    arc=2mm,
    boxrule=0.7pt,
    left=6pt,
    right=6pt,
    top=6pt,
    bottom=6pt,
    breakable,
}

% 6. 정의 박스
\newtcolorbox{definitionbox}[1][]{
    enhanced,
    colback=lightpink,
    colframe=purple!70!black,
    fonttitle=\bfseries,
    title=📌 정의: #1,
    arc=2mm,
    boxrule=0.7pt,
    left=6pt,
    right=6pt,
    top=6pt,
    bottom=6pt,
    breakable,
}

% 7. 중요 박스 (importantbox - warningbox와 유사)
\newtcolorbox{importantbox}[1][]{
    enhanced,
    colback=boxred,
    colframe=red!70!black,
    fonttitle=\bfseries,
    title=⚠️ 매우 중요: #1,
    arc=2mm,
    boxrule=0.7pt,
    left=6pt,
    right=6pt,
    top=6pt,
    bottom=6pt,
    breakable,
}

% 8. cautionbox (warningbox와 동일)
\let\cautionbox\warningbox
\let\endcautionbox\endwarningbox

%========================================================================================
% 코드 블록 설정 (밝은 배경)
%========================================================================================

\usepackage{listings}

\definecolor{codegray}{rgb}{0.5,0.5,0.5}
\definecolor{codepurple}{rgb}{0.58,0,0.82}
\definecolor{backcolour}{rgb}{0.95,0.95,0.95}

\lstset{
    basicstyle=\ttfamily\small,
    backgroundcolor=\color{lightgray},
    keywordstyle=\color{darkblue}\bfseries,
    commentstyle=\color{darkgreen}\itshape,
    stringstyle=\color{purple!80!black},
    numberstyle=\tiny\color{black!60},
    numbers=left,
    numbersep=8pt,
    breaklines=true,
    breakatwhitespace=false,
    frame=single,
    frameround=tttt,
    rulecolor=\color{black!30},
    captionpos=b,
    showstringspaces=false,
    tabsize=2,
    xleftmargin=15pt,
    xrightmargin=5pt,
    escapeinside={\%*}{*)}
}

% Python 코드 스타일
\lstdefinestyle{pythonstyle}{
    language=Python,
    morekeywords={self, True, False, None},
}

% SQL 코드 스타일
\lstdefinestyle{sqlstyle}{
    language=SQL,
    morekeywords={SELECT, FROM, WHERE, JOIN, GROUP, BY, ORDER, HAVING},
}

%========================================================================================
% 목차 스타일링
%========================================================================================

\usepackage{tocloft}
\renewcommand{\cftsecleader}{\cftdotfill{\cftdotsep}}
\setlength{\cftbeforesecskip}{0.4em}
\renewcommand{\cftsecfont}{\bfseries}
\renewcommand{\cftsubsecfont}{\normalfont}

%========================================================================================
% 표 및 그림
%========================================================================================

\usepackage{graphicx}              % 이미지
\usepackage{adjustbox}             % 표/박스 크기 조절

% 표 캡션 스타일
\usepackage{caption}
\captionsetup[table]{
    labelfont=bf,
    textfont=it,
    skip=5pt
}
\captionsetup[figure]{
    labelfont=bf,
    textfont=it,
    skip=5pt
}

%========================================================================================
% 수학
%========================================================================================

\usepackage{amsmath, amssymb, amsthm}

% 정리 환경
\theoremstyle{definition}
\newtheorem{theorem}{정리}[section]
\newtheorem{lemma}[theorem]{보조정리}
\newtheorem{proposition}[theorem]{명제}
\newtheorem{corollary}[theorem]{따름정리}
\newtheorem{definition}{정의}[section]
\newtheorem{example}{예제}[section]

%========================================================================================
% 하이퍼링크
%========================================================================================

\usepackage[
    colorlinks=true,
    linkcolor=blue!80!black,
    urlcolor=blue!80!black,
    citecolor=green!60!black,
    bookmarks=true,
    bookmarksnumbered=true,
    pdfborder={0 0 0}
]{hyperref}

% PDF 메타데이터는 각 문서에서 설정
\hypersetup{
    pdftitle={CS109A: 데이터 과학 입문 - Lecture 22},
    pdfauthor={강의 노트},
    pdfsubject={Academic Notes}
}

%========================================================================================
% 기타 유용한 패키지
%========================================================================================

\usepackage{enumitem}              % 리스트 커스터마이징
\setlist{nosep, leftmargin=*, itemsep=0.3em}

\usepackage{microtype}             % 타이포그래피 개선
\usepackage{footnote}              % 각주 개선
\usepackage{url}                   % URL 줄바꿈
\urlstyle{same}

%========================================================================================
% 사용자 정의 명령어
%========================================================================================

% 강조 텍스트
\newcommand{\important}[1]{\textbf{\textcolor{red!70!black}{#1}}}
\newcommand{\keyword}[1]{\textbf{#1}}
\newcommand{\term}[1]{\textit{#1}}
\newcommand{\code}[1]{\texttt{#1}}

% 용어 설명 (인라인)
\newcommand{\defterm}[2]{\textbf{#1}\footnote{#2}}

% 섹션 시작 전 페이지 분리
\newcommand{\newsection}[1]{\newpage\section{#1}}

%========================================================================================
% 문서 제목 스타일
%========================================================================================

\usepackage{titling}
\pretitle{\begin{center}\LARGE\bfseries}
\posttitle{\par\end{center}\vskip 0.5em}
\preauthor{\begin{center}\large}
\postauthor{\end{center}}
\predate{\begin{center}\large}
\postdate{\par\end{center}}

%========================================================================================
% 섹션 제목 간격
%========================================================================================

\usepackage{titlesec}
\titlespacing*{\section}{0pt}{1.5em}{0.8em}
\titlespacing*{\subsection}{0pt}{1.2em}{0.6em}
\titlespacing*{\subsubsection}{0pt}{1em}{0.5em}

%========================================================================================
% 메타 정보 박스 명령어
%========================================================================================

\newcommand{\metainfo}[4]{
\begin{tcolorbox}[
    colback=lightpurple,
    colframe=darkpurple,
    boxrule=1pt,
    arc=2mm,
    left=10pt,
    right=10pt,
    top=8pt,
    bottom=8pt
]
\begin{tabular}{@{}rl@{}}
▣ \textbf{강의명:} & #1 \\[0.3em]
▣ \textbf{주차:} & #2 \\[0.3em]
▣ \textbf{교수명:} & #3 \\[0.3em]
▣ \textbf{목적:} & \begin{minipage}[t]{0.75\textwidth}#4\end{minipage}
\end{tabular}
\end{tcolorbox}
}

%========================================================================================
% 끝
%========================================================================================


\begin{document}

\metainfo{CS109A: 데이터 과학 입문}{Lecture 22}{Pavlos Protopapas, Kevin Rader, Chris Gumb}{Lecture 22의 핵심 개념 학습}

\tableofcontents
\newpage


\thispagestyle{plain}

\section*{개요: 랜덤 포레스트와 변수 중요도}

\begin{tcolorbox}[infobox, title=문서 핵심 요약]
랜덤 포레스트(\emph{Random Forest})는 배깅(\emph{Bagging})의 변형으로, 다수의 결정 트리(\emph{Decision Tree})를 훈련하고 결과를 집계하여 예측의 분산(\emph{Variance})을 줄이는 앙상블 학습 기법입니다.
핵심은 각 트리의 분할 시마다 전체 특징(\emph{Feature}, 변수)의 부분집합을 무작위로 선택하여, 트리 간의 상관관계를 낮춰 성능을 극대화하는 것입니다.
트리 간의 낮은 상관관계는 모델의 안정성(\emph{Robustness})을 높이고 과적합(\emph{Overfitting}) 위험을 줄입니다.
변수 중요도(\emph{Variable Importance})는 모델의 해석력을 높이는 수단이며, 불순도 감소량 평균(\emph{Mean Decrease in Impurity}, MDI)과 순열 중요도(\emph{Permutation Importance}) 두 가지 방법으로 계산됩니다.
또한, 불균형 데이터(\emph{Imbalanced Data}) 문제와 결측값(\emph{Missing Data}) 처리 방법도 함께 다룹니다.
\end{tcolorbox}

\newpage
\section{용어 정리}

\begin{tcolorbox}[infobox, title=주요 용어 및 직관적 설명]
\begin{table}[h]
\centering
\begin{adjustbox}{width=\textwidth,center}
\begin{tabular}{lllc}
\toprule
\textbf{용어} & \textbf{쉬운 설명} & \textbf{원어} & \textbf{비고} \\
\midrule
랜덤 포레스트 & 숲을 이루는 수많은 나무(결정 트리)의 의견을 종합하는 방법. & Random Forest & 분산 감소가 목표 \\
배깅 & 무작위로 뽑은 여러 데이터 셋으로 여러 개의 모델을 만들어 합치는 것. & Bagging (Bootstrap Aggregating) & 부트스트랩 + 집계 \\
부트스트랩 & 전체 데이터에서 복원 추출로 동일 크기의 표본을 여러 개 만드는 것. & Bootstrap & 데이터 무작위화 \\
불순도 감소량 평균 & 어떤 특징이 트리를 얼마나 '순수'하게 만드는지(불순도를 얼마나 줄이는지) 평균. & Mean Decrease in Impurity (MDI) & 훈련 과정에서 측정 \\
순열 중요도 & 특정 특징을 무작위로 섞었을 때 모델 성능이 얼마나 떨어지는지 측정. & Permutation Importance & 성능 변화 기반으로 측정 \\
하이퍼파라미터 & 모델 훈련 전에 사람이 직접 설정하는 값 (예: 트리의 개수, 최대 깊이). & Hyperparameter & 교차 검증으로 최적화 \\
OOB 오류 & 부트스트랩 시 사용되지 않은 데이터를 검증에 사용하여 계산한 오류. & Out-of-Bag (OOB) Error & 교차 검증을 대체 \\
서로게이트 분할 & 결측값이 있는 주 예측 변수 대신 사용할 수 있는 대리 예측 변수. & Surrogate Split & 결정 트리의 결측값 처리 방법 \\
\bottomrule
\end{tabular}
\end{adjustbox}
\caption{랜덤 포레스트 관련 주요 용어 정리}
\label{tab:terms}
\end{table}
\end{tcolorbox}

\newpage
\section{핵심 개념: 랜덤 포레스트의 원리}

\subsection{배깅(\emph{Bagging})의 한계: 트리의 상관관계}

배깅은 부트스트랩 샘플(\emph{Bootstrap Sample})을 사용하여 다수의 결정 트리를 학습하고 그 결과를 집계(\emph{Aggregate})하여 분산을 줄이는 앙상블 기법입니다.
그러나 배깅에서 생성된 트리는 서로 \textbf{상관관계(\emph{Correlation})가 높게} 나타나는 경향이 있습니다.

\begin{tcolorbox}[examplebox, title=상관관계가 높은 이유에 대한 비유]
강력한 예측 변수(\emph{Strong Predictor})가 있는 경우, 모든 부트스트랩 샘플에서도 이 예측 변수가 가장 높은 불순도 감소 효과를 보일 것입니다.
마치 모든 의사(\emph{Estimator})가 동일한 환자 데이터를 기반으로 훈련받고, '휴식, 수분 섭취'와 같은 가장 확실한 조언을 첫 번째 조치로 내리는 것과 같습니다.
따라서, 대부분의 트리는 \textbf{최상위 노드(\emph{Root Node})}에서 동일한 특징으로 분할을 시작하게 되어, 트리가 서로 비슷한 구조와 예측을 하게 됩니다.
이러한 상관관계는 앙상블이 기대하는 분산 감소 효과($1/\sqrt{n}$)를 약화시킵니다.
\end{tcolorbox}

\subsection{랜덤 포레스트: 비상관화(\emph{Decorrelation})의 도입}

랜덤 포레스트는 배깅을 개선하여 트리 간의 상관관계를 낮추는 방법입니다.
트리의 상관관계를 낮추는 것이 분산(\emph{Variance})을 효과적으로 줄이는 핵심입니다.

\begin{tcolorbox}[infobox, title=랜덤 포레스트의 핵심 아이디어]
\textbf{각 분할 노드(\emph{Split Node})마다, 전체 특징 $J$개 중 무작위로 선택된 $J' < J$개의 특징 부분집합만 고려합니다.}
즉, 트리가 자라날 때마다 최고의 특징을 선택할 때 \textbf{전체 특징이 아닌} 일부 특징 중에서만 선택하게 강제합니다.
이로 인해 강력한 특징이 매번 선택될 확률이 낮아지고, 결과적으로 트리가 다양해지고 서로 덜 유사해집니다.
\end{tcolorbox}

\subsubsection*{랜덤 포레스트 요약 단계}
\begin{enumerate}
    \item $B$개의 부트스트랩 데이터셋을 생성합니다 (배깅과 동일).
    \item $B$개의 결정 트리를 초기화합니다.
    \item 각 트리 내의 \textbf{각 분할(\emph{Split})}마다:
    \begin{enumerate}
        \item 전체 특징 $J$개 중 \textbf{무작위로} $J'$개의 특징 부분집합을 선택합니다 ($J' < J$).
        \item 이 $J'$개의 특징 중에서 최적의 특징과 최적의 임계값(\emph{Threshold})를 선택하여 분할합니다.
    \end{enumerate}
    \item 최종적으로 모든 트리의 예측을 집계합니다 (분류는 다수결, 회귀는 평균).
\end{enumerate}
이때 $J'$는 각 분할마다 \textbf{새롭게} 무작위로 선택됩니다.

\subsection{하이퍼파라미터 튜닝}

랜덤 포레스트는 여러 개의 하이퍼파라미터를 가지며, 이는 모델의 성능과 훈련 속도에 영향을 미칩니다.

\begin{tcolorbox}[infobox, title=주요 하이퍼파라미터]
\begin{itemize}
    \item \textbf{분할 시 무작위 선택할 특징의 개수 $J'$}: 트리 간의 상관관계를 조절하는 가장 중요한 파라미터입니다.
    \item \textbf{앙상블 내 트리의 총 개수 $B$}: 분산 감소량을 결정합니다. 트리가 많을수록 분산은 줄어들지만 계산 시간이 늘어납니다.
    \item \textbf{트리 정지 조건}: 최대 깊이(\emph{Maximum Depth}), 최소 리프 노드 크기(\emph{Minimum Leaf Node Size}) 등.
    \item \textbf{분할 기준}: 지니 불순도(\emph{Gini Impurity}) 또는 엔트로피(\emph{Entropy}).
\end{itemize}
\end{tcolorbox}

\subsubsection*{하이퍼파라미터 튜닝과 전문가의 권장 사항}
\begin{enumerate}
    \item \textbf{최적의 $J'$ 선택}: 교차 검증(\emph{Cross-Validation})을 사용해야 하지만, 일반적인 경험 법칙(\emph{Rule of Thumb})이 있습니다.
    \begin{itemize}
        \item \textbf{분류(\emph{Classification})}: $\sqrt{N_j}$ (전체 특징 개수의 제곱근)
        \item \textbf{회귀(\emph{Regression})}: $N_j/3$
    \end{itemize}
    \item \textbf{트리의 개수 $B$}: OOB 오류가 더 이상 감소하지 않고 안정화되는 지점까지 늘립니다. 트리의 개수는 \textbf{과적합(\emph{Overfitting})을 유발하지 않습니다}. 단지 분산만 감소시킬 뿐입니다.
    \item \textbf{정지 조건 및 기준}: 보통 최대 깊이(\emph{Maximum Depth})나 지니 불순도(\emph{Gini})를 기본값으로 사용하고, 모델이 작동한 후에 더 세밀한 튜닝을 시도할 수 있습니다.
\end{enumerate}

\subsubsection*{OOB 오류를 활용한 검증}
랜덤 포레스트는 OOB(Out-of-Bag) 샘플, 즉 부트스트랩 과정에서 특정 트리 훈련에 사용되지 않은 데이터 포인트들을 사용하여 \textbf{별도의 검증 세트 없이} 모델을 평가할 수 있습니다. 이것이 교차 검증을 대체할 수 있는 효율적인 방법입니다.

\newpage
\section{변수 중요도(\emph{Variable Importance}) 평가}

앙상블 모델은 단일 결정 트리처럼 쉬운 규칙(\emph{Rule}) 형태로 해석하기 어렵기 때문에, 어떤 특징이 예측에 가장 큰 영향을 미쳤는지 평가하여 모델의 \textbf{해석력(\emph{Interpretability})}을 높여야 합니다.

\subsection{1. 불순도 감소량 평균(\emph{Mean Decrease in Impurity}, MDI)}

\begin{tcolorbox}[infobox]
MDI는 특정 특징이 트리의 불순도(예: 지니 불순도)를 평균적으로 얼마나 감소시켰는지를 측정하여 중요도를 산출하는 방법입니다.
\end{tcolorbox}

\subsubsection*{MDI 계산 절차}
\begin{enumerate}
    \item \textbf{노드별 불순도 감소량 계산}: 단일 트리 내의 각 노드 $q$에서 분할로 인한 불순도 감소량 $\Delta I_q$를 계산합니다.
    $$\Delta I_{q}=(\frac{n}{N})\left[Gini_{n}-\sum_{m\in \text{children}}(\frac{m}{n})Gini_{m}\right]$$
    ($n$: 노드의 샘플 수, $N$: 전체 데이터 샘플 수, $m$: 자식 노드의 샘플 수)
    \item \textbf{특징별 중요도 합산}: 특정 특징 $j$가 사용된 모든 노드 $n$의 불순도 감소량을 합하여 해당 특징의 중요도 $F_j^{(t)}$를 계산합니다.
    \item \textbf{정규화}: 중요도 합계를 모든 특징의 중요도 합계로 나누어 0에서 1 사이 값으로 정규화합니다.
    \item \textbf{앙상블 평균}: 모든 트리 $T$에 대해 정규화된 특징 중요도 $\hat{F}_j^{(t)}$를 평균하여 최종 MDI 중요도를 구합니다.
    $$\mathcal{F}_{j}=\frac{\sum_{t}\hat{F}_{j}^{(t)}}{T}$$
\end{enumerate}

\subsubsection*{MDI의 장단점}
\begin{itemize}
    \item \textbf{장점}: 계산 속도가 빠릅니다. 모든 필요한 값은 랜덤 포레스트 훈련 중에 계산됩니다.
    \item \textbf{단점}: 수치형 특징(\emph{Numerical Feature})이나 범주형 특징 중 고유값이 많은 특징(\emph{High Cardinality Categorical Feature})에 \textbf{편향(\emph{Bias})}되어 중요도를 과대평가하는 경향이 있습니다. 이들은 분할 지점(\emph{Split Point})이 많기 때문에 우연히 불순도를 크게 감소시키는 분할을 찾을 가능성이 높습니다.
\end{itemize}

\subsection{2. 순열 중요도(\emph{Permutation Importance})}

\begin{tcolorbox}[infobox]
순열 중요도는 특징의 값들을 무작위로 섞어(\emph{Permute}) 해당 특징과 결과 변수 간의 관계를 끊었을 때, 모델의 검증/OOB 성능이 얼마나 하락하는지를 측정하여 중요도를 산출하는 방법입니다.
\end{tcolorbox}

\subsubsection*{순열 중요도 계산 절차}
\begin{enumerate}
    \item \textbf{기준 성능 기록}: 원본 데이터셋에서의 OOB 또는 검증 정확도 $s$를 기록합니다.
    \item \textbf{특징 순열}: 관심 있는 특징 $j$의 데이터 열을 무작위로 섞습니다.
    \item \textbf{순열 성능 측정}: 섞인 데이터셋으로 모델을 실행하여 새로운 OOB/검증 정확도 $s_{k,j}$를 기록합니다.
    \item \textbf{반복 및 평균}: 2, 3단계를 $K$번 반복하고 평균 순열 정확도 $s_j$를 계산합니다.
    $$s_{j}=\frac{1}{K}\sum_{k=1}^{K}s_{k,j}$$
    \item \textbf{중요도 산출}: 기준 성능과 평균 순열 성능의 차이를 계산하여 중요도를 산출합니다.
    $$\text{특징 중요도} = s - s_{j}$$
\end{enumerate}

\subsubsection*{순열 중요도의 장단점}
\begin{itemize}
    \item \textbf{장점}: MDI의 편향이 없으며, 실제 모델 성능에 미치는 영향을 직접적으로 측정하므로 \textbf{더 직관적이고 신뢰할 만한} 중요도를 제공합니다.
    \item \textbf{단점}: MDI보다 계산 비용이 더 많이 듭니다. (각 특징마다 $K$번의 예측이 필요)
\end{itemize}

\subsection{랜덤 포레스트 vs. 배깅 변수 중요도 비교}
트리 간 상관관계가 높은 \textbf{배깅(\emph{Bagging})}은 소수의 강력한 예측 변수(예: \texttt{ChestPain}, \texttt{Ca})에 중요도가 \textbf{집중}되는 경향을 보입니다.
반면, 비상관화 과정을 거친 \textbf{랜덤 포레스트(\emph{Random Forest})}의 중요도는 여러 특징에 걸쳐 \textbf{더 부드럽고 분산된} 분포를 보여줍니다. 이는 트리들이 다양한 특징을 고려하도록 강제되었기 때문입니다.

\begin{tcolorbox}[cautionbox, title=랜덤 포레스트가 잘 작동하지 않는 경우]
전체 특징의 수가 매우 많지만, \textbf{실제로 중요한 특징의 수가 적을 때} (예: 1000개의 특징 중 10개만 중요), 랜덤 포레스트는 오히려 성능이 떨어질 수 있습니다.
각 분할에서 무작위로 선택된 특징 부분집합에 중요한 특징이 포함되지 않을 확률이 높아지기 때문에, 생성된 트리들이 대부분 '약한 모델'이 될 수 있습니다.
이 경우 PCA나 다른 특징 선택 기법을 사용해 불필요한 특징을 미리 제거하는 것이 좋습니다.
\end{tcolorbox}

\newpage
\section{불균형 데이터(\emph{Class Imbalance}) 처리}

데이터셋에서 특정 클래스(소수 클래스)의 샘플 수가 다른 클래스(다수 클래스)에 비해 현저히 적을 때 발생하는 문제입니다. 이 경우 모델이 다수 클래스만 예측해도 높은 '정확도'를 얻을 수 있어, 실제로는 소수 클래스 예측 능력이 매우 낮아집니다.

\begin{tcolorbox}[cautionbox, title=불균형 데이터 문제 진단]
\textbf{정확도(\emph{Accuracy})는 좋은 평가 지표가 아닙니다.}
99\% 대 1\%의 클래스 불균형에서 모델이 항상 99\%의 다수 클래스만 예측해도 정확도는 99\%가 나옵니다.
따라서, 불균형 데이터셋에서는 \textbf{F1-점수}나 \textbf{ROC 곡선 하 면적(\emph{AUC})}과 같은 지표를 사용해야 합니다.
\end{tcolorbox}

\subsection{처리 방법}

\begin{enumerate}
    \item \textbf{언더샘플링(\emph{Under-sampling}): 다수 클래스 샘플 수 감소}
    \begin{itemize}
        \item \textbf{무작위 언더샘플링}: 다수 클래스에서 무작위로 샘플을 제거하여 클래스 균형을 맞춥니다. (\emph{단점}: 중요한 정보를 포함하는 샘플이 손실될 수 있습니다.)
        \item \textbf{니어 미스(\emph{Near Miss})}: 결정 경계(\emph{Decision Boundary})에서 멀리 떨어진, 즉 분류에 덜 유익한 다수 클래스 샘플을 제거하여 정보 손실을 최소화합니다. (경계 근처의 샘플은 보존)
    \end{itemize}

    \item \textbf{오버샘플링(\emph{Over-sampling}): 소수 클래스 샘플 수 증가}
    \begin{itemize}
        \item \textbf{무작위 오버샘플링}: 소수 클래스의 샘플을 복원 추출(\emph{with replacement})하여 수를 늘립니다. (\emph{단점}: 단순 복제이므로 과적합을 유발하거나 데이터의 다양성을 해칠 수 있습니다.)
        \item \textbf{SMOTE (\emph{Synthetic Minority Oversampling Technique})}: 소수 클래스 샘플 주변에 \textbf{새로운 합성 데이터(\emph{Synthetic Data})}를 생성하여 수를 늘립니다. 데이터 영역을 확장하여 모델의 일반화 성능을 높입니다.
    \end{itemize}

    \item \textbf{클래스 가중치(\emph{Class Weighting})}
    \begin{itemize}
        \item 모델의 손실 함수(\emph{Loss Function})에서 소수 클래스의 오류에 \textbf{더 높은 가중치}를 부여하여 학습 시 소수 클래스에 더 많은 관심을 기울이도록 합니다.
        \item 사이킷런(\emph{scikit-learn})에서는 \texttt{class\_weight='balanced'} 옵션을 통해 자동으로 클래스 빈도에 반비례하여 가중치를 조정할 수 있습니다.
        $$W_{K}=\frac{N}{K \times N_{K}}$$
        ($N$: 전체 샘플 수, $K$: 클래스 수, $N_K$: 클래스 $K$의 샘플 수)
    \end{itemize}
\end{enumerate}

\begin{tcolorbox}[examplebox, title=실제 사용 전략]
데이터 균형을 맞출 때에는 세 가지 방법(언더샘플링, 오버샘플링, 가중치) 중 하나 또는 그 조합을 사용합니다. 예를 들어, 오버샘플링과 언더샘플링을 결합하거나, 데이터 재조정 후 클래스 가중치를 적용할 수 있습니다. 중요한 것은 \textbf{항상 데이터를 균형 있게 처리}하는 것입니다 (40\%/60\% 불균형에서도 적용 권장).
\end{tcolorbox}

\newpage
\section{결측값(\emph{Missing Data}) 처리: 서러게이트 분할}

결정 트리 기반 모델은 결측값을 처리하는 독특한 방법인 \textbf{서로게이트 분할(\emph{Surrogate Split})}을 사용할 수 있습니다.

\subsection{서로게이트 분할의 개념}

\begin{tcolorbox}[infobox]
서로게이트 분할은 트리를 훈련하는 과정에서 \textbf{최적의 분할 특징(\emph{Optimal Splitter})}의 결측값(\emph{Missing Value})을 대신할 수 있는 \textbf{차선책의 특징(\emph{Alternative Feature})}을 미리 찾아 순위를 매겨 놓는 방법입니다.
\end{tcolorbox}

\subsubsection*{작동 원리}
\begin{enumerate}
    \item \textbf{최적 분할 특징 선택}: 노드 $N$에서 불순도를 가장 크게 줄이는 최적의 분할 특징 $P_{opt}$를 찾습니다.
    \item \textbf{분할 분포 기록}: $P_{opt}$를 사용해 노드를 분할했을 때, 각 자식 노드로 이동하는 반응 변수(\emph{Response Variable})의 분포(예: \texttt{Yes}/\texttt{No} 개수)를 기록합니다.
    \item \textbf{대리 특징 찾기}: 나머지 모든 특징 $P_i$에 대해 $P_{opt}$와 동일한 분할을 시도했을 때, 반응 변수의 분포가 $P_{opt}$의 분포와 \textbf{가장 유사한} 특징을 찾습니다.
    \item \textbf{유사도 측정}: 두 특징의 분할 분포가 얼마나 유사한지(즉, 분포를 일치시키기 위해 얼마나 많은 '플립'이 필요한지)를 측정하여 유사도를 산출하고 순위를 매깁니다.
    \item \textbf{예측 시 사용}: 실제 예측 시점에 입력 데이터의 $P_{opt}$ 값이 결측이면, 미리 정의된 순위에 따라 가장 유사한 서러게이트 분할 특징을 대신 사용하여 데이터를 분할합니다.
\end{enumerate}

\begin{tcolorbox}[examplebox, title=서러게이트 분할 예시]
주요 특징이 '\texttt{Arteries Blocked}'라고 가정합니다. 이 특징으로 분할했을 때, 환자들의 '심장병 유무' 분포가 나옵니다.
\begin{itemize}
    \item \textbf{최적 분할 특징 (\texttt{Arteries Blocked})}: \texttt{FALSE} $\rightarrow$ [3 \texttt{No}, 1 \texttt{Yes}], \texttt{TRUE} $\rightarrow$ [0 \texttt{No}, 2 \texttt{Yes}]
    \item \textbf{대리 특징 후보 (\texttt{Chest Congested})}: \texttt{FALSE} $\rightarrow$ [3 \texttt{No}, 2 \texttt{Yes}], \texttt{TRUE} $\rightarrow$ [0 \texttt{No}, 1 \texttt{Yes}]
    \item 두 분포를 비교하여 유사도를 측정합니다. 이 예시에서는 '\texttt{Chest Congested}'가 '\texttt{Arteries Blocked}'와 가장 유사한 분할 분포를 보였으므로 최적의 서러게이트가 됩니다.
\end{itemize}
따라서, 예측할 데이터에서 '\texttt{Arteries Blocked}' 값이 결측이면, 대신 '\texttt{Chest Congested}' 특징을 사용하여 트리를 따라 내려갑니다.
\end{tcolorbox}

\subsection{서러게이트 분할의 이점}
\begin{itemize}
    \item \textbf{해석력 향상}: 서러게이트는 최적 분할 특징과 비슷한 역할을 하는 보조 특징을 보여주므로, 주 분할기의 작동 방식을 이해하는 데 도움이 됩니다.
    \item \textbf{다중 공선성(\emph{Multi-collinearity}) 활용}: 다중 공선성이 있는 경우(특징들이 서로 높은 상관관계를 가지는 경우), 대체할 서러게이트를 찾을 가능성이 높고 성능도 좋습니다.
    \item \textbf{명시적 대체 불필요}: 데이터 분석가가 별도의 결측값 대체(\emph{Imputation}) 방법을 고민할 필요 없이, 트리가 자체적으로 결측값을 처리할 수 있습니다.
\end{itemize}

\newpage
\section{FAQ 및 초심자 체크리스트}

\subsection{자주 묻는 질문 (FAQ)}

\begin{enumerate}
    \item Q: 랜덤 포레스트에서 트리의 개수(\emph{Number of Trees})가 많아지면 과적합되나요?
    \\
    A: \textbf{아닙니다.} 트리의 개수를 늘리는 것은 \textbf{분산(\emph{Variance})}만 줄이는 역할을 합니다. 이는 앙상블의 예측을 안정화할 뿐이며, 다른 하이퍼파라미터처럼 모델의 복잡도(\emph{Complexity})를 제어하지 않으므로 과적합 위험을 증가시키지 않습니다.

    \item Q: 랜덤 포레스트는 어떻게 클래스 예측 확률(\emph{Class Probabilities})을 반환하나요?
    \\
    A: 각 개별 트리는 최종적으로 클래스 예측 결과를 내놓습니다. 랜덤 포레스트 분류기(\emph{Classifier})는 모든 트리가 예측한 \textbf{클래스 예측의 평균}을 계산하여 이를 확률의 근사치(\emph{Proxy})로 사용합니다. 이 값을 임계값(\emph{Threshold})과 결합하여 ROC 곡선 등을 그릴 수 있습니다.

    \item Q: MDI와 순열 중요도 중 어떤 것을 사용해야 하나요?
    \\
    A: \textbf{순열 중요도(\emph{Permutation Importance})}가 더 신뢰할 수 있습니다. MDI는 계산이 빠르다는 장점이 있지만, 수치형이나 고유값이 많은 범주형 특징에 편향되어 중요도를 과대평가하는 경향이 있습니다. 순열 중요도는 모델의 실제 성능 변화를 측정하므로 더 직관적이고 정확합니다.

    \item Q: 결정 트리는 단일 모델로 불안정한데, 왜 배깅이나 랜덤 포레스트에서는 안정적인가요?
    \\
    A: 단일 결정 트리는 훈련 데이터의 작은 변화에도 민감하게 반응하여 매우 다른 구조로 학습될 수 있습니다(\textbf{고분산}). 앙상블은 부트스트랩을 통해 생성된 다양한 불안정한 트리를 평균하여, 그 '노이즈'로 인한 예측의 변동성(\emph{Variability})을 상쇄하고 \textbf{평균적인 안정적인 예측}을 얻습니다.
\end{enumerate}

\subsection{학습 및 구현 체크리스트}

\begin{tcolorbox}[infobox, title=랜덤 포레스트 구현 전 점검 사항]
\begin{itemize}
    \item \textbf{핵심 원리 이해}: 랜덤 포레스트가 배깅과의 차이점(특징 무작위 부분집합 선택)과 분산 감소 메커니즘을 명확히 설명할 수 있는가?
    \item \textbf{하이퍼파라미터 튜닝}: 가장 중요한 하이퍼파라미터 $J'$와 $B$의 역할 및 권장 기본값($\sqrt{N_j}$, $N_j/3$)을 아는가?
    \item \textbf{평가 지표 선택}: 데이터 불균형 여부를 확인하고, 불균형 시 F1-점수 또는 AUC를 주 평가 지표로 사용하는가?
    \item \textbf{불균형 처리}: 언더샘플링(Near Miss)이나 오버샘플링(SMOTE) 또는 클래스 가중치 중 최소 한 가지 방법을 적용하여 데이터를 균형 있게 조정했는가?
    \item \textbf{변수 중요도 평가}: MDI의 편향성을 인지하고, 가능하면 순열 중요도를 사용하여 특징의 기여도를 평가했는가?
    \item \textbf{성능 비교}: 단일 결정 트리, 배깅, 랜덤 포레스트의 RMSE 또는 정확도를 비교하여 분산 감소 효과를 확인했는가?
\end{itemize}
\end{tcolorbox}

\end{document}
