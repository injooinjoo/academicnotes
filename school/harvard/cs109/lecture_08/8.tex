%%%%%%%%%%%%%%%%%%%%%%%%%%%%%%%%%%%%%%%%%%%%%%%%%%%%%%%%%%%%%%%%%%%%%%%%%%%%%%%
% Harvard Academic Notes - 통합 마스터 템플릿
% 모든 강의 노트에 적용되는 통일된 스타일
% 버전: 2.1 - 가독성 개선 (선택적 최적화)
% 최종 수정일: 2025-11-17
%%%%%%%%%%%%%%%%%%%%%%%%%%%%%%%%%%%%%%%%%%%%%%%%%%%%%%%%%%%%%%%%%%%%%%%%%%%%%%%

\documentclass[11pt,a4paper]{article}

%========================================================================================
% 기본 패키지
%========================================================================================

% --- 한국어 지원 ---
\usepackage{kotex}

% --- 페이지 레이아웃 ---
\usepackage[top=20mm, bottom=20mm, left=20mm, right=18mm]{geometry}
\usepackage{setspace}
\onehalfspacing                      % 1.5배 줄간격
\setlength{\parskip}{0.5em}          % 문단 간격
\setlength{\parindent}{0pt}          % 들여쓰기 없음

% --- 표 관련 ---
\usepackage{booktabs}              % 고품질 표
\usepackage{tabularx}              % 자동 너비 조절 표
\usepackage{array}                 % 표 컬럼 확장
\usepackage{longtable}             % 여러 페이지 표
\renewcommand{\arraystretch}{1.1}  % 표 행간 조절

%========================================================================================
% 헤더 및 푸터
%========================================================================================

\usepackage{fancyhdr}
\pagestyle{fancy}
\fancyhf{}
\fancyhead[L]{\small\textit{CS109A: 데이터 과학 입문}}
\fancyhead[R]{\small\textit{Lecture 08}}
\fancyfoot[C]{\thepage}
\renewcommand{\headrulewidth}{0.5pt}
\renewcommand{\footrulewidth}{0.3pt}

% 첫 페이지는 헤더 없음
\fancypagestyle{firstpage}{
    \fancyhf{}
    \fancyfoot[C]{\thepage}
    \renewcommand{\headrulewidth}{0pt}
}

%========================================================================================
% 색상 정의 (파스텔 톤 + 다크모드 호환)
%========================================================================================

\usepackage[dvipsnames]{xcolor}

% 밝은 배경용 파스텔 색상
\definecolor{lightblue}{RGB}{220, 235, 255}      % 부드러운 파랑
\definecolor{lightgreen}{RGB}{220, 255, 235}     % 부드러운 초록
\definecolor{lightyellow}{RGB}{255, 250, 220}    % 부드러운 노랑
\definecolor{lightpurple}{RGB}{240, 230, 255}    % 부드러운 보라
\definecolor{lightgray}{gray}{0.95}              % 밝은 회색
\definecolor{lightpink}{RGB}{255, 235, 245}      % 부드러운 핑크
\definecolor{boxgray}{gray}{0.95}
\definecolor{boxblue}{rgb}{0.9, 0.95, 1.0}
\definecolor{boxred}{rgb}{1.0, 0.95, 0.95}

% 진한 색상 (테두리/제목용)
\definecolor{darkblue}{RGB}{50, 80, 150}
\definecolor{darkgreen}{RGB}{40, 120, 70}
\definecolor{darkorange}{RGB}{200, 100, 30}
\definecolor{darkpurple}{RGB}{100, 60, 150}

%========================================================================================
% 박스 환경 (tcolorbox) - 6가지 타입
%========================================================================================

\usepackage[most]{tcolorbox}
\tcbuselibrary{skins, breakable}

% 1. 개요 박스 (강의 시작 부분)
\newtcolorbox{overviewbox}[1][]{
    enhanced,
    colback=lightpurple,
    colframe=darkpurple,
    fonttitle=\bfseries\large,
    title=📚 강의 개요,
    arc=3mm,
    boxrule=1pt,
    left=8pt,
    right=8pt,
    top=8pt,
    bottom=8pt,
    breakable,
    #1
}

% 2. 요약 박스
\newtcolorbox{summarybox}[1][]{
    enhanced,
    colback=lightblue,
    colframe=darkblue,
    fonttitle=\bfseries,
    title=📝 핵심 요약,
    arc=2mm,
    boxrule=0.7pt,
    left=6pt,
    right=6pt,
    top=6pt,
    bottom=6pt,
    breakable,
    #1
}

% 3. 핵심 정보 박스
\newtcolorbox{infobox}[1][]{
    enhanced,
    colback=lightgreen,
    colframe=darkgreen,
    fonttitle=\bfseries,
    title=💡 핵심 정보,
    arc=2mm,
    boxrule=0.7pt,
    left=6pt,
    right=6pt,
    top=6pt,
    bottom=6pt,
    breakable,
    #1
}

% 4. 주의사항 박스
\newtcolorbox{warningbox}[1][]{
    enhanced,
    colback=lightyellow,
    colframe=darkorange,
    fonttitle=\bfseries,
    title=⚠️ 주의사항,
    arc=2mm,
    boxrule=0.7pt,
    left=6pt,
    right=6pt,
    top=6pt,
    bottom=6pt,
    breakable,
    #1
}

% 5. 예제 박스
\newtcolorbox{examplebox}[1][]{
    enhanced,
    colback=lightgray,
    colframe=black!60,
    fonttitle=\bfseries,
    title=📖 예제: #1,
    arc=2mm,
    boxrule=0.7pt,
    left=6pt,
    right=6pt,
    top=6pt,
    bottom=6pt,
    breakable,
}

% 6. 정의 박스
\newtcolorbox{definitionbox}[1][]{
    enhanced,
    colback=lightpink,
    colframe=purple!70!black,
    fonttitle=\bfseries,
    title=📌 정의: #1,
    arc=2mm,
    boxrule=0.7pt,
    left=6pt,
    right=6pt,
    top=6pt,
    bottom=6pt,
    breakable,
}

% 7. 중요 박스 (importantbox - warningbox와 유사)
\newtcolorbox{importantbox}[1][]{
    enhanced,
    colback=boxred,
    colframe=red!70!black,
    fonttitle=\bfseries,
    title=⚠️ 매우 중요: #1,
    arc=2mm,
    boxrule=0.7pt,
    left=6pt,
    right=6pt,
    top=6pt,
    bottom=6pt,
    breakable,
}

% 8. cautionbox (warningbox와 동일)
\let\cautionbox\warningbox
\let\endcautionbox\endwarningbox

%========================================================================================
% 코드 블록 설정 (밝은 배경)
%========================================================================================

\usepackage{listings}

\definecolor{codegray}{rgb}{0.5,0.5,0.5}
\definecolor{codepurple}{rgb}{0.58,0,0.82}
\definecolor{backcolour}{rgb}{0.95,0.95,0.95}

\lstset{
    basicstyle=\ttfamily\small,
    backgroundcolor=\color{lightgray},
    keywordstyle=\color{darkblue}\bfseries,
    commentstyle=\color{darkgreen}\itshape,
    stringstyle=\color{purple!80!black},
    numberstyle=\tiny\color{black!60},
    numbers=left,
    numbersep=8pt,
    breaklines=true,
    breakatwhitespace=false,
    frame=single,
    frameround=tttt,
    rulecolor=\color{black!30},
    captionpos=b,
    showstringspaces=false,
    tabsize=2,
    xleftmargin=15pt,
    xrightmargin=5pt,
    escapeinside={\%*}{*)}
}

% Python 코드 스타일
\lstdefinestyle{pythonstyle}{
    language=Python,
    morekeywords={self, True, False, None},
}

% SQL 코드 스타일
\lstdefinestyle{sqlstyle}{
    language=SQL,
    morekeywords={SELECT, FROM, WHERE, JOIN, GROUP, BY, ORDER, HAVING},
}

%========================================================================================
% 목차 스타일링
%========================================================================================

\usepackage{tocloft}
\renewcommand{\cftsecleader}{\cftdotfill{\cftdotsep}}
\setlength{\cftbeforesecskip}{0.4em}
\renewcommand{\cftsecfont}{\bfseries}
\renewcommand{\cftsubsecfont}{\normalfont}

%========================================================================================
% 표 및 그림
%========================================================================================

\usepackage{graphicx}              % 이미지
\usepackage{adjustbox}             % 표/박스 크기 조절

% 표 캡션 스타일
\usepackage{caption}
\captionsetup[table]{
    labelfont=bf,
    textfont=it,
    skip=5pt
}
\captionsetup[figure]{
    labelfont=bf,
    textfont=it,
    skip=5pt
}

%========================================================================================
% 수학
%========================================================================================

\usepackage{amsmath, amssymb, amsthm}

% 정리 환경
\theoremstyle{definition}
\newtheorem{theorem}{정리}[section]
\newtheorem{lemma}[theorem]{보조정리}
\newtheorem{proposition}[theorem]{명제}
\newtheorem{corollary}[theorem]{따름정리}
\newtheorem{definition}{정의}[section]
\newtheorem{example}{예제}[section]

%========================================================================================
% 하이퍼링크
%========================================================================================

\usepackage[
    colorlinks=true,
    linkcolor=blue!80!black,
    urlcolor=blue!80!black,
    citecolor=green!60!black,
    bookmarks=true,
    bookmarksnumbered=true,
    pdfborder={0 0 0}
]{hyperref}

% PDF 메타데이터는 각 문서에서 설정
\hypersetup{
    pdftitle={CS109A: 데이터 과학 입문 - Lecture 08},
    pdfauthor={강의 노트},
    pdfsubject={Academic Notes}
}

%========================================================================================
% 기타 유용한 패키지
%========================================================================================

\usepackage{enumitem}              % 리스트 커스터마이징
\setlist{nosep, leftmargin=*, itemsep=0.3em}

\usepackage{microtype}             % 타이포그래피 개선
\usepackage{footnote}              % 각주 개선
\usepackage{url}                   % URL 줄바꿈
\urlstyle{same}

%========================================================================================
% 사용자 정의 명령어
%========================================================================================

% 강조 텍스트
\newcommand{\important}[1]{\textbf{\textcolor{red!70!black}{#1}}}
\newcommand{\keyword}[1]{\textbf{#1}}
\newcommand{\term}[1]{\textit{#1}}
\newcommand{\code}[1]{\texttt{#1}}

% 용어 설명 (인라인)
\newcommand{\defterm}[2]{\textbf{#1}\footnote{#2}}

% 섹션 시작 전 페이지 분리
\newcommand{\newsection}[1]{\newpage\section{#1}}

%========================================================================================
% 문서 제목 스타일
%========================================================================================

\usepackage{titling}
\pretitle{\begin{center}\LARGE\bfseries}
\posttitle{\par\end{center}\vskip 0.5em}
\preauthor{\begin{center}\large}
\postauthor{\end{center}}
\predate{\begin{center}\large}
\postdate{\par\end{center}}

%========================================================================================
% 섹션 제목 간격
%========================================================================================

\usepackage{titlesec}
\titlespacing*{\section}{0pt}{1.5em}{0.8em}
\titlespacing*{\subsection}{0pt}{1.2em}{0.6em}
\titlespacing*{\subsubsection}{0pt}{1em}{0.5em}

%========================================================================================
% 메타 정보 박스 명령어
%========================================================================================

\newcommand{\metainfo}[4]{
\begin{tcolorbox}[
    colback=lightpurple,
    colframe=darkpurple,
    boxrule=1pt,
    arc=2mm,
    left=10pt,
    right=10pt,
    top=8pt,
    bottom=8pt
]
\begin{tabular}{@{}rl@{}}
▣ \textbf{강의명:} & #1 \\[0.3em]
▣ \textbf{주차:} & #2 \\[0.3em]
▣ \textbf{교수명:} & #3 \\[0.3em]
▣ \textbf{목적:} & \begin{minipage}[t]{0.75\textwidth}#4\end{minipage}
\end{tabular}
\end{tcolorbox}
}

%========================================================================================
% 끝
%========================================================================================


\begin{document}

\maketitle
\thispagestyle{firstpage}

\metainfo{CS109A: 데이터 과학 입문}{Lecture 08}{Pavlos Protopapas, Kevin Rader, Chris Gumb}{Lecture 08의 핵심 개념 학습}


\begin{summarybox}
\noindent
이 문서는 선형 회귀 모델의 단순한 예측을 넘어, 우리가 얻은 모델이 얼마나 신뢰할 수 있는지(정확성), 모델에 포함된 변수들이 실제로 의미가 있는지(유의성), 그리고 모델의 예측이 얼마나 확실한지(예측 구간)를 평가하는 통계적 추론 방법을 다룹니다.

데이터에 존재하는 불확실성을 이해하고, '부트스트래핑'이라는 시뮬레이션 기법을 통해 회귀 계수($\hat{\beta}$)의 분포를 추정합니다. 이를 바탕으로 신뢰구간을 계산하고, $\hat{t}$-검정 및 p-값을 이용해 각 예측 변수의 중요도와 통계적 유의성을 검증하는 방법을 배웁니다. 마지막으로 모델의 '평균 예측'에 대한 신뢰구간과 '개별 예측'에 대한 예측구간의 차이점을 명확히 구분합니다.
\end{summarybox}

\tableofcontents

\newpage

% ======================================================
\section{개요: 왜 '추론'이 필요한가?}
% ======================================================

선형 회귀 모델을 학습시키면 $\hat{y} = \hat{\beta}_0 + \hat{\beta}_1 x$와 같은 식을 얻습니다. 예를 들어, TV 광고 예산($x$)과 매출($y$) 간의 관계가 $\hat{y} = 1.01x + 0.005$라고 가정해 봅시다.

이 식의 해석은 "TV 광고 예산을 \$1,000 늘리면 매출이 \$1,010 증가한다 (순이익 \$10)"입니다. 하지만 이 '1.01'이라는 숫자는 우리가 가진 *하나의* 데이터 샘플로부터 얻은 *추정치*($\hat{\beta}_1$)일 뿐입니다.

\textbf{만약 우리가 다른 날짜에 데이터를 다시 수집했다면(다른 '실현' or 'realization') 어땠을까요?}
아마도 $\hat{\beta}_1 = 1.03$이나 $\hat{\beta}_1 = 0.98$처럼 미묘하게 다른 값을 얻었을 것입니다.

통계적 추론(Inference)은 이러한 불확실성을 다루는 학문입니다. 이 문서의 목표는 다음과 같은 질문에 답하는 것입니다.

\begin{itemize}
    \item \textbf{정확성 (Accuracy):} 우리가 얻은 $\hat{\beta}_1 = 1.01$이라는 값은 얼마나 정확하고 신뢰할 수 있는가? (1부)
    \item \textbf{유의성 (Significance):} $\hat{\beta}_1$이 '0'과 충분히 멀리 떨어져 있는가? 즉, TV 광고($x$)가 매출($y$)에 *정말로* 영향을 미치는가, 아니면 그냥 우연인가? (2부)
    \item \textbf{예측 불확실성 (Prediction Uncertainty):} 모델이 예측한 값 $\hat{y}$는 얼마나 믿을 수 있는가? (3부)
\end{itemize}

\newpage

% ======================================================
\section{핵심 용어 정리}
% ======================================================

본격적인 학습에 앞서 주요 용어들을 정리합니다.

\begin{table}[h!]
\centering
\caption{선형 회귀 추론 핵심 용어}
\label{tab:terms}
\begin{adjustbox}{width=\textwidth}
\begin{tabular}{@{}lll@{}}
\toprule
용어 & 원어 (English) & 쉬운 설명 \\
\midrule
\textbf{추론} & Inference & 제한된 데이터(샘플)를 가지고 더 큰 모집단의 특성을 추측하는 과정 \\
\textbf{$\hat{\beta}$ (계수 추정치)} & Coefficient Estimate & 우리가 가진 데이터로 계산한 회귀 계수. (e.g., 1.01) \\
\textbf{부트스트래핑} & Bootstrapping & 원본 데이터에서 '복원 추출'을 반복하여 가상의 데이터셋들을 만드는 기법 \\
\textbf{복원 추출} & Sampling with Replacement & 데이터를 뽑은 후 다시 집어넣고 다음 데이터를 뽑는 방식 (중복 허용) \\
\textbf{신뢰구간} & Confidence Interval (CI) & 실제 모수(e.g., 진짜 $\beta_1$)가 포함될 것이라 95\% 신뢰하는 범위 \\
\textbf{표준 오차} & Standard Error (SE) & $\hat{\beta}$ 값들이 평균으로부터 얼마나 흩어져 있는지를 나타내는 표준편차 \\
\textbf{가설 검정} & Hypothesis Testing & "효과가 없다"($H_0$)는 주장이 맞는지 데이터로 검증하는 절차 \\
\textbf{귀무가설 ($H_0$)} & Null Hypothesis & "아무런 효과가 없다"는 기본 가정. (e.g., "$\beta_1 = 0$이다.") \\
\textbf{t-검정 통계량} & t-test statistic & 계수 값이 0으로부터 표준 오차의 몇 배만큼 떨어져 있는지(신호 대 잡음비) \\
\textbf{p-값} & p-value & 귀무가설($H_0$)이 맞다고 할 때, 현재 데이터(혹은 더 극단적인)가 \\
& & *우연히* 관찰될 확률. (작을수록 $H_0$가 틀렸다고 확신) \\
\textbf{예측구간} & Prediction Interval (PI) & *새로운* 데이터 포인트 $y$ 하나가 존재할 것이라 95\% 신뢰하는 범위 \\
\bottomrule
\end{tabular}
\end{adjustbox}
\end{table}

\newpage

% ======================================================
\section{1부: 추정치의 정확성 평가 (Accuracy of Estimates)}
% ======================================================

\subsection{문제 제기: 불확실성은 어디에서 오는가?}

우리가 가진 데이터는 완벽하지 않습니다. 불확실성(오차)의 원인은 크게 두 가지입니다.

\begin{enumerate}
    \item \textbf{우연적 오차 (Aleatoric / Irreducible Error, $\epsilon$)}
        \begin{itemize}
            \item 시스템에 본질적으로 내재된 무작위성 또는 노이즈입니다.
            \item 예: 동일한 광고비를 써도 날씨, 경쟁사 프로모션 등 '측정되지 않은' 요인 때문에 매출이 매번 다르게 나옵니다.
            \item 이는 모델을 아무리 개선해도 줄일 수 없는 '축소 불가능한 오차'입니다.
        \end{itemize}
    \item \textbf{인식론적 오차 (Epistemic / Misspecification Error)}
        \begin{itemize}
            \item 우리가 모델을 잘못 설정했거나(e.g., 비선형인데 선형으로 가정) 데이터가 부족해서 발생하는 오차입니다.
            \item 우리가 가진 데이터는 수많은 가능한 현실 중 하나의 '실현(realization)'일 뿐입니다.
        \end{itemize}
\end{enumerate}

이러한 오차 때문에, 우리가 데이터를 다시 수집하면 $\hat{\beta}$ 값도 계속 바뀔 것입니다. 우리의 목표는 이 $\hat{\beta}$가 얼마나 \textbf{변동(Variability)}하는지 파악하는 것입니다.

\subsection{부트스트래핑 (Bootstrapping): '평행 우주' 시뮬레이션}

$\hat{\beta}$의 변동성을 알려면 "평행 우주"에서 데이터를 여러 번 가져와 $\hat{\beta}$를 여러 번 계산해 보면 됩니다. 하지만 현실에선 불가능합니다.

\textbf{해결책: 부트스트래핑 (Bootstrapping)}
우리가 가진 원본 데이터셋(크기 $N$)을 '모집단' 그 자체라고 간주하고, 이로부터 가상의 '평행 우주' 데이터셋들을 생성하는 기법입니다.

\begin{infobox}
\textbf{직관적 비유: 주머니 속 공 뽑기}

1.  우리에게 $N=5$개의 공(데이터)이 담긴 주머니(원본 데이터셋)가 있습니다. (공 번호: 1, 3, 5, 8, 9)
2.  이 주머니에서 공을 하나 꺼내 번호를 확인하고(e.g., 8번), 복제본을 만든 뒤, 새 주머니로 옮깁니다.
3.  \textbf{중요:} 꺼냈던 8번 공은 다시 \textbf{원본 주머니에 집어넣습니다.} (이것이 '복원 추출'입니다.)
4.  다시 주머니에서 공을 꺼냅니다. 아까 뽑았던 8번이 또 나올 수도 있습니다. (e.g., 8번) 복제본을 새 주머니로 옮깁니다.
5.  이 과정을 원본 크기 $N=5$가 될 때까지 반복합니다.
6.  \textbf{결과:}
    \begin{itemize}
        \item 원본 샘플: \{1, 3, 5, 8, 9\}
        \item 첫 번째 부트스트랩 샘플: \{8, 8, 3, 5, 1\} (8번 중복, 9번 누락)
    \end{itemize}
7.  이 1\~6의 과정을 $S$번(e.g., 1000번) 반복하여 $S$개의 '부트스트랩 샘플'(평행 우주)을 만듭니다.
\end{infobox}

\subsection{신뢰구간: $\beta$의 분포에서 정확성 찾기}

부트스트래핑으로 $S$개의 '가상' 데이터셋을 만들었습니다. 이제 $\hat{\beta}$의 분포를 찾을 수 있습니다.

\textbf{절차:}
\begin{enumerate}
    \item $S$개의 각 부트스트랩 샘플에 대해 선형 회귀 모델을 학습시킵니다.
    \item $S$개의 서로 다른 $\hat{\beta}^{(1)}, \hat{\beta}^{(2)}, ..., \hat{\beta}^{(S)}$ 값들을 얻습니다.
    \item 이 값들을 모아 히스토그램을 그리면 $\hat{\beta}$의 (근사적인) 샘플링 분포가 됩니다.
\end{enumerate}

이 분포가 바로 $\hat{\beta}$의 불확실성을 시각적으로 보여줍니다. 분포가 좁으면(표준편차가 작으면) 추정치가 매우 정확한 것이고, 넓으면(표준편차가 크면) 부정확한 것입니다.

\textit{[참고 이미지: $\hat{\beta}_1$의 부트스트랩 분포 - TV 광고는 변동성이 작고, Radio 광고는 변동성이 큼]}

\subsubsection{신뢰구간 (Confidence Interval, CI) 계산법}

이 분포를 사용해 "실제 $\beta$ 값이 존재할 것이라 95\% 신뢰하는 구간"을 계산할 수 있습니다.

\textbf{방법 (가정 없는 백분위수 방식):}
\begin{enumerate}
    \item $S$개의 $\hat{\beta}$ 값들을 크기순으로 정렬합니다. (e.g., $S=1000$개)
    \item \textbf{95\% 신뢰구간}을 원한다면, 하위 2.5\%와 상위 2.5\%를 잘라냅니다.
    \item 즉, 정렬된 값 중에서 [2.5 백분위수]와 [97.5 백분위수] 값을 찾습니다.
    \item (e.g., 1000개 샘플 중 25번째 값과 975번째 값)
    \item \textbf{예시:} $[11.50, 12.26, 12.81, ..., 15.21]$
        \begin{itemize}
            \item `np.percentile(betas, 2.5)` $\rightarrow$ $12.80$ (하한)
            \item `np.percentile(betas, 97.5)` $\rightarrow$ $13.71$ (상한)
            \item \textbf{95\% CI = [12.80, 13.71]}
        \end{itemize}
\end{enumerate}
\textbf{해석:} 우리의 $\hat{\beta}$ 추정치가 이 과정을 통해 얻어졌을 때, 실제(참) $\beta$ 값이 [12.80, 13.71] 구간 내에 존재한다고 95\% 신뢰할 수 있습니다.

\subsubsection{표준 오차 (Standard Error, SE)}

신뢰구간 외에 불확실성을 요약하는 또 다른 값입니다.

\begin{itemize}
    \item \textbf{표준 오차 (SE)}: $S$개 $\hat{\beta}$ 값들의 \textbf{표준편차}입니다. ($\sigma_{\hat{\beta}}$)
    \item \textbf{근사적 신뢰구간:} 만약 $\hat{\beta}$의 분포가 정규분포(종 모양)를 따른다고 *가정*한다면, 95\% CI는 대략 다음과 같이 근사할 수 있습니다.
    \[
        \text{95\% CI} \approx [\mu_{\hat{\beta}} - 2 \times SE_{\hat{\beta}}, \quad \mu_{\hat{\beta}} + 2 \times SE_{\hat{\beta}}]
    \]
    (여기서 $\mu_{\hat{\beta}}$는 $S$개 $\hat{\beta}$ 값들의 평균입니다.)
\end{itemize}

\begin{warningbox}
\textbf{표준 편차(Standard Deviation) vs. 표준 오차(Standard Error)}
\begin{itemize}
    \item \textbf{표준 편차 (SD):} 데이터 포인트($y$)가 평균($\bar{y}$)으로부터 얼마나 흩어져 있는가? (데이터 자체의 변동성)
    \item \textbf{표준 오차 (SE):} 추정치($\hat{\beta}$)가 실제 모수($\beta$)로부터 얼마나 흩어져 있을 것으로 *예상*되는가? (추정치의 변동성)
\end{itemize}
부트스트래핑에서는 $\hat{\beta}$ 샘플들의 표준편차를 표준 오차(SE)로 사용합니다.
\end{warningbox}

\newpage

% ======================================================
\section{2부: 예측 변수의 유의성 평가 (Significance)}
% ======================================================

\subsection{무엇이 '중요한' 예측 변수인가?}

이제 우리는 각 예측 변수(e.g., TV, Radio, Newspaper)의 $\hat{\beta}$ 분포를 알고 있습니다. 어떤 변수가 결과(매출)에 가장 큰 영향을 미칠까요?

\begin{itemize}
    \item \textbf{단순한 방법:} $\hat{\beta}$의 평균값 $|\mu_{\hat{\beta}}|$이 가장 큰 변수.
    \item \textbf{문제점:} Figure \ref{fig:beta_dist}에서 보듯이, Radio 광고($\mu_{\beta_1} = -0.05, \sigma_{\beta_1} = 0.1$)는 평균은 0에 가깝지만 변동성이 매우 큽니다. 이는 실제 값이 0.15일 수도, -0.25일 수도 있음을 의미합니다 (신뢰할 수 없음).
    \item 반면 TV 광고($\mu_{\beta_1} = 0.05, \sigma_{\beta_1} = 0.005$)는 값은 작지만 변동성이 매우 적어, 0이 아니라고 확실히 말할 수 있습니다.
\end{itemize}

즉, '중요도'는 계수의 \textbf{크기(신호)}뿐만 아니라 \textbf{불확실성(잡음)}도 함께 고려해야 합니다.

\subsection{$\hat{t}$-검정 통계량: 신뢰도를 반영한 중요도}

이 '신호 대 잡음비'를 측정하는 지표가 바로 \textbf{$\hat{t}$-검정 통계량}입니다. (강의에서는 $\sqrt{n}$을 생략한 $\hat{t}$-test hat을 사용)

\[
    \hat{t}\text{-test} = \frac{\mu_{\hat{\beta}}}{\sigma_{\hat{\beta}}} = \frac{\text{계수 평균 (신호)}}{\text{표준 오차 (잡음)}}
\]

이 값은 "계수 평균이 0으로부터 표준 오차의 몇 배만큼 떨어져 있는가?"를 의미합니다. $\hat{t}$ 통계량의 절댓값이 클수록, 그 변수는 불확실성에 비해 더 강력한 신호를 가집니다.

\begin{summarybox}
\textbf{특징 중요도 순위의 변화}

캘리포니아 주택 가격 데이터 예시에서, 중요도 순위는 어떤 지표를 쓰느냐에 따라 달라집니다.

\begin{table}[h!]
\caption{중요도 평가 지표에 따른 순위 비교}
\label{tab:importance}
\centering
\begin{tabular}{@{}ccc@{}}
\toprule
\textbf{순위} & \textbf{지표 1: $|\mu_{\hat{\beta}}|$ (계수 크기)} & \textbf{지표 2: $\hat{t}$-test (신뢰도 반영)} \\
\midrule
1 & AveBedrms (평균 방 수) & \textbf{MedInc (중간 소득)} \\
2 & MedInc (중간 소득) & \textbf{HouseAge (주택 연령)} \\
3 & AveRooms (평균 총 방 수) & Latitude (위도) \\
... & ... & ... \\
\bottomrule
\end{tabular}
\end{table}

\begin{itemize}
    \item '평균 방 수(AveBedrms)'는 계수 자체는 크지만, 부트스트래핑 결과 불확실성(SE)이 매우 커서 $\hat{t}$-test 순위는 낮아졌습니다.
    \item 반면 '중간 소득(MedInc)'은 계수 값도 크고 불확실성도 낮아, 가장 신뢰할 수 있는 중요한 예측 변수로 선정되었습니다.
\end{itemize}
\end{summarybox}

\subsection{p-값: '가장 중요한' 것이 '유의미'한가?}

$\hat{t}$-test로 '가장 중요한' 변수(MedInc)를 찾았습니다. 하지만 이런 질문이 남습니다.

\textit{"이 변수들이 모두 쓰레기(junk)이고, MedInc는 단지 '쓰레기 중 으뜸'인 것은 아닐까?"}

즉, MedInc가 매출에 미치는 영향이 \textbf{실제로 존재}하는지, 아니면 우리가 가진 데이터에서 \textbf{우연히} 그렇게 보인 것인지 검증해야 합니다. 이것이 \textbf{가설 검정(Hypothesis Testing)}입니다.

\begin{itemize}
    \item \textbf{귀무가설 ($H_0$):} "이 예측 변수는 $y$에 아무런 영향을 미치지 않는다."
        \begin{itemize}
            \item (수학적 표현: 실제 $\beta = 0$이다.)
            \item (이 경우, 우리가 관찰한 $\hat{t}$-test 값은 순전히 우연(random chance)의 산물이다.)
        \end{itemize}
    
    \item \textbf{대립가설 ($H_1$):} "이 예측 변수는 $y$에 유의미한 영향을 미친다."
        \begin{itemize}
            \item (수학적 표현: $\beta \neq 0$이다.)
        \end{itemize}
\end{itemize}

\subsection{p-값의 정의와 해석}

가설 검증은 $H_0$가 맞다고 가정하고 시작합니다.

\textbf{검증 절차:}
\begin{enumerate}
    \item $H_0$가 맞다고 가정합니다. (즉, $x$와 $y$는 아무 관계가 없다. $\beta=0$이다.)
    \item 이 가정 하에서, 데이터를 랜덤 노이즈로 간주하고 $\hat{t}$-test 값을 계산합니다.
    \item 이 과정을 수없이 반복하면, '순전히 우연'에 의해 발생하는 $\hat{t}$-test 값들의 분포(Null Distribution)를 얻을 수 있습니다. (이 분포는 \textbf{Student's t-distribution}으로 알려져 있습니다.)
    \item \textbf{핵심 질문:} 이 '우연의 분포'에서, 우리가 \textbf{실제 데이터로 계산한 $\hat{t}$-test 값} (e.g., $t^* = 49$) 또는 그보다 더 극단적인 값이 관찰될 확률은 얼마인가?
    \item 이 확률이 바로 \textbf{p-값 (p-value)}입니다.
\end{enumerate}

\[
    p\text{-value} = P(|t_{\text{random}}| \ge |t^{*}_{\text{our\_model}}| \quad | \quad H_0 \text{ is true})
\]

\textit{[참고 이미지: p-값의 시각적 이해 - t-분포 곡선에서 극단적인 꼬리 영역의 면적]}

\textbf{p-값 해석:}
\begin{itemize}
    \item \textbf{p-값이 크다 (e.g., p = 0.5):}
        \begin{itemize}
            \item "우리가 관찰한 $t^*$ 값은 '아무 효과가 없다'고 가정해도 50\%의 확률로 우연히 발생할 수 있다."
            \item $\rightarrow$ $H_0$를 기각할 근거가 부족하다. (변수가 유의미하다고 말할 수 없다.)
        \end{itemize}
    \item \textbf{p-값이 작다 (e.g., p = 0.01):}
        \begin{itemize}
            \item "우리가 관찰한 $t^*$ 값은 '아무 효과가 없다'고 가정하면 \textbf{단 1\%}의 확률로만 우연히 발생할 수 있다. (매우 희귀한 일)"
            \item $\rightarrow$ "우연이라기엔 너무 극단적이다. $H_0$ 가정이 틀렸을 것이다."
            \item $\rightarrow$ $H_0$를 기각하고 $H_1$을 채택한다. (변수가 \textbf{통계적으로 유의미하다}고 결론)
        \end{itemize}
\end{itemize}

\textbf{유의 수준 (Significance Level):} 통상적으로 $p < 0.05$ (5\%)를 $H_0$ 기각의 기준으로 삼습니다.

\newpage

% ======================================================
\section{3부: 예측의 불확실성 (Uncertainty in Predictions)}
% ======================================================

지금까지 $\beta$ 계수 자체의 불확실성을 다뤘습니다. 이제 모델의 \textbf{예측값} $\hat{y}$의 불확실성을 다룹니다.

\subsection{함수 $f$의 신뢰구간 (CI for $f$)}

부트스트래핑을 통해 $S$개의 다른 모델 $\hat{f}^{(i)}(x) = \hat{\beta}_0^{(i)} + \hat{\beta}_1^{(i)} x$을 얻었습니다. 이 $S$개의 회귀선을 모두 그리면 '스파게티 플롯(spaghetti plot)'이 됩니다.

\textit{[참고 이미지: 1000개의 부트스트랩 모델(회귀선)을 그린 '스파게티 플롯']}

이 플롯은 \textbf{$\hat{\beta}$의 불확실성}이 \textbf{$\hat{f}(x)$ 예측의 불확실성}으로 어떻게 전파되는지 보여줍니다.

\begin{itemize}
    \item 특정 $x$ 값(e.g., TV Budget = 200)에서, 1000개의 다른 예측값 $\hat{f}^{(i)}(200)$이 나옵니다.
    \item 이 값들의 95\% 백분위수 구간을 계산할 수 있습니다.
    \item 이 작업을 모든 $x$에 대해 수행하여 연결하면, 함수 $f$에 대한 95\% \textbf{신뢰구간(Confidence Interval)} 밴드(band)가 만들어집니다.
\end{itemize}

\textbf{해석:} "우리는 실제 \textbf{평균} 응답을 나타내는 회귀선 $f(x)$가 95\%의 신뢰로 이 밴드(연한 하늘색 영역) 안에 있다고 믿는다."

\subsection{새로운 $y$의 예측구간 (PI for $y$)}

$f(x)$의 신뢰구간은 \textbf{모델 자체의 불확실성}(즉, $\hat{\beta}$가 $\beta$와 다를 수 있는 오차)만 반영합니다.

하지만 우리가 예측하려는 \textbf{새로운 관측치 $y$}는 모델 $f(x)$에 더해 \textbf{축소 불가능한 오차 $\epsilon$}까지 포함하고 있습니다. ($y = f(x) + \epsilon$)

\textit{[참고 이미지: 신뢰구간(CI)과 예측구간(PI)의 비교 - PI가 $\epsilon$ 오차를 추가로 포함하므로 항상 더 넓음]}

따라서 '새로운 $y$ 값'이 존재할 범위를 나타내는 \textbf{예측구간(Prediction Interval)}은 $f(x)$의 신뢰구간보다 \textbf{항상 더 넓습니다.}

\begin{warningbox}
\textbf{신뢰구간 (CI) vs. 예측구간 (PI)}

두 개념은 예측 대상이 다릅니다.

\begin{itemize}
    \item \textbf{신뢰구간 (CI):} "TV 예산이 \$200,000일 때, \textbf{평균 매출($f(x)$)}은 95\% 신뢰로 [\$16.5M, \$17.5M] 사이에 있을 것이다."
        \begin{itemize}
            \item (오직 $\hat{\beta}$의 불확실성만 고려)
        \end{itemize}
    \item \textbf{예측구간 (PI):} "TV 예산이 \$200,000인 \textbf{새로운 매장 하나}의 \textbf{개별 매출($y$)}은 95\% 확률로 [\$14.0M, \$20.0M] 사이에 있을 것이다."
        \begin{itemize}
            \item ($\hat{\beta}$의 불확실성 + 개별 오차 $\epsilon$의 불확실성 모두 고려)
        \end{itemize}
\end{itemize}
\end{warningbox}

\subsection{구간의 '나팔' 모양}

Figure \ref{fig:ci_vs_pi}에서 볼 수 있듯이, 신뢰구간과 예측구간은 데이터의 중심(e.g., $x$의 평균 $\bar{x}$)에서 가장 좁고, 양 끝으로 갈수록 넓어지는 '나팔' 또는 '깔때기' 모양을 띱니다.

\textbf{이유:} 회귀선은 데이터의 중심($\bar{x}, \bar{y}$)을 축으로 회전하는 경향이 있습니다. $\hat{\beta}_1$(기울기)의 작은 불확실성이라도, 중심에서 멀어질수록 $\hat{y}$ 예측값에 큰 차이를 만들어내기 때문입니다.

\newpage

% ======================================================
\section{빠르게 훑어보기 (1-Page Summary)}
% ======================================================

\begin{summarybox}
\textbf{1. 부트스트래핑 (Bootstrapping)}
\begin{itemize}
    \item \textbf{목적:} $\hat{\beta}$ 추정치의 불확실성(변동성)을 알기 위해.
    \item \textbf{방법:} 원본 데이터(크기 $N$)에서 \textbf{복원 추출}을 $N$번 수행하여 '부트스트랩 샘플'을 만든다. 이 과정을 $S$번 반복한다.
    \item \textbf{결과:} $S$개의 모델과 $S$개의 $\hat{\beta}$ 분포를 얻는다.
\end{itemize}
\end{summarybox}

\begin{infobox}
\textbf{2. 신뢰구간 (CI) vs. 예측구간 (PI)}
\begin{itemize}
    \item \textbf{신뢰구간 (CI):} \textbf{평균} 응답 $\hat{f}(x)$의 불확실성.
        \begin{itemize}
            \item (오차 원인: $\hat{\beta}$의 불확실성)
        \end{itemize}
    \item \textbf{예측구간 (PI):} \textbf{새로운 관측치 $y$}의 불확실성.
        \begin{itemize}
            \item (오차 원인: $\hat{\beta}$의 불확실성 + 축소 불가능한 오차 $\epsilon$)
            \item \textbf{항상 PI가 CI보다 넓다.}
        \end{itemize}
\end{itemize}
\end{infobox}

\begin{tcolorbox}[
  colback=yellow!5!white,
  colframe=yellow!75!black,
  title=3. $\hat{t}$-test 와 p-value (유의성 검증),
  fonttitle=\bfseries
]
\begin{itemize}
    \item \textbf{질문:} 이 변수가 $y$에 미치는 영향이 우연인가, 실제인가?
    \item \textbf{귀무가설 ($H_0$):} 우연이다 ($\beta = 0$).
    \item \textbf{$\hat{t}$-test 통계량:} $\frac{\mu_{\hat{\beta}}}{\sigma_{\hat{\beta}}}$ (신호 대 잡음비). 이 값이 0에서 얼마나 먼가?
    \item \textbf{p-value:} $H_0$가 맞다고 가정할 때, 우리가 관찰한 $\hat{t}$-test 값(혹은 더 극단적인 값)이 \textbf{우연히} 나올 확률.
    \item \textbf{결론:} $p < 0.05$ 이면, "우연히 일어날 확률이 5\% 미만이므로 $H_0$는 틀렸을 것이다. 이 변수는 유의미하다."
\end{itemize}
\end{tcolorbox}

\end{document}
