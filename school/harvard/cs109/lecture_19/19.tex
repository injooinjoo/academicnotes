%%%%%%%%%%%%%%%%%%%%%%%%%%%%%%%%%%%%%%%%%%%%%%%%%%%%%%%%%%%%%%%%%%%%%%%%%%%%%%%
% Harvard Academic Notes - 통합 마스터 템플릿
% 모든 강의 노트에 적용되는 통일된 스타일
% 버전: 2.1 - 가독성 개선 (선택적 최적화)
% 최종 수정일: 2025-11-17
%%%%%%%%%%%%%%%%%%%%%%%%%%%%%%%%%%%%%%%%%%%%%%%%%%%%%%%%%%%%%%%%%%%%%%%%%%%%%%%

\documentclass[11pt,a4paper]{article}

%========================================================================================
% 기본 패키지
%========================================================================================

% --- 한국어 지원 ---
\usepackage{kotex}

% --- 페이지 레이아웃 ---
\usepackage[top=20mm, bottom=20mm, left=20mm, right=18mm]{geometry}
\usepackage{setspace}
\onehalfspacing                      % 1.5배 줄간격
\setlength{\parskip}{0.5em}          % 문단 간격
\setlength{\parindent}{0pt}          % 들여쓰기 없음

% --- 표 관련 ---
\usepackage{booktabs}              % 고품질 표
\usepackage{tabularx}              % 자동 너비 조절 표
\usepackage{array}                 % 표 컬럼 확장
\usepackage{longtable}             % 여러 페이지 표
\renewcommand{\arraystretch}{1.1}  % 표 행간 조절

%========================================================================================
% 헤더 및 푸터
%========================================================================================

\usepackage{fancyhdr}
\pagestyle{fancy}
\fancyhf{}
\fancyhead[L]{\small\textit{CS109A: 데이터 과학 입문}}
\fancyhead[R]{\small\textit{Lecture 19}}
\fancyfoot[C]{\thepage}
\renewcommand{\headrulewidth}{0.5pt}
\renewcommand{\footrulewidth}{0.3pt}

% 첫 페이지는 헤더 없음
\fancypagestyle{firstpage}{
    \fancyhf{}
    \fancyfoot[C]{\thepage}
    \renewcommand{\headrulewidth}{0pt}
}

%========================================================================================
% 색상 정의 (파스텔 톤 + 다크모드 호환)
%========================================================================================

\usepackage[dvipsnames]{xcolor}

% 밝은 배경용 파스텔 색상
\definecolor{lightblue}{RGB}{220, 235, 255}      % 부드러운 파랑
\definecolor{lightgreen}{RGB}{220, 255, 235}     % 부드러운 초록
\definecolor{lightyellow}{RGB}{255, 250, 220}    % 부드러운 노랑
\definecolor{lightpurple}{RGB}{240, 230, 255}    % 부드러운 보라
\definecolor{lightgray}{gray}{0.95}              % 밝은 회색
\definecolor{lightpink}{RGB}{255, 235, 245}      % 부드러운 핑크
\definecolor{boxgray}{gray}{0.95}
\definecolor{boxblue}{rgb}{0.9, 0.95, 1.0}
\definecolor{boxred}{rgb}{1.0, 0.95, 0.95}

% 진한 색상 (테두리/제목용)
\definecolor{darkblue}{RGB}{50, 80, 150}
\definecolor{darkgreen}{RGB}{40, 120, 70}
\definecolor{darkorange}{RGB}{200, 100, 30}
\definecolor{darkpurple}{RGB}{100, 60, 150}

%========================================================================================
% 박스 환경 (tcolorbox) - 6가지 타입
%========================================================================================

\usepackage[most]{tcolorbox}
\tcbuselibrary{skins, breakable}

% 1. 개요 박스 (강의 시작 부분)
\newtcolorbox{overviewbox}[1][]{
    enhanced,
    colback=lightpurple,
    colframe=darkpurple,
    fonttitle=\bfseries\large,
    title=📚 강의 개요,
    arc=3mm,
    boxrule=1pt,
    left=8pt,
    right=8pt,
    top=8pt,
    bottom=8pt,
    breakable,
    #1
}

% 2. 요약 박스
\newtcolorbox{summarybox}[1][]{
    enhanced,
    colback=lightblue,
    colframe=darkblue,
    fonttitle=\bfseries,
    title=📝 핵심 요약,
    arc=2mm,
    boxrule=0.7pt,
    left=6pt,
    right=6pt,
    top=6pt,
    bottom=6pt,
    breakable,
    #1
}

% 3. 핵심 정보 박스
\newtcolorbox{infobox}[1][]{
    enhanced,
    colback=lightgreen,
    colframe=darkgreen,
    fonttitle=\bfseries,
    title=💡 핵심 정보,
    arc=2mm,
    boxrule=0.7pt,
    left=6pt,
    right=6pt,
    top=6pt,
    bottom=6pt,
    breakable,
    #1
}

% 4. 주의사항 박스
\newtcolorbox{warningbox}[1][]{
    enhanced,
    colback=lightyellow,
    colframe=darkorange,
    fonttitle=\bfseries,
    title=⚠️ 주의사항,
    arc=2mm,
    boxrule=0.7pt,
    left=6pt,
    right=6pt,
    top=6pt,
    bottom=6pt,
    breakable,
    #1
}

% 5. 예제 박스
\newtcolorbox{examplebox}[1][]{
    enhanced,
    colback=lightgray,
    colframe=black!60,
    fonttitle=\bfseries,
    title=📖 예제: #1,
    arc=2mm,
    boxrule=0.7pt,
    left=6pt,
    right=6pt,
    top=6pt,
    bottom=6pt,
    breakable,
}

% 6. 정의 박스
\newtcolorbox{definitionbox}[1][]{
    enhanced,
    colback=lightpink,
    colframe=purple!70!black,
    fonttitle=\bfseries,
    title=📌 정의: #1,
    arc=2mm,
    boxrule=0.7pt,
    left=6pt,
    right=6pt,
    top=6pt,
    bottom=6pt,
    breakable,
}

% 7. 중요 박스 (importantbox - warningbox와 유사)
\newtcolorbox{importantbox}[1][]{
    enhanced,
    colback=boxred,
    colframe=red!70!black,
    fonttitle=\bfseries,
    title=⚠️ 매우 중요: #1,
    arc=2mm,
    boxrule=0.7pt,
    left=6pt,
    right=6pt,
    top=6pt,
    bottom=6pt,
    breakable,
}

% 8. cautionbox (warningbox와 동일)
\let\cautionbox\warningbox
\let\endcautionbox\endwarningbox

%========================================================================================
% 코드 블록 설정 (밝은 배경)
%========================================================================================

\usepackage{listings}

\definecolor{codegray}{rgb}{0.5,0.5,0.5}
\definecolor{codepurple}{rgb}{0.58,0,0.82}
\definecolor{backcolour}{rgb}{0.95,0.95,0.95}

\lstset{
    basicstyle=\ttfamily\small,
    backgroundcolor=\color{lightgray},
    keywordstyle=\color{darkblue}\bfseries,
    commentstyle=\color{darkgreen}\itshape,
    stringstyle=\color{purple!80!black},
    numberstyle=\tiny\color{black!60},
    numbers=left,
    numbersep=8pt,
    breaklines=true,
    breakatwhitespace=false,
    frame=single,
    frameround=tttt,
    rulecolor=\color{black!30},
    captionpos=b,
    showstringspaces=false,
    tabsize=2,
    xleftmargin=15pt,
    xrightmargin=5pt,
    escapeinside={\%*}{*)}
}

% Python 코드 스타일
\lstdefinestyle{pythonstyle}{
    language=Python,
    morekeywords={self, True, False, None},
}

% SQL 코드 스타일
\lstdefinestyle{sqlstyle}{
    language=SQL,
    morekeywords={SELECT, FROM, WHERE, JOIN, GROUP, BY, ORDER, HAVING},
}

%========================================================================================
% 목차 스타일링
%========================================================================================

\usepackage{tocloft}
\renewcommand{\cftsecleader}{\cftdotfill{\cftdotsep}}
\setlength{\cftbeforesecskip}{0.4em}
\renewcommand{\cftsecfont}{\bfseries}
\renewcommand{\cftsubsecfont}{\normalfont}

%========================================================================================
% 표 및 그림
%========================================================================================

\usepackage{graphicx}              % 이미지
\usepackage{adjustbox}             % 표/박스 크기 조절

% 표 캡션 스타일
\usepackage{caption}
\captionsetup[table]{
    labelfont=bf,
    textfont=it,
    skip=5pt
}
\captionsetup[figure]{
    labelfont=bf,
    textfont=it,
    skip=5pt
}

%========================================================================================
% 수학
%========================================================================================

\usepackage{amsmath, amssymb, amsthm}

% 정리 환경
\theoremstyle{definition}
\newtheorem{theorem}{정리}[section]
\newtheorem{lemma}[theorem]{보조정리}
\newtheorem{proposition}[theorem]{명제}
\newtheorem{corollary}[theorem]{따름정리}
\newtheorem{definition}{정의}[section]
\newtheorem{example}{예제}[section]

%========================================================================================
% 하이퍼링크
%========================================================================================

\usepackage[
    colorlinks=true,
    linkcolor=blue!80!black,
    urlcolor=blue!80!black,
    citecolor=green!60!black,
    bookmarks=true,
    bookmarksnumbered=true,
    pdfborder={0 0 0}
]{hyperref}

% PDF 메타데이터는 각 문서에서 설정
\hypersetup{
    pdftitle={CS109A: 데이터 과학 입문 - Lecture 19},
    pdfauthor={강의 노트},
    pdfsubject={Academic Notes}
}

%========================================================================================
% 기타 유용한 패키지
%========================================================================================

\usepackage{enumitem}              % 리스트 커스터마이징
\setlist{nosep, leftmargin=*, itemsep=0.3em}

\usepackage{microtype}             % 타이포그래피 개선
\usepackage{footnote}              % 각주 개선
\usepackage{url}                   % URL 줄바꿈
\urlstyle{same}

%========================================================================================
% 사용자 정의 명령어
%========================================================================================

% 강조 텍스트
\newcommand{\important}[1]{\textbf{\textcolor{red!70!black}{#1}}}
\newcommand{\keyword}[1]{\textbf{#1}}
\newcommand{\term}[1]{\textit{#1}}
\newcommand{\code}[1]{\texttt{#1}}

% 용어 설명 (인라인)
\newcommand{\defterm}[2]{\textbf{#1}\footnote{#2}}

% 섹션 시작 전 페이지 분리
\newcommand{\newsection}[1]{\newpage\section{#1}}

%========================================================================================
% 문서 제목 스타일
%========================================================================================

\usepackage{titling}
\pretitle{\begin{center}\LARGE\bfseries}
\posttitle{\par\end{center}\vskip 0.5em}
\preauthor{\begin{center}\large}
\postauthor{\end{center}}
\predate{\begin{center}\large}
\postdate{\par\end{center}}

%========================================================================================
% 섹션 제목 간격
%========================================================================================

\usepackage{titlesec}
\titlespacing*{\section}{0pt}{1.5em}{0.8em}
\titlespacing*{\subsection}{0pt}{1.2em}{0.6em}
\titlespacing*{\subsubsection}{0pt}{1em}{0.5em}

%========================================================================================
% 메타 정보 박스 명령어
%========================================================================================

\newcommand{\metainfo}[4]{
\begin{tcolorbox}[
    colback=lightpurple,
    colframe=darkpurple,
    boxrule=1pt,
    arc=2mm,
    left=10pt,
    right=10pt,
    top=8pt,
    bottom=8pt
]
\begin{tabular}{@{}rl@{}}
▣ \textbf{강의명:} & #1 \\[0.3em]
▣ \textbf{주차:} & #2 \\[0.3em]
▣ \textbf{교수명:} & #3 \\[0.3em]
▣ \textbf{목적:} & \begin{minipage}[t]{0.75\textwidth}#4\end{minipage}
\end{tabular}
\end{tcolorbox}
}

%========================================================================================
% 끝
%========================================================================================


\begin{document}

% --- 제목 페이지 ---
\maketitle
\thispagestyle{firstpage}

\metainfo{CS109A: 데이터 과학 입문}{Lecture 19}{Pavlos Protopapas, Kevin Rader, Chris Gumb}{Lecture 19의 핵심 개념 학습}


% --- 목차 ---
\tableofcontents


\newpage
%===============
% 개요
%===============
\section*{개요: 결정 트리 학습의 두 가지 핵심}

이 문서는 결정 트리(Decision Tree) 모델의 두 가지 핵심적인 확장 개념인 **회귀(Regression)**와 **가지치기(Pruning)**에 대해 다룹니다.

데이터 사이언스 초심자도 쉽게 이해할 수 있도록 각 개념의 필요성부터 작동 원리, 그리고 실제 적용 시 주의사항까지 단계별로 설명합니다.

\begin{boxSummary}
    \textbf{1. 회귀 트리 (Regression Trees):}
    '예/아니오' 같은 분류(Classification) 문제뿐만 아니라, '집값', '매출액' 등 연속적인 숫자(Quantitative outcome)를 예측하는 회귀 문제에 결정 트리를 사용하는 방법입니다. 핵심은 **'불순도(Impurity)' 대신 'MSE(평균 제곱 오차)'를 기준**으로 트리를 분할하는 것입니다.

    \textbf{2. 가지치기 (Pruning):}
    트리가 훈련 데이터에만 과도하게 최적화되는 **과적합(Overfitting)**을 방지하기 위한 핵심 기술입니다. 트리를 일단 최대로 성장시킨 후, 불필요한 가지들을 체계적으로 잘라내어 모델의 일반화 성능을 높입니다. 핵심은 **'비용 복잡도(Cost Complexity)'**라는 정규화 항을 도입하는 것입니다.
\end{boxSummary}


%===============
% 용어 정리
%===============
\section{핵심 용어 정리}

본격적인 학습에 앞서, 자주 등장하는 핵심 용어들을 정리합니다.

\begin{adjustbox}{width=\textwidth, center}
\begin{tabular}{l l l l}
\toprule
\textbf{용어} & \textbf{원어} & \textbf{쉬운 설명} & \textbf{비고} \\
\midrule
회귀 트리 & Regression Tree & 연속적인 숫자(예: 가격)를 예측하는 결정 트리 & 분류 트리의 반대 \\
분류 트리 & Classification Tree & 범주형 값(예: 생존/사망)을 예측하는 결정 트리 & \\
분할 기준 & Splitting Criterion & 트리의 가지를 나눌 때 사용하는 기준 & 분류: 지니 불순도, 엔트로피 \\
& & & 회귀: \textbf{MSE (평균 제곱 오차)} \\
단말 노드 & Leaf Node (Terminal Node) & 트리의 가장 마지막에 위치한 노드 (예측값 결정) & \\
과적합 & Overfitting & 모델이 훈련 데이터에만 너무 잘 맞아, \\
& & 새로운 데이터에 대한 예측 성능이 떨어지는 현상 & \\
가지치기 & Pruning & 과적합을 막기 위해 트리의 복잡도를 줄이는 과정 & (사후 가지치기) \\
비용 복잡도 & Cost Complexity & \textbf{모델의 오류(Error)}와 \textbf{복잡도(Complexity)}를 \\
& Pruning (CCP) & 동시에 고려하는 가지치기 공식 & \\
복잡도 매개변수 & Complexity Parameter ($\alpha$) & 트리의 복잡도에 얼마나 큰 페널티를 줄지 정하는 값 & 하이퍼파라미터 \\
\bottomrule
\end{tabular}
\end{adjustbox}
\captionof{table}{회귀 트리 및 가지치기 관련 핵심 용어}
\label{tab:terms}


\newpage
%===============
% 1. 회귀 트리 (Regression Trees)
%===============
\section{회귀 트리 (Regression Trees)}

우리는 결정 트리를 '스무고개'와 비슷하다고 배웠습니다. "A인가?", "B인가?" 질문을 통해 대상을 구분하는 것은 **분류(Classification)** 문제입니다.

그렇다면 "이 사람의 나이는 몇 살인가?" 또는 "이 집의 가격은 얼마인가?"처럼 \textbf{연속적인 숫자(Continuous Variable)}를 예측해야 하는 **회귀(Regression)** 문제에는 결정 트리를 어떻게 적용할 수 있을까요?

\begin{boxDefinition}{회귀 트리 (Regression Tree)란?}
    회귀 트리는 예측 결과값이 범주(Category)가 아닌 \textbf{연속적인 숫자}인 결정 트리 모델입니다.

    \begin{itemize}
        \item \textbf{분류 트리:} 각 단말 노드(Leaf Node)는 \textbf{다수결(Majority Vote)}을 통해 가장 흔한 클래스(예: '생존')를 예측합니다.
        \item \textbf{회귀 트리:} 각 단말 노드(Leaf Node)는 해당 노드에 속한 훈련 데이터 샘플들의 \textbf{평균(Mean)} 값을 예측합니다.
    \end{itemize}
\end{boxDefinition}

\begin{boxExample}
    \textbf{타이타닉 생존 예측 (분류 트리)}
    \begin{itemize}
        \item 질문: "성별이 여성인가?" $\rightarrow$ 예
        \item 질문: "객실 등급이 1등급인가?" $\rightarrow$ 예
        \item 예측: 해당 노드에 '생존' 80명, '사망' 5명이 있다면, 예측값은 \textbf{'생존'} (다수결)
    \end{itemize}

    \textbf{보스턴 집값 예측 (회귀 트리)}
    \begin{itemize}
        \item 질문: "방 개수가 6개 이상인가?" $\rightarrow$ 예
        \item 질문: "범죄율이 1\% 미만인가?" $\rightarrow$ 예
        \item 예측: 해당 노드에 속한 집들의 가격이 \{5억, 6억, 5.5억\} 이라면, 예측값은 \textbf{5.5억} (평균)
    \end{itemize}
\end{boxExample}

\subsection{핵심 질문: 어떻게 분할 기준을 정할까? (Splitting Criteria)}

분류 트리에서는 '지니 불순도(Gini)'나 '엔트로피(Entropy)'를 사용하여, 분할 후 두 그룹이 얼마나 더 '순수해졌는지(Pure)'를 측정했습니다. "생존/사망이 명확히 갈리는가?"가 중요했습니다.

회귀 트리에서도 비슷한 목표를 가집니다.
"분할된 두 그룹의 값들이 얼마나 \textbf{서로 비슷비슷하게 모여있는가?}"

\begin{itemize}
    \item \textbf{나쁜 분할:} 한 그룹에 \{1억, 5억, 10억\}이 섞여있음 $\rightarrow$ 평균 5.3억은 누구도 대표하지 못함 (분산이 큼)
    \item \textbf{좋은 분할:} 한 그룹에 \{1억, 1.1억, 1.2억\}이 모여있음 $\rightarrow$ 평균 1.1억은 그룹을 잘 대표함 (분산이 작음)
\end{itemize}

여기서 "값이 얼마나 흩어져 있는가(분산)"를 측정하는 가장 좋은 지표가 바로 \textbf{MSE(Mean Squared Error, 평균 제곱 오차)}입니다.

\begin{boxDefinition}{회귀 트리의 분할 기준: MSE 최소화}
    회귀 트리는 분할 후 생성될 두 자식 노드($R_1$, $R_2$)의 \textbf{MSE의 가중 평균}을 최소화하는 지점(변수 $p$와 임계값 $t_p$)을 찾습니다.

    각 노드 $R$의 MSE는 해당 노드의 \textbf{분산(Variance)}과 같습니다.
    
    $$ MSE(R) = \frac{1}{n_R} \sum_{i \in R} (y_i - \bar{y}_R)^2 $$
    ($n_R$: 노드 $R$의 샘플 수, $y_i$: 실제 값, $\bar{y}_R$: 노드 $R$의 평균 예측값)

    \vspace{1em}
    따라서, 트리가 찾는 최적의 분할 $(p, t_p)$는 다음 공식을 최소화하는 것입니다.

    $$ \min_{p, t_p} \left[ \frac{N_1}{N} MSE(R_1) + \frac{N_2}{N} MSE(R_2) \right] $$
    ($N$: 부모 노드의 총 샘플 수, $N_1, N_2$: 각 자식 노드의 샘플 수)
\end{boxDefinition}

\subsection{회귀 트리의 성장 및 예측 과정}

\begin{enumerate}
    \item \textbf{분할 (Greedy Algorithm):}
    모든 변수(Predictors)와 모든 가능한 분할 지점(Unique values)을 하나씩 다 시도해봅니다.
    그 중 MSE 가중 평균을 \textit{가장 많이} 낮춰주는 '최적의 질문' 하나를 선택하여 노드를 분할합니다.

    \item \textbf{성장:}
    위 1번 과정을 재귀적으로 반복하며 트리를 키워나갑니다.

    \item \textbf{중단 (Stopping Conditions):}
    미리 정해둔 중단 조건에 도달하면 성장을 멈춥니다.
    \begin{itemize}
        \item `max_depth`: 트리의 최대 깊이
        \item `min_samples_leaf`: 단말 노드가 가져야 할 최소 샘플 수
        \item \textbf{Accuracy Gain (MSE Reduction):} 분할로 인해 MSE가 줄어드는 양(`Gain(R)`)이 정해진 임계값(Threshold)보다 작으면 더 이상 분할하지 않습니다.
            $$ Gain(R) = MSE(R_{\text{parent}}) - \left( \frac{N_1}{N} MSE(R_1) + \frac{N_2}{N} MSE(R_2) \right) $$
    \end{itemize}
    
    \item \textbf{예측 (Prediction):}
    새로운 데이터($x_i$)가 들어오면, 트리의 질문을 따라가며 특정 단말 노드에 도달합니다.
    해당 단말 노드에 저장된 \textbf{훈련 데이터의 평균값($\bar{y}_{\text{leaf}}$)}을 최종 예측값($\hat{y}_i$)으로 반환합니다.
\end{enumerate}


\newpage
%===============
% 2. 범주형 변수 처리
%===============
\section{범주형 변수 처리 (Categorical Attributes)}

결정 트리는 "변수 < 임계값" 형태의 비교를 기반으로 작동합니다.
\begin{itemize}
    \item \textbf{수치형(Numerical):} "Sepal width > 4.0?" (가능)
    \item \textbf{범주형(Categorical):} "Color < Red?" (불가능)
\end{itemize}
'Red'보다 작다는 것은 수학적으로 의미가 없습니다.

\subsection{잘못된 접근: 순서형 인코딩 (Ordinal Encoding)}

가장 간단하게 숫자를 부여하는 방식입니다. (예: Yellow=0, Red=1, Purple=2)

\begin{boxWarning}
    \textbf{순서형 인코딩은 절대 사용하면 안 됩니다 (데이터가 실제 순서를 갖지 않는 한).}
    
    이 방식은 모델에게 \textbf{인공적인 순서(Artificial Order)}를 강제로 학습시킵니다.
    (예: "Purple(2) > Red(1)")
    
    만약 인코딩 순서를 "Yellow=2, Red=0, Purple=1"로 바꾸면, 트리의 분할 결과 자체가 완전히 달라집니다. 이는 모델의 성능이 데이터의 본질이 아닌, 프로그래머의 임의적인 인코딩 순서에 의존하게 됨을 의미합니다.
\end{boxWarning}

\subsection{올바른 접근: 원-핫 인코딩 (One-Hot Encoding, OHE)}

범주의 각 값(Category)을 자신만의 새로운 \textbf{이진(Binary) 변수}로 만드는 것입니다.

\begin{boxDefinition}{원-핫 인코딩 (OHE)}
    'Color'라는 하나의 변수를 'is\_Yellow', 'is\_Red', 'is\_Purple' 이라는 여러 개의 0 또는 1 값을 갖는 변수로 변환합니다.
\end{boxDefinition}

\begin{boxExample}
    \textbf{변환 전 (Original)}
    \begin{tabular}{l l} \toprule
    Sepal width & Color \\ \midrule
    3.0 mm & Yellow \\
    3.5 mm & Red \\
    3.7 mm & Purple \\ \bottomrule
    \end{tabular}
    
    \vspace{1em}
    \textbf{변환 후 (One-Hot Encoded)}
    \begin{tabular}{l c c c} \toprule
    Sepal width & Color\_Yellow & Color\_Red & Color\_Purple \\ \midrule
    3.0 mm & 1 & 0 & 0 \\
    3.5 mm & 0 & 1 & 0 \\
    3.7 mm & 0 & 0 & 1 \\ \bottomrule
    \end{tabular}
    \captionof{table}{원-핫 인코딩 예시}
\end{boxExample}

이렇게 변환하면, 트리는 "Color\_Red >= 1?" (즉, 색상이 빨간색인가?) 과 같은 논리적인 질문을 수치형으로 수행할 수 있게 됩니다.

\begin{boxWarning}
    \textbf{Scikit-Learn 사용 시 주의사항}
    
    `sklearn.tree.DecisionTreeClassifier` 나 `DecisionTreeRegressor`는 \textbf{범주형 변수를 자동으로 처리해주지 않습니다.}
    
    모델에 데이터를 넣기 전에, 사용자가 직접 Pandas의 `get_dummies`나 `sklearn.preprocessing.OneHotEncoder`를 사용하여 OHE 전처리를 \textbf{반드시} 수행해야 합니다.
    (참고: XGBoost, LightGBM, CatBoost 같은 다른 라이브러리들은 범주형 변수를 자체적으로 처리하는 기능을 지원하기도 합니다.)
\end{boxWarning}


\newpage
%===============
% 3. 가지치기 (Pruning)
%===============
\section{가지치기 (Pruning): 과적합과의 전쟁}

결정 트리를 아무런 제한 없이 끝까지 성장시키면(Full Tree), 훈련 데이터의 모든 샘플을 완벽하게 구분(분류)하거나 완벽하게 예측(회귀, R²=1)하려 합니다.

이는 훈련 데이터의 사소한 노이즈(Noise)까지 모두 학습하는 \textbf{과적합(Overfitting)} 상태로 이어집니다. 이런 모델은 새로운(Unseen) 데이터가 들어왔을 때 형편없는 성능을 보입니다.

\begin{boxExample}
    \textbf{과적합의 비유 (R²=1 밈)}
    
    "오늘 R²=1인 모델을 만들었어요!"
    "당장 내 집에서 나가!"
    
    R²=1은 모델이 훈련 데이터를 100\% 완벽하게 설명한다는 뜻입니다. 이는 현실에서는 불가능하며, 훈련 데이터에만 과적합된 '암기 기계'를 만들었다는 의미입니다. 이런 모델은 실제 문제 해결에 아무런 도움이 되지 않기 때문에 데이터 과학자들이 가장 경계하는 상황입니다.
\end{boxExample}

\subsection{과적합 제어: 사전 중단 vs 사후 가지치기}

\begin{enumerate}
    \item \textbf{사전 중단 (Pre-stopping):}
    트리가 성장하는 \textit{중에} 멈추는 방식입니다. (`max_depth`, `min_samples_leaf` 등)
    \begin{itemize}
        \item \textbf{문제점:} 최적의 중단 시점을 미리 알기 어렵습니다. 너무 일찍 멈추면 \textbf{과소적합(Underfitting)}이, 너무 늦게 멈추면 \textbf{과적합(Overfitting)}이 발생합니다.
    \end{itemize}
    
    \item \textbf{사후 가지치기 (Post-pruning / Pruning):}
    \textbf{"일단 끝까지 키우고, 나중에 잘라낸다."}
    \begin{itemize}
        \item \textbf{장점:} 트리의 전체 구조를 본 후에 불필요한 부분을 제거하므로, 더 유연하고 정교한 모델 복잡도 제어가 가능합니다.
        \item \textbf{방법:} 가장 표준적인 방법이 \textbf{비용 복잡도 가지치기(Cost Complexity Pruning, CCP)}입니다.
    \end{itemize}
\end{enumerate}

\subsection{비용 복잡도 가지치기 (Cost Complexity Pruning, CCP)}

CCP는 모델의 \textbf{오류(Error)}와 \textbf{복잡도(Complexity)} 사이에 균형점을 찾는 정규화(Regularization) 기법입니다.

\begin{boxDefinition}{비용 복잡도 (Cost Complexity) 공식}
    CCP는 트리의 '비용' $C(T)$를 최소화하는 것을 목표로 합니다.
    
    $$ C(T) = \text{Error}(T) + \alpha \cdot |T| $$

    \begin{itemize}
        \item $C(T)$: 트리 $T$의 총 비용 복잡도
        \item $\text{Error}(T)$: 트리 $T$가 훈련 데이터에서 발생시키는 \textbf{총 오류} (예: 분류 오류 수, 또는 총 MSE)
        \item $|T|$: 트리 $T$의 \textbf{단말 노드(Leaf) 개수}. (트리의 복잡도를 나타내는 척도)
        \item $\alpha$ (알파): \textbf{복잡도 매개변수 (Complexity Parameter)}. 사용자가 정하는 하이퍼파라미터입니다.
    \end{itemize}
\end{boxDefinition}

$\alpha$는 '복잡도에 대한 페널티'입니다.
\begin{itemize}
    \item $\alpha = 0$: 복잡도 페널티 없음. $C(T) = \text{Error}(T)$가 되므로, 훈련 오류가 가장 낮은 \textbf{가장 큰 트리(Full Tree)}가 선택됩니다.
    \item $\alpha \rightarrow \infty$: 복잡도 페널티가 매우 큼. 단말 노드가 2개인 것보다 1개인 것을 선호하게 되므로, 결국 \textbf{가장 단순한 트리(Root Node만 있는 트리)}가 선택됩니다.
    \item $0 < \alpha < \infty$: $\alpha$ 값이 커질수록, 오류(Error)가 조금 늘어나더라도 더 단순한(단말 노드가 적은) 트리를 선호하게 됩니다.
\end{itemize}

\begin{boxExample}
    $\alpha = 0.2$ 일 때, 두 트리를 비교해봅시다.
    
    \textbf{트리 T (Full Tree):}
    \begin{itemize}
        \item Error(T) = 0.32
        \item $|T|$ (단말 노드 수) = 8
        \item $C(T) = 0.32 + (0.2 \times 8) = \textbf{1.92}$
    \end{itemize}
    
    \textbf{트리 $T_{small}$ (가지치기 후):}
    \begin{itemize}
        \item Error($T_{small}$) = 0.33 (훈련 오류는 약간 증가)
        \item $|T_{small}|$ (단말 노드 수) = 7
        \item $C(T_{small}) = 0.33 + (0.2 \times 7) = \textbf{1.73}$
    \end{itemize}
    
    \textbf{결론:}
    비록 $T_{small}$이 훈련 오류는 0.01 더 높지만, 전체 비용 복잡도 $C(T)$는 더 낮습니다 (1.73 < 1.92).
    따라서 $\alpha=0.2$ 일 때는 $T_{small}$이 $T$보다 '더 좋은 트리'로 간주됩니다.
\end{boxExample}

\subsection{최적의 $\alpha$와 트리 찾는 과정 (Nested Cross-Validation)}

CCP의 알고리즘은 복잡해 보이지만, 본질은 \textbf{2단계 교차 검증}입니다.

\begin{enumerate}
    \item \textbf{[Inner Loop] 특정 $\alpha$에 대한 최적의 트리 찾기}
    
    먼저 $\alpha$ 값을 하나 고정합니다 (예: $\alpha=0.1$).
    알고리즘은 Full Tree($T_0$)에서 시작하여, '가장 약한 고리(Weakest Link)' (즉, 잘라냈을 때 $C(T)$가 가장 적게 증가하거나 오히려 감소하는 노드)부터 차례대로 제거합니다.
    
    이 과정을 통해 $\alpha=0.1$일 때 가능한 후보 트리들의 목록을 만듭니다:
    $\{ T_0(\text{Full}), T_1, T_2, ..., T_L(\text{Root}) \}$
    
    이 후보 트리들을 \textbf{검증 데이터(Validation Set)}에 적용하여, $\alpha=0.1$일 때 검증 성능(예: Validation MSE)이 가장 좋은 트리 $T_{\text{best\_for\_0.1}}$를 하나 선택합니다.

    \item \textbf{[Outer Loop] 모든 $\alpha$ 중 최적의 $\alpha$ 찾기}
    
    이제 \textit{다른} $\alpha$ 값 (예: $\alpha=0.2, \alpha=0.5, ...$)에 대해 1번 과정을 \textbf{반복}합니다.
    
    \begin{itemize}
        \item $\alpha=0.1$ 일 때 최적 트리: $T_{\text{best\_for\_0.1}}$ (검증 점수: 85점)
        \item $\alpha=0.2$ 일 때 최적 트리: $T_{\text{best\_for\_0.2}}$ (검증 점수: 90점)
        \item $\alpha=0.5$ 일 때 최적 트리: $T_{\text{best\_for\_0.5}}$ (검증 점수: 88점)
    \end{itemize}

    \item \textbf{최종 선택}
    
    모든 $\alpha$ 후보들 중에서 가장 높은 검증 점수(90점)를 기록한 $\alpha=0.2$와, 그 때의 트리 $T_{\text{best\_for\_0.2}}$를 \textbf{최종 모델}로 선택합니다.
\end{enumerate}


\newpage
%===============
% 4. FAQ 및 자가 점검
%===============
\section{FAQ 및 자가 점검}

강의 중 나온 퀴즈와 자주 묻는 질문들을 통해 이해도를 점검합니다.

\begin{boxExample}
    \textbf{Q1: 결정 트리 모델은 '탐욕적 알고리즘(Greedy Algorithm)'을 사용하는데, 왜 이것이 정당화되나요?}
    
    \textbf{A:} '탐욕적'이라는 것은 매 분할 시점마다 \textit{당장} MSE(또는 Gini)를 가장 많이 줄여주는 최적의 분할을 찾는다는 의미입니다. 이렇게 찾은 분할이 나중에 전체 트리의 최적(Global Optimum)을 보장하지는 않습니다.
    하지만, 모든 가능한 트리 조합을 탐색하는 것은 계산적으로 거의 불가능(NP-hard)합니다. 탐욕적 접근 방식은 계산 비용이 훨씬 저렴하면서도 \textbf{현실적으로 매우 준수한 성능의 모델}을 찾아주기 때문에 널리 사용됩니다.
\end{boxExample}

\begin{boxExample}
    \textbf{Q2: 하나의 결정 트리 안에서 동일한 변수(Feature)가 여러 번 사용될 수 있나요?}
    
    \textbf{A: 예, 가능합니다.}
    예를 들어, 상위 노드에서 "Work experience > 2"로 분할한 후, 그 자식 노드 중 하나에서 "Work experience > 4"로 \textbf{더 세분화된 분할}을 할 수 있습니다. 이는 모델이 데이터의 복잡한 관계를 더 정교하게 학습할 수 있게 해줍니다.
\end{boxExample}

\begin{boxExample}
    \textbf{Q3: 훈련된 회귀 트리(Regression Tree)가 새로운 데이터 포인트를 예측할 때, 마지막 단계는 무엇인가요?}
    
    \textbf{A:} 새로운 데이터가 트리를 따라 내려가 도달한 \textbf{단말 노드(Leaf Node)에 저장된 값}을 예측값으로 반환합니다. 이 저장된 값은 바로 해당 단말 노드에 속해있던 \textbf{훈련(Training) 데이터들의 평균(Average)} 값입니다.
\end{boxExample}

\begin{boxExample}
    \textbf{Q4: 어떤 회귀 트리가 4개의 단말 노드(Region)를 생성했습니다. 이 트리의 전체 MSE는 어떻게 계산하나요?}
    
    \begin{itemize}
        \item R1: 샘플 90개, MSE = 0.2
        \item R2: 샘플 5개, MSE = 1.2
        \item R3: 샘플 3개, MSE = 1.5
        \item R4: 샘플 2개, MSE = 1.8
    \end{itemize}
    
    \textbf{A:} 트리의 전체 MSE는 각 노드 MSE의 \textbf{가중 평균(Weighted Average)}입니다.
    
    \begin{itemize}
        \item 총 샘플 수(N) = 90 + 5 + 3 + 2 = 100
        \item 전체 MSE = $(\frac{90}{100} \times 0.2) + (\frac{5}{100} \times 1.2) + (\frac{3}{100} \times 1.5) + (\frac{2}{100} \times 1.8)$
        \item 전체 MSE = $(0.18) + (0.06) + (0.045) + (0.036) = \textbf{0.321}$
    \end{itemize}
    
    대부분의 샘플(90\%)이 R1에 속해있으므로, 전체 MSE는 R1의 MSE(0.2)에 가장 가깝게 나옵니다.
\end{boxExample}

\begin{boxExample}
    \textbf{Q5: `max_depth=1`로 설정된 결정 트리의 결정 경계(Decision Boundary)는 선형(Linear)인가요, 비선형(Non-linear)인가요?}
    
    \textbf{A: 선형(Linear)입니다.}
    `max_depth=1`은 트리가 \textbf{단 하나의 분할}만 수행한다는 의미입니다. (예: "x > 6.5")
    이는 전체 데이터 공간을 단 하나의 축에 수직인 직선(또는 초평면)으로 나누는 것과 같으므로, 결정 경계는 선형입니다. 트리의 깊이가 2 이상이 되어야 계단 형태의 비선형 경계가 만들어집니다.
\end{boxExample}

\end{document}
