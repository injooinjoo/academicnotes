%%%%%%%%%%%%%%%%%%%%%%%%%%%%%%%%%%%%%%%%%%%%%%%%%%%%%%%%%%%%%%%%%%%%%%%%%%%%%%%
% Harvard Academic Notes - 통합 마스터 템플릿
% 모든 강의 노트에 적용되는 통일된 스타일
% 버전: 2.1 - 가독성 개선 (선택적 최적화)
% 최종 수정일: 2025-11-17
%%%%%%%%%%%%%%%%%%%%%%%%%%%%%%%%%%%%%%%%%%%%%%%%%%%%%%%%%%%%%%%%%%%%%%%%%%%%%%%

\documentclass[11pt,a4paper]{article}

%========================================================================================
% 기본 패키지
%========================================================================================

% --- 한국어 지원 ---
\usepackage{kotex}

% --- 페이지 레이아웃 ---
\usepackage[top=20mm, bottom=20mm, left=20mm, right=18mm]{geometry}
\usepackage{setspace}
\onehalfspacing                      % 1.5배 줄간격
\setlength{\parskip}{0.5em}          % 문단 간격
\setlength{\parindent}{0pt}          % 들여쓰기 없음

% --- 표 관련 ---
\usepackage{booktabs}              % 고품질 표
\usepackage{tabularx}              % 자동 너비 조절 표
\usepackage{array}                 % 표 컬럼 확장
\usepackage{longtable}             % 여러 페이지 표
\renewcommand{\arraystretch}{1.1}  % 표 행간 조절

%========================================================================================
% 헤더 및 푸터
%========================================================================================

\usepackage{fancyhdr}
\pagestyle{fancy}
\fancyhf{}
\fancyhead[L]{\small\textit{CS109A: 데이터 과학 입문}}
\fancyhead[R]{\small\textit{Lecture 10}}
\fancyfoot[C]{\thepage}
\renewcommand{\headrulewidth}{0.5pt}
\renewcommand{\footrulewidth}{0.3pt}

% 첫 페이지는 헤더 없음
\fancypagestyle{firstpage}{
    \fancyhf{}
    \fancyfoot[C]{\thepage}
    \renewcommand{\headrulewidth}{0pt}
}

%========================================================================================
% 색상 정의 (파스텔 톤 + 다크모드 호환)
%========================================================================================

\usepackage[dvipsnames]{xcolor}

% 밝은 배경용 파스텔 색상
\definecolor{lightblue}{RGB}{220, 235, 255}      % 부드러운 파랑
\definecolor{lightgreen}{RGB}{220, 255, 235}     % 부드러운 초록
\definecolor{lightyellow}{RGB}{255, 250, 220}    % 부드러운 노랑
\definecolor{lightpurple}{RGB}{240, 230, 255}    % 부드러운 보라
\definecolor{lightgray}{gray}{0.95}              % 밝은 회색
\definecolor{lightpink}{RGB}{255, 235, 245}      % 부드러운 핑크
\definecolor{boxgray}{gray}{0.95}
\definecolor{boxblue}{rgb}{0.9, 0.95, 1.0}
\definecolor{boxred}{rgb}{1.0, 0.95, 0.95}

% 진한 색상 (테두리/제목용)
\definecolor{darkblue}{RGB}{50, 80, 150}
\definecolor{darkgreen}{RGB}{40, 120, 70}
\definecolor{darkorange}{RGB}{200, 100, 30}
\definecolor{darkpurple}{RGB}{100, 60, 150}

%========================================================================================
% 박스 환경 (tcolorbox) - 6가지 타입
%========================================================================================

\usepackage[most]{tcolorbox}
\tcbuselibrary{skins, breakable}

% 1. 개요 박스 (강의 시작 부분)
\newtcolorbox{overviewbox}[1][]{
    enhanced,
    colback=lightpurple,
    colframe=darkpurple,
    fonttitle=\bfseries\large,
    title=📚 강의 개요,
    arc=3mm,
    boxrule=1pt,
    left=8pt,
    right=8pt,
    top=8pt,
    bottom=8pt,
    breakable,
    #1
}

% 2. 요약 박스
\newtcolorbox{summarybox}[1][]{
    enhanced,
    colback=lightblue,
    colframe=darkblue,
    fonttitle=\bfseries,
    title=📝 핵심 요약,
    arc=2mm,
    boxrule=0.7pt,
    left=6pt,
    right=6pt,
    top=6pt,
    bottom=6pt,
    breakable,
    #1
}

% 3. 핵심 정보 박스
\newtcolorbox{infobox}[1][]{
    enhanced,
    colback=lightgreen,
    colframe=darkgreen,
    fonttitle=\bfseries,
    title=💡 핵심 정보,
    arc=2mm,
    boxrule=0.7pt,
    left=6pt,
    right=6pt,
    top=6pt,
    bottom=6pt,
    breakable,
    #1
}

% 4. 주의사항 박스
\newtcolorbox{warningbox}[1][]{
    enhanced,
    colback=lightyellow,
    colframe=darkorange,
    fonttitle=\bfseries,
    title=⚠️ 주의사항,
    arc=2mm,
    boxrule=0.7pt,
    left=6pt,
    right=6pt,
    top=6pt,
    bottom=6pt,
    breakable,
    #1
}

% 5. 예제 박스
\newtcolorbox{examplebox}[1][]{
    enhanced,
    colback=lightgray,
    colframe=black!60,
    fonttitle=\bfseries,
    title=📖 예제: #1,
    arc=2mm,
    boxrule=0.7pt,
    left=6pt,
    right=6pt,
    top=6pt,
    bottom=6pt,
    breakable,
}

% 6. 정의 박스
\newtcolorbox{definitionbox}[1][]{
    enhanced,
    colback=lightpink,
    colframe=purple!70!black,
    fonttitle=\bfseries,
    title=📌 정의: #1,
    arc=2mm,
    boxrule=0.7pt,
    left=6pt,
    right=6pt,
    top=6pt,
    bottom=6pt,
    breakable,
}

% 7. 중요 박스 (importantbox - warningbox와 유사)
\newtcolorbox{importantbox}[1][]{
    enhanced,
    colback=boxred,
    colframe=red!70!black,
    fonttitle=\bfseries,
    title=⚠️ 매우 중요: #1,
    arc=2mm,
    boxrule=0.7pt,
    left=6pt,
    right=6pt,
    top=6pt,
    bottom=6pt,
    breakable,
}

% 8. cautionbox (warningbox와 동일)
\let\cautionbox\warningbox
\let\endcautionbox\endwarningbox

%========================================================================================
% 코드 블록 설정 (밝은 배경)
%========================================================================================

\usepackage{listings}

\definecolor{codegray}{rgb}{0.5,0.5,0.5}
\definecolor{codepurple}{rgb}{0.58,0,0.82}
\definecolor{backcolour}{rgb}{0.95,0.95,0.95}

\lstset{
    basicstyle=\ttfamily\small,
    backgroundcolor=\color{lightgray},
    keywordstyle=\color{darkblue}\bfseries,
    commentstyle=\color{darkgreen}\itshape,
    stringstyle=\color{purple!80!black},
    numberstyle=\tiny\color{black!60},
    numbers=left,
    numbersep=8pt,
    breaklines=true,
    breakatwhitespace=false,
    frame=single,
    frameround=tttt,
    rulecolor=\color{black!30},
    captionpos=b,
    showstringspaces=false,
    tabsize=2,
    xleftmargin=15pt,
    xrightmargin=5pt,
    escapeinside={\%*}{*)}
}

% Python 코드 스타일
\lstdefinestyle{pythonstyle}{
    language=Python,
    morekeywords={self, True, False, None},
}

% SQL 코드 스타일
\lstdefinestyle{sqlstyle}{
    language=SQL,
    morekeywords={SELECT, FROM, WHERE, JOIN, GROUP, BY, ORDER, HAVING},
}

%========================================================================================
% 목차 스타일링
%========================================================================================

\usepackage{tocloft}
\renewcommand{\cftsecleader}{\cftdotfill{\cftdotsep}}
\setlength{\cftbeforesecskip}{0.4em}
\renewcommand{\cftsecfont}{\bfseries}
\renewcommand{\cftsubsecfont}{\normalfont}

%========================================================================================
% 표 및 그림
%========================================================================================

\usepackage{graphicx}              % 이미지
\usepackage{adjustbox}             % 표/박스 크기 조절

% 표 캡션 스타일
\usepackage{caption}
\captionsetup[table]{
    labelfont=bf,
    textfont=it,
    skip=5pt
}
\captionsetup[figure]{
    labelfont=bf,
    textfont=it,
    skip=5pt
}

%========================================================================================
% 수학
%========================================================================================

\usepackage{amsmath, amssymb, amsthm}

% 정리 환경
\theoremstyle{definition}
\newtheorem{theorem}{정리}[section]
\newtheorem{lemma}[theorem]{보조정리}
\newtheorem{proposition}[theorem]{명제}
\newtheorem{corollary}[theorem]{따름정리}
\newtheorem{definition}{정의}[section]
\newtheorem{example}{예제}[section]

%========================================================================================
% 하이퍼링크
%========================================================================================

\usepackage[
    colorlinks=true,
    linkcolor=blue!80!black,
    urlcolor=blue!80!black,
    citecolor=green!60!black,
    bookmarks=true,
    bookmarksnumbered=true,
    pdfborder={0 0 0}
]{hyperref}

% PDF 메타데이터는 각 문서에서 설정
\hypersetup{
    pdftitle={CS109A: 데이터 과학 입문 - Lecture 10},
    pdfauthor={강의 노트},
    pdfsubject={Academic Notes}
}

%========================================================================================
% 기타 유용한 패키지
%========================================================================================

\usepackage{enumitem}              % 리스트 커스터마이징
\setlist{nosep, leftmargin=*, itemsep=0.3em}

\usepackage{microtype}             % 타이포그래피 개선
\usepackage{footnote}              % 각주 개선
\usepackage{url}                   % URL 줄바꿈
\urlstyle{same}

%========================================================================================
% 사용자 정의 명령어
%========================================================================================

% 강조 텍스트
\newcommand{\important}[1]{\textbf{\textcolor{red!70!black}{#1}}}
\newcommand{\keyword}[1]{\textbf{#1}}
\newcommand{\term}[1]{\textit{#1}}
\newcommand{\code}[1]{\texttt{#1}}

% 용어 설명 (인라인)
\newcommand{\defterm}[2]{\textbf{#1}\footnote{#2}}

% 섹션 시작 전 페이지 분리
\newcommand{\newsection}[1]{\newpage\section{#1}}

%========================================================================================
% 문서 제목 스타일
%========================================================================================

\usepackage{titling}
\pretitle{\begin{center}\LARGE\bfseries}
\posttitle{\par\end{center}\vskip 0.5em}
\preauthor{\begin{center}\large}
\postauthor{\end{center}}
\predate{\begin{center}\large}
\postdate{\par\end{center}}

%========================================================================================
% 섹션 제목 간격
%========================================================================================

\usepackage{titlesec}
\titlespacing*{\section}{0pt}{1.5em}{0.8em}
\titlespacing*{\subsection}{0pt}{1.2em}{0.6em}
\titlespacing*{\subsubsection}{0pt}{1em}{0.5em}

%========================================================================================
% 메타 정보 박스 명령어
%========================================================================================

\newcommand{\metainfo}[4]{
\begin{tcolorbox}[
    colback=lightpurple,
    colframe=darkpurple,
    boxrule=1pt,
    arc=2mm,
    left=10pt,
    right=10pt,
    top=8pt,
    bottom=8pt
]
\begin{tabular}{@{}rl@{}}
▣ \textbf{강의명:} & #1 \\[0.3em]
▣ \textbf{주차:} & #2 \\[0.3em]
▣ \textbf{교수명:} & #3 \\[0.3em]
▣ \textbf{목적:} & \begin{minipage}[t]{0.75\textwidth}#4\end{minipage}
\end{tabular}
\end{tcolorbox}
}

%========================================================================================
% 끝
%========================================================================================


\begin{document}

\maketitle
\thispagestyle{firstpage}

\metainfo{CS109A: 데이터 과학 입문}{Lecture 10}{Pavlos Protopapas, Kevin Rader, Chris Gumb}{Lecture 10의 핵심 개념 학습}


\begin{summarybox}
    \textbf{이 문서의 핵심 요약} \\
    이 문서는 통계적 추론의 기본(신뢰구간, 가설검정)을 복습하고, '가능도(Likelihood)' 개념을 소개합니다. \\
    궁극적으로, 이 개념들을 조합하여 '베이즈 추론(Bayesian Inference)'의 핵심 원리를 설명합니다. \\
    베이즈 추론의 핵심은 \textbf{'데이터(증거)'를 바탕으로 '사전의 믿음(Prior)'을 '사후의 믿음(Posterior)'으로 갱신하는 것}입니다. \\
    진단 키트 예시를 통해 베이즈 정리의 직관을 배우고, 이를 통계 모델의 모수($\theta$) 추정에 적용하는 방법을 다룹니다.
\end{summarybox}

\tableofcontents

\newpage
\section{핵심 용어 정리}

본격적인 학습에 앞서, 이번 강의에서 사용되는 핵심 용어들을 정리합니다.

\begin{tcolorbox}[title=핵심 용어집 (Glossary)]
\begin{adjustbox}{width=\textwidth}
\begin{tabular}{lp{6cm}ll}
    \toprule
    \textbf{용어 (한글)} & \textbf{쉬운 설명} & \textbf{용어 (영어)} & \textbf{기호/비고} \\
    \midrule
    모수 & 우리가 알고 싶은 모집단의 실제 값 (e.g., 평균, 회귀계수) & Parameter & $\theta$, $\beta_1$, $\mu$ \\
    추정량 & 샘플 데이터를 이용해 계산한 '모수 추정치' (e.g., 표본평균) & Estimate / Estimator & $\hat{\theta}$, $\hat{\beta}_1$, $\bar{X}$ \\
    표준 오차 & 추정량이 얼마나 부정확한지(퍼져 있는지) 나타내는 척도 & Standard Error (SE) & $\hat{SE}(\hat{\beta}_1)$ \\
    신뢰 구간 & 모수가 포함될 것이라 '신뢰'하는 구간 (점 추정 + 불확실성) & Confidence Interval (CI) & e.g., $95\%$ CI \\
    예측 구간 & '새로운 단일 관측치'가 존재할 것이라 '예측'하는 구간 & Prediction Interval (PI) & 항상 신뢰구간보다 넓음 \\
    귀무 가설 & 우리가 기각하고 싶은 가설 (e.K., "관계가 없다", "차이가 없다") & Null Hypothesis & $H_0: \beta_1 = 0$ \\
    p-값 & 귀무가설이 맞다고 가정할 때, 현재 데이터를 얻을 확률 & p-value & p-value $< 0.05$ \\
    가능도 & 관찰된 데이터를 기반으로, 특정 모수가 얼마나 '그럴듯한지' & Likelihood & $L(\theta | \text{data})$ \\
    최대가능도추정 & 가능도를 최대화하는 모수 $\theta$를 찾는 방법 & Max Likelihood Est. (MLE) & \\
    사전 확률 & 데이터를 보기 전, 모수에 대해 갖는 초기 믿음(확률) & Prior Probability & $P(\theta)$, $f(\theta)$ \\
    사후 확률 & 데이터를 본 후, 갱신된 모수에 대한 믿음(확률) & Posterior Probability & $P(\theta | \text{data})$, $f(\theta|X)$ \\
    민감도 & 실제 '참'인 것을 '참'이라고 예측할 확률 (True Positive Rate) & Sensitivity & $P(T+ | D+)$ \\
    특이도 & 실제 '거짓'인 것을 '거짓'이라고 예측할 확률 (True Negative Rate) & Specificity & $P(T- | D-)$ \\
    \bottomrule
\end{tabular}
\end{adjustbox}
\captionof{table}{제10강의 핵심 통계 용어 정리}
\label{tab:glossary}
\end{tcolorbox}


\newpage
\section{통계적 추론 복습 (Inference Review)}

데이터 분석은 샘플(표본)을 통해 모집단의 특성을 알아내는 '추론' 과정입니다.
단순히 모델을 만드는 것을 넘어, 이 모델이 얼마나 신뢰할 수 있는지 평가해야 합니다.

\subsection{모집단 모델 vs. 표본 모델}

우리가 관심 있는 것은 모집단의 '진짜' 관계(Population Model)이지만, 우리가 가진 것은 샘플 데이터로 '추정'한 모델(Estimated Model)뿐입니다.

\begin{itemize}
    \item \textbf{모집단 모델 (이론):} $Y_i = \beta_0 + \beta_1 X_i + \epsilon_i$
        \begin{itemize}
            \item 우리가 찾고 싶은 진짜 값 $\beta_0, \beta_1$
            \item $\epsilon_i$: 측정 불가능한 오차 ($N(0, \sigma^2)$ 가정)
        \end{itemize}
    \item \textbf{표본 모델 (추정):} $\hat{y}_i = \hat{\beta}_0 + \hat{\beta}_1 x_i$ (e.g., $\hat{y} = 247.4 + 0.5898 \cdot x$)
        \begin{itemize}
            \item 우리가 가진 데이터로 계산한 추정치 $\hat{\beta}_0, \hat{\beta}_1$
            \item 이 값은 샘플이 달라지면 계속 바뀌는 '하나의 추측'에 불과합니다.
        \end{itemize}
\end{itemize}

\subsection{신뢰 구간 (Confidence Intervals)}

점 추정(Point Estimate) $\hat{\beta}_1 = 0.5898$은 하나의 추측일 뿐, 진짜 $\beta_1$이 0.6일지, 0.7일지, 아니면 0일지 알 수 없습니다.
이 '불확실성'을 표현하는 방법이 \textbf{신뢰 구간(CI)}입니다.

신뢰 구간을 만드는 방법은 크게 두 가지가 있습니다.
\begin{enumerate}
    \item \textbf{부트스트랩 (Bootstrap):} 데이터를 반복 재추출(resampling)하여 $\hat{\beta}_1$의 경험적 분포를 만듦 (컴퓨터 파워)
    \item \textbf{공식 기반 (Formula-based):} 확률 이론(정규분포, t-분포)에 기반한 공식을 사용 (수학)
\end{enumerate}

\subsubsection{공식 기반 신뢰 구간}

신뢰 구간은 추정치에서 표준 오차(Standard Error, SE)의 배수만큼 더하고 빼서 만듭니다.

$$ \text{95\% 신뢰 구간} = \hat{\beta}_1 \pm t^* \cdot \hat{SE}(\hat{\beta}_1) $$

\begin{itemize}
    \item $\hat{SE}(\hat{\beta}_1)$: 추정치 $\hat{\beta}_1$의 불확실성을 나타내는 척도.
    \item $t^*$: 신뢰 수준(e.g., 95\%)을 결정하는 값. (샘플이 충분히 크면 약 1.96 $\approx$ 2)
\end{itemize}

\begin{examplebox}[title=표준 오차(SE)를 줄이는 방법 (직관)]
    추정치의 불확실성($SE$)을 어떻게 줄일 수 있을까요? $SE(\hat{\beta}_1) \approx \frac{\hat{\sigma}_{\epsilon}}{\sqrt{\sum(x_i - \bar{x})^2}}$ 공식에서 힌트를 얻을 수 있습니다.

    \begin{itemize}
        \item \textbf{1. 데이터(n) 늘리기:} 샘플 크기 $n$이 커지면 분모의 $\sum(\dots)$ 항이 커져 $SE$가 줄어듭니다. (가장 확실한 방법)
        \item \textbf{2. X의 범위 넓히기:} $X$ 값들이 넓게 퍼져있을수록($Var(X)$ $\uparrow$), 분모가 커져 $SE$가 줄어듭니다. (좁은 범위의 $X$로 예측하면 불안정함)
        \item \textbf{3. 노이즈($\sigma_\epsilon$) 줄이기:} 데이터가 회귀선 주변에 빽빽하게 모여있을수록($\hat{\sigma}_\epsilon \downarrow$), 분자인 $\hat{\sigma}_\epsilon$이 작아져 $SE$가 줄어듭니다. (더 정확한 모델)
    \end{itemize}
\end{examplebox}

\subsubsection{왜 Z분포가 아닌 t-분포를 사용할까?}

\begin{itemize}
    \item \textbf{정규분포 (Z):} 모집단의 실제 노이즈($\sigma_\epsilon$)를 \textbf{알고 있을 때} 사용합니다.
    \item \textbf{t-분포 (t):} $\sigma_\epsilon$을 \textbf{모르기 때문에} 데이터의 잔차(residual)로 $\hat{\sigma}_\epsilon$를 \textbf{추정했을 때} 사용합니다.
\end{itemize}

$\sigma_\epsilon$를 추정하는 과정에서 추가적인 불확실성이 발생합니다. t-분포는 이 불확실성을 반영하기 위해 정규분포보다 \textbf{'꼬리가 두꺼운(fatter tails)'} 형태를 가집니다.
(단, 샘플 크기 $n$이 50개 이상으로 매우 커지면 t-분포는 정규분포와 거의 동일해집니다.)

\subsection{가설 검정 (Hypothesis Testing)}

"이 변수가 Y와 정말 관련이 있을까?" (e.g., $\beta_1$이 0이 아닐까?)를 통계적으로 답하는 과정입니다.

\textbf{가설 검정 5단계}
\begin{enumerate}
    \item \textbf{가설 설정:}
        \begin{itemize}
            \item \textbf{귀무 가설 ($H_0$):} 관계가 없다. ($H_0: \beta_1 = 0$)
            \item \textbf{대립 가설 ($H_A$):} 관계가 있다. ($H_A: \beta_1 \neq 0$)
        \end{itemize}
    \item \textbf{검정 통계량 선택:} t-통계량 $t = \frac{\hat{\beta}_1 - 0}{\hat{SE}(\hat{\beta}_1)}$
        (추정치가 0으로부터 표준 오차의 몇 배만큼 떨어져 있는가?)
    \item \textbf{데이터 수집 및 계산:} 샘플 데이터로 $\hat{\beta}_1$과 $\hat{SE}(\hat{\beta}_1)$을 계산하여 $t$값 확인.
    \item \textbf{결정 (p-value):}
        \begin{itemize}
            \item \textbf{p-value란?} $H_0$가 맞다(즉, $\beta_1=0$)고 가정할 때, 우리가 관찰한 $t$값보다 더 극단적인 값이 나올 확률.
            \item p-value가 매우 작으면 (e.g., $< 0.05$), "$H_0$가 맞다고 보기엔 너무 희귀한 일이 일어났군. $H_0$를 기각하자!"
        \end{itemize}
    \item \textbf{결론 도출:} "p-value가 0.001이므로, $X$와 $Y$ 사이에는 통계적으로 유의미한 연관성이 있다."
\end{enumerate}

\begin{warningbox}[title=가설 검정을 위한 부트스트랩? $\rightarrow$ 순열 검정!]
    신뢰 구간(CI)을 만들 때는 \textbf{부트스트랩(Bootstrap)}이 유용합니다. 하지만 가설 검정(p-value)에는 부트스트랩을 쓰면 안 됩니다. (Type I Error 증가 문제)

    가설 검정 시에는 \textbf{순열 검정(Permutation Test)}을 사용해야 합니다.

    \begin{itemize}
        \item \textbf{부트스트랩 (CI용):} $H_0$와 무관하게 원본 데이터에서 (복원) 재추출.
        \item \textbf{순열 검정 (p-value용):} $H_0$가 맞다(X, Y 관계 없음)고 가정하고, $Y$ 값의 순서를 무작위로 섞은(shuffle) 후 $X$와 다시 매칭하여 $\beta_1^*$을 계산.
    \end{itemize}
\end{warningbox}

\subsection{모델 가정과 추론 방법의 선택}

공식 기반 추론(t-분포)은 빠르고 편리하지만, 강력한 \textbf{모델 가정(Assumptions)}이 필요합니다.

\begin{itemize}
    \item \textbf{L}inearity (선형성): 관계는 선형이다.
    \item \textbf{I}ndependence (독립성): 오차(residual)들은 서로 독립이다. (가장 중요)
    \item \textbf{N}ormality (정규성): 오차는 정규분포를 따른다. (n이 크면(e.g., >50) 덜 중요)
    \item \textbf{E}qual Variance (등분산성): 오차의 분산은 $X$ 값에 상관없이 일정하다. (중요)
\end{itemize}

\begin{tcolorbox}[title=사례: 공식(statsmodels) vs. 부트스트랩]
    강의에서 주택 가격 데이터($n=592$)를 분석한 결과, 잔차 그림에서 $X$가 커질수록 분산이 커지는 '부채꼴 모양(fanning out)'이 나타났습니다. 이는 \textbf{등분산성(E) 가정이 위반}되었음을 의미합니다.

    이때 두 방법으로 구한 $\beta_1$ (sqft)의 95\% CI는 다음과 같습니다.
    \begin{itemize}
        \item \textbf{공식 (statsmodels):} (0.544, 0.636) $\rightarrow$ 가정이 깨져서 \textbf{잘못된(너무 좁은)} 구간.
        \item \textbf{부트스트랩:} (0.487, 0.705) $\rightarrow$ 등분산성 가정이 필요 없으므로, 이 상황에서 \textbf{더 신뢰할 수 있는} 구간.
    \end{itemize}
    결론: 모델 가정이 깨졌을 때는 부트스트랩 같은 비모수적(non-parametric) 방법이 더 강건(robust)합니다.
\end{tcolorbox}


\newpage
\section{신뢰 구간 vs. 예측 구간 (CI vs. PI)}

통계적 추론에서 가장 혼동하기 쉬운 두 가지 '구간'이 있습니다.
강의 중 두 번째 퀴즈가 이 차이점을 정확히 짚었습니다.

\begin{itemize}
    \item \textbf{신뢰 구간 (Confidence Interval):}
        \begin{itemize}
            \item \textbf{질문:} "2860 평방피트($x$)를 가진 \textbf{집단의 평균 가격}은 어디쯤 있을까?"
            \item \textbf{의미:} 회귀선 자체의 불확실성. (e.g., $\hat{y} \pm t^* \cdot \text{SE(fit)}$)
        \end{itemize}
    \item \textbf{예측 구간 (Prediction Interval):}
        \begin{itemize}
            \item \textbf{질문:} "2860 평방피트($x$)를 가진 \textbf{새로운 집 한 채}의 가격은 어디쯤 있을까?"
            \item \textbf{의미:} 회귀선의 불확실성 + 개별 데이터의 고유한 변동성(노이즈).
        \end{itemize}
\end{itemize}

\begin{warningbox}[title=예측 구간(PI)은 항상 더 넓습니다]
    예측 구간은 \textbf{두 가지 불확실성}을 모두 고려해야 합니다.

    \begin{enumerate}
        \item \textbf{모델(회귀선)의 불확실성:} $\hat{\beta}_0, \hat{\beta}_1$ 추정치의 불확실성. (신뢰구간이 다루는 영역)
        \item \textbf{개별 데이터의 불확실성 (Irreducible Error, $\hat{\sigma}_\epsilon$):} 아무리 모델이 완벽해도, 개별 주택 가격은 '진짜' 평균선 주위에 흩어져 있습니다.
    \end{enumerate}

    따라서 예측 구간의 표준 오차는 두 불확실성을 모두 합산합니다.
    $$ \text{PI} = \hat{y} \pm t^* \cdot \sqrt{(\text{SE(fit)})^2 + \hat{\sigma}_\epsilon^2} $$
    (강의 퀴즈의 D번 보기 $247.4+0.5898(2860) \pm 2\sqrt{0.023^{2}+\hat{\sigma}^{2}}$ 와 동일한 형태)
\end{warningbox}



\newpage
\section{가능도 (Likelihood)}

베이즈 추론을 이해하기 위한 핵심 구성요소인 '가능도'에 대해 알아봅니다.

\subsection{정의: PDF를 뒤집어 생각하기}

지금까지 우리는 확률 밀도 함수(PDF)를 $P(\text{data} | \theta)$로 생각했습니다.
(e.g., $\mu=0, \sigma=1$일 때, $x=2$가 나올 확률은?)

\textbf{가능도 (Likelihood)}는 이 관점을 뒤집습니다.
\begin{itemize}
    \item \textbf{데이터($x$)를 '고정된 관찰 값'으로 봅니다.}
    \item \textbf{모수($\theta$)를 '변수'로 봅니다.}
    \item \textbf{질문:} "우리가 \textbf{관찰한 데이터} $x$가 있을 때, 어떤 $\theta$ 값이 이 데이터를 가장 '그럴듯하게(likely)' 설명하는가?"
\end{itemize}

$$ L(\theta | \text{data}) = P(\text{data} | \theta) $$

수학적 형태는 PDF와 같지만, $\theta$에 대한 함수라는 점이 다릅니다.

\subsection{최대 가능도 추정 (Maximum Likelihood Estimation, MLE)}

가능도 $L(\theta | \text{data})$를 최대로 만드는 $\theta$ 값을 찾는 것을 \textbf{MLE}라고 부릅니다.

\begin{itemize}
    \item \textbf{데이터:} $x_1, x_2, \dots, x_n$ (서로 독립이라 가정)
    \item \textbf{가능도:} $L(\theta) = \prod_{i=1}^{n} P(x_i | \theta)$ (모든 데이터가 동시에 관찰될 확률)
\end{itemize}

\begin{examplebox}[title=로그 가능도 (Log-Likelihood)]
    여러 확률을 곱하는 것($\prod$)은 계산이 복잡하고 언더플로우(underflow) 위험이 있습니다.
    대신 로그(log)를 취하면 곱셈이 덧셈($\sum$)으로 바뀝니다.
    $$ \log(L(\theta)) = \sum_{i=1}^{n} \log(P(x_i | \theta)) $$
    $L(\theta)$를 최대화하는 $\theta$는 $\log(L(\theta))$를 최대화하는 $\theta$와 동일합니다.

    \textbf{예시:} $\sigma^2=4$인 정규분포에서 [3, 5, 10] (평균 6) 데이터가 관찰됨.
    $\log(L(\mu | \text{data}))$ 그래프는 $\mu=6$일 때 정확히 최댓값을 가집니다. 즉, $\hat{\mu}_{MLE} = 6 = \bar{X}$ 입니다.
\end{examplebox}

\textbf{MLE를 찾는 방법}
\begin{enumerate}
    \item \textbf{수학적 방법:} $\log(L(\theta))$를 $\theta$에 대해 미분하여 0이 되는 지점을 찾습니다.
    \item \textbf{컴퓨터 방법:} 경사 하강법(Gradient Descent)을 사용합니다.
        (단, 경사 하강법은 '최소화' 알고리즘이므로, $\log(L(\theta))$ 대신 \textbf{음의 로그 가능도 (Negative Log-Likelihood)} $-\log(L(\theta))$를 최소화합니다.)
\end{enumerate}

\begin{summarybox}[title=OLS와 MLE의 중요한 관계]
    선형 회귀(Ordinary Least Squares, OLS)와 MLE는 깊은 관련이 있습니다.

    만약 선형 회귀의 오차($\epsilon_i$)가 \textbf{정규분포를 따른다}고 가정한다면,
    가능도 $L(\beta_0, \beta_1, \sigma^2 | \text{data})$를 계산할 수 있습니다.

    이때, \textbf{음의 로그 가능도(Negative Log-Likelihood)}를 최소화하는 것은
    수학적으로 \textbf{오차 제곱합(Sum of Squared Errors, SSE)} $\sum (y_i - (\beta_0 + \beta_1 x_i))^2$ 을 최소화하는 것과 \textbf{정확히 동일}합니다.

    \textbf{결론: OLS는 '오차가 정규분포를 따른다'는 가정 하의 MLE입니다.}
\end{summarybox}


\newpage
\section{베이즈 정리 (Bayes' Rule)}

베이즈 추론의 기반이 되는 베이즈 정리는 '조건부 확률'을 뒤집는 방법을 제공합니다.

\subsection{기본 공식}

두 사건 A, B에 대하여 B가 일어났을 때 A가 일어날 확률은 다음과 같습니다.

$$ P(A|B) = \frac{P(B|A) P(A)}{P(B)} $$

\begin{examplebox}[title=CS 전공자 vs. STAT 전공자]
    어떤 수업에 STAT 전공자(A)가 40\%, CS 전공자(B)가 60\%이며, 둘 다 전공하는 학생(A and B)은 20\%라고 가정해봅시다.

    \begin{itemize}
        \item \textbf{질문 1: STAT 전공자(A) 중에서 CS도 전공(B)할 확률은?}
            $$ P(B|A) = \frac{P(A \text{ and } B)}{P(A)} = \frac{0.20}{0.40} = 0.50 \quad (50\%) $$
        \item \textbf{질문 2: CS 전공자(B) 중에서 STAT도 전공(A)할 확률은?}
            $$ P(A|B) = \frac{P(A \text{ and } B)}{P(B)} = \frac{0.20}{0.60} = 0.333 \quad (33.3\%) $$
    \end{itemize}
    이처럼 $P(A|B)$와 $P(B|A)$는 전혀 다른 값입니다. 베이즈 정리는 한쪽을 알 때 다른 쪽을 계산할 수 있게 해줍니다.
\end{examplebox}

\subsection{베이즈 정리의 직관: 믿음의 갱신 (Belief Update)}

베이즈 정리는 '증거(Evidence)'를 바탕으로 '사전의 믿음(Prior Belief)'을 '사후의 믿음(Posterior Belief)'으로 갱신하는 논리적 과정입니다.

\begin{examplebox}[title=직관적 예시: 임신 진단 키트]
    어떤 사람이 임신했을 확률($D+$)이 30\%라고 \textbf{'초기에 믿고(Prior)'} 있습니다.
    이 사람이 사용한 키트의 정보는 다음과 같습니다.
    \begin{itemize}
        \item \textbf{민감도 (Sensitivity):} $P(T+|D+) = 0.97$ (실제 임신 시, '양성'이 나올 확률)
        \item \textbf{특이도 (Specificity):} $P(T-|D-) = 0.99$
            ( $\rightarrow$ 실제 비임신 시, '양성'이 나올 확률 $P(T+|D-) = 1 - 0.99 = 0.01$ )
    \end{itemize}
    이 사람이 검사 후 \textbf{'양성($T+$)'이라는 증거(Evidence)}를 얻었습니다.
    이제 이 사람이 실제 임신($D+$)했을 확률 $P(D+|T+)$은 얼마일까요?

    베이즈 정리를 사용합니다.
    $$ P(D+|T+) = \frac{P(T+|D+) P(D+)}{P(T+)} $$
    여기서 $P(T+)$는 양성이 나올 모든 경우의 확률 (Law of Total Probability)입니다.
    $$ P(T+) = P(T+|D+)P(D+) + P(T+|D-)P(D-) $$
    $$ P(T+) = (0.97 \times 0.30) + (0.01 \times 0.70) = 0.291 + 0.007 = 0.298 $$
    
    따라서 사후 확률은:
    $$ P(D+|T+) = \frac{0.291}{0.298} \approx 0.9765 \quad (97.65\%) $$

    \textbf{결론: 30\%였던 '믿음'이 '양성'이라는 '증거'를 통해 97.65\%로 '갱신'되었습니다.}
\end{examplebox}


\newpage
\section{베이즈 추론 (Bayesian Inference)}

이제 베이즈 정리를 통계적 모수($\theta$) 추론에 적용합니다.

\subsection{베이즈 추론의 핵심 공식}

진단 키트 예시의 구성요소를 통계 모델의 모수($\theta$)와 데이터($X$)로 치환합니다.

\begin{itemize}
    \item $D+$ (실제 상태) $\rightarrow \theta$ (우리가 모르는 실제 모수)
    \item $T+$ (검사 결과) $\rightarrow X$ (우리가 관찰한 데이터)
\end{itemize}

\begin{summarybox}[title=베이즈 추론 공식]
    $$ f(\theta | X) = \frac{f(X | \theta) \cdot f(\theta)}{f(X)} $$

    이 공식의 각 요소를 이해하는 것이 베이즈 추론의 전부입니다.

    \begin{itemize}
        \item $f(\theta | X)$ --- \textbf{사후 분포 (Posterior Distribution)}
            \begin{itemize}
                \item \textbf{의미:} 데이터를 관찰한 \textbf{후}에 갱신된 $\theta$에 대한 믿음의 분포.
                \item \textbf{결과물:} 이것이 우리가 얻고자 하는 최종 결과입니다.
            \end{itemize}
        \item $f(X | \theta)$ --- \textbf{가능도 (Likelihood)}
            \begin{itemize}
                \item \textbf{의미:} 특정 모수 $\theta$를 가정했을 때, 현재 데이터 $X$가 관찰될 확률.
                \item \textbf{역할:} 데이터 $X$가 $\theta$의 어떤 값을 지지하는지 알려주는 '증거'의 역할. (앞서 배운 MLE의 그 가능도입니다.)
            \end{itemize}
        \item $f(\theta)$ --- \textbf{사전 분포 (Prior Distribution)}
            \begin{itemize}
                \item \textbf{의미:} 데이터를 관찰하기 \textbf{전}에 $\theta$에 대해 우리가 가졌던 초기 믿음.
                \item \textbf{역할:} 이전 연구나 상식 등, 데이터 외의 정보를 모델에 반영합니다.
            \end{itemize}
        \item $f(X)$ --- \textbf{증거 (Evidence) 또는 정규화 상수}
            \begin{itemize}
                \item \textbf{의미:} $f(X) = \int f(X|\theta)f(\theta) d\theta$. (모든 $\theta$에 대해 가능도를 평균낸 값)
                \item \textbf{역할:} 사후 분포 $f(\theta|X)$의 총합이 1이 되도록 맞춰주는 '상수' 역할. 계산이 복잡해서 종종 생략하고 비례 관계로 표현합니다.
            \end{itemize}
    \end{itemize}

    \textbf{더 간단한 표현: $\quad \text{Posterior} \propto \text{Likelihood} \times \text{Prior}$}
\end{summarybox}

\subsection{빈도주의 (Frequentist) vs. 베이지안 (Bayesian)}

전통적인 통계(빈도주의)와 베이즈 추론은 '확률'과 '모수'를 바라보는 근본적인 관점이 다릅니다.

\begin{adjustbox}{width=\textwidth}
\begin{tabular}{l|l|l}
    \toprule
    \textbf{항목} & \textbf{빈도주의 (Frequentist) (e.g., OLS, MLE)} & \textbf{베이지안 (Bayesian)} \\
    \midrule
    \textbf{모수($\theta$) 관점} & \textbf{고정된 상수}. (단지 우리가 모를 뿐) & \textbf{확률 변수}. (믿음의 분포를 가짐) \\
    \textbf{확률의 정의} & 장기적인 실험에서 발생하는 \textbf{빈도} & \textbf{주관적 믿음}의 정도 (Belief) \\
    \textbf{핵심 질문} & "$\theta$가 0이라는 $H_0$를 기각할 수 있는가?" & "데이터 $X$를 본 후, $\theta$에 대한 믿음은?" \\
    \textbf{결과물} & 점 추정치 $\hat{\theta}$와 \textbf{신뢰구간(CI)} & \textbf{사후 분포 $f(\theta|X)$} (전체 분포) \\
    \textbf{구간의 해석} & "이 \textbf{절차}로 CI를 100번 만들면, 95개는 $\theta$를 포함한다." & "$\theta$가 이 \textbf{구간}에 있을 확률(믿음)이 95\%이다." \\
    \bottomrule
\end{tabular}
\end{adjustbox}
\captionof{table}{빈도주의와 베이지안 추론의 관점 비교}
\label{tab:freq_vs_bayes}

\subsection{이산적 베이즈 추론 예시: 3개의 동전}

베이즈 추론이 '믿음을 갱신'하는 과정을 간단한 이산적 예시로 살펴봅니다.

\begin{examplebox}[title=어떤 동전을 뽑았을까?]
    주머니 속에 3개의 동전($\theta$)이 있습니다: $p=0.1$ (가짜), $p=0.5$ (공정), $p=0.9$ (가짜).
    이 중 하나를 랜덤하게 뽑아 4번($n=4$) 던졌더니, \textbf{3번의 앞면(H)과 1번의 뒷면(T)}이라는 데이터($X$)가 나왔습니다.

    \textbf{질문: 우리는 3개의 동전 중 어떤 것을 뽑았다고 믿어야 할까요?}

    \textbf{1. 사전 분포 (Prior) $f(\theta)$ 설정}
    데이터를 보기 전, 각 동전을 뽑을 확률은 공평하게 1/3입니다.
    $P(p=0.1) = 1/3, \quad P(p=0.5) = 1/3, \quad P(p=0.9) = 1/3$

    \textbf{2. 가능도 (Likelihood) $f(X|\theta)$ 계산}
    각 동전($\theta$)을 가정했을 때, '3H 1T' 데이터($X$)가 나올 확률 (이항분포)
    \begin{itemize}
        \item $L(p=0.1) = P(X|p=0.1) = \binom{4}{3} (0.1)^3 (0.9)^1 = 4 \times 0.001 \times 0.9 = \textbf{0.0036}$
        \item $L(p=0.5) = P(X|p=0.5) = \binom{4}{3} (0.5)^3 (0.5)^1 = 4 \times 0.125 \times 0.5 = \textbf{0.2500}$
        \item $L(p=0.9) = P(X|p=0.9) = \binom{4}{3} (0.9)^3 (0.1)^1 = 4 \times 0.729 \times 0.1 = \textbf{0.2916}$
    \end{itemize}
    (데이터 $X$는 $p=0.9$일 가능성을 가장 그럴듯하게 봅니다.)

    \textbf{3. 사후 분포 (Posterior) $\propto$ Likelihood $\times$ Prior}
    \begin{itemize}
        \item $P(p=0.1|X) \propto 0.0036 \times (1/3) = 0.0012$
        \item $P(p=0.5|X) \propto 0.2500 \times (1/3) = 0.0833$
        \item $P(p=0.9|X) \propto 0.2916 \times (1/3) = 0.0972$
    \end{itemize}

    \textbf{4. 정규화 (Normalize) $\rightarrow$ 총합을 1로 만들기}
    총합 = $0.0012 + 0.0833 + 0.0972 = 0.1817$
    \begin{itemize}
        \item $P(p=0.1|X) = 0.0012 / 0.1817 \approx \textbf{0.007 \quad (0.7\%)}$
        \item $P(p=0.5|X) = 0.0833 / 0.1817 \approx \textbf{0.458 \quad (45.8\%)}$
        \item $P(p=0.9|X) = 0.0972 / 0.1817 \approx \textbf{0.535 \quad (53.5\%)}$
    \end{itemize}

    \textbf{결론:} '3H 1T'라는 데이터를 관찰한 후, 우리의 믿음은 (1/3, 1/3, 1/3)에서 (0.7\%, 45.8\%, 53.5\%)로 \textbf{갱신}되었습니다. 이제 우리는 $p=0.9$ 동전을 뽑았다고 가장 강하게 믿게 되었습니다.
\end{examplebox}

\subsection{미리보기: 연속적인 모수 (Normal-Normal 모델)}

위 예시는 모수 $p$가 3개의 값만 가지는 이산적(discrete) 경우였습니다.
실제로는 모수($\mu$, $\beta_1$ 등)가 연속적(continuous)인 경우가 많습니다.

\begin{itemize}
    \item \textbf{모델:} $X_i \sim N(\mu, \sigma^2)$ (단, $\sigma^2$은 안다고 가정)
    \item \textbf{사전 분포 (Prior):} $\mu$에 대한 초기 믿음. (e.g., $\mu \sim N(\mu_0, \sigma_0^2)$)
    \item \textbf{데이터:} $X = \{x_1, \dots, x_n\}$ (표본 평균 $\bar{X}$)
    \item \textbf{사후 분포 (Posterior):} $f(\mu|X)$는 놀랍게도 또 다른 정규분포가 됩니다.
\end{itemize}

이때 사후 분포의 평균(우리의 최종 $\mu$ 추정치)은 다음과 같습니다.
$$ \hat{\mu}_{\text{posterior}} = \frac{(\sigma^2 \cdot \mu_0) + (n \sigma_0^2 \cdot \bar{X})}{\sigma^2 + n \sigma_0^2} $$
이 식은 \textbf{사전 믿음($\mu_0$)과 데이터($\bar{X}$)의 가중 평균}입니다.

\begin{warningbox}[title=데이터가 사전 믿음을 이긴다]
    위 가중 평균 식에서 샘플 크기 $n$이 매우 커지면($n \to \infty$) 어떻게 될까요?
    \begin{itemize}
        \item $\mu_0$의 가중치 $\rightarrow 0$
        \item $\bar{X}$의 가중치 $\rightarrow 1$
    \end{itemize}
    \textbf{결론: 데이터가 충분히 많아지면, 초기에 어떤 사전 믿음($\mu_0$)을 가졌든 상관없이 사후 분포는 데이터가 말해주는 값($\bar{X}$)으로 수렴합니다.}
\end{warningbox}


\newpage
\section{참고: `statsmodels` 결과표 분석}

강의에서 사용된 `statsmodels` (Python 라이브러리)의 OLS 회귀분석 결과표를 해석하는 방법입니다.

\begin{tcolorbox}[title=단순 선형 회귀: `price ~ sqft`]
\begin{lstlisting}[language={}, caption={단순 선형 회귀(OLS) 결과}, label={lst:ols_simple}]
==============================================================================
Dep. Variable:                  price   R-squared:                       0.519
Model:                            OLS   Adj. R-squared:                  0.518
Method:                 Least Squares   F-statistic:                     635.6
Date:                Tue, 03 Oct 2023   Prob (F-statistic):           9.97e-96
Time:                        22:00:05   Log-Likelihood:                -4566.2
No. Observations:                 592   AIC:                             9136.
Df Residuals:                     590   BIC:                             9145.
Df Model:                           1                                         
Covariance Type:            nonrobust                                         
==============================================================================
                 coef    std err          t      P>|t|      [0.025      0.975]
------------------------------------------------------------------------------
Intercept    247.4382     45.388      5.452      0.000     158.296     336.581
sqft           0.5898      0.023     25.211      0.000       0.544       0.636
==============================================================================
\end{lstlisting}

\textbf{핵심 해석:}
\begin{itemize}
    \item \textbf{coef (계수):} `Intercept`($\hat{\beta}_0$) = 247.4, `sqft`($\hat{\beta}_1$) = 0.5898
        (평방 피트가 1 증가할 때마다 가격이 0.5898 (천 달러) 증가)
    \item \textbf{std err (표준 오차):} `sqft`의 $\hat{SE}(\hat{\beta}_1)$ = 0.023
        (0.5898 이라는 추정치의 불확실성 정도)
    \item \textbf{t (t-통계량):} 25.211 = (0.5898 - 0) / 0.023
        ($H_0: \beta_1=0$ 으로부터 25 표준오차만큼 떨어져 있음)
    \item \textbf{P>|t| (p-값):} 0.000
        ($\beta_1=0$인데 이런 결과가 나올 확률이 0에 가까움 $\rightarrow H_0$ 기각)
    \item \textbf{[0.025 0.975] (95\% 신뢰구간):} (0.544, 0.636)
        (모집단의 실제 $\beta_1$이 0.544와 0.636 사이에 있을 것이라 95\% 신뢰함)
\end{itemize}
\end{tcolorbox}

\begin{tcolorbox}[title=다중 선형 회귀: `price ~ sqft + dist + ... + type`]
\begin{lstlisting}[language={}, caption={다중 선형 회귀(OLS) 결과 (일부)}, label={lst:ols_multi}]
... (R-squared: 0.733, Adj. R-squared: 0.729) ...
======================================================================================
                         coef    std err          t      P>|t|      [0.025      0.975]
--------------------------------------------------------------------------------------
Intercept          -1949.0670    745.203     -2.615      0.009   -3412.677    -485.457
type[T.multifamily] -452.2352     77.451     -5.839      0.000    -604.352    -300.119
type[T.singlefamily] 335.7612     54.642      6.145      0.000     228.441     443.081
type[T.townhouse]    -76.4372     56.859     -1.344      0.179    -188.111      35.237
sqft                   0.6411      0.044     14.720      0.000       0.556       0.727
dist                -173.5430     20.099     -8.634      0.000    -213.018    -134.067
beds                 -89.9345     23.532     -3.822      0.000    -136.152     -43.717
baths                198.4646     31.332      6.334      0.000     136.928     260.002
year                   1.2300      0.388      3.169      0.002       0.468       1.992
======================================================================================
\end{lstlisting}

\textbf{핵심 해석 (다중 회귀):}
\begin{itemize}
    \item \textbf{계수 해석의 변화:} `sqft`의 계수가 0.5898 $\rightarrow$ 0.6411로 변경되었습니다. 이는 다른 변수(dist, beds 등)의 효과를 '통제'했기 때문입니다.
    \item \textbf{`beds` 계수 (-89.93):} "다른 모든 변수(sqft, baths, dist 등)가 \textbf{동일하게 고정}되어 있다면, 침실 수가 1개 증가할 때 오히려 가격이 89.9 (천 달러) 감소한다."
    \item \textbf{인과관계 주의:} 이는 상관관계(association)일 뿐, 침실을 늘리면 집값이 떨어진다는 인과관계(causation)를 의미하지 않습니다. (e.g., 같은 크기의 집에 침실만 무리하게 늘리면 가치가 떨어질 수 있음)
\end{itemize}
\end{tcolorbox}


\newpage
\section{학습 체크리스트}

이 강의를 통해 다음 질문에 답할 수 있는지 확인해 보세요.

\begin{tcolorbox}[title=최소 학습 목표 (Checklist)]
    \begin{itemize}
        \item [$\square$] 모집단 모델($\beta_1$)과 표본 모델($\hat{\beta}_1$)의 차이를 설명할 수 있는가?
        \item [$\square$] 표준 오차(SE)가 무엇인지, 그리고 SE를 줄이는 3가지 방법을 아는가?
        \item [$\square$] 왜 $Z$-분포 대신 $t$-분포를 사용하며, 둘의 형태상 차이는 무엇인가?
        \item [$\square$] p-value가 0.05보다 작다는 것이 (e.g., $H_0: \beta_1=0$일 때) 무엇을 의미하는지 설명할 수 있는가?
        \item [$\square$] \textbf{신뢰 구간}과 \textbf{예측 구간}의 차이를 '질문'과 '불확실성'의 관점에서 설명할 수 있는가?
        \item [$\square$] 가능도(Likelihood)가 확률(Probability)과 어떻게 다른지, $L(\theta|X)$로 설명할 수 있는가?
        \item [$\square$] OLS(최소제곱법)와 MLE(최대가능도추정)의 관계는 무엇인가? (힌트: 정규분포 가정)
        \item [$\square$] 베이즈 정리를 '믿음의 갱신' 과정 (사전확률 $\rightarrow$ 증거 $\rightarrow$ 사후확률)으로 설명할 수 있는가?
        \item [$\square$] 베이즈 추론 공식의 4가지 요소($f(\theta|X), f(X|\theta), f(\theta), f(X)$)를 각각 명명하고 설명할 수 있는가?
        \item [$\square$] 빈도주의자와 베이지안이 '모수($\theta$)'를 바라보는 근본적인 관점 차이는 무엇인가?
    \end{itemize}
\end{tcolorbox}

\section{FAQ (자주 묻는 질문)}

초심자가 흔히 가질 수 있는 질문들을 정리했습니다.

\begin{warningbox}[title=Q: 부트스트랩 샘플 수($B$)를 1000번에서 100만 번으로 늘리면 신뢰구간이 더 좁아지나요?]
    \textbf{A: 아닙니다.}
    부트스트랩 샘플 수 $B$를 늘리는 것은 단지 우리가 추정한 샘플링 분포($\hat{\beta}_1$의 히스토그램)를 더 \textbf{'매끄럽게(smoother)'} 만들 뿐입니다. 이는 계산의 정밀도를 높이는 것이지, 원본 추정치의 근본적인 불확실성을 줄여주지 않습니다.

    신뢰 구간을 \textbf{좁히는(불확실성을 줄이는) 유일한 방법}은 원본 데이터의 샘플 크기 \textbf{$n$을 늘리는 것}입니다. (e.g., 500개 홈 데이터 $\rightarrow$ 5000개 홈 데이터)
\end{warningbox}

\begin{examplebox}[title=Q: 사전확률(Prior)은 그냥 '찍는' 건가요? 너무 주관적이지 않나요?]
    \textbf{A: 좋은 지적입니다. 하지만 '근거 없는 찍기'와는 다릅니다.}
    \begin{enumerate}
        \item \textbf{정보 사전확률 (Informative Prior):} 과거의 연구 결과나 해당 분야의 전문가적 합의(e.g., "회귀계수가 0에서 10 사이일 것")를 반영합니다.
        \item \textbf{무정보 사전확률 (Uninformative Prior):} 정말 아는 것이 없을 때, 모든 가능성을 평평하게(uniform) 두는 사전확률을 사용합니다. (e.g., $p$가 0~1 사이 모든 값일 확률이 동일)
        \item \textbf{데이터의 힘:} 가장 중요한 점은, \textbf{데이터($n$)가 충분히 많아지면}, 초기에 어떤 사전확률을 설정했더라도 사후확률(결과)은 \textbf{거의 같은 결론으로 수렴}한다는 것입니다. 데이터가 주관적인 믿음을 '압도'합니다.
    \end{enumerate}
\end{examplebox}

\begin{examplebox}[title=Q: 빈도주의와 베이지안 중 어느 것이 더 좋은가요?]
    \textbf{A: 상황에 따라 다릅니다. (It depends.)}
    \begin{itemize}
        \item \textbf{빈도주의 (Frequentist):} 고전적인 방법이며, 계산이 간단하고 빠릅니다. 객관적인 '절차'를 중시하는 분야(e.g., 의학 임상시험)에서 표준으로 사용됩니다.
        \item \textbf{베이지안 (Bayesian):} 계산이 복잡하지만(최근 10+년간 MCMC 등 컴퓨터 파워로 해결), '사전 지식'을 모델에 통합할 수 있습니다. 데이터가 적거나($n$이 작을 때), 모수에 대한 불확실성을 '분포' 자체로 얻고 싶을 때 매우 강력합니다.
    \end{itemize}
    현대 데이터 과학에서는 두 가지 접근법을 모두 이해하고, 문제 상황에 맞게 사용하거나 두 결과를 비교 분석하는 경우가 많습니다.
\end{examplebox}


\newpage
\section{한눈에 훑어보기 (1-Page Summary)}


\begin{tcolorbox}[title=신뢰 구간 (Mean) vs. 예측 구간 (Single)]
    \begin{itemize}
        \item \textbf{신뢰 구간 (CI):} "2860 sqft 집들의 \textbf{평균} 가격은?"
            (불확실성 1개: 모델)
        \item \textbf{예측 구간 (PI):} "2860 sqft 집 \textbf{한 채}의 가격은?"
            (불확실성 2개: 모델 + 개별 노이즈 $\hat{\sigma}_\epsilon$)
        \item $\rightarrow$ PI는 항상 CI보다 넓습니다.
    \end{itemize}
\end{tcolorbox}

\begin{tcolorbox}[title=OLS = MLE]
    선형 회귀에서 오차($\epsilon$)가 \textbf{정규분포}를 따른다고 가정하면,
    \textbf{오차 제곱합을 최소화(OLS)}하는 것은
    \textbf{가능도를 최대화(MLE)}하는 것과
    수학적으로 동일합니다.
\end{tcolorbox}

\begin{tcolorbox}[title=베이즈 정리의 핵심 흐름]
    \textbf{사전 믿음 (Prior)}
    ($P(D+)=30\%$)
    $\quad + \quad$
    \textbf{증거 (Evidence)}
    (테스트 '양성')
    $\quad \to \quad$
    \textbf{사후 믿음 (Posterior)}
    ($P(D+|T+)=97.7\%$)
\end{tcolorbox}

\begin{tcolorbox}[title=베이즈 추론의 심장]
    $$ \underbrace{f(\theta | X)}_{\text{사후 분포}} \propto \underbrace{f(X | \theta)}_{\text{가능도}} \times \underbrace{f(\theta)}_{\text{사전 분포}} $$
    (Posterior $\propto$ Likelihood $\times$ Prior)
\end{tcolorbox}

\begin{tcolorbox}[title=빈도주의 vs. 베이지안]
    \begin{itemize}
        \item \textbf{빈도주의자:} 모수($\theta$)는 \textbf{고정된 값}. 나의 추정치($\hat{\theta}$)가 확률적.
        \item \textbf{베이지안:} 모수($\theta$)는 \textbf{확률 변수}(분포). 나의 믿음이 데이터에 따라 갱신됨.
    \end{itemize}
\end{tcolorbox}

\newpage
\section{부록: 몬티 홀 문제 (Monty Hall Problem)}

강의에서 간단히 언급된 몬티 홀 문제는 조건부 확률과 베이즈 정리의 직관을 보여주는 유명한 예시입니다.

\textbf{문제 상황}
\begin{enumerate}
    \item 3개의 문(Door 1, 2, 3) 뒤에 1개는 자동차(Car), 2개는 염소(Goat)가 있습니다.
    \item 당신은 하나의 문(e.g., Door 1)을 선택합니다.
    \item 진행자(Monty)는 \textbf{당신이 선택하지 않은} 두 개의 문(Door 2, 3) 중에서, \textbf{염소가 있는 문 하나}를 열어서 보여줍니다. (e.g., Door 3에 염소가 있다고 열어줌)
    \item 진행자가 묻습니다: "처음 선택(Door 1)을 \textbf{유지}하시겠습니까, 아니면 남은 문(Door 2)으로 \textbf{바꾸}시겠습니까?"
\end{enumerate}

\textbf{결론: 무조건 바꾸는 것이 유리합니다.}
\begin{itemize}
    \item \textbf{유지(Stay)할 경우 승률: 1/3}
    \item \textbf{변경(Switch)할 경우 승률: 2/3}
\end{itemize}

\begin{examplebox}[title=왜 바꾸는 것이 유리한가?]
    \textbf{시나리오 1: 맨 처음에 '자동차'를 선택한 경우 (확률 1/3)}
    \begin{itemize}
        \item 당신: Door 1 (Car) 선택
        \item 진행자: Door 2(Goat) 또는 Door 3(Goat) 중 하나를 열어줌
        \item 당신: Door 2(Goat) 또는 Door 3(Goat)으로 \textbf{바꾼다. $\to$ 패배!}
    \end{itemize}

    \textbf{시나리오 2: 맨 처음에 '염소'를 선택한 경우 (확률 2/3)}
    \begin{itemize}
        \item 당신: Door 1 (Goat) 선택
        \item 진행자: Door 2(Car), Door 3(Goat) 중 \textbf{염소가 있는 문(Door 3)을 반드시 열어야 함}
        \item 당신: 남은 문(Door 2)으로 \textbf{바꾼다. $\to$ 승리!}
    \end{itemize}

    \textbf{요약:}
    \begin{itemize}
        \item 처음 선택을 '유지'하면, 처음에 자동차를 골랐을 때(1/3)만 이깁니다.
        \item 처음 선택을 '변경'하면, 처음에 염소를 골랐을 때(2/3) 항상 이깁니다.
    \end{itemize}
    진행자의 행동($P(\text{염소 문 열기} | \text{내 선택})$)은 새로운 정보를 제공하며, 이 정보가 조건부 확률을 변경하여 바꾸는 것의 승률을 2/3로 높여줍니다. (100개 문으로 확장하면, 99/100의 확률로 이길 수 있습니다.)
\end{examplebox}

\end{document}
