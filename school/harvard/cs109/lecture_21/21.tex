%%%%%%%%%%%%%%%%%%%%%%%%%%%%%%%%%%%%%%%%%%%%%%%%%%%%%%%%%%%%%%%%%%%%%%%%%%%%%%%
% Harvard Academic Notes - 통합 마스터 템플릿
% 모든 강의 노트에 적용되는 통일된 스타일
% 버전: 2.1 - 가독성 개선 (선택적 최적화)
% 최종 수정일: 2025-11-17
%%%%%%%%%%%%%%%%%%%%%%%%%%%%%%%%%%%%%%%%%%%%%%%%%%%%%%%%%%%%%%%%%%%%%%%%%%%%%%%

\documentclass[11pt,a4paper]{article}

%========================================================================================
% 기본 패키지
%========================================================================================

% --- 한국어 지원 ---
\usepackage{kotex}

% --- 페이지 레이아웃 ---
\usepackage[top=20mm, bottom=20mm, left=20mm, right=18mm]{geometry}
\usepackage{setspace}
\onehalfspacing                      % 1.5배 줄간격
\setlength{\parskip}{0.5em}          % 문단 간격
\setlength{\parindent}{0pt}          % 들여쓰기 없음

% --- 표 관련 ---
\usepackage{booktabs}              % 고품질 표
\usepackage{tabularx}              % 자동 너비 조절 표
\usepackage{array}                 % 표 컬럼 확장
\usepackage{longtable}             % 여러 페이지 표
\renewcommand{\arraystretch}{1.1}  % 표 행간 조절

%========================================================================================
% 헤더 및 푸터
%========================================================================================

\usepackage{fancyhdr}
\pagestyle{fancy}
\fancyhf{}
\fancyhead[L]{\small\textit{CS109A: 데이터 과학 입문}}
\fancyhead[R]{\small\textit{Lecture 21}}
\fancyfoot[C]{\thepage}
\renewcommand{\headrulewidth}{0.5pt}
\renewcommand{\footrulewidth}{0.3pt}

% 첫 페이지는 헤더 없음
\fancypagestyle{firstpage}{
    \fancyhf{}
    \fancyfoot[C]{\thepage}
    \renewcommand{\headrulewidth}{0pt}
}

%========================================================================================
% 색상 정의 (파스텔 톤 + 다크모드 호환)
%========================================================================================

\usepackage[dvipsnames]{xcolor}

% 밝은 배경용 파스텔 색상
\definecolor{lightblue}{RGB}{220, 235, 255}      % 부드러운 파랑
\definecolor{lightgreen}{RGB}{220, 255, 235}     % 부드러운 초록
\definecolor{lightyellow}{RGB}{255, 250, 220}    % 부드러운 노랑
\definecolor{lightpurple}{RGB}{240, 230, 255}    % 부드러운 보라
\definecolor{lightgray}{gray}{0.95}              % 밝은 회색
\definecolor{lightpink}{RGB}{255, 235, 245}      % 부드러운 핑크
\definecolor{boxgray}{gray}{0.95}
\definecolor{boxblue}{rgb}{0.9, 0.95, 1.0}
\definecolor{boxred}{rgb}{1.0, 0.95, 0.95}

% 진한 색상 (테두리/제목용)
\definecolor{darkblue}{RGB}{50, 80, 150}
\definecolor{darkgreen}{RGB}{40, 120, 70}
\definecolor{darkorange}{RGB}{200, 100, 30}
\definecolor{darkpurple}{RGB}{100, 60, 150}

%========================================================================================
% 박스 환경 (tcolorbox) - 6가지 타입
%========================================================================================

\usepackage[most]{tcolorbox}
\tcbuselibrary{skins, breakable}

% 1. 개요 박스 (강의 시작 부분)
\newtcolorbox{overviewbox}[1][]{
    enhanced,
    colback=lightpurple,
    colframe=darkpurple,
    fonttitle=\bfseries\large,
    title=📚 강의 개요,
    arc=3mm,
    boxrule=1pt,
    left=8pt,
    right=8pt,
    top=8pt,
    bottom=8pt,
    breakable,
    #1
}

% 2. 요약 박스
\newtcolorbox{summarybox}[1][]{
    enhanced,
    colback=lightblue,
    colframe=darkblue,
    fonttitle=\bfseries,
    title=📝 핵심 요약,
    arc=2mm,
    boxrule=0.7pt,
    left=6pt,
    right=6pt,
    top=6pt,
    bottom=6pt,
    breakable,
    #1
}

% 3. 핵심 정보 박스
\newtcolorbox{infobox}[1][]{
    enhanced,
    colback=lightgreen,
    colframe=darkgreen,
    fonttitle=\bfseries,
    title=💡 핵심 정보,
    arc=2mm,
    boxrule=0.7pt,
    left=6pt,
    right=6pt,
    top=6pt,
    bottom=6pt,
    breakable,
    #1
}

% 4. 주의사항 박스
\newtcolorbox{warningbox}[1][]{
    enhanced,
    colback=lightyellow,
    colframe=darkorange,
    fonttitle=\bfseries,
    title=⚠️ 주의사항,
    arc=2mm,
    boxrule=0.7pt,
    left=6pt,
    right=6pt,
    top=6pt,
    bottom=6pt,
    breakable,
    #1
}

% 5. 예제 박스
\newtcolorbox{examplebox}[1][]{
    enhanced,
    colback=lightgray,
    colframe=black!60,
    fonttitle=\bfseries,
    title=📖 예제: #1,
    arc=2mm,
    boxrule=0.7pt,
    left=6pt,
    right=6pt,
    top=6pt,
    bottom=6pt,
    breakable,
}

% 6. 정의 박스
\newtcolorbox{definitionbox}[1][]{
    enhanced,
    colback=lightpink,
    colframe=purple!70!black,
    fonttitle=\bfseries,
    title=📌 정의: #1,
    arc=2mm,
    boxrule=0.7pt,
    left=6pt,
    right=6pt,
    top=6pt,
    bottom=6pt,
    breakable,
}

% 7. 중요 박스 (importantbox - warningbox와 유사)
\newtcolorbox{importantbox}[1][]{
    enhanced,
    colback=boxred,
    colframe=red!70!black,
    fonttitle=\bfseries,
    title=⚠️ 매우 중요: #1,
    arc=2mm,
    boxrule=0.7pt,
    left=6pt,
    right=6pt,
    top=6pt,
    bottom=6pt,
    breakable,
}

% 8. cautionbox (warningbox와 동일)
\let\cautionbox\warningbox
\let\endcautionbox\endwarningbox

%========================================================================================
% 코드 블록 설정 (밝은 배경)
%========================================================================================

\usepackage{listings}

\definecolor{codegray}{rgb}{0.5,0.5,0.5}
\definecolor{codepurple}{rgb}{0.58,0,0.82}
\definecolor{backcolour}{rgb}{0.95,0.95,0.95}

\lstset{
    basicstyle=\ttfamily\small,
    backgroundcolor=\color{lightgray},
    keywordstyle=\color{darkblue}\bfseries,
    commentstyle=\color{darkgreen}\itshape,
    stringstyle=\color{purple!80!black},
    numberstyle=\tiny\color{black!60},
    numbers=left,
    numbersep=8pt,
    breaklines=true,
    breakatwhitespace=false,
    frame=single,
    frameround=tttt,
    rulecolor=\color{black!30},
    captionpos=b,
    showstringspaces=false,
    tabsize=2,
    xleftmargin=15pt,
    xrightmargin=5pt,
    escapeinside={\%*}{*)}
}

% Python 코드 스타일
\lstdefinestyle{pythonstyle}{
    language=Python,
    morekeywords={self, True, False, None},
}

% SQL 코드 스타일
\lstdefinestyle{sqlstyle}{
    language=SQL,
    morekeywords={SELECT, FROM, WHERE, JOIN, GROUP, BY, ORDER, HAVING},
}

%========================================================================================
% 목차 스타일링
%========================================================================================

\usepackage{tocloft}
\renewcommand{\cftsecleader}{\cftdotfill{\cftdotsep}}
\setlength{\cftbeforesecskip}{0.4em}
\renewcommand{\cftsecfont}{\bfseries}
\renewcommand{\cftsubsecfont}{\normalfont}

%========================================================================================
% 표 및 그림
%========================================================================================

\usepackage{graphicx}              % 이미지
\usepackage{adjustbox}             % 표/박스 크기 조절

% 표 캡션 스타일
\usepackage{caption}
\captionsetup[table]{
    labelfont=bf,
    textfont=it,
    skip=5pt
}
\captionsetup[figure]{
    labelfont=bf,
    textfont=it,
    skip=5pt
}

%========================================================================================
% 수학
%========================================================================================

\usepackage{amsmath, amssymb, amsthm}

% 정리 환경
\theoremstyle{definition}
\newtheorem{theorem}{정리}[section]
\newtheorem{lemma}[theorem]{보조정리}
\newtheorem{proposition}[theorem]{명제}
\newtheorem{corollary}[theorem]{따름정리}
\newtheorem{definition}{정의}[section]
\newtheorem{example}{예제}[section]

%========================================================================================
% 하이퍼링크
%========================================================================================

\usepackage[
    colorlinks=true,
    linkcolor=blue!80!black,
    urlcolor=blue!80!black,
    citecolor=green!60!black,
    bookmarks=true,
    bookmarksnumbered=true,
    pdfborder={0 0 0}
]{hyperref}

% PDF 메타데이터는 각 문서에서 설정
\hypersetup{
    pdftitle={CS109A: 데이터 과학 입문 - Lecture 21},
    pdfauthor={강의 노트},
    pdfsubject={Academic Notes}
}

%========================================================================================
% 기타 유용한 패키지
%========================================================================================

\usepackage{enumitem}              % 리스트 커스터마이징
\setlist{nosep, leftmargin=*, itemsep=0.3em}

\usepackage{microtype}             % 타이포그래피 개선
\usepackage{footnote}              % 각주 개선
\usepackage{url}                   % URL 줄바꿈
\urlstyle{same}

%========================================================================================
% 사용자 정의 명령어
%========================================================================================

% 강조 텍스트
\newcommand{\important}[1]{\textbf{\textcolor{red!70!black}{#1}}}
\newcommand{\keyword}[1]{\textbf{#1}}
\newcommand{\term}[1]{\textit{#1}}
\newcommand{\code}[1]{\texttt{#1}}

% 용어 설명 (인라인)
\newcommand{\defterm}[2]{\textbf{#1}\footnote{#2}}

% 섹션 시작 전 페이지 분리
\newcommand{\newsection}[1]{\newpage\section{#1}}

%========================================================================================
% 문서 제목 스타일
%========================================================================================

\usepackage{titling}
\pretitle{\begin{center}\LARGE\bfseries}
\posttitle{\par\end{center}\vskip 0.5em}
\preauthor{\begin{center}\large}
\postauthor{\end{center}}
\predate{\begin{center}\large}
\postdate{\par\end{center}}

%========================================================================================
% 섹션 제목 간격
%========================================================================================

\usepackage{titlesec}
\titlespacing*{\section}{0pt}{1.5em}{0.8em}
\titlespacing*{\subsection}{0pt}{1.2em}{0.6em}
\titlespacing*{\subsubsection}{0pt}{1em}{0.5em}

%========================================================================================
% 메타 정보 박스 명령어
%========================================================================================

\newcommand{\metainfo}[4]{
\begin{tcolorbox}[
    colback=lightpurple,
    colframe=darkpurple,
    boxrule=1pt,
    arc=2mm,
    left=10pt,
    right=10pt,
    top=8pt,
    bottom=8pt
]
\begin{tabular}{@{}rl@{}}
▣ \textbf{강의명:} & #1 \\[0.3em]
▣ \textbf{주차:} & #2 \\[0.3em]
▣ \textbf{교수명:} & #3 \\[0.3em]
▣ \textbf{목적:} & \begin{minipage}[t]{0.75\textwidth}#4\end{minipage}
\end{tabular}
\end{tcolorbox}
}

%========================================================================================
% 끝
%========================================================================================


\begin{document}

\metainfo{CS109A: 데이터 과학 입문}{Lecture 21}{Pavlos Protopapas, Kevin Rader, Chris Gumb}{Lecture 21의 핵심 개념 학습}

\tableofcontents
\newpage


\section{개요: 배깅(Bagging) 기법 및 OOB 오류}
\label{sec:overview}

\begin{summarybox}{문서 핵심 요약}
배깅($\mathbf{Bagging}$, \textbf{B}ootstrap \textbf{Agg}regat\textbf{ing})은 여러 모델의 예측을 결합하여 성능을 높이는 \textbf{앙상블 학습(Ensemble Learning)} 기법입니다.
단일 결정 트리의 \textbf{높은 분산(High Variance)}과 \textbf{과적합(Overfitting)} 문제를 해결하기 위해 도입되었습니다.
핵심은 \textbf{부트스트랩(Bootstrap)}을 통해 원본 데이터셋으로부터 여러 개의 새로운 훈련 데이터셋을 생성하고, 각각의 데이터셋으로 모델을 훈련시킨 후, 그 결과를 \textbf{집계(Aggregating)}하여 최종 예측을 도출하는 것입니다.
특히, \textbf{OOB 오류(Out-of-Bag Error)}는 부트스트랩 과정에서 모델 훈련에 사용되지 않은 데이터를 활용하여 별도의 \textbf{검증 세트} 없이도 모델의 성능을 평가할 수 있게 해줍니다.
\end{summarybox}

\newpage
\section{용어 정리}
\label{sec:terms}

\begin{table}[h!]
\centering
\caption{핵심 용어 및 개념 비교}
\label{tab:terms}
\begin{adjustbox}{max width=\textwidth}
\begin{tabular}{lllc}
\toprule
\textbf{용어 (원어)} & \textbf{쉬운 설명} & \textbf{기술적 정의} & \textbf{비고} \\
\midrule
\textbf{배깅 (Bagging)} & 여러 모델의 예측을 모아 평균/투표로 최종 결론을 내는 기법. & 부트스트랩 샘플로 여러 모델을 훈련 후 예측을 집계하는 앙상블 방법. & \textbf{B}ootstrap \textbf{Agg}regat\textbf{ing}의 줄임말 \\
\textbf{앙상블 학습} & 다수의 전문가(모델)에게 물어 최종 결론의 정확도를 높이는 기법. & 여러 기본 모델(Base Model)을 결합하여 단일 최적 모델을 생성. & 분류 시 \textit{다수결}, 회귀 시 \textit{평균} 사용 \\
\textbf{부트스트랩} & 원본 데이터에서 중복을 허용하며 같은 크기의 데이터를 뽑아내는 것. & \textbf{복원 추출(Sampling with Replacement)}을 통해 가상 데이터셋 생성. & 모델 학습을 위한 다양한 데이터 확보 \\
\textbf{과적합} & 모델이 훈련 데이터의 노이즈까지 암기하여 실제 데이터에서 성능이 저하되는 현상. & 모델 복잡도가 너무 높아 분산(Variance)이 증가하는 상태. & 주로 깊은 트리에 발생 \\
\textbf{OOB 오류} & 훈련에 사용되지 않은 샘플(OOB)로 측정한 오류율. & 부트스트랩 샘플에 포함되지 않은 데이터를 사용하여 모델 성능 평가. & 교차 검증(CV) 대체, 계산 비용 절감 \\
\bottomrule
\end{tabular}
\end{adjustbox}
\end{table}

\newpage
\section{앙상블 학습의 배경 및 원리}
\label{sec:ensemble_principle}

\subsection{단일 결정 트리의 한계}

단일 \textbf{결정 트리(Decision Tree)} 모델은 이해하기 쉽고(Interpretable) 학습 속도가 빠르다는 장점이 있습니다.
하지만 데이터가 복잡한 \textbf{결정 경계(Decision Boundary)}를 가질 경우, 이 경계를 축에 정렬된 분할(Axis-aligned Splits)만으로 정확히 근사하려면 트리를 깊게 만들어야 합니다.

\begin{itemize}
    \item \textbf{얕은 트리($\text{Depth} \le 4$)}: 데이터의 충분한 패턴을 포착하지 못해 \textbf{과소적합(Underfitting)}됩니다. 이는 \textbf{높은 편향(High Bias)}과 \textbf{낮은 분산(Low Variance)}을 의미합니다.
    \item \textbf{깊은 트리($\text{Depth} \ge 20$)}: 입력 공간을 너무 많이 분할하여 데이터의 \textbf{노이즈(Noise)}까지 학습하게 되어 \textbf{과적합(Overfitting)}됩니다. 이는 \textbf{낮은 편향(Low Bias)}과 \textbf{높은 분산(High Variance)}을 의미하며, 실제 환경에서 성능이 떨어집니다.
\end{itemize}

결론적으로, 단일 결정 트리는 복잡한 데이터에 대해 성능이 다른 분류/회귀 방법론에 비해 떨어지는 경우가 많습니다.

\subsection{앙상블 학습의 직관}

앙상블 학습은 단일 모델의 한계를 극복하기 위해 다수의 모델을 결합하는 기법입니다.
직관적으로, 이는 한 명의 전문가(단일 모델)에게만 의존하는 것이 아니라, 여러 전문가에게서 의견을 듣고 \textbf{종합적인 결론}을 내리는 것과 같습니다.

\begin{examplebox}{현실 비유: MRI 진단}
뇌종양 진단을 위해 환자의 MRI 스캔을 여러 의사(모델)에게 보냅니다.
\begin{itemize}
    \item 각 의사는 각기 다른 훈련 데이터(병원 A, B, C의 사례)로 학습했습니다.
    \item 각 의사는 스캔을 보고 '종양 있음(Yes)' 또는 '종양 없음(No)'을 예측합니다.
    \item 최종 진단은 모든 의사의 의견을 모아 \textbf{다수결(Plurality)}로 결정합니다. (예: 3명 중 2명이 'No'이면 최종 결론은 'No')
\end{itemize}
이처럼 다수의 모델의 예측을 취합하면, 개별 모델의 \textbf{고유한 오류(Error)}는 서로 상쇄되어 최종 예측의 \textbf{정확도(Accuracy)}가 향상됩니다.
\end{examplebox}

\subsection{앙상블 학습의 기술적 정의 및 종류}

앙상블 학습은 여러 기본 모델을 결합하여 하나의 최적 예측 모델을 만드는 기계 학습 기법입니다.

\begin{itemize}
    \item \textbf{분류(Classification)}: 여러 모델의 예측 중 \textbf{가장 많은 표}를 얻은 클래스(다수결)를 최종 예측으로 반환합니다.
    \item \textbf{회귀(Regression)}: 여러 모델의 예측 값들의 \textbf{평균}을 최종 예측으로 반환합니다.
\end{itemize}

앙상블 방법의 주요 목표는 \textbf{편향(Bias)}과 \textbf{분산(Variance)}을 줄이는 데 있습니다. 다양한 앙상블 기법이 있으며, 이들은 기본 모델을 구성하고 결합하는 방식에 따라 나뉩니다.

\begin{table}[h!]
\centering
\caption{주요 앙상블 학습 기법}
\label{tab:ensemble_methods}
\begin{tabular}{lll}
\toprule
\textbf{기법} & \textbf{핵심 원리} & \textbf{주요 목표} \\
\midrule
\textbf{배깅 (Bagging)} & 부트스트랩 샘플로 병렬 훈련 후 예측을 집계(평균/다수결) & \textbf{분산(Variance) 감소} \\
\textbf{부스팅 (Boosting)} & 이전 모델의 오류를 보완하도록 순차적으로 모델 훈련 & \textbf{편향(Bias) 감소} \\
\textbf{스태킹 (Stacking)} & 기본 모델의 예측 결과를 새로운 메타 모델의 입력으로 사용 & 성능 최적화 \\
\textbf{블렌딩 (Blending)} & 검증 세트에서 메타 모델을 훈련시키는 스태킹의 변형 & 성능 최적화 \\
\bottomrule
\end{tabular}
\end{table}

\newpage
\section{배깅(Bagging)의 원리: 부트스트랩 + 집계}
\label{sec:bagging_principle}

배깅(\textbf{B}ootstrap \textbf{Agg}regat\textbf{ing})은 앙상블 학습 중 가장 기본이 되는 기법입니다. 다수의 모델을 훈련시키기 위해 다양한 훈련 데이터셋을 생성하고, 그 결과를 모아 최종 예측을 합니다.

\subsection{1단계: 부트스트랩 (Bootstrap)}

배깅을 위해 여러 모델을 훈련시키려면 \textbf{다양한 훈련 데이터셋}이 필요합니다. 하지만 현실에서는 하나의 원본 데이터셋만 주어지는 경우가 대부분입니다. 이 문제를 해결하기 위해 \textbf{부트스트랩(Bootstrapping)} 기법을 사용합니다.

\begin{itemize}
    \item \textbf{정의}: 원본 데이터셋에서 \textbf{복원 추출(Sampling with Replacement)}을 통해 여러 개의 새로운 데이터셋을 생성하는 과정입니다.
    \item \textbf{특징}:
    \begin{enumerate}
        \item 새로 생성된 부트스트랩 데이터셋의 크기는 원본 데이터셋과 같습니다.
        \item 복원 추출로 인해 원본 데이터의 일부 관측치($\sim 36.8\%$)는 샘플에 포함되지 않고, 일부 관측치는 중복되어 포함됩니다.
    \end{enumerate}
    \item \textbf{목적}: 모델을 훈련시킬 때마다 데이터셋을 다르게 제공하여, 개별 모델의 \textbf{독립성}을 확보하고 \textbf{다양성(Diversity)}을 부여합니다.
\end{itemize}

\subsection{2단계: 집계 (Aggregating)}

부트스트랩 샘플을 준비했다면, 이 샘플들을 이용해 여러 개의 기본 모델(여기서는 깊은 결정 트리)을 훈련시키고, 이 모델들의 예측을 결합하여 최종 예측을 도출합니다.

\begin{figure}[h!]
    \centering
    \includegraphics[width=0.9\textwidth]{example_bagging_process.png}
    \caption{배깅 과정: 부트스트랩 및 집계의 결합}
    \label{fig:bagging_process}
\end{figure}

\begin{enumerate}
    \item \textbf{부트스트랩 샘플 생성}: 원본 데이터에서 $N$개의 부트스트랩 샘플을 생성합니다.
    \item \textbf{모델 훈련}: 각 샘플에 대해 $N$개의 결정 트리($\text{Tree } 1, \dots, \text{Tree } N$)를 훈련시킵니다. 일반적으로 깊은 트리($\text{Deep Tree}$)를 사용하여 \textbf{높은 표현력(High Expressiveness)}을 가지도록 합니다.
    \item \textbf{예측 및 집계}: 새로운 테스트 데이터가 들어오면, $N$개의 모든 트리가 예측($\text{Prediction } 1, \dots, \text{Prediction } N$)을 수행합니다.
    \begin{itemize}
        \item 분류: $N$개의 예측 중 \textbf{다수결}로 최종 예측을 결정합니다.
        \item 회귀: $N$개의 예측 값의 \textbf{평균}을 최종 예측으로 결정합니다.
    \end{itemize}
\end{enumerate}

\subsection{배깅의 장점}
배깅은 특히 단일 깊은 트리가 가진 문제를 효과적으로 해결합니다.

\begin{itemize}
    \item \textbf{높은 표현력(High Expressiveness)}: 개별 트리가 깊이를 제한받지 않고 복잡한 패턴을 학습할 수 있게 하여, 복잡한 함수와 결정 경계를 근사할 수 있습니다.
    \item \textbf{낮은 분산(Low Variance)}: 여러 모델의 예측을 평균/다수결로 \textbf{집계(Averaging)}함으로써, 개별 모델이 가진 과적합 경향(높은 분산)을 상쇄하고 최종 예측의 분산을 크게 줄입니다.
    \item \textbf{과적합 완화}: 개별 트리는 과적합될 수 있지만(높은 분산), 다양한 데이터셋으로 훈련되었기 때문에 그 오류가 독립적이며, 집계 과정을 통해 서로 상쇄됩니다.
\end{itemize}

\begin{cautionbox}{주의: 깊이 제어의 필요성}
이론적으로 배깅은 트리를 끝까지 키워도(Full Tree) 평균화로 인해 과적합 문제가 해소된다고 설명되기도 합니다.
그러나 실제로는 여전히 트리의 \textbf{최대 깊이(Max Depth)} 같은 \textbf{정지 조건(Stopping Condition)}을 제어해야 합니다.
\begin{itemize}
    \item \textbf{너무 얕은 트리}: 여러 개의 과소적합된(Underfit) 트리를 합쳐도 여전히 \textbf{과소적합}된 결과를 낼 수 있습니다.
    \item \textbf{너무 깊은 트리}: 트리의 깊이가 너무 깊으면 계산 비용이 커지고, 부트스트랩 샘플 간의 상관관계(Correlation)가 증가하여 분산 감소 효과가 줄어들 수 있습니다.
\end{itemize}
따라서, 배깅에서도 \textbf{교차 검증(Cross-Validation)}을 통해 개별 트리의 적절한 깊이(Complexity)를 결정해야 합니다.
\end{cautionbox}

\newpage
\section{배깅의 성능 제어 하이퍼파라미터}
\label{sec:hyperparameters}

배깅 모델에는 단일 결정 트리에서 사용되는 매개변수 외에 추가적인 하이퍼파라미터가 있습니다.

\subsection{1. 나무 깊이(Tree Depth)}

\begin{itemize}
    \item \textbf{역할}: 개별 트리의 \textbf{모델 복잡도(Complexity)}를 제어합니다.
    \item \textbf{원리}: 단일 트리와 마찬가지로, 깊이를 제어하는 것은 \textbf{편향-분산 상충 관계(Bias-Variance Trade-off)}를 제어하는 것과 같습니다.
    \item \textbf{결정}: \textbf{교차 검증(Cross Validation, CV)} 또는 \textbf{OOB 오류}를 통해 최적의 깊이를 선택합니다.
\end{itemize}

\subsection{2. 추정기 개수(Number of Estimators, $N$)}

\begin{itemize}
    \item \textbf{역할}: 앙상블에 참여하는 나무(\textbf{Estimator})의 개수입니다. 이는 \textbf{모델 복잡도}를 직접 제어하기보다는 \textbf{분산}을 제어합니다.
    \item \textbf{원리}: 추정기 개수를 늘릴수록 예측의 \textbf{분산(Variance)}이 감소하고 모델이 \textbf{안정화(Stabilization)}됩니다.
    \item \textbf{과적합 위험}: 추정기 개수를 늘리는 것은 \textbf{과적합의 위험을 증가시키지 않습니다.} 나무의 수가 많아질수록 분산은 계속 감소하다가 안정적인 값에 수렴합니다.
\end{itemize}

\begin{figure}[h!]
    \centering
    \includegraphics[width=0.8\textwidth]{example_estimator_error.png}
    \caption{추정기 개수 증가에 따른 오류 변화 (나무 깊이 고정)}
    \label{fig:estimator_error}
\end{figure}

그림 \ref{fig:estimator_error}에서 볼 수 있듯이, 추정기($N$) 개수가 증가할수록 \textbf{검증 오류(Validation Error)}는 꾸준히 감소하다가 특정 시점 이후 안정화됩니다. 이는 분산 감소 효과를 보여줍니다.

\begin{cautionbox}{잠재적 문제점: 복잡한 하이퍼파라미터 튜닝}
배깅은 단일 트리보다 훨씬 많은 하이퍼파라미터를 최적화해야 합니다.
\begin{itemize}
    \item \textbf{기준}: 지니 불순도, 엔트로피, MSE 등
    \item \textbf{정지 조건}: 최대 깊이, 리프당 최소 샘플 수 등
    \item \textbf{추정기 수}: $N$ (나무 개수)
\end{itemize}
이 모든 조합을 \textbf{K-겹 교차 검증(K-Fold CV)}으로 최적화하려면 수천 개의 트리를 훈련해야 하므로 \textbf{계산 비용(Computational Cost)}이 매우 커집니다. 이는 OOB 오류 개념이 필요한 주요 배경입니다.
\end{cautionbox}

\newpage
\section{OOB 오류 (Out-of-Bag Error)}
\label{sec:oob_error}

\subsection{OOB 오류의 개념}

\textbf{OOB 오류(Out-of-Bag Error)}는 배깅 과정의 부산물로 얻어지는 성능 측정 지표입니다. 별도의 \textbf{검증 데이터셋(Validation Set)}을 만들 필요 없이 모델의 일반화 성능을 측정할 수 있게 해줍니다.

\begin{itemize}
    \item \textbf{발생 배경}: 부트스트랩 과정에서 원본 데이터의 약 $\sim 36.8\%$는 개별 트리를 훈련시키는 데 사용되지 않습니다. 이 사용되지 않은 데이터($\text{Out-of-Bag}$)를 해당 트리의 \textbf{검증 데이터}로 활용합니다.
    \item \textbf{목적}:
    \begin{enumerate}
        \item \textbf{일반화 능력 측정}: 훈련 과정에서 생성된 데이터를 활용하여 모델의 \textbf{일반화(Generalizability)} 능력을 측정합니다.
        \item \textbf{검증 세트 대체}: 교차 검증을 위한 별도의 \textbf{검증 세트}를 만들 필요가 없어 계산 비용을 절감하고 데이터 활용도를 높입니다.
    \end{enumerate}
\end{itemize}

\subsection{OOB 오류 계산 절차}

OOB 오류는 전체 훈련 세트의 모든 관측치($i=1$부터 $N$까지)에 대해 \textbf{점별 OOB 예측(Point-wise OOB Prediction)}을 계산하고, 이 오류들을 평균하여 구합니다.

\begin{enumerate}
    \item \textbf{OOB 모델 식별}: 전체 앙상블에서 특정 관측치 $\mathbf{x}_i$를 훈련에 사용하지 않은 모든 트리(모델 집합 $\mathbf{B}_{i}^{\text{OOB}}$)를 식별합니다.
    \item \textbf{점별 예측($\hat{y}_{i,pw}$)}: 식별된 OOB 트리들($\mathbf{B}_{i}^{\text{OOB}}$)의 예측을 집계하여 $\mathbf{x}_i$에 대한 \textbf{점별 예측} $\hat{y}_{i,pw}$를 구합니다.
    \begin{itemize}
        \item 분류: $\hat{y}_{i,pw} = \text{majority}(\hat{y}_{i}^{j})$ ($\forall j \in \mathbf{B}_{i}^{\text{OOB}}$)
        \item 회귀: $\hat{y}_{i,pw} = \frac{1}{|\mathbf{B}_{i}^{\text{OOB}}|}\sum_{j \in \mathbf{B}_{i}^{\text{OOB}}}\hat{y}_{i,j}$
    \end{itemize}
    \item \textbf{점별 오류($e_i$)}: 점별 예측과 실제 값($y_i$)을 비교하여 \textbf{점별 OOB 오류}를 계산합니다.
    \begin{itemize}
        \item 분류: $e_{i}=\mathbb{I}(\hat{y}_{i,pw}\ne y_{i})$ ($\mathbb{I}$는 지시 함수, 오류 시 1, 정확 시 0)
        \item 회귀: $e_{i}=(y_{i}-\hat{y}_{i,pw})^{2}$ (평균 제곱 오차)
    \end{itemize}
    \item \textbf{OOB 최종 오류}: 전체 훈련 세트($N$)의 모든 점별 오류를 평균합니다.
    \begin{itemize}
        \item 분류: $Error_{\text{OOB}}=\frac{1}{N}\sum_{i=1}^{N}e_{i}$ (오류율)
        \item 회귀: $Error_{\text{OOB}}=\frac{1}{N}\sum_{i=1}^{N}e_{i}$ (평균 제곱 오차)
    \end{itemize}
\end{enumerate}

\subsection{OOB 오류의 장점}

OOB 오류는 교차 검증(CV)에 비해 다음과 같은 이점을 제공합니다.

\begin{itemize}
    \item \textbf{계산 효율성}: 별도의 CV 절차 없이도 모델의 성능을 측정할 수 있어 \textbf{계산 비용}을 절감합니다.
    \item \textbf{데이터 누출 방지}: CV에서는 검증 세트가 다른 폴드의 훈련 세트로 사용될 수 있어 데이터 정보가 미세하게 \textbf{유출(Leakage)}될 위험이 있지만, OOB 데이터는 해당 트리 훈련에 전혀 사용되지 않았으므로 누출 위험이 낮아 더 \textbf{강건한(Robust)} 성능 평가를 제공합니다.
\end{itemize}

\begin{cautionbox}{OOB와 테스트 세트}
OOB 오류는 \textbf{검증 세트}를 대체하는 수단입니다. OOB 오류를 사용하여 \textbf{하이퍼파라미터 튜닝}이나 \textbf{모델 선택}을 수행합니다.

최종적으로 모델의 성능을 평가하고 대중에게 공개할 때는, 훈련과 검증에 \textbf{전혀 사용되지 않은} 독립적인 \textbf{테스트 세트(Test Set)}를 반드시 사용해야 합니다. 테스트 세트는 모델 성능 평가를 위한 '마지막 보루'로 끝까지 따로 보관해야 합니다.
\end{cautionbox}

\newpage
\section{배깅의 주요 단점 및 다음 과제}
\label{sec:drawbacks}

배깅은 단일 결정 트리의 분산 문제를 해결하지만, 새로운 문제점과 한계도 발생합니다.

\subsection{1. 해석력(Interpretability) 상실}

\begin{itemize}
    \item \textbf{문제}: 단일 결정 트리는 '이유와 결론'을 \textbf{논리적인 흐름(Logical Flow)}으로 추적할 수 있어 해석력이 높았습니다. 그러나 배깅은 수많은 트리의 예측을 단순히 평균하거나 다수결로 집계하기 때문에, 최종 예측이 어떤 논리적 경로를 따랐는지 \textbf{추적하는 것이 불가능}합니다.
    \item \textbf{해결 방향}: 이는 앙상블 학습의 공통적인 문제이며, \textbf{랜덤 포레스트(Random Forest)} 강의에서 \textbf{MDI (Mean Decrease Impurity)}나 \textbf{순열 중요도(Permutation Importance)}와 같은 \textbf{특성 중요도(Feature Importance)} 방법을 통해 간접적인 해석력을 확보하는 방법을 다룰 예정입니다.
\end{itemize}

\subsection{2. 과소적합/과적합의 여전한 위험}

배깅이 분산을 줄여주지만, 개별 트리의 복잡도를 적절히 제어하지 않으면 단일 트리와 같은 문제가 여전히 발생할 수 있습니다.

\begin{itemize}
    \item \textbf{과소적합}: 개별 트리가 \textbf{너무 얕으면(Shallow)} 데이터의 실제 패턴을 포착하지 못합니다. 과소적합된 여러 트리를 합쳐도 결과적으로 \textbf{과소적합}될 위험이 있습니다.
    \item \textbf{과적합}: 개별 트리가 \textbf{너무 깊으면(Deep)} 앙상블 모델이 여전히 \textbf{과적합}될 수 있습니다. (다만, 단일 트리보다는 완화됩니다.)
\end{itemize}

\subsection{3. 나무 간 상관관계 (Correlation of Trees)}

이것은 배깅의 근본적인 한계로, 다음 강의에서 랜덤 포레스트가 해결할 핵심 과제입니다.

\begin{itemize}
    \item \textbf{문제}: 앙상블 학습은 개별 모델의 오류가 \textbf{독립적}일수록(Independent) 집계 효과가 극대화된다고 가정합니다. 그러나 현실에서는 \textbf{나무 간의 상관관계(Correlation)}가 매우 높게 나타납니다.
    \item \textbf{원인}: 데이터셋에 \textbf{극도로 강력한 예측 변수($x_j$)}가 존재할 경우, 모든 부트스트랩 샘플은 해당 변수를 이용해 \textbf{트리의 최상단 노드(Root Node)}를 분할하는 경향이 있습니다.
    \item \textbf{결과}: 모든 트리가 비슷한 모양으로 시작되어 예측이 \textbf{매우 유사해집니다.} 이는 분산 감소 효과를 저해하고, 결과적으로 예측 성능 향상에 한계가 생깁니다.
\end{itemize}

\begin{examplebox}{트리의 상관관계 예시}
당뇨병 예측 모델을 만들 때, 혈당 수치(\texttt{Glucose})가 압도적으로 중요한 특성이라고 가정해 봅시다.

\begin{itemize}
    \item \textbf{트리 1의 루트}: \texttt{Glucose}로 분할
    \item \textbf{트리 2의 루트}: \texttt{Glucose}로 분할
    \item \textbf{트리 N의 루트}: \texttt{Glucose}로 분할
\end{itemize}
모든 트리가 동일한 첫 번째 분할을 하기 때문에, 그 이후의 구조가 달라도 \textbf{서로 높은 상관관계}를 가지게 됩니다. 이 문제는 \textbf{랜덤 포레스트}에서 \textbf{특성 랜덤성(Feature Randomness)}을 추가하여 해결하게 됩니다.
\end{examplebox}

\newpage
\section{빠르게 훑어보기: 핵심 체크리스트}

\begin{summarybox}{배깅(Bagging) 및 OOB 핵심 정리}
\begin{itemize}
    \item \textbf{배깅이란?} \textbf{B}ootstrap \textbf{Agg}regat\textbf{ing}의 줄임말. 깊은 트리를 \textbf{병렬}로 훈련 후 결과를 \textbf{평균/다수결}로 집계하여 최종 예측을 수행합니다.
    \item \textbf{도입 목적}: 단일 결정 트리의 \textbf{높은 분산(Variance)}과 \textbf{과적합(Overfitting)} 문제 해결.
    \item \textbf{하이퍼파라미터}: 나무의 깊이(복잡도 제어), 추정기 수(분산 제어).
    \item \textbf{추정기 수 증가}: 분산을 \textbf{감소}시키며, 과적합의 위험이 \textbf{없다}.
    \item \textbf{집계 방식}:
    \begin{itemize}
        \item 분류: \textbf{다수결(Plurality)}
        \item 회귀: \textbf{평균(Average)}
    \end{itemize}
    \item \textbf{OOB 오류}: 부트스트랩에 사용되지 않은 $\sim 36.8\%$의 데이터를 활용하여 \textbf{검증 세트}를 대체하는 효율적인 성능 측정 지표.
    \item \textbf{OOB 장점}: \textbf{계산 효율성} 향상 및 \textbf{데이터 누출} 없이 강건한 성능 측정.
    \item \textbf{단점}: 모델 \textbf{해석력(Interpretability)} 상실 및 나무 간 \textbf{높은 상관관계} 문제.
\end{itemize}
\end{summarybox}

\end{document}
