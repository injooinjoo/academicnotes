%%%%%%%%%%%%%%%%%%%%%%%%%%%%%%%%%%%%%%%%%%%%%%%%%%%%%%%%%%%%%%%%%%%%%%%%%%%%%%%
% Harvard Academic Notes - 통합 마스터 템플릿
% 모든 강의 노트에 적용되는 통일된 스타일
% 버전: 2.0
% 최종 수정일: 2025-10-26
%%%%%%%%%%%%%%%%%%%%%%%%%%%%%%%%%%%%%%%%%%%%%%%%%%%%%%%%%%%%%%%%%%%%%%%%%%%%%%%

\documentclass[11pt,a4paper]{article}

%========================================================================================
% 기본 패키지
%========================================================================================

% --- 한국어 지원 ---
\usepackage{kotex}

% --- 페이지 레이아웃 ---
\usepackage[margin=25mm]{geometry}
\usepackage{setspace}
\onehalfspacing                      % 1.5배 줄간격
\setlength{\parskip}{0.6em}          % 문단 간격
\setlength{\parindent}{0pt}          % 들여쓰기 없음

% --- 표 관련 ---
\usepackage{booktabs}              % 고품질 표
\usepackage{tabularx}              % 자동 너비 조절 표
\usepackage{array}                 % 표 컬럼 확장
\usepackage{longtable}             % 여러 페이지 표
\renewcommand{\arraystretch}{1.2}  % 표 행간 조절

%========================================================================================
% 헤더 및 푸터
%========================================================================================

\usepackage{fancyhdr}
\pagestyle{fancy}
\fancyhf{}
\fancyhead[L]{\small\textit{CS109A: 데이터 과학 입문}}
\fancyhead[R]{\small\textit{Lecture 01}}
\fancyfoot[C]{\thepage}
\renewcommand{\headrulewidth}{0.5pt}
\renewcommand{\footrulewidth}{0.3pt}

% 첫 페이지는 헤더 없음
\fancypagestyle{firstpage}{
    \fancyhf{}
    \fancyfoot[C]{\thepage}
    \renewcommand{\headrulewidth}{0pt}
}

%========================================================================================
% 색상 정의 (파스텔 톤 + 다크모드 호환)
%========================================================================================

\usepackage[dvipsnames]{xcolor}

% 밝은 배경용 파스텔 색상
\definecolor{lightblue}{RGB}{220, 235, 255}      % 부드러운 파랑
\definecolor{lightgreen}{RGB}{220, 255, 235}     % 부드러운 초록
\definecolor{lightyellow}{RGB}{255, 250, 220}    % 부드러운 노랑
\definecolor{lightpurple}{RGB}{240, 230, 255}    % 부드러운 보라
\definecolor{lightgray}{gray}{0.95}              % 밝은 회색
\definecolor{lightpink}{RGB}{255, 235, 245}      % 부드러운 핑크
\definecolor{boxgray}{gray}{0.95}
\definecolor{boxblue}{rgb}{0.9, 0.95, 1.0}
\definecolor{boxred}{rgb}{1.0, 0.95, 0.95}

% 진한 색상 (테두리/제목용)
\definecolor{darkblue}{RGB}{50, 80, 150}
\definecolor{darkgreen}{RGB}{40, 120, 70}
\definecolor{darkorange}{RGB}{200, 100, 30}
\definecolor{darkpurple}{RGB}{100, 60, 150}

%========================================================================================
% 박스 환경 (tcolorbox) - 6가지 타입
%========================================================================================

\usepackage[most]{tcolorbox}
\tcbuselibrary{skins, breakable}

% 1. 개요 박스 (강의 시작 부분)
\newtcolorbox{overviewbox}[1][]{
    enhanced,
    colback=lightpurple,
    colframe=darkpurple,
    fonttitle=\bfseries\large,
    title=📚 강의 개요,
    arc=3mm,
    boxrule=1pt,
    left=8pt,
    right=8pt,
    top=8pt,
    bottom=8pt,
    breakable,
    #1
}

% 2. 요약 박스
\newtcolorbox{summarybox}[1][]{
    enhanced,
    colback=lightblue,
    colframe=darkblue,
    fonttitle=\bfseries,
    title=📝 핵심 요약,
    arc=2mm,
    boxrule=0.7pt,
    left=6pt,
    right=6pt,
    top=6pt,
    bottom=6pt,
    breakable,
    #1
}

% 3. 핵심 정보 박스
\newtcolorbox{infobox}[1][]{
    enhanced,
    colback=lightgreen,
    colframe=darkgreen,
    fonttitle=\bfseries,
    title=💡 핵심 정보,
    arc=2mm,
    boxrule=0.7pt,
    left=6pt,
    right=6pt,
    top=6pt,
    bottom=6pt,
    breakable,
    #1
}

% 4. 주의사항 박스
\newtcolorbox{warningbox}[1][]{
    enhanced,
    colback=lightyellow,
    colframe=darkorange,
    fonttitle=\bfseries,
    title=⚠️ 주의사항,
    arc=2mm,
    boxrule=0.7pt,
    left=6pt,
    right=6pt,
    top=6pt,
    bottom=6pt,
    breakable,
    #1
}

% 5. 예제 박스
\newtcolorbox{examplebox}[1][]{
    enhanced,
    colback=lightgray,
    colframe=black!60,
    fonttitle=\bfseries,
    title=📖 예제: #1,
    arc=2mm,
    boxrule=0.7pt,
    left=6pt,
    right=6pt,
    top=6pt,
    bottom=6pt,
    breakable,
}

% 6. 정의 박스
\newtcolorbox{definitionbox}[1][]{
    enhanced,
    colback=lightpink,
    colframe=purple!70!black,
    fonttitle=\bfseries,
    title=📌 정의: #1,
    arc=2mm,
    boxrule=0.7pt,
    left=6pt,
    right=6pt,
    top=6pt,
    bottom=6pt,
    breakable,
}

% 7. 중요 박스 (importantbox - warningbox와 유사)
\newtcolorbox{importantbox}[1][]{
    enhanced,
    colback=boxred,
    colframe=red!70!black,
    fonttitle=\bfseries,
    title=⚠️ 매우 중요: #1,
    arc=2mm,
    boxrule=0.7pt,
    left=6pt,
    right=6pt,
    top=6pt,
    bottom=6pt,
    breakable,
}

% 8. cautionbox (warningbox와 동일)
\let\cautionbox\warningbox
\let\endcautionbox\endwarningbox

%========================================================================================
% 코드 블록 설정 (밝은 배경)
%========================================================================================

\usepackage{listings}

\definecolor{codegray}{rgb}{0.5,0.5,0.5}
\definecolor{codepurple}{rgb}{0.58,0,0.82}
\definecolor{backcolour}{rgb}{0.95,0.95,0.95}

\lstset{
    basicstyle=\ttfamily\small,
    backgroundcolor=\color{lightgray},
    keywordstyle=\color{darkblue}\bfseries,
    commentstyle=\color{darkgreen}\itshape,
    stringstyle=\color{purple!80!black},
    numberstyle=\tiny\color{black!60},
    numbers=left,
    numbersep=8pt,
    breaklines=true,
    breakatwhitespace=false,
    frame=single,
    frameround=tttt,
    rulecolor=\color{black!30},
    captionpos=b,
    showstringspaces=false,
    tabsize=2,
    xleftmargin=15pt,
    xrightmargin=5pt,
    escapeinside={\%*}{*)}
}

% Python 코드 스타일
\lstdefinestyle{pythonstyle}{
    language=Python,
    morekeywords={self, True, False, None},
}

% SQL 코드 스타일
\lstdefinestyle{sqlstyle}{
    language=SQL,
    morekeywords={SELECT, FROM, WHERE, JOIN, GROUP, BY, ORDER, HAVING},
}

%========================================================================================
% 목차 스타일링
%========================================================================================

\usepackage{tocloft}
\renewcommand{\cftsecleader}{\cftdotfill{\cftdotsep}}
\setlength{\cftbeforesecskip}{0.4em}
\renewcommand{\cftsecfont}{\bfseries}
\renewcommand{\cftsubsecfont}{\normalfont}

%========================================================================================
% 표 및 그림
%========================================================================================

\usepackage{graphicx}              % 이미지
\usepackage{adjustbox}             % 표/박스 크기 조절

% 표 캡션 스타일
\usepackage{caption}
\captionsetup[table]{
    labelfont=bf,
    textfont=it,
    skip=5pt
}
\captionsetup[figure]{
    labelfont=bf,
    textfont=it,
    skip=5pt
}

%========================================================================================
% 수학
%========================================================================================

\usepackage{amsmath, amssymb, amsthm}

% 정리 환경
\theoremstyle{definition}
\newtheorem{theorem}{정리}[section]
\newtheorem{lemma}[theorem]{보조정리}
\newtheorem{proposition}[theorem]{명제}
\newtheorem{corollary}[theorem]{따름정리}
\newtheorem{definition}{정의}[section]
\newtheorem{example}{예제}[section]

%========================================================================================
% 하이퍼링크
%========================================================================================

\usepackage[
    colorlinks=true,
    linkcolor=blue!80!black,
    urlcolor=blue!80!black,
    citecolor=green!60!black,
    bookmarks=true,
    bookmarksnumbered=true,
    pdfborder={0 0 0}
]{hyperref}

% PDF 메타데이터는 각 문서에서 설정
\hypersetup{
    pdftitle={CS109A: 데이터 과학 입문 - Lecture 01},
    pdfauthor={강의 노트},
    pdfsubject={Academic Notes}
}

%========================================================================================
% 기타 유용한 패키지
%========================================================================================

\usepackage{enumitem}              % 리스트 커스터마이징
\setlist{nosep, leftmargin=*, itemsep=0.3em}

\usepackage{microtype}             % 타이포그래피 개선
\usepackage{footnote}              % 각주 개선
\usepackage{url}                   % URL 줄바꿈
\urlstyle{same}

%========================================================================================
% 사용자 정의 명령어
%========================================================================================

% 강조 텍스트
\newcommand{\important}[1]{\textbf{\textcolor{red!70!black}{#1}}}
\newcommand{\keyword}[1]{\textbf{#1}}
\newcommand{\term}[1]{\textit{#1}}
\newcommand{\code}[1]{\texttt{#1}}

% 용어 설명 (인라인)
\newcommand{\defterm}[2]{\textbf{#1}\footnote{#2}}

% 섹션 시작 전 페이지 분리
\newcommand{\newsection}[1]{\newpage\section{#1}}

%========================================================================================
% 문서 제목 스타일
%========================================================================================

\usepackage{titling}
\pretitle{\begin{center}\LARGE\bfseries}
\posttitle{\par\end{center}\vskip 0.5em}
\preauthor{\begin{center}\large}
\postauthor{\end{center}}
\predate{\begin{center}\large}
\postdate{\par\end{center}}

%========================================================================================
% 섹션 제목 간격
%========================================================================================

\usepackage{titlesec}
\titlespacing*{\section}{0pt}{1.5em}{0.8em}
\titlespacing*{\subsection}{0pt}{1.2em}{0.6em}
\titlespacing*{\subsubsection}{0pt}{1em}{0.5em}

%========================================================================================
% 메타 정보 박스 명령어
%========================================================================================

\newcommand{\metainfo}[4]{
\begin{tcolorbox}[
    colback=lightpurple,
    colframe=darkpurple,
    boxrule=1pt,
    arc=2mm,
    left=10pt,
    right=10pt,
    top=8pt,
    bottom=8pt
]
\begin{tabular}{@{}rl@{}}
▣ \textbf{강의명:} & #1 \\[0.3em]
▣ \textbf{주차:} & #2 \\[0.3em]
▣ \textbf{교수명:} & #3 \\[0.3em]
▣ \textbf{목적:} & \begin{minipage}[t]{0.75\textwidth}#4\end{minipage}
\end{tabular}
\end{tcolorbox}
}

%========================================================================================
% 끝
%========================================================================================


\begin{document}

\maketitle
\thispagestyle{firstpage}

\metainfo{CS109A: 데이터 과학 입문}{Lecture 01}{Pavlos Protopapas, Kevin Rader, Chris Gumb}{Lecture 01의 핵심 개념 학습}


\begin{summarybox}
본 문서는 CS109A '데이터 과학 입문' 과정의 첫 번째 강의와 실습 세션을 통합한 종합 노트입니다. 데이터 과학의 정의, 역사적 발전, 5단계 프로세스 등 핵심 이론을 다룹니다.
또한 과정의 상세한 평가 기준, 특히 중요한 출석 정책을 설명합니다.
마지막으로, 파이썬(Python)을 활용한 웹 스크레이핑(Web Scraping) 기초 실습을 단계별로 상세히 안내하여 이론과 실습을 연결합니다.
\end{summarybox}

\tableofcontents

\newpage

%========================================
\section{과정 개요 (Course Overview)}
%========================================

\subsection{CS109A: 데이터 과학의 첫걸음}
본 과정(CS109A)은 데이터 과학과 인공지능(AI) 분야의 전문가가 되기 위한 여정의 시작입니다. 최신 모델(예: LLM)을 단순히 사용하는 것을 넘어, 그 근간이 되는 \textbf{기초 원리(Fundamentals)}를 탄탄히 다지는 것을 목표로 합니다.

이 과정은 무술을 배울 때 기본자세(예: "Wax on, Wax off")를 반복 숙달하는 것과 같습니다. 때로는 지루하게 느껴질 수 있지만, 이 기초가 없으면 복잡한 모델을 제대로 이해하고 활용할 수 없습니다.

\subsection{예상 학습 로드맵}
본 과정은 데이터 과학의 전체 흐름을 따르며 다음과 같은 순서로 진행됩니다.

\begin{enumerate}
    \item \textbf{데이터 수집 및 탐색 (Weeks 1-2)}: 웹 스크레이핑, 데이터 정제(Wrangling), 탐색적 데이터 분석(EDA) 및 시각화.
    \item \textbf{회귀 (Regression) (Weeks 3-5)}: K-최근접 이웃(KNN), 선형 회귀, 다중/다항 회귀, 모델 선택(Cross Validation), 추론, 정규화(Ridge, Lasso).
    \item \textbf{베이지안 모델링 (Bayesian) (Week 6)}: 베이지안 추론 프레임워크, 베이지안 선형 회귀.
    \item \textbf{분류 (Classification) (Weeks 7-9)}: KNN 분류, 로지스틱 회귀, 계층적 모델링. (중간고사 포함)
    \item \textbf{데이터 이슈 (Data Issues) (Week 10)}: 결측치(Missingness), 인과 추론(Causal Inference), 편향성 및 윤리.
    \item \textbf{트리 기반 모델 (Tree-Based) (Weeks 11-14)}: 의사결정 나무, 배깅(Bagging), 랜덤 포레스트(Random Forest), 부스팅(Boosting).
\end{enumerate}

\subsection{과정의 3대 목표}
본 과정은 이론, 실습, 그리고 실제 영향력이라는 세 가지 축을 중심으로 구성됩니다.

\begin{itemize}
    \item \textbf{이론과 직관 (Theory/Intuition)}: 통계 분석 및 머신러닝의 핵심 개념을 이해하고, 모델 평가 지표를 학습하며, 분석 결과로부터 통찰력을 추출합니다.
    \item \textbf{실습 (Practice)}: 파이썬 라이브러리(Pandas, Scikit-learn 등)를 사용하여 머신러닝 및 딥러닝 모델을 구현하고, 다양한 종류의 데이터를 다루는 법을 배웁니다.
    \item \textbf{영향력 (Impact)}: 데이터 과학을 사용해 실제 문제를 해결하고, 그 과정에서 발생할 수 있는 사회적, 윤리적 영향을 평가합니다.
\end{itemize}

\newpage

%========================================
\section{데이터 과학(Data Science)이란 무엇인가?}
%========================================

\subsection{데이터 과학의 정의: 3단계 접근}
데이터 과학을 이해하는 가장 좋은 방법은 그 역사적 맥락과 구성 요소를 살펴보는 것입니다.

\begin{itemize}
    \item \textbf{1단계 (핵심 요약)}: 데이터 과학은 \textbf{데이터로부터 의미 있는 통찰과 가치를 추출}하는 모든 과정을 다루는 융합 학문입니다.
    \item \textbf{2단계 (비유)}: 데이터 과학자는 데이터라는 원석을 캐내어(수집), 불순물을 제거하고(정제), 세공하여(모델링), 아름다운 보석(통찰)으로 만들어내는 장인과 같습니다.
    \item \textbf{3단계 (기술적 설명)}: 정형/비정형 데이터를 수집, 관리, 탐색하고, 통계 및 머신러닝 모델을 적용하여 패턴을 발견하거나 미래를 예측하며, 그 결과를 시각화하여 설득력 있게 전달하는 전 과정을 포함합니다.
\end{itemize}

\subsection{데이터 과학의 역사적 발전 4단계}
인류가 세상을 이해하는 방식은 다음과 같이 발전해왔으며, 데이터 과학은 가장 최신의 패러다임입니다.

\begin{enumerate}
    \item \textbf{경험적 관찰 (Empirical Observation) (고대)}
    \begin{itemize}
        \item 밤하늘의 별을 세거나 농작물 수확량을 기록하는 등, 직접적인 관찰과 경험을 통해 데이터를 수집했습니다.
        \item 이는 통계학의 초기 형태라 볼 수 있습니다.
    \end{itemize}
    \item \textbf{방정식 (Equations) (근대 과학)}
    \begin{itemize}
        \item 뉴턴, 아인슈타인 등이 등장하며 세상이 작동하는 근본 원리(First Principles)를 수학 방정식으로 설명하기 시작했습니다.
        \item 예: $F=ma$, $E=mc^2$
    \end{itemize}
    \item \textbf{컴퓨팅 (Computation) (20세기)}
    \begin{itemize}
        \item 1단계의 방정식들이 너무 복잡하여 손으로 풀기 어려워지자, 컴퓨터를 사용하여 시뮬레이션하고 해를 구하기 시작했습니다.
    \end{itemize}
    \item \textbf{데이터 과학 (Data Science) (현대)}
    \begin{itemize}
        \item \textbf{2단계(방정식)를 건너뛰는 경향}이 나타납니다.
        \item 세상의 근본 원리(방정식)를 완벽히 이해하지 못하더라도, \textbf{방대한 데이터(1단계)와 강력한 컴퓨팅(3단계)}을 결합하여 세상의 작동 방식을 근사(approximate)하거나 예측합니다.
    \end{itemize}
\end{enumerate}

\subsection{데이터 과학의 3대 구성 요소}
데이터 과학은 세 가지 핵심 분야가 교차하는 지점에 있습니다.

\begin{itemize}
    \item \textbf{컴퓨터 과학 / IT (Computer Science / IT)}: 데이터 수집, 저장, 처리, 소프트웨어 개발.
    \item \textbf{수학 / 통계 (Math \& Statistics)}: 모델링, 가설 검증, 예측의 수학적 기반.
    \item \textbf{도메인 지식 / 비즈니스 (Domain Knowledge)}: 해당 분야(예: 천문학, 의학, 금융)에 대한 전문 지식.
\end{itemize}

\begin{warningbox}
\textbf{도메인 지식의 중요성}

도메인 지식이 없는 데이터 과학은 "엉뚱한 문제"를 풀 위험이 큽니다.

한 천문학자가 컴퓨터 과학자에게 데이터 분석을 의뢰했습니다. 한 달 후, 컴퓨터 과학자는 "문제를 완벽히 해결했다"고 했지만, 알고 보니 그는 데이터의 의미와 천문학적 맥락을 전혀 이해하지 못해 완전히 잘못된 문제를 푼 것이었습니다.

데이터 과학은 자동차 정비소에 차를 맡기듯 문제를 던지고 끝내는 것이 아닙니다. \textbf{반드시 해당 분야의 전문가와 긴밀히 소통하며 문제 자체를 함께 정의해야 합니다.}
\end{warningbox}

\subsection{데이터 과학의 잠재력과 위험성}
데이터 과학은 강력한 도구이며, 그에 따른 잠재력과 위험성을 동시에 가집니다.

\begin{tcolorbox}[colback=green!5, colframe=green!50!black, title=👍 데이터 과학의 잠재력]
\begin{itemize}
    \item \textbf{질병 진단}: 혈액 도말 샘플 이미지로 말라리아 감염 여부 진단.
    \item \textbf{신약 개발}: 언어 모델(LM)을 활용하여 새로운 약물 조합 발견.
    \item \textbf{생성형 AI (Generative AI)}: 텍스트 프롬프트(예: "안경 쓴 그리스인 교수")로부터 이미지 생성.
    \item \textbf{자율 주행}: 야간에도 안전하게 운행하는 자율주행 트럭.
\end{itemize}
\end{tcolorbox}

\begin{tcolorbox}[colback=red!5, colframe=red!50!black, title=👎 데이터 과학의 위험성과 윤리]
\begin{itemize}
    \item \textbf{성별 편향 (Gender Bias)}: 특정 직군(예: 엔지니어) 채용 모델이 남성 지원자에게 유리하게 작동.
    \item \textbf{인종 편향 (Racial Bias)}: 미국 법원에서 사용되는 재범 위험도 예측 모델이 유색인종에게 불리하게 편향됨.
\end{itemize}
\end{tcolorbox}

이러한 위험성 때문에 우리는 데이터 과학을 맹목적으로 사용해서는 안 되며, 항상 \textbf{비판적 사고(Critical Thinking)}를 견지하고 모델의 공정성과 윤리성을 점검해야 합니다.

\newpage

%========================================
\section{데이터 과학의 5단계 프로세스}
%========================================

데이터 과학 프로젝트는 일반적으로 다음 5단계를 순환하며 진행됩니다.

\begin{enumerate}
    \item \textbf{흥미로운 질문하기 (Ask an interesting question)}
    \begin{itemize}
        \item 가장 중요하고 첫 번째 단계입니다. "데이터가 있으니 뭔가 찾아봐"가 아니라, 명확한 가설이나 과학적 목표를 설정해야 합니다.
        \item (예: "우리가 예측/추정하려는 것은 무엇인가?", "모든 데이터가 있다면 무엇을 할 것인가?")
    \end{itemize}
    \item \textbf{데이터 획득하기 (Get the Data)}
    \begin{itemize}
        \item 질문에 답하기 위해 필요한 데이터를 수집합니다.
        \item (예: "데이터는 어떻게 샘플링되었는가?", "어떤 데이터가 관련 있는가?", "라이선스나 개인정보 보호 문제는 없는가?")
    \end{itemize}
    \item \textbf{데이터 탐색하기 (Explore the Data - EDA)}
    \begin{itemize}
        \item 데이터를 시각화하고 요약하며 패턴을 찾습니다. 이 단계에서 많은 시간을 절약할 수 있습니다. (때로는 이 단계만으로도 충분한 답을 얻기도 합니다.)
        \item (예: "데이터를 플롯팅해 보았는가?", "이상치(anomaly)나 심각한 오류는 없는가?", "어떤 패턴이 보이는가?")
    \end{itemize}
    \item \textbf{데이터 모델링하기 (Model the Data)}
    \begin{itemize}
        \item 데이터의 패턴을 학습하거나 미래를 예측하는 통계/머신러닝 모델을 구축합니다.
        \item (예: "모델 구축 (Build) -> 모델 학습 (Fit) -> 모델 검증 (Validate)")
    \end{itemize}
    \item \textbf{결과 전달/시각화하기 (Communicate/Visualize the Results)}
    \begin{itemize}
        \item 분석 결과를 비전문가도 이해할 수 있도록 스토리텔링과 시각화를 통해 전달합니다.
        \item (예: "우리는 무엇을 배웠는가?", "결과가 말이 되는가(make sense)?", "효과적으로 스토리를 전달할 수 있는가?")
    \end{itemize}
\end{enumerate}

\newpage

%========================================
\section{CS109A 과정 상세 안내}
%========================================

\subsection{교수진 및 조교(TAs)}
\begin{itemize}
    \item \textbf{Pavlos Protopapas}: 데이터 과학 석사 과정의 Scientific Director. 천문학과 머신러닝 연구를 수행하며, 요리 자격증을 보유하고 있습니다.
    \item \textbf{Kevin Rader}: 통계학과 선임 지도교수(Senior Preceptor). 학부 교육을 담당하며, 스포츠 및 의학 분야 데이터 분석에 관심이 많습니다. (필라델피아 이글스 팬 - "Go Birds!")
    \item \textbf{Chris Gumb}: 지도교수(Preceptor). 약 30명에 달하는 TF 팀을 조율하고 과정 운영을 지원합니다.
    \item \textbf{Teaching Fellows (TFs)}: 약 30명의 TF가 섹션(실습)과 오피스 아워를 담당합니다.
\end{itemize}

\subsection{학습 철학: "Wax on, Wax off"}
\begin{examplebox}
영화 <베스트 키드>에서 미야기 사부는 제자에게 가라데 대신 자동차 왁스 칠(Wax on, Wax off)만 반복시킵니다. 제자는 불평하지만, 이 무의미해 보이는 반복 작업이 실제 대련에서 방어 동작의 완벽한 기초가 되었음을 깨닫습니다.

CS109A도 마찬가지입니다.
데이터 탐색, 모델 학습, 검증 등 \textbf{기본적인 절차를 반복 숙달}시키는 과제가 많을 것입니다. 이는 단순히 코드를 `fit()` 시키는 것을 넘어, \textbf{모델이 내부에서 어떻게 작동하는지, 왜 그렇게 작동하는지}를 깊이 이해하기 위한 필수 과정입니다.
\end{examplebox}

\subsection{학습 도구}
본 과정은 두 가지 주요 플랫폼을 사용합니다.
\begin{itemize}
    \item \textbf{Edstem}: 강의 슬라이드, 섹션(실습) 자료, 공지사항, \textbf{토론 포럼(Q\&A)}이 이루어지는 메인 허브입니다.
    \item \textbf{Canvas}: 강의 비디오 녹화본, 과제 제출, 공식 일정, \textbf{성적 확인}에 사용됩니다.
\end{itemize}

\subsection{도움 받는 방법 (How to get help)}
문제가 생겼을 때 다음 순서로 도움을 요청하세요.

\begin{enumerate}
    \item \textbf{Edstem (토론 포럼)}: 가장 빠른 방법. 동료 학생이나 TF가 답변해줍니다. (개인적인 내용 제외)
    \item \textbf{오피스 아워 (Office Hours)}: 개념 이해나 과제에 대한 심층적인 도움이 필요할 때 가장 좋은 방법입니다.
    \item \textbf{과정 헬프라인 (Email)}: 수강 변경 등 개인적인 행정 문의.
    \item \textbf{교수진 (Email)}: 매우 사적인 문제나 민감한 사안.
\end{enumerate}

\newpage

%========================================
\section{평가 및 주요 정책}
%========================================

\subsection{5대 평가 요소 및 비중}
학생 평가는 5가지 요소를 합산하여 이루어집니다.

\begin{table}[h!]
  \centering
  \caption{CS109A 평가 요소 및 비중}
  \label{tab:grading}
  \begin{tabular}{l c p{10cm}}
    \toprule
    \textbf{평가 요소} & \textbf{비중} & \textbf{세부 내용} \\
    \midrule
    \textbf{숙제 (Homework)} & 30\% & HW0 (1\%) + HW 1-5 (29\%). \newline 2인 1조(Pair) 작업 권장. \\
    \textbf{섹션 퀴즈} & 10\% & 섹션(실습) 시간 중 실시하는 30분 분량의 퀴즈 2회. \\
    \textbf{중간고사 (Midterm)} & 18\% & 섹션 시간 중 치르는 개념 파트 + 별도의 코딩 파트(Take-home)로 구성. \\
    \textbf{기말고사 (Final Exam)} & 22\% & 3시간 동안 지정된 좌석에서 치르는 시험. (개념 + 코딩) \\
    \textbf{프로젝트 (Project)} & 20\% & 3-5인 1조의 그룹 프로젝트. 공개된(public) 데이터를 활용하여 주제 제안 가능. \\
    \bottomrule
  \end{tabular}
\end{table}

\subsection{숙제(Homework) 상세}
\begin{itemize}
    \item \textbf{HW 0}: 과정 시작 시 배포되며, \textbf{선수과목(Prerequisites) 충족 여부를 스스로 진단}하기 위한 목적입니다. (성실히 제출 시 1% 반영)
    \item \textbf{HW 1-5}: 2인 1조(Pair)로 제출하는 것을 적극 권장합니다.
    \item \textbf{제출 기한}: (별도 공지 없는 한) 매주 화요일 오후 10시.
\end{itemize}

\subsection{선수과목 (Prerequisites) 진단}
HW0는 다음 3가지 영역에 대한 준비 상태를 점검합니다.

\begin{itemize}
    \item \textbf{파이썬 (Python) 코딩}:
        \begin{itemize}
            \item \textbf{(필수)} 프로그래밍 경험이 \textbf{전혀 없다면} 이 과정을 수강하기 매우 어렵습니다.
            \item \textbf{(괜찮음)} 파이썬에 능숙하지 않더라도, 다른 언어(예: CS50 수강) 경험이 있다면 따라올 수 있습니다.
        \end{itemize}
    \item \textbf{기초 수학 (Calculus)}: 기본적인 미적분 지식이 필요합니다. (예: Math 1B 수준)
    \item \textbf{기초 통계/확률 (Stats/Probability)}:
        \begin{itemize}
            \item Stat 104 (데이터 분석 중심) 수강생이 가장 이상적입니다.
            \item Stat 110 (수학적 확률론 중심)도 좋지만, 실제 데이터를 다루는 부분은 본 과정에서 새로 배워야 할 수 있습니다.
        \end{itemize}
    \item \textbf{결론}: 수학/통계 지식의 공백은 TF와 교수진의 도움으로 메울 수 있지만, \textbf{코딩 경험의 부재}는 심각한 장애물이 될 수 있습니다.
\end{itemize}

\subsection{출석 및 지각 정책}

\begin{warningbox}
\textbf{CS109A 출석 정책은 매우 중요하며 성적에 직접적인 영향을 미칩니다.}

\begin{itemize}
    \item \textbf{출석은 필수입니다 (On-campus 학생)}: 모든 강의와 섹션은 출석이 요구됩니다.
    \item \textbf{성적 등급 자격 (Qualification)}: 출석률은 받을 수 있는 \textbf{최고 성적을 제한}하는 "자격" 요건입니다. (이 출석률을 만족한다고 해당 성적을 보장하는 것은 아닙니다.)
        \begin{itemize}
            \item \textbf{A} 등급을 받으려면 $\rightarrow$ 최소 \textbf{66\%} (2/3) 출석 필요
            \item \textbf{A-} 등급을 받으려면 $\rightarrow$ 최소 \textbf{50\%} (1/2) 출석 필요
            \item \textbf{B+} 등급을 받으려면 $\rightarrow$ 최소 \textbf{33\%} (1/3) 출석 필요
        \end{itemize}
    \item \textbf{지각 제출권 (Late Days) 획득}:
        \item 출석(강의 또는 섹션) \textbf{4회당 1개의 지각 제출권(Late Day)}을 획득합니다.
        \item (예: 24회 출석 시 6개의 Late Day 획득)
        \item \textbf{DCE 학생}은 출석 확인이 어려운 점을 감안하여 자동으로 \textbf{4개의 Late Day}가 부여됩니다.
    \item \textbf{Late Day 사용}:
        \item 획득한 Late Day는 숙제(HW) 제출 시 사용할 수 있습니다.
        \item 한 숙제당 최대 \textbf{2개}의 Late Day만 사용할 수 있습니다.
\end{itemize}
\end{warningbox}

\newpage

%========================================
\section{실습 (Section 1): 웹 스크레이핑 입문}
%========================================

\subsection{웹 스크레이핑이란?}
웹 스크레이핑(Web Scraping)은 웹사이트에서 프로그래밍 방식을 통해 자동으로 데이터를 추출하고 수집하는 기술입니다.

첫 번째 실습과 숙제(HW1)는 이 기술을 사용하여 노벨상(Nobel Prize) 웹사이트에서 데이터를 수집하는 것을 목표로 합니다.

\subsection{실습 라이브러리}
\begin{itemize}
    \item \textbf{requests}: 웹사이트에 접속하여 원본 HTML 코드를 가져오는 라이브러리. (HTTP 요청)
    \item \textbf{BeautifulSoup}: 가져온 HTML 코드를 파이썬이 다루기 쉬운 객체 구조로 변환(Parsing)하고, 원하는 정보를 쉽게 찾도록 도와주는 라이브러리.
    \item \textbf{pandas}: 추출한 데이터를 표(DataFrame) 형태로 정리하고 분석하는 라이브러리.
    \item \textbf{matplotlib}: 데이터를 시각화하는 라이브러리.
\end{itemize}

\subsection{1단계: 웹 스크레이핑 윤리 및 규칙 확인}
\begin{warningbox}
모든 웹사이트가 데이터 수집을 허용하는 것은 아닙니다.

\begin{itemize}
    \item \textbf{robots.txt 확인}: 웹사이트 도메인 뒤에 \texttt{/robots.txt}를 붙여(예: \texttt{google.com/robots.txt}) 어떤 페이지의 수집을 허용/금지하는지 확인해야 합니다.
    \item \textbf{과도한 요청 금지 (Rate Limit)}: 서버에 부담을 주지 않도록 짧은 시간에 너무 많은 요청을 보내지 않아야 합니다. (예: "1분에 500회 요청 제한")
\end{itemize}
\end{warningbox}

\subsection{2단계: HTML 기초와 브라우저 '검사' 도구}
웹사이트는 HTML(HyperText Markup Language)이라는 언어로 구성됩니다.

\begin{itemize}
    \item \textbf{요소 (Element)}: \texttt{<tag>}로 시작하여 \texttt{</tag>}로 끝나는 전체 구조.
    \item \textbf{태그 (Tag)}: 요소의 종류를 정의합니다. (예: \texttt{<h1>}(제목), \texttt{<p>}(문단), \texttt{<a>}(링크), \texttt{<div>}(구역))
    \item \textbf{속성 (Attribute)}: 태그에 추가 정보를 제공합니다. (예: \texttt{<a href="..." >}(링크 주소), \texttt{<div class="..." >}(요소의 별명))
\end{itemize}

\begin{examplebox}
\textbf{브라우저 '검사' 도구 활용하기}

웹사이트에서 원하는 정보(예: 수상자 이름)가 어떤 태그와 클래스(class)로 구성되어 있는지 확인하는 가장 쉬운 방법입니다.
\begin{enumerate}
    \item 웹페이지에서 원하는 부분에 마우스 오른쪽 클릭
    \item \textbf{'검사(Inspect)'} 메뉴 선택
    \item 개발자 도구가 열리면, 왼쪽 상단의 \textbf{'선택 도구(Picker Tool)'} 아이콘(화살표 모양)을 클릭
    \item 페이지에서 원하는 요소를 클릭하면, 해당 요소의 HTML 코드가 하이라이트됩니다.
\end{enumerate}
\end{examplebox}

\subsection{3단계: \texttt{requests}로 데이터 가져오기}
먼저 웹페이지의 HTML 소스 코드를 가져와야 합니다.

\begin{lstlisting}[language=Python, caption={requests를 사용한 HTML 코드 요청}, label={code:requests}]
import requests

# 1. 수집할 웹사이트 URL
url = "https://www.nobelprize.org/all-nobel-prizes/"

# 2. HTTP GET 요청 보내기
response = requests.get(url)

# 3. 상태 코드 확인 (200이면 성공)
print(f"상태 코드: {response.status_code}")

# 4. HTML 텍스트 내용 확인 (일부만)
html_text = response.text
print(html_text[:200])
\end{lstlisting}

\begin{itemize}
    \item \textbf{Status Code 200}: 요청이 성공적으로 완료되었음을 의미합니다. (OK)
    \item \textbf{Status Code 404}: 해당 URL을 찾을 수 없음을 의미합니다. (Not Found)
\end{itemize}

\subsection{4단계: \texttt{BeautifulSoup}로 HTML 파싱하기}
\texttt{response.text}는 다루기 힘든 거대한 문자열입니다. 이를 \texttt{BeautifulSoup}를 이용해 "수프(soup)" 객체로 만듭니다.

\begin{lstlisting}[language=Python, caption={BeautifulSoup로 HTML 파싱하기}, label={code:soup}]
from bs4 import BeautifulSoup

# 1. 'html.parser'를 이용해 html_text를 soup 객체로 변환
soup = BeautifulSoup(html_text, 'html.parser')

# 2. 원하는 정보 찾기 (예: 페이지 제목)
page_title = soup.title.get_text()
print(f"페이지 제목: {page_title}")

# 3. CSS 선택자(Selector)로 특정 요소 찾기
# (예: 클래스가 'card-prize'인 모든 div 요소)
prize_blocks = soup.select('div.card-prize')
print(f"총 수상 블록 개수: {len(prize_blocks)}")

# 4. 첫 번째 블록에서 텍스트만 추출 (공백 제거)
first_block = prize_blocks[0]
block_text = first_block.get_text().strip()
print(block_text)
\end{lstlisting}

\subsection{5단계: 데이터 추출 및 구조화 (노벨상 예제)}
여러 개의 수상 블록(\texttt{prize\_blocks})을 순회하며 원하는 정보를 추출하여 리스트와 딕셔너리로 저장합니다.

\begin{lstlisting}[language=Python, caption={반복문을 통한 데이터 추출 및 구조화}, label={code:extract}]
import re # 정규표현식 (Regular Expression) 라이브러리
from collections import defaultdict, Counter

# 1. 람다(lambda)를 이용한 간단한 헬퍼 함수 정의
get_title = lambda block: block.select_one('h3').get_text().strip()
get_year = lambda block: re.search(r'(\d{4})', block.select_one('h3').get_text()).group(1)
get_description = lambda block: block.select_one('blockquote').get_text().strip()

# 2. 데이터를 저장할 리스트
nobel_data = []

# 3. 모든 수상 블록을 순회(loop)
for block in prize_blocks:
    try:
        title = get_title(block)
        year = get_year(block)
        description = get_description(block)
        
        # 4. 딕셔너리 형태로 저장
        nobel_data.append({
            'title': title,
            'year': int(year),
            'description': description
        })
    except Exception as e:
        print(f"데이터 추출 중 오류 발생: {e}")

# 5. (예제) 고유한 수상 분야 찾기
unique_titles = set(item['title'] for item in nobel_data)
print(f"고유 수상 분야: {unique_titles}")

# 6. (예제) 경제학상이 처음 수여된 연도 찾기
econ_years = [item['year'] for item in nobel_data if 'Economic Sciences' in item['title']]
first_econ_year = min(econ_years)
print(f"경제학상 최초 수여 연도: {first_econ_year}")

# 7. (예제) 연도별 수상자 수 집계 (defaultdict 사용)
winners_per_year = defaultdict(int)
for item in nobel_data:
    winners_per_year[item['year']] += 1 # 0으로 자동 초기화됨
print(f"2023년 수상자 수: {winners_per_year[2023]}")
\end{lstlisting}

\subsection{6단계: \texttt{pandas}로 데이터 프레임 변환}
스크레이핑한 데이터(딕셔너리 리스트)는 \texttt{pandas}의 \texttt{DataFrame}으로 변환하면 분석하기 매우 용이합니다.

\begin{lstlisting}[language=Python, caption={pandas DataFrame으로 변환 및 저장}, label={code:pandas}]
import pandas as pd

# 1. 리스트를 데이터 프레임으로 변환
df = pd.DataFrame(nobel_data)

# 2. 데이터 프레임 상위 5개 확인
print(df.head())

# 3. CSV 파일로 저장
df.to_csv('nobel_prizes.csv', index=False)
\end{lstlisting}

\subsection{(심화) 비동기(Async) 스크레이핑}
수백, 수천 개의 페이지를 스크레이핑할 때는 한 번에 하나씩 요청하면 매우 느립니다.

\textbf{비동기(Asynchronous) 프로그래밍} (예: \texttt{asyncio}, \texttt{httpx} 라이브러리)은 여러 개의 요청을 동시에 처리하여 속도를 높이는 고급 기법입니다.

이는 서버의 "Rate Limit"(요청 제한)을 존중하면서도 효율적으로 데이터를 수집하기 위해 사용됩니다. (예: "1초에 5개씩" 또는 "한 번에 10개씩 묶어서(batch) 요청")

\newpage

%========================================
\section{자주 묻는 질문 (FAQ)}
%========================================

\begin{itemize}
    \item \textbf{Q: 이 수업을 청강(Audit)할 수 있나요?}
    \item A: 네, 가능합니다. 다만 청강으로 수강한 경우, 나중에 동일 과목을 학점 이수(for grade)로 다시 수강할 수 없습니다.

    \item \textbf{Q: 비동기(Asynchronous) 방식(녹화본 시청)으로만 수강할 수 있나요?}
    \item A: \textbf{(대학생)} 허용되지 않습니다. \textbf{(대학원생)} 권장하지 않습니다. 출석은 이 수업의 매우 중요한 부분이며, 출석률이 33\% 미만일 경우 B+ 이상의 성적을 받을 자격이 박탈됩니다.

    \item \textbf{Q: 선수과목이 부족한데 수강할 수 있을까요?}
    \item A: \textbf{HW0}를 풀어보고 판단하세요. 수학/통계 지식은 도움을 받아 채울 수 있지만, \textbf{프로그래밍(코딩) 경험이 전혀 없다면} 수강을 다음 학기로 미루는 것을 강력히 권장합니다.

    \item \textbf{Q: 중간고사 기간에 여행 계획이 있습니다. 시험을 일찍 보거나 미룰 수 있나요?}
    \item A: 아니요. 중간고사(10/22-24주간)와 기말고사(12/11 예정)는 지정된 날짜에만 치러야 합니다. 일정을 미리 확인하세요.

    \item \textbf{Q: 제가 가진 개인 프로젝트 아이디어를 사용해도 되나요?}
    \item A: 네, 가능합니다. 단, \textbf{데이터가 공개(public)되어 있어야 하며, 3-5명의 그룹 프로젝트}로 진행해야 합니다.
\end{itemize}

\end{document}
