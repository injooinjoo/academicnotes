%%%%%%%%%%%%%%%%%%%%%%%%%%%%%%%%%%%%%%%%%%%%%%%%%%%%%%%%%%%%%%%%%%%%%%%%%%%%%%%
% Harvard Academic Notes - 통합 마스터 템플릿
% 모든 강의 노트에 적용되는 통일된 스타일
% 버전: 2.1 - 가독성 개선 (선택적 최적화)
% 최종 수정일: 2025-11-17
%%%%%%%%%%%%%%%%%%%%%%%%%%%%%%%%%%%%%%%%%%%%%%%%%%%%%%%%%%%%%%%%%%%%%%%%%%%%%%%

\documentclass[11pt,a4paper]{article}

%========================================================================================
% 기본 패키지
%========================================================================================

% --- 한국어 지원 ---
\usepackage{kotex}

% --- 페이지 레이아웃 ---
\usepackage[top=20mm, bottom=20mm, left=20mm, right=18mm]{geometry}
\usepackage{setspace}
\onehalfspacing                      % 1.5배 줄간격
\setlength{\parskip}{0.5em}          % 문단 간격
\setlength{\parindent}{0pt}          % 들여쓰기 없음

% --- 표 관련 ---
\usepackage{booktabs}              % 고품질 표
\usepackage{tabularx}              % 자동 너비 조절 표
\usepackage{array}                 % 표 컬럼 확장
\usepackage{longtable}             % 여러 페이지 표
\renewcommand{\arraystretch}{1.1}  % 표 행간 조절

%========================================================================================
% 헤더 및 푸터
%========================================================================================

\usepackage{fancyhdr}
\pagestyle{fancy}
\fancyhf{}
\fancyhead[L]{\small\textit{CS109A: 데이터 과학 입문}}
\fancyhead[R]{\small\textit{Lecture 25}}
\fancyfoot[C]{\thepage}
\renewcommand{\headrulewidth}{0.5pt}
\renewcommand{\footrulewidth}{0.3pt}

% 첫 페이지는 헤더 없음
\fancypagestyle{firstpage}{
    \fancyhf{}
    \fancyfoot[C]{\thepage}
    \renewcommand{\headrulewidth}{0pt}
}

%========================================================================================
% 색상 정의 (파스텔 톤 + 다크모드 호환)
%========================================================================================

\usepackage[dvipsnames]{xcolor}

% 밝은 배경용 파스텔 색상
\definecolor{lightblue}{RGB}{220, 235, 255}      % 부드러운 파랑
\definecolor{lightgreen}{RGB}{220, 255, 235}     % 부드러운 초록
\definecolor{lightyellow}{RGB}{255, 250, 220}    % 부드러운 노랑
\definecolor{lightpurple}{RGB}{240, 230, 255}    % 부드러운 보라
\definecolor{lightgray}{gray}{0.95}              % 밝은 회색
\definecolor{lightpink}{RGB}{255, 235, 245}      % 부드러운 핑크
\definecolor{boxgray}{gray}{0.95}
\definecolor{boxblue}{rgb}{0.9, 0.95, 1.0}
\definecolor{boxred}{rgb}{1.0, 0.95, 0.95}

% 진한 색상 (테두리/제목용)
\definecolor{darkblue}{RGB}{50, 80, 150}
\definecolor{darkgreen}{RGB}{40, 120, 70}
\definecolor{darkorange}{RGB}{200, 100, 30}
\definecolor{darkpurple}{RGB}{100, 60, 150}

%========================================================================================
% 박스 환경 (tcolorbox) - 6가지 타입
%========================================================================================

\usepackage[most]{tcolorbox}
\tcbuselibrary{skins, breakable}

% 1. 개요 박스 (강의 시작 부분)
\newtcolorbox{overviewbox}[1][]{
    enhanced,
    colback=lightpurple,
    colframe=darkpurple,
    fonttitle=\bfseries\large,
    title=📚 강의 개요,
    arc=3mm,
    boxrule=1pt,
    left=8pt,
    right=8pt,
    top=8pt,
    bottom=8pt,
    breakable,
    #1
}

% 2. 요약 박스
\newtcolorbox{summarybox}[1][]{
    enhanced,
    colback=lightblue,
    colframe=darkblue,
    fonttitle=\bfseries,
    title=📝 핵심 요약,
    arc=2mm,
    boxrule=0.7pt,
    left=6pt,
    right=6pt,
    top=6pt,
    bottom=6pt,
    breakable,
    #1
}

% 3. 핵심 정보 박스
\newtcolorbox{infobox}[1][]{
    enhanced,
    colback=lightgreen,
    colframe=darkgreen,
    fonttitle=\bfseries,
    title=💡 핵심 정보,
    arc=2mm,
    boxrule=0.7pt,
    left=6pt,
    right=6pt,
    top=6pt,
    bottom=6pt,
    breakable,
    #1
}

% 4. 주의사항 박스
\newtcolorbox{warningbox}[1][]{
    enhanced,
    colback=lightyellow,
    colframe=darkorange,
    fonttitle=\bfseries,
    title=⚠️ 주의사항,
    arc=2mm,
    boxrule=0.7pt,
    left=6pt,
    right=6pt,
    top=6pt,
    bottom=6pt,
    breakable,
    #1
}

% 5. 예제 박스
\newtcolorbox{examplebox}[1][]{
    enhanced,
    colback=lightgray,
    colframe=black!60,
    fonttitle=\bfseries,
    title=📖 예제: #1,
    arc=2mm,
    boxrule=0.7pt,
    left=6pt,
    right=6pt,
    top=6pt,
    bottom=6pt,
    breakable,
}

% 6. 정의 박스
\newtcolorbox{definitionbox}[1][]{
    enhanced,
    colback=lightpink,
    colframe=purple!70!black,
    fonttitle=\bfseries,
    title=📌 정의: #1,
    arc=2mm,
    boxrule=0.7pt,
    left=6pt,
    right=6pt,
    top=6pt,
    bottom=6pt,
    breakable,
}

% 7. 중요 박스 (importantbox - warningbox와 유사)
\newtcolorbox{importantbox}[1][]{
    enhanced,
    colback=boxred,
    colframe=red!70!black,
    fonttitle=\bfseries,
    title=⚠️ 매우 중요: #1,
    arc=2mm,
    boxrule=0.7pt,
    left=6pt,
    right=6pt,
    top=6pt,
    bottom=6pt,
    breakable,
}

% 8. cautionbox (warningbox와 동일)
\let\cautionbox\warningbox
\let\endcautionbox\endwarningbox

%========================================================================================
% 코드 블록 설정 (밝은 배경)
%========================================================================================

\usepackage{listings}

\definecolor{codegray}{rgb}{0.5,0.5,0.5}
\definecolor{codepurple}{rgb}{0.58,0,0.82}
\definecolor{backcolour}{rgb}{0.95,0.95,0.95}

\lstset{
    basicstyle=\ttfamily\small,
    backgroundcolor=\color{lightgray},
    keywordstyle=\color{darkblue}\bfseries,
    commentstyle=\color{darkgreen}\itshape,
    stringstyle=\color{purple!80!black},
    numberstyle=\tiny\color{black!60},
    numbers=left,
    numbersep=8pt,
    breaklines=true,
    breakatwhitespace=false,
    frame=single,
    frameround=tttt,
    rulecolor=\color{black!30},
    captionpos=b,
    showstringspaces=false,
    tabsize=2,
    xleftmargin=15pt,
    xrightmargin=5pt,
    escapeinside={\%*}{*)}
}

% Python 코드 스타일
\lstdefinestyle{pythonstyle}{
    language=Python,
    morekeywords={self, True, False, None},
}

% SQL 코드 스타일
\lstdefinestyle{sqlstyle}{
    language=SQL,
    morekeywords={SELECT, FROM, WHERE, JOIN, GROUP, BY, ORDER, HAVING},
}

%========================================================================================
% 목차 스타일링
%========================================================================================

\usepackage{tocloft}
\renewcommand{\cftsecleader}{\cftdotfill{\cftdotsep}}
\setlength{\cftbeforesecskip}{0.4em}
\renewcommand{\cftsecfont}{\bfseries}
\renewcommand{\cftsubsecfont}{\normalfont}

%========================================================================================
% 표 및 그림
%========================================================================================

\usepackage{graphicx}              % 이미지
\usepackage{adjustbox}             % 표/박스 크기 조절

% 표 캡션 스타일
\usepackage{caption}
\captionsetup[table]{
    labelfont=bf,
    textfont=it,
    skip=5pt
}
\captionsetup[figure]{
    labelfont=bf,
    textfont=it,
    skip=5pt
}

%========================================================================================
% 수학
%========================================================================================

\usepackage{amsmath, amssymb, amsthm}

% 정리 환경
\theoremstyle{definition}
\newtheorem{theorem}{정리}[section]
\newtheorem{lemma}[theorem]{보조정리}
\newtheorem{proposition}[theorem]{명제}
\newtheorem{corollary}[theorem]{따름정리}
\newtheorem{definition}{정의}[section]
\newtheorem{example}{예제}[section]

%========================================================================================
% 하이퍼링크
%========================================================================================

\usepackage[
    colorlinks=true,
    linkcolor=blue!80!black,
    urlcolor=blue!80!black,
    citecolor=green!60!black,
    bookmarks=true,
    bookmarksnumbered=true,
    pdfborder={0 0 0}
]{hyperref}

% PDF 메타데이터는 각 문서에서 설정
\hypersetup{
    pdftitle={CS109A: 데이터 과학 입문 - Lecture 25},
    pdfauthor={강의 노트},
    pdfsubject={Academic Notes}
}

%========================================================================================
% 기타 유용한 패키지
%========================================================================================

\usepackage{enumitem}              % 리스트 커스터마이징
\setlist{nosep, leftmargin=*, itemsep=0.3em}

\usepackage{microtype}             % 타이포그래피 개선
\usepackage{footnote}              % 각주 개선
\usepackage{url}                   % URL 줄바꿈
\urlstyle{same}

%========================================================================================
% 사용자 정의 명령어
%========================================================================================

% 강조 텍스트
\newcommand{\important}[1]{\textbf{\textcolor{red!70!black}{#1}}}
\newcommand{\keyword}[1]{\textbf{#1}}
\newcommand{\term}[1]{\textit{#1}}
\newcommand{\code}[1]{\texttt{#1}}

% 용어 설명 (인라인)
\newcommand{\defterm}[2]{\textbf{#1}\footnote{#2}}

% 섹션 시작 전 페이지 분리
\newcommand{\newsection}[1]{\newpage\section{#1}}

%========================================================================================
% 문서 제목 스타일
%========================================================================================

\usepackage{titling}
\pretitle{\begin{center}\LARGE\bfseries}
\posttitle{\par\end{center}\vskip 0.5em}
\preauthor{\begin{center}\large}
\postauthor{\end{center}}
\predate{\begin{center}\large}
\postdate{\par\end{center}}

%========================================================================================
% 섹션 제목 간격
%========================================================================================

\usepackage{titlesec}
\titlespacing*{\section}{0pt}{1.5em}{0.8em}
\titlespacing*{\subsection}{0pt}{1.2em}{0.6em}
\titlespacing*{\subsubsection}{0pt}{1em}{0.5em}

%========================================================================================
% 메타 정보 박스 명령어
%========================================================================================

\newcommand{\metainfo}[4]{
\begin{tcolorbox}[
    colback=lightpurple,
    colframe=darkpurple,
    boxrule=1pt,
    arc=2mm,
    left=10pt,
    right=10pt,
    top=8pt,
    bottom=8pt
]
\begin{tabular}{@{}rl@{}}
▣ \textbf{강의명:} & #1 \\[0.3em]
▣ \textbf{주차:} & #2 \\[0.3em]
▣ \textbf{교수명:} & #3 \\[0.3em]
▣ \textbf{목적:} & \begin{minipage}[t]{0.75\textwidth}#4\end{minipage}
\end{tabular}
\end{tcolorbox}
}

%========================================================================================
% 끝
%========================================================================================


\begin{document}

\metainfo{CS109A: 데이터 과학 입문}{Lecture 25}{Pavlos Protopapas, Kevin Rader, Chris Gumb}{Lecture 25의 핵심 개념 학습}

\tableofcontents
\newpage


% 제목 섹션
\begin{center}
    \LARGE \textbf{CS109A 수업 노트: 앙상블 학습의 확장}\\
    \large (Blending, Stacking, and Mixture of Experts)
\end{center}
\vspace{0.5cm}

% -----------------------------------------------------------------------------
% 1. 개요 (Overview)
% -----------------------------------------------------------------------------
\section{개요: 왜 더 복잡한 앙상블이 필요한가?}

이 문서는 기존의 랜덤 포레스트나 부스팅(Bagging/Boosting)을 넘어, 서로 다른 종류의 모델들을 결합하여 성능을 극대화하는 고급 앙상블 기법을 다룹니다.

\begin{summarybox}{핵심 목표 및 요약}
\begin{itemize}
    \item \textbf{한계 극복:} 단일 모델(Decision Tree 등)이나 동일한 모델의 집합(Random Forest)만으로는 해결하기 어려운 복잡한 데이터 패턴을 학습합니다.
    \item \textbf{이종 결합:} 선형 회귀, KNN, SVM 등 성격이 완전히 다른 모델들을 하나로 합쳐 각 모델의 장점만 취합니다.
    \item \textbf{메타 학습:} 모델들의 예측값(Output)을 다시 학습 데이터로 사용하여 최종 결론을 내리는 '관리자 모델(Meta-model)'을 도입합니다.
\end{itemize}
\end{summarybox}

\vspace{0.5cm}

% -----------------------------------------------------------------------------
% 2. 용어 정리 (Terminology)
% -----------------------------------------------------------------------------
\section{필수 용어 정리}

본격적인 학습 전에 아래 용어를 숙지하면 이해가 훨씬 빠릅니다.

\begin{table}[h]
\centering
\resizebox{\textwidth}{!}{%
\begin{tabular}{l l l}
\toprule
\textbf{용어 (한글/영문)} & \textbf{설명} & \textbf{비고} \\
\midrule
\textbf{기반 모델 (Base Model)} & 1차적으로 데이터를 학습하여 예측을 수행하는 개별 모델들 & 전문가 팀원 \\
\textbf{메타 모델 (Meta Model)} & 기반 모델들의 예측값을 입력받아 최종 판단을 내리는 모델 & 팀장/관리자 \\
\textbf{동질성 (Homogeneous)} & 같은 종류의 알고리즘(예: 트리)만 모아서 앙상블 하는 것 & Random Forest 등 \\
\textbf{이질성 (Heterogeneous)} & 서로 다른 알고리즘(예: 선형회귀+KNN)을 섞어서 사용하는 것 & Blending/Stacking \\
\textbf{홀드아웃 세트 (Holdout Set)} & 메타 모델을 학습시키기 위해 따로 떼어둔 검증용 데이터 & 커닝 방지용 \\
\textbf{게이팅 (Gating Network)} & 데이터의 특성에 따라 '어떤 전문가 모델'을 쓸지 결정하는 망 & 상담원/분류기 \\
\bottomrule
\end{tabular}%
}
\caption{앙상블 심화 학습을 위한 핵심 용어}
\end{table}

\newpage

% -----------------------------------------------------------------------------
% 3. 핵심 개념: 앙상블의 진화
% -----------------------------------------------------------------------------
\section{핵심 개념: 앙상블의 진화}

우리가 이전에 배운 Bagging과 Boosting은 강력하지만, 주로 \textbf{'같은 종류의 모델(주로 결정 트리)'}을 사용하는 한계가 있습니다. Blending과 Stacking은 \textbf{'서로 다른 모델'}을 결합한다는 점에서 차이가 있습니다.

\begin{analogybox}{어벤져스 팀 구성하기}
\begin{itemize}
    \item \textbf{Bagging (Random Forest):} 100명의 의사가 모여서 진단하고 다수결로 결정합니다. (모두가 의사라는 점에서 동질적임)
    \item \textbf{Blending/Stacking:} 의사, 변호사, 회계사가 모입니다. 그리고 이들의 의견을 종합해서 최종 결정을 내리는 '판사(Meta Model)'가 있습니다. (서로 다른 직업군이 모여 이질적임)
\end{itemize}
\end{analogybox}

% 비교 표
\begin{table}[h]
\centering
\label{tab:comparison}
\resizebox{\textwidth}{!}{%
\begin{tabular}{l l l l}
\toprule
\textbf{구분} & \textbf{구성 모델} & \textbf{결합 방식} & \textbf{특징} \\
\midrule
\textbf{Bagging} & 동질적 (주로 트리) & 병렬 학습 $\rightarrow$ 평균/투표 & 분산(Variance) 감소, 과적합 방지 \\
\textbf{Boosting} & 동질적 (약한 모델) & 순차 학습 $\rightarrow$ 가중치 합 & 편향(Bias) 감소, 오답 집중 학습 \\
\textbf{Blending} & \textbf{이질적 (다양함)} & 데이터 분할 $\rightarrow$ 메타 모델 학습 & 구현이 쉬우나 데이터 낭비 발생 \\
\textbf{Stacking} & \textbf{이질적 (다양함)} & 교차 검증(CV) $\rightarrow$ 메타 모델 학습 & 데이터 효율 높음, 연산 비용 큼 \\
\bottomrule
\end{tabular}%
}
\caption{다양한 앙상블 기법 비교}
\end{table}

\vspace{0.5cm}

% -----------------------------------------------------------------------------
% 4. Blending (블렌딩)
% -----------------------------------------------------------------------------
\section{Blending: 단순하고 직관적인 결합}

\subsection{기본 원리}
Blending은 데이터를 여러 조각으로 나누어, 일부는 기반 모델(Base Model)을 학습시키고, 나머지는 그 모델들을 평가(예측)하여 메타 모델(Meta Model)을 학습시키는 방식입니다.

\subsection{수행 절차 (Step-by-Step)}

\begin{enumerate}
    \item \textbf{데이터 분할:} 전체 데이터를 Train / Validation / Holdout / Test 세트로 나눕니다.
    \item \textbf{기반 모델 학습:} Train 세트로 여러 모델(선형회귀, 결정트리, KNN 등)을 학습시킵니다.
    \item \textbf{예측 생성 (1차):} 학습된 기반 모델들로 Validation 세트를 예측합니다.
    \item \textbf{메타 모델 학습:}
    \begin{itemize}
        \item 입력($X$): 기반 모델들이 내놓은 예측값들 (필요시 원본 데이터도 포함 가능)
        \item 정답($Y$): Validation 세트의 실제 정답
        \item 위 데이터를 사용해 메타 모델(보통 간단한 선형 모델)을 학습시킵니다.
    \end{itemize}
    \item \textbf{최종 평가:} Test 세트로 최종 성능을 확인합니다.
\end{enumerate}

\begin{warningbox}{Blending의 단점: 데이터 효율성}
Blending은 Validation과 Holdout 세트를 따로 떼어놓아야 하므로, 실제로 모델 학습에 사용할 수 있는 데이터의 양이 줄어듭니다. 데이터가 아주 많다면 괜찮지만, 데이터가 적다면 성능 저하가 올 수 있습니다.
\end{warningbox}

\newpage

% -----------------------------------------------------------------------------
% 5. Stacking (스태킹)
% -----------------------------------------------------------------------------
\section{Stacking: 데이터 낭비 없는 정교한 결합}

\subsection{Blending과의 차이점}
Stacking은 Blending과 비슷하지만, \textbf{교차 검증(Cross-Validation)}을 사용하여 데이터를 낭비하지 않고 모든 데이터를 학습에 활용합니다. 조금 더 복잡하지만 성능은 일반적으로 더 좋습니다.

\subsection{수행 절차 (Step-by-Step)}

\begin{enumerate}
    \item \textbf{데이터 분할:} 전체 데이터를 Train과 Test로 나눕니다.
    \item \textbf{교차 검증 (CV) 수행:} Train 데이터를 $K$개의 폴드(Fold)로 나눕니다 (예: 3-Fold).
    \item \textbf{예측값 생성 (Out-of-Fold Prediction):}
    \begin{itemize}
        \item Fold 1, 2로 학습하고 $\rightarrow$ Fold 3을 예측합니다.
        \item Fold 1, 3으로 학습하고 $\rightarrow$ Fold 2를 예측합니다.
        \item Fold 2, 3으로 학습하고 $\rightarrow$ Fold 1을 예측합니다.
        \item 이렇게 하면 모든 Train 데이터에 대한 '예측값'을 얻을 수 있습니다.
    \end{itemize}
    \item \textbf{메타 데이터 생성:} 위에서 얻은 예측값들을 모아서 새로운 입력 데이터셋($X_{meta}$)을 만듭니다.
    \item \textbf{메타 모델 학습:} $X_{meta}$와 실제 정답($Y$)을 이용해 메타 모델을 학습시킵니다.
    \item \textbf{최종 추론:} Test 데이터를 입력하면, 기반 모델들이 예측값을 내놓고, 이를 메타 모델이 받아 최종 결과를 출력합니다.
\end{enumerate}

\begin{analogybox}{Stacking 과정 비유}
팀원들이 돌아가며 모의고사를 봅니다.
\begin{itemize}
    \item A가 시험 볼 때 B, C가 공부한 걸로 채점해주고,
    \item B가 시험 볼 때 A, C가 도와주는 식입니다.
    \item 결과적으로 팀원 모두의 모의고사 성적표(예측값)가 모이게 되고, 선생님(메타 모델)은 "철수는 수학을 잘하고 영희는 영어를 잘하네"라는 패턴을 학습하여 실전 수능(Test Set)에 대비합니다.
\end{itemize}
\end{analogybox}

\subsection{Passthrough (패스스루)}
메타 모델을 학습시킬 때, 단순히 기반 모델의 예측값($\hat{y}$)만 쓰는 것이 아니라, \textbf{원본 데이터($X$)도 함께 넣어주는 기법}입니다.
\begin{itemize}
    \item \textbf{장점:} 메타 모델이 "어떤 데이터 상황에서 어떤 모델을 믿어야 할지" 더 잘 판단할 수 있습니다.
    \item \textbf{주의:} 데이터 차원이 커지고 복잡해질 수 있습니다.
\end{itemize}

\newpage

% -----------------------------------------------------------------------------
% 6. Mixture of Experts (전문가 혼합 모델)
% -----------------------------------------------------------------------------
\section{Mixture of Experts (MoE)}

\subsection{개념: 협력이 아닌 분업}
지금까지의 앙상블은 모든 모델이 힘을 합쳐(평균, 투표) 결과를 냈습니다. 하지만 MoE는 \textbf{"이 문제는 네가 전문가니까 네가 처리해"}라고 맡기는 방식입니다. 최근 대형 언어 모델(LLM)에서도 효율성을 위해 많이 사용되는 핵심 개념입니다.

\subsection{구조: 전문가와 게이트}
\begin{itemize}
    \item \textbf{전문가 (Experts):} 특정 데이터 영역이나 패턴에 강한 모델들입니다. (예: $X < 0$일 때는 선형회귀 A, $X \ge 0$일 때는 선형회귀 B)
    \item \textbf{게이팅 네트워크 (Gating Network):} 입력 데이터($X$)를 보고 어떤 전문가에게 일을 맡길지 결정하는 '분류기'입니다. 주로 Softmax 함수를 사용하여 각 전문가에 대한 가중치(확률)를 출력합니다.
\end{itemize}

\begin{equation}
\text{최종 예측 } \hat{y} = \sum_{i=1}^{K} g_i(x) \cdot f_i(x)
\end{equation}
\begin{itemize}
    \item $g_i(x)$: 게이팅 네트워크가 부여한 $i$번째 전문가의 가중치 (합은 1)
    \item $f_i(x)$: $i$번째 전문가의 예측값
\end{itemize}

\begin{analogybox}{병원 진료 시스템}
환자(데이터)가 병원에 오면, 접수처 직원(Gating Network)이 증상을 듣고 정형외과(Expert 1)로 보낼지, 내과(Expert 2)로 보낼지 결정합니다. 모든 의사가 다 같이 진료하는 것이 아니라, 가장 적합한 의사가 주도적으로 치료합니다.
\end{analogybox}

\subsection{주요 이슈: 전문가 붕괴 (Expert Collapse)}
학습 과정에서 특정 전문가 한 명만 너무 똑똑해지거나, 게이팅 네트워크가 한쪽으로만 데이터를 몰아주는 현상이 발생할 수 있습니다.
\begin{itemize}
    \item \textbf{증상:} 모든 데이터가 Expert 1에게만 가고, 나머지 Expert들은 놉니다.
    \item \textbf{원인:} 초기에 성능이 조금이라도 좋은 쪽에 몰아주다 보니 빈익빈 부익부 현상이 발생함.
    \item \textbf{해결:} 학습 시 손실 함수(Loss Function)를 조정하여 전문가들이 골고루 학습되도록 유도해야 합니다.
\end{itemize}

\newpage

% -----------------------------------------------------------------------------
% 7. 실습 및 구현 가이드 (Implementation)
% -----------------------------------------------------------------------------
\section{구현 및 활용 가이드 (Python/Scikit-Learn)}

Stacking은 `sklearn` 라이브러리를 통해 쉽게 구현할 수 있습니다. 아래는 개념적인 의사 코드(Pseudo-code) 흐름입니다.

\begin{tcolorbox}[colback=codebg, title=Stacking Regressor 구현 흐름]
\begin{verbatim}
from sklearn.ensemble import StackingRegressor
from sklearn.linear_model import LinearRegression
from sklearn.ensemble import RandomForestRegressor
from sklearn.neighbors import KNeighborsRegressor

# 1. 기반 모델(Experts) 정의
estimators = [
    ('rf', RandomForestRegressor(n_estimators=10)),
    ('knn', KNeighborsRegressor(n_neighbors=5)),
    ('lr', LinearRegression())
]

# 2. 메타 모델(Manager) 정의 - 보통 단순한 모델 사용
meta_model = LinearRegression()

# 3. Stacking 모델 구성 (cv=5는 5-Fold 교차검증 의미)
# passthrough=True 옵션: 원본 데이터도 메타 모델에 전달
clf = StackingRegressor(
    estimators=estimators,
    final_estimator=meta_model,
    cv=5,
    passthrough=True 
)

# 4. 학습 및 예측
clf.fit(X_train, y_train)
y_pred = clf.predict(X_test)
\end{verbatim}
\end{tcolorbox}

\subsection{언제 어떤 모델을 써야 할까? (의사결정)}
\begin{itemize}
    \item \textbf{데이터가 적고 빠른 결과가 필요하다:} $\rightarrow$ 단일 모델 또는 Random Forest
    \item \textbf{성능을 0.1\%라도 더 올리고 싶다 (Kaggle 등):} $\rightarrow$ Stacking (다양한 모델 결합)
    \item \textbf{데이터의 특성이 영역별로 확연히 다르다:} $\rightarrow$ Mixture of Experts (또는 Tree 모델)
    \item \textbf{해석(Interpretation)이 중요하다:} $\rightarrow$ Stacking을 쓰되, 메타 모델로 선형 회귀를 써서 각 기반 모델의 가중치(계수)를 확인한다.
\end{itemize}

\newpage

% -----------------------------------------------------------------------------
% 8. 체크리스트 및 FAQ
% -----------------------------------------------------------------------------
\section{학습 점검 체크리스트 \& FAQ}

\begin{itemize}[label=$\square$]
    \item \textbf{기반 모델의 다양성 확보:} 서로 비슷한 모델(예: 트리 3개)만 섞지 않았는가? (KNN, 선형, 트리를 섞는 것이 좋음)
    \item \textbf{데이터 누수(Leakage) 주의:} Stacking 시 Test 데이터를 학습 과정에 실수로 포함시키지 않았는가?
    \item \textbf{메타 모델의 단순성:} 메타 모델을 너무 복잡하게(예: 딥러닝) 만들어서 과적합되지 않았는가? (보통 선형 모델 권장)
    \item \textbf{전처리 일관성:} KNN 같은 거리 기반 모델을 섞을 때는 스케일링(StandardScaler)을 했는가? (트리는 필요 없지만 KNN은 필수)
\end{itemize}

\vspace{0.5cm}

\subsection{자주 묻는 질문 (FAQ)}

\textbf{Q1. Stacking이 Random Forest보다 무조건 좋은가요?} \\
A. 아닙니다. Stacking은 연산량이 훨씬 많고 복잡합니다. 데이터에 따라 Random Forest 같은 단일 앙상블 기법이 더 효율적일 수 있습니다. Stacking은 '마지막 성능 쥐어짜기' 용도로 주로 쓰입니다.

\textbf{Q2. Passthrough는 언제 켜야 하나요?} \\
A. 기반 모델들이 데이터의 모든 정보를 다 캡처하지 못했다고 판단될 때 켭니다. 하지만 원본 데이터의 차원(Feature 수)이 너무 많으면 메타 모델이 과적합될 수 있으니 주의해야 합니다.

\textbf{Q3. Mixture of Experts에서 '전문가'는 사람이 정해주나요?} \\
A. 아닙니다. 데이터 학습 과정(Gradient Descent)을 통해 모델이 스스로 "이 데이터 영역은 내가 처리할게"라고 학습하게 됩니다. 하지만 초기 설정이나 손실 함수 설계가 잘못되면 한 명의 전문가만 일하는 쏠림 현상이 발생할 수 있습니다.

\vspace{1cm}
\hrule
\vspace{0.3cm}
\begin{center}
    \small \textbf{학습 팁:} 처음에는 Random Forest나 Gradient Boosting 같은 검증된 단일 앙상블 기법을 먼저 마스터하세요. 그 후 성능의 한계를 느낄 때, 서로 다른 모델들을 섞는 Stacking을 시도해보는 것이 가장 효율적인 학습 순서입니다.
\end{center}

\end{document}
