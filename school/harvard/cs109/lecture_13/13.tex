%%%%%%%%%%%%%%%%%%%%%%%%%%%%%%%%%%%%%%%%%%%%%%%%%%%%%%%%%%%%%%%%%%%%%%%%%%%%%%%
% Harvard Academic Notes - 통합 마스터 템플릿
% 모든 강의 노트에 적용되는 통일된 스타일
% 버전: 2.1 - 가독성 개선 (선택적 최적화)
% 최종 수정일: 2025-11-17
%%%%%%%%%%%%%%%%%%%%%%%%%%%%%%%%%%%%%%%%%%%%%%%%%%%%%%%%%%%%%%%%%%%%%%%%%%%%%%%

\documentclass[11pt,a4paper]{article}

%========================================================================================
% 기본 패키지
%========================================================================================

% --- 한국어 지원 ---
\usepackage{kotex}

% --- 페이지 레이아웃 ---
\usepackage[top=20mm, bottom=20mm, left=20mm, right=18mm]{geometry}
\usepackage{setspace}
\onehalfspacing                      % 1.5배 줄간격
\setlength{\parskip}{0.5em}          % 문단 간격
\setlength{\parindent}{0pt}          % 들여쓰기 없음

% --- 표 관련 ---
\usepackage{booktabs}              % 고품질 표
\usepackage{tabularx}              % 자동 너비 조절 표
\usepackage{array}                 % 표 컬럼 확장
\usepackage{longtable}             % 여러 페이지 표
\renewcommand{\arraystretch}{1.1}  % 표 행간 조절

%========================================================================================
% 헤더 및 푸터
%========================================================================================

\usepackage{fancyhdr}
\pagestyle{fancy}
\fancyhf{}
\fancyhead[L]{\small\textit{CS109A: 데이터 과학 입문}}
\fancyhead[R]{\small\textit{Lecture 13}}
\fancyfoot[C]{\thepage}
\renewcommand{\headrulewidth}{0.5pt}
\renewcommand{\footrulewidth}{0.3pt}

% 첫 페이지는 헤더 없음
\fancypagestyle{firstpage}{
    \fancyhf{}
    \fancyfoot[C]{\thepage}
    \renewcommand{\headrulewidth}{0pt}
}

%========================================================================================
% 색상 정의 (파스텔 톤 + 다크모드 호환)
%========================================================================================

\usepackage[dvipsnames]{xcolor}

% 밝은 배경용 파스텔 색상
\definecolor{lightblue}{RGB}{220, 235, 255}      % 부드러운 파랑
\definecolor{lightgreen}{RGB}{220, 255, 235}     % 부드러운 초록
\definecolor{lightyellow}{RGB}{255, 250, 220}    % 부드러운 노랑
\definecolor{lightpurple}{RGB}{240, 230, 255}    % 부드러운 보라
\definecolor{lightgray}{gray}{0.95}              % 밝은 회색
\definecolor{lightpink}{RGB}{255, 235, 245}      % 부드러운 핑크
\definecolor{boxgray}{gray}{0.95}
\definecolor{boxblue}{rgb}{0.9, 0.95, 1.0}
\definecolor{boxred}{rgb}{1.0, 0.95, 0.95}

% 진한 색상 (테두리/제목용)
\definecolor{darkblue}{RGB}{50, 80, 150}
\definecolor{darkgreen}{RGB}{40, 120, 70}
\definecolor{darkorange}{RGB}{200, 100, 30}
\definecolor{darkpurple}{RGB}{100, 60, 150}

%========================================================================================
% 박스 환경 (tcolorbox) - 6가지 타입
%========================================================================================

\usepackage[most]{tcolorbox}
\tcbuselibrary{skins, breakable}

% 1. 개요 박스 (강의 시작 부분)
\newtcolorbox{overviewbox}[1][]{
    enhanced,
    colback=lightpurple,
    colframe=darkpurple,
    fonttitle=\bfseries\large,
    title=📚 강의 개요,
    arc=3mm,
    boxrule=1pt,
    left=8pt,
    right=8pt,
    top=8pt,
    bottom=8pt,
    breakable,
    #1
}

% 2. 요약 박스
\newtcolorbox{summarybox}[1][]{
    enhanced,
    colback=lightblue,
    colframe=darkblue,
    fonttitle=\bfseries,
    title=📝 핵심 요약,
    arc=2mm,
    boxrule=0.7pt,
    left=6pt,
    right=6pt,
    top=6pt,
    bottom=6pt,
    breakable,
    #1
}

% 3. 핵심 정보 박스
\newtcolorbox{infobox}[1][]{
    enhanced,
    colback=lightgreen,
    colframe=darkgreen,
    fonttitle=\bfseries,
    title=💡 핵심 정보,
    arc=2mm,
    boxrule=0.7pt,
    left=6pt,
    right=6pt,
    top=6pt,
    bottom=6pt,
    breakable,
    #1
}

% 4. 주의사항 박스
\newtcolorbox{warningbox}[1][]{
    enhanced,
    colback=lightyellow,
    colframe=darkorange,
    fonttitle=\bfseries,
    title=⚠️ 주의사항,
    arc=2mm,
    boxrule=0.7pt,
    left=6pt,
    right=6pt,
    top=6pt,
    bottom=6pt,
    breakable,
    #1
}

% 5. 예제 박스
\newtcolorbox{examplebox}[1][]{
    enhanced,
    colback=lightgray,
    colframe=black!60,
    fonttitle=\bfseries,
    title=📖 예제: #1,
    arc=2mm,
    boxrule=0.7pt,
    left=6pt,
    right=6pt,
    top=6pt,
    bottom=6pt,
    breakable,
}

% 6. 정의 박스
\newtcolorbox{definitionbox}[1][]{
    enhanced,
    colback=lightpink,
    colframe=purple!70!black,
    fonttitle=\bfseries,
    title=📌 정의: #1,
    arc=2mm,
    boxrule=0.7pt,
    left=6pt,
    right=6pt,
    top=6pt,
    bottom=6pt,
    breakable,
}

% 7. 중요 박스 (importantbox - warningbox와 유사)
\newtcolorbox{importantbox}[1][]{
    enhanced,
    colback=boxred,
    colframe=red!70!black,
    fonttitle=\bfseries,
    title=⚠️ 매우 중요: #1,
    arc=2mm,
    boxrule=0.7pt,
    left=6pt,
    right=6pt,
    top=6pt,
    bottom=6pt,
    breakable,
}

% 8. cautionbox (warningbox와 동일)
\let\cautionbox\warningbox
\let\endcautionbox\endwarningbox

%========================================================================================
% 코드 블록 설정 (밝은 배경)
%========================================================================================

\usepackage{listings}

\definecolor{codegray}{rgb}{0.5,0.5,0.5}
\definecolor{codepurple}{rgb}{0.58,0,0.82}
\definecolor{backcolour}{rgb}{0.95,0.95,0.95}

\lstset{
    basicstyle=\ttfamily\small,
    backgroundcolor=\color{lightgray},
    keywordstyle=\color{darkblue}\bfseries,
    commentstyle=\color{darkgreen}\itshape,
    stringstyle=\color{purple!80!black},
    numberstyle=\tiny\color{black!60},
    numbers=left,
    numbersep=8pt,
    breaklines=true,
    breakatwhitespace=false,
    frame=single,
    frameround=tttt,
    rulecolor=\color{black!30},
    captionpos=b,
    showstringspaces=false,
    tabsize=2,
    xleftmargin=15pt,
    xrightmargin=5pt,
    escapeinside={\%*}{*)}
}

% Python 코드 스타일
\lstdefinestyle{pythonstyle}{
    language=Python,
    morekeywords={self, True, False, None},
}

% SQL 코드 스타일
\lstdefinestyle{sqlstyle}{
    language=SQL,
    morekeywords={SELECT, FROM, WHERE, JOIN, GROUP, BY, ORDER, HAVING},
}

%========================================================================================
% 목차 스타일링
%========================================================================================

\usepackage{tocloft}
\renewcommand{\cftsecleader}{\cftdotfill{\cftdotsep}}
\setlength{\cftbeforesecskip}{0.4em}
\renewcommand{\cftsecfont}{\bfseries}
\renewcommand{\cftsubsecfont}{\normalfont}

%========================================================================================
% 표 및 그림
%========================================================================================

\usepackage{graphicx}              % 이미지
\usepackage{adjustbox}             % 표/박스 크기 조절

% 표 캡션 스타일
\usepackage{caption}
\captionsetup[table]{
    labelfont=bf,
    textfont=it,
    skip=5pt
}
\captionsetup[figure]{
    labelfont=bf,
    textfont=it,
    skip=5pt
}

%========================================================================================
% 수학
%========================================================================================

\usepackage{amsmath, amssymb, amsthm}

% 정리 환경
\theoremstyle{definition}
\newtheorem{theorem}{정리}[section]
\newtheorem{lemma}[theorem]{보조정리}
\newtheorem{proposition}[theorem]{명제}
\newtheorem{corollary}[theorem]{따름정리}
\newtheorem{definition}{정의}[section]
\newtheorem{example}{예제}[section]

%========================================================================================
% 하이퍼링크
%========================================================================================

\usepackage[
    colorlinks=true,
    linkcolor=blue!80!black,
    urlcolor=blue!80!black,
    citecolor=green!60!black,
    bookmarks=true,
    bookmarksnumbered=true,
    pdfborder={0 0 0}
]{hyperref}

% PDF 메타데이터는 각 문서에서 설정
\hypersetup{
    pdftitle={CS109A: 데이터 과학 입문 - Lecture 13},
    pdfauthor={강의 노트},
    pdfsubject={Academic Notes}
}

%========================================================================================
% 기타 유용한 패키지
%========================================================================================

\usepackage{enumitem}              % 리스트 커스터마이징
\setlist{nosep, leftmargin=*, itemsep=0.3em}

\usepackage{microtype}             % 타이포그래피 개선
\usepackage{footnote}              % 각주 개선
\usepackage{url}                   % URL 줄바꿈
\urlstyle{same}

%========================================================================================
% 사용자 정의 명령어
%========================================================================================

% 강조 텍스트
\newcommand{\important}[1]{\textbf{\textcolor{red!70!black}{#1}}}
\newcommand{\keyword}[1]{\textbf{#1}}
\newcommand{\term}[1]{\textit{#1}}
\newcommand{\code}[1]{\texttt{#1}}

% 용어 설명 (인라인)
\newcommand{\defterm}[2]{\textbf{#1}\footnote{#2}}

% 섹션 시작 전 페이지 분리
\newcommand{\newsection}[1]{\newpage\section{#1}}

%========================================================================================
% 문서 제목 스타일
%========================================================================================

\usepackage{titling}
\pretitle{\begin{center}\LARGE\bfseries}
\posttitle{\par\end{center}\vskip 0.5em}
\preauthor{\begin{center}\large}
\postauthor{\end{center}}
\predate{\begin{center}\large}
\postdate{\par\end{center}}

%========================================================================================
% 섹션 제목 간격
%========================================================================================

\usepackage{titlesec}
\titlespacing*{\section}{0pt}{1.5em}{0.8em}
\titlespacing*{\subsection}{0pt}{1.2em}{0.6em}
\titlespacing*{\subsubsection}{0pt}{1em}{0.5em}

%========================================================================================
% 메타 정보 박스 명령어
%========================================================================================

\newcommand{\metainfo}[4]{
\begin{tcolorbox}[
    colback=lightpurple,
    colframe=darkpurple,
    boxrule=1pt,
    arc=2mm,
    left=10pt,
    right=10pt,
    top=8pt,
    bottom=8pt
]
\begin{tabular}{@{}rl@{}}
▣ \textbf{강의명:} & #1 \\[0.3em]
▣ \textbf{주차:} & #2 \\[0.3em]
▣ \textbf{교수명:} & #3 \\[0.3em]
▣ \textbf{목적:} & \begin{minipage}[t]{0.75\textwidth}#4\end{minipage}
\end{tabular}
\end{tcolorbox}
}

%========================================================================================
% 끝
%========================================================================================


\begin{document}

\maketitle
\thispagestyle{firstpage}

\metainfo{CS109A: 데이터 과학 입문}{Lecture 13}{Pavlos Protopapas, Kevin Rader, Chris Gumb}{Lecture 13의 핵심 개념 학습}


\tableofcontents

\newpage

% --- 개요 ---
\section*{개요 (Overview)}

\begin{summarybox}
이 문서는 데이터 사이언스의 핵심 주제인 \textbf{'분류(Classification)'}를 다룹니다.

지금까지 다룬 '회귀(Regression)'가 \textbf{숫자(예: 주택 가격)}를 예측하는 문제였다면,
'분류'는 \textbf{범주(예: 심장병 유무, 전공)}를 예측하는 문제입니다.

이를 위해, 선형 회귀를 분류 문제에 바로 적용할 때 발생하는 문제점들을 살펴보고,
분류를 위한 핵심적인 파라메트릭 모델인 \textbf{로지스틱 회귀(Logistic Regression)}를 배웁니다.

\medskip
\textbf{주요 학습 목표:}
\begin{itemize}
    \item 회귀와 분류의 근본적인 차이를 설명할 수 있습니다.
    \item 왜 선형 회귀를 분류 문제에 사용하면 안 되는지 이해합니다.
    \item 로지스틱 회귀가 '확률'을 모델링하기 위해 \textbf{시그모이드(Sigmoid) 함수}를 사용하는 원리를 배웁니다.
    \item 로지스틱 회귀의 계수가 \textbf{로그-오즈(Log-Odds)} 관점에서 어떻게 해석되는지 설명할 수 있습니다.
    \item 로지스틱 회귀가 \textbf{최대가능도추정(MLE)}과 \textbf{이진 교차 엔트로피(BCE)}를 통해 어떻게 학습되는지 이해합니다.
    \item 모델의 \textbf{결정 경계(Decision Boundary)}가 어떻게 형성되는지 이해합니다.
\end{itemize}
또한, 이 주제에 앞서 중간고사에 포함될 수 있는 가설 검정, 순열 검정, 상호작용 항, 예측 구간 등에 대한 핵심 내용을 복습합니다.
\end{summarybox}

% --- 용어 정리 ---
\newpage
\section*{주요 용어 정리 (Terminology)}

본격적인 학습에 앞서, 오늘 다룰 핵심 용어들을 정리합니다.

\begin{table}[h!]
  \centering
  \caption{분류 및 로지스틱 회귀 핵심 용어}
  \label{tab:terminology}
  \begin{adjustbox}{width=\textwidth, center}
  \begin{tabular}{@{}llll@{}}
    \toprule
    \textbf{용어 (Korean)} & \textbf{원어 (English)} & \textbf{쉬운 설명} & \textbf{비고} \\
    \midrule
    가설 검정 & Hypothesis Testing & 데이터가 특정 가설(주장)을 지지하는지 통계적으로 판단하는 과정. & \textit{예: $\beta_1 = 0$인가?} \\
    p-value & p-value & '귀무가설(예: 관계가 없다)'이 맞다고 할 때, 현재 데이터만큼 극단적인 결과가 우연히 나올 확률. & \textit{낮을수록(보통 < 0.05) 관계가 있다고 봄.} \\
    순열 검정 & Permutation Test & 데이터의 라벨(Y)을 무작위로 섞어(순열), 우연만으로 원본 결과가 나오기 힘든 일인지 검증하는 기법. & \textit{t-test의 가정(정규성 등)이 깨졌을 때 유용.} \\
    부트스트랩 & Bootstrap & 원본 데이터에서 중복을 허용하여(복원추출) 여러 번 샘플링하는 기법. & \textit{주로 신뢰구간 '추정(Estimation)'에 사용.} \\
    상호작용 항 & Interaction Term & 한 변수(X1)의 효과가 다른 변수(X2)의 수준에 따라 달라지는 효과를 나타내는 항. & \textit{예: \texttt{sqft:type}} \\
    예측 구간 & Prediction Interval & \textit{새로운 개별 관측치(single point)}가 존재할 것으로 예상되는 범위. & \textit{신뢰 구간(평균)보다 항상 넓음.} \\
    \midrule
    분류 & Classification & 데이터가 어떤 '범주(Category)'에 속하는지 예측하는 문제. & \textit{예: 스팸(1) vs. 정상(0)} \\
    회귀 & Regression & 데이터로부터 '연속적인 숫자(Number)'를 예측하는 문제. & \textit{예: 주택 가격 예측} \\
    시그모이드 & Sigmoid Function & 모든 입력을 0과 1 사이의 S자 곡선으로 매핑하는 함수. & \textit{로지스틱 함수의 별명.} \\
    로지스틱 회귀 & Logistic Regression & 시그모이드 함수를 사용해 데이터가 특정 범주(예: 1)에 속할 \textbf{확률}을 모델링하는 기법. & \textit{이름은 '회귀'지만 '분류' 모델임.} \\
    오즈 & Odds & 성공 확률(p)을 실패 확률(1-p)로 나눈 값. \textit{($\frac{p}{1-p}$)} & \textit{p=0.8 이면 Odds = 4 (성공이 4배)} \\
    로그-오즈 & Log-Odds (Logit) & 오즈에 자연로그($\ln$)를 취한 값. \textit{$\ln(\frac{p}{1-p})$} & \textit{로지스틱 회귀는 로그-오즈를 선형 모델링함.} \\
    최대가능도추정 & MLE (Max Likelihood) & 주어진 데이터가 관측될 '가능성(Likelihood)'을 최대로 만드는 모델 파라미터를 찾는 방법. & \textit{로지스틱 회귀의 학습 원리.} \\
    이진 교차 엔트로피 & BCE (Binary Cross-Entropy) & 로지스틱 회귀의 손실 함수(Loss Function). (음의 로그 가능도) & \textit{BCE를 최소화 = 가능도를 최대화.} \\
    결정 경계 & Decision Boundary & 모델이 클래스 0과 1을 구분하는 경계선. (즉, $P(Y=1)=0.5$가 되는 지점) & \textit{기본은 선형, 다항식 항 추가 시 곡선 가능.} \\
    \bottomrule
  \end{tabular}
  \end{adjustbox}
\end{table}

% --- 복습 섹션 ---
\newpage
\section{복습: 선형 회귀와 통계적 추론 (Review)}

분류 모델을 배우기에 앞서, 선형 회귀 모델의 통계적 추론 방식을 복습합니다.

\subsection{가설 검정과 p-value}

\begin{itemize}
    \item \textbf{가설 검정(Hypothesis Testing)}이란, 우리가 모델에서 발견한 관계(예: 주택 크기와 가격의 관계)가 '진짜'인지, 아니면 '단순한 우연'인지 통계적으로 판단하는 공식적인 절차입니다.
    \item \textbf{귀무가설 ($H_0$):} "관계가 없다." (즉, $\beta_1 = 0$ 이다.)
    \item \textbf{대립가설 ($H_A$):} "관계가 있다." (즉, $\beta_1 \neq 0$ 이다.)
\end{itemize}

이때 사용되는 핵심 도구가 \textbf{t-통계량(t-statistic)}과 \textbf{p-value}입니다.

\begin{itemize}
    \item \textbf{t-통계량:} 우리가 추정한 계수($\hat{\beta}_1$)가 표준 오차(SE)에 비해 얼마나 큰지 나타내는 값입니다. (즉, 0에서 얼마나 멀리 떨어져 있는가?)
    $$ t = \frac{\hat{\beta}_1 - 0}{\text{SE}(\hat{\beta}_1)} $$
    \item \textbf{p-value:} 만약 귀무가설($H_0$)이 사실(관계가 없음)이라면, 우리가 관찰한 t-통계량만큼 극단적인 값이 순전히 '우연'에 의해 관찰될 확률입니다.
\end{itemize}

\begin{examplebox}[title=p-value의 직관적 해석]
    p-value가 0.001이라는 것은, "만약 주택 크기와 가격이 아무 관계가 없다면, 우리가 현재 데이터에서 본 것과 같은 강한 관계가 우연히 나타날 확률이 0.1\%밖에 되지 않는다"는 의미입니다.

    이 확률이 매우 낮기(보통 0.05 미만), 우리는 "이건 우연이 아니다"라고 결론 내리고 귀무가설을 기각합니다.
    즉, "주택 크기와 가격 사이에는 통계적으로 유의미한 관계가 있다"고 말합니다.
\end{examplebox}

\subsection{순열 검정 (Permutation Test)}

t-test는 데이터가 정규분포를 따르고, 분산이 동일하다(등분산성)는 가정이 필요합니다. 만약 데이터가 이 가정을 만족하지 못하면(예: 에러가 한쪽으로 몰려있음), t-test의 p-value를 신뢰할 수 없습니다.

\textbf{순열 검정}은 이러한 가정 없이 p-value를 계산하는 강력한 \textbf{재표본추출(Resampling)} 기법입니다.

\begin{itemize}
    \item \textbf{핵심 아이디어:} 귀무가설($H_0$)이 "X와 Y는 관계가 없다"는 것이므로, 이 가설을 시뮬레이션하기 위해 Y값(예: 주택 가격)을 무작위로 뒤섞어버립니다(shuffling).
    \item 이렇게 하면 X와 Y 사이의 실제 관계가 인위적으로 파괴됩니다.
    \item \textbf{절차:}
    \begin{enumerate}
        \item 원본 데이터에서 t-통계량 (또는 $\hat{\beta}_1$ 값)을 계산합니다. (예: $\hat{\beta}_1 = 0.5898$)
        \item Y 값을 무작위로 섞은 후, X와 다시 짝지어 모델을 적합하고 $\hat{\beta}_1^*$ 값을 계산합니다. (당연히 0에 가까운 값이 나올 것입니다.)
        \item 이 과정을 1,000번 (또는 10,000번) 반복하여, '관계가 없을 때' 나올 수 있는 $\hat{\beta}_1^*$ 값들의 분포(귀무 분포)를 만듭니다.
        \item 원본 값(0.5898)이 이 귀무 분포(대부분 0 근처)에서 얼마나 극단적인 위치에 있는지 확인하여 p-value를 계산합니다.
    \end{enumerate}
\end{itemize}

\subsection{순열 검정 vs. 부트스트랩 (Permutation vs. Bootstrap)}

두 기법 모두 데이터를 재표본추출하지만, 목적과 방식이 다릅니다.

\begin{table}[h!]
  \centering
  \caption{순열 검정과 부트스트랩 비교}
  \label{tab:perm_vs_boot}
  \begin{tabular}{@{}lll@{}}
    \toprule
    \textbf{특징} & \textbf{부트스트랩 (Bootstrap)} & \textbf{순열 검정 (Permutation Test)} \\
    \midrule
    \textbf{목적} & \textbf{추정 (Estimation)} & \textbf{가설 검정 (Hypothesis Testing)} \\
    \textbf{주요 산출물} & 신뢰 구간 (Confidence Interval) & p-value \\
    \textbf{샘플링 방식} & \textbf{복원 추출} (With Replacement) & \textbf{비복원 추출} (Shuffling) \\
    \textbf{기본 가정} & 원본 데이터가 모집단을 잘 대표함 & \textbf{귀무가설($H_0$)이 참} (X-Y 관계 없음) \\
    \bottomrule
  \end{tabular}
\end{table}

\subsection{상호작용 항 (Interaction Terms) 해석}

상호작용 항은 "X1의 효과가 X2의 수준에 따라 달라지는" 효과를 모델링합니다.
예를 들어, \texttt{price $\sim$ sqft + type + sqft:type} 모델을 살펴봅니다. (여기서 \texttt{type}의 기준(reference) 범주는 'Condo'입니다.)

$$ \text{price} = \beta_0 + \beta_1 \cdot \text{sqft} + \beta_2 \cdot \text{type[multifamily]} + \beta_3 \cdot (\text{sqft} \times \text{type[multifamily]}) + \dots $$

\begin{itemize}
    \item $\hat{\beta}_1$ (예: 0.6659): \textbf{기준 범주(Condo)의} 1 평방 피트당 가격 효과입니다.
    \item $\hat{\beta}_3$ (예: -0.2863): \textbf{Multifamily의} 1 평방 피트당 가격 효과가 \textbf{Condo에 비해} 얼마나 \textbf{다른지(차이)}를 나타냅니다.
    \item \textbf{Multifamily의 실제 평방 피트당 가격 효과}는 $\hat{\beta}_1 + \hat{\beta}_3$ (즉, $0.6659 - 0.2863$) 입니다.
    \item 이 상호작용 항의 p-value가 유의미하다면(예: < 0.05), "평방 피트에 따른 가격 변화율이 주택 유형(Condo vs. Multifamily)에 따라 통계적으로 유의미하게 다르다"고 결론 내릴 수 있습니다.
\end{itemize}

\subsection{신뢰 구간 vs. 예측 구간}

\begin{itemize}
    \item \textbf{신뢰 구간 (Confidence Interval):} (더 좁은 구간)
    \begin{itemize}
        \item "특정 X 값에 대한 \textbf{평균 Y값} ($\hat{Y}$)이 존재할 범위"에 대한 구간입니다.
        \item 즉, "우리의 \textit{회귀선 자체}가 얼마나 정확한가"를 보여줍니다.
        \item 데이터가 많아질수록 0에 가깝게 좁아질 수 있습니다.
    \end{itemize}
    \item \textbf{예측 구간 (Prediction Interval):} (더 넓은 구간)
    \begin{itemize}
        \item "특정 X 값에 대한 \textbf{새로운 개별 Y값} (single new observation)이 존재할 범위"에 대한 구간입니다.
        \item 이는 회귀선의 불확실성뿐만 아니라, 데이터 고유의 노이즈(\textbf{줄일 수 없는 오차, $\epsilon$})까지 포함합니다.
        \item 데이터가 무한히 많아져도, 이 고유의 노이즈 때문에 일정 수준 이하로 좁아지지 않습니다.
    \end{itemize}
\end{itemize}

% --- 분류(Classification)란? ---
\newpage
\section{분류 (Classification)란 무엇인가?}

\subsection{회귀 vs. 분류 (Regression vs. Classification)}

지금까지 우리가 다룬 문제는 대부분 \textbf{회귀(Regression)}였습니다.

\begin{itemize}
    \item \textbf{회귀 (Regression):} 예측하려는 값(Y)이 \textbf{연속적인 숫자(quantitative)}입니다.
    \begin{itemize}
        \item 예: 내일의 기온(25.5도), 주택 가격(\$500,000), 광고비 대비 매출액(\$18.5)
    \end{itemize}
    \item \textbf{분류 (Classification):} 예측하려는 값(Y)이 \textbf{범주형(qualitative, categorical)}입니다.
    \begin{itemize}
        \item 예: 내일 날씨(맑음, 흐림, 비), 환자의 심장병 유무(Yes, No), 학생의 전공(CS, Stats, Other)
    \end{itemize}
\end{itemize}

\begin{examplebox}[title=회귀 질문 vs. 분류 질문]
    \begin{itemize}
        \item \textbf{회귀 질문:} "이 학생의 최고 심박수는 \textbf{몇}입니까?" (예측: 150 bpm)
        \item \textbf{분류 질문:} "이 학생은 심장병이 \textbf{있습니까, 없습니까}?" (예측: Yes)
    \end{itemize}
\end{examplebox}

\subsection{왜 선형 회귀를 분류 문제에 쓰면 안 되는가?}

Y값이 범주형일 때, 선형 회귀($Y = \beta_0 + \beta_1 X$)를 그냥 사용하면 두 가지 심각한 문제가 발생합니다.

\subsubsection{문제 1: 다중 클래스의 잘못된 순서 (False Ordering)}

Y값이 3개 이상의 범주(multi-class)를 가질 때를 생각해봅시다. (예: 전공)
우리가 이 범주를 숫자로 강제 인코딩했다고 가정합니다.
($Y = 1$ if CS, $Y = 2$ if Statistics, $Y = 3$ if Otherwise)

\begin{warningbox}
선형 회귀는 이 숫자들 사이에 \textbf{수학적인 관계가 있다고 가정}합니다.
\begin{itemize}
    \item 모델은 'CS(1)에서 Stats(2)로의 변화' (+1)와 'Stats(2)에서 Other(3)로의 변화' (+1)를 \textbf{동일한 크기의 변화로 취급}합니다.
    \item 이는 완전히 무의미한 가정입니다. 만약 'CS=3, Stats=1'로 순서를 바꾸면 모델의 결과가 완전히 달라집니다.
\end{itemize}
범주형 변수에는 자연스러운 순서나 등간격이 없습니다. (이러한 변수를 \textit{nominal}하다고 합니다.)
\end{warningbox}

\subsubsection{문제 2: 확률 범위를 벗어남 (Probability Bounds Violation)}

Y값이 2개의 범주(binary)만 가질 때(예: 심장병 Yes=1, No=0)는 순서 문제는 없지만, 더 심각한 문제가 발생합니다.

이때 선형 회귀는 $P(Y=1)$ (즉, 심장병에 걸릴 '확률')을 예측하도록 학습될 수 있습니다.
하지만 \textbf{확률은 반드시 0과 1 사이의 값}이어야 합니다.

\begin{warningbox}
선형 회귀의 예측값($\hat{Y}$)은 직선이므로, \textbf{범위의 제한이 없습니다.}
\begin{itemize}
    \item X가 매우 작으면(예: MaxHR이 매우 낮음), 모델이 $P(Y=1) = 1.1$ (110\%) 과 같이 1보다 큰 값을 예측할 수 있습니다.
    \item X가 매우 크면(예: MaxHR이 매우 높음), 모델이 $P(Y=1) = -0.1$ (-10\%) 과 같이 0보다 작은 값을 예측할 수 있습니다.
\end{itemize}
이는 수학적으로, 논리적으로 완전히 잘못된 예측입니다.
\end{warningbox}

\begin{center}
    \textit{(참고: 강의 슬라이드 30페이지의 그림은 선형 회귀선이 Y=0과 Y=1 데이터를 벗어나 각각 0 미만, 1 초과의 확률을 예측하는 문제점을 시각적으로 보여줍니다.)}
\end{center}

% --- 로지스틱 회귀 ---
\newpage
\section{로지스틱 회귀 (Logistic Regression)}

\subsection{핵심 아이디어: S-커브 (The S-Curve)}

선형 회귀의 문제(0~1 범위를 벗어남)를 해결하기 위해, 우리는 예측값이 항상 0과 1 사이에 머무르도록 하는 새로운 함수가 필요합니다.

\begin{itemize}
    \item \textbf{(1) 한 줄 핵심 요약:} 모든 입력을 0과 1 사이의 S자 곡선으로 '압축'시키는 \textbf{시그모이드(Sigmoid) 함수}를 사용해 확률을 모델링합니다.
    \item \textbf{(2) 직관적 예시:} 선형 회귀의 무한한 직선($h$)을 가져와서, 이 직선을 양쪽 끝에서 눌러 0이라는 '바닥'과 1이라는 '천장'에 닿도록 찌그러뜨린 모양을 상상하면 됩니다.
    \item \textbf{(3) 기술적 설명:}
        \begin{itemize}
            \item 먼저, 선형 회귀와 똑같은 부분($h$)을 계산합니다: $h = \beta_0 + \beta_1 X$
            \item 이 $h$ 값을 시그모이드(로지스틱) 함수에 통과시켜 확률 $p$를 얻습니다.
        \end{itemize}
\end{itemize}

$$ p = P(Y=1) = \frac{1}{1 + e^{-h}} = \frac{1}{1 + e^{-(\beta_0 + \beta_1 X)}} $$

이 함수는 $h$ 값에 따라 항상 0과 1 사이의 값을 반환합니다.
\begin{itemize}
    \item $h \to +\infty$ (아주 큰 양수)이면, $e^{-h} \to 0$, 따라서 $p \to \frac{1}{1+0} = 1$
    \item $h \to -\infty$ (아주 큰 음수)이면, $e^{-h} \to \infty$, 따라서 $p \to \frac{1}{1+\infty} = 0$
    \item $h = 0$ 이면, $e^{0} = 1$, 따라서 $p \to \frac{1}{1+1} = 0.5$
\end{itemize}

\begin{tcolorbox}[colback=gray!5, colframe=gray!60, title=로지스틱 회귀의 이름]
    이름은 '회귀(Regression)'이지만, 하는 일은 \textbf{'분류(Classification)'}입니다.
    이는 모델이 확률이라는 '연속적인 숫자'를 예측한 뒤, 그 확률을 기준으로 '범주'를 결정하기 때문입니다.
\end{tcolorbox}


\subsection{계수(Coefficient)의 해석: 오즈(Odds)와 로그-오즈(Log-Odds)}

$p = \frac{1}{1 + e^{-h}}$ 공식은 $\beta_1$을 해석하기 매우 어렵습니다.
"X가 1 증가할 때, $p$는 $\frac{1}{1 + e^{-(\beta_0 + \beta_1 (X+1))}}$ ... 만큼 변한다"는 식은 직관적이지 않습니다.

대신, 위 공식을 $h$에 대해 정리하면(즉, 역함수를 구하면) 훨씬 강력한 해석이 가능해집니다.

$$ \ln\left( \frac{p}{1-p} \right) = \beta_0 + \beta_1 X $$

이 방정식의 좌변을 이해하기 위해 두 가지 개념을 도입합니다.

\begin{itemize}
    \item \textbf{오즈 (Odds):} (성공 확률) / (실패 확률)
    $$ \text{Odds} = \frac{p}{1-p} $$
    \begin{itemize}
        \item $p=0.5$ (확률 50\%) $\implies$ Odds = 1 (1:1)
        \item $p=0.8$ (확률 80\%) $\implies$ Odds = 4 (실패보다 성공이 4배 높음)
        \item $p=0.2$ (확률 20\%) $\implies$ Odds = 0.25 (성공보다 실패가 4배 높음)
    \end{itemize}
    \item \textbf{로그-오즈 (Log-Odds) 또는 로짓(Logit):} 오즈에 자연로그($\ln$)를 취한 값.
    $$ \text{Logit}(p) = \ln(\text{Odds}) = \ln\left( \frac{p}{1-p} \right) $$
\end{itemize}

\begin{summarybox}
\textbf{로지스틱 회귀의 핵심 해석:}
로지스틱 회귀는 \textbf{로그-오즈(Log-Odds)를 선형 회귀로 모델링}하는 것입니다!

"X가 1단위 증가할 때, \textbf{로그-오즈}가 $\beta_1$만큼 \textbf{더하기(additive)}로 변한다."

이것은 "X가 1단위 증가할 때, \textbf{오즈(Odds)}가 $e^{\beta_1}$만큼 \textbf{곱하기(multiplicative)}로 변한다."는 의미와 같습니다.
\end{summarybox}

\begin{examplebox}[title= $\beta_1$ 값 해석하기]
    심장병 예측 모델에서 $X=$ MaxHR (최대 심박수)에 대한 계수가 $\hat{\beta}_1 = -0.0434$ 라고 가정합니다.

    \begin{itemize}
        \item \textbf{로그-오즈 해석 (어려움):}
        최대 심박수가 1 증가할 때마다, 심장병에 걸릴 로그-오즈가 -0.0434만큼 감소합니다.
        
        \item \textbf{오즈 해석 (쉬움):}
        $e^{\beta_1} = e^{-0.0434} \approx 0.957$
        
        이는 최대 심박수가 1 증가할 때마다, 심장병에 걸릴 \textbf{오즈}가 약 \textbf{0.957배}가 된다는 의미입니다. (즉, 약 4.3\%씩 감소합니다.)
        
        \item 만약 $\hat{\beta}_1 = 0$ 이었다면? $e^0 = 1$ 이므로, 오즈가 1배 (변화 없음)가 됩니다. 즉, X와 Y는 관계가 없습니다.
        \item 만약 $\hat{\beta}_1 = 0.7$ 이었다면? $e^{0.7} \approx 2.01$ 이므로, 오즈가 약 2배 증가합니다.
    \end{itemize}
\end{examplebox}

\subsection{모델 추정: 최대가능도추정 (MLE)}

최적의 S-커브 (즉, 최적의 $\beta_0, \beta_1$)는 어떻게 찾을까요?

\begin{itemize}
    \item \textbf{선형 회귀의 경우:} 손실 함수인 \textbf{MSE(평균 제곱 오차)}를 최소화하는 $\beta$값을 찾았습니다. (이는 Y가 정규분포를 따른다고 가정한 것과 같습니다.)
    \item \textbf{로지스틱 회귀의 경우:} Y가 0 또는 1이므로, \textbf{베르누이 분포(Bernoulli Distribution)} (동전 던지기)를 따른다고 가정합니다.
\end{itemize}

이때 사용하는 학습 원리가 \textbf{최대가능도추정 (MLE, Maximum Likelihood Estimation)}입니다.

\begin{itemize}
    \item \textbf{가능도 (Likelihood):} "현재 우리가 가정한 S-커브(모델)가, 지금 우리가 가진 데이터(Y=0 또는 1)를 만들어 냈을 총 확률"입니다.
    \item 모델이 예측한 확률이 $p_i$일 때:
    \begin{itemize}
        \item 실제 값이 $y_i = 1$ (성공)이면: 이 관측치의 가능도는 $p_i$
        \item 실제 값이 $y_i = 0$ (실패)이면: 이 관측치의 가능도는 $1 - p_i$
    \end{itemize}
    \item 이를 하나의 수식으로 표현하면 $L_i = p_i^{y_i} \cdot (1-p_i)^{1-y_i}$ 입니다.
    \item \textbf{전체 가능도 (Total Likelihood):} 모든 데이터가 독립이라고 가정하므로, 모든 관측치의 가능도를 곱합니다.
    $$ L(\beta) = \prod_{i=1}^n L_i = \prod_{i=1}^n p_i^{y_i} \cdot (1-p_i)^{1-y_i} $$
    \item \textbf{MLE의 목표:} 이 $L(\beta)$ 값을 \textbf{최대}로 만드는 $\beta$ (즉, $\beta_0, \beta_1$)를 찾는 것입니다.
\end{itemize}

\begin{warningbox}[title=손실 함수: 이진 교차 엔트로피 (BCE)]
곱셈($\prod$)은 미분하기 매우 어렵습니다. 따라서 계산을 쉽게 하기 위해 양변에 로그($\log$)를 취합니다. 이를 \textbf{로그 가능도(Log-Likelihood)}라고 합니다. (로그를 취해도 최대가 되는 지점은 변하지 않습니다.)

$$ l(\beta) = \log L(\beta) = \sum_{i=1}^n \left[ y_i \log(p_i) + (1-y_i) \log(1-p_i) \right] $$

컴퓨터는 보통 '최소화' 문제를 풉니다. 따라서 위 값에 마이너스(-)를 붙인 \textbf{음의 로그 가능도 (Negative Log-Likelihood)}를 \textbf{최소화}합니다.

$$ \text{Loss} = -l(\beta) = -\sum_{i=1}^n \left[ y_i \log(p_i) + (1-y_i) \log(1-p_i) \right] $$

이 손실 함수를 \textbf{이진 교차 엔트로피 (Binary Cross-Entropy, BCE)}라고 부릅니다. 선형 회귀가 MSE를 최소화하듯, 로지스틱 회귀는 BCE를 최소화합니다.
\end{warningbox}

% --- 다중 로지스틱 회귀와 결정 경계 ---
\newpage
\section{다중 로지스틱 회귀와 결정 경계}

\subsection{다중 로지스틱 회귀 (Multiple Logistic Regression)}

선형 회귀를 다중 선형 회귀로 확장했듯이, 로지스틱 회귀도 여러 개의 예측 변수(X)를 사용하도록 쉽게 확장할 수 있습니다.

단순히 로그-오즈에 대한 선형 방정식을 확장하면 됩니다.

$$ \ln\left( \frac{p}{1-p} \right) = \beta_0 + \beta_1 X_1 + \beta_2 X_2 + \dots + \beta_p X_p $$

\begin{itemize}
    \item \textbf{해석:} $\beta_j$의 해석은 \textbf{"다른 모든 변수($X_k$)가 일정하다고 가정할 때"}라는 조건이 추가됩니다.
    \item 즉, $e^{\beta_j}$는 다른 변수들이 고정된 상태에서 $X_j$가 1단위 증가할 때 오즈(Odds)의 곱셈 변화량입니다.
    \item 다중 공선성, 과적합 등 다중 선형 회귀에서 발생했던 문제들이 여기서도 동일하게 발생하며, 정규화(Ridge, Lasso) 등이 필요할 수 있습니다.
\end{itemize}

\subsection{결정 경계 (Decision Boundaries)}

로지스틱 회귀는 확률($p$)을 반환합니다. 이를 '분류' (Yes/No)로 바꾸려면 \textbf{결정 임계값(Threshold)}이 필요합니다.

\begin{itemize}
    \item \textbf{기본 임계값:} $p \ge 0.5$ 이면 $Y=1$ (Yes)로 분류, $p < 0.5$ 이면 $Y=0$ (No)로 분류합니다.
    \item \textbf{결정 경계 (Decision Boundary):} 모델이 Yes와 No를 구분하는 경계선, 즉 $p=0.5$가 되는 지점입니다.
\end{itemize}

$p=0.5$는 $\text{Odds} = \frac{0.5}{1-0.5} = 1$을 의미하고, $\text{Log-Odds} = \ln(1) = 0$을 의미합니다.
따라서 결정 경계는 로지스틱 회귀의 선형 방정식 부분이 0이 되는 지점입니다.

$$ \textbf{결정 경계:} \quad \beta_0 + \beta_1 X_1 + \beta_2 X_2 + \dots + \beta_p X_p = 0 $$

\subsection{결정 경계의 형태: 선형과 비선형}

\begin{itemize}
    \item \textbf{선형 경계 (Linear Boundary):}
    기본적인 다중 로지스틱 회귀 모델($\beta_0 + \beta_1 X_1 + \beta_2 X_2 = 0$)은 $X_1$과 $X_2$에 대한 1차 방정식이므로, 결정 경계는 항상 \textbf{직선} (또는 3D에서는 평면)이 됩니다.
    
        \begin{center}
        \textit{(참고: 강의 슬라이드 83페이지는 MaxHR과 Chol 변수만 사용했을 때, 두 클래스를 나누는 경계가 직선으로 나타나는 것을 보여줍니다.)}
    \end{center}

    \item \textbf{비선형 경계 (Non-linear Boundary):}
    하지만 데이터가 직선으로 잘 나뉘지 않는 경우가 많습니다.
    
        \begin{center}
        \textit{(참고: 강의 슬라이드 84페이지는 두 클래스가 곡선 형태로 섞여있어 직선 경계로는 잘 나눌 수 없는 예시를 보여줍니다.)}
    \end{center}

    \textbf{해결책:} 선형 회귀에서 다항 회귀를 사용했듯이, 로지스틱 회귀에도 \textbf{변수들을 변형하여 추가}합니다.
    
    \begin{enumerate}
        \item \textbf{상호작용 항 추가:} $X_3 = X_1 \cdot X_2$
        \item \textbf{다항 항 추가:} $X_4 = X_1^2$, $X_5 = X_2^2$
    \end{enumerate}

    이렇게 변형된 변수들로 모델을 만들면,
    $$ \ln\left( \frac{p}{1-p} \right) = \beta_0 + \beta_1 X_1 + \beta_2 X_2 + \beta_3 (X_1 \cdot X_2) + \beta_4 X_1^2 + \beta_5 X_2^2 $$
    
    결정 경계 ($= 0$)는 $X_1, X_2$에 대한 2차 방정식이 되므로, \textbf{곡선, 원, 타원} 형태의 비선형 경계를 만들어낼 수 있습니다.
\end{itemize}

\begin{warningbox}[title=모델은 여전히 '선형'입니다]
    비선형 경계를 만들었음에도 불구하고, 이 모델은 여전히 '선형 모델'로 분류됩니다.
    
    왜냐하면 모델은 \textbf{계수($\beta$)}에 대해 선형이기 때문입니다. ($X_1^2$를 그냥 $Z_4$라는 새로운 변수로 보면, 모델은 $\beta_0 + \beta_1 Z_1 + \dots + \beta_5 Z_5$ 형태의 선형 결합입니다.)
    
    우리는 입력 \textbf{변수(X)를 비선형으로 변환(feature engineering)}하여, 선형 모델로 비선형 경계를 찾도록 한 것입니다.
\end{warningbox}

% --- 실습 코드 ---
\newpage
\section{실습 코드 예제 (Python)}

강의에서 사용된 \texttt{statsmodels} 및 \texttt{scikit-learn} 코드 예제입니다.

\subsection{순열 검정 (Permutation Test) 예제 (Numpy)}
\begin{lstlisting}[language=Python, caption={Numpy를 이용한 순열 검정 구현}, label={lst:permutation_test}, breaklines=true]
import numpy as np
import sklearn.linear_model

nsims = 1000
X = homes[['sqft']]
y = homes[['price']]

indices = np.arange(0, len(homes))
beta1_permute = []

for i in np.arange(0, nsims):
    # 1. 귀무가설을 시뮬레이션하기 위해 Y의 인덱스를 섞습니다.
    np.random.shuffle(indices)
    y_permute = y.iloc[indices]
    
    # 2. 섞인 Y와 원본 X로 모델을 적합합니다.
    permute_ols = sk.linear_model.LinearRegression().fit(X, y_permute)
    
    # 3. 귀무가설 하의 기울기(beta1)를 저장합니다.
    beta1_permute.append(permute_ols.coef_[0][0])

# beta1_permute의 분포 (귀무 분포)와
# 원본 데이터의 기울기(beta1_observed)를 비교하여 p-value를 계산합니다.
\end{lstlisting}

\subsection{상호작용 모델 (Statsmodels)}
\begin{lstlisting}[language=Python, caption={Statsmodels를 이용한 상호작용 모델 적합}, label={lst:interaction_model}, breaklines=true]
import statsmodels.formula.api as smf

# price ~ sqft + type + sqft:type 모델을 적합합니다.
# 'type'은 범주형 변수로 자동 인식됩니다.
interaction_ols = smf.ols(formula="price ~ sqft * type", 
                          data=homes).fit()

# 결과 요약 출력
print(interaction_ols.summary())

# coef
# ---------------------------------------------------------------------
# Intercept                 170.5182
# type[T.multifamily]       142.0626
# type[T.singlefamily]     -708.8103
# sqft                        0.6659  <-- Condo(기준)의 sqft 기울기
# sqft:type[T.multifamily]   -0.2863  <-- Condo 대비 multifamily의 기울기 '차이'
# sqft:type[T.singlefamily]   0.4769  <-- Condo 대비 singlefamily의 기울기 '차이'
# ...
\end{lstlisting}

\subsection{단순 로지스틱 회귀 (Scikit-learn)}
\begin{lstlisting}[language=Python, caption={Scikit-learn을 이용한 단순 로지스틱 회귀}, label={lst:simple_logistic}, breaklines=true]
from sklearn.linear_model import LogisticRegression

# X (예측 변수, 2D 배열이어야 함)
X_hr = df_heart[['MaxHR']] 
# Y (반응 변수, 1D 배열)
y_ahd = df_heart['AHD'] 

# penalty='none' : 정규화(Ridge/Lasso)를 사용하지 않음
logreg = LogisticRegression(penalty='none')
logreg.fit(X_hr, y_ahd)

# beta_1 (기울기)
print('Estimated beta1: \n', logreg.coef_)
# [[-0.04341112]]

# beta_0 (절편)
print('Estimated beta0: \n', logreg.intercept_)
# [6.3249492]

# 모델: log(odds) = 6.325 - 0.0434 * MaxHR
\end{lstlisting}

\subsection{비선형 결정을 위한 다항 로지스틱 회귀 (Scikit-learn)}
\begin{lstlisting}[language=Python, caption={다항 및 상호작용 항을 사용한 로지스틱 회귀}, label={lst:poly_logistic}, breaklines=true]
# 1. 비선형 특성 생성
df_heart['Interaction'] = df_heart.MaxHR * df_heart.Chol
df_heart['MaxHR_sq'] = df_heart.MaxHR\textbf{2
df_heart['Chol_sq'] = df_heart.Chol}2

# 사용할 변수 리스트
features = ['MaxHR', 'Chol', 'Interaction', 'MaxHR_sq', 'Chol_sq']

data_x = df_heart[features]
data_y = df_heart['AHD']

# 2. 모델 적합
logreg_poly = LogisticRegression(penalty='none', fit_intercept=True, max_iter=1000)
logreg_poly.fit(data_x, data_y)

# 3. 계수 확인
print('Estimated betas: \n', logreg_poly.coef_)
print('Estimated beta0: \n', logreg_poly.intercept_)

# 이 모델의 결정 경계 (log_odds = 0)는 X1, X2에 대한 2차식이 되어
# 원형 또는 타원형의 곡선 경계를 생성합니다.
\end{lstlisting}


% --- 1페이지 요약 ---
\newpage
\section*{빠르게 훑어보기 (1-Page Summary)}

\begin{tcolorbox}[colback=white, colframe=black!70, title=분류(Classification)와 로지스틱 회귀 핵심 요약]
    
    \begin{tcolorbox}[colback=blue!5, colframe=blue!60, title=1. 문제 정의: 회귀 vs. 분류]
        \begin{itemize}
            \item \textbf{회귀 (Regression):} \textbf{숫자} 예측 (예: 가격, 온도). \textit{Tool: 선형 회귀}
            \item \textbf{분류 (Classification):} \textbf{범주} 예측 (예: Yes/No, 스팸/정상). \textit{Tool: 로지스틱 회귀}
        \end{itemize}
    \end{tcolorbox}
    
    \begin{tcolorbox}[colback=red!5, colframe=red!60, title=2. 왜 선형 회귀는 분류에 실패하는가?]
        \begin{itemize}
            \item \textbf{이유 1 (다중 클래스):} \texttt{1=CS}, \texttt{2=Stats} 처럼 강제 인코딩 시, 무의미한 \textbf{순서}와 \textbf{간격}을 가정하게 됨.
            \item \textbf{이유 2 (이진 클래스):} 예측값이 \textbf{확률의 범위 [0, 1]을 벗어남} (예: 110\% 또는 -10\%).
        \end{itemize}
    \end{tcolorbox}
    
    \begin{tcolorbox}[colback=green!5, colframe=green!70, title=3. 해결책: 로지스틱 회귀와 시그모이드]
        \begin{itemize}
            \item 선형 회귀의 결과($h = \beta_0 + \beta_1 X$)를 \textbf{시그모이드(Sigmoid) 함수}에 넣어 0~1 사이의 확률값($p$)으로 '압축'시킴.
            $$ p = P(Y=1) = \frac{1}{1 + e^{-h}} $$
        \end{itemize}
    \end{tcolorbox}
    
    \begin{tcolorbox}[colback=orange!5, colframe=orange!70, title=4. 핵심 해석: 로그-오즈 (Log-Odds)]
        \begin{itemize}
            \item 모델의 공식을 변형하면 \textbf{로그-오즈(Logit)}가 X에 대한 선형 함수임을 알 수 있음.
            $$ \ln\left( \frac{p}{1-p} \right) = \beta_0 + \beta_1 X $$
            \item \textbf{$\beta_1$의 해석:} X가 1 증가할 때, \textbf{오즈(Odds)}가 $\mathbf{e^{\beta_1}}$ \textbf{배}가 된다.
            \item $e^{\beta_1} > 1$ (긍정적 관계), $e^{\beta_1} = 1$ (관계 없음), $e^{\beta_1} < 1$ (부정적 관계)
        \end{itemize}
    \end{tcolorbox}
    
    \begin{tcolorbox}[colback=purple!5, colframe=purple!70, title=5. 학습 원리: MLE와 BCE]
        \begin{itemize}
            \item \textbf{가정:} Y는 \textbf{베르누이 분포}를 따름 (동전 던지기).
            \item \textbf{목표:} 관측된 데이터가 나타날 \textbf{가능도(Likelihood)}를 \textbf{최대}로 만드는 $\beta$를 찾음 (MLE).
            \item \textbf{손실 함수:} \textbf{이진 교차 엔트로피(BCE)} (음의 로그 가능도)를 \textbf{최소화}함.
        \end{itemize}
    \end{tcolorbox}

    \begin{tcolorbox}[colback=gray!5, colframe=gray!70, title=6. 결정 경계 (Decision Boundary)]
        \begin{itemize}
            \item 모델이 0과 1을 나누는 경계선. (즉, $p=0.5$ 또는 $\text{Log-Odds}=0$이 되는 지점)
            \item \textbf{기본 모델:} $\beta_0 + \beta_1 X_1 + \beta_2 X_2 = 0 \implies$ \textbf{직선} 경계.
            \item \textbf{다항/상호작용 모델:} $\beta_0 + \dots + \beta_4 X_1^2 + \beta_5 X_2^2 = 0 \implies$ \textbf{곡선} 경계.
        \end{itemize}
    \end{tcolorbox}
    
\end{tcolorbox}

\end{document}
