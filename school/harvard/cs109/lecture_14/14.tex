%%%%%%%%%%%%%%%%%%%%%%%%%%%%%%%%%%%%%%%%%%%%%%%%%%%%%%%%%%%%%%%%%%%%%%%%%%%%%%%
% Harvard Academic Notes - 통합 마스터 템플릿
% 모든 강의 노트에 적용되는 통일된 스타일
% 버전: 2.0
% 최종 수정일: 2025-10-26
%%%%%%%%%%%%%%%%%%%%%%%%%%%%%%%%%%%%%%%%%%%%%%%%%%%%%%%%%%%%%%%%%%%%%%%%%%%%%%%

\documentclass[11pt,a4paper]{article}

%========================================================================================
% 기본 패키지
%========================================================================================

% --- 한국어 지원 ---
\usepackage{kotex}

% --- 페이지 레이아웃 ---
\usepackage[margin=25mm]{geometry}
\usepackage{setspace}
\onehalfspacing                      % 1.5배 줄간격
\setlength{\parskip}{0.6em}          % 문단 간격
\setlength{\parindent}{0pt}          % 들여쓰기 없음

% --- 표 관련 ---
\usepackage{booktabs}              % 고품질 표
\usepackage{tabularx}              % 자동 너비 조절 표
\usepackage{array}                 % 표 컬럼 확장
\usepackage{longtable}             % 여러 페이지 표
\renewcommand{\arraystretch}{1.2}  % 표 행간 조절

%========================================================================================
% 헤더 및 푸터
%========================================================================================

\usepackage{fancyhdr}
\pagestyle{fancy}
\fancyhf{}
\fancyhead[L]{\small\textit{CS109A: 데이터 과학 입문}}
\fancyhead[R]{\small\textit{Lecture 14}}
\fancyfoot[C]{\thepage}
\renewcommand{\headrulewidth}{0.5pt}
\renewcommand{\footrulewidth}{0.3pt}

% 첫 페이지는 헤더 없음
\fancypagestyle{firstpage}{
    \fancyhf{}
    \fancyfoot[C]{\thepage}
    \renewcommand{\headrulewidth}{0pt}
}

%========================================================================================
% 색상 정의 (파스텔 톤 + 다크모드 호환)
%========================================================================================

\usepackage[dvipsnames]{xcolor}

% 밝은 배경용 파스텔 색상
\definecolor{lightblue}{RGB}{220, 235, 255}      % 부드러운 파랑
\definecolor{lightgreen}{RGB}{220, 255, 235}     % 부드러운 초록
\definecolor{lightyellow}{RGB}{255, 250, 220}    % 부드러운 노랑
\definecolor{lightpurple}{RGB}{240, 230, 255}    % 부드러운 보라
\definecolor{lightgray}{gray}{0.95}              % 밝은 회색
\definecolor{lightpink}{RGB}{255, 235, 245}      % 부드러운 핑크
\definecolor{boxgray}{gray}{0.95}
\definecolor{boxblue}{rgb}{0.9, 0.95, 1.0}
\definecolor{boxred}{rgb}{1.0, 0.95, 0.95}

% 진한 색상 (테두리/제목용)
\definecolor{darkblue}{RGB}{50, 80, 150}
\definecolor{darkgreen}{RGB}{40, 120, 70}
\definecolor{darkorange}{RGB}{200, 100, 30}
\definecolor{darkpurple}{RGB}{100, 60, 150}

%========================================================================================
% 박스 환경 (tcolorbox) - 6가지 타입
%========================================================================================

\usepackage[most]{tcolorbox}
\tcbuselibrary{skins, breakable}

% 1. 개요 박스 (강의 시작 부분)
\newtcolorbox{overviewbox}[1][]{
    enhanced,
    colback=lightpurple,
    colframe=darkpurple,
    fonttitle=\bfseries\large,
    title=📚 강의 개요,
    arc=3mm,
    boxrule=1pt,
    left=8pt,
    right=8pt,
    top=8pt,
    bottom=8pt,
    breakable,
    #1
}

% 2. 요약 박스
\newtcolorbox{summarybox}[1][]{
    enhanced,
    colback=lightblue,
    colframe=darkblue,
    fonttitle=\bfseries,
    title=📝 핵심 요약,
    arc=2mm,
    boxrule=0.7pt,
    left=6pt,
    right=6pt,
    top=6pt,
    bottom=6pt,
    breakable,
    #1
}

% 3. 핵심 정보 박스
\newtcolorbox{infobox}[1][]{
    enhanced,
    colback=lightgreen,
    colframe=darkgreen,
    fonttitle=\bfseries,
    title=💡 핵심 정보,
    arc=2mm,
    boxrule=0.7pt,
    left=6pt,
    right=6pt,
    top=6pt,
    bottom=6pt,
    breakable,
    #1
}

% 4. 주의사항 박스
\newtcolorbox{warningbox}[1][]{
    enhanced,
    colback=lightyellow,
    colframe=darkorange,
    fonttitle=\bfseries,
    title=⚠️ 주의사항,
    arc=2mm,
    boxrule=0.7pt,
    left=6pt,
    right=6pt,
    top=6pt,
    bottom=6pt,
    breakable,
    #1
}

% 5. 예제 박스
\newtcolorbox{examplebox}[1][]{
    enhanced,
    colback=lightgray,
    colframe=black!60,
    fonttitle=\bfseries,
    title=📖 예제: #1,
    arc=2mm,
    boxrule=0.7pt,
    left=6pt,
    right=6pt,
    top=6pt,
    bottom=6pt,
    breakable,
}

% 6. 정의 박스
\newtcolorbox{definitionbox}[1][]{
    enhanced,
    colback=lightpink,
    colframe=purple!70!black,
    fonttitle=\bfseries,
    title=📌 정의: #1,
    arc=2mm,
    boxrule=0.7pt,
    left=6pt,
    right=6pt,
    top=6pt,
    bottom=6pt,
    breakable,
}

% 7. 중요 박스 (importantbox - warningbox와 유사)
\newtcolorbox{importantbox}[1][]{
    enhanced,
    colback=boxred,
    colframe=red!70!black,
    fonttitle=\bfseries,
    title=⚠️ 매우 중요: #1,
    arc=2mm,
    boxrule=0.7pt,
    left=6pt,
    right=6pt,
    top=6pt,
    bottom=6pt,
    breakable,
}

% 8. cautionbox (warningbox와 동일)
\let\cautionbox\warningbox
\let\endcautionbox\endwarningbox

%========================================================================================
% 코드 블록 설정 (밝은 배경)
%========================================================================================

\usepackage{listings}

\definecolor{codegray}{rgb}{0.5,0.5,0.5}
\definecolor{codepurple}{rgb}{0.58,0,0.82}
\definecolor{backcolour}{rgb}{0.95,0.95,0.95}

\lstset{
    basicstyle=\ttfamily\small,
    backgroundcolor=\color{lightgray},
    keywordstyle=\color{darkblue}\bfseries,
    commentstyle=\color{darkgreen}\itshape,
    stringstyle=\color{purple!80!black},
    numberstyle=\tiny\color{black!60},
    numbers=left,
    numbersep=8pt,
    breaklines=true,
    breakatwhitespace=false,
    frame=single,
    frameround=tttt,
    rulecolor=\color{black!30},
    captionpos=b,
    showstringspaces=false,
    tabsize=2,
    xleftmargin=15pt,
    xrightmargin=5pt,
    escapeinside={\%*}{*)}
}

% Python 코드 스타일
\lstdefinestyle{pythonstyle}{
    language=Python,
    morekeywords={self, True, False, None},
}

% SQL 코드 스타일
\lstdefinestyle{sqlstyle}{
    language=SQL,
    morekeywords={SELECT, FROM, WHERE, JOIN, GROUP, BY, ORDER, HAVING},
}

%========================================================================================
% 목차 스타일링
%========================================================================================

\usepackage{tocloft}
\renewcommand{\cftsecleader}{\cftdotfill{\cftdotsep}}
\setlength{\cftbeforesecskip}{0.4em}
\renewcommand{\cftsecfont}{\bfseries}
\renewcommand{\cftsubsecfont}{\normalfont}

%========================================================================================
% 표 및 그림
%========================================================================================

\usepackage{graphicx}              % 이미지
\usepackage{adjustbox}             % 표/박스 크기 조절

% 표 캡션 스타일
\usepackage{caption}
\captionsetup[table]{
    labelfont=bf,
    textfont=it,
    skip=5pt
}
\captionsetup[figure]{
    labelfont=bf,
    textfont=it,
    skip=5pt
}

%========================================================================================
% 수학
%========================================================================================

\usepackage{amsmath, amssymb, amsthm}

% 정리 환경
\theoremstyle{definition}
\newtheorem{theorem}{정리}[section]
\newtheorem{lemma}[theorem]{보조정리}
\newtheorem{proposition}[theorem]{명제}
\newtheorem{corollary}[theorem]{따름정리}
\newtheorem{definition}{정의}[section]
\newtheorem{example}{예제}[section]

%========================================================================================
% 하이퍼링크
%========================================================================================

\usepackage[
    colorlinks=true,
    linkcolor=blue!80!black,
    urlcolor=blue!80!black,
    citecolor=green!60!black,
    bookmarks=true,
    bookmarksnumbered=true,
    pdfborder={0 0 0}
]{hyperref}

% PDF 메타데이터는 각 문서에서 설정
\hypersetup{
    pdftitle={CS109A: 데이터 과학 입문 - Lecture 14},
    pdfauthor={강의 노트},
    pdfsubject={Academic Notes}
}

%========================================================================================
% 기타 유용한 패키지
%========================================================================================

\usepackage{enumitem}              % 리스트 커스터마이징
\setlist{nosep, leftmargin=*, itemsep=0.3em}

\usepackage{microtype}             % 타이포그래피 개선
\usepackage{footnote}              % 각주 개선
\usepackage{url}                   % URL 줄바꿈
\urlstyle{same}

%========================================================================================
% 사용자 정의 명령어
%========================================================================================

% 강조 텍스트
\newcommand{\important}[1]{\textbf{\textcolor{red!70!black}{#1}}}
\newcommand{\keyword}[1]{\textbf{#1}}
\newcommand{\term}[1]{\textit{#1}}
\newcommand{\code}[1]{\texttt{#1}}

% 용어 설명 (인라인)
\newcommand{\defterm}[2]{\textbf{#1}\footnote{#2}}

% 섹션 시작 전 페이지 분리
\newcommand{\newsection}[1]{\newpage\section{#1}}

%========================================================================================
% 문서 제목 스타일
%========================================================================================

\usepackage{titling}
\pretitle{\begin{center}\LARGE\bfseries}
\posttitle{\par\end{center}\vskip 0.5em}
\preauthor{\begin{center}\large}
\postauthor{\end{center}}
\predate{\begin{center}\large}
\postdate{\par\end{center}}

%========================================================================================
% 섹션 제목 간격
%========================================================================================

\usepackage{titlesec}
\titlespacing*{\section}{0pt}{1.5em}{0.8em}
\titlespacing*{\subsection}{0pt}{1.2em}{0.6em}
\titlespacing*{\subsubsection}{0pt}{1em}{0.5em}

%========================================================================================
% 메타 정보 박스 명령어
%========================================================================================

\newcommand{\metainfo}[4]{
\begin{tcolorbox}[
    colback=lightpurple,
    colframe=darkpurple,
    boxrule=1pt,
    arc=2mm,
    left=10pt,
    right=10pt,
    top=8pt,
    bottom=8pt
]
\begin{tabular}{@{}rl@{}}
▣ \textbf{강의명:} & #1 \\[0.3em]
▣ \textbf{주차:} & #2 \\[0.3em]
▣ \textbf{교수명:} & #3 \\[0.3em]
▣ \textbf{목적:} & \begin{minipage}[t]{0.75\textwidth}#4\end{minipage}
\end{tabular}
\end{tcolorbox}
}

%========================================================================================
% 끝
%========================================================================================


\begin{document}

\maketitle
\thispagestyle{firstpage}

\metainfo{CS109A: 데이터 과학 입문}{Lecture 14}{Pavlos Protopapas, Kevin Rader, Chris Gumb}{Lecture 14의 핵심 개념 학습}


\thispagestyle{fancy} % 첫 페이지에도 fancy 스타일 적용

\begin{summarybox}
본 문서는 로지스틱 회귀의 심화 주제를 다룹니다.
단순 모델을 넘어 다중 로지스틱 회귀, 상호작용 항의 해석, 정규화(Ridge)를 통한 과적합 방지 방법을 배웁니다.

또한, 로지스틱 회귀를 분류(Classification) 문제에 활용하는 방법,
즉 결정 경계(Decision Boundary)의 개념과 다중 클래스(K>2) 분류를 위한 OvR, 다항 로지스틱 회귀(Softmax)를 학습합니다.

마지막으로, 분류 모델의 성능을 평가하는 핵심 지표인 혼동 행렬(Confusion Matrix),
민감도, 특이도, ROC 커브, AUC의 개념을 상세히 설명합니다.
\end{summarybox}

\tableofcontents

\newpage

%================================================================================
\section{로지스틱 회귀 복습 (Review)}
%================================================================================

\subsection{왜 로지스틱 회귀인가?}

선형 회귀(Linear Regression)는 예측 값이 연속적인 숫자(예: 집값, 온도)일 때 사용합니다.
하지만 우리가 예측하려는 대상($Y$)이 '성공/실패', '합격/불합격', '생존/사망'처럼 두 가지 범주 중 하나라면(Binary) 선형 회귀는 적합하지 않습니다.

로지스틱 회귀(Logistic Regression)는 $Y$가 범주형일 때,
특히 이진(binary) 분류 문제에서 사용됩니다.

\begin{warningbox}
  \textbf{오해 피하기: 선형 회귀를 이진 분류에 쓰면 안 되는 이유}

  선형 회귀 모델($Y = \beta_0 + \beta_1 X$)을 그대로 사용하면 두 가지 큰 문제가 발생합니다.
  \begin{itemize}
    \item \textbf{범위 초과:} $Y$는 0 또는 1이어야 하지만, 선형 회귀의 예측 값은 1을 넘거나 0보다 작아질 수 있습니다. 이는 '확률'로 해석할 수 없게 만듭니다.
    \item \textbf{관계 왜곡:} 0과 1 사이의 관계가 직선적(linear)이라고 가정하지만, 실제로는 특정 지점에서 급격히 변하는 S자 형태(비선형)일 가능성이 높습니다.
  \end{itemize}
\end{warningbox}

\subsection{핵심 아이디어: 확률을 직접 모델링하지 않는다}

로지스틱 회귀는 $P(Y=1|X)$ (성공 확률)을 직접 모델링하는 대신,
'확률'을 변형한 **'로그-오즈(Log-Odds)'**를 선형 회귀 형태로 모델링합니다.

\begin{enumerate}
  \item \textbf{확률 (Probability, $P$):} 0과 1 사이의 값. (예: $P = 0.8$)
  \item \textbf{오즈 (Odds):} 성공 확률 / 실패 확률. (예: $Odds = 0.8 / (1-0.8) = 4$). 0부터 무한대($\infty$)까지의 값을 가집니다. "실패보다 성공할 확률이 4배 높다"는 의미입니다.
      $$ \text{Odds} = \frac{P(Y=1)}{1 - P(Y=1)} = \frac{P}{1-P} $$
  \item \textbf{로그-오즈 (Log-Odds) 또는 로짓(Logit):} 오즈에 자연로그($\ln$)를 취한 값. (예: $\ln(4) \approx 1.386$). 음의 무한대($-\infty$)부터 양의 무한대($+\infty$)까지 모든 값을 가질 수 있습니다.
      $$ \text{Logit}(P) = \ln(\text{Odds}) = \ln\left(\frac{P}{1-P}\right) $$
\end{enumerate}

로그-오즈는 선형 회귀의 예측 값처럼 범위에 제한이 없으므로,
이를 $X$에 대한 선형 결합으로 모델링할 수 있습니다.

$$ \underbrace{\ln\left(\frac{P}{1-P}\right)}_{\text{Log-Odds}} = \beta_0 + \beta_1 X_1 + \dots + \beta_p X_p $$

\subsection{추정: 최대가능도 추정법 (MLE)}

선형 회귀는 $\beta$를 찾기 위해 오차제곱합(SSE)을 최소화했습니다 (최소제곱법, OLS).

로지스틱 회귀는 **최대가능도 추정법 (Maximum Likelihood Estimation, MLE)**을 사용합니다.
직관적으로, "우리가 가진 데이터($Y$ 값들)가 관찰될 확률을 가장 높게 만드는 $\beta$ 값을 찾자"는 의미입니다.

이는 수학적으로 **'음의 로그-가능도(Negative Log-Likelihood)'**를 최소화하는 것과 같으며,
이 손실 함수(Loss Function)를 **'이진 교차 엔트로피 (Binary Cross-Entropy)'**라고 부릅니다.

$$ \text{Loss (Binary Cross-Entropy)} = - \sum_{i=1}^{n} \left[ y_i \ln(p_i) + (1-y_i) \ln(1-p_i) \right] $$

* $y_i$: 실제 값 (0 또는 1)
* $p_i$: 모델이 예측한 $P(Y_i=1)$ 확률

선형 회귀와 달리 한 번에 풀리는 해(Closed-form solution)가 없으며,
**경사 하강법(Gradient Descent)**과 같은 수치 최적화(Numerical Optimization) 기법을 사용해 $\beta$ 값을 반복적으로 찾아나갑니다.

\newpage

%================================================================================
\section{로지스틱 회귀의 추론 (Inference)}
%================================================================================

추론(Inference)의 목적은 모델의 계수($\beta$)가 통계적으로 유의미한지,
그리고 그 값의 신뢰구간(Confidence Interval)이 어느 정도인지 파악하는 것입니다.

\subsection{선형 회귀(t-분포) vs 로지스틱 회귀(Z-분포)}

\begin{itemize}
  \item \textbf{선형 회귀:} 오차항의 분산($\sigma^2$)을 별도로 '추정'해야 합니다. 이 불확실성 때문에 $\beta$의 분포는 $t$-분포를 따릅니다.
  \item \textbf{로지스틱 회귀:} $Y$가 베르누이 분포($Y \sim \text{Bernoulli}(P)$)를 따릅니다.
  베르누이 분포의 분산은 $P(1-P)$로, 평균($P$)이 정해지면 분산이 '공짜로' 결정됩니다.
  별도로 추정할 분산이 없으므로, (샘플이 충분히 크다면) $\beta$는 정규분포($Z$-분포)를 따른다고 가정합니다.
\end{itemize}

\begin{examplebox}
  \textbf{Z-분포 vs t-분포의 실제적 차이}

  95\% 신뢰구간을 계산할 때,
  \begin{itemize}
    \item \textbf{Z-분포 (로지스틱):} $\hat{\beta} \pm \mathbf{1.96} \times (\text{표준오차})$
    \item \textbf{t-분포 (선형):} $\hat{\beta} \pm \mathbf{t_{\alpha/2, df}} \times (\text{표준오차})$ (보통 2에 가까운 값)
  \end{itemize}
  실제 계산에서는 큰 차이가 없으나, 통계적 근거가 다릅니다.
  `statsmodels` 라이브러리는 이러한 Z-통계량, p-value, 신뢰구간을 제공해줍니다.
\end{examplebox}

\subsection{계수(Coefficient) 해석하기}

$\beta$ 값 자체보다 $e^{\beta}$ (지수 변환) 값이 훨씬 직관적입니다.

\begin{itemize}
  \item $\boldsymbol{\beta_j}$: $X_j$가 1단위 증가할 때, \textbf{로그-오즈(Log-Odds)}가 $\beta_j$만큼 \textbf{증가(덧셈)}합니다.
  \item $\boldsymbol{e^{\beta_j}}$: $X_j$가 1단위 증가할 때, \textbf{오즈(Odds)}가 $e^{\beta_j}$만큼 \textbf{곱해집니다(배수)}.
\end{itemize}

이를 **오즈비 (Odds Ratio, OR)**라고 부릅니다.
* $e^{\beta_j} > 1$: $X_j$가 증가하면 성공 오즈가 증가합니다. (긍정적 관계)
* $e^{\beta_j} = 1$: $X_j$는 성공 오즈와 관계 없습니다. ($\beta_j=0$)
* $e^{\beta_j} < 1$: $X_j$가 증가하면 성공 오즈가 감소합니다. (부정적 관계)

\begin{examplebox}
  \textbf{예: 이진 예측변수 (Binary Predictor) 해석 (성별과 심장병)}

  심장병 발병($Y=1$) 여부를 성별로 예측하는 모델을 가정합니다.
  (기준 그룹: 남성)

  $$ \ln(\text{Odds}) = \beta_0 + \beta_1 \cdot \text{Female} \quad (\text{Female}=1, \text{Male}=0) $$

  여기서 $\beta_0 = 0.214$이고 $\beta_1 = -1.272$라고 가정합시다.

  \begin{enumerate}
    \item \textbf{$\beta_0$의 해석 (기준 그룹):}
        \begin{itemize}
          \item $\beta_0 = 0.214$는 \textbf{남성(기준 그룹)}의 \textbf{로그-오즈}입니다.
          \item $e^{\beta_0} = e^{0.214} \approx 1.24$는 \textbf{남성}의 \textbf{오즈}입니다.
          \item (심장병에 걸릴 오즈가 안 걸릴 오즈보다 1.24배 높다.)
        \end{itemize}

    \item \textbf{$\beta_1$의 해석 (차이):}
        \begin{itemize}
          \item $\beta_1 = -1.272$는 \textbf{여성}의 로그-오즈가 \textbf{남성}보다 1.272만큼 \textbf{낮다}는 의미입니다.
          \item $e^{\beta_1} = e^{-1.272} \approx 0.28$은 \textbf{오즈비(Odds Ratio)}입니다.
          \item \textbf{해석:} "다른 조건이 같다면, 여성의 심장병 발병 \textbf{오즈}는 남성의 \textbf{0.28배 (즉, 72\% 낮다)}."
        \end{itemize}
  \end{enumerate}
\end{examplebox}

\begin{warningbox}
  \textbf{오즈(Odds)와 확률(Probability)을 혼동하지 마세요!}

  계수($\beta$) 해석은 항상 \textbf{오즈(Odds)} 관점에서 이루어집니다.
  "여성의 심장병 발병 \textit{확률}이 72\% 낮다"라고 해석하면 \textbf{틀립니다}.
  확률로 변환하려면 $P = \text{Odds} / (1 + \text{Odds})$ 공식을 사용해야 하며,
  이는 기준이 되는 확률(baseline probability)에 따라 그 변화량이 달라지는 비선형(non-linear) 관계입니다.
\end{warningbox}

\newpage

%================================================================================
\section{다중 로지스틱 회귀와 상호작용}
%================================================================================

\subsection{다중 로지스틱 회귀 (Multiple Logistic Regression)}

선형 회귀와 마찬가지로, 여러 개의 예측 변수($X_1, \dots, X_p$)를 사용하여 모델을 구성할 수 있습니다.

$$ \ln\left(\frac{P}{1-P}\right) = \beta_0 + \beta_1 X_1 + \beta_2 X_2 + \dots + \beta_p X_p $$

\textbf{해석:} $\beta_j$ (또는 $e^{\beta_j}$)의 의미는, \textbf{"다른 모든 예측 변수($X_k$)를 통제(일정하게 유지)했을 때"} $X_j$가 1단위 변할 때의 로그-오즈 (또는 오즈비) 변화량입니다.

\textbf{주의점:} 선형 회귀와 동일한 문제들, 즉 \textbf{다중공선성(Multicollinearity)}과 \textbf{과적합(Overfitting)}이 여기서도 동일하게 발생합니다.

\subsection{상호작용 (Interactions)}

상호작용 항은 "한 변수($X_1$)가 결과($Y$)에 미치는 영향이 다른 변수($X_2$)의 값에 따라 달라질 때" 사용됩니다.

\begin{examplebox}
  \textbf{예: 상호작용 해석 (Age $\times$ Female)}

  심장병 모델에 나이(Age)와 성별(Female), 그리고 둘의 상호작용 항을 추가합니다.

  $$ \ln(\text{Odds}) = \beta_0 + \beta_1 \text{Age} + \beta_2 \text{Female} + \beta_3 (\text{Age} \times \text{Female}) $$

  이 모델은 성별에 따라 두 개의 다른 모델로 분리됩니다.

  \begin{enumerate}
    \item \textbf{남성 (Male, Female=0)의 모델:}
        Female=0을 대입하면 $\beta_2, \beta_3$ 항이 사라집니다.
        $$ \ln(\text{Odds})_{\text{Male}} = \beta_0 + \beta_1 \text{Age} $$
        (절편: $\beta_0$, 나이의 기울기: $\beta_1$)

    \item \textbf{여성 (Female, Female=1)의 모델:}
        Female=1을 대입합니다.
        $$ \ln(\text{Odds})_{\text{Female}} = \beta_0 + \beta_1 \text{Age} + \beta_2(1) + \beta_3 (\text{Age} \times 1) $$
        $$ \ln(\text{Odds})_{\text{Female}} = (\beta_0 + \beta_2) + (\beta_1 + \beta_3) \text{Age} $$
        (절편: $\beta_0 + \beta_2$, 나이의 기울기: $\beta_1 + \beta_3$)
  \end{enumerate}

  \textbf{계수 해석:}
  \begin{itemize}
    \item $\beta_1$: \textbf{남성(기준 그룹)}의 나이 1살 증가에 따른 로그-오즈 변화량.
    \item $\beta_3$: \textbf{여성}의 나이 1살 증가에 따른 로그-오즈 변화량이 \textbf{남성 대비} 얼마나 \textbf{다른지 (그 차이)}
  \end{itemize}
  만약 $\beta_3$가 0이라면 (상호작용이 없다면), 두 그룹의 나이 기울기는 $\beta_1$로 동일할 것입니다.
\end{examplebox}

\newpage

%================================================================================
\section{분류와 결정 경계 (Classification \& Decision Boundary)}
%================================================================================

\subsection{확률에서 분류로: 임계값 (Threshold)}

로지스틱 회귀는 $P(Y=1|X)$ 확률을 예측합니다.
이를 0 또는 1의 분류로 바꾸려면 **'임계값(Threshold)'** (보통 0.5)을 정해야 합니다.

\begin{itemize}
  \item $P(Y=1) \ge 0.5$ 이면, $\hat{Y} = 1$ (성공)으로 분류한다.
  \item $P(Y=1) < 0.5$ 이면, $\hat{Y} = 0$ (실패)로 분류한다.
\end{itemize}

\subsection{결정 경계 (Decision Boundary)}

결정 경계는 모델의 예측이 $\hat{Y}=0$에서 $\hat{Y}=1$로 바뀌는 지점, 즉 $P(Y=1) = 0.5$가 되는 지점의 선 또는 면을 의미합니다.

$P=0.5$라는 것은 어떤 의미일까요?
\begin{itemize}
  \item $P = 0.5$
  \item $\text{Odds} = P / (1-P) = 0.5 / 0.5 = 1$
  \item $\ln(\text{Odds}) = \ln(1) = 0$
\end{itemize}

즉, 결정 경계는 **로그-오즈가 0이 되는 지점**입니다.
$$ \ln(\text{Odds}) = \beta_0 + \beta_1 X_1 + \dots + \beta_p X_p = 0 $$

\begin{warningbox}
  \textbf{오해 피하기: 결정 경계는 항상 선형인가?}

  결정 경계가 선형(직선, 평면)일 수도 있고, 비선형(곡선, 곡면)일 수도 있습니다.
  이는 \textbf{모델에 어떤 항을 포함했는지}에 달려있습니다.

  \begin{itemize}
    \item \textbf{선형 경계:}
        모델이 $\beta_0 + \beta_1 X_1 + \beta_2 X_2 = 0$ 처럼 $X$의 1차항만 포함하면,
        결정 경계는 $X_1$과 $X_2$ 공간에서 \textbf{직선}이 됩니다.

    \item \textbf{비선형 경계:}
        모델이 $X_1^2, X_2^2, X_1 X_2$ 같은 \textbf{다항식(Polynomial) 항이나 상호작용 항}을 포함하면,
        (예: $\beta_0 + \beta_1 X_1 + \beta_2 X_2 + \beta_3 X_1^2 + \beta_4 X_2^2 = 0$)
        결정 경계는 $X_1$과 $X_2$ 공간에서 \textbf{곡선 (원, 타원 등)}이 됩니다.
  \end{itemize}
  비선형 항을 추가함으로써 로지스틱 회귀는 복잡한 데이터 패턴도 분류할 수 있게 됩니다.
\end{warningbox}

% \begin{figure}[h]
%   \centering
%   % \includegraphics[width=0.45\textwidth]{linear_boundary.png}
%   % \includegraphics[width=0.45\textwidth]{nonlinear_boundary.png}
%   \caption{결정 경계의 예시: (왼쪽) 선형 항만 사용한 선형 경계, (오른쪽) 다항식 항을 사용한 비선형 경계}
%   \label{fig:boundary}
% \end{figure}

\textit{[이미지 삽입: 왼쪽은 직선으로 두 클래스를 나누는 결정 경계, 오른쪽은 곡선(원형)으로 두 클래스를 나누는 결정 경계를 보여줌]}

\newpage

%================================================================================
\section{정규화 (Regularization)}
%================================================================================

모델에 다항식 항이나 상호작용 항을 많이 추가하면 결정 경계가 매우 복잡해지면서 훈련 데이터에만 꼭 맞는 \textbf{과적합(Overfitting)}이 발생할 수 있습니다.

\textbf{정규화(Regularization)}는 모델의 복잡도에 페널티를 부과하여 과적합을 방지하는 기법입니다.
계수($\beta$)의 크기가 너무 커지지 않도록 손실 함수(Loss Function)에 **페널티 항(Penalty Term)**을 추가합니다.

\subsection{손실 함수 + L2 (Ridge) 페널티}

로지스틱 회귀의 손실 함수(Binary Cross-Entropy)에 L2 페널티(계수 제곱의 합, Ridge)를 더합니다.

$$ \text{Loss}_{\text{Regularized}} = \underbrace{\text{Loss (Binary Cross-Entropy)}}_{\text{모델이 데이터에 얼마나 잘 맞는지}} + \underbrace{\lambda \sum_{j=1}^{p} \beta_j^2}_{\text{모델이 얼마나 복잡한지 (페널티)}} $$

\begin{itemize}
  \item $\lambda$ (람다): 정규화의 강도를 조절하는 하이퍼파라미터입니다.
    \begin{itemize}
      \item $\lambda=0$: 페널티 없음 (표준 로지스틱 회귀).
      \item $\lambda \to \infty$: 페널티가 매우 강해져 모든 $\beta$가 0에 가까워집니다 (모델이 매우 단순해짐).
    \end{itemize}
  \item 페널티는 보통 절편($\beta_0$)을 제외하고 적용됩니다.
\end{itemize}

\begin{warningbox}
  \textbf{sklearn의 `C` 파라미터 이해하기}

  `sklearn.linear_model.LogisticRegression`에서는 $\lambda$ 대신 $C$라는 파라미터를 사용합니다.
  $C$는 $\lambda$의 역수 ($C = 1/\lambda$) 개념입니다.

  \begin{itemize}
    \item \textbf{높은 `C` (예: $C=100$) $\implies$ 낮은 $\lambda$:}
        정규화(페널티)가 \textbf{약합니다}. 모델이 복잡해지고 훈련 데이터에 더 강하게 맞춰집니다 (과적합 위험).

    \item \textbf{낮은 `C` (예: $C=0.01$) $\implies$ 높은 $\lambda$:}
        정규화(페널티)가 \textbf{강합니다}. 모델이 단순해지고 $\beta$ 계수들이 0에 가까워집니다 (과소적합 위험).
  \end{itemize}

  최적의 $C$ 값은 \textbf{교차 검증(Cross-Validation)}을 통해 찾아야 합니다.
\end{warningbox}

\begin{lstlisting}[language=Python, caption={sklearn에서 Ridge 정규화를 적용한 로지스틱 회귀}, label={lst:sklearn_c}]
from sklearn.linear_model import LogisticRegression
from sklearn.model_selection import GridSearchCV

# C 값이 낮을수록 정규화가 강해짐
logreg = LogisticRegression(penalty='l2', C=0.1)

# 교차 검증으로 최적의 C 값을 찾을 수도 있음
params = {'C': [0.01, 0.1, 1, 10, 100]}
grid_search = GridSearchCV(LogisticRegression(penalty='l2'), params, cv=5)
# grid_search.fit(X_train, y_train)
# print(grid_search.best_params_)
\end{lstlisting}

\newpage

%================================================================================
\section{다중 클래스 로지스틱 회귀 (Multiclass)}
%================================================================================

$Y$의 범주가 3개 이상일 때 (예: 'CS', 'Stat', 'Other') 사용하는 방법입니다.
여기서는 순서가 없는 **명목형(Nominal)** 범주를 가정합니다.

\subsection{접근법 1: One-vs-Rest (OvR)}

가장 간단하고 직관적인 방법입니다. 범주가 $K$개일 때, $K$개의 독립적인 이진 분류기를 만듭니다.

\begin{itemize}
  \item \textbf{분류기 1:} 'CS' vs 'Not CS' (즉, 'Stat' + 'Other')
  \item \textbf{분류기 2:} 'Stat' vs 'Not Stat' (즉, 'CS' + 'Other')
  \item \textbf{분류기 3:} 'Other' vs 'Not Other' (즉, 'CS' + 'Stat')
\end{itemize}

새로운 데이터가 들어오면, 3개의 분류기를 모두 돌려서 각각의 확률(또는 점수)을 계산한 뒤, 가장 높은 확률(점수)을 보인 클래스로 예측합니다. (`sklearn`의 `multi_class='ovr'`)

\subsection{접근법 2: 다항 로지스틱 회귀 (Multinomial)}

OvR과 달리, $K$개의 클래스를 한 번에 처리하는 단일 모델을 만듭니다.
하나의 클래스(예: 'Other')를 **기준(Reference) 클래스**로 정합니다.
그리고 $K-1$개의 로그-오즈 모델을 만듭니다.

\begin{itemize}
  \item \textbf{모델 1:} $\ln\left(\frac{P(\text{CS})}{P(\text{Other})}\right) = \beta_0^{(1)} + \beta_1^{(1)} X + \dots$
  \item \textbf{모델 2:} $\ln\left(\frac{P(\text{Stat})}{P(\text{Other})}\right) = \beta_0^{(2)} + \beta_1^{(2)} X + \dots$
\end{itemize}
(`sklearn`의 `multi_class='multinomial'`)

\subsection{Softmax: 점수를 확률로 변환하기}

OvR이든 다항 로지스틱이든, 각 클래스 $k$에 대한 '점수(Score)' 또는 '로짓(Logit)' $s_k$가 나옵니다.
이 점수들은 합쳐도 1이 되지 않기 때문에 확률로 사용하기 어렵습니다.

**소프트맥스(Softmax)** 함수는 이 점수($s_k$)들을 0과 1 사이의 값으로 변환하고,
모든 클래스의 확률 총합이 1이 되도록 정규화해줍니다.

$$ P(Y=k|X) = \frac{e^{s_k}}{\sum_{j=1}^{K} e^{s_j}} $$

\subsection{다중 클래스 분류 (K>2)}

임계값 0.5는 더 이상 의미가 없습니다.
대신 **'다수결 원칙(Plurality Wins)'**을 사용합니다.
Softmax를 통해 계산된 $K$개의 확률 중, **가장 높은 확률을 가진 클래스**로 예측합니다.

예: $P(\text{CS})=0.2$, $P(\text{Stat})=0.4$, $P(\text{Other})=0.4$
$\implies$ 가장 높은 확률이 0.4로 두 개이므로, 둘 중 하나를 선택 (혹은 추가 규칙 적용).
만약 $P(\text{Stat})=0.41$, $P(\text{Other})=0.39$ 였다면 $\hat{Y} = \text{Stat}$로 예측.

\newpage

%================================================================================
\section{분류 모델 평가 (Evaluation)}
%================================================================================

모델을 만들었다면, 이 모델이 얼마나 '좋은' 분류기인지 평가해야 합니다.
단순 정확도(Accuracy)는 특히 데이터가 불균형할 때(예: 99\%가 'No', 1\%가 'Yes') 성능을 오해하게 만듭니다.

\subsection{혼동 행렬 (Confusion Matrix)}

분류 결과(예측)와 실제 값을 $2 \times 2$ 표로 정리한 것입니다.

\begin{center}
\begin{adjustbox}{max width=\textwidth}
\begin{tabular}{cc|cc}
\multicolumn{2}{c}{} & \multicolumn{2}{c}{\textbf{모델의 예측 (Predicted)}} \\
\multicolumn{2}{c}{} & \textbf{Negative (0)} & \textbf{Positive (1)} \\ \hline
\textbf{실제 값} & \textbf{Negative (0)} & \textbf{True Negative (TN)} & \textbf{False Positive (FP)} \\
\textbf{(Actual)} & \textbf{Positive (1)} & \textbf{False Negative (FN)} & \textbf{True Positive (TP)} \\
\end{tabular}
\end{adjustbox}
\end{center}

\begin{itemize}
  \item \textbf{True Positive (TP):} \textbf{정답.} 실제 Positive, 예측 Positive (예: 암환자를 암으로 진단)
  \item \textbf{True Negative (TN):} \textbf{정답.} 실제 Negative, 예측 Negative (예: 건강한 사람을 건강하다고 진단)
  \item \textbf{False Positive (FP):} \textbf{1종 오류.} 실제 Negative, 예측 Positive (예: 건강한 사람을 암으로 오진)
  \item \textbf{False Negative (FN):} \textbf{2종 오류.} 실제 Positive, 예측 Negative (예: 암환자를 건강하다고 오진) $\leftarrow$ \textit{치명적 오류!}
\end{itemize}

\subsection{핵심 평가지표}

\begin{enumerate}
  \item \textbf{민감도 (Sensitivity) = 재현율 (Recall) = True Positive Rate (TPR)}
      $$ \text{Sensitivity} = \frac{TP}{TP + FN} \quad (\text{실제 Positive 중 맞춘 비율}) $$
      \textbf{의미:} 실제 암환자 중 몇 \%를 '암'이라고 잡아냈는가? (FN을 줄이는 데 초점)
      의료 진단에서 매우 중요합니다. (놓치면 안 됨)

  \item \textbf{특이도 (Specificity) = True Negative Rate (TNR)}
      $$ \text{Specificity} = \frac{TN}{TN + FP} \quad (\text{실제 Negative 중 맞춘 비율}) $$
      \textbf{의미:} 실제 건강한 사람 중 몇 \%를 '건강'이라고 판단했는가? (FP를 줄이는 데 초점)
      FP의 비용이 클 때 (예: 스팸 필터가 중요한 메일을 스팸 처리) 중요합니다.

  \item \textbf{정밀도 (Precision) = Positive Predictive Value (PPV)}
      $$ \text{Precision} = \frac{TP}{TP + FP} \quad (\text{예측 Positive 중 맞춘 비율}) $$
      \textbf{의미:} 모델이 '암'이라고 예측한 사람들 중, 실제 암환자는 몇 \%인가?

  \item \textbf{False Positive Rate (FPR)}
      $$ \text{FPR} = 1 - \text{Specificity} = \frac{FP}{TN + FP} \quad (\text{실제 Negative 중 틀린 비율}) $$
      \textbf{의미:} 건강한 사람 중 몇 \%를 '암'이라고 잘못 예측했는가?
\end{enumerate}

\begin{warningbox}
  \textbf{베이즈 정리와 낮은 유병률(Prevalence) 문제}

  베이즈 정리에 따르면, 아무리 테스트기(모델)의 민감도(99\%)와 특이도(99\%)가 높아도,
  질병 자체가 매우 희귀하다면(예: 유병률 0.1\%),
  테스트 결과가 '양성(Positive)'이 나왔더라도 실제 환자일 확률(PPV/정밀도)은 매우 낮을 수 있습니다.
  대부분이 False Positive이기 때문입니다.
\end{warningbox}

\newpage

\subsection{임계값(Threshold)의 트레이드오프}

분류 임계값 0.5는 절대적인 기준이 아닙니다.
임계값을 조절하면 민감도(TPR)와 특이도(1-FPR)가 반비례 관계(Trade-off)로 움직입니다.

\begin{itemize}
  \item \textbf{임계값을 낮추면 (예: 0.5 $\to$ 0.3):}
      모델이 'Positive'라고 더 쉽게 예측합니다.
      TP $\uparrow$ (좋음), FN $\downarrow$ (좋음) $\implies$ \textbf{민감도(TPR) 상승}
      FP $\uparrow$ (나쁨), TN $\downarrow$ (나쁨) $\implies$ \textbf{FPR 상승 (특이도 하락)}
      (예: "일단 암일 가능성이 조금만 있어도 양성으로 판정" $\to$ FN은 줄지만 FP가 늘어남)

  \item \textbf{임계값을 높이면 (예: 0.5 $\to$ 0.7):}
      모델이 'Positive'라고 더 보수적으로 예측합니다.
      TP $\downarrow$ (나쁨), FN $\uparrow$ (나쁨) $\implies$ \textbf{민감도(TPR) 하락}
      FP $\downarrow$ (좋음), TN $\uparrow$ (좋음) $\implies$ \textbf{FPR 하락 (특이도 상승)}
      (예: "확실히 암일 때만 양성으로 판정" $\to$ FP는 줄지만 FN이 늘어남)
\end{itemize}

\subsection{ROC 커브와 AUC}

\textbf{ROC 커브 (Receiver Operating Characteristic Curve)}는
이 트레이드오프를 시각화한 그래프입니다.

\begin{itemize}
  \item \textbf{X축:} False Positive Rate (FPR) (1 - 특이도)
  \item \textbf{Y축:} True Positive Rate (TPR) (민감도)
\end{itemize}

모든 가능한 임계값(0에서 1까지)에 대해 (FPR, TPR) 좌표를 찍어서 연결한 선입니다.

% \begin{figure}[h]
%   \centering
%   % \includegraphics[width=0.6\textwidth]{roc_curve.png}
%   \caption{ROC 커브의 예시}
%   \label{fig:roc}
% \end{figure}

\textit{[이미지 삽입: ROC 커브 그래프. (0,0)에서 (1,1)을 잇는 점선(Random Classifier),
(0,0)에서 (0,1)을 거쳐 (1,1)로 가는 실선(Perfect Classifier),
그리고 그 사이를 지나는 실제 모델의 ROC 커브를 보여줌]}

\begin{itemize}
  \item \textbf{완벽한 모델 (Perfect Classifier):} (0, 1) 지점을 통과합니다 (FPR=0, TPR=1).
  \item \textbf{랜덤 모델 (Random Classifier):} $y=x$ 대각선. (FPR과 TPR이 같음)
  \item \textbf{좋은 모델:} 커브가 왼쪽 위 (0, 1)에 최대한 가까이 붙습니다.
\end{itemize}

\textbf{AUC (Area Under the Curve)}는 이 ROC 커브 아래의 면적입니다.
0부터 1 사이의 값을 가지며, 모델의 전체적인 성능을 하나의 숫자로 요약해줍니다.

\begin{itemize}
  \item \textbf{AUC = 1.0:} 완벽한 분류기
  \item \textbf{AUC = 0.5:} 쓸모없는 분류기 (랜덤 추측)
  \item \textbf{AUC $\approx$ 0.8\textasciitilde{}0.9:} 매우 좋은 분류기
\end{itemize}
AUC는 임계값에 상관없이 모델이 'Positive' 샘플을 'Negative' 샘플보다 얼마나 더 높은 확률로 예측하는지(순서)를 나타내는 지표입니다.

\newpage

%================================================================================
\section{핵심 용어 정리}
%================================================================================

\begin{table}[h]
\caption{로지스틱 회귀 및 분류 평가 핵심 용어}
\label{tab:glossary}
\begin{adjustbox}{max width=\textwidth}
\begin{tabular}{@{}lll@{}}
\toprule
\textbf{용어} & \textbf{원어} & \textbf{쉬운 설명} \\ \midrule
오즈 & Odds & 성공 확률 / 실패 확률. ($P/(1-P)$) \\
로그-오즈 & Log-Odds & 오즈에 자연로그를 취한 값. $\ln(P/(1-P)$). 로지스틱 회귀의 $Y$값. \\
오즈비 & Odds Ratio (OR) & $X$가 1단위 증가할 때, 오즈가 몇 '배' 변하는지. ($e^\beta$) \\
MLE & Max Likelihood Estimation & 데이터가 관찰될 확률을 최대화하는 $\beta$를 찾는 추정 방식. \\
이진 교차 엔트로피 & Binary Cross-Entropy & 로지스틱 회귀의 손실 함수(Loss Function). 음의 로그-가능도. \\
결정 경계 & Decision Boundary & 예측 클래스가 0에서 1로 바뀌는 경계선. $P=0.5$ (즉, Log-Odds=0)인 지점. \\
정규화 & Regularization & 모델 복잡도에 페널티를 주어 과적합을 막는 기법. (예: L2/Ridge) \\
C 파라미터 & C (in sklearn) & $1/\lambda$. 정규화 강도의 역수. (C가 낮을수록 정규화가 강함) \\
OvR & One-vs-Rest & K개 클래스 분류 시, K개의 이진 분류기('A' vs 'Not A')를 만듦. \\
다항 회귀 & Multinomial Regression & K개 클래스 분류 시, 1개의 기준 클래스 대비 K-1개 모델을 만듦. \\
소프트맥스 & Softmax & K개의 클래스 점수(Logit)를 총합 1인 확률로 변환하는 함수. \\
혼동 행렬 & Confusion Matrix & 모델의 예측(TP, FP, FN, TN)과 실제 값을 비교한 표. \\
민감도 (재현율) & Sensitivity (Recall) & 실제 '성공' 중 모델이 '성공'으로 맞춘 비율. ($TP / (TP+FN)$) \\
특이도 & Specificity & 실제 '실패' 중 모델이 '실패'로 맞춘 비율. ($TN / (TN+FP)$) \\
정밀도 & Precision & 모델이 '성공' 예측 중 실제 '성공'인 비율. ($TP / (TP+FP)$) \\
ROC 커브 & ROC Curve & 모든 임계값에 대해 (FPR, TPR)을 그린 그래프. \\
AUC & Area Under the Curve & ROC 커브 아래 면적. 1에 가까울수록 좋은 모델. \\ \bottomrule
\end{tabular}
\end{adjustbox}
\end{table}

\newpage

%================================================================================
\section{학습 체크리스트}
%================================================================================

\begin{tcolorbox}{title=최종 점검 체크리스트}
\begin{itemize}
    \item[$\square$] 왜 이진 분류 문제에 선형 회귀 대신 로지스틱 회귀를 써야 하는지 설명할 수 있는가?
    \item[$\square$] 확률(P), 오즈(Odds), 로그-오즈(Log-Odds)의 관계를 설명할 수 있는가?
    \item[$\square$] $\beta_1$ 계수와 $e^{\beta_1}$ (오즈비)의 해석상 차이를 설명할 수 있는가?
    \item[$\square$] 다중 로지스틱 회귀에서 $\beta_j$를 해석할 때 "다른 변수를 통제할 때"라는 조건이 왜 붙는지 아는가?
    \item[$\square$] 상호작용 항($X_1 \times X_2$)이 모델의 절편과 기울기에 각각 어떤 영향을 미치는지 설명할 수 있는가?
    \item[$\square$] 결정 경계(Decision Boundary)가 $P=0.5$ 지점, 즉 $X\beta=0$ 지점과 같다는 것을 수학적으로 유도할 수 있는가?
    \item[$\square$] 모델에 다항식 항을 추가하면 결정 경계가 어떻게 변하는지 아는가?
    \item[$\square$] 정규화(Regularization)가 필요한 이유(과적합 방지)를 설명할 수 있는가?
    \item[$\square$] sklearn의 `C` 파라미터가 낮을수록 정규화가 강해진다는 것을 아는가? (C $\approx 1/\lambda$)
    \item[$\square$] 다중 클래스 분류의 2가지 접근법 (OvR, Multinomial)을 비교할 수 있는가?
    \item[$\square$] Softmax 함수의 역할(점수 $\to$ 확률 정규화)을 아는가?
    \item[$\square$] 혼동 행렬의 TP, FP, FN, TN이 각각 무엇을 의미하는지 아는가?
    \item[$\square$] 민감도(재현율), 특이도, 정밀도의 차이를 (공식과 의미) 설명할 수 있는가?
    \item[$\square$] 분류 임계값(Threshold)을 조절하면 민감도와 특이도가 어떻게 변하는지(Trade-off) 아는가?
    \item[$\square$] ROC 커브의 X축(FPR)과 Y축(TPR)이 무엇이며, AUC가 왜 0.5면 '랜덤'인지 설명할 수 있는가?
\end{itemize}
\end{tcolorbox}

\newpage

%================================================================================
\section{1페이지 요약 (1-Page Summary)}
%================================================================================

\begin{tcolorbox}{title=1. 로지스틱 회귀 모델}
$Y$가 0 또는 1일 때, 성공 확률 $P$를 직접 모델링하지 않고, \textbf{로그-오즈}를 $X$에 대한 선형식으로 모델링합니다.
$$ \ln\left(\frac{P}{1-P}\right) = \beta_0 + \beta_1 X_1 + \dots + \beta_p X_p $$
$\beta$는 MLE (최대가능도 추정법) 또는 경사하강법으로 추정합니다.
\end{tcolorbox}

\begin{tcolorbox}{title=2. 계수 해석}
$\beta_j$는 $X_j$가 1단위 증가할 때 \textbf{로그-오즈}의 \textbf{덧셈} 변화량입니다.
$e^{\beta_j}$ (오즈비)는 $X_j$가 1단위 증가할 때 \textbf{오즈}의 \textbf{곱셈} 변화량 (배수)입니다.
\end{tcolorbox}

\begin{tcolorbox}{title=3. 결정 경계 (Decision Boundary)}
분류 임계값을 $P=0.5$로 두면, 결정 경계는 $\ln(\text{Odds})=0$이 되는 지점,
즉 $\beta_0 + \beta_1 X_1 + \dots + \beta_p X_p = 0$이 됩니다.
$X$의 1차항만 있으면 직선, 다항식/상호작용 항이 있으면 곡선이 됩니다.
\end{tcolorbox}

\begin{tcolorbox}{title=4. 정규화 (Regularization)}
과적합을 막기 위해 손실 함수(이진 교차 엔트로피)에 페널티 항을 추가합니다.
(L2/Ridge) $\text{Loss}_{\text{Reg}} = \text{Loss} + \lambda \sum \beta_j^2$.
`sklearn`에서는 $C \approx 1/\lambda$를 사용하며, $C$가 낮을수록 정규화가 강합니다.
최적의 $C$는 교차 검증(Cross-Validation)으로 찾습니다.
\end{tcolorbox}

\begin{tcolorbox}{title=5. 다중 클래스 (Multiclass) K>2}
\begin{itemize}
    \item \textbf{OvR (One-vs-Rest):} K개의 이진 분류기('A' vs 'Not A')를 만듦.
    \item \textbf{Multinomial:} 1개의 기준 클래스 대비 K-1개 모델을 만듦.
    \item \textbf{Softmax:} K개의 점수(Logit)를 총합 1인 확률로 변환.
    \item \textbf{분류:} 가장 높은 확률을 가진 클래스로 예측 (Plurality Wins).
\end{itemize}
\end{tcolorbox}

\begin{tcolorbox}{title=6. 분류 평가 (Evaluation)}
\begin{itemize}
    \item \textbf{혼동 행렬:} TP, FP, FN, TN
    \item \textbf{민감도(TPR):} $\frac{TP}{TP+FN}$ (실제 P 중 예측 P)
    \item \textbf{특이도(TNR):} $\frac{TN}{TN+FP}$ (실제 N 중 예측 N)
    \item \textbf{정밀도(PPV):} $\frac{TP}{TP+FP}$ (예측 P 중 실제 P)
    \item \textbf{ROC 커브:} X축=FPR (1-특이도), Y축=TPR (민감도). 모든 임계값에서의 성능 시각화.
    \item \textbf{AUC:} ROC 커브 아래 면적. 1에 가까울수록 좋은 분류기.
\end{itemize}
\end{tcolorbox}

\end{document}
