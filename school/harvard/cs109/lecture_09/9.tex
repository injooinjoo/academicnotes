%%%%%%%%%%%%%%%%%%%%%%%%%%%%%%%%%%%%%%%%%%%%%%%%%%%%%%%%%%%%%%%%%%%%%%%%%%%%%%%
% Harvard Academic Notes - 통합 마스터 템플릿
% 모든 강의 노트에 적용되는 통일된 스타일
% 버전: 2.0
% 최종 수정일: 2025-10-26
%%%%%%%%%%%%%%%%%%%%%%%%%%%%%%%%%%%%%%%%%%%%%%%%%%%%%%%%%%%%%%%%%%%%%%%%%%%%%%%

\documentclass[11pt,a4paper]{article}

%========================================================================================
% 기본 패키지
%========================================================================================

% --- 한국어 지원 ---
\usepackage{kotex}

% --- 페이지 레이아웃 ---
\usepackage[margin=25mm]{geometry}
\usepackage{setspace}
\onehalfspacing                      % 1.5배 줄간격
\setlength{\parskip}{0.6em}          % 문단 간격
\setlength{\parindent}{0pt}          % 들여쓰기 없음

% --- 표 관련 ---
\usepackage{booktabs}              % 고품질 표
\usepackage{tabularx}              % 자동 너비 조절 표
\usepackage{array}                 % 표 컬럼 확장
\usepackage{longtable}             % 여러 페이지 표
\renewcommand{\arraystretch}{1.2}  % 표 행간 조절

%========================================================================================
% 헤더 및 푸터
%========================================================================================

\usepackage{fancyhdr}
\pagestyle{fancy}
\fancyhf{}
\fancyhead[L]{\small\textit{CS109A: 데이터 과학 입문}}
\fancyhead[R]{\small\textit{Lecture 09}}
\fancyfoot[C]{\thepage}
\renewcommand{\headrulewidth}{0.5pt}
\renewcommand{\footrulewidth}{0.3pt}

% 첫 페이지는 헤더 없음
\fancypagestyle{firstpage}{
    \fancyhf{}
    \fancyfoot[C]{\thepage}
    \renewcommand{\headrulewidth}{0pt}
}

%========================================================================================
% 색상 정의 (파스텔 톤 + 다크모드 호환)
%========================================================================================

\usepackage[dvipsnames]{xcolor}

% 밝은 배경용 파스텔 색상
\definecolor{lightblue}{RGB}{220, 235, 255}      % 부드러운 파랑
\definecolor{lightgreen}{RGB}{220, 255, 235}     % 부드러운 초록
\definecolor{lightyellow}{RGB}{255, 250, 220}    % 부드러운 노랑
\definecolor{lightpurple}{RGB}{240, 230, 255}    % 부드러운 보라
\definecolor{lightgray}{gray}{0.95}              % 밝은 회색
\definecolor{lightpink}{RGB}{255, 235, 245}      % 부드러운 핑크
\definecolor{boxgray}{gray}{0.95}
\definecolor{boxblue}{rgb}{0.9, 0.95, 1.0}
\definecolor{boxred}{rgb}{1.0, 0.95, 0.95}

% 진한 색상 (테두리/제목용)
\definecolor{darkblue}{RGB}{50, 80, 150}
\definecolor{darkgreen}{RGB}{40, 120, 70}
\definecolor{darkorange}{RGB}{200, 100, 30}
\definecolor{darkpurple}{RGB}{100, 60, 150}

%========================================================================================
% 박스 환경 (tcolorbox) - 6가지 타입
%========================================================================================

\usepackage[most]{tcolorbox}
\tcbuselibrary{skins, breakable}

% 1. 개요 박스 (강의 시작 부분)
\newtcolorbox{overviewbox}[1][]{
    enhanced,
    colback=lightpurple,
    colframe=darkpurple,
    fonttitle=\bfseries\large,
    title=📚 강의 개요,
    arc=3mm,
    boxrule=1pt,
    left=8pt,
    right=8pt,
    top=8pt,
    bottom=8pt,
    breakable,
    #1
}

% 2. 요약 박스
\newtcolorbox{summarybox}[1][]{
    enhanced,
    colback=lightblue,
    colframe=darkblue,
    fonttitle=\bfseries,
    title=📝 핵심 요약,
    arc=2mm,
    boxrule=0.7pt,
    left=6pt,
    right=6pt,
    top=6pt,
    bottom=6pt,
    breakable,
    #1
}

% 3. 핵심 정보 박스
\newtcolorbox{infobox}[1][]{
    enhanced,
    colback=lightgreen,
    colframe=darkgreen,
    fonttitle=\bfseries,
    title=💡 핵심 정보,
    arc=2mm,
    boxrule=0.7pt,
    left=6pt,
    right=6pt,
    top=6pt,
    bottom=6pt,
    breakable,
    #1
}

% 4. 주의사항 박스
\newtcolorbox{warningbox}[1][]{
    enhanced,
    colback=lightyellow,
    colframe=darkorange,
    fonttitle=\bfseries,
    title=⚠️ 주의사항,
    arc=2mm,
    boxrule=0.7pt,
    left=6pt,
    right=6pt,
    top=6pt,
    bottom=6pt,
    breakable,
    #1
}

% 5. 예제 박스
\newtcolorbox{examplebox}[1][]{
    enhanced,
    colback=lightgray,
    colframe=black!60,
    fonttitle=\bfseries,
    title=📖 예제: #1,
    arc=2mm,
    boxrule=0.7pt,
    left=6pt,
    right=6pt,
    top=6pt,
    bottom=6pt,
    breakable,
}

% 6. 정의 박스
\newtcolorbox{definitionbox}[1][]{
    enhanced,
    colback=lightpink,
    colframe=purple!70!black,
    fonttitle=\bfseries,
    title=📌 정의: #1,
    arc=2mm,
    boxrule=0.7pt,
    left=6pt,
    right=6pt,
    top=6pt,
    bottom=6pt,
    breakable,
}

% 7. 중요 박스 (importantbox - warningbox와 유사)
\newtcolorbox{importantbox}[1][]{
    enhanced,
    colback=boxred,
    colframe=red!70!black,
    fonttitle=\bfseries,
    title=⚠️ 매우 중요: #1,
    arc=2mm,
    boxrule=0.7pt,
    left=6pt,
    right=6pt,
    top=6pt,
    bottom=6pt,
    breakable,
}

% 8. cautionbox (warningbox와 동일)
\let\cautionbox\warningbox
\let\endcautionbox\endwarningbox

%========================================================================================
% 코드 블록 설정 (밝은 배경)
%========================================================================================

\usepackage{listings}

\definecolor{codegray}{rgb}{0.5,0.5,0.5}
\definecolor{codepurple}{rgb}{0.58,0,0.82}
\definecolor{backcolour}{rgb}{0.95,0.95,0.95}

\lstset{
    basicstyle=\ttfamily\small,
    backgroundcolor=\color{lightgray},
    keywordstyle=\color{darkblue}\bfseries,
    commentstyle=\color{darkgreen}\itshape,
    stringstyle=\color{purple!80!black},
    numberstyle=\tiny\color{black!60},
    numbers=left,
    numbersep=8pt,
    breaklines=true,
    breakatwhitespace=false,
    frame=single,
    frameround=tttt,
    rulecolor=\color{black!30},
    captionpos=b,
    showstringspaces=false,
    tabsize=2,
    xleftmargin=15pt,
    xrightmargin=5pt,
    escapeinside={\%*}{*)}
}

% Python 코드 스타일
\lstdefinestyle{pythonstyle}{
    language=Python,
    morekeywords={self, True, False, None},
}

% SQL 코드 스타일
\lstdefinestyle{sqlstyle}{
    language=SQL,
    morekeywords={SELECT, FROM, WHERE, JOIN, GROUP, BY, ORDER, HAVING},
}

%========================================================================================
% 목차 스타일링
%========================================================================================

\usepackage{tocloft}
\renewcommand{\cftsecleader}{\cftdotfill{\cftdotsep}}
\setlength{\cftbeforesecskip}{0.4em}
\renewcommand{\cftsecfont}{\bfseries}
\renewcommand{\cftsubsecfont}{\normalfont}

%========================================================================================
% 표 및 그림
%========================================================================================

\usepackage{graphicx}              % 이미지
\usepackage{adjustbox}             % 표/박스 크기 조절

% 표 캡션 스타일
\usepackage{caption}
\captionsetup[table]{
    labelfont=bf,
    textfont=it,
    skip=5pt
}
\captionsetup[figure]{
    labelfont=bf,
    textfont=it,
    skip=5pt
}

%========================================================================================
% 수학
%========================================================================================

\usepackage{amsmath, amssymb, amsthm}

% 정리 환경
\theoremstyle{definition}
\newtheorem{theorem}{정리}[section]
\newtheorem{lemma}[theorem]{보조정리}
\newtheorem{proposition}[theorem]{명제}
\newtheorem{corollary}[theorem]{따름정리}
\newtheorem{definition}{정의}[section]
\newtheorem{example}{예제}[section]

%========================================================================================
% 하이퍼링크
%========================================================================================

\usepackage[
    colorlinks=true,
    linkcolor=blue!80!black,
    urlcolor=blue!80!black,
    citecolor=green!60!black,
    bookmarks=true,
    bookmarksnumbered=true,
    pdfborder={0 0 0}
]{hyperref}

% PDF 메타데이터는 각 문서에서 설정
\hypersetup{
    pdftitle={CS109A: 데이터 과학 입문 - Lecture 09},
    pdfauthor={강의 노트},
    pdfsubject={Academic Notes}
}

%========================================================================================
% 기타 유용한 패키지
%========================================================================================

\usepackage{enumitem}              % 리스트 커스터마이징
\setlist{nosep, leftmargin=*, itemsep=0.3em}

\usepackage{microtype}             % 타이포그래피 개선
\usepackage{footnote}              % 각주 개선
\usepackage{url}                   % URL 줄바꿈
\urlstyle{same}

%========================================================================================
% 사용자 정의 명령어
%========================================================================================

% 강조 텍스트
\newcommand{\important}[1]{\textbf{\textcolor{red!70!black}{#1}}}
\newcommand{\keyword}[1]{\textbf{#1}}
\newcommand{\term}[1]{\textit{#1}}
\newcommand{\code}[1]{\texttt{#1}}

% 용어 설명 (인라인)
\newcommand{\defterm}[2]{\textbf{#1}\footnote{#2}}

% 섹션 시작 전 페이지 분리
\newcommand{\newsection}[1]{\newpage\section{#1}}

%========================================================================================
% 문서 제목 스타일
%========================================================================================

\usepackage{titling}
\pretitle{\begin{center}\LARGE\bfseries}
\posttitle{\par\end{center}\vskip 0.5em}
\preauthor{\begin{center}\large}
\postauthor{\end{center}}
\predate{\begin{center}\large}
\postdate{\par\end{center}}

%========================================================================================
% 섹션 제목 간격
%========================================================================================

\usepackage{titlesec}
\titlespacing*{\section}{0pt}{1.5em}{0.8em}
\titlespacing*{\subsection}{0pt}{1.2em}{0.6em}
\titlespacing*{\subsubsection}{0pt}{1em}{0.5em}

%========================================================================================
% 메타 정보 박스 명령어
%========================================================================================

\newcommand{\metainfo}[4]{
\begin{tcolorbox}[
    colback=lightpurple,
    colframe=darkpurple,
    boxrule=1pt,
    arc=2mm,
    left=10pt,
    right=10pt,
    top=8pt,
    bottom=8pt
]
\begin{tabular}{@{}rl@{}}
▣ \textbf{강의명:} & #1 \\[0.3em]
▣ \textbf{주차:} & #2 \\[0.3em]
▣ \textbf{교수명:} & #3 \\[0.3em]
▣ \textbf{목적:} & \begin{minipage}[t]{0.75\textwidth}#4\end{minipage}
\end{tabular}
\end{tcolorbox}
}

%========================================================================================
% 끝
%========================================================================================


\begin{document}

\maketitle
\thispagestyle{firstpage}

\metainfo{CS109A: 데이터 과학 입문}{Lecture 09}{Pavlos Protopapas, Kevin Rader, Chris Gumb}{Lecture 09의 핵심 개념 학습}


\part{개요}

\begin{tcolorbox}[colback=blue!5!white, colframe=blue!75!black, title=강의 핵심 요약]
이번 강의는 데이터 과학의 핵심인 **선형 회귀** 모델을 **확률론적 관점**에서 재해석합니다.

\begin{itemize}
    \item \textbf{왜 확률인가?}: 우리가 가진 데이터는 더 큰 모집단(혹은 데이터 생성 프로세스)에서 나온 하나의 '무작위 실현'일 뿐이므로, 불확실성을 다루기 위해 확률론이 필요합니다.
    \item \textbf{핵심 연결 고리}: \textbf{최소제곱법(OLS)}을 사용하여 평균제곱오차(MSE)를 최소화하는 것은, 만약 \textbf{잔차(residuals)가 정규 분포를 따른다고 가정}한다면, 통계적 추론 방법인 \textbf{최대가능도추정(MLE)}을 수행하는 것과 수학적으로 동일함을 보입니다.
    \item \textbf{주요 개념}: 확률 변수, PMF/PDF, 정규 분포, 이항 분포, 가능도(Likelihood), 최대가능도추정(MLE)의 기본 원리를 학습합니다.
    \item \textbf{추론 방법 비교}: 모델의 불확실성을 측정하는 두 가지 방법, 즉 공식(t-검정) 기반의 신뢰구간과 부트스트래핑(Bootstrapping)을 비교합니다.
    \item \textbf{실전 결론}: 모델의 가정(예: 등분산성)이 깨졌을 때, 공식 기반 추론은 잘못된(지나치게 낙관적인) 결론을 내릴 수 있으며, 가정이 적은 부트스트래핑이 더 신뢰할 수 있는 대안이 됩니다.
\end{itemize}
\end{tcolorbox}


\part{핵심 용어 정리}
\label{part:terms}

강의에서 다루는 주요 확률 및 통계 용어를 정리합니다.

\begin{adjustbox}{width=\textwidth, center}
\begin{tabular}{lp{6cm}p{4cm}l}
\toprule
\textbf{용어} & \textbf{쉬운 설명 (직관)} & \textbf{원어} & \textbf{비고 (예시)} \\
\midrule
확률 변수 & 어떤 무작위 현상의 결과를 '숫자'로 바꿔주는 변수 & Random Variable (RV) & 동전 던지기(현상) $\to$ 앞면=1, 뒷면=0 (RV) \\
PMF & 이산 확률 변수(셀 수 있는)가 특정 값을 가질 '확률' & Probability Mass Func. & 주사위 '3'이 나올 확률 = 1/6 \\
PDF & 연속 확률 변수(셀 수 없는)의 '상대적 가능성' (밀도) & Probability Density Func. & 키가 170cm일 확률은 0이지만, 170-171cm 사이일 '확률'은 계산 가능 (곡선 아래 면적) \\
정규 분포 & 자연 현상에서 가장 흔히 발견되는 종 모양(bell-shaped)의 연속 분포 & Normal (Gaussian) Dist. & 사람들의 키, 시험 성적 등 \\
CLT & '많은' 샘플의 '평균'은, 원래 데이터가 어떻든 상관없이, 정규 분포를 따른다는 마법 같은 정리 & Central Limit Theorem & 모집단이 정규분포가 아니어도 $\bar{X}$는 정규분포를 따름 \\
이항 분포 & $n$번의 독립 시도에서 $k$번 성공할 확률 (이산 분포) & Binomial Distribution & 동전을 10번 던져 앞면이 3번 나올 확률 \\
가능도 & '데이터가 관찰된 지금', 어떤 모델(파라미터)이 가장 그럴듯한지에 대한 '믿음의 정도' & Likelihood & $P(\text{Data}|\theta)$가 아닌 $L(\theta|\text{Data})$로 관점을 바꾼 것 \\
MLE & 가능도를 '최대화'하는 파라미터를 찾는 추정 방법. "이 데이터가 나올 확률이 가장 높은 모델은 무엇인가?" & Max. Likelihood Est. & 동전을 10번 던져 앞면이 7번 나왔다면, $p=0.7$을 MLE로 추정 \\
로그 가능도 & 가능도 함수에 로그(log)를 씌운 것. 미분을 쉽게 하기 위해 사용. (곱셈 $\to$ 덧셈) & Log-Likelihood & $\log(L(\theta|\text{Data}))$ \\
통계적 추론 & 샘플 데이터를 이용해 모집단의 특성(파라미터)에 대해 추측하는 과정 & Statistical Inference & 샘플 평균으로 모집단 평균을 추측 \\
표준 오차 & 추정치의 불확실성(변동성)을 나타내는 값. (추정치의 표준 편차) & Standard Error (SE) & $\hat{\beta}_1$의 SE: "나의 기울기 추정치가 평균적으로 얼마나 빗나갈까?" \\
신뢰 구간 & 모집단 파라미터가 존재할 것이라고 '신뢰'하는 구간 (예: 95% CI) & Confidence Interval (CI) & 95\% CI: [0.5, 0.7]. "이런 샘플링을 100번 하면 95번은 이 구간이 실제 값을 포함한다." \\
가설 검정 & $H_0$(귀무가설)과 $H_A$(대립가설)을 세우고, 데이터로 $H_0$을 기각할지 결정하는 절차 & Hypothesis Testing & $H_0$: $\beta_1 = 0$ (효과 없음) \\
p-value & 귀무가설($H_0$)이 '맞다'고 가정할 때, 지금 관찰된 데이터 혹은 더 극단적인 데이터가 나올 확률 & p-value & p-value가 낮으면($<0.05$), "$H_0$이 맞는데 이런 일이? $\to H_0$을 기각하자" \\
다중공선성 & 예측 변수(X)들끼리 강한 상관관계를 보이는 현상 & Collinearity & '방의 개수'와 '집 크기(sqft)'가 강하게 비례하는 경우 \\
이분산성 & 오차(잔차)의 분산이 일정하지 않은 현상 & Heteroscedasticity & 집 크기가 클수록 가격 예측 오차의 변동 폭도 커짐 \\
\bottomrule
\end{tabular}
\end{adjustbox}
\caption{제9강 핵심 용어 정리표}
\label{table:terms}


\newpage
\part{데이터 분석 예시: 주택 가격 예측}
\label{part:example}

강의는 케임브리지/서머빌 지역의 주택 가격 데이터를 분석하는 예시로 시작합니다.

\section{문제 정의 및 데이터 탐색 (EDA)}

\begin{itemize}
    \item \textbf{데이터 소스}: Redfin.com (온라인 부동산 사이트)
    \item \textbf{데이터 규모}: $n=592$개의 주택 판매 기록
    \item \textbf{질문 (목표)}: 어떤 변수(특성)가 주택 판매 가격과 연관되어 있는가?
    \item \textbf{반응 변수 (Y)}: \texttt{price} (주택 판매 가격)
    \item \textbf{예측 변수 (X)}:
        \begin{itemize}
            \item \texttt{type}: 주택 유형 (콘도, 단독주택, 다가구 등) - (범주형)
            \item \texttt{beds}: 침실 수 - (이산형)
            \item \texttt{baths}: 욕실 수 - (이산형)
            \item \texttt{sqft}: 면적 (제곱피트) - (연속형)
            \item \texttt{lotsize}: 대지 크기 - (연속형)
            \item \texttt{year}: 건축 연도 - (이산형)
            \item \texttt{dist}: 하버드 스퀘어 T-stop(지하철역)까지의 거리 - (연속형)
        \end{itemize}
\end{itemize}

\subsection{데이터 전처리 (Cleaning)}

분석 전, 몇 가지 데이터 정리 작업이 수행되었습니다.
\begin{enumerate}
    \item \textbf{결측치 처리 (Missing Data)}:
        \begin{itemize}
            \item \texttt{lotsize} (대지 크기)와 \texttt{hoa} (주택소유자협회비) 변수에서 많은 결측치(NA)가 발견되었습니다.
            \item \texttt{lotsize}는 주로 콘도나 타운하우스에서 결측되었고, \texttt{hoa}는 그 반대였습니다.
            \item 이는 데이터가 누락된 것이 아니라 '해당 없음'을 의미할 가능성이 높습니다.
            \item 따라서 이 결측치들을 \textbf{0으로 대체(Imputation)}했습니다. 이는 합리적인 가정입니다.
        \end{itemize}
    \item \textbf{스케일 조정 (Rescaling)}:
        \begin{itemize}
            \item \texttt{price} 변수의 단위를 '달러(\$)'에서 '천 달러(\$1000s)'로 변경했습니다.
            \item 예: \$1,250,000 $\to$ 1250.
            \item 이는 모델의 계수(coefficient)를 해석하기 쉽게 만듭니다.
        \end{itemize}
    \item \textbf{타입 변환 (Type Conversion)}:
        \begin{itemize}
            \item \texttt{zip} (우편번호)는 숫자(int)로 저장되어 있었지만, 실제로는 순서나 크기가 없는 범주형 데이터입니다.
            \item 따라서 문자열(string) 타입으로 변환하여 모델이 이를 연속형 숫자로 오해하지 않도록 했습니다.
        \end{itemize}
\end{enumerate}

\section{데이터 시각화 및 주요 발견}

데이터 탐색(EDA)을 통해 두 가지 중요한 통계적 문제를 발견했습니다.

\subsection{발견 1: 이분산성 (Heteroscedasticity)}

\begin{itemize}
    \item \texttt{price} (가격)와 \texttt{sqft} (면적)의 산점도(scatter plot)를 확인했습니다.
    \item \textbf{현상}: 면적(\texttt{sqft})이 작은 집들은 가격 변동성이 작은 반면 (즉, 예측 오차가 적음), 면적이 큰 집들은 가격 변동성이 매우 컸습니다 (즉, 예측 오차가 큼).
    \item \textbf{정의}: 이처럼 예측 변수(X)의 값에 따라 오차(잔차)의 분산이 일정하지 않은 현상을 \textbf{이분산성(Heteroscedasticity)}이라고 합니다.
    \item \textbf{문제점}: 이는 표준적인 선형 회귀의 기본 가정('오차의 분산은 일정하다')을 위배합니다. 이 가정이 깨지면, 모델이 추정한 계수($\hat{\beta}$)는 괜찮을지 몰라도, 그 계수의 \textbf{표준 오차(SE)와 신뢰 구간(CI) 계산이 부정확}해집니다.
\end{itemize}

% \begin{figure}[h]
% \centering
% % \includegraphics[width=0.7\textwidth]{path/to/hetero_plot.png} % 이미지 파일 경로
% \framebox(300,150){[Price vs. Sqft 산점도 이미지]}
% \caption{이분산성(Heteroscedasticity) 예시. X(sqft)가 증가함에 따라 Y(price)의 분산(퍼짐 정도)이 커집니다.}
% \label{fig:hetero}
% \end{figure}

\subsection{발견 2: 다중공선성 (Collinearity)}

여러 예측 변수를 모두 포함한 다중 선형 회귀 모델을 피팅했습니다.

\begin{tcolorbox}[colback=gray!10!white, colframe=black, title=다중 회귀 모델 결과 (일부)]
\begin{verbatim}
==================================================================
                 coef    std err          t      P>|t|
------------------------------------------------------------------
Intercept  -1949.0670   745.203     -2.615      0.009
sqft             0.6411     0.044     14.720      0.000
beds           -89.9345    23.532     -3.822      0.000  <-- (주목!)
baths          198.4646    31.332      6.334      0.000
...
==================================================================
\end{verbatim}
\end{tcolorbox}

\begin{itemize}
    \item \textbf{이상한 점}: \texttt{beds} (침실 수)의 계수가 \textbf{음수(-89.9)}로 나타났습니다.
    \item \textbf{직관적 해석}: "다른 모든 변수(면적, 욕실 수 등)를 \textbf{고정한 채로}, 침실 수만 1개 늘리면 집값이 약 \$89,934 하락한다."
    \item \textbf{이게 말이 되나?}: 상식적으로 침실이 많으면 집값이 올라야 합니다. 이것은 모델의 오류일까요?
    \item \textbf{원인 (다중공선성)}: 아닙니다. 이는 \texttt{beds}와 \texttt{sqft} 간의 강한 \textbf{다중공선성(Collinearity)} 때문입니다.
    \item \textbf{올바른 해석 (비유)}:
        \begin{itemize}
            \item '다른 모든 변수를 고정한다'는 것은, 특히 '총 면적(\texttt{sqft})을 고정한다'는 의미입니다.
            \item 즉, \textbf{동일한 총 면적의 집}에서 침실 수를 1개 늘린다는 것은, 기존의 방들을 쪼개서 더 \textbf{작고 답답한 침실들}을 만든다는 뜻입니다.
            \item 따라서 "집이 넓어지지 않는데 방만 억지로 구겨 넣으면 (cramming another bedroom) 집의 가치가 떨어진다"는 합리적인 해석이 가능합니다.
        \end{itemize}
\end{itemize}

\subsection{모델링 라이브러리: \texttt{statsmodels}}

\begin{itemize}
    \item 데이터 과학에서는 \texttt{sklearn} 라이브러리를 예측에 주로 사용하지만, 통계적 추론과 해석에는 \texttt{statsmodels} 라이브러리가 더 편리할 수 있습니다.
    \item \texttt{statsmodels}는 R 언어처럼 공식(formula)을 사용하여 모델을 정의할 수 있게 해줍니다. (예: \texttt{"price \textasciitilde{} sqft + type + dist"})
    \item 이는 특히 범주형 변수(type)를 다룰 때 자동으로 더미 변수(dummy variable)를 생성해주는 등 편리함을 제공합니다.
\end{itemize}

\newpage
\part{복습: 모델 검증 및 규제}
\label{part:review}

본격적인 확률론에 들어가기 전, 이전 강의의 핵심 개념인 교차 검증과 규제를 복습합니다.

\section{교차 검증 (Cross-Validation)의 활용}

\begin{tcolorbox}[colback=green!5!white, colframe=green!60!black, title=Q: 교차 검증(CV)은 언제 사용하나요?]
교차 검증은 특정 모델에 국한된 기술이 아니라, \textbf{모델 선택(Model Selection)과 관련된 모든 의사결정}에 사용할 수 있는 일반적인 도구입니다.

\begin{itemize}
    \item \textbf{A. k-NN 모델}에서 최적의 $k$ (이웃 수)를 고를 때
    \item \textbf{B. Ridge/Lasso 모델}에서 최적의 $\lambda$ (규제 강도)를 고를 때
    \item \textbf{C. 선형 회귀}에서 어떤 예측 변수(feature) 조합이 최선인지 고를 때
    \item \textbf{D. 서로 다른 모델 계열} (예: k-NN vs. 선형 회귀) 중 어느 것이 더 나은지 비교할 때
\end{itemize}
\textbf{정답: A, B, C, D 모두.} 이 모든 것은 '모델을 선택'하는 과정이며, CV는 이 선택의 성능을 평가하는 표준 방법입니다.
\end{tcolorbox}

\section{데이터 표준화 (Standardization)}

\begin{itemize}
    \item \textbf{Q}: 예측 변수(X)들을 표준화(Standardize)해야 하는 경우는 언제인가요?
    \item \textbf{A}: (D) "예측 변수들을 \textbf{동등하게 처리(treat equally)}하고 싶을 때"입니다.
    \item \textbf{이유}:
        \begin{itemize}
            \item \textbf{k-NN (거리 기반)}: 표준화를 하지 않으면, '집값'(단위: \$1000s) 같은 큰 스케일의 변수가 '방 개수'(단위: 1) 같은 작은 스케일의 변수보다 거리에 훨씬 큰 영향을 미칩니다. 즉, 스케일이 큰 변수에 편향됩니다.
            \item \textbf{Ridge/Lasso (규제 기반)}: 규제는 계수($\beta$)의 '크기'에 페널티를 줍니다. 만약 \texttt{sqft}가 피트(ft)가 아닌 인치(inch) 단위라면, 스케일이 커져서 계수($\beta$) 값은 0에 매우 가까워질 것입니다. 스케일이 다르면 페널티가 불공평하게 적용되므로, 표준화가 권장됩니다.
        \end{itemize}
    \item \textbf{주의}: 표준화는 '항상(Always)' 정답은 아닙니다. 예를 들어, 범주형 변수를 더미(0/1)로 만들었을 때, 이 변수들을 다른 연속형 변수와 동일한 스케일로 맞추는 것이 오히려 모델 해석을 어렵게 하거나 성능을 저하시킬 수도 있습니다.
\end{itemize}

\section{규제 모델 궤적도 (Trajectory Plots) 해석}

Lasso와 Ridge 모델에서 규제 강도($\lambda$ 또는 $\alpha$)를 변화시킬 때, 각 변수의 회귀 계수($\beta$)가 어떻게 변하는지 그린 그래프입니다.

\begin{tcolorbox}[colback=yellow!5!white, colframe=yellow!75!black, title=Lasso vs. Ridge 궤적도]
\begin{itemize}
    \item \textbf{Lasso (라쏘)}: $\lambda$가 커지면 계수가 정확히 \textbf{0}이 됩니다. 0이 된 변수는 모델에서 '탈락'한 것이므로, \textbf{특성 선택(Feature Selection)}에 유용합니다.
    \item \textbf{Ridge (릿지)}: $\lambda$가 커져도 계수가 0에 가까워질 뿐, 정확히 0이 되지는 않습니다. 모든 변수를 유지하되 영향력을 줄입니다.
\end{itemize}
\end{tcolorbox}

\begin{itemize}
    \item \textbf{이상적인 궤적도}: 교과서에 나오는 궤적도는 매우 매끄러운 곡선을 그리며 0으로 수렴합니다. 이는 모든 예측 변수(X)들이 서로 \textbf{독립(independent)}이라고 가정한, 비현실적인 상황입니다.
    \item \textbf{현실적인 궤적도}: 실제 데이터에서는 \textbf{다중공선성(Collinearity)} 때문에 궤적도가 매우 지저분합니다.
    \item \textbf{다중공선성의 징후}:
        \begin{itemize}
            \item 특정 계수가 0으로 수렴하다가 갑자기 부호가 바뀌거나 (0을 통과함)
            \item 0으로 수축하는 대신 \textbf{오히려 일시적으로 값이 커졌다가} 줄어드는 경우
            \item 이는 $\lambda$가 커짐에 따라 한 변수(A)가 페널티를 받아 영향력이 줄어들 때, 그와 상관관계가 높은 다른 변수(B)가 A의 예측력(power)을 '넘겨받아' 계수가 커지는 현상을 나타냅니다.
        \end{itemize}
\end{itemize}


\newpage
\part{확률론의 기초}
\label{part:probability}

데이터 과학 모델의 근간이 되는 확률 이론의 기본 개념들을 복습합니다.

\section{확률과 확률 변수}

\begin{itemize}
    \item \textbf{확률(Probability)이란?}
        \begin{itemize}
            \item \textbf{정의}: 어떤 사건이 발생할 \textbf{장기적인 상대 빈도(long-run, relative frequency)}.
            \item \textbf{범위}: 0 (절대 발생 안 함) 에서 1 (항상 발생) 사이의 값을 가집니다.
        \end{itemize}
    \item \textbf{왜 데이터 과학에서 확률이 중요한가?}
        \begin{itemize}
            \item 우리가 가진 데이터(샘플)는 더 큰 \textbf{데이터 생성 프로세스(Data Generating Process)} 또는 모집단에서 나온 하나의 \textbf{무작위적인 실현(random realization)}에 불과합니다.
            \item 확률론은 이 '불확실성'을 수학적으로 다루고, 샘플을 넘어선 모집단에 대한 추론을 가능하게 하는 언어입니다.
        \end{itemize}
    \item \textbf{확률 변수(Random Variable, RV)란?}
        \begin{itemize}
            \item 무작위적인 현상(phenomenon)의 결과를 \textbf{숫자(numeric outcome)}로 대응시키는 함수(또는 변수)입니다.
            \item \textbf{예시}: "하버드 학생이 Mac 노트북을 사용하는가?"라는 현상을 조사할 때
            \item $X_1$ = 첫 번째 학생의 응답
            \item $X_1 = 1$ (Mac 사용자), $X_1 = 0$ (그 외 사용자) $\to$ $X_1$은 확률 변수입니다.
        \end{itemize}
\end{itemize}

\section{핵심 구분: PMF vs. PDF (이산형 vs. 연속형)}
\label{sec:pmf_pdf}

확률 변수는 크게 두 종류로 나뉘며, 이에 따라 확률을 기술하는 방식이 달라집니다.

\begin{tcolorbox}[colback=blue!5!white, colframe=blue!75!black, title=이산형 확률 변수 (Discrete RV)]
\begin{itemize}
    \item \textbf{정의}: 변수가 가질 수 있는 값이 '셀 수 있는' 경우 (예: 0, 1, 2, ... 또는 정수 값).
    \item \textbf{확률 함수}: \textbf{확률 질량 함수 (Probability Mass Function, PMF)}.
    \item \textbf{특징}: $P(X=x)$ 값이 특정 '확률'을 가집니다. 막대 그래프(bar chart)로 시각화됩니다.
    \item \textbf{예시}: 동전 던지기 (0, 1), 주사위 굴리기 (1, 2, 3, 4, 5, 6), 침실 수 (1, 2, 3, ...).
\end{itemize}
\end{tcolorbox}

\begin{tcolorbox}[colback=red!5!white, colframe=red!75!black, title=연속형 확률 변수 (Continuous RV)]
\begin{itemize}
    \item \textbf{정의}: 변수가 특정 범위 내의 '모든 실수' 값을 가질 수 있는 경우.
    \item \textbf{확률 함수}: \textbf{확률 밀도 함수 (Probability Density Function, PDF)}.
    \item \textbf{특징}:
        \begin{itemize}
            \item $f(x)$ 값(곡선의 높이)은 확률이 아니라 '밀도(density)' 또는 '상대적 가능성'입니다.
            \item \textbf{특정 값에서의 확률은 0입니다.} (예: $P(\text{키}=175.000... \text{cm}) = 0$).
            \item \textbf{확률은 항상 '구간'으로 계산}되며, 이는 PDF 곡선 아래의 \textbf{면적(Area)}과 같습니다.
        \end{itemize}
    \item \textbf{예시}: 사람의 키, 주택 가격(\texttt{price}), 지하철역까지의 거리(\texttt{dist}).
\end{itemize}
\end{tcolorbox}

\section{주요 확률 분포}

\subsection{베르누이 분포 (Bernoulli Distribution)}
\begin{itemize}
    \item \textbf{정의}: '단 한 번'의 시도($n=1$)에서 '성공(1)' 또는 '실패(0)' 두 가지 결과만 나오는 분포. (가장 단순한 이산 분포)
    \item \textbf{파라미터}: $p$ (성공할 확률)
    \item \textbf{예시}: 동전 1번 던지기 (앞면=1, 뒷면=0), 학생 1명이 Mac 유저(1)인지 아닌지(0).
    \item \textbf{PMF}: $P(X=x) = p^x (1-p)^{1-x}$, (단, $x \in \{0, 1\}$)
\end{itemize}

\subsection{이항 분포 (Binomial Distribution)}
\begin{itemize}
    \item \textbf{정의}: '서로 독립'인 베르누이 시도를 $n$번 반복했을 때, '성공 횟수(x)'를 나타내는 이산 분포.
    \item \textbf{파라미터}: $n$ (총 시도 횟수), $p$ (각 시도의 성공 확률)
    \item \textbf{예시}: 동전을 10번($n=10$) 던졌을 때, 앞면이 3번($x=3$) 나올 확률 (단, $p=0.5$).
    \item \textbf{PMF}: $P(X=x) = \binom{n}{x} p^x (1-p)^{n-x}$
    \item \textbf{$\binom{n}{x}$ (n choose x)의 의미}: "$n$번의 시도 중 성공이 $x$번 발생하는 모든 \textbf{경우의 수}"를 의미합니다. (예: HHH...TTT, HTHTH... 등). 이항 계수(Binomial Coefficient)라고 부릅니다.
\end{itemize}

\subsection{정규 분포 (Normal Distribution)와 중심극한정리 (CLT)}
\begin{itemize}
    \item \textbf{정의}: 자연계와 사회 현상에서 가장 흔하게 발견되는 \textbf{종 모양(bell-shaped)}의 연속 분포. \textbf{가우시안 분포(Gaussian Distribution)}라고도 합니다.
    \item \textbf{파라미터}: $\mu$ (평균, 분포의 중심), $\sigma^2$ (분산, 분포의 퍼짐 정도)
    \item \textbf{표준 정규 분포 (Standard Normal)}: 평균이 0이고 표준편차가 1인($Z \sim N(0, 1)$) 특별한 정규 분포.
    \item \textbf{표준화 (Standardization)}: 어떤 정규 분포 $X \sim N(\mu, \sigma^2)$라도, $Z = \frac{X - \mu}{\sigma}$ 공식을 통해 표준 정규 분포로 변환할 수 있습니다. $Z$값은 "평균에서 몇 표준편차만큼 떨어져 있는가?"를 의미합니다.
\end{itemize}

\begin{tcolorbox}[colback=yellow!5!white, colframe=yellow!75!black, title=핵심 원리: 중심극한정리 (Central Limit Theorem, CLT)]
\textbf{Q: 왜 정규 분포가 이렇게 중요한가요?}\\
\textbf{A: 중심극한정리(CLT) 때문입니다.}

\textbf{CLT란?} "모집단이 어떤 분포(정규분포가 아니어도 됨)를 따르든 상관없이, 거기서 뽑은 샘플의 크기 $n$이 충분히 크다면, 그 \textbf{샘플들의 평균($\bar{X}$)}이 이루는 분포는 \textbf{정규 분포}에 근사한다."

\begin{itemize}
    \item \textbf{직관적 예시 (신장)}: 한 사람의 키(Height)는 수많은 작은 요인들(유전, 영양, 수면, ...)의 '합' 또는 '평균'으로 결정됩니다. CLT에 의해, 이렇게 많은 요인들이 합쳐진 결과물은 정규 분포를 따르게 됩니다.
    \item \textbf{수학적 표현}: $X_i \sim (\mu, \sigma^2)$ 일 때, $\bar{X} \sim N(\mu, \frac{\sigma^2}{n})$
    \item \textbf{중요한 함의}:
        \begin{itemize}
            \item 샘플 평균($\bar{X}$)의 평균은 모집단 평균($\mu$)과 같습니다.
            \item 샘플 평균($\bar{X}$)의 분산은 $n$이 커질수록 작아집니다 ($n$으로 나눔).
            \item (직관: 샘플 1개를 뽑는 것보다, 100개를 뽑아 평균내는 것이 훨씬 더 안정적이고 실제 평균에 가깝습니다.)
        \end{itemize}
\end{itemize}
\end{tcolorbox}


\newpage
\part{최대가능도추정 (MLE)과 선형 회귀의 연결}
\label{part:mle}

이번 강의의 가장 핵심적인 부분으로, OLS 회귀 모델이 어떻게 확률론적 MLE와 연결되는지 설명합니다.

\section{추론: 확률의 역방향 문제}

확률과 통계적 추론은 동전의 양면과 같습니다.

\begin{tcolorbox}[colback=blue!5!white, colframe=blue!75!black, title=확률 (Probability) vs. 추론 (Inference)]
\begin{itemize}
    \item \textbf{확률 (Deduction)}: \textbf{모델(파라미터) $\to$ 데이터}
        \begin{itemize}
            \item \textbf{질문}: "공정한 동전($p=0.5$)이 주어졌을 때, 10번 던져 앞면이 8번 나올 확률($P(\text{Data}|\text{Model})$)은 얼마인가?"
            \item \textbf{방향}: 원인 $\to$ 결과
        \end{itemize}
    \item \textbf{통계적 추론 (Inference)}: \textbf{데이터 $\to$ 모델(파라미터)}
        \begin{itemize}
            \item \textbf{질문}: "동전을 10번 던져 앞면이 8번 나왔다($\text{Data}$). 이 동전은 공정한가($p=0.5$), 아니면 편향되었는가($p=0.8$)? \textbf{어떤 모델이 이 데이터를 가장 잘 설명하는가?}"
            \item \textbf{방향}: 결과 $\to$ 원인 (우리가 데이터 과학에서 하는 일!)
        \end{itemize}
\end{itemize}
\end{tcolorbox}

\section{가능도 함수 (Likelihood Function)}

\begin{itemize}
    \item \textbf{정의}: 가능도 함수 $L(\theta | \text{data})$는 PMF 또는 PDF와 \textbf{수학적으로는 동일한 함수}입니다.
    \item \textbf{관점의 차이}:
        \begin{itemize}
            \item PDF/PMF, $f(\text{data} | \theta)$: $\theta$(파라미터)를 고정하고 $\text{data}$를 변수로 봅니다.
            \item Likelihood, $L(\theta | \text{data})$: $\text{data}$를 (우리가 관찰한) 고정된 값으로 보고, $\theta$(파라미터)를 변수로 봅니다.
        \end{itemize}
    \item \textbf{의미}: 이 함수는 "관찰된 데이터 하에서, 이 파라미터 $\theta$가 얼마나 그럴듯한지(likely)"를 측정합니다.
    \item \textbf{독립 가정}: 만약 $n$개의 데이터가 모두 독립적으로($i.i.d.$) 샘플링되었다면, 전체의 가능도는 각 데이터 포인트의 가능도의 \textbf{곱(product)}으로 표현됩니다.
    $$L(\theta | x_1, ..., x_n) = \prod_{i=1}^{n} L(\theta | x_i) = \prod_{i=1}^{n} f(x_i | \theta)$$
\end{itemize}

\section{로그 가능도 (Log-Likelihood)와 MLE}

\begin{itemize}
    \item \textbf{문제점}: 수많은 확률값을 '곱하는' 것은 수학적으로 매우 다루기 어렵습니다. (값이 0에 가깝게 작아지거나, 미분이 복잡해짐)
    \item \textbf{해결책}: \textbf{로그 가능도 (Log-Likelihood)} $l(\theta) = \log(L(\theta))$를 사용합니다.
    \item \textbf{이유}:
        \begin{enumerate}
            \item $\log$ 함수는 단조 증가(monotonic) 함수이므로, $L(\theta)$를 최대화하는 $\theta$값은 $l(\theta)$를 최대화하는 $\theta$값과 \textbf{동일}합니다.
            \item 로그의 성질($\log(A \times B) = \log(A) + \log(B)$) 덕분에, 거대한 \textbf{곱셈(Product)}이 간단한 \textbf{덧셈(Sum)}으로 바뀝니다.
            $$l(\theta | \text{data}) = \log \left( \prod_{i=1}^{n} f(x_i | \theta) \right) = \sum_{i=1}^{n} \log(f(x_i | \theta))$$
        \end{enumerate}
    \item \textbf{최대가능도추정 (Maximum Likelihood Estimator, MLE)}:
        \begin{itemize}
            \item \textbf{정의}: 이 로그 가능도 함수 $l(\theta)$를 \textbf{최대화}하는 파라미터 $\hat{\theta}$를 찾는 방법입니다.
            \item \textbf{직관}: "내가 가진 데이터를 만들어 냈을 가능성이 가장 높은 파라미터(모델)를 찾겠다!"
            \item \textbf{예시}: 데이터 [3, 5, 10]이 $N(\mu, \sigma^2=4)$에서 나왔다고 가정할 때, 이 데이터의 로그 가능도 함수는 $\mu=6$일 때 최대가 됩니다. 따라서 MLE $\hat{\mu} = 6$이며, 이는 샘플 평균($\bar{x}$)과 같습니다.
        \end{itemize}
    \item \textbf{MLE를 찾는 방법}:
        \begin{enumerate}
            \item \textbf{분석적 방법 (Analytical)}: $l(\theta)$를 $\theta$에 대해 미분하고, 그 값을 0으로 만드는 $\theta$를 찾습니다. (수학적으로 해가 구해지는 경우)
            \item \textbf{수치적 방법 (Numerical)}: 해가 복잡할 때 컴퓨터를 사용합니다. \textbf{경사 하강법(Gradient Descent)}을 이용해 \textbf{'음의' 로그 가능도 (Negative Log-Likelihood)}를 \textbf{최소화(minimize)}합니다. (최대화 $\to$ 음수 최소화)
        \end{enumerate}
\end{itemize}

\section{OLS와 MLE의 통합 (The Big Connection)}
\label{sec:ols_mle_connection}

이제, 왜 우리가 선형 회귀에서 손실 함수로 \textbf{평균제곱오차(MSE)}를 사용했는지에 대한 확률론적 정당성을 찾게 됩니다.

\begin{tcolorbox}[colback=yellow!5!white, colframe=red!75!black, title=핵심 결론: OLS는 MLE의 특별한 경우이다]
\textbf{가정}: 선형 회귀 모델 $Y_i = \beta_0 + \beta_1 X_i + \epsilon_i$ 에서, \textbf{잔차(error term) $\epsilon_i$가 서로 독립이며 평균이 0인 정규 분포 $\epsilon_i \sim N(0, \sigma^2)$를 따른다고 가정}하자.

\begin{enumerate}
    \item \textbf{모델 재정의}:
        위 가정은 $Y_i$ 자체가 $X_i$에 조건부로 정규 분포를 따른다는 의미입니다.
        $$Y_i | X_i \sim N(\text{mean}=\beta_0 + \beta_1 X_i, \text{variance}=\sigma^2)$$
    
    \item \textbf{가능도 함수 작성}:
        $n$개의 모든 데이터에 대한 (로그) 가능도 함수를 작성합니다. 정규 분포 PDF 공식을 사용하고 $\log$를 씌워 덧셈으로 만듭니다.
        $$l(\beta_0, \beta_1, \sigma^2 | \text{Data}) = \sum_{i=1}^{n} \log \left( \frac{1}{\sqrt{2\pi\sigma^2}} e^{-\frac{1}{2} \left(\frac{Y_i - (\beta_0 + \beta_1 X_i)}{\sigma}\right)^2} \right)$$

    \item \textbf{함수 정리}:
        로그를 풀면 두 부분으로 나뉩니다.
        $$l(...) = \underbrace{ \sum_{i=1}^{n} \log\left(\frac{1}{\sqrt{2\pi\sigma^2}}\right) }_{\text{(A) $\beta$와 무관한 상수항}} - \underbrace{ \frac{1}{2\sigma^2} \sum_{i=1}^{n} \left( Y_i - (\beta_0 + \beta_1 X_i) \right)^2 }_{\text{(B) $\beta$가 포함된 항}} $$
    
    \item \textbf{MLE 목표}:
        우리의 목표는 이 $l(...)$ 함수를 \textbf{최대화}하는 $\beta_0$와 $\beta_1$를 찾는 것입니다 (MLE).
        
    \item \textbf{결론}:
        \begin{itemize}
            \item (A) 부분은 $\beta_0, \beta_1$와 아무 상관이 없으므로 무시할 수 있습니다.
            \item $l(...)$를 최대화하려면, (B) 부분을 최대화해야 합니다.
            \item (B) 앞에는 음수(-) 부호가 붙어 있으므로, (B)를 최대화하는 것은 괄호 안의 $\sum (\dots)^2$ 부분을 \textbf{최소화}하는 것과 같습니다.
            \item $\sum_{i=1}^{n} \left( Y_i - (\beta_0 + \beta_1 X_i) \right)^2$ $\leftarrow$ 이것이 바로 \textbf{오차제곱합 (Sum of Squared Errors, SSE)}입니다!
        \end{itemize}
\end{enumerate}

\textbf{최종 요약: '잔차가 정규 분포를 따른다'고 가정한 상태에서 MLE를 찾는 것은, 수학적으로 OLS(최소제곱법)를 사용하여 SSE(혹은 MSE)를 최소화하는 것과 정확히 일치합니다.}

이는 우리가 왜 손실 함수로 MSE를 사용하는지에 대한 강력한 이론적 근거를 제공합니다.
\end{tcolorbox}


\newpage
\part{통계적 추론: 불확실성 정량화}
\label{part:inference}

모델이 $\hat{\beta}_1 = 0.5898$이라는 '하나의 값'(점 추정치)을 주었지만, 이 추정치가 얼마나 신뢰할 수 있는지(불확실성)를 알아야 합니다.

\section{점 추정(Point Estimate)의 한계}

\begin{itemize}
    \item 우리가 얻은 $\hat{\beta}_1 = 0.5898$은 $n=592$개의 '샘플'로 계산한 값입니다.
    \item 만약 우리가 다른 592개의 샘플을 뽑았다면, $\hat{\beta}_1$ 값은 $0.59$나 $0.57$처럼 약간 다른 값이 나왔을 것입니다.
    \item 즉, 우리의 추정치($\hat{\beta}$) 자체도 '불확실성'을 가집니다.
    \item 통계적 추론의 목표는 이 불확실성을 정량화하는 것입니다. (예: "진짜 $\beta_1$이 0.6일 가능성은?", "0일 가능성은?")
    \item 우리는 두 가지 방법(신뢰 구간, 가설 검정)을 사용합니다.
\end{itemize}

\section{신뢰 구간 (Confidence Interval, CI)}

\begin{itemize}
    \item \textbf{목표}: "진짜 $\beta_1$ (모집단의 기울기)이 존재할 것이라고 95\% 신뢰하는 \textbf{구간}을 제공하자."
    \item \textbf{방법 1 (부트스트래핑, Bootstrapping)}: (이전 강의에서 배움)
        \begin{enumerate}
            \item 원본 데이터(592개)에서 중복을 허용하여 592개를 다시 뽑습니다 (Resampling).
            \item 모델을 다시 피팅하여 $\hat{\beta}_1^*$ (부트스트랩 추정치)를 기록합니다.
            \item 이 과정을 1000번 반복하여 $\hat{\beta}_1^*$의 분포를 만듭니다.
            \item 이 분포의 하위 2.5\%와 상위 97.5\% 지점을 찾아 95\% 신뢰 구간으로 삼습니다.
        \end{enumerate}
    \item \textbf{방법 2 (공식 기반, Formula-based)}: (이번 강의)
        \begin{itemize}
            \item 확률 이론(CLT, t-분포)을 바탕으로 신뢰 구간을 계산하는 공식을 사용합니다.
            \item \textbf{신뢰 구간 공식}: $\text{추정치} \pm (\text{임계값}) \times (\text{표준 오차})$
            \item $$ \hat{\beta}_1 \pm t^* \cdot \hat{SE}(\hat{\beta}_1) $$
        \end{itemize}
\end{itemize}

\subsection{표준 오차 (Standard Error, SE)}
\begin{itemize}
    \item \textbf{정의}: 표준 오차($SE(\hat{\beta}_1)$)는 추정치($\hat{\beta}_1$)의 표준 편차입니다.
    \item \textbf{직관}: "우리의 추정치가 평균적으로 얼마나 부정확한가?" (불확실성의 크기)
    \item 선형 회귀의 가정(정규성, 등분산성 등)이 모두 맞는다면, $\hat{SE}(\hat{\beta}_1)$를 계산하는 수학 공식이 존재합니다.
\end{itemize}

\section{가설 검정 (Hypothesis Testing)}

\begin{itemize}
    \item \textbf{목표}: "특정 가설(예: 'sqft는 가격에 영향이 없다')이 맞는지 틀리는지 데이터로 검증하자."
    \item \textbf{단계 (t-검정 예시)}:
        \begin{enumerate}
            \item \textbf{가설 설정}:
                \begin{itemize}
                    \item $H_0$ (귀무가설): $\beta_1 = 0$ (sqft는 가격과 관계가 없다. 기울기는 0이다.)
                    \item $H_A$ (대립가설): $\beta_1 \neq 0$ (sqft는 가격과 관계가 있다. 기울기는 0이 아니다.)
                \end{itemize}
            \item \textbf{검정 통계량 (Test Statistic) 계산}:
                \begin{itemize}
                    \item $t = \frac{\hat{\beta}_1 - 0}{\hat{SE}(\hat{\beta}_1)}$
                    \item \textbf{직관}: "우리가 관찰한 기울기($\hat{\beta}_1$)가, 0으로부터 \textbf{몇 표준 오차(SE)만큼} 떨어져 있는가?"
                    \item $t$값이 크다는 것은 $\hat{\beta}_1$이 0과 매우 멀리 떨어져 있다는 뜻이며, 이는 $H_0$에 불리한 증거입니다.
                \end{itemize}
            \item \textbf{p-value 계산}:
                \begin{itemize}
                    \item "$H_0$이 정말 맞다($\beta_1=0$)고 가정할 때, 지금 우리가 관찰한 $t$값만큼 (혹은 더) 극단적인 $t$값이 우연히 나올 확률은 얼마인가?"
                \end{itemize}
            \item \textbf{결정}:
                \begin{itemize}
                    \item p-value가 매우 낮으면 (관례적으로 $< 0.05$), "우연이라고 보기엔 너무 드문 일이다. $H_0$이 틀린 것 같다." $\to$ \textbf{$H_0$을 기각(Reject)}합니다.
                    \item "sqft는 가격에 통계적으로 유의미한(statistically significant) 영향을 미친다"고 결론 내립니다.
                \end{itemize}
        \end{enumerate}
\end{itemize}

\section{부트스트래핑 vs. 공식: 최종 비교}

주택 가격 예시(\texttt{price \textasciitilde{} sqft})로 돌아가, 두 가지 방법으로 계산된 $\beta_1$(sqft의 계수)의 95\% 신뢰 구간을 비교합니다.

\begin{tcolorbox}[colback=red!5!white, colframe=red!75!black, title=가정 위반의 결과: 부트스트래핑의 승리]
\begin{itemize}
    \item \textbf{방법 1 (부트스트래핑 결과)}:
        \begin{itemize}
            \item CI: $[0.487, 0.705]$
            \item 구간의 너비: $0.705 - 0.487 = 0.218$
        \end{itemize}
    \item \textbf{방법 2 (공식 기반 \texttt{statsmodels} 결과)}:
        \begin{itemize}
            \item CI: $[0.544, 0.636]$
            \item 구간의 너비: $0.636 - 0.544 = 0.092$
        \end{itemize}
\end{itemize}

\textbf{결과 분석}:
\begin{enumerate}
    \item 공식 기반의 신뢰 구간이 부트스트래핑 신뢰 구간보다 훨씬 \textbf{좁습니다(narrower)}.
    \item 이는 공식 기반 방법이 "우리의 추정치가 매우 정확하다"고 \textbf{지나치게 낙관적인(overly optimistic)} 결론을 내렸음을 의미합니다.
    \item \textbf{왜 이런 일이 발생했는가?}
        \begin{itemize}
            \item 공식 기반 방법은 선형 회귀의 \textbf{모든 가정이 완벽하게 충족될 때}만 정확합니다.
            \item 하지만 우리는 EDA 과정에서 \textbf{이분산성(Heteroscedasticity)}을 발견했습니다. 이는 '오차의 분산이 일정하다'는 핵심 가정을 위반한 것입니다.
            \item 가정이 깨졌기 때문에, 공식을 통해 계산된 표준 오차($\hat{SE}$) 값이 \textbf{잘못(너무 작게)} 계산된 것입니다.
        \end{itemize}
    \item \textbf{최종 결론}:
        \begin{itemize}
            \item 부트스트래핑은 데이터의 실제 분포(이분산성 포함)를 그대로 사용하여 추정치의 불확실성을 계산합니다.
            \item 따라서 회귀 모델의 가정이 위반되었을 때, \textbf{부트스트래핑이 공식 기반 방법보다 더 정직하고 신뢰할 수 있는(reliable) 불확실성 추정치를 제공}합니다.
        \end{itemize}
\end{enumerate}
\end{tcolorbox}


\newpage
\part{빠르게 훑어보기 (1페이지 요약)}
\label{part:summary}

\begin{tcolorbox}[colback=yellow!5!white, colframe=yellow!75!black, title=1. 핵심 연결: OLS = MLE (가장 중요한 결론)]
\textbf{최소제곱법(OLS)}을 사용하여 \textbf{MSE(오차제곱합)}를 최소화하는 것은,
만약 \textbf{"잔차가 정규 분포를 따른다"}고 가정한다면,
확률론적인 \textbf{최대가능도추정(MLE)}을 수행하는 것과
\textbf{수학적으로 완벽하게 동일}합니다.

이것이 우리가 회귀 모델의 손실 함수로 MSE를 사용하는 강력한 이론적 근거입니다.
\end{tcolorbox}

\begin{tcolorbox}[colback=blue!5!white, colframe=blue!75!black, title=2. PMF vs. PDF (핵심 구분)]
\begin{itemize}
    \item \textbf{PMF (확률 질량 함수)}: \textbf{이산형} (셀 수 있음, 예: 방 개수).
    \begin{itemize}
        \item $P(X=3)$ $\to$ "방 3개일 확률". \textbf{확률(Mass)}을 가짐.
        \item (시각화: 막대 그래프)
    \end{itemize}
    \item \textbf{PDF (확률 밀도 함수)}: \textbf{연속형} (셀 수 없음, 예: 집 면적).
    \begin{itemize}
        \item $P(X=1500.0) = 0$ $\to$ "정확히 1500.0 sqft일 확률은 0".
        \item 확률은 항상 \textbf{구간(면적)}으로 계산됨. (예: $P(1500 < X < 1501)$)
        \item (시각화: 부드러운 곡선)
    \end{itemize}
\end{itemize}
\end{tcolorbox}

\begin{tcolorbox}[colback=gray!10!white, colframe=black, title=3. 중심극한정리 (CLT)]
\textbf{왜 정규 분포(종 모양)가 중요한가?}

모집단이 주사위처럼 생긴 분포(Uniform)이든, 삐딱한 분포(Skewed)이든 상관없이,
거기서 뽑은 \textbf{샘플의 평균($\bar{X}$)}은
$n$이 충분히 크다면 \textbf{무조건 정규 분포}를 따릅니다.
\end{tcolorbox}

\begin{tcolorbox}[colback=gray!10!white, colframe=black, title=4. 다중공선성 (Collinearity) 함정]
\textbf{"침실 수(\texttt{beds})의 계수가 음수(-89.9)가 나왔습니다. 오류인가요?"}

아닙니다. 이는 \texttt{beds}와 \texttt{sqft}가 강하게 연관(collinear)되어 있기 때문입니다.
\textbf{해석}: "다른 변수, 특히 \textbf{총 면적(\texttt{sqft})을 고정한 채로} 침실 수만 1개 늘리면 (즉, 방을 억지로 쪼개면) 집값이 \$89.9k 하락한다"는 합리적인 의미입니다.
\end{tcolorbox}

\begin{tcolorbox}[colback=red!5!white, colframe=red!75!black, title=5. 이분산성 (Heteroscedasticity)과 추론]
\textbf{"모델의 신뢰 구간(CI)은 어떤 방법을 믿어야 하나요?"}

\begin{itemize}
    \item \textbf{공식 기반 CI} (t-검정): $[0.544, 0.636]$ (좁음 $\to$ 지나치게 낙관적)
    \item \textbf{부트스트랩 CI} (Resampling): $[0.487, 0.705]$ (넓음 $\to$ 더 정직함)
\end{itemize}

\textbf{결론}: 우리 데이터는 '이분산성' (집이 클수록 오차가 커짐)을 보였고, 이는 공식 기반 방법의 가정을 위배합니다.
따라서 가정이 깨졌을 때는 \textbf{부트스트래핑}이 더 신뢰할 수 있는 불확실성 추정 방법입니다.
\end{tcolorbox}

\end{document}
