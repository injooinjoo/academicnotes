\documentclass[a4paper,12pt]{article}
\usepackage{kotex}
\usepackage{amsmath, amssymb, amsthm}
\usepackage{geometry}
\usepackage{graphicx}
\usepackage{adjustbox}  % 표/박스 크기 조절
\usepackage{xcolor}
\usepackage[most]{tcolorbox}
\tcbuselibrary{breakable}
\usepackage{hyperref}
\usepackage{enumitem}
\usepackage{booktabs}
\usepackage{tabularx}
\usepackage{fancyhdr}
\usepackage{bm}

% 페이지 설정
\geometry{left=25mm, right=25mm, top=30mm, bottom=30mm}
\pagestyle{fancy}
\fancyhead[L]{MIT 6.419x Module 4}
\fancyhead[R]{Causal Inference}

% 색상 정의
\definecolor{mainblue}{RGB}{0, 51, 102}
\definecolor{subblue}{RGB}{230, 240, 255}
\definecolor{warningred}{RGB}{204, 0, 0}
\definecolor{conceptgreen}{RGB}{0, 102, 51}
\definecolor{storypurple}{RGB}{102, 0, 102}

% 박스 스타일 정의
\newtcolorbox{summarybox}[1]{
  colback=subblue, colframe=mainblue, 
  title=\textbf{#1}, fonttitle=\bfseries,
  boxrule=0.5mm, arc=2mm
}

\newtcolorbox{warningbox}[1]{
  colback=white, colframe=warningred, 
  title=\textbf{⚠️ #1}, fonttitle=\bfseries,
  boxrule=0.5mm, arc=0mm,
  coltitle=white
}

\newtcolorbox{conceptbox}[1]{
  colback=white, colframe=conceptgreen, 
  title=\textbf{💡 #1}, fonttitle=\bfseries,
  boxrule=0.5mm, arc=2mm,
  coltitle=white
}

\newtcolorbox{storybox}[1]{
  colback=white, colframe=storypurple, 
  title=\textbf{🎬 #1}, fonttitle=\bfseries,
  boxrule=0.5mm, arc=2mm,
  coltitle=white
}

\title{\textbf{MIT 6.419x: 데이터 과학의 성배}}
\author{Module 4: Causal Inference (인과 추론)}
\date{}

\begin{document}

\maketitle

% 1. 전체 목차 (TOC)
\tableofcontents
\vspace{1cm}
\hrule
\vspace{1cm}

\section*{Course Structure \& Current Focus}
\begin{itemize}
    \item Module 1: High-Dimensional Data (데이터 보기)
    \item Module 2: Clustering \& Networks (구조 찾기)
    \item Module 3: Time Series \& Spatial Data (시공간 분석)
    \item \textbf{\textcolor{mainblue}{Module 4: Causal Inference (현재 단원: 원인 찾기)}}
    \begin{itemize}
        \item 4.1 Correlation vs Causation (관찰 vs 개입)
        \item 4.2 Potential Outcomes Framework (반사실의 세계)
        \item 4.3 ATE \& Selection Bias (평균 처치 효과)
        \item 4.4 Causal Diagrams \& Matching (심화 예고)
    \end{itemize}
\end{itemize}

\newpage

% 2. 현재 단원 제목
\section{Module 4. 인과 추론 (Causal Inference)}

% 3. 이전 단원과의 연결
\begin{quote}
\textit{지금까지 우리는 "A가 변할 때 B도 변하더라(상관관계)"는 패턴을 기가 막히게 찾아냈습니다. 하지만 비즈니스 현장에서는 이것만으로는 부족합니다. 우리가 알고 싶은 건 \textbf{"그래서 내가 A를 바꾸면 B가 바뀔까?(인과관계)"}입니다. 이번 단원에서는 데이터 과학의 가장 높은 벽인 '인과의 세계'를 넘습니다.}
\end{quote}

% 4. 개요
\subsection*{📌 개요 (Overview)}
인과 추론은 "내가 이 행동(Treatment)을 하지 않았더라면 결과가 어땠을까?"라는 \textbf{반사실(Counterfactual)}적 질문을 수학적으로 정의하는 과정입니다. 상관관계와 인과관계의 명확한 차이, 루빈의 잠재적 결과 프레임워크, 그리고 실제 비즈니스 효과(ATE)를 측정할 때 발생하는 선택 편향(Selection Bias)의 위험성을 다룹니다.

% 5. 용어 정리 표
\subsection*{📝 핵심 용어 사전}
\begin{table}[h]
\centering
\begin{tabularx}{\textwidth}{|p{0.28\textwidth}|X|}
\hline
\textbf{용어 (Term)} & \textbf{직관적 의미 (Meaning)} \\
\hline
\textbf{Treatment ($W$)} & 원인이 되는 행위나 개입 (약물 투여, 쿠폰 발송). \\
\hline
\textbf{Outcome ($Y$)} & 처치로 인해 나타난 결과 (혈압, 구매액). \\
\hline
\textbf{Confounder (교란변수)} & 원인과 결과 모두에게 영향을 미쳐 착시를 일으키는 제3의 변수. \\
\hline
\textbf{Counterfactual (반사실)} & "만약 그 약을 안 먹었더라면..." (현실에선 일어나지 않은 가정). \\
\hline
\textbf{ATE} & "이 정책을 도입하면 전체적으로 평균 얼마의 이득이 있는가?" \\
\hline
\end{tabularx}
\end{table}

\vspace{0.5cm}\hrule\vspace{0.5cm}

% 6. 핵심 개념 상세 설명
\subsection{1. 상관관계 vs 인과관계 (Correlation vs Causation)}

\begin{conceptbox}{개념 1: 보기만 하는 것 vs 행동하는 것}
\textbf{한 줄 요약:} 비가 올 때 우산이 펴지는 것을 본다고 해서(상관), 내가 우산을 편다고 비가 오게 만들 수는 없습니다(인과 X).
\end{conceptbox}

\subsubsection*{1) 상관관계 (Correlation)}
\begin{itemize}
    \item \textbf{관찰(Observation)의 영역.} $P(Y | X)$.
    \item 예시: "아이스크림 판매량($X$)이 늘면 상어 습격($Y$)도 는다."
    \item 통계적으로 두 변수는 같이 움직입니다. 하지만 아이스크림을 금지한다고 상어가 사라지진 않습니다.
    \item \textbf{이유:} \textbf{기온(Summer)}이라는 \textbf{교란 변수(Confounder)}가 둘 다를 증가시켰기 때문입니다.
\end{itemize}

\subsubsection*{2) 인과관계 (Causation)}
\begin{itemize}
    \item \textbf{개입(Intervention)의 영역.} $P(Y | \text{do}(X))$. (Judea Pearl의 do-calculus)
    \item 의미: "내가 강제로 $X$를 변화시켰을 때, $Y$가 변하는가?"
    \item 데이터 과학의 목표: 우리는 단순히 예측(Prediction)을 넘어, 세상을 바꾸는 개입(Intervention)의 효과를 알고 싶습니다.
\end{itemize}

\vspace{0.5cm}\hrule\vspace{0.5cm}

\subsection{2. 잠재적 결과 프레임워크 (Potential Outcomes Framework)}

\begin{conceptbox}{개념 2: 가지 않은 길 (The Road Not Taken)}
\textbf{한 줄 요약:} 인과 효과를 안다는 것은 '평행우주'의 나를 훔쳐보는 것과 같습니다. "약을 먹은 나"와 "안 먹은 나"를 동시에 비교해야 합니다.
\end{conceptbox}

\subsubsection*{1) 루빈 인과 모델 (Rubin Causal Model)}
어떤 대상 $i$에 대해 두 가지 잠재적 결과가 존재합니다.
\begin{itemize}
    \item $Y_i(1)$: 처치($W=1$)를 받았을 때의 결과.
    \item $Y_i(0)$: 처치($W=0$)를 받지 않았을 때의 결과.
    \item \textbf{인과 효과 (Causal Effect):} $\tau_i = Y_i(1) - Y_i(0)$
\end{itemize}

\subsubsection*{2) 인과 추론의 근본 문제 (Fundamental Problem)}
현실 세계에서 우리는 한 사람에 대해 \textbf{둘 중 하나만 관측}할 수 있습니다.
$$ Y_i^{obs} = W_i Y_i(1) + (1-W_i) Y_i(0) $$
약을 먹었다면 $Y_i(1)$은 보이지만, $Y_i(0)$은 영원히 알 수 없는 \textbf{반사실(Counterfactual)}이 됩니다. 이것은 본질적으로 \textbf{결측치(Missing Data) 문제}입니다.

\vspace{0.5cm}\hrule\vspace{0.5cm}

\subsection{3. 처치 효과 (ATE)와 선택 편향}

\begin{conceptbox}{개념 3: 비교할 수 없는 것을 비교하지 마라}
\textbf{한 줄 요약:} 병원 간 사람은 아프고, 안 간 사람은 건강합니다. 단순히 병원 간 사람의 사망률이 높다고 해서 "병원이 사람을 죽인다"고 할 수 없습니다. 이것이 선택 편향입니다.
\end{conceptbox}

\subsubsection*{1) 평균 처치 효과 (ATE, Average Treatment Effect)}
개개인의 인과 효과는 알 수 없으니, 집단 전체의 평균을 봅니다.
$$ \text{ATE} = \mathbb{E}[Y(1) - Y(0)] = \mathbb{E}[Y(1)] - \mathbb{E}[Y(0)] $$

\subsubsection*{2) 단순 비교의 함정 (Naive Comparison)}
우리가 데이터에서 계산하는 단순 차이는 다음과 같이 분해됩니다.
$$ \underbrace{\mathbb{E}[Y|W=1] - \mathbb{E}[Y|W=0]}_{\text{관측된 차이}} = \underbrace{\text{ATE}}_{\text{진짜 효과}} + \underbrace{(\mathbb{E}[Y(0)|W=1] - \mathbb{E}[Y(0)|W=0])}_{\text{선택 편향 (Selection Bias)}} $$
\begin{itemize}
    \item \textbf{선택 편향:} 처치를 받은 그룹($W=1$)과 안 받은 그룹($W=0$)은 애초에 기본 상태($Y(0)$)가 다릅니다.
    \item \textbf{예:} 멤버십 가입자($W=1$)는 가입 안 해도($Y(0)$) 원래 구매력이 높은 사람들($W=0$보다 큼)일 가능성이 높습니다. 따라서 관측된 매출 차이를 온전히 멤버십 효과라고 보면 과대평가하게 됩니다.
\end{itemize}

\vspace{0.5cm}\hrule\vspace{0.5cm}

\section{실전 시나리오: 넥슨 멤버십 효과 분석}

\begin{storybox}{Scenario: 월정액 상품이 진짜 돈을 벌어줄까?}
당신은 '던전앤파이터'의 PM입니다. 월 1만 원짜리 '아라드 패스(멤버십)'를 출시했습니다. 패스 구매자의 월평균 결제액은 10만 원, 비구매자는 3만 원입니다. 패스의 효과가 +7만 원이라고 보고하면 될까요?
\end{storybox}

\begin{enumerate}
    \item \textbf{단순 비교 (Correlation):} $10\text{만} - 3\text{만} = +7\text{만}$.
    \item \textbf{의심 (Selection Bias):} "패스를 산 사람들은 원래 게임을 열심히 하는 '핵과금러'들이잖아. 패스가 없었어도 9만 원은 썼을걸?"
    \item \textbf{잠재적 결과 추정:}
        \begin{itemize}
            \item $Y(1)$: 패스 구매 시 결제액 (관측됨: 10만)
            \item $Y(0)$: 패스 미구매 시 결제액 (\textbf{반사실}: 9만 원으로 추정)
        \end{itemize}
    \item \textbf{인과 효과 (ATE):} $10\text{만} - 9\text{만} = +1\text{만}$.
    \item \textbf{결론:} 실제 증분 매출(Incrementality)은 7만 원이 아니라 1만 원입니다. 단순 비교로 보고했다면 성과를 7배 뻥튀기하는 치명적인 실수를 범했을 것입니다.
\end{enumerate}

\vspace{0.5cm}\hrule\vspace{0.5cm}

\section{자주 묻는 질문 (FAQ)}

\begin{description}
    \item[Q1. 그럼 선택 편향을 어떻게 없애나요?]
    \textbf{A.} 가장 확실한 방법은 \textbf{A/B 테스트(RCT, 무작위 대조 실험)}입니다. 동전을 던져서 강제로 처치 그룹과 통제 그룹을 나누면, 두 그룹의 성향이 통계적으로 같아져서 선택 편향이 0이 됩니다.
    
    \item[Q2. A/B 테스트를 못 하는 상황이면요?]
    \textbf{A.} 관측 데이터만으로 인과를 추론해야 합니다. 이때는 \textbf{성향 점수 매칭(Propensity Score Matching)}이나 \textbf{도구 변수(Instrumental Variables)} 같은 고급 통계 기법을 사용하여, 최대한 비슷한 사람끼리 비교하도록 데이터를 보정합니다. (다음 시간에 배울 내용입니다.)
\end{description}

% 10. 다음 단원 연결
\vspace{1cm}
\begin{quote}
\textbf{Next Step:} 인과 추론의 개념을 잡았습니다. 하지만 현실은 A가 B를, B가 C를, C가 다시 A를 건드리는 복잡한 \textbf{인과 사슬}로 얽혀 있습니다. 다음 시간에는 이런 복잡한 관계를 그림으로 풀어내는 \textbf{인과 그래프(Causal Diagrams)}와, 실험 없이도 인과를 찾아내는 관찰 연구 기법들을 배웁니다.
\end{quote}

% 11. 단원 요약 박스
\begin{summarybox}{Module 4 핵심 요약}
\begin{itemize}
    \item \textbf{상관 $\neq$ 인과:} 같이 움직인다고 해서 원인과 결과는 아니다. (교란변수 주의)
    \item \textbf{잠재적 결과:} 인과 효과 = (약 먹었을 때의 나) - (약 안 먹었을 때의 나). 하나는 볼 수 없으므로 추론해야 한다.
    \item \textbf{ATE:} 집단 전체의 평균적인 인과 효과. 비즈니스 임팩트의 핵심 지표.
    \item \textbf{선택 편향:} "원래 그런 애들이 선택했다." 단순 비교는 효과를 왜곡한다.
\end{itemize}
\end{summarybox}

\end{document}