\documentclass[a4paper,12pt]{article}
\usepackage{kotex}
\usepackage{amsmath, amssymb, amsthm}
\usepackage{geometry}
\usepackage{graphicx}
\usepackage{adjustbox}  % 표/박스 크기 조절
\usepackage{xcolor}
\usepackage[most]{tcolorbox}
\tcbuselibrary{breakable}
\usepackage{hyperref}
\usepackage{enumitem}
\usepackage{booktabs}
\usepackage{tabularx}
\usepackage{fancyhdr}
\usepackage{bm}

% 페이지 설정
\geometry{left=25mm, right=25mm, top=30mm, bottom=30mm}
\pagestyle{fancy}
\fancyhead[L]{MIT 6.419x Module 4 (Part B)}
\fancyhead[R]{Observational Studies}

% 색상 정의
\definecolor{mainblue}{RGB}{0, 51, 102}
\definecolor{subblue}{RGB}{230, 240, 255}
\definecolor{warningred}{RGB}{204, 0, 0}
\definecolor{conceptgreen}{RGB}{0, 102, 51}
\definecolor{storypurple}{RGB}{102, 0, 102}

% 박스 스타일 정의
\newtcolorbox{summarybox}[1]{
  colback=subblue, colframe=mainblue, 
  title=\textbf{#1}, fonttitle=\bfseries,
  boxrule=0.5mm, arc=2mm
}

\newtcolorbox{warningbox}[1]{
  colback=white, colframe=warningred, 
  title=\textbf{⚠️ #1}, fonttitle=\bfseries,
  boxrule=0.5mm, arc=0mm,
  coltitle=white
}

\newtcolorbox{conceptbox}[1]{
  colback=white, colframe=conceptgreen, 
  title=\textbf{💡 #1}, fonttitle=\bfseries,
  boxrule=0.5mm, arc=2mm,
  coltitle=white
}

\newtcolorbox{storybox}[1]{
  colback=white, colframe=storypurple, 
  title=\textbf{🎬 #1}, fonttitle=\bfseries,
  boxrule=0.5mm, arc=2mm,
  coltitle=white
}

\title{\textbf{MIT 6.419x: 실험 없이 원인을 찾다}}
\author{Module 4 (Part B): Causal Inference in Observational Studies}
\date{}

\begin{document}

\maketitle

% 1. 전체 목차 (TOC)
\tableofcontents
\vspace{1cm}
\hrule
\vspace{1cm}

\section*{Course Structure \& Current Focus}
\begin{itemize}
    \item Module 4 (Part A): Causal Basics (ATE, Selection Bias)
    \item \textbf{\textcolor{mainblue}{Module 4 (Part B): Observational Studies (현재 단원: 관찰 연구 기법)}}
    \begin{itemize}
        \item 10.1 Confounders \& Simpson's Paradox (함정 피하기)
        \item 10.2 Matching \& Propensity Score (비슷한 것끼리 묶기)
        \item 10.3 Instrumental Variables (도구 변수 활용)
        \item 10.4 Regression Discontinuity (경계선 활용)
    \end{itemize}
    \item Module 5: Deep Learning \& Applications (심화 응용)
\end{itemize}

\newpage

% 2. 현재 단원 제목
\section{Unit 10. 관찰 연구에서의 인과 추론}

% 3. 이전 단원과의 연결
\begin{quote}
\textit{Part A에서 우리는 "A/B 테스트(RCT)가 최고다"라고 배웠습니다. 하지만 현실에서는 담배의 해로움을 알기 위해 강제로 담배를 피우게 할 수 없고, 넥슨의 모든 유저에게 강제로 캐시를 지급할 수도 없습니다. 실험이 불가능한 상황에서, 이미 쌓여있는 로그 데이터(Observational Data)만으로 인과관계를 밝혀내는 고급 기술들을 배웁니다.}
\end{quote}

% 4. 개요
\subsection*{📌 개요 (Overview)}
이 단원에서는 관찰 데이터에 숨어있는 \textbf{교란 변수(Confounder)}의 위험성을 심슨의 역설을 통해 이해하고, 이를 통제하기 위한 3대장 기법을 배웁니다.
1. \textbf{성향 점수 매칭 (PSM):} 비슷한 사람끼리 비교하기.
2. \textbf{도구 변수 (IV):} 외부 충격을 이용하기.
3. \textbf{회귀 불연속 (RD):} 컷오프 경계선의 마법 활용하기.

% 5. 용어 정리 표
\subsection*{📝 핵심 용어 사전}
\begin{table}[h]
\centering
\begin{tabularx}{\textwidth}{|p{0.28\textwidth}|X|}
\hline
\textbf{용어 (Term)} & \textbf{직관적 의미 (Meaning)} \\
\hline
\textbf{Observational Study} & 연구자가 개입하지 않고, 자연 발생한 데이터를 관찰하는 연구. \\
\hline
\textbf{Confounder (교란요인)} & 원인과 결과 뒤에서 둘 다 조종하는 흑막. ($Z \to X, Z \to Y$) \\
\hline
\textbf{Simpson's Paradox} & 쪼개서 볼 때와 합쳐서 볼 때 결과가 정반대가 되는 기이한 현상. \\
\hline
\textbf{Propensity Score} & "이 사람이 처치(약/쿠폰)를 받을 확률". 복잡한 특성을 점수 하나로 요약. \\
\hline
\textbf{Instrumental Variable} & 원인($X$)에만 영향을 주고 결과($Y$)에는 직접 영향을 안 주는 제3의 변수. \\
\hline
\end{tabularx}
\end{table}

\vspace{0.5cm}\hrule\vspace{0.5cm}

% 6. 핵심 개념 상세 설명
\subsection{1. 교란 변수와 심슨의 역설}


\begin{conceptbox}{개념 1: 통계의 착시 현상}
\textbf{한 줄 요약:} 데이터를 뭉뚱그려(Aggregation) 보면 진실이 가려집니다. 쪼개서(Stratification) 봐야 합니다.
\end{conceptbox}

\subsubsection*{심슨의 역설 (Simpson's Paradox)}
\begin{itemize}
    \item \textbf{상황:} 신약 A의 완치율이 B보다 높습니다. (전체 통계)
    \item \textbf{진실:} 남성 그룹에서도 B가 높고, 여성 그룹에서도 B가 높습니다. (???)
    \item \textbf{원인:} A약은 주로 병이 가벼운 사람들에게 투여되었고, B약은 중환자들에게 투여되었기 때문입니다. \textbf{'환자의 중증도'}라는 교란 변수를 무시하고 합쳤기 때문에 발생한 착시입니다.
    \item \textbf{교훈:} 인과 추론의 제1원칙은 "비교 가능한 것끼리 비교하라(Ceteris Paribus)"입니다.
\end{itemize}

\vspace{0.5cm}\hrule\vspace{0.5cm}

\subsection{2. 성향 점수 매칭 (Propensity Score Matching, PSM)}


\begin{conceptbox}{개념 2: 도플갱어 찾기}
\textbf{한 줄 요약:} 약을 먹은 사람과, 안 먹었지만 먹을 뻔했던(성향이 비슷한) 사람을 짝지어서 비교합니다.
\end{conceptbox}

\subsubsection*{1) 원리}
변수가 많으면(나이, 성별, 소득, 지역...) 완벽하게 똑같은 짝을 찾기 힘듭니다. 그래서 모든 특성을 \textbf{'처치 받을 확률(Propensity Score)'}이라는 하나의 숫자로 압축합니다.
$$ e(x) = P(T=1 | X=x) $$

\subsubsection*{2) 매칭 (Matching)}
\begin{enumerate}
    \item 각 유저가 쿠폰을 받을 확률(성향 점수)을 계산합니다.
    \item 실제 쿠폰 받은 사람(T=1, 점수 0.7)과 안 받은 사람(T=0, 점수 0.7)을 매칭합니다.
    \item 이 둘의 구매액 차이를 계산하면 순수한 쿠폰 효과를 알 수 있습니다.
\end{enumerate}

\vspace{0.5cm}\hrule\vspace{0.5cm}

\subsection{3. 도구 변수 (Instrumental Variables, IV)}


\begin{conceptbox}{개념 3: 자연 실험(Natural Experiment) 이용하기}
\textbf{한 줄 요약:} 우리가 직접 실험할 수 없을 때, 우연히 발생한 외부 충격(도구)을 이용해 인과관계를 발라냅니다.
\end{conceptbox}

\subsubsection*{1) 도구 변수의 조건}
도구 변수 $Z$는 $X$(원인)는 건드리지만, $Y$(결과)는 직접 건드리지 않아야 합니다. ($Z \to X \to Y$)
\begin{itemize}
    \item \textbf{예시:} "군 복무($X$)가 평생 소득($Y$)에 미치는 영향"
    \item \textbf{문제:} 애국심이나 건강 상태 같은 교란 변수가 섞여 있습니다.
    \item \textbf{도구($Z$):} \textbf{생일 추첨(징병제 복권)}. 생일은 랜덤이므로 개인 특성과 무관하며, 오직 군대 가는 여부($X$)에만 영향을 줍니다.
\end{itemize}

\subsubsection*{2) 2단계 최소자승법 (2SLS)}
\begin{itemize}
    \item \textbf{Stage 1:} $Z$로 $X$를 예측합니다. ($\hat{X}$ 생성) $\rightarrow$ $X$에서 교란 변수의 때를 벗겨냄.
    \item \textbf{Stage 2:} 깨끗해진 $\hat{X}$로 $Y$를 설명합니다.
\end{itemize}

\vspace{0.5cm}\hrule\vspace{0.5cm}

\subsection{4. 회귀 불연속 (Regression Discontinuity, RD)}


\begin{conceptbox}{개념 4: 깻잎 한 장 차이}
\textbf{한 줄 요약:} 커트라인(90점) 바로 위(90점)와 바로 아래(89점) 사람은 실력 차이가 거의 없습니다. 운명만 갈렸을 뿐이죠. 이들을 비교합니다.
\end{conceptbox}

\subsubsection*{1) 원리}
임의의 컷오프(Threshold) 주변에서는 마치 \textbf{무작위 할당(Random Assignment)}이 일어난 것과 같다고 가정합니다.
\begin{itemize}
    \item \textbf{예:} 장학금 기준이 토익 800점.
    \item 799점(탈락)과 800점(수혜) 학생을 비교하면 장학금의 인과 효과를 측정할 수 있습니다.
\end{itemize}

\subsubsection*{2) 시각적 확인}
$X$축(점수)에 따른 $Y$축(결과) 그래프를 그렸을 때, 컷오프 지점에서 그래프가 \textbf{'점프(Jump)'} 한다면 그것이 바로 인과 효과입니다.

\vspace{0.5cm}\hrule\vspace{0.5cm}

\section{실전 시나리오: 넥슨 PC방 혜택 분석}

\begin{storybox}{Scenario: PC방 혜택이 진짜 접속 시간을 늘릴까?}
당신은 넥슨의 사업 PM입니다. "PC방에서 접속하면 경험치 +20\%" 혜택을 줍니다. PC방 유저의 플레이 시간이 집 유저보다 2배 깁니다. 이게 혜택 덕분일까요?
\end{storybox}

\begin{enumerate}
    \item \textbf{문제 (Selection Bias):} 원래 게임을 오래 하는 '하드코어 유저'들이 PC방을 더 많이 갑니다. (단순 비교 불가)
    
    \item \textbf{해결책 1 (PSM):} 집 유저 중에서도 하드코어한 성향(레벨, 장비 수준 등)이 PC방 유저와 비슷한 사람들을 찾아 매칭하여 비교합니다.
    
    \item \textbf{해결책 2 (IV - 도구 변수):} \textbf{"비 오는 날($Z$)"}을 도구로 씁니다.
    \begin{itemize}
        \item 비가 오면 귀찮아서 PC방을 덜 갑니다 ($Z \to X$).
        \item 비 자체가 게임 실력($Y$)에 영향을 주진 않습니다.
        \item 비 때문에 '강제로' 집에서 하게 된 유저들의 플레이 시간 변화를 분석하여 순수한 PC방 효과를 추정합니다.
    \end{itemize}
\end{enumerate}

\vspace{0.5cm}\hrule\vspace{0.5cm}

\section{자주 묻는 질문 (FAQ)}

\begin{description}
    \item[Q1. 셋 중 뭐가 제일 좋나요?]
    \textbf{A.} 상황에 따라 다릅니다.
    \begin{itemize}
        \item \textbf{RD (회귀 불연속):} 가장 신뢰도가 높지만(거의 RCT급), 컷오프 규칙이 있는 경우에만 쓸 수 있습니다.
        \item \textbf{IV (도구 변수):} 강력하지만, 완벽한 도구 변수($Z$)를 찾는 것이 하늘의 별 따기입니다. (대부분 논문감)
        \item \textbf{PSM (매칭):} 가장 범용적이고 쉽지만, "우리가 관찰하지 못한 변수(Unobserved Confounder)"는 통제 못 한다는 약점이 있습니다.
    \end{itemize}
    
    \item[Q2. 머신러닝이랑 인과 추론을 섞어 쓸 수 있나요?]
    \textbf{A.} 네, 최신 트렌드인 \textbf{Causal ML}입니다. 'Double Machine Learning' 같은 기법은 성향 점수나 결과 예측 모델에 딥러닝(XGBoost 등)을 사용하여 추정의 정확도를 높입니다. (주로 Moloco 같은 애드테크에서 많이 씁니다.)
\end{description}

% 10. 다음 단원 연결
\vspace{1cm}
\begin{quote}
\textbf{Next Step:} 드디어 MIT 6.419x의 긴 여정이 마무리되었습니다. 고차원 데이터 시각화부터, 네트워크 분석, 시공간 통계, 그리고 인과 추론까지. 이제 이 모든 지식을 종합하여 넥슨의 데이터를 바라보면, 전에는 보이지 않던 \textbf{'구조'와 '원인'}이 보이기 시작할 것입니다. 고생 많으셨습니다!
\end{quote}

% 11. 단원 요약 박스
\begin{summarybox}{Module 4 (Part B) 핵심 요약}
\begin{itemize}
    \item \textbf{심슨의 역설:} 데이터를 뭉뚱그려 보면 인과관계가 뒤집힐 수 있다. 쪼개 봐야 한다.
    \item \textbf{PSM:} 성향이 비슷한 사람끼리 짝지어(Matching) 비교한다.
    \item \textbf{IV:} 외부 충격(도구)을 이용해 자연 실험 상황을 만든다.
    \item \textbf{RD:} 컷오프 경계선에서의 불연속 점프를 인과 효과로 간주한다.
\end{itemize}
\end{summarybox}

\end{document}