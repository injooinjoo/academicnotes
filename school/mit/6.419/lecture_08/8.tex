\documentclass[a4paper,12pt]{article}
\usepackage{kotex}
\usepackage{amsmath, amssymb, amsthm}
\usepackage{geometry}
\usepackage{graphicx}
\usepackage{adjustbox}  % 표/박스 크기 조절
\usepackage{xcolor}
\usepackage[most]{tcolorbox}
\tcbuselibrary{breakable}
\usepackage{hyperref}
\usepackage{enumitem}
\usepackage{booktabs}
\usepackage{tabularx}
\usepackage{fancyhdr}
\usepackage{bm}

% 페이지 설정
\geometry{left=25mm, right=25mm, top=30mm, bottom=30mm}
\pagestyle{fancy}
\fancyhead[L]{MIT 6.419x Module 3 (Part B)}
\fancyhead[R]{Spatial Statistics}

% 색상 정의
\definecolor{mainblue}{RGB}{0, 51, 102}
\definecolor{subblue}{RGB}{230, 240, 255}
\definecolor{warningred}{RGB}{204, 0, 0}
\definecolor{conceptgreen}{RGB}{0, 102, 51}
\definecolor{storypurple}{RGB}{102, 0, 102}

% 박스 스타일 정의
\newtcolorbox{summarybox}[1]{
  colback=subblue, colframe=mainblue, 
  title=\textbf{#1}, fonttitle=\bfseries,
  boxrule=0.5mm, arc=2mm
}

\newtcolorbox{warningbox}[1]{
  colback=white, colframe=warningred, 
  title=\textbf{⚠️ #1}, fonttitle=\bfseries,
  boxrule=0.5mm, arc=0mm,
  coltitle=white
}

\newtcolorbox{conceptbox}[1]{
  colback=white, colframe=conceptgreen, 
  title=\textbf{💡 #1}, fonttitle=\bfseries,
  boxrule=0.5mm, arc=2mm,
  coltitle=white
}

\newtcolorbox{storybox}[1]{
  colback=white, colframe=storypurple, 
  title=\textbf{🎬 #1}, fonttitle=\bfseries,
  boxrule=0.5mm, arc=2mm,
  coltitle=white
}

\title{\textbf{MIT 6.419x: 지도의 빈칸을 채우다}}
\author{Module 3 (Part B): Spatial Statistics (GP \& Kriging)}
\date{}

\begin{document}

\maketitle

% 1. 전체 목차 (TOC)
\tableofcontents
\vspace{1cm}
\hrule
\vspace{1cm}

\section*{Course Structure \& Current Focus}
\begin{itemize}
    \item Module 3 (Part A): Time Series Analysis (시간의 흐름)
    \item \textbf{\textcolor{mainblue}{Module 3 (Part B): Spatial Statistics (현재 단원: 공간의 분포)}}
    \begin{itemize}
        \item 3.5 Gaussian Processes (함수의 확률 분포)
        \item 3.6 Kriging \& Variograms (지구통계학적 보간)
        \item 3.7 Kernel Methods (공간적 유사도 정의)
    \end{itemize}
\end{itemize}

\newpage

% 2. 현재 단원 제목
\section{Module 3 (Part B). 공간 통계 (Spatial Statistics)}

% 3. 이전 단원과의 연결
\begin{quote}
\textit{지난 시간(Part A)에는 "오늘의 나는 어제의 나"라는 시간적 자기상관(Autocorrelation)을 다뤘습니다. 이번에는 \textbf{"모든 것은 다른 모든 것과 관련이 있지만, 가까운 것이 더 관련이 있다(Tobler's First Law)"}는 지리학의 제1법칙을 바탕으로, 공간적 자기상관을 다룹니다.}
\end{quote}

% 4. 개요
\subsection*{📌 개요 (Overview)}
이 단원에서는 듬성듬성한 관측 데이터(예: 기상청 관측소)를 바탕으로 전체 공간(예: 대한민국 전체 미세먼지 지도)을 추정하는 방법을 배웁니다. 특히 \textbf{가우시안 프로세스(GP)}를 통해 예측의 불확실성을 정량화하고, 이를 지구통계학에 적용한 \textbf{크리깅(Kriging)} 기법을 마스터합니다.

% 5. 용어 정리 표
\subsection*{📝 핵심 용어 사전}
\begin{table}[h]
\centering
\begin{tabularx}{\textwidth}{|p{0.28\textwidth}|X|}
\hline
\textbf{용어 (Term)} & \textbf{직관적 의미 (Meaning)} \\
\hline
\textbf{Interpolation (보간)} & 점들 사이의 빈 공간을 부드럽게 채워 넣는 기술. \\
\hline
\textbf{Uncertainty} & "이 지역 값은 내가 확실히 아는데, 저기는 데이터가 없어서 잘 몰라(분산이 커)." \\
\hline
\textbf{Variogram} & 거리가 멀어질수록 데이터 간의 차이(분산)가 어떻게 커지는지 나타내는 그래프. \\
\hline
\textbf{Kernel ($k$)} & 두 지점이 '가깝다'는 것을 수학적으로 정의하는 함수. \\
\hline
\end{tabularx}
\end{table}

\vspace{0.5cm}\hrule\vspace{0.5cm}

% 6. 핵심 개념 상세 설명
\subsection{1. 가우시안 프로세스 (Gaussian Processes, GP)}


\begin{conceptbox}{개념 1: 함수의 확률 분포}
\textbf{한 줄 요약:} 점 하나하나를 예측하는 게 아니라, "데이터를 설명할 수 있는 모든 가능한 곡선(함수)들의 모임"을 그립니다.
\end{conceptbox}

\subsubsection*{1) 핵심 아이디어}
일반적인 회귀분석($y = wx + b$)은 파라미터 $w$를 찾지만, GP는 \textbf{함수 $f(x)$ 자체를 확률 변수}로 봅니다.
\begin{itemize}
    \item \textbf{관측된 곳:} 오차가 거의 없이 점을 지나가야 하므로 불확실성(분산)이 0에 가깝습니다.
    \item \textbf{관측 안 된 곳:} 어떤 값이든 될 수 있으므로 불확실성(분산)이 풍선처럼 부풀어 오릅니다.
\end{itemize}

\subsubsection*{2) 구성 요소}
$$ f(x) \sim GP(m(x), k(x, x')) $$
\begin{itemize}
    \item \textbf{평균 함수 ($m$):} 데이터의 전반적인 추세.
    \item \textbf{커널 함수 ($k$):} $x$와 $x'$가 가까우면 $f(x)$와 $f(x')$도 비슷한 값을 가질 확률이 높다는 '상관관계'를 정의합니다.
\end{itemize}

\vspace{0.5cm}\hrule\vspace{0.5cm}

\subsection{2. 크리깅 (Kriging)}


\begin{conceptbox}{개념 2: 지능적인 가중 평균}
\textbf{한 줄 요약:} 단순히 가까운 점을 평균 내는 게 아니라, 공간의 특성(배리오그램)을 분석해서 가장 오차가 적은 황금 비율(가중치)로 섞습니다.
\end{conceptbox}

\subsubsection*{1) 작동 원리 (BLUE)}
미지의 지점 $x_0$의 값은 주변 데이터 $Z(x_i)$들의 선형 결합입니다.
$$ \hat{Z}(x_0) = \sum_{i=1}^{n} \lambda_i Z(x_i) $$
여기서 가중치 $\lambda_i$를 구하는 것이 핵심인데, \textbf{배리오그램(Variogram)}을 이용해 오차 분산을 최소화하는 최적의 $\lambda$를 찾아냅니다.

\subsubsection*{2) 배리오그램 (Variogram)}
"거리가 $h$만큼 떨어져 있으면, 값은 얼마나 차이가 날까?"를 나타내는 함수입니다.
\begin{itemize}
    \item 가까운 거리: 차이가 적음 (상관관계 높음).
    \item 먼 거리: 차이가 큼 (상관관계 낮음).
    \item 크리깅은 이 구조를 파악해서, 멀리 있어도 상관관계가 유지되는 방향이라면 가중치를 더 줍니다.
\end{itemize}

\vspace{0.5cm}\hrule\vspace{0.5cm}

\subsection{3. 커널 방법론 (The Kernel Connection)}


\begin{conceptbox}{개념 3: 공간의 질감(Texture) 정의하기}
\textbf{한 줄 요약:} "이 지역은 부드러운 언덕인가(RBF), 아니면 거친 자갈밭인가(Matern)?"를 수학식(커널)으로 결정합니다.
\end{conceptbox}

\subsubsection*{주요 커널 종류}
\begin{enumerate}
    \item \textbf{RBF (Squared Exponential):}
    \begin{itemize}
        \item 아주 매끄러운(Smooth) 곡선을 만듭니다.
        \item 현실 세계의 거친 지형이나 급격한 변화를 설명하기엔 너무 이상적일 수 있습니다.
    \end{itemize}
    \item \textbf{Matern Kernel:}
    \begin{itemize}
        \item 거칠기(Roughness)를 조절하는 파라미터가 있습니다.
        \item 공간 통계에서 가장 널리 쓰이며, 현실적인 물리 현상을 잘 반영합니다.
    \end{itemize}
    \item \textbf{Periodic Kernel:}
    \begin{itemize}
        \item 일정한 패턴이 반복될 때 사용합니다 (예: 계절에 따른 온도 변화).
    \end{itemize}
\end{enumerate}

\vspace{0.5cm}\hrule\vspace{0.5cm}

\section{실전 시나리오: 넥슨 글로벌 서버 랙(Lag) 지도}

\begin{storybox}{Scenario: 전 세계 유저들의 네트워크 품질 추정}
당신은 넥슨 글로벌 서비스 본부장입니다. 전 세계 곳곳에 흩어진 유저들의 핑(Ping) 데이터를 이용해, 어느 지역에 서버를 증설해야 할지 결정해야 합니다.
\end{storybox}

\begin{enumerate}
    \item \textbf{데이터:} 일부 유저($n=10,000$)에게서 수집된 [위도, 경도, Ping] 데이터. 대부분의 지역은 데이터가 없는 공백지입니다.
    \item \textbf{Kriging 적용:}
    \begin{itemize}
        \item 단순히 근처 유저 값을 평균 내는 게 아니라, 대륙별 인터넷망의 특성(Variogram)을 고려해 보간합니다.
        \item 결과: 데이터가 없던 '동남아 시골 마을'의 핑도 주변 도시 데이터를 기반으로 정밀하게 추정해냅니다.
    \end{itemize}
    \item \textbf{Gaussian Process 활용 (핵심):}
    \begin{itemize}
        \item GP는 \textbf{불확실성(Variance) 지도}를 함께 줍니다.
        \item "아프리카 지역은 핑이 200ms로 예상되지만, 불확실성이 매우 높다($\pm 100$)."
        \item \textbf{의사결정:} 불확실성이 높은 지역은 섣불리 서버를 짓기보다, 먼저 테스트 서버를 열어 데이터를 더 수집하기로 결정합니다.
    \end{itemize}
\end{enumerate}

\vspace{0.5cm}\hrule\vspace{0.5cm}

\section{자주 묻는 질문 (FAQ)}

\begin{description}
    \item[Q1. 단순 평균(Nearest Neighbor)과 크리깅의 결정적 차이는?]
    \textbf{A.} \textbf{불확실성(Uncertainty)}과 \textbf{구조(Structure)}입니다.
    단순 평균은 "값"만 채우지만, 크리깅은 "이 값이 얼마나 믿을만한지(Variance)"를 알려줍니다. 또한, 크리깅은 데이터가 한쪽에 쏠려 있으면(Clustered), 그쪽 데이터들의 가중치를 낮춰서(De-clustering) 전체적인 균형을 맞춥니다.
    
    \item[Q2. 커널은 어떻게 고르나요?]
    \textbf{A.} 데이터에 대한 \textbf{도메인 지식}이 필요합니다. "현상은 부드럽게 변하는가, 급격히 변하는가?", "주기적인가?"를 고민해야 합니다. 보통은 Matern 커널로 시작해서 Cross-Validation으로 최적의 커널을 찾습니다.
\end{description}

% 10. 다음 단원 연결
\vspace{1cm}
\begin{quote}
\textbf{Next Step:} 지금까지 정형 데이터(행렬, 그래프, 시계열, 공간)를 다뤘습니다. 이제 마지막 모듈에서는 비정형 데이터의 끝판왕인 \textbf{이미지와 텍스트}를 다루기 위해 다시 딥러닝(CNN, NLP)의 심화 주제로 돌아가거나, 강화학습의 응용을 다루게 됩니다. (MIT 커리큘럼에 따라 유동적)
\end{quote}

% 11. 단원 요약 박스
\begin{summarybox}{Module 3 (Part B) 핵심 요약}
\begin{itemize}
    \item \textbf{Gaussian Processes:} 함수를 확률적으로 추정하여 예측값과 불확실성(범위)을 동시에 제공한다.
    \item \textbf{Kriging:} 공간적 상관관계(Variogram)를 분석하여 미관측 지점을 정밀하게 보간하는 지구통계학 기법.
    \item \textbf{Kernel:} 공간적 유사도를 정의하는 핵심 엔진. (RBF = Smooth, Matern = Realistic).
\end{itemize}
\end{summarybox}

\end{document}