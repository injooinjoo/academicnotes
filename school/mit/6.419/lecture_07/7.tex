\documentclass[a4paper,12pt]{article}
\usepackage{kotex}
\usepackage{amsmath, amssymb, amsthm}
\usepackage{geometry}
\usepackage{graphicx}
\usepackage{adjustbox}  % 표/박스 크기 조절
\usepackage{xcolor}
\usepackage[most]{tcolorbox}
\tcbuselibrary{breakable}
\usepackage{hyperref}
\usepackage{enumitem}
\usepackage{booktabs}
\usepackage{tabularx}
\usepackage{fancyhdr}
\usepackage{bm}

% 페이지 설정
\geometry{left=25mm, right=25mm, top=30mm, bottom=30mm}
\pagestyle{fancy}
\fancyhead[L]{MIT 6.419x Module 3}
\fancyhead[R]{Time Series Analysis}

% 색상 정의
\definecolor{mainblue}{RGB}{0, 51, 102}
\definecolor{subblue}{RGB}{230, 240, 255}
\definecolor{warningred}{RGB}{204, 0, 0}
\definecolor{conceptgreen}{RGB}{0, 102, 51}
\definecolor{storypurple}{RGB}{102, 0, 102}

% 박스 스타일 정의
\newtcolorbox{summarybox}[1]{
  colback=subblue, colframe=mainblue, 
  title=\textbf{#1}, fonttitle=\bfseries,
  boxrule=0.5mm, arc=2mm
}

\newtcolorbox{warningbox}[1]{
  colback=white, colframe=warningred, 
  title=\textbf{⚠️ #1}, fonttitle=\bfseries,
  boxrule=0.5mm, arc=0mm,
  coltitle=white
}

\newtcolorbox{conceptbox}[1]{
  colback=white, colframe=conceptgreen, 
  title=\textbf{💡 #1}, fonttitle=\bfseries,
  boxrule=0.5mm, arc=2mm,
  coltitle=white
}

\newtcolorbox{storybox}[1]{
  colback=white, colframe=storypurple, 
  title=\textbf{🎬 #1}, fonttitle=\bfseries,
  boxrule=0.5mm, arc=2mm,
  coltitle=white
}

\title{\textbf{MIT 6.419x: 시간의 패턴을 읽다}}
\author{Module 3 (Part A): Time Series Analysis}
\date{}

\begin{document}

\maketitle

% 1. 전체 목차 (TOC)
\tableofcontents
\vspace{1cm}
\hrule
\vspace{1cm}

\section*{Course Structure \& Current Focus}
\begin{itemize}
    \item Module 1 \& 2: Static Data (정적 데이터 분석)
    \item \textbf{\textcolor{mainblue}{Module 3: Time Series \& Spatial Data (현재 단원: 시공간 데이터)}}
    \begin{itemize}
        \item \textbf{3.1 Trend \& Seasonality (정상성 만들기)}
        \item \textbf{3.2 ARMA / ARIMA (통계적 예측 모델)}
        \item \textbf{3.3 Kalman Filter (상태 추정 및 노이즈 제거)}
        \item 3.4 Spatial Statistics (Kriging)
    \end{itemize}
\end{itemize}

\newpage

% 2. 현재 단원 제목
\section{Module 3 (Part A). 시계열 분석 (Time Series Analysis)}

% 3. 이전 단원과의 연결
\begin{quote}
\textit{지금까지 우리는 데이터의 순서를 신경 쓰지 않았습니다(i.i.d 가정). 하지만 주가, 게임 접속자 수, 서버 로그는 순서가 핵심입니다. 어제의 데이터가 오늘의 데이터에 영향을 미치는 \textbf{자기상관(Autocorrelation)}의 세계로 들어갑니다.}
\end{quote}

% 4. 개요
\subsection*{📌 개요 (Overview)}
이 단원에서는 시계열 데이터를 분석 가능한 형태로 가공하는 \textbf{분해(Decomposition)} 기법, 과거 데이터를 기반으로 미래를 예측하는 \textbf{ARIMA} 모델, 그리고 노이즈가 섞인 실시간 데이터에서 진짜 상태를 추정하는 \textbf{칼만 필터(Kalman Filter)}를 다룹니다.

% 5. 용어 정리 표
\subsection*{📝 핵심 용어 사전}
\begin{table}[h]
\centering
\begin{tabularx}{\textwidth}{|p{0.28\textwidth}|X|}
\hline
\textbf{용어 (Term)} & \textbf{직관적 의미 (Meaning)} \\
\hline
\textbf{Stationarity (정상성)} & 평균과 분산이 시간에 따라 변하지 않고 일정한 상태. (예측의 전제 조건) \\
\hline
\textbf{Seasonality (계절성)} & 주말마다 접속자가 늘어나는 것처럼 주기적인 패턴. \\
\hline
\textbf{Lag (시차)} & 현재 시점($t$)과 과거 시점($t-k$) 사이의 간격. \\
\hline
\textbf{White Noise} & 패턴이 전혀 없는 완전 무작위 잡음. (예측 불가능한 영역) \\
\hline
\textbf{State-Space Model} & 관측된 데이터 뒤에 숨겨진 '진짜 상태'가 있다고 가정하는 모델. \\
\hline
\end{tabularx}
\end{table}

\vspace{0.5cm}\hrule\vspace{0.5cm}

% 6. 핵심 개념 상세 설명
\subsection{1. 추세(Trend)와 계절성(Seasonality) 제거}


\begin{conceptbox}{개념 1: 양파 껍질 벗기기}
\textbf{한 줄 요약:} 데이터에서 '뻔한 패턴(상승세, 주말 효과)'을 걷어내야, 진짜 중요한 '변화(신호)'가 보입니다. 이를 위해 데이터를 분해합니다.
\end{conceptbox}

\subsubsection*{1) 시계열 분해 (Decomposition)}
$$ Y_t = \underbrace{T_t}_{\text{추세}} + \underbrace{S_t}_{\text{계절성}} + \underbrace{R_t}_{\text{잔차}} $$
우리는 $Y_t$에서 $T_t$와 $S_t$를 제거하여, \textbf{정상성(Stationarity)}을 가진 잔차 $R_t$만 남기고 싶어 합니다. 왜냐하면 통계적 모델은 "평균이 일정한 데이터"에서 가장 잘 작동하기 때문입니다.

\subsubsection*{2) 제거 방법}
\begin{itemize}
    \item \textbf{차분 (Differencing, $I$):} $Y_t - Y_{t-1}$. 오늘의 값에서 어제의 값을 뺍니다. 추세(기울기)를 제거하여 수평으로 만듭니다.
    \item \textbf{로그 변환:} 분산이 점점 커지는 경우(깔때기 모양), 로그를 씌워 변동 폭을 일정하게 맞춥니다.
    \item \textbf{계절 차분:} $Y_t - Y_{t-7}$ (일주일 주기인 경우). 이번 주 토요일 데이터에서 지난주 토요일 데이터를 뺍니다.
\end{itemize}

\vspace{0.5cm}\hrule\vspace{0.5cm}

\subsection{2. 자기회귀 이동평균 모델 (ARMA / ARIMA)}


\begin{conceptbox}{개념 2: 관성(Inertia)과 충격(Shock)}
\textbf{한 줄 요약:} "관성대로 가려는 성질(AR)"과 "외부 충격에 반응하는 성질(MA)"을 합쳐서 미래를 그립니다.
\end{conceptbox}

\subsubsection*{1) 구성 요소: ARIMA(p, d, q)}
\begin{itemize}
    \item \textbf{AR ($p$, AutoRegressive):} 자기회귀.
    $$ Y_t = \phi Y_{t-1} + \epsilon_t $$
    "어제 접속자가 많았으니, 관성에 의해 오늘도 많을 거야."
    \item \textbf{I ($d$, Integrated):} 차분.
    추세를 제거하기 위해 데이터를 몇 번 미분(뺄셈)했는가? (정상성 확보용)
    \item \textbf{MA ($q$, Moving Average):} 이동평균.
    $$ Y_t = \epsilon_t + \theta \epsilon_{t-1} $$
    "어제 예측보다 사람이 갑자기 더 몰렸어(Shock $\epsilon_{t-1}$). 그 여파가 오늘까지 이어질 거야."
\end{itemize}

\subsubsection*{2) 모델 선택}
보통 \textbf{ACF(자기상관함수)}와 \textbf{PACF(편자기상관함수)} 그래프를 그려서 $p$와 $q$ 값을 결정합니다.

\vspace{0.5cm}\hrule\vspace{0.5cm}

\subsection{3. 칼만 필터 (Kalman Filter)}


\begin{conceptbox}{개념 3: 네비게이션의 원리}
\textbf{한 줄 요약:} "내 예측(계산)"과 "센서값(측정)"은 둘 다 부정확합니다. 이 둘을 적절히 섞어서(가중 평균) 최적의 "진짜 위치"를 찾아냅니다.
\end{conceptbox}

\subsubsection*{1) 상태 공간 모델 (State-Space Model)}
\begin{itemize}
    \item \textbf{Hidden State ($x_t$):} 진짜 값 (예: 유저의 실제 게임 실력). 눈에 안 보임.
    \item \textbf{Measurement ($z_t$):} 관측 값 (예: 승패 결과). 노이즈가 섞여 있음.
\end{itemize}

\subsubsection*{2) 작동 원리: 무한 루프}
\begin{enumerate}
    \item \textbf{Predict (예측):} 어제의 상태를 바탕으로 오늘의 상태를 추측합니다. (불확실성 증가)
    \item \textbf{Update (보정):} 실제 데이터($z_t$)가 들어옵니다. 내 예측과 실제 데이터 사이에서, \textbf{더 신뢰할 수 있는 쪽으로} 값을 수정합니다. (불확실성 감소)
\end{enumerate}

\subsubsection*{3) 강점}
과거 데이터를 다 들고 있을 필요 없이, \textbf{직전 상태($t-1$)}만 있으면 됩니다. 그래서 실시간(Real-time) 시스템에 매우 유리합니다.

\vspace{0.5cm}\hrule\vspace{0.5cm}

\section{실전 시나리오: 넥슨 게임 지표 분석}

\begin{storybox}{Scenario 1: 주말 이벤트 효과 분석 (Decomposition)}
토요일에 경험치 2배 이벤트를 했습니다. 접속자가 평일 대비 30\% 늘었습니다. 이게 이벤트 덕분일까요, 아니면 원래 토요일이라서 그런 걸까요?
\end{storybox}
\begin{itemize}
    \item \textbf{해결:} 시계열 분해를 통해 '계절성(토요일 효과)' 성분을 제거합니다.
    \item \textbf{결과:} 계절성을 뺐더니 상승분이 5\%밖에 안 남았습니다. 이벤트 효과는 생각보다 미미했습니다.
\end{itemize}

\begin{storybox}{Scenario 2: 매치메이킹 시스템 (Kalman Filter)}
유저의 실력(MMR)을 측정해야 합니다. 고수도 운 나쁘면 지고, 초보도 버스 타면 이깁니다. 승패(Measurement)는 노이즈 투성이입니다.
\end{storybox}
\begin{itemize}
    \item \textbf{해결:} 칼만 필터를 적용합니다.
    \item \textbf{과정:} 유저가 이겼을 때, MMR을 확 올리지 않고 "이 유저의 실력 불확실성(분산)"을 고려해 조금만 올립니다. 판수가 쌓일수록 불확실성이 줄어들며 '진짜 실력'에 수렴합니다. (TrueSkill 알고리즘의 기초)
\end{itemize}

\vspace{0.5cm}\hrule\vspace{0.5cm}

\section{자주 묻는 질문 (FAQ)}

\begin{description}
    \item[Q1. 딥러닝(RNN/LSTM)이 ARIMA보다 무조건 좋은가요?]
    \textbf{A.} 아닙니다. 데이터가 적거나(수백 개), 패턴이 단순하고 주기적이라면 \textbf{ARIMA}가 훨씬 정확하고 설명력(Why)도 좋습니다. 반면, 비선형적이고 복잡한 패턴(언어, 불규칙한 로그)은 딥러닝이 강합니다.
    
    \item[Q2. 칼만 필터는 예측 모델인가요?]
    \textbf{A.} 엄밀히 말하면 \textbf{'추정(Estimation)'} 모델입니다. 현재의 '진짜 상태'를 가장 잘 맞추는 것이 목표입니다. 하지만 추정된 상태를 바탕으로 다음 단계를 Predict 하므로 예측에도 쓰입니다.
\end{description}

% 10. 다음 단원 연결
\vspace{1cm}
\begin{quote}
\textbf{Next Step:} 시간($t$)에 대한 분석은 마쳤습니다. 그런데 데이터에는 시간뿐만 아니라 \textbf{위치($x, y$)} 정보도 중요합니다. "강남구의 집값이 오르면 서초구도 오를까?" 다음 시간에는 공간적 상관관계를 다루는 \textbf{공간 통계(Spatial Statistics)와 크리깅(Kriging)}을 배웁니다.
\end{quote}

% 11. 단원 요약 박스
\begin{summarybox}{Module 3 (Part A) 핵심 요약}
\begin{itemize}
    \item \textbf{Decomposition:} $Y = T + S + R$. 추세와 계절성을 제거해야 예측이 가능하다.
    \item \textbf{ARIMA:} 과거의 나(AR)와 과거의 오차(MA)를 합쳐 미래를 예측하는 통계적 도구.
    \item \textbf{Kalman Filter:} 예측(Predict)과 보정(Update)을 반복하며 노이즈 속에서 진짜 상태를 찾아내는 알고리즘. (실시간 처리에 강함)
\end{itemize}
\end{summarybox}

\end{document}