\documentclass[a4paper, 11pt]{book}

%===========================================================
% 패키지 설정
%===========================================================
\usepackage[utf8]{inputenc}
\usepackage[T1]{fontenc}
\usepackage{kotex}
\usepackage{amsmath, amssymb, amsfonts, amsthm}
\usepackage{geometry}
\geometry{left=25mm, right=25mm, top=30mm, bottom=30mm}
\usepackage{xcolor}
\usepackage{graphicx}
\usepackage{adjustbox}  % 표/박스 크기 조절
\usepackage{hyperref}
\usepackage{booktabs}
\usepackage{enumitem}
\usepackage[most]{tcolorbox}
\tcbuselibrary{breakable}
\usepackage{tikz}

%===========================================================
% 커스텀 스타일 정의
%===========================================================
\definecolor{mainblue}{RGB}{0, 80, 160}
\definecolor{subgray}{RGB}{240, 240, 240}
\definecolor{alertred}{RGB}{200, 50, 50}
\definecolor{examplegreen}{RGB}{50, 150, 50}
\definecolor{conceptpurple}{RGB}{100, 50, 150}
\definecolor{goldstar}{RGB}{220, 160, 0}

\newcommand{\chaptertitle}[2]{
    \begin{tcolorbox}[colback=mainblue, colframe=mainblue, sharp corners, boxrule=0pt, top=10pt, bottom=10pt]
        \centering \Large \bfseries \textcolor{white}{#1: #2}
    \end{tcolorbox}
}

\newtcolorbox{conceptbox}[1]{
    colback=white, colframe=mainblue, fonttitle=\bfseries,
    title={💡 #1}, rounded corners, drop shadow
}

\newtcolorbox{warningbox}[1]{
    colback=red!5!white, colframe=alertred, fonttitle=\bfseries,
    title={⚠️ 잠깐! 오해하기 쉬워요: #1}, rounded corners
}

\newtcolorbox{examplebox}[1]{
    colback=green!5!white, colframe=examplegreen, fonttitle=\bfseries,
    title={📝 예시: #1}, rounded corners
}

\newtcolorbox{storybox}[1]{
    colback=yellow!10!white, colframe=goldstar, fonttitle=\bfseries,
    title={📖 실전 시나리오: #1}, rounded corners, fontupper=\small
}

\newtcolorbox{mathstep}[1]{
    colback=conceptpurple!5!white, colframe=conceptpurple, fonttitle=\bfseries,
    title={🛠️ CLT 해결 레시피: #1}, rounded corners
}

\begin{document}

%===========================================================
% 1. 전체 목차 (TOC) - 구조 명시
%===========================================================
\tableofcontents
\newpage

\section*{📚 이 교재의 전체 구조 (Roadmap)}
\begin{itemize}
    \item Unit 1: 확률의 기초와 이산 확률 변수 (완료)
    \item Unit 2: 일반 확률 변수 (완료)
    \item \textbf{Unit 3: 확률 과정과 극한 정리 (Random Processes \& Limit Theorems)}
    \begin{itemize}
        \item[$\to$] \textbf{\textcolor{mainblue}{Chapter 7: 극한 정리와 통계적 추론 (Limit Theorems) \textit{<-- 현재 위치}}}
        \item Chapter 8: 베르누이 프로세스와 포아송 프로세스
        \item Chapter 9: 마르코프 체인 (Markov Chains)
    \end{itemize}
\end{itemize}
\vspace{1cm}

%===========================================================
% 2. 현재 단원 제목
%===========================================================
\chaptertitle{Unit 3 - Chapter 7}{극한 정리: 무한($\infty$)이 주는 선물}

%===========================================================
% 3. 이전 단원과의 연결 \& 4. 개요
%===========================================================
\section*{Intro: 정확한 계산에서 '근사(Approximation)'로}

\textbf{[이전 단계와의 연결]} \\
지금까지는 확률 변수의 분포(PMF/PDF)를 정확히 알고 있다고 가정하고, $P(X \le 3)$ 같은 값을 '정확하게' 계산했습니다.
하지만 현실 세계(빅데이터)에서는 데이터가 너무 많거나 분포를 모르는 경우가 대부분입니다.
이제 우리는 \textbf{"데이터가 무한히 많아지면($n \to \infty$), 분포를 몰라도 결과를 예측할 수 있다"}는 강력한 도구를 배웁니다.

\begin{tcolorbox}[colback=subgray, title={\textbf{📌 이 단원의 핵심 개요 (Overview)}}]
\begin{enumerate}
    \item \textbf{부등식 (Bounds):} 분포를 몰라도 평균과 분산만 알면 '최악의 시나리오'를 막을 수 있다. (마르코프, 체비셰프)
    \item \textbf{대수의 법칙 (LLN):} 데이터가 많아지면 통계치(평균)는 결국 진실(True Mean)로 수렴한다.
    \item \textbf{중심 극한 정리 (CLT):} 데이터가 많아지면 합과 평균의 분포는 무조건 \textbf{정규 분포(Bell Curve)}가 된다.
\end{enumerate}
\end{tcolorbox}

%===========================================================
% 5. 용어 정리 표
%===========================================================
\section{필수 용어 사전}

\begin{center}
\begin{tabular}{|c|l|l|}
\hline
\rowcolor{mainblue!20} \textbf{기호} & \textbf{용어} & \textbf{직관적 의미} \\ \hline
Bounds & 한계(부등식) & 정확한 값은 모르지만, "적어도 이 선은 넘지 않는다"는 상한선 \\ \hline
$M_n$ & 표본 평균 & 데이터 $n$개를 모아서 낸 평균 ($\frac{X_1+\dots+X_n}{n}$) \\ \hline
Converge & 수렴 ($\to$) & $n$이 커질수록 특정 값이나 분포에 가까워지는 현상 \\ \hline
LLN & 대수의 법칙 & "많이 던지면 결국 확률대로 나온다." (진실의 발견) \\ \hline
CLT & 중심 극한 정리 & "많이 합치면 종 모양(정규 분포)이 된다." (형태의 발견) \\ \hline
\end{tabular}
\end{center}

%===========================================================
% 6. 핵심 개념 1: 부등식 (Inequalities)
%===========================================================
\section{핵심 개념 1: 분포를 모를 때의 생존법 (부등식)}

분포의 모양(PDF)을 몰라도, \textbf{평균($\mu$)}과 \textbf{분산($\sigma^2$)}만 안다면 확률의 한계(Limit)를 설정할 수 있습니다. 이것은 리스크 관리(Risk Management)의 기초입니다.

\subsection{1. 마르코프 부등식 (Markov Inequality)}
\begin{conceptbox}{평균의 시소 원리}
\begin{itemize}
    \item \textbf{조건:} 데이터가 음수가 아닐 때 ($X \ge 0$).
    \item \textbf{직관:} 평균(무게중심)이 정해져 있다면, 아주 큰 값이 나올 확률은 작아야만 합니다. 그래야 시소가 균형을 잡으니까요.
    \item \textbf{공식:}
    \[ P(X \ge a) \le \frac{E[X]}{a} \]
\end{itemize}
\end{conceptbox}

\begin{examplebox}{연봉 분포 미스터리}
어떤 회사의 평균 연봉이 5,000만 원이라는 것만 알고 있습니다. 이 회사에서 5억 원(평균의 10배) 이상 받는 사람은 최대 몇 명일까요?
\begin{itemize}
    \item 정확한 분포는 모릅니다. 하지만 마르코프 부등식에 의해:
    \[ P(X \ge \text{5억}) \le \frac{\text{5천만}}{\text{5억}} = \frac{1}{10} = 10\% \]
    \item \textbf{결론:} 고액 연봉자는 아무리 많아봤자 전 직원의 10\%를 넘을 수 없습니다. (절대적 상한선)
\end{itemize}
\end{examplebox}

\subsection{2. 체비셰프 부등식 (Chebyshev Inequality)}
\begin{conceptbox}{분산이 알려주는 거리의 제약}
\begin{itemize}
    \item \textbf{조건:} 모든 분포에 적용 가능. (분산 $\sigma^2$을 알 때)
    \item \textbf{직관:} 분산(퍼짐)이 작다면, 데이터는 평균 근처에 모여 있어야 합니다. 평균에서 멀리 떨어진 꼬리(Tail) 확률은 급격히 줄어듭니다.
    \item \textbf{공식:}
    \[ P(|X - \mu| \ge c) \le \frac{\text{var}(X)}{c^2} \]
    (거리 $c$가 멀어질수록 확률은 제곱($c^2$)으로 줄어듭니다.)
\end{itemize}
\end{conceptbox}

%===========================================================
% 7. 핵심 개념 2: 대수의 법칙 (LLN)
%===========================================================
\section{핵심 개념 2: 대수의 법칙 (Laws of Large Numbers)}

\textbf{"데이터가 깡패다."} 데이터($n$)가 많아지면 표본 평균($M_n$)은 흔들림을 멈추고 진실($\mu$)로 수렴합니다.

\subsection{분산의 축소 (Variance Shrinking)}
왜 수렴할까요? 수학적 엔진은 바로 \textbf{분산이 줄어들기 때문}입니다.
\[ M_n = \frac{X_1 + \dots + X_n}{n} \implies \text{var}(M_n) = \frac{\sigma^2}{n} \]

\begin{itemize}
    \item $n \to \infty$이면 분산 $\to 0$이 됩니다.
    \item 분산이 0이라는 뜻은, 분포가 평균 $\mu$ 한 점에 뾰족하게 모인다는 뜻입니다. (불확실성 소멸)
\end{itemize}

\begin{warningbox}{약법칙(Weak) vs 강법칙(Strong)}
\begin{itemize}
    \item \textbf{약법칙 (WLLN):} "오차가 발생할 확률이 0이 된다." (가끔 튀는 놈이 나올 수는 있지만, 그 가능성이 희박해짐)
    \item \textbf{강법칙 (SLLN):} "결과값 자체가 100\% 확률로 진실에 꽂힌다." (시뮬레이션 그래프가 진동하다가 결국 일직선이 됨)
    \item \textbf{실무적 결론:} 엔지니어링에서는 둘 다 "데이터 많으면 평균 믿어도 된다"로 해석해도 무방합니다.
\end{itemize}
\end{warningbox}

%===========================================================
% 8. 핵심 개념 3: 중심 극한 정리 (CLT)
%===========================================================
\section{핵심 개념 3: 중심 극한 정리 (Central Limit Theorem)}

이 챕터의 하이라이트입니다. LLN이 "평균으로 간다"는 \textbf{위치}를 알려줬다면, CLT는 "\textbf{어떤 모양}으로 가는가?"를 알려줍니다.

\begin{conceptbox}{모든 길은 정규 분포로 통한다}
원래 데이터가 이항 분포든, 균등 분포든, 지수 분포든 상관없습니다.
\textbf{"독립적인 확률 변수들을 많이 더하면($S_n$) 그 합의 분포는 정규 분포($N$)에 가까워진다."}
\end{conceptbox}


\subsection{CLT 문제 해결 3단계 레시피}
복잡한 합의 확률을 구해야 할 때, 다음 절차를 따르세요.

\begin{mathstep}{표준화(Standardization) 후 근사}
\textbf{상황:} $S_n = X_1 + \dots + X_n$ (합계)의 확률을 구하고 싶다.

\textbf{1단계: 평균과 분산 계산}
\begin{itemize}
    \item 합의 평균: $E[S_n] = n\mu$
    \item 합의 분산: $\text{var}(S_n) = n\sigma^2$ (표준편차는 $\sigma\sqrt{n}$)
\end{itemize}

\textbf{2단계: 표준화 (Z-score 변환)}
정규 분포표를 쓰기 위해 $Z$로 바꿉니다.
\[ Z_n = \frac{S_n - n\mu}{\sigma\sqrt{n}} \]

\textbf{3단계: 근사 (Approximation)}
이제 $Z_n$을 표준 정규 분포 $N(0, 1)$로 취급하여 표에서 확률 $\Phi(z)$를 찾습니다.
\end{mathstep}

%===========================================================
% 9. 실전 시나리오
%===========================================================
\section{실전 응용: 게임 서버 용량 설계 (Capacity Planning)}

\begin{storybox}{넥슨 서버 개발자의 고민}
신규 게임 출시를 앞두고 있습니다.
\begin{itemize}
    \item 유저 1명의 접속 트래픽($X_i$)은 평균 10MB($\mu$), 표준편차 5MB($\sigma$)입니다. (분포는 모름, 아마도 지수 분포?)
    \item 동시 접속자가 10,000명($n$)일 때, 총 트래픽($S_n$)이 서버 용량인 100,500MB를 초과하여 \textbf{서버가 터질 확률}은?
\end{itemize}

\textbf{해결 과정 (CLT 적용):}
\begin{enumerate}
    \item \textbf{기초값:} $n=10000$, $\mu=10$, $\sigma=5$.
    \item \textbf{합의 통계량:}
    \begin{itemize}
        \item 평균 $E[S_n] = 10000 \times 10 = 100,000$ MB
        \item 분산 $\text{var}(S_n) = 10000 \times 25$ $\implies$ 표준편차 $\sqrt{250000} = 500$ MB
    \end{itemize}
    \item \textbf{표준화:} 우리가 궁금한 임계값 100,500을 $Z$로 변환.
    \[ Z = \frac{100,500 - 100,000}{500} = \frac{500}{500} = 1 \]
    \item \textbf{결론:} $P(Z > 1)$을 구하면 됩니다. 정규 분포표에서 $P(Z \le 1) \approx 0.84$.
    \[ P(S_n > 100500) \approx 1 - 0.84 = 0.16 \]
    \textbf{해석:} 서버가 터질 확률이 16\%나 됩니다! (위험함). 용량을 늘려야 합니다.
\end{enumerate}
\end{storybox}

%===========================================================
% 10. FAQ
%===========================================================
\section{초심자가 자주 묻는 질문 (FAQ)}

\begin{description}
    \item[Q1. $n$이 얼마나 커야 정규 분포라고 볼 수 있나요?] \hfill \\
    보통 통계학 교과서에서는 \textbf{$n \ge 30$}이면 충분하다고 봅니다. 하지만 분포가 심하게 치우친(Skewed) 경우라면 데이터가 더 많이 필요할 수 있습니다.
    
    \item[Q2. 왜 $\sqrt{n}$으로 나누나요?] \hfill \\
    이게 핵심입니다! 합($S_n$)은 $n$에 비례해서 커지지만, 변동성(표준편차)은 $\sqrt{n}$에 비례해서 커집니다.
    즉, 신호(Signal, $n$)가 소음(Noise, $\sqrt{n}$)보다 더 빨리 커지기 때문에, 데이터가 많을수록 비율적으로 안정되는 것입니다.
    
    \item[Q3. 마르코프와 체비셰프 중 뭘 써야 하나요?] \hfill \\
    정보가 많을수록 더 강력한(Tight) 부등식을 쓸 수 있습니다.
    \begin{itemize}
        \item 평균만 안다 $\to$ 마르코프 (범위가 넓음)
        \item 분산까지 안다 $\to$ 체비셰프 (범위가 좁혀짐)
        \item 분포 모양(종 모양)까지 안다 $\to$ 정규 분포 계산 (정확함)
    \end{itemize}
\end{description}

%===========================================================
% 11. 요약 및 다음 단계
%===========================================================
\section{요약 및 다음 단계}

\begin{tcolorbox}[colback=mainblue!10, colframe=mainblue, title={\textbf{📝 Unit 3-1 핵심 요약}}]
\begin{enumerate}
    \item \textbf{부등식:} 분포를 몰라도 평균($\mu$)과 분산($\sigma^2$)만 있으면 리스크의 상한선을 그을 수 있다.
    \item \textbf{LLN:} 데이터가 많으면 표본 평균 $\approx$ 진짜 평균이다. (믿어라!)
    \item \textbf{CLT:} 데이터 합계의 분포는 결국 정규 분포($Z$)로 수렴한다.
    \item \textbf{공학적 의의:} 복잡한 세상의 현상을 정규 분포라는 단순한 모델로 \textbf{근사(Approximation)}하여 해석할 수 있게 해준다.
\end{enumerate}
\end{tcolorbox}

\vspace{0.5cm}
\textbf{[다음 단원 예고]} \\
지금까지는 $n \to \infty$로 갈 때의 '분포'를 봤습니다. 이제 시간 $t$가 흐르면서 사건이 발생하는 \textbf{'과정(Process)'}을 다룹니다. \\
다음 장 \textbf{Chapter 8: 베르누이와 포아송 프로세스}에서는 "동전을 영원히 던지는 상황"과 "콜센터 전화가 계속 오는 상황"을 모델링합니다.

\end{document}