\documentclass[a4paper, 11pt]{book}

%===========================================================
% 패키지 설정 (일관된 스타일 유지)
%===========================================================
\usepackage[utf8]{inputenc}
\usepackage[T1]{fontenc}
\usepackage{kotex}
\usepackage{amsmath, amssymb, amsfonts, amsthm}
\usepackage{geometry}
\geometry{left=25mm, right=25mm, top=30mm, bottom=30mm}
\usepackage{xcolor}
\usepackage{graphicx}
\usepackage{adjustbox}  % 표/박스 크기 조절
\usepackage{hyperref}
\usepackage{booktabs}
\usepackage{enumitem}
\usepackage[most]{tcolorbox}
\tcbuselibrary{breakable}
\usepackage{tikz} % 간단한 그래프 표현용 (선택사항)

%===========================================================
% 커스텀 스타일 정의
%===========================================================
\definecolor{mainblue}{RGB}{0, 80, 160}
\definecolor{subgray}{RGB}{240, 240, 240}
\definecolor{alertred}{RGB}{200, 50, 50}
\definecolor{examplegreen}{RGB}{50, 150, 50}
\definecolor{conceptpurple}{RGB}{100, 50, 150}

\newcommand{\chaptertitle}[2]{
    \begin{tcolorbox}[colback=mainblue, colframe=mainblue, sharp corners, boxrule=0pt, top=10pt, bottom=10pt]
        \centering \Large \bfseries \textcolor{white}{#1: #2}
    \end{tcolorbox}
}

\newtcolorbox{conceptbox}[1]{
    colback=white, colframe=mainblue, fonttitle=\bfseries,
    title={💡 #1}, rounded corners, drop shadow
}

\newtcolorbox{warningbox}[1]{
    colback=red!5!white, colframe=alertred, fonttitle=\bfseries,
    title={⚠️ 0의 역설 (Zero Probability Paradox): #1}, rounded corners
}

\newtcolorbox{examplebox}[1]{
    colback=green!5!white, colframe=examplegreen, fonttitle=\bfseries,
    title={📝 예시: #1}, rounded corners
}

\newtcolorbox{storybox}[1]{
    colback=yellow!10!white, colframe=orange, fonttitle=\bfseries,
    title={📖 Story Matching: #1}, rounded corners, fontupper=\small
}

\newtcolorbox{mathstep}[1]{
    colback=conceptpurple!5!white, colframe=conceptpurple, fonttitle=\bfseries,
    title={🛠️ 절차적 기술 (Skill): #1}, rounded corners
}

\begin{document}

%===========================================================
% 1. 전체 목차 (TOC) - 구조 명시
%===========================================================
\tableofcontents
\newpage

\section*{📚 이 교재의 전체 구조 (Roadmap)}
\begin{itemize}
    \item Unit 1: 확률의 기초와 이산 확률 변수 (완료)
    \item \textbf{Unit 2: 일반 확률 변수 (General Random Variables)}
    \begin{itemize}
        \item[$\to$] \textbf{\textcolor{mainblue}{Chapter 4: 연속 확률 변수와 PDF (Continuous RVs) \textit{<-- 현재 위치}}}
        \item Chapter 5: 다변수 연속 확률 변수
        \item Chapter 6: 확률 변수의 변환과 유도
    \end{itemize}
    \item Unit 3: 확률 과정과 극한 정리
\end{itemize}
\vspace{1cm}

%===========================================================
% 2. 현재 단원 제목
%===========================================================
\chaptertitle{Unit 2 - Chapter 4}{연속 확률 변수: 레고에서 유체로}

%===========================================================
% 3. 이전 단원과의 연결 \& 4. 개요
%===========================================================
\section*{Intro: 더하기($\sum$)에서 적분($\int$)으로}

\textbf{[이전 단계와의 연결]} \\
Unit 1에서는 주사위 눈이나 동전 개수처럼 \textbf{하나하나 셀 수 있는(Countable)} '레고 블록' 같은 데이터를 다뤘습니다. 이때는 확률을 블록의 높이로 생각하고 단순히 더했습니다($\sum$).

하지만 현실의 시간, 온도, 속도, 전압은 뚝뚝 끊어져 있지 않고 \textbf{연속적으로 흐르는 유체(Fluid)}와 같습니다. 물의 양을 잴 때 블록 개수를 세지 않고 '부피'를 측정하듯이, 이제 우리는 확률을 \textbf{'구간의 면적(Area)'}으로 다루는 미적분의 세계로 들어갑니다.

\begin{tcolorbox}[colback=subgray, title={\textbf{📌 이 단원의 핵심 개요 (Overview)}}]
\begin{enumerate}
    \item \textbf{PDF (밀도):} 점의 확률은 0입니다. 이제 '밀도'를 적분해야 확률이 됩니다.
    \item \textbf{분포 동물원 (The Zoo):} 균등(Uniform), 지수(Exponential), 정규(Normal) 분포의 탄생 배경을 배웁니다.
    \item \textbf{표준화 (Standardization):} 복잡한 정규 분포를 $Z$-score로 변환하여 쉽게 계산하는 법을 익힙니다.
    \item \textbf{혼합 추론 (Mixed Bayes):} 연속된 신호(전압)를 보고 이산적 원인(0 or 1)을 맞추는 통신의 기초를 다룹니다.
\end{enumerate}
\end{tcolorbox}

%===========================================================
% 5. 용어 정리 표
%===========================================================
\section{필수 용어 사전}

\begin{center}
\begin{tabular}{|c|l|l|}
\hline
\rowcolor{mainblue!20} \textbf{기호} & \textbf{용어} & \textbf{직관적 의미} \\ \hline
PDF & 확률 밀도 함수 ($f_X(x)$) & 특정 지점에서의 확률의 \textbf{농도(높이)}. (확률 아님!) \\ \hline
CDF & 누적 분포 함수 ($F_X(x)$) & $-\infty$부터 $x$까지 그래프 아래의 \textbf{면적} (진짜 확률) \\ \hline
Uniform & 균등 분포 & "정보가 없다." 모든 구간의 확률 밀도가 평평함. \\ \hline
Exponential & 지수 분포 & "대기 시간." 사건이 터질 때까지 걸리는 시간. \\ \hline
Normal & 정규 분포 (Gaussian) & "오차의 합." 자연계의 잡음(Noise)을 설명하는 종 모양. \\ \hline
$Z$ & 표준 정규 분포 & 평균 0, 분산 1로 통일된 기준 정규 분포. \\ \hline
\end{tabular}
\end{center}

%===========================================================
% 6. 핵심 개념 1: PDF와 0의 역설
%===========================================================
\section{핵심 개념 1: 밀도(Density)는 확률이 아니다}

\subsection{확률 밀도 함수 (PDF, $f_X(x)$)}
[Image of Integration Area under Curve]

\begin{conceptbox}{높이가 아니라 '면적'이 확률이다}
\begin{itemize}
    \item \textbf{한 줄 요약:} PDF 그래프의 $y$값(높이)은 확률이 아니라 '밀도'이며, 이를 구간으로 \textbf{적분(넓이)}해야 비로소 확률이 됩니다.
    \item \textbf{직관적 비유:} 금속 막대의 어느 한 점을 바늘로 찔렀을 때의 무게는 0입니다. 하지만 일정 길이(구간)를 잘라내면 무게(확률)가 생깁니다.
    \item \textbf{공식:}
    \[ P(a \le X \le b) = \int_{a}^{b} f_X(x) \, dx \]
\end{itemize}
\end{conceptbox}

\begin{warningbox}{점의 확률은 0이다 ($P(X=x)=0$)}
많은 분들이 가장 혼란스러워하는 부분입니다.
\begin{itemize}
    \item 친구가 약속 장소에 "정확히 2.00000...초"에 도착할 확률은? \textbf{0입니다.}
    \item 수학적으로 \textbf{선(Line)의 넓이는 0}이기 때문입니다.
    \item 따라서 연속 확률 변수에서는 등호의 유무가 중요하지 않습니다.
    \[ P(X \ge a) = P(X > a) \]
\end{itemize}
\end{warningbox}

%===========================================================
% 7. 핵심 개념 2: 연속 분포의 동물원 (The Continuous Zoo)
%===========================================================
\section{핵심 개념 2: 주요 연속 분포 (Story Matching)}

\begin{storybox}{1. 균등 분포 (Uniform): "정보 부재의 공평함"}
\begin{itemize}
    \item \textbf{스토리:} 버스가 0분에서 10분 사이에 온다는 것만 알고, 다른 정보는 전혀 없을 때.
    \item \textbf{특징:} 그래프가 평평한 박스 모양입니다.
    \item \textbf{확률:} 전체 구간 길이 분의 내가 원하는 구간 길이 (기하학적 확률).
\end{itemize}
\end{storybox}

\begin{storybox}{2. 지수 분포 (Exponential): "무기억성(Memoryless)"}
\begin{itemize}
    \item \textbf{스토리:} 콜센터 전화가 올 때까지 걸리는 대기 시간. (이산 기하 분포의 연속 버전)
    \item \textbf{핵심 성질 (무기억성):} "지난 30분 동안 전화가 안 왔다고 해서, 지금 당장 올 확률이 높아지는 건 아니다."
    \[ P(T > t+s | T > t) = P(T > s) \]
    \item 과거를 잊어버리고 매 순간 새롭게 시작하는(Fresh Start) 성질입니다.
\end{itemize}
\end{storybox}

\begin{storybox}{3. 정규 분포 (Normal): "세상의 모든 잡음(Noise)"}
\begin{itemize}
    \item \textbf{스토리:} 키, 몸무게, 시험 점수, 측정 오차 등 수많은 작은 원인들이 합쳐진 결과.
    \item \textbf{중요성:} 중심극한정리(CLT)에 의해, 데이터가 충분히 많으면 세상 만사는 정규 분포(종 모양)를 따르게 됩니다.
\end{itemize}
\end{storybox}

%===========================================================
% 8. 공식/절차: 정규 분포의 표준화
%===========================================================
\section{필수 스킬: 정규 분포의 표준화 (Standardization)}

정규 분포 곡선 $e^{-x^2}$은 손으로 적분하는 것이 불가능합니다. 그래서 우리는 모든 정규 분포를 \textbf{표준 정규 분포($Z$)}로 변환하여 미리 계산된 표(Standard Normal Table)를 사용합니다.

\begin{mathstep}{모든 정규 분포를 $N(0, 1)$로 만드는 마법}
변수 $X$가 평균 $\mu$, 표준편차 $\sigma$를 가질 때 ($X \sim N(\mu, \sigma^2)$):

\textbf{1단계: Z-score 변환 (공식 암기 필수)}
\[ Z = \frac{X - \mu}{\sigma} \]
(의미: 내 점수가 평균에서 표준편차의 몇 배만큼 떨어져 있는가?)

\textbf{2단계: 표 찾기}
변환된 $Z$값을 이용해 표(CDF)에서 확률 $\Phi(Z)$를 찾습니다.
\end{mathstep}

\begin{examplebox}{수능 점수 계산}
평균이 50점($\mu=50$), 표준편차가 10점($\sigma=10$)인 시험에서 70점 이상 받을 확률은?
\begin{enumerate}
    \item \textbf{표준화:} $z = \frac{70 - 50}{10} = 2$. (나는 평균보다 2$\sigma$만큼 잘했다.)
    \item \textbf{확률 확인:} 표준 정규 분포 표에서 $P(Z \le 2) \approx 0.977$.
    \item \textbf{결론:} 상위 약 2.3\% ($1 - 0.977$) 안에 듭니다.
\end{enumerate}
\end{examplebox}

%===========================================================
% 9. 예시 시나리오: 혼합형 베이즈 정리
%===========================================================
\section{실전 응용: 디지털 통신과 베이즈 정리}

\textbf{[상황]} AI 모델이 이미지를 보고 '고양이(0)'인지 '개(1)'인지 분류하려 합니다.
\begin{itemize}
    \item \textbf{원인(Discrete):} 실제 정답 $K \in \{0, 1\}$.
    \item \textbf{결과(Continuous):} 모델이 뱉어낸 확신 점수(Score) $X$ (0.0 ~ 1.0 사이의 실수).
\end{itemize}
우리는 점수 $X=0.7$을 관측했을 때, 이것이 실제로 개($K=1$)일 확률을 알고 싶습니다.

\[ P(K=1 | X=0.7) = \frac{P(K=1) \cdot f_{X|K}(0.7|1)}{f_X(0.7)} \]

\begin{itemize}
    \item \textbf{이산과 연속의 만남:} 분자는 '이산 확률 $\times$ 연속 밀도'입니다.
    \item \textbf{분모(전체 확률):} 전체 밀도 $f_X(0.7)$은 고양이일 때의 밀도와 개일 때의 밀도를 가중 합(Total Probability)하여 구합니다.
    \item 이 공식이 바로 \textbf{모든 머신러닝 분류기(Classifier)}가 작동하는 수학적 원리입니다.
\end{itemize}

%===========================================================
% 10. FAQ
%===========================================================
\section{초심자가 자주 묻는 질문 (FAQ)}

\begin{description}
    \item[Q1. PDF 값(높이)이 1보다 클 수 있나요?] \hfill \\
    \textbf{네, 가능합니다!} 확률은 1을 넘을 수 없지만, 밀도(Density)는 넘을 수 있습니다.
    \begin{itemize}
        \item 예: 구간 [0, 0.1]에 확률 1이 꽉 차 있다면, 높이(밀도)는 10이어야 면적이 1이 됩니다. ($10 \times 0.1 = 1$)
    \end{itemize}
    
    \item[Q2. 왜 적분을 해야 확률이 나오나요?] \hfill \\
    속도($v$)를 적분하면 이동 거리($s$)가 나오죠? 마찬가지로 확률의 '변화율(밀도)'을 쌓아야 전체 '양(확률)'이 나오기 때문입니다.
    
    \item[Q3. 정규 분포 식을 외워야 하나요?] \hfill \\
    복잡한 식($\frac{1}{\sqrt{2\pi}\sigma}e^{-\dots}$) 자체를 외울 필요는 없습니다. 중요한 것은 그 식을 \textbf{$Z$로 변환하는 방법}과 그래프의 \textbf{대칭성(Symmetry)}을 활용하는 능력입니다.
\end{description}

%===========================================================
% 11. 요약 및 다음 단계
%===========================================================
\section{요약 및 다음 단계}

\begin{tcolorbox}[colback=mainblue!10, colframe=mainblue, title={\textbf{📝 Unit 2-1 핵심 요약}}]
\begin{enumerate}
    \item \textbf{적분 마인드:} 점($x$)은 0이다. 구간($dx$)을 적분해야 확률(면적)이다.
    \item \textbf{분포 매칭:} 랜덤 $\to$ Uniform / 대기시간 $\to$ Exponential / 오차합 $\to$ Normal.
    \item \textbf{표준화:} 정규 분포 문제는 무조건 $Z = \frac{X-\mu}{\sigma}$로 바꿔서 푼다.
    \item \textbf{혼합 추론:} 연속된 데이터(신호)로 이산적 원인(디지털 정보)을 추측한다.
\end{enumerate}
\end{tcolorbox}

\vspace{0.5cm}
\textbf{[다음 단원 예고]} \\
하나의 변수(키)만으로는 세상을 설명하기 부족합니다. 키($X$)와 몸무게($Y$)는 서로 관계가 있지 않을까요? \\
다음 장 \textbf{Chapter 5: 다변수 연속 확률 변수}에서는 2차원 평면 위에서의 적분(이중 적분)과 상관관계(Correlation)를 다룹니다. "적분 기호가 두 개($\iint$)로 늘어납니다!"

\end{document}