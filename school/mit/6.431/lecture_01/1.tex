\documentclass[a4paper, 11pt]{book}

%===========================================================
% 패키지 설정
%===========================================================
\usepackage[utf8]{inputenc}
\usepackage[T1]{fontenc}
\usepackage{kotex} % 한글 지원
\usepackage{amsmath, amssymb, amsfonts, amsthm} % 수식 관련
\usepackage{geometry} % 여백 설정
\geometry{left=25mm, right=25mm, top=30mm, bottom=30mm}
\usepackage{xcolor} % 색상 지원
\usepackage{graphicx}
\usepackage{adjustbox}  % 표/박스 크기 조절 % 이미지 지원
\usepackage{hyperref} % 하이퍼링크
\usepackage{booktabs} % 표 디자인
\usepackage{enumitem} % 리스트 디자인
\usepackage[most]{tcolorbox}
\tcbuselibrary{breakable} % 박스 디자인

%===========================================================
% 커스텀 스타일 정의
%===========================================================
% 메인 색상 정의
\definecolor{mainblue}{RGB}{0, 80, 160}
\definecolor{subgray}{RGB}{240, 240, 240}
\definecolor{alertred}{RGB}{200, 50, 50}
\definecolor{examplegreen}{RGB}{50, 150, 50}

% 챕터 제목 스타일
\newcommand{\chaptertitle}[2]{
    \begin{tcolorbox}[colback=mainblue, colframe=mainblue, sharp corners, boxrule=0pt, top=10pt, bottom=10pt]
        \centering \Large \bfseries \textcolor{white}{#1: #2}
    \end{tcolorbox}
}

% 개념 정리 박스
\newtcolorbox{conceptbox}[1]{
    colback=white, colframe=mainblue, fonttitle=\bfseries,
    title={💡 #1}, rounded corners, drop shadow
}

% 오해 방지 박스 (Warning)
\newtcolorbox{warningbox}[1]{
    colback=red!5!white, colframe=alertred, fonttitle=\bfseries,
    title={⚠️ 잠깐! 오해하기 쉬워요: #1}, rounded corners
}

% 예제 박스
\newtcolorbox{examplebox}[1]{
    colback=green!5!white, colframe=examplegreen, fonttitle=\bfseries,
    title={📝 예시: #1}, rounded corners
}

% 스토리 박스
\newtcolorbox{storybox}[1]{
    colback=yellow!10!white, colframe=orange, fonttitle=\bfseries,
    title={📖 스토리 시나리오: #1}, rounded corners, fontupper=\small
}

\begin{document}

%===========================================================
% 1. 전체 목차 (TOC) - 구조 명시
%===========================================================
\tableofcontents
\newpage

% 목차 구조 시각화 (현재 위치 강조)
\section*{📚 이 교재의 전체 구조 (Roadmap)}
\begin{itemize}
    \item \textbf{Unit 1: 확률의 기초와 이산 확률 변수 (Fundamentals \& Discrete Random Variables)}
    \begin{itemize}
        \item[$\to$] \textbf{\textcolor{mainblue}{Chapter 1: 확률 모형의 수립과 공리 (Probability Models \& Axioms) \textit{<-- 현재 위치}}}
        \item Chapter 2: 조건부 확률과 베이즈 정리
        \item Chapter 3: 독립성과 이산 확률 변수
    \end{itemize}
    \item Unit 2: 일반 확률 변수 (General Random Variables)
    \item Unit 3: 확률 과정과 극한 정리 (Random Processes \& Limit Theorems)
\end{itemize}
\vspace{1cm}

%===========================================================
% 2. 현재 단원 제목
%===========================================================
\chaptertitle{Unit 1 - Chapter 1}{확률 모형의 수립과 기초 공리}

%===========================================================
% 3. 이전 단원과의 연결 \& 4. 개요
%===========================================================
\section*{Intro: 불확실한 세상으로의 첫걸음}

\textbf{[이전 단계와의 연결]} \\
지금까지 여러분은 $1+1=2$와 같이 결과가 정해져 있는 '결정론적(Deterministic)' 세계에서 수학을 배웠습니다. 하지만 현실은 "내일 비가 올까?", "이 주식이 오를까?"처럼 불확실성으로 가득 차 있습니다. 이제 우리는 확실하지 않은 미래를 숫자로 다루는 법을 배웁니다.

\begin{tcolorbox}[colback=subgray, title={\textbf{📌 이 단원의 핵심 개요 (Overview)}}]
이 단원에서는 현실 세계의 모호한 문제를 수학적 '집합'으로 번역하는 방법을 배웁니다.
\begin{enumerate}
    \item \textbf{모델링:} 일어날 수 있는 모든 일을 \textbf{표본 공간($\Omega$)}이라는 우주로 정의합니다.
    \item \textbf{규칙(공리):} 확률이 무너지지 않게 지탱하는 3가지 절대 규칙(콜모고로프 공리)을 배웁니다.
    \item \textbf{도구(셈하기):} 경우의 수를 정확히 세서 확률을 계산하는 기술(순열/조합)을 익힙니다.
\end{enumerate}
\end{tcolorbox}

%===========================================================
% 5. 용어 정리 표
%===========================================================
\section{필수 용어 사전 (Terminology)}
확률론은 '집합의 언어'를 사용합니다. 이 번역표를 머릿속에 넣어두세요.

\begin{center}
\begin{tabular}{|c|l|l|}
\hline
\rowcolor{mainblue!20} \textbf{기호} & \textbf{용어 (한국어/영어)} & \textbf{직관적 의미} \\ \hline
$\Omega$ & 표본 공간 (Sample Space) & 일어날 수 있는 \textbf{모든} 결과의 전체 집합 (우주) \\ \hline
$\omega$ & 근원 사건 (Outcome) & 실험의 가장 작은 결과 단위 (원소) \\ \hline
$A, B$ & 사건 (Event) & 우리가 관심 있는 특정 결과들의 모임 (부분집합) \\ \hline
$P(A)$ & 확률 (Probability) & 사건 A가 일어날 믿음의 정도 (크기, 무게) \\ \hline
$A \cap B$ & 교집합 (Intersection) & A \textbf{그리고} B가 동시에 일어남 \\ \hline
$A \cup B$ & 합집합 (Union) & A \textbf{또는} B가 일어남 \\ \hline
\end{tabular}
\end{center}

%===========================================================
% 6. 핵심 개념 상세 설명
%===========================================================
\section{핵심 개념 1: 확률 모형의 수립 (Modeling)}

\subsection{실험과 표본 공간 ($\Omega$)}
\begin{conceptbox}{표본 공간은 우리가 노는 '운동장'이다}
\begin{itemize}
    \item \textbf{한 줄 요약:} 실험에서 나올 수 있는 모든 결과를 빠짐없이 모아둔 집합입니다.
    \item \textbf{직관적 비유:} 식당의 \textbf{전체 메뉴판}. 내가 무엇을 주문하든, 반드시 메뉴판 안에 있는 것이어야 합니다.
    \item \textbf{기술적 정의:} 상호 배타적(Mutually Exclusive)이고 전체를 포괄하는(Collectively Exhaustive) 모든 결과들의 집합 $\Omega$.
    \item \textbf{구체적 예시:} 동전을 한 번 던질 때 $\Omega = \{ \text{앞면(H)}, \text{뒷면(T)} \}$.
\end{itemize}
\end{conceptbox}

\begin{warningbox}{Outcome vs Event 구분하기}
\begin{itemize}
    \item \textbf{Outcome (근원 사건):} 주사위를 던져 '1'이 나오는 것. (더 쪼갤 수 없음)
    \item \textbf{Event (사건):} 주사위를 던져 '홀수'가 나오는 것 ($\{1, 3, 5\}$). (Outcome들의 묶음)
    \item 확률론에서는 주로 \textbf{Event(집합)}에 확률을 부여합니다.
\end{itemize}
\end{warningbox}

\subsection{사건(Event)과 집합 연산}
현실의 언어를 집합의 언어로 번역해야 합니다.
\begin{itemize}
    \item "적어도 한 번 앞면" $\rightarrow$ $\{HHT, HTH, THH, HHH, \dots\}$
    \item "A가 일어나지 않음" $\rightarrow$ $A^c$ (여집합)
    \item "A와 B가 겹치지 않음" $\rightarrow$ $A \cap B = \emptyset$ (배반 사건, Disjoint)
\end{itemize}

%===========================================================
% 7. 공식/절차 + 예시 계산 (공리)
%===========================================================
\section{핵심 개념 2: 확률의 3공리 (The Axioms)}

확률은 감으로 찍는 것이 아닙니다. 러시아 수학자 콜모고로프가 만든 3가지 절대 규칙 위에서만 작동합니다.

\begin{conceptbox}{콜모고로프의 3공리}
\begin{enumerate}
    \item \textbf{Non-negativity (비음수성):} 확률은 절대 음수가 될 수 없다.
    \[ P(A) \ge 0 \]
    \item \textbf{Normalization (정규화):} 전체 우주(모든 가능성)의 확률 합은 1(100\%)이다.
    \[ P(\Omega) = 1 \]
    \item \textbf{Additivity (가산성):} 서로 겹치지 않는(Disjoint) 사건들의 합집합 확률은 각 확률의 단순 합과 같다.
    \[ \text{If } A \cap B = \emptyset, \text{ then } P(A \cup B) = P(A) + P(B) \]
\end{enumerate}
\end{conceptbox}

\begin{examplebox}{공리의 적용: 케이크 자르기}
\begin{itemize}
    \item 전체 케이크($\Omega$)의 크기는 1입니다. (공리 2)
    \item 케이크 한 조각($A$)의 크기는 0보다 큽니다. (공리 1)
    \item 딸기 조각($A$)과 초코 조각($B$)이 겹치지 않는다면, 두 조각을 합친 크기는 그냥 두 조각의 무게를 더하면 됩니다. (공리 3)
\end{itemize}
\end{examplebox}

\subsection{공리에서 유도된 유용한 도구들}
\begin{itemize}
    \item \textbf{포함-배제 원리:} 겹치는 게 있다면?
    \[ P(A \cup B) = P(A) + P(B) - P(A \cap B) \]
    (두 번 더해진 교집합 부분을 한 번 빼줘야 함)
    
    \item \textbf{Union Bound (안전빵 부등식):} 겹치는지 아닌지 모를 때 최대 확률은?
    \[ P(A \cup B) \le P(A) + P(B) \]
    (엔지니어링에서 최악의 시나리오를 계산할 때 매우 중요!)
\end{itemize}

%===========================================================
% 8. 셈하기 (Counting) 및 스토리 시나리오
%===========================================================
\section{핵심 개념 3: 셈하기 (Counting)}

모든 결과가 동등하게 일어날 때(주사위 등), 확률 계산은 결국 \textbf{"분모와 분자를 잘 세는 싸움"}입니다.
\[ P(A) = \frac{\text{사건 A의 경우의 수}}{\text{전체 경우의 수}} \]

\subsection{셈하기 결정 트리 (Decision Tree)}
문제를 보자마자 다음 두 질문을 던지세요.
\begin{enumerate}
    \item \textbf{순서가 중요한가? (Order matters?)}
    \begin{itemize}
        \item Yes $\rightarrow$ \textbf{순열 (Permutation)}: 비밀번호 1234와 4321은 다르다.
        \item No $\rightarrow$ \textbf{조합 (Combination)}: 로또 번호 1, 5, 10과 10, 5, 1은 같다.
    \end{itemize}
    \item \textbf{중복을 허용하는가? (Replacement?)}
    \begin{itemize}
        \item 뽑고 다시 넣나(복원), 아니면 버리나(비복원).
    \end{itemize}
\end{enumerate}

\begin{storybox}{비밀요원 K의 가방 열기 (Counting 시나리오)}
비밀요원 K가 3자리 숫자 자물쇠가 달린 가방을 열어야 합니다.
\begin{enumerate}
    \item \textbf{상황 A (순열):} 비밀번호는 0~9 숫자 중 서로 다른 3개로 이루어져 있고, 순서가 중요합니다.
    \begin{itemize}
        \item 계산: 첫 번째 칸 10개 $\times$ 두 번째 칸 9개 $\times$ 세 번째 칸 8개
        \item $10 \times 9 \times 8 = 720$가지.
    \end{itemize}
    
    \item \textbf{상황 B (조합):} 사실 자물쇠가 아니라, 10개의 버튼 중 3개를 동시에 누르면 열리는 방식이었습니다. (순서 상관 없음)
    \begin{itemize}
        \item 계산: 순열(720)에서 순서 섞이는 경우($3! = 3 \times 2 \times 1 = 6$)를 나눠줘야 합니다.
        \item $\binom{10}{3} = \frac{720}{6} = 120$가지.
    \end{itemize}
\end{enumerate}
\textbf{교훈:} "순서"가 사라지니 경우의 수가 720에서 120으로 확 줄어듭니다!
\end{storybox}

%===========================================================
% 9. 예시 시나리오 (실전 적용)
%===========================================================
\section{실전 응용: 시스템 안정성 (Union Bound)}

\textbf{상황:} 당신은 서버 관리자입니다. 서버 A가 다운될 확률은 5\%($0.05$), 서버 B가 다운될 확률은 10\%($0.10$)입니다. 두 서버가 동시에 다운되는지, 서로 영향을 주는지는 복잡해서 정확히 모릅니다.

\textbf{질문:} "적어도 하나의 서버가 다운될 최악의 확률은 얼마인가?"

\textbf{해결:} Union Bound를 사용합니다.
\[ P(A \cup B) \le P(A) + P(B) = 0.05 + 0.10 = 0.15 \]
\textbf{결론:} 정확한 상관관계를 몰라도, "어쨌든 전체 시스템 에러율은 15\%를 넘지는 않겠군"이라고 보수적으로 판단하고 대비할 수 있습니다. 이것이 공학적 사고입니다.

%===========================================================
% 10. 자주 하는 질문 (FAQ)
%===========================================================
\section{초심자가 자주 묻는 질문 (FAQ)}

\begin{description}
    \item[Q1. 왜 확률은 1을 넘을 수 없나요?] \hfill \\
    확률은 '전체(표본 공간 $\Omega$)'에 대한 '부분(사건 $A$)'의 비율(무게)입니다. 부분이 전체보다 클 수는 없습니다. 만약 계산 결과가 1.2가 나왔다면, 어딘가(주로 겹치는 부분)를 중복해서 더한 것입니다.
    
    \item[Q2. 언제 곱하고, 언제 더하나요?] \hfill \\
    \begin{itemize}
        \item \textbf{곱할 때:} 단계적으로 일이 진행될 때 ("A 하고, \textbf{그리고 나서} B 한다"). 예: 옷 입기 (상의 $\times$ 하의).
        \item \textbf{더할 때:} 경우를 나눌 때 ("A인 경우 \textbf{또는} B인 경우"). 예: 등교 방법 (버스 타는 수 + 지하철 타는 수).
    \end{itemize}
    
    \item[Q3. 주사위 2개를 던질 때 표본 공간을 $\{2, 3, \dots, 12\}$로 잡으면 안 되나요?] \hfill \\
    가능은 합니다. 하지만 추천하지 않습니다. 왜냐하면 합이 2가 되는 경우(1,1)와 합이 7이 되는 경우(1,6, 2,5, 3,4...)의 \textbf{확률이 다르기 때문}입니다. 계산을 쉽게 하려면 모든 근원 사건의 확률이 같은 '가장 잘게 쪼개진' 모델($\{(1,1), \dots, (6,6)\}$)을 쓰는 것이 좋습니다.
\end{description}

%===========================================================
% 11. 단원 요약 및 다음 단계
%===========================================================
\section{요약 및 다음 단계}

\begin{tcolorbox}[colback=mainblue!10, colframe=mainblue, title={\textbf{📝 Unit 1-1 핵심 요약}}]
\begin{enumerate}
    \item \textbf{모델링:} 문제를 집합($\Omega$)과 부분집합(Event)으로 바꿔라.
    \item \textbf{공리:} 확률의 3원칙(0 이상, 전체는 1, 겹치지 않으면 더하기)을 항상 기억하라.
    \item \textbf{셈하기:} 순서가 있는지(Permutation), 없는지(Combination) 먼저 판단하라.
\end{enumerate}
\end{tcolorbox}

\vspace{0.5cm}
\textbf{[다음 단원 예고]} \\
이번 장에서는 모든 사건이 똑같은 확률로 일어난다고 가정하거나, 이미 확률이 주어졌다고 쳤습니다. 하지만 현실에서 내일 비가 올 확률은 어떻게 알 수 있을까요? 새로운 정보(구름이 꼈다)가 들어오면 확률은 어떻게 바뀔까요? \\
다음 장 \textbf{Chapter 2: 조건부 확률(Conditional Probability)}에서 그 비밀을 밝혀냅니다. "정보가 곧 확률을 바꿉니다."

\end{document}