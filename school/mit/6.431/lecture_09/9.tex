\documentclass[a4paper, 11pt]{book}

%===========================================================
% 패키지 설정
%===========================================================
\usepackage[utf8]{inputenc}
\usepackage[T1]{fontenc}
\usepackage{kotex}
\usepackage{amsmath, amssymb, amsfonts, amsthm}
\usepackage{geometry}
\geometry{left=25mm, right=25mm, top=30mm, bottom=30mm}
\usepackage{xcolor}
\usepackage{graphicx}
\usepackage{adjustbox}  % 표/박스 크기 조절
\usepackage{hyperref}
\usepackage{booktabs}
\usepackage{enumitem}
\usepackage[most]{tcolorbox}
\tcbuselibrary{breakable}
\usepackage{tikz}

%===========================================================
% 커스텀 스타일 정의
%===========================================================
\definecolor{mainblue}{RGB}{0, 80, 160}
\definecolor{subgray}{RGB}{240, 240, 240}
\definecolor{alertred}{RGB}{200, 50, 50}
\definecolor{examplegreen}{RGB}{50, 150, 50}
\definecolor{conceptpurple}{RGB}{100, 50, 150}
\definecolor{finalgold}{RGB}{218, 165, 32}

\newcommand{\chaptertitle}[2]{
    \begin{tcolorbox}[colback=mainblue, colframe=mainblue, sharp corners, boxrule=0pt, top=10pt, bottom=10pt]
        \centering \Large \bfseries \textcolor{white}{#1: #2}
    \end{tcolorbox}
}

\newtcolorbox{conceptbox}[1]{
    colback=white, colframe=mainblue, fonttitle=\bfseries,
    title={💡 #1}, rounded corners, drop shadow
}

\newtcolorbox{warningbox}[1]{
    colback=red!5!white, colframe=alertred, fonttitle=\bfseries,
    title={⚠️ 잠깐! 오해하기 쉬워요: #1}, rounded corners
}

\newtcolorbox{examplebox}[1]{
    colback=green!5!white, colframe=examplegreen, fonttitle=\bfseries,
    title={📝 예시: #1}, rounded corners
}

\newtcolorbox{storybox}[1]{
    colback=yellow!10!white, colframe=finalgold, fonttitle=\bfseries,
    title={🏆 실전 시나리오: #1}, rounded corners, fontupper=\small
}

\newtcolorbox{mathstep}[1]{
    colback=conceptpurple!5!white, colframe=conceptpurple, fonttitle=\bfseries,
    title={🛠️ 문제 해결 스킬: #1}, rounded corners
}

\begin{document}

%===========================================================
% 1. 전체 목차 (TOC) - 구조 명시
%===========================================================
\tableofcontents
\newpage

\section*{📚 이 교재의 전체 구조 (Roadmap)}
\begin{itemize}
    \item Unit 1: 확률의 기초와 이산 확률 변수 (완료)
    \item Unit 2: 일반 확률 변수 (완료)
    \item \textbf{Unit 3: 확률 과정과 극한 정리 (Random Processes \& Limit Theorems)}
    \begin{itemize}
        \item Chapter 7: 극한 정리와 통계적 추론 (완료)
        \item Chapter 8: 베르누이와 포아송 프로세스 (완료)
        \item[$\to$] \textbf{\textcolor{mainblue}{Chapter 9: 마르코프 체인 (Markov Chains) \textit{<-- 현재 위치 (완결)}}}
    \end{itemize}
\end{itemize}
\vspace{1cm}

%===========================================================
% 2. 현재 단원 제목
%===========================================================
\chaptertitle{Unit 3 - Chapter 9}{마르코프 체인: 과거는 잊고 미래로}

%===========================================================
% 3. 이전 단원과의 연결 \& 4. 개요
%===========================================================
\section*{Intro: 무작위성(Randomness)에 구조(State)를 입히다}

\textbf{[이전 단계와의 연결]} \\
Chapter 8의 베르누이/포아송 과정은 "매 순간이 똑같다(Memoryless)"는 가정하에 진행되었습니다. 하지만 현실은 다릅니다. '맑음' 뒤에는 '맑음'일 확률이 높고, '비' 뒤에는 '흐림'일 확률이 높습니다.
이제 우리는 \textbf{"현재의 상태(State)가 미래의 확률을 결정하는"} 조금 더 똑똑한 시스템, 마르코프 체인을 배웁니다.

\begin{tcolorbox}[colback=subgray, title={\textbf{📌 이 단원의 핵심 개요 (Overview)}}]
\begin{enumerate}
    \item \textbf{마르코프 성질:} "미래는 오직 현재에만 달려있다." 과거 이력은 깡그리 무시합니다.
    \item \textbf{전이 행렬 (Map):} 상태들 사이를 이동하는 확률 지도를 행렬로 표현합니다.
    \item \textbf{구조 분석 (Class):} 한번 가면 못 돌아오는 길(Transient)과 갇히는 길(Recurrent)을 구분합니다.
    \item \textbf{정상 상태 ($\pi$):} 시간이 아주 오래 흐르면 시스템은 어떤 평형 상태에 도달하는가?
\end{enumerate}
\end{tcolorbox}

%===========================================================
% 5. 용어 정리 표
%===========================================================
\section{필수 용어 사전}

\begin{center}
\begin{tabular}{|c|l|l|}
\hline
\rowcolor{mainblue!20} \textbf{기호} & \textbf{용어} & \textbf{직관적 의미} \\ \hline
State ($i$) & 상태 & 시스템이 머무를 수 있는 위치 (예: 맑음, 흐림, 비) \\ \hline
$p_{ij}$ & 전이 확률 & 상태 $i$에서 $j$로 한 번에 건너갈 확률 \\ \hline
$P$ & 전이 행렬 & 모든 $p_{ij}$를 모아놓은 지도 (행의 합은 1) \\ \hline
Recurrent & 재귀 상태 & 언젠가는 반드시 다시 돌아오는 상태 (집) \\ \hline
Transient & 일시적 상태 & 떠나면 다시는 못 돌아올 수도 있는 상태 (호텔 로비) \\ \hline
$\pi_j$ & 정상 상태 확률 & 먼 미래($n \to \infty$)에 내가 $j$에 있을 확률 \\ \hline
\end{tabular}
\end{center}

%===========================================================
% 6. 핵심 개념 1: 마르코프 성질과 전이 행렬
%===========================================================
\section{핵심 개념 1: 개구리의 점프 (Markov Property)}

[Image of Markov Chain state transition diagram with 3 nodes and arrows]

\subsection{마르코프 성질 (The Markov Property)}
\begin{conceptbox}{과거 세탁의 원칙}
\begin{itemize}
    \item \textbf{정의:} 내일의 상태($X_{n+1}$)를 예측하는 데 필요한 정보는 오직 오늘($X_n$)뿐입니다. 어제($X_{n-1}$) 날씨가 어땠는지는 상관없습니다.
    \item \textbf{수식:}
    \[ P(X_{n+1} = j \mid X_n = i, X_{n-1}, \dots) = P(X_{n+1} = j \mid X_n = i) = p_{ij} \]
    \item \textbf{의미:} 모든 "역사(History)" 정보를 현재의 "상태(State)" 변수 하나에 압축해 넣어야 모델링이 성공한 것입니다.
\end{itemize}
\end{conceptbox}

\subsection{전이 행렬 (Transition Matrix $P$)}
확률을 표(Matrix)로 정리합니다.
\[ P = \begin{bmatrix} p_{11} & p_{12} \\ p_{21} & p_{22} \end{bmatrix} \]
\begin{itemize}
    \item \textbf{규칙:} 각 행(Row)의 합은 반드시 1이어야 합니다. (어디론가는 가야 하니까요.)
    \item \textbf{n-단계 전이:} $n$번 점프해서 $i \to j$로 갈 확률은 행렬을 $n$번 곱한 $P^n$의 성분과 같습니다. (행렬 연산의 강력함!)
\end{itemize}

%===========================================================
% 7. 핵심 개념 2: 상태의 분류 (Classification)
%===========================================================
\section{핵심 개념 2: 구조 파악하기 (Topology)}

계산기를 들기 전에, 반드시 \textbf{그림(다이어그램)}을 그려서 화살표를 따라가 봐야 합니다.

[Image of Recurrent vs Transient states diagram]

\begin{itemize}
    \item \textbf{재귀 (Recurrent):} "개미지옥". 한번 들어가면 그 집합 안에서만 뱅뱅 돌고, 절대 밖으로 못 나옵니다. 무한히 방문합니다.
    \item \textbf{일시적 (Transient):} "환승역". 잠시 머물 수는 있지만, 언젠가는 떠나서 다시는 돌아오지 않습니다. 방문 횟수가 유한합니다.
    \item \textbf{흡수 (Absorbing):} 나한테서 나로 가는 확률이 100\%($p_{ii}=1$)인 재귀 상태. (블랙홀)
\end{itemize}

\begin{warningbox}{접근 가능성(Accessibility) 체크}
서로 왔다 갔다 할 수 있는 상태끼리 묶어서 '클래스(Class)'라고 부릅니다.
마르코프 체인 문제를 풀 땐 가장 먼저 \textbf{"전체 시스템이 몇 개의 덩어리(Recurrent Classes)로 쪼개져 있는가?"}를 파악해야 합니다.
\end{warningbox}

%===========================================================
% 8. 핵심 개념 3: 정상 상태 확률 (Steady State)
%===========================================================
\section{핵심 개념 3: 먼 미래의 평형 (Long-run Equilibrium)}

시간이 아주 오래 흐르면($n \to \infty$), 내가 어디에 있을지 예측할 수 있을까요?

\subsection{평형 방정식 (Balance Equations)}
수조의 물 높이가 일정하다면, \textbf{"들어오는 물 = 나가는 물"}이어야 합니다.

\begin{conceptbox}{Global Balance Equation}
상태 $j$에 대해:
\[ \pi_j = \sum_{k} \pi_k p_{kj} \]
\textbf{(나가는 양) = (들어오는 양의 합)}
\begin{itemize}
    \item $\pi_j$: 내가 상태 $j$에 있을 확률 (다음 턴에 무조건 나가므로 유출량과 같음)
    \item $\sum \pi_k p_{kj}$: 다른 모든 곳($k$)에서 나($j$)한테 올 확률의 합.
\end{itemize}
\end{conceptbox}

\begin{mathstep}{정상 상태($\pi$) 구하는 법}
\begin{enumerate}
    \item \textbf{연립 방정식 세우기:} 모든 상태 $j$에 대해 $\pi_j = \sum \pi_k p_{kj}$를 씁니다.
    \item \textbf{정규화 조건 추가:} 위 식만으로는 해가 안 나옵니다. 반드시 아래 식을 추가하세요.
    \[ \sum_{j} \pi_j = 1 \]
    \item \textbf{풀기:} 연립 방정식을 풀면 유일한 해 $\pi$가 나옵니다. (단, 단일 재귀 클래스일 때)
\end{enumerate}
\end{mathstep}

%===========================================================
% 9. 핵심 개념 4: 흡수 확률과 시간 (First Step Analysis)
%===========================================================
\section{핵심 개념 4: 첫 스텝 분석법 (First Step Analysis)}

MIT 6.431에서 가장 우아하고 강력한 문제 해결 테크닉입니다. 무한 급수를 계산하는 대신, \textbf{"딱 한 발자국"}만 생각해서 재귀식을 세웁니다.

\textbf{문제:} 상태 $i$에서 시작해서 흡수 상태(Goal)까지 가는 평균 시간 $\mu_i$는?
\[ \mu_i = 1 + \sum_{j} p_{ij} \mu_j \]
\begin{itemize}
    \item $1$: 일단 한 발자국 움직임 (시간 1 소모).
    \item $\sum p_{ij} \mu_j$: 그 다음 도착한 곳($j$)에서 겪게 될 평균 시간의 가중 평균.
\end{itemize}
이 연립 방정식을 풀면 복잡한 미로 찾기의 평균 시간을 아주 쉽게 구할 수 있습니다.

%===========================================================
% 10. 실전 시나리오
%===========================================================
\section{실전 응용: 게임 유저 이탈 분석 (Churn Analysis)}

[Image of User Funnel Markov Chain: Install -> Tutorial -> Play -> Purchase or Churn]

\begin{storybox}{Nexon 모바일 게임 유저 분석}
유저의 상태를 다음과 같이 정의합니다.
\begin{itemize}
    \item 상태 1: 튜토리얼 (시작)
    \item 상태 2: 일반 플레이 (Recurrent 같지만 Transient임)
    \item 상태 3: \textbf{구매 (Absorbing Goal)} - 우리가 원하는 목표
    \item 상태 4: \textbf{삭제/이탈 (Absorbing Bad)} - 원치 않는 결과
\end{itemize}

\textbf{질문 1: 튜토리얼을 시작한 유저가 결국 구매(상태 3)에 도달할 확률은?}
\begin{itemize}
    \item 흡수 확률 $a_i$에 대한 First Step Analysis 방정식을 세워 풉니다.
    \item $a_1 = p_{12}a_2 + p_{14}a_4$ (여기서 $a_3=1, a_4=0$)
\end{itemize}

\textbf{질문 2: 구매까지 평균 몇 번의 플레이(Click)를 하는가?}
\begin{itemize}
    \item 기대 시간 $\mu_i$ 방정식을 세워 풉니다.
    \item 이 데이터를 바탕으로 "튜토리얼 난이도를 낮춰야 구매 전환율이 오르겠구나" 같은 의사결정을 합니다.
\end{itemize}
\end{storybox}

%===========================================================
% 11. 요약 및 마무리
%===========================================================
\section{Course Summary: 확률론의 여정을 마치며}

\begin{tcolorbox}[colback=mainblue!10, colframe=mainblue, title={\textbf{📝 Unit 3-9 핵심 요약}}]
\begin{enumerate}
    \item \textbf{모델링:} "상태(State)"를 잘 정의하면 과거를 잊을 수 있다(Markov).
    \item \textbf{구조:} 한번 갇히면 못 나오는 방(Recurrent)과 지나가는 방(Transient)을 구분해라.
    \item \textbf{평형:} $\pi P = \pi$. 들어오는 만큼 나간다. (장기적 확률)
    \item \textbf{첫 스텝 분석:} 무한히 더하지 말고, 재귀 방정식($x = 1 + \sum px$)을 세워라.
\end{enumerate}
\end{tcolorbox}

\vspace{0.5cm}
\textbf{[Final Comment]} \\
수고하셨습니다! 🎉 \\
우리는 \textbf{불확실성을 집합으로 정의}하는 것부터 시작해(Unit 1), \textbf{확률 변수라는 함수}로 세상을 모델링하고(Unit 2), 무한한 데이터 속에서 \textbf{극한의 법칙}을 발견하고, 마지막으로 \textbf{시간에 따른 변화(Process)}까지 다루었습니다(Unit 3).

이제 여러분은 "불확실한 세상"을 두려워하는 것이 아니라, \textbf{수학적 모델로 구조화하고 예측할 수 있는 눈}을 가지게 되었습니다. 이 지식은 머신러닝, 통계, 금융, 공학 어디서든 가장 든든한 무기가 될 것입니다.

\end{document}