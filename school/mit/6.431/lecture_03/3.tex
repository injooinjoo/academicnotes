\documentclass[a4paper, 11pt]{book}

%===========================================================
% 패키지 설정 (Chapter 1, 2와 동일한 스타일 유지)
%===========================================================
\usepackage[utf8]{inputenc}
\usepackage[T1]{fontenc}
\usepackage{kotex}
\usepackage{amsmath, amssymb, amsfonts, amsthm}
\usepackage{geometry}
\geometry{left=25mm, right=25mm, top=30mm, bottom=30mm}
\usepackage{xcolor}
\usepackage{graphicx}
\usepackage{adjustbox}  % 표/박스 크기 조절
\usepackage{hyperref}
\usepackage{booktabs}
\usepackage{enumitem}
\usepackage[most]{tcolorbox}
\tcbuselibrary{breakable}

%===========================================================
% 커스텀 스타일 정의
%===========================================================
\definecolor{mainblue}{RGB}{0, 80, 160}
\definecolor{subgray}{RGB}{240, 240, 240}
\definecolor{alertred}{RGB}{200, 50, 50}
\definecolor{examplegreen}{RGB}{50, 150, 50}
\definecolor{storyorange}{RGB}{255, 140, 0}
\definecolor{purplebox}{RGB}{120, 80, 180}

\newcommand{\chaptertitle}[2]{
    \begin{tcolorbox}[colback=mainblue, colframe=mainblue, sharp corners, boxrule=0pt, top=10pt, bottom=10pt]
        \centering \Large \bfseries \textcolor{white}{#1: #2}
    \end{tcolorbox}
}

\newtcolorbox{conceptbox}[1]{
    colback=white, colframe=mainblue, fonttitle=\bfseries,
    title={💡 #1}, rounded corners, drop shadow
}

\newtcolorbox{warningbox}[1]{
    colback=red!5!white, colframe=alertred, fonttitle=\bfseries,
    title={⚠️ 잠깐! 오해하기 쉬워요: #1}, rounded corners
}

\newtcolorbox{examplebox}[1]{
    colback=green!5!white, colframe=examplegreen, fonttitle=\bfseries,
    title={📝 예시: #1}, rounded corners
}

\newtcolorbox{storybox}[1]{
    colback=yellow!10!white, colframe=storyorange, fonttitle=\bfseries,
    title={📖 Story Matching: #1}, rounded corners, fontupper=\small
}

\newtcolorbox{toolbox}[1]{
    colback=purplebox!5!white, colframe=purplebox, fonttitle=\bfseries,
    title={🛠️ 강력한 도구: #1}, rounded corners
}

\begin{document}

%===========================================================
% 1. 전체 목차 (TOC) - 구조 명시
%===========================================================
\tableofcontents
\newpage

\section*{📚 이 교재의 전체 구조 (Roadmap)}
\begin{itemize}
    \item \textbf{Unit 1: 확률의 기초와 이산 확률 변수}
    \begin{itemize}
        \item Chapter 1: 확률 모형의 수립과 공리 (완료)
        \item Chapter 2: 조건부 확률과 베이즈 정리 (완료)
        \item[$\to$] \textbf{\textcolor{mainblue}{Chapter 3: 이산 확률 변수와 기댓값 (Discrete Random Variables) \textit{<-- 현재 위치}}}
    \end{itemize}
    \item Unit 2: 일반 확률 변수 (General Random Variables)
    \item Unit 3: 확률 과정과 극한 정리 (Random Processes)
\end{itemize}
\vspace{1cm}

%===========================================================
% 2. 현재 단원 제목
%===========================================================
\chaptertitle{Unit 1 - Chapter 3}{이산 확률 변수와 분포의 세계}

%===========================================================
% 3. 이전 단원과의 연결 \& 4. 개요
%===========================================================
\section*{Intro: 결과(Outcome)를 숫자(Number)로 바꾸다}

\textbf{[이전 단계와의 연결]} \\
Chapter 2까지 우리는 "앞면/뒷면", "성공/실패" 같은 추상적인 사건(Event)을 다뤘습니다. 하지만 데이터 분석이나 공학에서는 \textbf{"그래서 성공을 몇 번 했는데?"}, \textbf{"평균적으로 얼마를 버는데?"}와 같이 \textbf{숫자}로 요약된 정보가 필요합니다. 이제 우리는 추상적인 사건을 숫자로 매핑(Mapping)하는 법을 배웁니다.

\begin{tcolorbox}[colback=subgray, title={\textbf{📌 이 단원의 핵심 개요 (Overview)}}]
\begin{enumerate}
    \item \textbf{확률 변수 (RV):} 현실의 결과를 숫자로 바꿔주는 '함수'를 정의합니다.
    \item \textbf{PMF와 CDF:} 확률이 어떻게 분포해 있는지 그래프(질량/누적)로 그립니다.
    \item \textbf{기댓값과 분산:} 전체 데이터를 '중심'과 '퍼짐'이라는 두 개의 숫자로 요약합니다.
    \item \textbf{분포 동물원 (The Zoo):} 이항, 기하, 포아송 등 상황별로 꺼내 쓰는 표준 모델을 배웁니다.
\end{enumerate}
\end{tcolorbox}

%===========================================================
% 5. 용어 정리 표
%===========================================================
\section{필수 용어 사전}

\begin{center}
\begin{tabular}{|c|l|l|}
\hline
\rowcolor{mainblue!20} \textbf{기호} & \textbf{용어} & \textbf{직관적 의미} \\ \hline
$X$ & 확률 변수 (Random Variable) & 결과를 숫자로 바꿔주는 함수 ($HH \to 2$) \\ \hline
$p_X(k)$ & 확률 질량 함수 (PMF) & 숫자 $k$가 나올 확률의 높이 (무게) \\ \hline
$F_X(x)$ & 누적 분포 함수 (CDF) & $x$까지 쌓아 올린 확률의 합 \\ \hline
$E[X]$ & 기댓값 (Expectation) & 확률을 무게로 쳤을 때의 무게 중심 (평균) \\ \hline
$\text{var}(X)$ & 분산 (Variance) & 중심으로부터 데이터가 얼마나 퍼져 있는가 \\ \hline
\end{tabular}
\end{center}

%===========================================================
% 6. 핵심 개념 1: 확률 변수와 PMF
%===========================================================
\section{핵심 개념 1: 확률 변수는 '함수'다}

\subsection{확률 변수의 정의 (Mapping)}
\begin{conceptbox}{Outcome $\to$ Number 변환기}
\begin{itemize}
    \item \textbf{한 줄 요약:} 세상의 모든 결과(Outcome)에 꼬리표(숫자)를 붙이는 규칙입니다.
    \item \textbf{직관적 비유:} \textbf{자판기}입니다. 버튼(Outcome: 앞면, 앞면)을 누르면 음료수(Number: 2)가 나옵니다.
    \item \textbf{정의:} 표본 공간 $\Omega$의 각 원소를 실수 $\mathbb{R}$로 대응시키는 함수 $X(\omega)$.
\end{itemize}
\end{conceptbox}

\begin{examplebox}{동전 2개 던지기}
\begin{itemize}
    \item 표본 공간 $\Omega = \{ HH, HT, TH, TT \}$ (4가지 결과)
    \item \textbf{확률 변수 $X$ 정의:} "앞면의 개수"
    \item \textbf{매핑 과정:}
    \begin{itemize}
        \item $X(HH) = 2$
        \item $X(HT) = 1, X(TH) = 1$
        \item $X(TT) = 0$
    \end{itemize}
    \item 이제 우리는 $\{HH, \dots\}$가 아니라 숫자 $0, 1, 2$를 다룹니다.
\end{itemize}
\end{examplebox}

\subsection{PMF (확률 질량 함수) vs CDF (누적 분포 함수)}

\begin{itemize}
    \item \textbf{PMF ($p_X(k)$):} "정확히 $k$일 확률은?" (막대 그래프의 높이)
        \[ \sum p_X(k) = 1 \]
    \item \textbf{CDF ($F_X(x)$):} "$x$ 이하일 확률은?" (계단식으로 쌓이는 그래프)
        \[ F_X(x) = P(X \le x) = \sum_{k \le x} p_X(k) \]
\end{itemize}

%===========================================================
% 7. 핵심 개념 2: 기댓값과 분산
%===========================================================
\section{핵심 개념 2: 데이터를 요약하는 기술}

\subsection{기댓값 (Expectation, $E[X]$)}
기댓값은 단순한 '산술 평균(N빵)'이 아닙니다. \textbf{확률 가중치가 반영된 무게 중심}입니다.

\begin{itemize}
    \item \textbf{공식:} $E[X] = \sum x \cdot p_X(x)$
    \item \textbf{비유:} 시소(See-saw)의 받침점. 확률(무게)이 큰 쪽으로 중심이 이동합니다.
\end{itemize}

\begin{toolbox}{기댓값의 선형성 (Linearity of Expectation)}
복잡한 문제를 푸는 마법의 열쇠입니다.
\[ E[X + Y] = E[X] + E[Y] \]
\textbf{중요:} $X$와 $Y$가 독립이 아니어도(상관 있어도) 무조건 성립합니다.
\begin{itemize}
    \item \textbf{활용 팁:} 전체 $E[X]$를 구하기 어려우면, 문제를 아주 작은 단위(지시 변수, Indicator)로 쪼개서 각각 기댓값을 구한 뒤 더하세요.
\end{itemize}
\end{toolbox}

\subsection{분산 (Variance, $\text{var}(X)$)}
\begin{itemize}
    \item \textbf{개념:} 데이터가 기댓값 주위에 모여 있나(안정적), 퍼져 있나(Risk)?
    \item \textbf{계산 꿀팁:} 정의대로 계산하면 복잡합니다. 아래 공식을 쓰세요.
    \[ \text{var}(X) = E[X^2] - (E[X])^2 \]
    (제곱의 평균 - 평균의 제곱)
\end{itemize}

%===========================================================
% 8. 핵심 개념 3: 분포의 동물원 (The Zoo)
%===========================================================
\section{핵심 개념 3: 주요 이산 분포 (Story Matching)}
공식을 외우기 전에, 어떤 \textbf{상황(Story)}에서 이 분포를 쓰는지 파악해야 합니다.

\begin{storybox}{1. 베르누이 \& 이항 분포 (Binomial)}
\textbf{"동전 던지기의 반복"}
\begin{itemize}
    \item \textbf{상황:} 성공 확률 $p$인 실험을 $n$번 독립적으로 반복했을 때, 성공 횟수 $X$.
    \item \textbf{예시:} 고객 100명($n$)에게 메일을 보냈을 때, 클릭($p=0.1$)한 사람의 수.
    \item \textbf{기댓값:} $np$ (100명 $\times$ 0.1 = 10명)
\end{itemize}
\end{storybox}

\begin{storybox}{2. 기하 분포 (Geometric)}
\textbf{"성공할 때까지 도전!" (존버 정신)}
\begin{itemize}
    \item \textbf{상황:} 성공 확률 $p$인 실험을 \textbf{성공할 때까지} 계속할 때, 시도 횟수 $X$.
    \item \textbf{예시:} 소개팅에서 마음에 드는 사람($p=0.01$)을 만날 때까지 나가는 횟수.
    \item \textbf{특징 (Memoryless):} "지금까지 10번 차였다고 해서, 11번째 확률이 높아지는 건 아니다." (슬프지만 수학적 사실)
\end{itemize}
\end{storybox}

\begin{storybox}{3. 포아송 분포 (Poisson)}
\textbf{"매우 드문 사건의 카운팅"}
\begin{itemize}
    \item \textbf{상황:} 이항 분포에서 횟수 $n \to \infty$, 확률 $p \to 0$인 극한 상황.
    \item \textbf{예시:} 
    \begin{itemize}
        \item 책 한 페이지의 오타 수 (글자 수는 많고, 오타 확률은 낮음)
        \item 1시간 동안 콜센터에 걸려오는 전화 수
    \end{itemize}
    \item \textbf{특징:} 평균($\lambda$)과 분산($\lambda$)이 같습니다.
\end{itemize}
\end{storybox}

%===========================================================
% 9. 예시 시나리오 (실전 적용)
%===========================================================
\section{실전 응용: 게임 아이템 뽑기 (Gacha)}

당신은 게임 기획자입니다. 전설 아이템이 나올 확률은 $1\%(p=0.01)$입니다.

\textbf{Q1. 유저가 전설템을 먹을 때까지 평균 몇 번 뽑아야 할까? (Geometric)}
\[ E[X] = \frac{1}{p} = \frac{1}{0.01} = 100\text{번} \]
$\to$ "평균 100번은 시도해야 나옵니다."

\textbf{Q2. 유저가 50번 뽑기를 했을 때, 전설템을 1개도 못 먹을 확률은? (Binomial)}
\begin{itemize}
    \item 50번 시도($n=50$), 성공 횟수 $k=0$
    \item $P(X=0) = \binom{50}{0} (0.01)^0 (0.99)^{50} \approx 0.605$
\end{itemize}
$\to$ "60.5\%의 유저는 50번을 뽑아도 꽝입니다."

\textbf{Q3. 서버에 1분당 평균 5명의 접속자가 온다. 1분 동안 아무도 안 올 확률은? (Poisson)}
\begin{itemize}
    \item 평균 $\lambda = 5$
    \item $P(X=0) = e^{-5} \frac{5^0}{0!} = e^{-5} \approx 0.0067$
\end{itemize}
$\to$ "0.67\%, 즉 거의 일어나지 않는 일입니다."

%===========================================================
% 10. 결합 PMF (Joint PMF) - 짧은 요약
%===========================================================
\section{확장: 변수가 2개일 때 (Joint PMF)}
$X$(키)와 $Y$(몸무게)처럼 두 변수를 동시에 고려할 때는, 2차원 표(Grid)를 상상하세요.
\begin{itemize}
    \item \textbf{Joint PMF:} 각 칸 $(x,y)$의 확률. $\sum \sum p_{X,Y}(x,y) = 1$.
    \item \textbf{Marginal PMF:} $X$만 알고 싶으면, $Y$ 축을 따라 확률을 다 더해버리면(압축) 됩니다.
\end{itemize}

%===========================================================
% 11. 요약 및 다음 단계
%===========================================================
\section{요약 및 다음 단계}

\begin{tcolorbox}[colback=mainblue!10, colframe=mainblue, title={\textbf{📝 Unit 1-3 핵심 요약}}]
\begin{enumerate}
    \item \textbf{확률 변수:} 결과를 숫자로 바꾸는 함수.
    \item \textbf{기댓값:} 선형성($E[X+Y]=E[X]+E[Y]$)을 적극 활용해라.
    \item \textbf{패턴 매칭:} 
    \begin{itemize}
        \item 횟수 고정 + 성공 수 $\to$ \textbf{이항 분포}
        \item 성공할 때까지 $\to$ \textbf{기하 분포}
        \item 아주 드문 사건 $\to$ \textbf{포아송 분포}
    \end{itemize}
\end{enumerate}
\end{tcolorbox}

\vspace{0.5cm}
\textbf{[다음 단원 예고]} \\
지금까지는 1, 2, 3처럼 뚝뚝 끊어지는 '이산(Discrete)' 데이터만 다뤘습니다. 하지만 시간, 온도, 속도처럼 연속적으로 변하는 데이터는 어떻게 다룰까요? \\
Unit 2에서는 $\sum$(더하기)가 $\int$(적분)으로 바뀌는 \textbf{연속 확률 변수(Continuous Random Variables)}의 세계로 넘어갑니다.

\end{document}