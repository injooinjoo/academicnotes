\documentclass[a4paper, 11pt]{book}

%===========================================================
% 패키지 설정 (Chapter 1과 동일)
%===========================================================
\usepackage[utf8]{inputenc}
\usepackage[T1]{fontenc}
\usepackage{kotex}
\usepackage{amsmath, amssymb, amsfonts, amsthm}
\usepackage{geometry}
\geometry{left=25mm, right=25mm, top=30mm, bottom=30mm}
\usepackage{xcolor}
\usepackage{graphicx}
\usepackage{adjustbox}  % 표/박스 크기 조절
\usepackage{hyperref}
\usepackage{booktabs}
\usepackage{enumitem}
\usepackage[most]{tcolorbox}
\tcbuselibrary{breakable}

%===========================================================
% 커스텀 스타일 정의
%===========================================================
\definecolor{mainblue}{RGB}{0, 80, 160}
\definecolor{subgray}{RGB}{240, 240, 240}
\definecolor{alertred}{RGB}{200, 50, 50}
\definecolor{examplegreen}{RGB}{50, 150, 50}
\definecolor{bayespurple}{RGB}{100, 50, 150} % 베이즈 정리용 보라색

\newcommand{\chaptertitle}[2]{
    \begin{tcolorbox}[colback=mainblue, colframe=mainblue, sharp corners, boxrule=0pt, top=10pt, bottom=10pt]
        \centering \Large \bfseries \textcolor{white}{#1: #2}
    \end{tcolorbox}
}

\newtcolorbox{conceptbox}[1]{
    colback=white, colframe=mainblue, fonttitle=\bfseries,
    title={💡 #1}, rounded corners, drop shadow
}

\newtcolorbox{warningbox}[1]{
    colback=red!5!white, colframe=alertred, fonttitle=\bfseries,
    title={⚠️ 잠깐! 오해하기 쉬워요: #1}, rounded corners
}

\newtcolorbox{examplebox}[1]{
    colback=green!5!white, colframe=examplegreen, fonttitle=\bfseries,
    title={📝 예시: #1}, rounded corners
}

\newtcolorbox{storybox}[1]{
    colback=yellow!10!white, colframe=orange, fonttitle=\bfseries,
    title={📖 스토리 시나리오: #1}, rounded corners, fontupper=\small
}

\newtcolorbox{bayesbox}[1]{
    colback=bayespurple!5!white, colframe=bayespurple, fonttitle=\bfseries,
    title={🔮 #1}, rounded corners
}

\begin{document}

%===========================================================
% 1. 전체 목차 (TOC) - 구조 명시
%===========================================================
\tableofcontents
\newpage

\section*{📚 이 교재의 전체 구조 (Roadmap)}
\begin{itemize}
    \item \textbf{Unit 1: 확률의 기초와 이산 확률 변수}
    \begin{itemize}
        \item Chapter 1: 확률 모형의 수립과 공리 (완료)
        \item[$\to$] \textbf{\textcolor{mainblue}{Chapter 2: 조건부 확률과 베이즈 정리 (Conditional Probability \& Bayes) \textit{<-- 현재 위치}}}
        \item Chapter 3: 독립성과 이산 확률 변수
    \end{itemize}
    \item Unit 2: 일반 확률 변수
    \item Unit 3: 확률 과정과 극한 정리
\end{itemize}
\vspace{1cm}

%===========================================================
% 2. 현재 단원 제목
%===========================================================
\chaptertitle{Unit 1 - Chapter 2}{조건부 확률과 추론 (Inference)}

%===========================================================
% 3. 이전 단원과의 연결 \& 4. 개요
%===========================================================
\section*{Intro: 정보는 확률을 바꾼다}

\textbf{[이전 단계와의 연결]} \\
Chapter 1에서는 아무런 정보가 없는 상태에서 확률을 계산했습니다. 하지만 현실에서는 끊임없이 \textbf{"정보(단서)"}가 들어옵니다. "구름이 끼었다"는 정보를 알았을 때, 비가 올 확률은 변해야 합니다. 이번 장에서는 \textbf{정보가 주어졌을 때 확률을 업데이트하는 법}을 배웁니다.

\begin{tcolorbox}[colback=subgray, title={\textbf{📌 이 단원의 핵심 개요 (Overview)}}]
\begin{enumerate}
    \item \textbf{조건부 확률 (Renormalization):} 새로운 정보 $B$가 주어지면, $B$가 곧 새로운 우주(분모)가 됩니다.
    \item \textbf{베이즈 정리 (Inference):} 결과를 보고 원인을 역추적하는 '탐정의 도구'를 배웁니다.
    \item \textbf{독립성 (Independence):} 정보가 가치가 있는지(확률에 영향을 주는지) 판단합니다.
\end{enumerate}
\end{tcolorbox}

%===========================================================
% 5. 용어 정리 표
%===========================================================
\section{필수 용어 사전}

\begin{center}
\begin{tabular}{|c|l|l|}
\hline
\rowcolor{mainblue!20} \textbf{기호} & \textbf{용어} & \textbf{직관적 의미} \\ \hline
$P(A|B)$ & 조건부 확률 & $B$가 일어났다는 전제 하에 $A$가 일어날 확률 \\ \hline
$P(A \cap B)$ & 결합 확률 & $A$와 $B$가 동시에 일어날 확률 (교집합) \\ \hline
Partition & 분할 & 전체를 겹치지 않게 조각내는 것 (케이크 자르기) \\ \hline
Prior & 사전 확률 & 데이터를 보기 전의 믿음 ($P(\text{원인})$) \\ \hline
Posterior & 사후 확률 & 데이터를 본 후 수정된 믿음 ($P(\text{원인}|\text{결과})$) \\ \hline
Independent & 독립 & $B$를 아는 것이 $A$ 예측에 도움이 안 됨 ($P(A|B)=P(A)$) \\ \hline
\end{tabular}
\end{center}

%===========================================================
% 6. 핵심 개념 1: 조건부 확률
%===========================================================
\section{핵심 개념 1: 조건부 확률 (Conditional Probability)}

\subsection{표본 공간의 재정의 (Renormalization)}
\begin{conceptbox}{새로운 정보는 세상을 축소시킨다}
\begin{itemize}
    \item \textbf{한 줄 요약:} 사건 $B$가 일어났다면, $B$ 바깥의 세상은 소멸하고 $B$가 새로운 전체($\Omega_{new}$)가 됩니다.
    \item \textbf{직관적 비유:} 지도 어플에서 '서울시'를 검색하면, 지도 화면(표본 공간)이 대한민국 전체에서 서울시로 확대(Zoom-in)되는 것과 같습니다.
    \item \textbf{공식:}
    \[ P(A|B) = \frac{P(A \cap B)}{P(B)} \]
    (분모가 1이 아니라 $P(B)$로 바뀜 $\to$ 이것이 Renormalization!)
\end{itemize}
\end{conceptbox}

\begin{examplebox}{주사위 던지기}
주사위를 던졌는데 \textbf{"짝수가 나왔다(B)"}는 힌트를 받았습니다. 이때 그 숫자가 \textbf{"2 이하(A)"}일 확률은?
\begin{enumerate}
    \item 원래 세상($\Omega$): $\{1, 2, 3, 4, 5, 6\}$. 확률은 $2/6 = 1/3$.
    \item 바뀐 세상($B$): $\{2, 4, 6\}$. 이제 전체는 3개입니다.
    \item 겹치는 부분($A \cap B$): $\{2\}$ 하나뿐입니다.
    \item 계산: $P(A|B) = \frac{1}{3}$. (3개 중 1개)
\end{enumerate}
\end{examplebox}

\subsection{곱셈 법칙 (Multiplication Rule)}
조건부 확률 식을 살짝 바꾸면, 사건이 \textbf{순차적}으로 일어나는 과정을 설명할 수 있습니다.
\[ P(A \cap B) = P(B) \cdot P(A|B) \]
\textbf{해석:} "먼저 $B$가 일어나고, 그 상황 속에서 $A$가 일어난다."

%===========================================================
% 7. 핵심 개념 2: 베이즈 정리 (Inference)
%===========================================================
\section{핵심 개념 2: 전체 확률과 베이즈 정리 (The Core)}

이 부분은 MIT 수업에서 가장 강조하는 \textbf{"시스템적 추론(Inference)"}의 핵심입니다.

\subsection{전체 확률의 정리 (Total Probability Theorem)}
복잡한 문제를 해결하는 전략: \textbf{"분할 정복(Divide and Conquer)"}
\begin{itemize}
    \item $B$가 일어날 확률을 한 번에 구하기 어렵다면, 원인별 시나리오($A_1, A_2, \dots$)로 쪼개서 계산한 뒤 합칩니다.
    \[ P(B) = P(A_1)P(B|A_1) + P(A_2)P(B|A_2) + \dots \]
\end{itemize}

\subsection{베이즈 정리 (Bayes' Rule)}
\begin{bayesbox}{시간을 거스르는 추론}
\begin{itemize}
    \item \textbf{핵심 사고:} 결과를 관측($B$)한 뒤, 그 원인이 무엇이었는지($A$) 역추적합니다.
    \item \textbf{공식:}
    \[ P(A_i|B) = \frac{P(A_i)P(B|A_i)}{P(B)} \]
    \item \textbf{구조:}
    \[ \text{사후 확률(Posterior)} = \frac{\text{사전 확률(Prior)} \times \text{우도(Likelihood)}}{\text{전체 확률(Evidence)}} \]
\end{itemize}
\end{bayesbox}

\begin{storybox}{공장 불량품 추적 시나리오}
당신은 품질 관리자입니다.
\begin{itemize}
    \item \textbf{공장 A:} 전체 부품의 60\% 생산 ($P(A)=0.6$), 불량률 1\% ($P(E|A)=0.01$)
    \item \textbf{공장 B:} 전체 부품의 40\% 생산 ($P(B)=0.4$), 불량률 2\% ($P(E|B)=0.02$)
\end{itemize}
랜덤하게 뽑은 부품이 \textbf{불량(Error, E)}이었습니다. 이 부품이 \textbf{공장 A}에서 왔을 확률은?

\textbf{1단계: 전체 불량률 구하기 (분모)}
\[ P(E) = (0.6 \times 0.01) + (0.4 \times 0.02) = 0.006 + 0.008 = 0.014 \]
(전체 중 1.4\%가 불량)

\textbf{2단계: 베이즈 정리 적용 (역추적)}
\[ P(A|E) = \frac{P(A)P(E|A)}{P(E)} = \frac{0.006}{0.014} = \frac{3}{7} \approx 42.8\% \]

\textbf{결론:} 공장 A가 점유율은 더 높지만(60\%), 불량품이 나왔다는 \textbf{증거(Evidence)}를 보고 나니 공장 A일 확률이 42.8\%로 줄어들었습니다. (공장 B일 확률이 높아짐)
\end{storybox}

%===========================================================
% 8. 핵심 개념 3: 독립성
%===========================================================
\section{핵심 개념 3: 사건의 독립성 (Independence)}

\begin{conceptbox}{정보의 가치 평가}
\begin{itemize}
    \item \textbf{정의:} $P(A \cap B) = P(A)P(B)$
    \item \textbf{의미:} $P(A|B) = P(A)$. 즉, "$B$를 알게 되어도 $A$에 대한 나의 믿음은 변하지 않는다."
    \item 예: "어제 주식 가격이 올랐다($B$)"는 정보는 "오늘 내가 점심에 카레를 먹을지($A$)"를 예측하는 데 아무런 도움이 안 됩니다. (독립)
\end{itemize}
\end{conceptbox}

\begin{warningbox}{독립(Independent) vs 배반(Disjoint)}
이 둘을 절대 혼동하면 안 됩니다!
\begin{itemize}
    \item \textbf{배반 (Disjoint):} $A$와 $B$는 겹치지 않음 ($A \cap B = \emptyset$).
        \begin{itemize}
            \item $A$가 일어나면 $B$는 \textbf{절대} 안 일어남.
            \item 즉, $A$는 $B$에 대해 \textbf{엄청난 정보}를 줌. $\to$ \textbf{강한 종속성!}
        \end{itemize}
    \item \textbf{독립 (Independent):} $A$가 일어나든 말든 $B$ 발생 확률은 그대로임.
    \item \textbf{결론:} 배반 사건은 (확률이 0이 아닌 이상) \textbf{절대로 독립일 수 없습니다.}
\end{itemize}
\end{warningbox}

%===========================================================
% 9. 실전 적용 시나리오
%===========================================================
\section{실전 응용: 스팸 필터 (Naive Bayes)}

이메일에 "무료"라는 단어가 들어있을 때($B$), 이것이 스팸($A$)일 확률을 어떻게 구할까요?
\begin{enumerate}
    \item 과거 데이터를 통해 평소 스팸 올 확률 $P(A)$를 구합니다. (Prior)
    \item 스팸 메일 중에서 "무료"라는 단어가 나올 확률 $P(B|A)$를 구합니다. (Likelihood)
    \item "무료"라는 단어가 포함된 메일이 도착했습니다. 베이즈 정리를 돌려 $P(A|B)$를 계산합니다.
    \item 이 확률이 95\%를 넘으면 스팸함으로 보냅니다.
\end{enumerate}
우리가 쓰는 지메일(Gmail) 필터의 기초 원리가 바로 이것입니다.

%===========================================================
% 10. FAQ
%===========================================================
\section{초심자가 자주 묻는 질문 (FAQ)}

\begin{description}
    \item[Q1. $P(A|B)$와 $P(B|A)$는 같은 거 아닌가요?] \hfill \\
    완전히 다릅니다! 이를 \textbf{"검사의 오류(Prosecutor's Fallacy)"}라고 합니다.
    \begin{itemize}
        \item $P(\text{죽음}|\text{상어에 물림}) \approx 100\%$ (매우 높음)
        \item $P(\text{상어에 물림}|\text{죽음}) \approx 0\%$ (대부분의 죽음은 상어와 무관함)
    \end{itemize}
    순서를 바꾸면 의미가 완전히 달라집니다.
    
    \item[Q2. 독립성을 직관적으로 어떻게 아나요?] \hfill \\
    직관은 위험합니다. 반드시 $P(A \cap B) = P(A)P(B)$인지 계산해보고 판단하세요. 특히 "조건부 독립(Conditional Independence)" 상황에서는 직관이 자주 틀립니다.
\end{description}

%===========================================================
% 11. 요약 및 다음 단계
%===========================================================
\section{요약 및 다음 단계}

\begin{tcolorbox}[colback=mainblue!10, colframe=mainblue, title={\textbf{📝 Unit 1-2 핵심 요약}}]
\begin{enumerate}
    \item \textbf{조건부 확률:} 정보($B$)는 분모를 바꾼다. ($/P(B)$)
    \item \textbf{전체 확률:} 복잡하면 시나리오별로 쪼개서($\Sigma$) 계산해라.
    \item \textbf{베이즈 정리:} $P(\text{원인}|\text{결과})$를 구하려면 순서를 뒤집어라.
    \item \textbf{독립성:} $P(A|B) = P(A)$일 때만 정보다 가치가 없다. (배반과는 다르다!)
\end{enumerate}
\end{tcolorbox}

\vspace{0.5cm}
\textbf{[다음 단원 예고]} \\
지금까지는 "앞면/뒷면", "성공/실패" 같은 사건(Event) 중심이었습니다. 하지만 "앞면이 100번 중 몇 번 나왔지?"처럼 숫자로 요약하고 싶을 때가 많습니다. \\
다음 장 \textbf{Chapter 3: 이산 확률 변수(Discrete Random Variables)}에서는 사건을 숫자로 매핑하는 '확률 변수'와 'PMF'라는 강력한 도구를 만납니다.

\end{document}