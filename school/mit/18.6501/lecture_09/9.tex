\documentclass[a4paper,12pt]{article}
\usepackage{kotex}
\usepackage{amsmath, amssymb, amsthm}
\usepackage{geometry}
\usepackage{graphicx}
\usepackage{adjustbox}  % 표/박스 크기 조절
\usepackage{xcolor}
\usepackage[most]{tcolorbox}
\tcbuselibrary{breakable}
\usepackage{hyperref}
\usepackage{enumitem}
\usepackage{booktabs}
\usepackage{tabularx}
\usepackage{fancyhdr}

% 페이지 설정
\geometry{left=25mm, right=25mm, top=30mm, bottom=30mm}
\pagestyle{fancy}
\fancyhead[L]{MIT 18.6501 Unit 9}
\fancyhead[R]{Generalized Linear Models}

% 색상 정의
\definecolor{mainblue}{RGB}{0, 51, 102}
\definecolor{subblue}{RGB}{230, 240, 255}
\definecolor{warningred}{RGB}{204, 0, 0}
\definecolor{conceptgreen}{RGB}{0, 102, 51}
\definecolor{storypurple}{RGB}{102, 0, 102}

% 박스 스타일 정의
\newtcolorbox{summarybox}[1]{
  colback=subblue, colframe=mainblue, 
  title=\textbf{#1}, fonttitle=\bfseries,
  boxrule=0.5mm, arc=2mm
}

\newtcolorbox{warningbox}[1]{
  colback=white, colframe=warningred, 
  title=\textbf{⚠️ #1}, fonttitle=\bfseries,
  boxrule=0.5mm, arc=0mm,
  coltitle=white
}

\newtcolorbox{conceptbox}[1]{
  colback=white, colframe=conceptgreen, 
  title=\textbf{💡 #1}, fonttitle=\bfseries,
  boxrule=0.5mm, arc=2mm,
  coltitle=white
}

\newtcolorbox{storybox}[1]{
  colback=white, colframe=storypurple, 
  title=\textbf{🎬 #1}, fonttitle=\bfseries,
  boxrule=0.5mm, arc=2mm,
  coltitle=white
}

\title{\textbf{MIT 18.6501: 모든 데이터를 위한 통계}}
\author{Unit 9: Generalized Linear Models (GLM)}
\date{}

\begin{document}

\maketitle

% 1. 전체 목차 (TOC)
\tableofcontents
\vspace{1cm}
\hrule
\vspace{1cm}

\section*{Course Structure \& Current Focus}
\begin{itemize}
    \item Unit 3: Linear Regression (연속형 데이터 예측)
    \item Unit 7: Goodness of Fit (모델 검증)
    \item \textbf{\textcolor{mainblue}{Unit 9: Generalized Linear Models (현재 단원: 확장팩 설치)}}
    \begin{itemize}
        \item 9.1 Exponential Family (공통의 조상)
        \item 9.2 Link Function (데이터 범위 해결)
        \item 9.3 Logistic Regression (이진 분류)
        \item 9.4 Poisson Regression (카운트 데이터)
        \item 9.5 IRLS Algorithm (해를 구하는 법)
    \end{itemize}
    \item Unit 10: Classification (머신러닝으로의 연결)
\end{itemize}

\newpage

% 2. 현재 단원 제목
\section{Unit 9. 일반화 선형 모형 (GLM)}

% 3. 이전 단원과의 연결
\begin{quote}
\textit{Unit 3에서 배운 선형 회귀($Y=X\beta$)는 강력하지만 치명적인 약점이 있습니다. $Y$가 정규분포를 따라야 한다는 점이죠. 만약 "내일 비가 올까(0/1)?"를 선형 회귀로 예측했더니 "확률이 120\%"라거나 "-30\%"라는 황당한 답이 나온다면 어떡할까요? GLM은 이 문제를 해결하기 위해 선형 회귀에 \textbf{'유연한 연결고리(Link)'}를 달아주는 과정입니다.}
\end{quote}

% 4. 개요
\subsection*{📌 개요 (Overview)}
GLM은 정규분포, 베르누이 분포, 포아송 분포 등을 \textbf{지수족(Exponential Family)}이라는 하나의 수학적 틀로 묶고, \textbf{연결 함수(Link Function)}를 통해 선형 예측($X\beta$)을 데이터의 특성에 맞는 범위(예: 0~1)로 매핑하는 기법입니다. 이를 통해 분류(Classification)와 회귀(Regression)를 통합적으로 다룹니다.

% 5. 용어 정리 표
\subsection*{📝 핵심 용어 사전}
\begin{table}[h]
\centering
\begin{tabularx}{\textwidth}{|p{0.28\textwidth}|X|}
\hline
\textbf{용어 (Term)} & \textbf{직관적 의미 (Meaning)} \\
\hline
\textbf{Exponential Family} & "다르게 생겼지만 사실은 한 가족". 미분과 계산이 편한 분포들의 모임. \\
\hline
\textbf{Link Function ($g$)} & $X\beta$(무한대 범위)를 $\mu$(제한된 범위)로 통역해주는 번역기. \\
\hline
\textbf{Logit Link} & 확률(0~1)을 $\pm\infty$로 펴주는 함수. (로지스틱 회귀용) \\
\hline
\textbf{Log Link} & 양수(0~$\infty$)를 전체 실수로 펴주는 함수. (포아송 회귀용) \\
\hline
\textbf{IRLS} & GLM은 공식 한 방에 안 풀리므로, 조금씩 정답을 찾아가는 반복 알고리즘. \\
\hline
\end{tabularx}
\end{table}

\vspace{0.5cm}\hrule\vspace{0.5cm}

% 6. 핵심 개념 상세 설명
\subsection{1. 지수족 분포 (Exponential Family)}

\begin{conceptbox}{개념 1: 통계학의 만능 플랫폼}
\textbf{한 줄 요약:} 정규분포, 베르누이, 포아송 등은 겉보기엔 다르지만, 사실 $f(y) = h(y)e^{\eta T(y) - A(\eta)}$라는 \textbf{공통 DNA}를 가진 가족입니다.
\end{conceptbox}

\subsubsection*{1) 왜 배우는가?}
서로 다른 분포들을 매번 따로 연구할 필요가 없습니다. 이 "가족"에 속하기만 하면 다음과 같은 강력한 성질(VIP 혜택)을 공유하기 때문입니다.
\begin{itemize}
    \item \textbf{Convexity:} 로그 우도 함수가 항상 위로 볼록(Concave)합니다. 즉, 최적화할 때 \textbf{'가짜 봉우리(Local Optima)'가 없어서} 안심하고 답을 찾을 수 있습니다.
    \item \textbf{Automatic Moments:} 복잡한 적분 없이 미분만으로 평균과 분산을 구할 수 있습니다. ($A'(\eta) = \text{평균}, A''(\eta) = \text{분산}$)
\end{itemize}

\subsubsection*{2) 수학적 구조}
$$ f(y; \theta) = h(y) \exp\left( \eta(\theta) \cdot T(y) - A(\theta) \right) $$
\begin{itemize}
    \item $\eta$: 자연 파라미터 (Natural Parameter). 분포의 모양을 결정하는 핵심.
    \item $T(y)$: 충분 통계량.
    \item $A(\theta)$: 로그 분배 함수 (정규화 상수). 확률의 합이 1이 되게 맞추는 역할.
\end{itemize}

\vspace{0.5cm}\hrule\vspace{0.5cm}

\subsection{2. 연결 함수 (Link Function)}


\begin{conceptbox}{개념 2: 무한의 세계와 유한의 세계를 잇다}
\textbf{한 줄 요약:} 선형 회귀 결과값 $X\beta$는 $-\infty \sim \infty$ 범위를 가지지만, 우리가 원하는 평균 $\mu$는 범위가 제한적입니다(확률은 0~1). 이 둘을 이어주는 다리입니다.
\end{conceptbox}

\subsubsection*{1) 로지스틱 회귀 (Logistic Regression)}
\begin{itemize}
    \item \textbf{상황:} 합격(1) vs 불합격(0). $Y \sim \text{Bernoulli}(p)$.
    \item \textbf{문제:} $X\beta = 100$이 나오면 확률이 1을 넘어버림.
    \item \textbf{해결 (Logit Link):} 확률($p$) 대신 \textbf{로그 오즈(Log-Odds)}를 예측합니다.
    $$ g(p) = \log \left( \frac{p}{1-p} \right) = X\beta $$
    \item \textbf{복원:} 역함수를 취하면 그 유명한 \textbf{시그모이드(Sigmoid)} 함수가 됩니다.
    $$ p = \frac{e^{X\beta}}{1 + e^{X\beta}} = \frac{1}{1 + e^{-X\beta}} $$
\end{itemize}

\subsubsection*{2) 포아송 회귀 (Poisson Regression)}
\begin{itemize}
    \item \textbf{상황:} 하루 교통사고 건수. $Y \in \{0, 1, 2, ...\}$. (음수 불가능)
    \item \textbf{문제:} 선형 회귀는 음수 값을 예측할 수도 있음.
    \item \textbf{해결 (Log Link):} 평균($\lambda$)에 로그를 씌워서 예측합니다.
    $$ g(\lambda) = \log(\lambda) = X\beta \quad \implies \quad \lambda = e^{X\beta} $$
    \item \textbf{의미:} $X$가 증가하면 $Y$는 \textbf{기하급수적(Multiplicative)}으로 증가합니다.
\end{itemize}

\vspace{0.5cm}\hrule\vspace{0.5cm}

\subsection{3. 추정 알고리즘: IRLS}

\begin{conceptbox}{개념 3: 산을 오르는 반복적인 걸음}
\textbf{한 줄 요약:} 선형 회귀처럼 한 번에 답($\beta$)이 나오지 않습니다. 직선을 긋고(Linear), 가중치를 조절하고(Reweight), 다시 긋는 과정을 반복(Iterative)합니다.
\end{conceptbox}

\subsubsection*{1) 문제점}
GLM의 로그 우도 함수를 미분하면, $\beta$가 지수함수($e^{X\beta}$) 안에 갇혀 있어서 깔끔하게 정리가 안 됩니다 (No closed-form solution).

\subsubsection*{2) 해결책: 뉴턴-랩슨 \& IRLS}
근사적으로 해를 구하는 수치해석 기법을 씁니다.
\begin{itemize}
    \item \textbf{Iteratively Reweighted Least Squares (IRLS):}
    매 단계마다 분산이 다른 것을 고려하여 \textbf{가중치(Weight)를 둔 선형 회귀(WLS)}를 푸는 문제로 바꿔서 풉니다. 컴퓨터가 아주 빠르게 반복해서 최적의 $\beta$를 찾아냅니다.
\end{itemize}

\vspace{0.5cm}\hrule\vspace{0.5cm}

\section{실전 시나리오: 대학원 합격 예측 (로지스틱)}

\begin{storybox}{Scenario: GPA와 합격률의 관계}
학생들의 GPA($X$)에 따른 합격 여부($Y \in \{0, 1\}$)를 분석합니다.
\end{storybox}

\begin{enumerate}
    \item \textbf{데이터:} GPA 3.0은 불합격, 3.5는 합격... 이런 데이터 100개.
    \item \textbf{모델링:} $Y \sim \text{Bernoulli}(p)$, $\log(\frac{p}{1-p}) = \beta_0 + \beta_1 X$.
    \item \textbf{결과:} $\hat{\beta}_0 = -10, \hat{\beta}_1 = 3$.
    \item \textbf{예측 계산 (GPA 3.8인 학생):}
    \begin{itemize}
        \item 선형 예측값: $\eta = -10 + 3(3.8) = -10 + 11.4 = 1.4$.
        \item 이것은 확률이 아니라 \textbf{로그 오즈}입니다.
        \item 확률 변환: $p = \frac{1}{1 + e^{-1.4}} \approx \frac{1}{1 + 0.246} \approx 0.80$.
    \end{itemize}
    \item \textbf{해석:} GPA가 3.8인 학생은 합격 확률이 약 80\%입니다.
    \item \textbf{계수 해석:} $\beta_1=3$은 양수이므로, GPA가 높을수록 합격 오즈가 $e^3 \approx 20$배 증가합니다 (매우 강력한 영향).
\end{enumerate}

\vspace{0.5cm}\hrule\vspace{0.5cm}

\section{자주 묻는 질문 (FAQ)}

\begin{description}
    \item[Q1. 왜 확률을 바로 $X\beta$로 두지 않고 굳이 '오즈'에 로그를 씌우나요?]
    \textbf{A.} 만약 $P(Y=1) = \beta_0 + \beta_1 X$로 두면, $X$가 아주 커졌을 때 확률이 1.5가 되거나 -0.2가 되는 모순이 발생합니다. 로그 오즈는 범위가 $(-\infty, \infty)$이므로, $X\beta$와 매칭시키기에 수학적으로 가장 안전하고 자연스럽습니다.
    
    \item[Q2. 정준 연결 함수(Canonical Link)가 뭔가요?]
    \textbf{A.} '순정 부품' 같은 겁니다. 수학적으로 가장 깔끔하게 떨어지는 연결 함수 조합입니다.
    \begin{itemize}
        \item Normal $\to$ Identity ($Y=X\beta$)
        \item Bernoulli $\to$ Logit
        \item Poisson $\to$ Log
    \end{itemize}
    다른 걸 써도 되지만(예: Probit), 정준 링크를 쓰면 계산이 편하고 해석이 명확합니다.
\end{description}

% 10. 다음 단원 연결
\vspace{1cm}
\begin{quote}
\textbf{Next Step:} 우리는 GLM을 통해 $Y$가 0/1인 경우(로지스틱)까지 다루었습니다. 하지만 만약 $Y$가 0/1이 아니라 \textbf{"개, 고양이, 사자"처럼 3개 이상의 범주}라면 어떡할까요? 다음 \textbf{Unit 10}에서는 이를 해결하는 다중 분류(Multiclass Classification)와 현대적 분류 기법들을 배웁니다.
\end{quote}

% 11. 단원 요약 박스
\begin{summarybox}{Unit 9 핵심 요약}
\begin{itemize}
    \item \textbf{지수족(Exponential Family):} GLM의 수학적 기반. 로그 우도가 오목(Concave)하여 최적화가 쉽다.
    \item \textbf{연결 함수(Link Function):} $g(\mu) = X\beta$. 선형 예측값을 데이터 범위에 맞게 변환한다.
    \item \textbf{로지스틱 회귀:} $g = \text{logit}$. 이진 분류에 사용. 결과는 로그 오즈.
    \item \textbf{포아송 회귀:} $g = \log$. 카운트 데이터에 사용.
    \item \textbf{IRLS:} GLM의 해를 구하는 반복적 알고리즘.
\end{itemize}
\end{summarybox}

\end{document}