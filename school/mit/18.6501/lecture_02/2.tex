\documentclass[a4paper,12pt]{article}
\usepackage{kotex}
\usepackage{amsmath, amssymb, amsthm}
\usepackage{geometry}
\usepackage{graphicx}
\usepackage{adjustbox}  % 표/박스 크기 조절
\usepackage{xcolor}
\usepackage[most]{tcolorbox}
\tcbuselibrary{breakable}
\usepackage{hyperref}
\usepackage{enumitem}
\usepackage{booktabs}
\usepackage{tabularx}
\usepackage{fancyhdr}

% 페이지 설정
\geometry{left=25mm, right=25mm, top=30mm, bottom=30mm}
\pagestyle{fancy}
\fancyhead[L]{MIT 18.6501 Unit 2}
\fancyhead[R]{Point Estimation}

% 색상 정의
\definecolor{mainblue}{RGB}{0, 51, 102}
\definecolor{subblue}{RGB}{230, 240, 255}
\definecolor{warningred}{RGB}{204, 0, 0}
\definecolor{conceptgreen}{RGB}{0, 102, 51}
\definecolor{storypurple}{RGB}{102, 0, 102}

% 박스 스타일 정의
\newtcolorbox{summarybox}[1]{
  colback=subblue, colframe=mainblue, 
  title=\textbf{#1}, fonttitle=\bfseries,
  boxrule=0.5mm, arc=2mm
}

\newtcolorbox{warningbox}[1]{
  colback=white, colframe=warningred, 
  title=\textbf{⚠️ #1}, fonttitle=\bfseries,
  boxrule=0.5mm, arc=0mm,
  coltitle=white
}

\newtcolorbox{conceptbox}[1]{
  colback=white, colframe=conceptgreen, 
  title=\textbf{💡 #1}, fonttitle=\bfseries,
  boxrule=0.5mm, arc=2mm,
  coltitle=white
}

\newtcolorbox{storybox}[1]{
  colback=white, colframe=storypurple, 
  title=\textbf{🎬 #1}, fonttitle=\bfseries,
  boxrule=0.5mm, arc=2mm,
  coltitle=white
}

\title{\textbf{MIT 18.6501: 통계적 추론의 핵심}}
\author{Unit 2: Point Estimation (점 추정)}
\date{}

\begin{document}

\maketitle

% 1. 전체 목차 (TOC)
\tableofcontents
\vspace{1cm}
\hrule
\vspace{1cm}

\section*{Course Structure \& Current Focus}
\begin{itemize}
    \item Unit 1: Modeling, Identifiability, Likelihood (완료)
    \item \textbf{\textcolor{mainblue}{Unit 2: Point Estimation (현재 단원)}}
    \begin{itemize}
        \item 2.1 Estimator: Definition \& Intuition
        \item 2.2 Maximum Likelihood Estimation (MLE)
        \item 2.3 Method of Moments (MoM)
        \item 2.4 Properties: Bias, Variance, MSE
    \end{itemize}
    \item Unit 3: Confidence Intervals (구간 추정)
    \item Unit 4: Hypothesis Testing (가설 검정)
\end{itemize}

\newpage

% 2. 현재 단원 제목
\section{Unit 2. 점 추정 (Point Estimation)}

% 3. 이전 단원과의 연결
\begin{quote}
\textit{Unit 1에서 우리는 '우도(Likelihood)'라는 강력한 도구를 만들었습니다. 이제 질문을 던질 차례입니다. "그래서 그 우도를 가장 높게 만드는 범인($\theta$)은 정확히 누구인가?" 이번 단원에서는 미지의 파라미터를 하나의 숫자(Point)로 찍어 맞추는 방법과, 그 방법의 성적을 매기는 기준을 배웁니다.}
\end{quote}

% 4. 개요
\subsection*{📌 개요 (Overview)}
점 추정은 데이터를 함수에 넣어 미지의 파라미터 $\theta$에 대한 최적의 추측값 $\hat{\theta}$를 계산하는 과정입니다. 우리는 두 가지 주요 방법론인 \textbf{MLE(최적화 접근)}와 \textbf{MoM(대수의 법칙 접근)}을 배우고, 추정량이 좋은지 나쁜지를 판별하는 \textbf{MSE(Bias-Variance Trade-off)} 프레임워크를 익힙니다. 이것은 머신러닝의 손실 함수(Loss Function) 개념의 기초가 됩니다.

% 5. 용어 정리 표
\subsection*{📝 핵심 용어 사전}
\begin{table}[h]
\centering
\begin{tabularx}{\textwidth}{|p{0.28\textwidth}|X|}
\hline
\textbf{용어 (Term)} & \textbf{직관적 의미 (Meaning)} \\
\hline
\textbf{Estimator ($\hat{\theta}_n$)} & 데이터를 입력받아 추정값을 뱉어내는 '함수' (공식 그 자체) \\
\hline
\textbf{Estimate} & 실제 데이터를 넣어 계산된 구체적인 '숫자' \\
\hline
\textbf{Bias (편향)} & 영점 조절 상태. 평균적으로 정답을 맞히는가? \\
\hline
\textbf{Variance (분산)} & 정밀도. 데이터가 조금 바뀔 때 추정값이 얼마나 흔들리는가? \\
\hline
\textbf{MSE} & 종합 점수. 편향의 제곱과 분산을 합친 총 에러. \\
\hline
\end{tabularx}
\end{table}

\vspace{0.5cm}\hrule\vspace{0.5cm}

% 6. 핵심 개념 상세 설명
\subsection{1. 추정량(Estimator)의 정의와 본질}

\begin{conceptbox}{개념 1: 추정량은 숫자가 아니라 '함수'이자 '확률변수'다}
\textbf{한 줄 요약:} 추정량은 데이터가 들어오면 정답을 내놓는 '기계'이며, 들어오는 데이터가 랜덤하므로 기계가 내놓는 답도 랜덤하게 변합니다.
\end{conceptbox}

\subsubsection*{1) 직관적 비유: 여론조사 기관}
대통령 지지율($\theta$)을 알고 싶습니다.
\begin{itemize}
    \item \textbf{추정량($\hat{\theta}$):} "지나가는 사람 100명에게 물어보고 찬성 비율을 계산한다"라는 \textbf{규칙}입니다.
    \item \textbf{확률변수로서의 성격:} 오늘 100명을 조사했을 때와, 내일 100명을 조사했을 때 결과는 다를 것입니다. 즉, $\hat{\theta}$는 고정된 값이 아니라 \textbf{분포(Distribution)}를 가집니다.
\end{itemize}

\subsubsection*{2) 기술적 정의}
추정량 $\hat{\theta}_n$은 표본 $X_1, ..., X_n$의 함수입니다.
$$ \hat{\theta}_n = g(X_1, ..., X_n) $$
\textbf{중요 조건:} 이 함수 $g$ 안에는 미지의 파라미터 $\theta$가 들어 있으면 안 됩니다. (우리가 모르는 값을 이용해 계산할 수는 없으니까요.)

\vspace{0.5cm}\hrule\vspace{0.5cm}

\subsection{2. 최대우도추정 (MLE, Maximum Likelihood Estimation)}


\begin{conceptbox}{개념 2: 가장 그럴듯한 범인 찾기}
\textbf{한 줄 요약:} "현재의 데이터가 관찰될 확률을 수학적으로 가장 높여주는 $\theta$ 값을 정답으로 채택하자."
\end{conceptbox}

\subsubsection*{1) 직관적 비유: 등산하기}
안개 낀 산(로그 우도 함수)에서 가장 높은 봉우리(최대값)를 찾아가는 과정입니다.
\begin{itemize}
    \item 산의 높이: $\ell_n(\theta)$ (로그 우도)
    \item 전략: 기울기(미분값)가 0이 되는 지점을 찾는다.
\end{itemize}

\subsubsection*{2) 수학적 절차 (Algorithm)}
\begin{enumerate}
    \item \textbf{Log-Likelihood:} $\ell_n(\theta) = \sum_{i=1}^n \log f(X_i; \theta)$ 를 구합니다.
    \item \textbf{Differentiate:} $\theta$에 대해 미분합니다. $\frac{\partial}{\partial \theta} \ell_n(\theta)$.
    \item \textbf{Solve:} 미분값이 0이 되는 방정식(Estimating Equation)을 풉니다.
    \item \textbf{Check:} 두 번 미분하여 음수인지(위로 볼록, Concave) 확인하여 최대값임을 보장합니다.
\end{enumerate}

\subsubsection*{3) 계산 예시: 포아송 분포 (웹사이트 방문자 수)}
하루 방문자 수 $X$가 포아송 분포 $\text{Pois}(\lambda)$를 따른다고 가정합시다. ($\theta = \lambda$)
$$ P(X=x) = \frac{e^{-\lambda}\lambda^x}{x!} $$
데이터: $X_1, ..., X_n$
\begin{itemize}
    \item 로그 우도: $\ell_n(\lambda) = \sum (- \lambda + X_i \ln \lambda - \ln X_i!) = -n\lambda + (\sum X_i)\ln\lambda - C$
    \item 미분: $\frac{\partial}{\partial \lambda} \ell_n(\lambda) = -n + \frac{\sum X_i}{\lambda}$
    \item 0으로 놓기: $-n + \frac{\sum X_i}{\hat{\lambda}} = 0 \implies \hat{\lambda}_{MLE} = \frac{1}{n}\sum X_i = \bar{X}$
\end{itemize}
결론: 포아송 분포의 MLE는 직관적이게도 \textbf{표본 평균}입니다.

\vspace{0.5cm}\hrule\vspace{0.5cm}

\subsection{3. 적률법 (MoM, Method of Moments)}

\begin{conceptbox}{개념 3: 평균을 평균에 맞춘다}
\textbf{한 줄 요약:} "이론적인 평균(Population Mean)이 실제 데이터의 평균(Sample Mean)과 같아야 한다"는 단순한 믿음에서 출발합니다.
\end{conceptbox}

\subsubsection*{1) 직관적 비유: 요리 간 맞추기}
국물 맛(데이터)을 봅니다.
\begin{itemize}
    \item 이론: 소금($\theta$)을 1스푼 넣으면 짠맛 농도($E[X]$)가 10이 되어야 한다.
    \item 실제: 국물을 떠서 맛보니 짠맛 농도($\bar{X}$)가 20이다.
    \item 추론: "아, 소금이 2스푼 들어갔겠구나($\hat{\theta}=2$)."
\end{itemize}

\subsubsection*{2) 수학적 정의}
$k$번째 이론적 적률(Moment) $m_k(\theta) = \mathbb{E}[X^k]$을 구하고, 이를 표본 적률 $\frac{1}{n}\sum X_i^k$과 같다고 둡니다.
$$ \mathbb{E}[X] = \bar{X} \quad (\text{1차 모멘트 매칭}) $$
미지수가 2개면 2차 모멘트($E[X^2]$)까지 사용합니다.

\begin{warningbox}{MLE vs MoM: 누가 더 좋은가?}
일반적으로 \textbf{MLE가 더 정밀(Efficient)}합니다. 하지만 MLE는 계산이 복잡하거나 미분이 불가능할 수 있습니다. 이때 계산이 쉬운 \textbf{MoM}을 먼저 구해서 MLE를 찾기 위한 \textbf{초기값(Initial Guess)}으로 사용하는 경우가 많습니다.
\end{warningbox}

\vspace{0.5cm}\hrule\vspace{0.5cm}

\subsection{4. 추정량의 성질 (Properties): 성적표 매기기}


\begin{conceptbox}{개념 4: Bias-Variance Trade-off}
\textbf{한 줄 요약:} 좋은 추정량은 영점이 잘 잡혀있어야 하고(Low Bias), 쏠 때마다 탄착군이 좁게 모여야 합니다(Low Variance).
\end{conceptbox}

\subsubsection*{1) 편향 (Bias): 영점 조절}
$$ \text{Bias}(\hat{\theta}) = \mathbb{E}[\hat{\theta}] - \theta $$
추정량을 무한히 반복했을 때, 그 \textbf{평균}이 과연 정답($\theta$)과 일치하는가?
\begin{itemize}
    \item $Bias = 0$: 비편향 추정량 (Unbiased). 영점이 정확함.
    \item $Bias \neq 0$: 편향 추정량 (Biased). 영점이 틀어짐.
\end{itemize}

\subsubsection*{2) 분산 (Variance): 탄착군 크기}
$$ \text{Var}(\hat{\theta}) = \mathbb{E}[(\hat{\theta} - \mathbb{E}[\hat{\theta}])^2] $$
데이터가 바뀔 때마다 추정값이 얼마나 들쑥날쑥한가? (안정성)

\subsubsection*{3) MSE (Mean Squared Error): 최종 점수}
우리의 목표는 에러의 총합을 줄이는 것입니다.
$$ \text{MSE}(\hat{\theta}) = \mathbb{E}[(\hat{\theta} - \theta)^2] = \underbrace{\text{Bias}(\hat{\theta})^2}_{\text{영점 오차}} + \underbrace{\text{Var}(\hat{\theta})}_{\text{흔들림}} $$
\textbf{인사이트:} 때로는 영점이 조금 틀어지더라도(Bias 존재), 탄착군을 획기적으로 좁힐 수 있다면(Variance 감소), 그게 더 좋은 추정량일 수 있습니다. (이것이 현대 머신러닝의 핵심 철학입니다.)

\vspace{0.5cm}\hrule\vspace{0.5cm}

\section{실전 시나리오: 모바일 게임 가챠 확률 검증}

\begin{storybox}{Scenario: 유저들의 불만}
당신은 게임 회사 넥슨의 PM입니다. "전설 아이템 획득 확률($p$)이 1\%라고 했는데, 너무 안 나온다!"라는 유저 불만이 폭주합니다.
데이터를 확인해보니, 100명의 유저가 가챠를 돌렸고 그중 딱 1명만 성공했습니다. ($n=100, \sum x_i = 1$)
\end{storybox}

\begin{enumerate}
    \item \textbf{모델링:} 성공/실패이므로 베르누이 분포 $X \sim \text{Ber}(p)$입니다.
    \item \textbf{MLE 적용:} 베르누이의 MLE는 표본 평균입니다.
    $$ \hat{p}_{MLE} = \frac{1}{100} = 0.01 \, (1\%) $$
    회사 측 입장: "데이터로 추정해 보니 1\%가 맞습니다. 시스템은 정상입니다."
    
    \item \textbf{Bayesian 관점 (심화 예고):} 하지만 만약 유저가 3명만 돌려서 0명이 나왔다면? $\hat{p}=0\%$. 확률이 0이라고 단정할 수 있을까요? 이때는 "과거의 경험(Prior)"을 섞는 베이지안 방식이 필요할 수 있습니다.
\end{enumerate}

\vspace{0.5cm}\hrule\vspace{0.5cm}

\section{자주 묻는 질문 (FAQ)}

\begin{description}
    \item[Q1. 비편향(Unbiased) 추정량이 무조건 좋은 건가요?]
    \textbf{A.} 아닙니다! 비편향이지만 분산이 태평양만큼 넓다면 쓸모가 없습니다. 약간의 편향을 감수하더라도 분산이 매우 작은 추정량(예: Ridge Regression)을 선택하는 경우가 많습니다. 결국 \textbf{MSE(총 에러)}가 작은 것이 장땡입니다.

    \item[Q2. MLE는 항상 정답을 주나요?]
    \textbf{A.} 데이터가 무한히 많다면($n \to \infty$) MLE는 참값으로 수렴합니다(Consistency). 하지만 데이터가 적을 때($n$이 작을 때)는 MLE도 틀릴 수 있고, 심지어 편향(Biased)되어 있을 수도 있습니다. (예: 정규분포 분산의 MLE는 편향되어 있음).
\end{description}

% 10. 다음 단원 연결
\vspace{1cm}
\begin{quote}
\textbf{Next Step:} 우리는 점 추정을 통해 $\hat{\theta} = 0.01$이라는 숫자를 얻었습니다. 하지만 이 숫자가 \textbf{얼마나 확실한지}는 아직 모릅니다. 0.01이라곤 했지만, 실제로는 0.005일 수도 있고 0.015일 수도 있지 않을까요? 다음 \textbf{Unit 3}에서는 이 불확실성을 감안하여 정답이 있을 만한 \textbf{구간(Interval)}을 구하는 법을 배웁니다.
\end{quote}

% 11. 단원 요약 박스
\begin{summarybox}{Unit 2 핵심 요약}
\begin{itemize}
    \item \textbf{추정량(Estimator):} 데이터의 함수이자 확률변수. 분포를 가진다.
    \item \textbf{MLE:} 우도(Likelihood)를 최대화하는 값을 찾는다. (미분 $\nabla \ell = 0$)
    \item \textbf{MoM:} 표본 평균을 이론적 평균과 같다고 둔다. (계산이 쉬움)
    \item \textbf{Bias-Variance Trade-off:} $\text{MSE} = \text{Bias}^2 + \text{Var}$. 편향과 분산의 균형을 맞춰 전체 에러를 줄이는 것이 목표다.
\end{itemize}
\end{summarybox}

\end{document}