\documentclass[a4paper,12pt]{article}
\usepackage{kotex}
\usepackage{amsmath, amssymb, amsthm}
\usepackage{geometry}
\usepackage{graphicx}
\usepackage{adjustbox}  % 표/박스 크기 조절
\usepackage{xcolor}
\usepackage[most]{tcolorbox}
\tcbuselibrary{breakable}
\usepackage{hyperref}
\usepackage{enumitem}
\usepackage{booktabs}
\usepackage{tabularx}
\usepackage{fancyhdr}
\usepackage{bm}

% 페이지 설정
\geometry{left=25mm, right=25mm, top=30mm, bottom=30mm}
\pagestyle{fancy}
\fancyhead[L]{MIT 6.86x Unit 4}
\fancyhead[R]{Backpropagation}

% 색상 정의
\definecolor{mainblue}{RGB}{0, 51, 102}
\definecolor{subblue}{RGB}{230, 240, 255}
\definecolor{warningred}{RGB}{204, 0, 0}
\definecolor{conceptgreen}{RGB}{0, 102, 51}
\definecolor{storypurple}{RGB}{102, 0, 102}

% 박스 스타일 정의
\newtcolorbox{summarybox}[1]{
  colback=subblue, colframe=mainblue, 
  title=\textbf{#1}, fonttitle=\bfseries,
  boxrule=0.5mm, arc=2mm
}

\newtcolorbox{warningbox}[1]{
  colback=white, colframe=warningred, 
  title=\textbf{⚠️ #1}, fonttitle=\bfseries,
  boxrule=0.5mm, arc=0mm,
  coltitle=white
}

\newtcolorbox{conceptbox}[1]{
  colback=white, colframe=conceptgreen, 
  title=\textbf{💡 #1}, fonttitle=\bfseries,
  boxrule=0.5mm, arc=2mm,
  coltitle=white
}

\newtcolorbox{storybox}[1]{
  colback=white, colframe=storypurple, 
  title=\textbf{🎬 #1}, fonttitle=\bfseries,
  boxrule=0.5mm, arc=2mm,
  coltitle=white
}

\title{\textbf{MIT 6.86x: 오차를 통해 배우다}}
\author{Unit 4: Backpropagation \& Optimization Details}
\date{}

\begin{document}

\maketitle

% 1. 전체 목차 (TOC)
\tableofcontents
\vspace{1cm}
\hrule
\vspace{1cm}

\section*{Course Structure \& Current Focus}
\begin{itemize}
    \item Unit 3: Neural Networks Basics (구조 설계)
    \item \textbf{\textcolor{mainblue}{Unit 4: Backpropagation (현재 단원: 학습 메커니즘)}}
    \begin{itemize}
        \item 8.1 The Engine: Backpropagation Algorithm
        \item 8.2 The Scorecard: Loss Functions (MSE vs Cross-Entropy)
        \item 8.3 Setting the Stage: Initialization \& Learning Rate
        \item 8.4 Stabilization: Batch Norm \& Dropout
    \end{itemize}
    \item Unit 5: Recurrent Neural Networks (순차 데이터)
\end{itemize}

\newpage

% 2. 현재 단원 제목
\section{Unit 4. 역전파와 학습의 원리 (Backpropagation)}

% 3. 이전 단원과의 연결
\begin{quote}
\textit{Unit 3에서 우리는 수천 개의 뉴런이 연결된 뇌(FNN)를 만들었습니다. 하지만 갓 태어난 뇌는 아는 것이 없습니다(가중치 $W$가 랜덤임). 이제 이 뇌에게 정답을 보여주고, 틀릴 때마다 \textbf{"누가 잘못했는지"}를 따져서 조금씩 똑똑해지게 만들어야 합니다. 이 과정이 바로 학습(Training)입니다.}
\end{quote}

% 4. 개요
\subsection*{📌 개요 (Overview)}
이 단원에서는 신경망 학습의 핵심 엔진인 \textbf{역전파(Backpropagation)} 알고리즘을 배웁니다. 미분의 \textbf{연쇄 법칙(Chain Rule)}을 이용해 오차의 원인을 찾아내고, \textbf{손실 함수(Loss Function)}를 통해 성적을 매기며, \textbf{초기화(Initialization)}와 \textbf{정규화(Normalization)} 기법으로 학습을 안정화시키는 전체 파이프라인을 다룹니다.

% 5. 용어 정리 표
\subsection*{📝 핵심 용어 사전}
\begin{table}[h]
\centering
\begin{tabularx}{\textwidth}{|p{0.28\textwidth}|X|}
\hline
\textbf{용어 (Term)} & \textbf{직관적 의미 (Meaning)} \\
\hline
\textbf{Forward Pass} & 입력 $\to$ 출력. 예측값을 뽑아내는 과정. \\
\hline
\textbf{Backward Pass} & 출력 $\to$ 입력. 틀린 이유(미분값)를 배달하는 과정. \\
\hline
\textbf{Chain Rule} & 꼬리에 꼬리를 무는 미분. (나비효과 추적) \\
\hline
\textbf{Epoch} & 전체 문제집을 한 번 다 풀어보는 것. \\
\hline
\textbf{Dropout} & 일부 뉴런을 랜덤하게 꺼서, 특정 뉴런 편애를 막는 기술. \\
\hline
\end{tabularx}
\end{table}

\vspace{0.5cm}\hrule\vspace{0.5cm}

% 6. 핵심 개념 상세 설명
\subsection{1. 역전파 알고리즘 (Backpropagation)}

\begin{conceptbox}{개념 1: 책임 소재 따지기 (The Blame Game)}
\textbf{한 줄 요약:} 결과가 틀렸을 때, 출력층에서부터 거꾸로 거슬러 올라가며 각 뉴런이 오차에 기여한 만큼(기울기) 가중치를 수정합니다.
\end{conceptbox}

\subsubsection*{1) 작동 원리: 연쇄 법칙 (Chain Rule)}
우리는 최종 오차 $L$을 줄이기 위해 가중치 $w$를 조절하고 싶습니다. 즉, $\frac{\partial L}{\partial w}$가 필요합니다. 하지만 $L$과 $w$는 멀리 떨어져 있습니다.
$$ \frac{\partial L}{\partial w} = \underbrace{\frac{\partial L}{\partial y}}_{\text{오차 변화}} \cdot \underbrace{\frac{\partial y}{\partial h}}_{\text{출력 변화}} \cdot \underbrace{\frac{\partial h}{\partial w}}_{\text{은닉층 변화}} $$
\begin{itemize}
    \item 마치 통역을 거치듯, 미분값을 층층이 곱해서 전달합니다.
    \item **** (위 이미지 참고): $\delta$값(오차 신호)이 오른쪽($a^4$)에서 왼쪽($a^0$)으로 전파되는 것을 볼 수 있습니다.
\end{itemize}

\subsubsection*{2) 숫자 예시}
간단한 식 $L = (y - wx)^2$ 에서, $y=10, x=2, w=3$이라고 합시다.
\begin{enumerate}
    \item \textbf{순전파:} 예측 $\hat{y} = 3 \times 2 = 6$.
    \item \textbf{오차:} $L = (10 - 6)^2 = 16$.
    \item \textbf{역전파:} $w$를 조절하면 $L$이 얼마나 변할까?
    $$ \frac{\partial L}{\partial w} = 2(y - \hat{y}) \cdot (-x) = 2(4) \cdot (-2) = -16 $$
    \item \textbf{의미:} 기울기가 음수(-16)이므로, $w$를 키워야 오차가 줄어듭니다.
\end{enumerate}

\vspace{0.5cm}\hrule\vspace{0.5cm}

\subsection{2. 손실 함수 (Loss Function)}

\begin{conceptbox}{개념 2: 채점 기준표}
\textbf{한 줄 요약:} 문제의 유형(회귀 vs 분류)에 따라 "얼마나 틀렸는지"를 계산하는 방식이 다릅니다.
\end{conceptbox}

\begin{table}[h]
\centering
\begin{tabular}{l|l|l}
\toprule
\textbf{구분} & \textbf{MSE (Mean Squared Error)} & \textbf{Cross-Entropy} \\
\midrule
\textbf{용도} & \textbf{회귀} (집값, 온도 예측) & \textbf{분류} (스팸, 아이템 등급) \\
\textbf{수식} & $\frac{1}{n}\sum (y - \hat{y})^2$ & $-\sum y \log(\hat{y})$ \\
\textbf{특징} & 거리가 멀수록 제곱으로 혼냄. & 확률이 틀릴수록 급격히 혼냄. \\
\bottomrule
\end{tabular}
\end{table}

\begin{warningbox}{왜 분류 문제에 MSE를 안 쓰나요?}
Sigmoid + MSE 조합을 쓰면, 오차가 엄청 큰데도 미분값(Gradient)이 0에 가까워져서 학습이 안 되는 현상이 발생합니다. 반면 \textbf{Cross-Entropy}는 로그 함수 특성상 틀리면 미분값이 폭발적으로 커져서 "정신 차려!"라고 강하게 피드백을 줍니다.
\end{warningbox}

\vspace{0.5cm}\hrule\vspace{0.5cm}

\subsection{3. 초기화와 학습률 (Initialization \& Learning Rate)}

\begin{conceptbox}{개념 3: 시작이 반이다}
\textbf{한 줄 요약:} 출발점(초기화)을 잘 잡고, 보폭(학습률)을 적절히 조절해야 산 아래로 잘 내려갈 수 있습니다.
\end{conceptbox}

\subsubsection*{1) 가중치 초기화 (Initialization)}
\begin{itemize}
    \item \textbf{0으로 초기화하면? (The Trap of Symmetry):} 모든 뉴런이 똑같은 값을 내뱉고, 똑같이 업데이트됩니다. 100개를 써도 1개 쓰는 것과 같습니다.
    \item \textbf{해결책:}
    \begin{itemize}
        \item \textbf{He Initialization:} ReLU 함수를 쓸 때 표준. (분산을 $2/n$로 맞춤)
        \item \textbf{Xavier Initialization:} Sigmoid/Tanh를 쓸 때 표준.
    \end{itemize}
\end{itemize}

\subsubsection*{2) 학습률 스케줄링 (Learning Rate Decay)}
처음엔 성큼성큼(큰 $\eta$) 가다가, 정답 근처에서는 조심조심(작은 $\eta$) 가야 합니다.
$$ \eta_t = \eta_0 / (1 + kt) \quad (\text{시간이 갈수록 줄어듦}) $$

\vspace{0.5cm}\hrule\vspace{0.5cm}

\subsection{4. 학습 안정화 기법: 배치 정규화 \& 드롭아웃}

\begin{conceptbox}{개념 4: 흔들리지 않는 편안함}
\textbf{한 줄 요약:} 딥러닝은 깊어질수록 학습이 불안정해집니다. 이를 막기 위해 데이터를 강제로 정렬하거나(Batch Norm), 일부러 핸디캡을 줍니다(Dropout).
\end{conceptbox}

\subsubsection*{A. 배치 정규화 (Batch Normalization)}
\begin{itemize}
    \item \textbf{문제:} 앞단 레이어의 가중치가 바뀌면, 뒷단 레이어 입장에서는 들어오는 데이터 분포가 계속 바뀝니다 (Internal Covariate Shift).
    \item \textbf{해결:} 각 층마다 들어오는 데이터를 \textbf{평균 0, 분산 1}로 강제 조정해 버립니다.
    \item \textbf{효과:} 학습 속도가 비약적으로 빨라지고, 초기화에 덜 민감해집니다.
\end{itemize}

\subsubsection*{B. 드롭아웃 (Dropout)}
\begin{itemize}
    \item \textbf{개념:} 학습할 때마다 은닉층 뉴런의 50\%를 랜덤하게 꺼버립니다.
    \item \textbf{비유:} 축구팀에서 에이스(특정 뉴런) 한 명만 믿는 것을 방지하기 위해, 연습 때 에이스를 벤치에 앉혀두고 나머지 선수들끼리 훈련시킵니다. 팀 전체 전력이 올라갑니다(과적합 방지).
\end{itemize}

\vspace{0.5cm}\hrule\vspace{0.5cm}

\section{실전 시나리오: 넥슨 게임 이탈(Churn) 방지 모델}

\begin{storybox}{Scenario: VIP 유저가 떠나는 이유 찾기}
당신은 넥슨의 PM입니다. "이번 달에 접속할 것"이라고 예측(0.9)했던 VIP 유저가 실제로는 접속하지 않았습니다(Target 0.0). 오차가 큽니다.
\end{storybox}

\begin{enumerate}
    \item \textbf{Forward:} 유저 데이터(길드 활동, 결제액 등) $\to$ FNN $\to$ 예측확률 90\%.
    \item \textbf{Loss:} Cross-Entropy로 계산하니 손실이 매우 큽니다. ("확신했는데 틀렸군!")
    \item \textbf{Backward:}
    \begin{itemize}
        \item 역전파가 시작됩니다. 출력층에서 은닉층으로 거슬러 올라갑니다.
        \item 범인 색출: "길드 활동 점수"에 연결된 가중치가 범인이었습니다. "길드 활동이 많으면 무조건 남는다"고 과대평가하고 있었네요.
    \end{itemize}
    \item \textbf{Update:} 해당 가중치를 대폭 삭감합니다.
    \item \textbf{Result:} 모델은 이제 "길드 활동이 많아도 최근 접속이 뜸하면 이탈할 수 있다"는 미묘한 패턴을 배우게 됩니다.
\end{enumerate}

\vspace{0.5cm}\hrule\vspace{0.5cm}

\section{자주 묻는 질문 (FAQ)}

\begin{description}
    \item[Q1. 역전파를 손으로 계산할 줄 알아야 하나요?]
    \textbf{A.} 개념만 알면 됩니다. PyTorch나 TensorFlow 같은 프레임워크가 `loss.backward()` 한 줄이면 자동으로 미분을 다 해줍니다. (이걸 \textbf{Autograd}라고 합니다.)
    
    \item[Q2. Dropout은 테스트(실전) 때도 쓰나요?]
    \textbf{A.} \textbf{절대 아닙니다!} 학습(Training) 때는 뉴런을 끄지만, 테스트(Inference) 때는 모든 뉴런을 켜서 전력을 다해야 합니다. 대신 출력값에 비율(0.5)을 곱해서 보정해 줍니다.
\end{description}

% 10. 다음 단원 연결
\vspace{1cm}
\begin{quote}
\textbf{Next Step:} 지금까지 배운 신경망은 정지된 이미지나 테이블 데이터에는 잘 작동하지만, "문장"이나 "주가"처럼 \textbf{순서가 있는(Sequential) 데이터}에는 약합니다. 다음 \textbf{Unit 5}에서는 기억력(Memory)을 가진 신경망, \textbf{RNN(Recurrent Neural Networks)}과 LSTM을 배웁니다.
\end{quote}

% 11. 단원 요약 박스
\begin{summarybox}{Unit 4 핵심 요약}
\begin{itemize}
    \item \textbf{역전파 (Backprop):} Chain Rule을 이용해 오차의 책임(Gradient)을 각 가중치에 배분한다.
    \item \textbf{손실 함수:} 회귀는 MSE, 분류는 Cross-Entropy가 국룰이다.
    \item \textbf{He Initialization:} ReLU를 쓸 때는 초기화를 잘해야 학습이 시작된다.
    \item \textbf{Batch Norm \& Dropout:} 학습을 빠르고 안정적으로 만들고 과적합을 막는 필수 테크닉.
\end{itemize}
\end{summarybox}

\end{document}