\documentclass[a4paper,12pt]{article}
\usepackage{kotex}
\usepackage{amsmath, amssymb, amsthm}
\usepackage{geometry}
\usepackage{graphicx}
\usepackage{adjustbox}  % 표/박스 크기 조절
\usepackage{xcolor}
\usepackage[most]{tcolorbox}
\tcbuselibrary{breakable}
\usepackage{hyperref}
\usepackage{enumitem}
\usepackage{booktabs}
\usepackage{tabularx}
\usepackage{fancyhdr}
\usepackage{bm}

% 페이지 설정
\geometry{left=25mm, right=25mm, top=30mm, bottom=30mm}
\pagestyle{fancy}
\fancyhead[L]{MIT 6.86x Unit 4 (Part B)}
\fancyhead[R]{Dimension Reduction \& Matrix Factorization}

% 색상 정의
\definecolor{mainblue}{RGB}{0, 51, 102}
\definecolor{subblue}{RGB}{230, 240, 255}
\definecolor{warningred}{RGB}{204, 0, 0}
\definecolor{conceptgreen}{RGB}{0, 102, 51}
\definecolor{storypurple}{RGB}{102, 0, 102}

% 박스 스타일 정의
\newtcolorbox{summarybox}[1]{
  colback=subblue, colframe=mainblue, 
  title=\textbf{#1}, fonttitle=\bfseries,
  boxrule=0.5mm, arc=2mm
}

\newtcolorbox{warningbox}[1]{
  colback=white, colframe=warningred, 
  title=\textbf{⚠️ #1}, fonttitle=\bfseries,
  boxrule=0.5mm, arc=0mm,
  coltitle=white
}

\newtcolorbox{conceptbox}[1]{
  colback=white, colframe=conceptgreen, 
  title=\textbf{💡 #1}, fonttitle=\bfseries,
  boxrule=0.5mm, arc=2mm,
  coltitle=white
}

\newtcolorbox{storybox}[1]{
  colback=white, colframe=storypurple, 
  title=\textbf{🎬 #1}, fonttitle=\bfseries,
  boxrule=0.5mm, arc=2mm,
  coltitle=white
}

\title{\textbf{MIT 6.86x: 데이터 압축과 추천의 원리}}
\author{Unit 4 (Part B): Dimension Reduction \& Matrix Factorization}
\date{}

\begin{document}

\maketitle

% 1. 전체 목차 (TOC)
\tableofcontents
\vspace{1cm}
\hrule
\vspace{1cm}

\section*{Course Structure \& Current Focus}
\begin{itemize}
    \item Unit 4 (Part A): Clustering (비슷한 것끼리 묶기)
    \item \textbf{\textcolor{mainblue}{Unit 4 (Part B): Dimension Reduction \& Matrix Factorization}}
    \begin{itemize}
        \item 12.1 PCA: Reconstruction Error View (데이터 압축)
        \item 12.2 Matrix Completion Problem (빈칸 채우기)
        \item 12.3 Matrix Factorization (추천 시스템의 엔진)
        \item 12.4 Collaborative Filtering (협업 필터링)
    \end{itemize}
    \item Unit 5: Generative Models
\end{itemize}

\newpage

% 2. 현재 단원 제목
\section{Unit 4 (Part B). 차원 축소와 행렬 분해}

% 3. 이전 단원과의 연결
\begin{quote}
\textit{지난 파트(Clustering)에서는 데이터를 그룹핑하여 구조를 파악했습니다. 이번에는 데이터 자체를 \textbf{'압축'}하거나, 비어 있는 데이터를 \textbf{'복원'}하는 방법을 배웁니다. 이는 마치 저화질 이미지를 고화질로 복원하거나, 내가 보지 않은 영화의 평점을 예측하는 마법 같은 기술의 기초가 됩니다.}
\end{quote}

% 4. 개요
\subsection*{📌 개요 (Overview)}
이 단원에서는 \textbf{PCA(주성분 분석)}를 '분산 최대화'가 아닌 \textbf{'재구성 오차 최소화'}라는 머신러닝 관점에서 재해석합니다. 또한, 넷플릭스 추천 시스템의 근간이 되는 \textbf{행렬 분해(Matrix Factorization)}를 통해 희소 행렬(Sparse Matrix)의 빈칸을 채우고 사용자 취향을 예측하는 원리를 학습합니다.

% 5. 용어 정리 표
\subsection*{📝 핵심 용어 사전}
\begin{table}[h]
\centering
\begin{tabularx}{\textwidth}{|p{0.28\textwidth}|X|}
\hline
\textbf{용어 (Term)} & \textbf{직관적 의미 (Meaning)} \\
\hline
\textbf{Reconstruction Error} & 원본 $\to$ 압축 $\to$ 복원했을 때, 원본과 얼마나 달라졌는가? \\
\hline
\textbf{Autoencoder} & PCA의 딥러닝 버전. 입력을 그대로 출력하도록 학습하는 신경망. \\
\hline
\textbf{Sparse Matrix} & 대부분이 0이나 비어있는(NaN) 행렬. (유저는 대부분의 영화를 안 봄) \\
\hline
\textbf{Latent Factor} & 데이터 뒤에 숨어있는 진짜 원인. (예: 장르 선호도, 영화의 분위기) \\
\hline
\textbf{Collaborative Filtering} & "너랑 비슷한 사람이 이거 좋아했으니 너도 좋아할 거야." \\
\hline
\end{tabularx}
\end{table}

\vspace{0.5cm}\hrule\vspace{0.5cm}

% 6. 핵심 개념 상세 설명
\subsection{1. PCA의 머신러닝적 접근: 재구성 오차}


\begin{conceptbox}{개념 1: 파일 압축과 해제}
\textbf{한 줄 요약:} 좋은 압축이란, 압축을 풀었을 때 원본과 가장 똑같은 상태여야 합니다. 손실(Error)을 최소화하는 압축 방법을 찾습니다.
\end{conceptbox}

\subsubsection*{1) 통계학 vs 머신러닝 관점 비교}
\begin{itemize}
    \item \textbf{통계학 (18.6501):} "데이터가 가장 넓게 퍼진(분산이 큰) 방향을 찾자."
    \item \textbf{머신러닝 (6.86x):} "데이터를 저차원 평면에 수직으로 내리꽂았을 때(투영), \textbf{이동 거리가 가장 짧은(오차가 적은)} 평면을 찾자."
\end{itemize}
결국 피타고라스 정리에 의해 두 목표는 \textbf{수학적으로 동치}입니다. (분산 최대화 $\iff$ 재구성 오차 최소화)

\subsubsection*{2) 목적 함수 (Objective Function)}
$$ \text{Minimize } J = \frac{1}{N} \sum_{n=1}^{N} || x_n - \hat{x}_n ||^2 $$
\begin{itemize}
    \item $x_n$: 원본 데이터 (고차원)
    \item $\hat{x}_n$: 압축했다가 다시 복원한 데이터 (재구성된 값)
    \item 의미: 원본과 복원본 사이의 거리 제곱 합을 최소화하는 축(Principal Component)을 찾습니다.
\end{itemize}

\vspace{0.5cm}\hrule\vspace{0.5cm}

\subsection{2. 추천 시스템: 행렬 분해 (Matrix Factorization)}


\begin{conceptbox}{개념 2: 빈칸 채우기 퍼즐}
\textbf{한 줄 요약:} 거대한 구멍 숭숭 뚫린 행렬을 두 개의 꽉 찬 작은 행렬의 곱으로 쪼개서, 빈칸의 값을 추론합니다.
\end{conceptbox}

\subsubsection*{1) 문제 상황: 희소 행렬 (Sparse Matrix)}
넷플릭스에 영화가 1만 개 있다고 칩시다. 한 유저가 본 영화는 기껏해야 100개입니다.
평점 행렬 $R$은 99\%가 비어 있습니다 (NaN). 우리의 목표는 이 빈칸에 들어갈 예상 평점을 맞추는 것입니다.

\subsubsection*{2) 핵심 원리: $R \approx U \times V^T$}
평점 행렬 $R$ ($N \times M$)을 두 개의 행렬로 분해합니다.
\begin{itemize}
    \item \textbf{$U$ (User Matrix, $N \times K$):} 유저들의 \textbf{잠재 취향} (예: 액션 선호도, 로맨스 선호도...)
    \item \textbf{$V$ (Item Matrix, $M \times K$):} 영화들의 \textbf{잠재 특성} (예: 액션 지수, 로맨스 지수...)
\end{itemize}

\subsubsection*{3) 예측 (Prediction)}
유저 $u$가 영화 $i$에 줄 예상 평점 $\hat{r}_{ui}$는 두 잠재 벡터의 \textbf{내적(Dot Product)}입니다.
$$ \hat{r}_{ui} = \mathbf{u}_u \cdot \mathbf{v}_i = \sum_{k=1}^{K} u_{uk} v_{ik} $$
\begin{itemize}
    \item \textbf{직관:} (내 액션 선호도 $\times$ 영화의 액션성) + (내 로맨스 선호도 $\times$ 영화의 로맨스성) = 총점
    \item 취향과 특성이 일치할수록(내적 값이 클수록) 높은 평점이 예측됩니다.
\end{itemize}

\subsubsection*{4) 최적화 (Optimization)}
실제 평점이 있는 데이터에 대해서만 오차를 줄이도록 학습합니다.
$$ \min_{U, V} \sum_{(u,i) \in \text{Observed}} (r_{ui} - \mathbf{u}_u \cdot \mathbf{v}_i)^2 + \lambda(\|U\|^2 + \|V\|^2) $$
주로 \textbf{ALS (Alternating Least Squares)} 알고리즘을 사용하여 $U$를 고정하고 $V$를 풀고, $V$를 고정하고 $U$를 푸는 방식을 반복합니다.

\vspace{0.5cm}\hrule\vspace{0.5cm}

\section{실전 시나리오: 넷플릭스 영화 추천}

\begin{storybox}{Scenario: 당신이 아직 안 본 '명작' 찾기}
당신은 넷플릭스 엔지니어입니다. 유저 A가 '아이언맨', '어벤져스'에는 5점을 줬고, '노트북'에는 평점을 안 남겼습니다. A에게 '노트북'을 추천해야 할까요?
\end{storybox}

\begin{enumerate}
    \item \textbf{잠재 요인 학습 ($K=2$라고 가정):}
    \begin{itemize}
        \item 차원 1: [액션/폭파], 차원 2: [감동/로맨스]
        \item 행렬 분해를 해보니, 유저 A의 벡터 $u_A = [5.0, 0.2]$ (액션광)
        \item 영화 '노트북'의 벡터 $v_{Nb} = [0.1, 4.8]$ (완전 로맨스)
    \end{itemize}
    
    \item \textbf{예상 평점 계산 (내적):}
    $$ \hat{r} = (5.0 \times 0.1) + (0.2 \times 4.8) = 0.5 + 0.96 = 1.46 $$
    
    \item \textbf{결정:} 예상 평점이 1.46점으로 매우 낮습니다.
    $\rightarrow$ \textbf{추천하지 않습니다.} (대신 액션 지수가 높은 '트랜스포머'를 추천합니다.)
\end{enumerate}

\vspace{0.5cm}\hrule\vspace{0.5cm}

\section{자주 묻는 질문 (FAQ)}

\begin{description}
    \item[Q1. 잠재 요인(Latent Factor)이 무엇인지 컴퓨터가 알려주나요?]
    \textbf{A.} 아니요, 컴퓨터는 그냥 숫자 벡터만 줍니다. $k=1$번째 차원이 "액션"인지 "공포"인지는 사람이 결과를 보고 해석해야 합니다. 때로는 해석 불가능한 추상적인 특징일 수도 있습니다.
    
    \item[Q2. 콜드 스타트(Cold Start) 문제는 뭔가요?]
    \textbf{A.} 새로 가입한 유저나 막 개봉한 영화는 데이터(평점)가 하나도 없어서 $U$나 $V$ 벡터를 학습할 수 없는 문제입니다. 이 경우엔 협업 필터링 대신, 나이/성별 기반의 베스트셀러 추천(Top-N)이나 콘텐츠 기반 필터링을 섞어서 사용합니다.
\end{description}

% 10. 다음 단원 연결
\vspace{1cm}
\begin{quote}
\textbf{Next Step:} 지금까지는 주어진 데이터를 분석하는 데 집중했습니다. 그런데 AI가 데이터를 분석하는 것을 넘어, \textbf{새로운 데이터를 창조}할 수는 없을까요? 다음 \textbf{Unit 5: 생성 모델 (Generative Models)}에서는 데이터를 직접 만들어내는 VAE와 GAN에 대해 배웁니다.
\end{quote}

% 11. 단원 요약 박스
\begin{summarybox}{Unit 4 (Part B) 핵심 요약}
\begin{itemize}
    \item \textbf{PCA:} 데이터의 재구성 오차를 최소화하는 평면을 찾아 차원을 축소한다.
    \item \textbf{행렬 분해 (MF):} $R \approx UV^T$. 희소 행렬을 분해하여 숨겨진 패턴을 찾는다.
    \item \textbf{협업 필터링:} 유저와 아이템의 잠재 벡터 내적을 통해 선호도를 예측한다.
    \item \textbf{활용:} 넷플릭스, 유튜브, 쇼핑몰 등의 개인화 추천 시스템의 핵심 원리.
\end{itemize}
\end{summarybox}

\end{document}