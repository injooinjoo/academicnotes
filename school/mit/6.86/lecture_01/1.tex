\documentclass[a4paper,12pt]{article}
\usepackage{kotex}
\usepackage{amsmath, amssymb, amsthm}
\usepackage{geometry}
\usepackage{graphicx}
\usepackage{adjustbox}  % 표/박스 크기 조절
\usepackage{xcolor}
\usepackage[most]{tcolorbox}
\tcbuselibrary{breakable}
\usepackage{hyperref}
\usepackage{enumitem}
\usepackage{booktabs}
\usepackage{tabularx}
\usepackage{fancyhdr}

% 페이지 설정
\geometry{left=25mm, right=25mm, top=30mm, bottom=30mm}
\pagestyle{fancy}
\fancyhead[L]{MIT 6.86x Unit 1}
\fancyhead[R]{Introduction to ML}

% 색상 정의
\definecolor{mainblue}{RGB}{0, 51, 102}
\definecolor{subblue}{RGB}{230, 240, 255}
\definecolor{warningred}{RGB}{204, 0, 0}
\definecolor{conceptgreen}{RGB}{0, 102, 51}
\definecolor{storypurple}{RGB}{102, 0, 102}

% 박스 스타일 정의
\newtcolorbox{summarybox}[1]{
  colback=subblue, colframe=mainblue, 
  title=\textbf{#1}, fonttitle=\bfseries,
  boxrule=0.5mm, arc=2mm
}

\newtcolorbox{warningbox}[1]{
  colback=white, colframe=warningred, 
  title=\textbf{⚠️ #1}, fonttitle=\bfseries,
  boxrule=0.5mm, arc=0mm,
  coltitle=white
}

\newtcolorbox{conceptbox}[1]{
  colback=white, colframe=conceptgreen, 
  title=\textbf{💡 #1}, fonttitle=\bfseries,
  boxrule=0.5mm, arc=2mm,
  coltitle=white
}

\newtcolorbox{storybox}[1]{
  colback=white, colframe=storypurple, 
  title=\textbf{🎬 #1}, fonttitle=\bfseries,
  boxrule=0.5mm, arc=2mm,
  coltitle=white
}

\title{\textbf{MIT 6.86x: 예측의 미학}}
\author{Unit 1: Introduction to Machine Learning}
\date{}

\begin{document}

\maketitle

% 1. 전체 목차 (TOC)
\tableofcontents
\vspace{1cm}
\hrule
\vspace{1cm}

\section*{Course Structure \& Current Focus}
\begin{itemize}
    \item \textbf{\textcolor{mainblue}{Unit 1: Introduction to ML (현재 단원: 운동장 세팅)}}
    \begin{itemize}
        \item 1.1 Formalizing the Problem (함수 근사 문제)
        \item 1.2 Hypothesis Space ($\mathcal{H}$)
        \item 1.3 Learning Paradigms (지도 vs 비지도)
        \item 1.4 Generalization ($E_{in}$ vs $E_{out}$)
    \end{itemize}
    \item Unit 2: Linear Classifiers (Perceptron)
    \item Unit 3: Neural Networks
\end{itemize}

\newpage

% 2. 현재 단원 제목
\section{Unit 1. 학습의 기초 (Fundamentals of Learning)}

% 3. 이전 단원과의 연결
\begin{quote}
\textit{우리는 18.6501(통계학)에서 "데이터가 어떤 분포에서 나왔는가(Inference)?"를 고민했습니다. 6.86x(머신러닝)에서는 질문을 바꿉니다. \textbf{"그래서, 내일 주가가 오를까 내릴까(Prediction)?"} 분포의 모양보다는, 정답을 맞히는 성능에 목숨을 거는 새로운 여정이 시작됩니다.}
\end{quote}

% 4. 개요
\subsection*{📌 개요 (Overview)}
이 단원은 코드를 짜기 전, 머신러닝 문제를 수학적으로 정의하는 단계입니다. 머신러닝을 **"미지의 목표 함수 $f^*$를 찾기 위해 가설 공간 $\mathcal{H}$를 탐색하는 과정"\textbf{으로 정의하고, 학습의 궁극적 목표인 }일반화(Generalization)**의 개념을 확립합니다.

% 5. 용어 정리 표
\subsection*{📝 핵심 용어 사전}
\begin{table}[h]
\centering
\begin{tabularx}{\textwidth}{|p{0.28\textwidth}|X|}
\hline
\textbf{용어 (Term)} & \textbf{직관적 의미 (Meaning)} \\
\hline
\textbf{Target Function ($f^*$)} & 신(God)만이 아는 진짜 정답 규칙. 우리는 이걸 모른다. \\
\hline
\textbf{Hypothesis ($h$)} & 우리가 $f^*$라고 추측한 모델 (가설). \\
\hline
\textbf{Hypothesis Space ($\mathcal{H}$)} & $h$를 찾을 수색 범위 (예: 직선들, 신경망들). \\
\hline
\textbf{Inductive Bias} & 데이터를 보기도 전에 정해둔 편견 (예: "답은 직선일 거야"). \\
\hline
\textbf{Generalization} & 안 배운 문제(Test Data)도 잘 맞히는 능력. \\
\hline
\end{tabularx}
\end{table}

\vspace{0.5cm}\hrule\vspace{0.5cm}

% 6. 핵심 개념 상세 설명
\subsection{1. 학습 문제의 정의 (Formalizing the Problem)}


\begin{conceptbox}{개념 1: 함수 근사 (Function Approximation)}
\textbf{한 줄 요약:} 머신러닝은 입력($x$)과 출력($y$) 사이의 숨겨진 관계식($f^*$)을 데이터($S_n$)를 통해 역추적하는 과정입니다.
\end{conceptbox}

\subsubsection*{1) 직관적 비유: 장인의 레시피 복원}
유명 맛집의 비법 소스($f^*$)가 있습니다. 사장님은 레시피를 절대 안 알려줍니다.
\begin{itemize}
    \item \textbf{Data ($S_n$):} 우리가 맛볼 수 있는 건 결과물(음식) 뿐입니다.
    \item \textbf{Goal:} 맛을 보고 재료와 비율을 역추적해서, 사장님의 맛과 거의 똑같은 맛을 내는 나만의 레시피($h$)를 만드는 것입니다.
\end{itemize}

\subsubsection*{2) 수학적 구성 요소}
\begin{itemize}
    \item \textbf{Input ($\mathcal{X}$):} 재료 (벡터 $x \in \mathbb{R}^d$)
    \item \textbf{Output ($\mathcal{Y}$):} 맛 (값 $y$)
    \item **Unknown Target ($f^*$):** $f^*: \mathcal{X} \to \mathcal{Y}$. (이상적인 정답 규칙)
    \item \textbf{Data ($S_n$):} $\{(x^{(i)}, y^{(i)})\}_{i=1}^n$. ($f^*$에 노이즈가 섞인 관측치)
    \item \textbf{Final Hypothesis ($h$):} 우리가 찾은 최적의 함수. $h \approx f^*$이기를 희망함.
\end{itemize}

\vspace{0.5cm}\hrule\vspace{0.5cm}

\subsection{2. 가설 공간 (Hypothesis Space, $\mathcal{H}$)}

\begin{conceptbox}{개념 2: 수색 범위 설정}
\textbf{한 줄 요약:} "세상의 모든 함수" 중에서 정답을 찾을 수는 없습니다. "직선 중에서 찾자" 혹은 "곡선 중에서 찾자"처럼 탐색 범위를 제한해야 합니다.
\end{conceptbox}

\subsubsection*{1) 귀납적 편향 (Inductive Bias)}
가설 공간 $\mathcal{H}$를 정하는 순간, 우리는 데이터에 대한 \textbf{편견(Bias)}을 갖게 됩니다.
\begin{itemize}
    \item 선형 회귀를 쓴다면? $\rightarrow$ "세상은 선형적일 거야"라는 편견.
    \item 편견이 나쁜가요? \textbf{아니요, 필수입니다.} 편견(제약 조건)이 없으면, 데이터 점들을 잇는 방법이 무한히 많아서 학습 자체가 불가능합니다.
\end{itemize}

\subsubsection*{2) Trade-off}
\begin{itemize}
    \item $\mathcal{H}$가 너무 작음 (예: 상수 함수): 정답을 표현 못 함 (Underfitting).
    \item $\mathcal{H}$가 너무 큼 (예: 100차 다항식): 노이즈까지 다 외워버림 (Overfitting).
\end{itemize}

\vspace{0.5cm}\hrule\vspace{0.5cm}

\subsection{3. 지도 학습 vs 비지도 학습}

\begin{conceptbox}{개념 3: 정답지의 유무}
\textbf{한 줄 요약:} 문제집 뒤에 해설지가 있으면 지도 학습, 해설지 없이 문제들의 유형을 스스로 정리해야 하면 비지도 학습입니다.
\end{conceptbox}

\subsubsection*{1) 지도 학습 (Supervised)}
\begin{itemize}
    \item 데이터: $(x, y)$ 쌍.
    \item 목표: $h(x) \approx y$ 인 $h$를 찾음. (회귀, 분류)
\end{itemize}

\subsubsection*{2) 비지도 학습 (Unsupervised)}
\begin{itemize}
    \item 데이터: $x$만 있음.
    \item 목표: 데이터의 구조, 패턴, 군집을 찾음. (클러스터링, 차원 축소)
    \item 예시: 넥슨 유저들의 행동 로그($x$)만 보고 "이들은 하드코어 유저군이다"라고 그룹핑하기.
\end{itemize}

\vspace{0.5cm}\hrule\vspace{0.5cm}

\subsection{4. 일반화 (Generalization)}


\begin{conceptbox}{개념 4: 머신러닝의 존재 이유}
\textbf{한 줄 요약:} 연습문제(Training Data)를 100점 맞는 건 의미 없습니다. 실전 수능(Unseen Data)을 잘 보는 것이 진짜 목표입니다.
\end{conceptbox}

\subsubsection*{1) 두 가지 위험 (Risk)}
\begin{itemize}
    \item \textbf{경험적 위험 ($E_{in}$, Empirical Risk):}
    지금 가진 학습 데이터($S_n$)에서의 오차. (모의고사 성적)
    \item \textbf{실제 위험 ($E_{out}$, True Risk):}
    전체 데이터 분포($\mathcal{D}$)에서의 기대 오차. (수능 성적)
\end{itemize}

\subsubsection*{2) 근본적인 긴장 관계 (Fundamental Tension)}
우리는 알고리즘을 통해 $E_{in}$을 줄입니다. 하지만 $E_{in}$을 0으로 만든다고 $E_{out}$이 0이 될까요?
\begin{itemize}
    \item \textbf{Overfitting:} $E_{in} \approx 0$이지만 $E_{out}$은 매우 큰 상태. (답을 달달 외워서 응용을 못함)
    \item \textbf{Goal:} $E_{in}$도 작게 하면서, 동시에 \textbf{$E_{out} \approx E_{in}$}이 되도록 보장하는 것. 이것이 6.86x 과정의 핵심 질문입니다.
\end{itemize}

\vspace{0.5cm}\hrule\vspace{0.5cm}

\section{실전 시나리오: 넥슨 게임 이탈자 예측}

\begin{storybox}{Scenario: 완벽한 예측 모델의 함정}
당신은 유저 이탈 예측 모델을 만들었습니다.
과거 1년 치 데이터($S_n$)로 학습시켰더니, 정확도 99.9\%($E_{in} \approx 0$)가 나왔습니다.
"완벽해!"라고 외치며 실 서버에 배포했습니다.
\end{storybox}

\begin{enumerate}
    \item \textbf{결과:} 다음 달, 모델은 이탈자를 하나도 못 맞췄습니다. ($E_{out}$ 폭망)
    \item \textbf{원인 분석:} 모델을 뜯어보니 이런 규칙이 있었습니다.
    \textit{"아이디가 'User1234'이고 3월 5일에 접속한 사람은 이탈한다."}
    \item \textbf{해석:} 가설 공간 $\mathcal{H}$를 너무 복잡하게 잡아서, 유저의 행동 패턴($f^*$)을 배운 게 아니라 특정 유저의 ID와 날짜(노이즈)를 외워버린 것입니다.
    \item \textbf{해결:} 모델을 단순화(Regularization)하거나 데이터를 더 모아서 \textbf{일반화 성능($E_{out}$)}을 챙겨야 합니다.
\end{enumerate}

\vspace{0.5cm}\hrule\vspace{0.5cm}

\section{자주 묻는 질문 (FAQ)}

\begin{description}
    \item[Q1. $f^*$와 $h$의 차이가 정확히 뭔가요?]
    \textbf{A.} $f^*$는 '진리(Truth)'이고 $h$는 '추측(Guess)'입니다.
    예를 들어, 물리 법칙($F=ma$)은 $f^*$입니다. 우리가 실험 데이터를 통해 $F = 0.98 m a + 0.02$라는 식을 얻었다면 이것이 $h$입니다. 우리는 영원히 $f^*$를 완벽하게 알 수는 없고, $h$를 $f^*$에 가깝게 만들 뿐입니다.
    
    \item[Q2. Inductive Bias가 왜 필요한가요?]
    \textbf{A.} 편견 없이는 학습도 없습니다(No Free Lunch Theorem). "모든 것이 가능하다"는 말은 "아무것도 알 수 없다"는 말과 같습니다. "답은 연속적일 거야", "답은 간단할 거야" 같은 가정이 있어야만 유한한 데이터로부터 무한한 미래를 예측할 수 있습니다.
\end{description}

% 10. 다음 단원 연결
\vspace{1cm}
\begin{quote}
\textbf{Next Step:} "일반화"가 중요하다는 것은 알았습니다. 그렇다면 가장 단순하면서도 강력한 가설 공간인 \textbf{'선형 모델(Linear Model)'}부터 시작해볼까요? 다음 \textbf{Unit 2}에서는 직선 하나로 데이터를 분류하는 \textbf{퍼셉트론(Perceptron)} 알고리즘을 배웁니다.
\end{quote}

% 11. 단원 요약 박스
\begin{summarybox}{Unit 1 핵심 요약}
\begin{itemize}
    \item \textbf{목표:} 데이터($S_n$)를 이용해 미지의 함수 $f^*$에 근사하는 $h$를 찾는다.
    \item \textbf{가설 공간($\mathcal{H}$):} 탐색 범위를 제한하는 것. 필연적으로 편견(Bias)을 동반한다.
    \item \textbf{일반화:} 학습 데이터 오차($E_{in}$)가 아니라, 보지 못한 데이터의 오차($E_{out}$)를 줄이는 것이 진짜 목표다.
    \item \textbf{Overfitting:} $E_{in}$은 낮지만 $E_{out}$이 높은 상태. (암기왕)
\end{itemize}
\end{summarybox}

\end{document}