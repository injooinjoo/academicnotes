\documentclass[a4paper, 11pt]{article}

% --- 패키지 설정 ---
\usepackage{kotex} % 한글 지원
\usepackage{geometry} % 여백 설정
\geometry{left=25mm, right=25mm, top=25mm, bottom=25mm}
\usepackage{amsmath, amssymb, amsfonts} % 수식 패키지
\usepackage{graphicx}
\usepackage{adjustbox}  % 표/박스 크기 조절 % 이미지 삽입
\usepackage{hyperref} % 하이퍼링크
\usepackage{xcolor} % 색상 지원
\usepackage{listings} % 코드 블록
\usepackage[most]{tcolorbox}
\tcbuselibrary{breakable} % 박스 디자인
\usepackage{enumitem} % 리스트 스타일
\usepackage{booktabs} % 표 디자인
\usepackage{array} % 표 정렬

% --- 색상 정의 ---
\definecolor{conceptblue}{RGB}{60, 100, 160}
\definecolor{analogygreen}{RGB}{80, 160, 100}
\definecolor{alertred}{RGB}{200, 60, 60}
\definecolor{exampleorange}{RGB}{230, 120, 30}
\definecolor{codegray}{rgb}{0.5,0.5,0.5}
\definecolor{backcolour}{rgb}{0.96,0.96,0.96}

% --- 코드 스타일 설정 ---
\lstdefinestyle{mystyle}{
    backgroundcolor=\color{backcolour},   
    commentstyle=\color{analogygreen},
    keywordstyle=\color{conceptblue},
    numberstyle=\tiny\color{codegray},
    stringstyle=\color{exampleorange},
    basicstyle=\ttfamily\footnotesize,
    breakatwhitespace=false,         
    breaklines=true,                 
    captionpos=b,                    
    keepspaces=true,                 
    numbers=left,                    
    numbersep=5pt,                  
    showspaces=false,                
    showstringspaces=false,
    showtabs=false,                  
    tabsize=4,
    frame=single
}
\lstset{style=mystyle}

% --- 박스 스타일 정의 ---
\newtcolorbox{summarybox}[1]{
    colback=conceptblue!5!white,
    colframe=conceptblue!80!black,
    fonttitle=\bfseries,
    title=📌 #1
}

\newtcolorbox{analogybox}[1]{
    colback=analogygreen!5!white,
    colframe=analogygreen!80!black,
    fonttitle=\bfseries,
    title=💡 #1 (직관적 비유)
}

\newtcolorbox{warningbox}[1]{
    colback=alertred!5!white,
    colframe=alertred!80!black,
    fonttitle=\bfseries,
    title=⚠️ #1 (오해 방지 가이드)
}

\newtcolorbox{mathbox}[1]{
    colback=exampleorange!5!white,
    colframe=exampleorange!80!black,
    fonttitle=\bfseries,
    title=🧮 #1 (수학적 원리)
}

% --- 문서 정보 ---
\title{\textbf{[CS230] Optimization Algorithms: \\ Adam Optimizer}}
\author{Lecturer: Gemini (Integrated Editor)}
\date{}

\begin{document}

\maketitle

% --- 1. 전체 목차 (TOC) ---
\section*{📚 Course Table of Contents}
\begin{itemize}
    \item[Chapter 1-4.] Neural Networks Basics \textit{- Completed}
    \item[\textbf{Chapter 5.}] \textbf{Practical Aspects of Deep Learning (Current Unit)}
    \begin{itemize}
        \item 5.1-5.5 Regularization \& Data Setup \textit{- Completed}
        \item 5.6 Mini-batch Gradient Descent \textit{- Completed}
        \item 5.7 Momentum \& RMSprop \textit{- Completed}
        \item \textbf{5.8 Adam Optimizer}
        \begin{itemize}
            \item The Ultimate Fusion: Momentum + RMSprop
            \item Bias Correction Mechanism
            \item Hyperparameter Standards ($\alpha, \beta_1, \beta_2, \epsilon$)
            \item Implementation from Scratch
        \end{itemize}
        \item 5.9 Hyperparameter Tuning Strategy \textit{- Upcoming}
    \end{itemize}
\end{itemize}

\vspace{0.5cm}
\hrule
\vspace{0.5cm}

% --- 3. 이전 단원 연결 ---
\section*{🔗 지난 시간 복습 및 연결}
우리는 지난 두 강의를 통해 '관성'을 이용해 속도를 높이는 \textbf{Momentum}과, '보폭'을 조절해 진동을 줄이는 \textbf{RMSprop}을 배웠습니다.
그렇다면 자연스러운 질문이 생깁니다. \textbf{"이 두 가지 장점을 모두 합칠 수는 없을까?"}
그 해답이 바로 \textbf{Adam (Adaptive Moment Estimation)}입니다. Adam은 현재 딥러닝 학계와 현업에서 \textbf{'Default Optimizer(기본 설정)'}로 통합니다. 어떤 옵티마이저를 쓸지 고민될 때, 일단 Adam을 쓰면 80점 이상은 갑니다.

% --- 4. 개요 ---
\section{Unit Overview}
\begin{summarybox}{핵심 목표}
이 단원은 현대 딥러닝의 표준인 \textbf{Adam Optimizer}의 내부 구조를 해부합니다.
\begin{itemize}
    \item \textbf{통합:} Adam이 Momentum의 평균(1차)과 RMSprop의 분산(2차)을 어떻게 결합하는지 수식으로 이해합니다.
    \item \textbf{보정:} 학습 초기에 0으로 쏠리는 현상을 막기 위한 \textbf{편향 보정(Bias Correction)}을 익힙니다.
    \item \textbf{표준:} $\beta_1, \beta_2, \epsilon$ 등 하이퍼파라미터의 국룰(Standard Value)을 배웁니다.
    \item \textbf{구현:} Python으로 편향 보정이 포함된 전체 알고리즘을 밑바닥부터 구현합니다.
\end{itemize}
\end{summarybox}

% --- 5. 용어 정리 ---
\section{Essential Terminology}
\begin{center}
\begin{tabular}{|c|c|l|}
\hline
\textbf{기호} & \textbf{표준값} & \textbf{역할} \\ \hline
$\alpha$ & 튜닝 필요 & \textbf{학습률 (Learning Rate)}. 가장 중요함. \\ \hline
$\beta_1$ & 0.9 & \textbf{Momentum 계수}. (기울기의 지수 평균) \\ \hline
$\beta_2$ & 0.999 & \textbf{RMSprop 계수}. (기울기 제곱의 지수 평균) \\ \hline
$\epsilon$ & $10^{-8}$ & \textbf{안정성 상수}. 0으로 나누기 방지. \\ \hline
\end{tabular}
\end{center}

% --- 6. 핵심 개념 상세 설명 ---
\section{Core Concepts: 최강의 융합}

\subsection{1. The Fusion Algorithm}
Adam은 매 스텝($t$)마다 다음 4단계를 수행합니다.

\begin{enumerate}
    \item \textbf{Momentum ($v$):} 속도를 계산합니다. (1차 모멘트)
    $$ v_t = \beta_1 v_{t-1} + (1 - \beta_1) dW $$
    
    \item \textbf{RMSprop ($s$):} 가속도 제어(마찰력)를 계산합니다. (2차 모멘트)
    $$ s_t = \beta_2 s_{t-1} + (1 - \beta_2) dW^2 $$
    
    \item \textbf{Bias Correction (핵심):} 초기 0으로 쏠린 값을 보정합니다.
    $$ v^{corr}_t = \frac{v_t}{1 - \beta_1^t}, \quad s^{corr}_t = \frac{s_t}{1 - \beta_2^t} $$
    
    \item \textbf{Update:} 파라미터를 갱신합니다.
    $$ W = W - \alpha \frac{v^{corr}_t}{\sqrt{s^{corr}_t} + \epsilon} $$
\end{enumerate}

\vspace{0.5cm}\hrule\vspace{0.5cm}

\section{Deep Dive: Bias Correction (편향 보정)}

"교수님, 왜 굳이 $(1 - \beta^t)$로 나눠주나요?"
이것은 Adam의 정교함을 보여주는 대목입니다.

\begin{mathbox}{초기값 0의 저주}
우리는 $v_0 = 0$으로 시작합니다. 첫 번째 스텝($t=1$)을 봅시다.
($\beta_1 = 0.9$ 가정)

$$ v_1 = 0.9 \times 0 + 0.1 \times dW = 0.1 dW $$

\textbf{문제점:} 실제 기울기($dW$)의 \textbf{10분의 1(0.1)}밖에 반영되지 않습니다. 학습 초반에 거북이처럼 느려집니다.

\textbf{해결책 (보정):}
$$ 1 - \beta_1^1 = 1 - 0.9 = 0.1 $$
$$ v^{corr}_1 = \frac{v_1}{0.1} = \frac{0.1 dW}{0.1} = dW $$

\textbf{결과:} 보정 덕분에 초기에도 기울기를 100\% 반영할 수 있습니다.
$t$가 커지면 $\beta^t \to 0$이 되어, 분모가 1이 되므로 보정 효과는 자연스럽게 사라집니다.
\end{mathbox}

% --- 7. 구현 코드 ---
\section{Implementation: Adam from Scratch}

Adam 구현 시 가장 중요한 것은 현재 반복 횟수 \textbf{$t$}를 추적하는 것입니다.

\begin{lstlisting}[language=Python, caption=Adam Optimizer Implementation, breaklines=true]
import numpy as np

def update_parameters_with_adam(parameters, grads, v, s, t, learning_rate=0.01,
                                beta1=0.9, beta2=0.999, epsilon=1e-8):
    """
    t: 현재 iteration count (1부터 시작해야 함!)
    v, s: 이전 스텝까지 누적된 Momentum, RMSprop 변수
    """
    L = len(parameters) // 2
    v_corrected = {} 
    s_corrected = {} 
    
    for l in range(1, L + 1):
        # --- 1. Momentum (v) ---
        v["dW" + str(l)] = beta1 * v["dW" + str(l)] + (1 - beta1) * grads["dW" + str(l)]
        v["db" + str(l)] = beta1 * v["db" + str(l)] + (1 - beta1) * grads["db" + str(l)]
        
        # --- 2. Bias Correction (v) ---
        # 1 - beta^t 로 나눔
        v_corrected["dW" + str(l)] = v["dW" + str(l)] / (1 - np.power(beta1, t))
        v_corrected["db" + str(l)] = v["db" + str(l)] / (1 - np.power(beta1, t))
        
        # --- 3. RMSprop (s) ---
        # 기울기 제곱(square) 주의!
        s["dW" + str(l)] = beta2 * s["dW" + str(l)] + (1 - beta2) * np.square(grads["dW" + str(l)])
        s["db" + str(l)] = beta2 * s["db" + str(l)] + (1 - beta2) * np.square(grads["db" + str(l)])
        
        # --- 4. Bias Correction (s) ---
        s_corrected["dW" + str(l)] = s["dW" + str(l)] / (1 - np.power(beta2, t))
        s_corrected["db" + str(l)] = s["db" + str(l)] / (1 - np.power(beta2, t))
        
        # --- 5. Update Parameters ---
        # 분모: sqrt(s_corr) + epsilon
        parameters["W" + str(l)] -= learning_rate * (v_corrected["dW" + str(l)] / (np.sqrt(s_corrected["dW" + str(l)]) + epsilon))
        parameters["b" + str(l)] -= learning_rate * (v_corrected["db" + str(l)] / (np.sqrt(s_corrected["db" + str(l)]) + epsilon))
        
    return parameters, v, s
\end{lstlisting}

% --- 8. FAQ ---
\section{FAQ \& Pitfalls}

\begin{warningbox}{Iteration Count $t$ 주의}
함수를 호출할 때 $t$는 반드시 1부터 시작해야 합니다. 만약 $t=0$이면 $1 - \beta^0 = 1 - 1 = 0$이 되어 \textbf{ZeroDivisionError}가 발생합니다.
\end{warningbox}

\textbf{Q. Adam이 항상 최고인가요?} \\
\textbf{A.} 대부분의 경우(CV, NLP, GAN) 그렇습니다. 하지만 아주 정교한 수렴이 필요할 때(SOTA 논문 등)는 일반 SGD+Momentum이 더 좋은 성능을 낼 때도 있습니다. 그래도 시작은 무조건 Adam을 추천합니다.

% --- 9. 다음 단원 연결 ---
\section*{🔗 다음 단계 (Next Step)}
이로써 우리는 최적화 알고리즘의 정점인 Adam을 정복했습니다. 이제 여러분은 어떤 모델이든 빠르고 안정적으로 학습시킬 수 있는 엔진을 갖췄습니다.

하지만 엔진 성능이 좋아도, 기어 변속(하이퍼파라미터 설정)을 잘못하면 차가 나가지 않습니다. $\alpha$, $\beta$, 배치 크기, 은닉층 개수... 도대체 무엇부터 조절해야 할까요?
다음 시간에는 이 수많은 다이얼을 어떤 순서로 돌려야 하는지, \textbf{[Hyperparameter Tuning]}의 체계적인 전략(Random Search vs Grid Search)을 알려드리겠습니다.

\vspace{0.5cm}

\begin{summarybox}{단원 요약 (Cheat Sheet)}
\begin{enumerate}
    \item \textbf{Adam:} Momentum(속도) + RMSprop(가속도 제어) + Bias Correction(초기 보정).
    \item \textbf{Standard Params:} $\alpha$(튜닝), $\beta_1(0.9)$, $\beta_2(0.999)$, $\epsilon(10^{-8})$.
    \item \textbf{Bias Correction:} 학습 초반에 파라미터 업데이트가 너무 작아지는 것을 막아준다.
    \item \textbf{Memory:} $v$와 $s$를 따로 저장해야 하므로 일반 SGD보다 메모리를 더 쓴다.
\end{enumerate}
\end{summarybox}

\end{document}