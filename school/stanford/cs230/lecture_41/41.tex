\documentclass[a4paper, 11pt]{article}

% --- 패키지 설정 ---
\usepackage{kotex} % 한글 지원
\usepackage{geometry} % 여백 설정
\geometry{left=25mm, right=25mm, top=25mm, bottom=25mm}
\usepackage{amsmath, amssymb, amsfonts} % 수식 패키지
\usepackage{graphicx}
\usepackage{adjustbox}  % 표/박스 크기 조절 % 이미지 삽입
\usepackage{hyperref} % 하이퍼링크
\usepackage{xcolor} % 색상 지원
\usepackage{listings} % 코드 블록
\usepackage[most]{tcolorbox}
\tcbuselibrary{breakable} % 박스 디자인
\usepackage{enumitem} % 리스트 스타일
\usepackage{booktabs} % 표 디자인
\usepackage{array} % 표 정렬

% --- 색상 정의 ---
\definecolor{conceptblue}{RGB}{60, 100, 160}
\definecolor{analogygreen}{RGB}{80, 160, 100}
\definecolor{alertred}{RGB}{200, 60, 60}
\definecolor{exampleorange}{RGB}{230, 120, 30}
\definecolor{codegray}{rgb}{0.5,0.5,0.5}
\definecolor{backcolour}{rgb}{0.96,0.96,0.96}

% --- 코드 스타일 설정 ---
\lstdefinestyle{mystyle}{
    backgroundcolor=\color{backcolour},   
    commentstyle=\color{analogygreen},
    keywordstyle=\color{conceptblue},
    numberstyle=\tiny\color{codegray},
    stringstyle=\color{exampleorange},
    basicstyle=\ttfamily\footnotesize,
    breakatwhitespace=false,         
    breaklines=true,                 
    captionpos=b,                    
    keepspaces=true,                 
    numbers=left,                    
    numbersep=5pt,                  
    showspaces=false,                
    showstringspaces=false,
    showtabs=false,                  
    tabsize=4,
    frame=single
}
\lstset{style=mystyle}

% --- 박스 스타일 정의 ---
\newtcolorbox{summarybox}[1]{
    colback=conceptblue!5!white,
    colframe=conceptblue!80!black,
    fonttitle=\bfseries,
    title=📌 #1
}

\newtcolorbox{analogybox}[1]{
    colback=analogygreen!5!white,
    colframe=analogygreen!80!black,
    fonttitle=\bfseries,
    title=💡 #1 (직관적 비유)
}

\newtcolorbox{warningbox}[1]{
    colback=alertred!5!white,
    colframe=alertred!80!black,
    fonttitle=\bfseries,
    title=⚠️ #1 (오해 방지 가이드)
}

\newtcolorbox{formulabox}[1]{
    colback=exampleorange!5!white,
    colframe=exampleorange!80!black,
    fonttitle=\bfseries,
    title=🧮 #1 (핵심 수식)
}

% --- 문서 정보 ---
\title{\textbf{[CS230] Sequence Models: \\ Vanishing Gradients \& GRU}}
\author{Lecturer: Gemini (Integrated Editor)}
\date{}

\begin{document}

\maketitle

% --- 1. 전체 목차 (TOC) ---
\section*{📚 Course Table of Contents}
\begin{itemize}
    \item[Chapter 1-9.] Deep Learning Fundamentals \& CNNs \textit{- Completed}
    \item[\textbf{Chapter 10.}] \textbf{Sequence Models (Current Part)}
    \begin{itemize}
        \item 10.1 RNN Basics \textit{- Completed}
        \item 10.2 Various RNN Architectures \textit{- Completed}
        \item \textbf{10.3 Vanishing Gradients \& Gated Units}
        \begin{itemize}
            \item The Problem: Vanishing Gradients
            \item The Solution: Gating Mechanism
            \item GRU (Gated Recurrent Unit)
            \item Implementation: Update \& Reset Gates
        \end{itemize}
        \item 10.4 LSTM (Long Short-Term Memory) \textit{- Upcoming}
    \end{itemize}
\end{itemize}

\vspace{0.5cm}
\hrule
\vspace{0.5cm}

% --- 3. 이전 단원 연결 ---
\section*{🔗 지난 시간 복습 및 연결}
지난 시간에 배운 기본 RNN은 이론적으로는 완벽해 보입니다. 과거 정보를 현재로 전달하니까요.
하지만 실제로는 문장이 10단어만 넘어가도 앞부분 내용을 까맣게 잊어버립니다.
\textbf{"The cat, which ate ..., was full."}
RNN은 중간 수식어가 길어지면 주어 'cat(단수)'을 잊어버리고 동사 'was'를 예측하지 못합니다.
이것이 바로 \textbf{기울기 소실(Vanishing Gradient)} 문제입니다. 오늘은 이 난제를 해결한 \textbf{GRU}를 배웁니다.

% --- 4. 개요 ---
\section{Unit Overview}
\begin{summarybox}{핵심 목표}
이 단원은 RNN의 장기 의존성 문제를 해결하는 게이트(Gate) 메커니즘을 다룹니다.
\begin{itemize}
    \item \textbf{원인:} 역전파 시 기울기가 지수적으로 작아져서 초기 기억이 사라지는 수학적 원리를 이해합니다.
    \item \textbf{해결:} 정보를 "얼마나 유지하고 버릴지" 결정하는 \textbf{게이트(Gate)} 개념을 도입합니다.
    \item \textbf{모델:} LSTM의 간소화 버전인 \textbf{GRU}의 구조와 수식을 마스터합니다.
\end{itemize}
\end{summarybox}

% --- 5. 용어 정리 ---
\section{Essential Terminology}
\begin{center}
\begin{tabular}{|c|l|l|}
\hline
\textbf{용어} & \textbf{의미} & \textbf{역할} \\ \hline
\textbf{Vanishing Gradient} & 기울기 소실 & 깊은 신경망 학습을 방해하는 주범. \\ \hline
\textbf{Gate ($\Gamma$)} & 문지기 (0~1) & 정보 흐름을 제어하는 시그모이드 함수. \\ \hline
\textbf{Update Gate ($\Gamma_u$)} & 업데이트 게이트 & 과거 기억을 얼마나 유지할지 결정. \\ \hline
\textbf{Reset Gate ($\Gamma_r$)} & 리셋 게이트 & 과거 기억을 얼마나 무시할지 결정. \\ \hline
\end{tabular}
\end{center}

% --- 6. 핵심 개념 상세 설명 ---
\section{Core Concepts: 기억의 보존}

\subsection{1. The Problem: Vanishing Gradients}
RNN에서 $t=100$ 시점의 오차를 $t=1$ 시점까지 전파하려면 가중치 행렬 $W$를 100번 곱해야 합니다.

\begin{analogybox}{복사본의 복사본}
문서를 복사하고, 그 복사본을 또 복사하는 과정을 100번 반복한다고 상상해 보세요.
조금이라도 흐릿해지면($W < 1$), 100번째 복사본은 백지가 되어버립니다.
반대로 진해지면($W > 1$), 검은색 잉크 덩어리가 됩니다(Exploding).
\end{analogybox}

\subsection{2. The Solution: Gated Recurrent Unit (GRU)}
핵심 아이디어는 \textbf{"기억을 위한 전용 금고(Memory Cell)를 만들고 문지기를 세우자"}는 것입니다.

\begin{formulabox}{GRU의 핵심 수식}
$$ c^{\langle t \rangle} = \Gamma_u \cdot \tilde{c}^{\langle t \rangle} + (1 - \Gamma_u) \cdot c^{\langle t-1 \rangle} $$
\begin{itemize}
    \item $\Gamma_u$ (Update Gate): 0이면 문을 닫습니다.
    \item $\Gamma_u \approx 0$ 일 때: $c^{\langle t \rangle} \approx c^{\langle t-1 \rangle}$.
    \item \textbf{의미:} 과거의 기억($c^{\langle t-1 \rangle}$)이 아무런 변형 없이(행렬 곱셈 없이) 그대로 복사되어 다음으로 넘어갑니다. \textbf{기울기 소실 없이 고속도로처럼 전달}됩니다.
\end{itemize}
\end{formulabox}

\vspace{0.5cm}\hrule\vspace{0.5cm}

\section{Deep Dive: GRU Gates Detail}

GRU는 두 개의 게이트를 사용합니다.

\begin{enumerate}
    \item \textbf{Update Gate ($\Gamma_u$):} "이 기억을 계속 가져갈까?" (단수/복수 정보 유지)
    $$ \Gamma_u = \sigma(W_u[c^{\langle t-1 \rangle}, x^{\langle t \rangle}] + b_u) $$
    
    \item \textbf{Reset Gate ($\Gamma_r$):} "이제 과거는 잊을까?" (문침표가 나오면 리셋)
    $$ \Gamma_r = \sigma(W_r[c^{\langle t-1 \rangle}, x^{\langle t \rangle}] + b_r) $$
    
    \item \textbf{Candidate Memory ($\tilde{c}$):} 새로운 기억 후보
    $$ \tilde{c}^{\langle t \rangle} = \tanh(W_c[\Gamma_r \cdot c^{\langle t-1 \rangle}, x^{\langle t \rangle}] + b_c) $$
\end{enumerate}

% --- 7. 구현 코드 ---
\section{Implementation: GRU Cell}

\begin{lstlisting}[language=Python, caption=GRU Cell Forward Implementation, breaklines=true]
import numpy as np

def sigmoid(x):
    return 1 / (1 + np.exp(-x))

def gru_cell_forward(xt, a_prev, parameters):
    """
    xt: 현재 입력
    a_prev: 과거 은닉 상태 (GRU에선 Memory Cell c와 동일)
    """
    # 파라미터 로드
    W_u = parameters["Wu"] # Update Gate Weights
    W_r = parameters["Wr"] # Reset Gate Weights
    W_c = parameters["Wc"] # Candidate Memory Weights
    
    # 입력 결합 (Concatenate)
    concat = np.concatenate((a_prev, xt), axis=0)
    
    # 1. Update Gate (Gamma_u)
    Gamma_u = sigmoid(np.dot(W_u, concat) + parameters["bu"])
    
    # 2. Reset Gate (Gamma_r)
    Gamma_r = sigmoid(np.dot(W_r, concat) + parameters["br"])
    
    # 3. Candidate Memory (c_tilde)
    # 과거 기억에 Reset Gate 적용
    concat_reset = np.concatenate((Gamma_r * a_prev, xt), axis=0)
    c_candidate = np.tanh(np.dot(W_c, concat_reset) + parameters["bc"])
    
    # 4. Final Memory Update (The Magic Formula)
    # u가 0이면 과거 기억 보존, 1이면 새 기억으로 덮어씀
    c_next = Gamma_u * c_candidate + (1 - Gamma_u) * a_prev
    
    return c_next
\end{lstlisting}

% --- 8. FAQ ---
\section{FAQ \& Pitfalls}

\begin{warningbox}{Gradient Clipping (기울기 자르기)}
기울기 소실의 반대는 \textbf{기울기 폭발(Exploding Gradient)}입니다. 숫자가 너무 커져서 `NaN`이 뜹니다.
해결책은 간단합니다. 기울기 벡터의 크기(Norm)가 특정 값(예: 5)을 넘으면 강제로 줄여버리는 \textbf{Gradient Clipping}을 사용합니다.
\end{warningbox}

\textbf{Q. GRU와 LSTM 중 뭐가 더 좋나요?} \\
\textbf{A.} \textbf{정답은 없습니다.} GRU는 구조가 단순해서(게이트 2개) 연산이 빠르고 데이터가 적을 때 유리합니다. LSTM은 구조가 복잡하지만(게이트 3개) 표현력이 더 좋습니다. 보통 LSTM을 기본으로 쓰고, 속도가 중요하면 GRU를 씁니다.

% --- 9. 다음 단원 연결 ---
\section*{🔗 다음 단계 (Next Step)}
GRU는 훌륭하고 단순합니다. 하지만 때로는 더 섬세한 제어가 필요합니다. GRU보다 조금 더 복잡하지만, 더 강력하고 널리 쓰이는 형님이 있습니다.

다음 시간에는 GRU의 확장판이자 RNN의 사실상 표준(De facto Standard), \textbf{[LSTM (Long Short-Term Memory)]}에 대해 다루겠습니다. 3개의 게이트(Forget, Input, Output)가 어떻게 기억을 수술하듯 정교하게 다루는지 알아보겠습니다.

\vspace{0.5cm}

\begin{summarybox}{단원 요약 (Cheat Sheet)}
\begin{enumerate}
    \item \textbf{Vanishing Gradient:} 시퀀스가 길어지면 초기 정보가 사라지는 문제.
    \item \textbf{Gate:} 시그모이드를 사용해 정보의 통과 여부(0~1)를 결정하는 밸브.
    \item \textbf{GRU:} Update/Reset 게이트를 사용해 기억을 보존하거나 갱신한다.
    \item \textbf{Key:} $\Gamma_u=0$일 때 과거 기억이 손실 없이 그대로 전달된다.
\end{enumerate}
\end{summarybox}

\end{document}