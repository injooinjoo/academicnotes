\documentclass[a4paper, 11pt]{article}

% --- 패키지 설정 ---
\usepackage{kotex} % 한글 지원
\usepackage{geometry} % 여백 설정
\geometry{left=25mm, right=25mm, top=25mm, bottom=25mm}
\usepackage{amsmath, amssymb, amsfonts} % 수식 패키지
\usepackage{graphicx}
\usepackage{adjustbox}  % 표/박스 크기 조절 % 이미지 삽입
\usepackage{hyperref} % 하이퍼링크
\usepackage{xcolor} % 색상 지원
\usepackage{listings} % 코드 블록
\usepackage[most]{tcolorbox}
\tcbuselibrary{breakable} % 박스 디자인
\usepackage{enumitem} % 리스트 스타일
\usepackage{booktabs} % 표 디자인
\usepackage{array} % 표 정렬

% --- 색상 정의 ---
\definecolor{conceptblue}{RGB}{60, 100, 160}
\definecolor{analogygreen}{RGB}{80, 160, 100}
\definecolor{alertred}{RGB}{200, 60, 60}
\definecolor{exampleorange}{RGB}{230, 120, 30}
\definecolor{codegray}{rgb}{0.5,0.5,0.5}
\definecolor{backcolour}{rgb}{0.96,0.96,0.96}

% --- 코드 스타일 설정 ---
\lstdefinestyle{mystyle}{
    backgroundcolor=\color{backcolour},   
    commentstyle=\color{analogygreen},
    keywordstyle=\color{conceptblue},
    numberstyle=\tiny\color{codegray},
    stringstyle=\color{exampleorange},
    basicstyle=\ttfamily\footnotesize,
    breakatwhitespace=false,         
    breaklines=true,                 
    captionpos=b,                    
    keepspaces=true,                 
    numbers=left,                    
    numbersep=5pt,                  
    showspaces=false,                
    showstringspaces=false,
    showtabs=false,                  
    tabsize=4,
    frame=single
}
\lstset{style=mystyle}

% --- 박스 스타일 정의 ---
\newtcolorbox{summarybox}[1]{
    colback=conceptblue!5!white,
    colframe=conceptblue!80!black,
    fonttitle=\bfseries,
    title=📌 #1
}

\newtcolorbox{analogybox}[1]{
    colback=analogygreen!5!white,
    colframe=analogygreen!80!black,
    fonttitle=\bfseries,
    title=💡 #1 (직관적 비유)
}

\newtcolorbox{warningbox}[1]{
    colback=alertred!5!white,
    colframe=alertred!80!black,
    fonttitle=\bfseries,
    title=⚠️ #1 (오해 방지 가이드)
}

\newtcolorbox{strategybox}[1]{
    colback=exampleorange!5!white,
    colframe=exampleorange!80!black,
    fonttitle=\bfseries,
    title=🧭 #1 (전략 가이드)
}

% --- 문서 정보 ---
\title{\textbf{[CS230] Structuring Machine Learning Projects: \\ Satisficing and Optimizing Metrics}}
\author{Lecturer: Gemini (Integrated Editor)}
\date{}

\begin{document}

\maketitle

% --- 1. 전체 목차 (TOC) ---
\section*{📚 Course Table of Contents}
\begin{itemize}
    \item[Chapter 1-5.] Deep Learning Fundamentals \textit{- Completed}
    \item[\textbf{Chapter 6.}] \textbf{Structuring ML Projects (Current Unit)}
    \begin{itemize}
        \item 6.1 Orthogonalization Strategy \textit{- Completed}
        \item 6.2 Single Number Evaluation Metric \textit{- Completed}
        \item \textbf{6.3 Satisficing and Optimizing Metrics}
        \begin{itemize}
            \item The Dilemma: Accuracy vs Latency
            \item Definition: Optimization vs Constraint
            \item The $N$-Metric Rule
            \item Implementation: Filtering Logic
        \end{itemize}
        \item 6.4 Human-level Performance \textit{- Upcoming}
    \end{itemize}
\end{itemize}

\vspace{0.5cm}
\hrule
\vspace{0.5cm}

% --- 3. 이전 단원 연결 ---
\section*{🔗 지난 시간 복습 및 연결}
지난 시간에 우리는 정밀도와 재현율을 F1 Score로 합치는 법을 배웠습니다.
그런데 현실 문제는 더 복잡합니다.
\textbf{모델 A: 정확도 99\%, 응답시간 1.5초}
\textbf{모델 B: 정확도 98\%, 응답시간 0.05초}
A는 성능은 좋지만 1.5초나 걸려서 아무도 안 쓸 겁니다. 그렇다고 정확도와 시간을 평균 낼 수도 없습니다(단위가 다름).
이런 딜레마를 해결하기 위해 앤드류 응 교수는 \textbf{"만족(Satisficing) 지표와 최적화(Optimizing) 지표의 분리"}를 제안합니다.

% --- 4. 개요 ---
\section{Unit Overview}
\begin{summarybox}{핵심 목표}
이 단원은 여러 개의 목표가 충돌할 때, 우선순위를 정하는 \textbf{전략적 프레임워크}를 다룹니다.
\begin{itemize}
    \item \textbf{구분:} 평가 지표를 \textbf{'최대한 좋게 할 것(Optimizing)'}과 \textbf{'기준만 넘기면 되는 것(Satisficing)'}으로 나눕니다.
    \item \textbf{규칙:} $N$개의 지표가 있다면, 1개만 최적화하고 나머지 $N-1$개는 제약 조건으로 둡니다.
    \item \textbf{구현:} 여러 모델 후보 중 최적의 모델을 자동으로 선별하는 파이썬 코드를 작성합니다.
\end{itemize}
\end{summarybox}

% --- 5. 용어 정리 ---
\section{Essential Terminology}
\begin{center}
\begin{tabular}{|c|l|l|}
\hline
\textbf{구분} & \textbf{정의} & \textbf{예시} \\ \hline
\textbf{Optimizing Metric} & 다다익선. 무한히 좋아질수록 좋은 단 하나의 지표. & \textbf{정확도(Accuracy)}, F1 Score \\ \hline
\textbf{Satisficing Metric} & 임계값(Threshold)만 넘으면 통과(Pass). & \textbf{실행 시간(Latency)} $\le$ 100ms \\ \hline
\end{tabular}
\end{center}

% --- 6. 핵심 개념 상세 설명 ---
\section{Core Concepts: 올림픽 달리기}

\subsection{1. The Rule of $N$ Metrics}
고려해야 할 지표가 3개(정확도, 속도, 메모리)라면 어떻게 해야 할까요?
\textbf{"3마리 토끼를 다 잡으려다가는 다 놓칩니다."}

\begin{analogybox}{올림픽 달리기와 도핑 테스트}
\begin{itemize}
    \item \textbf{Optimizing (달리기 기록):} 0.01초라도 빠르면 무조건 좋습니다. 금메달의 기준입니다.
    \item \textbf{Satisficing (도핑 테스트):} 약물이 "아주 조금 검출됨"이나 "전혀 검출 안 됨"이나 똑같이 \textbf{통과(Pass)}입니다. 기준치만 안 넘으면 됩니다. 더 깨끗하다고 가산점을 주진 않습니다.
\end{itemize}
\end{analogybox}

\vspace{0.5cm}\hrule\vspace{0.5cm}

\subsection{2. Mathematical Formulation (수학적 정의)}
이 전략은 머신러닝 문제를 \textbf{'제약 조건이 있는 최적화 문제'}로 바꿉니다.

$$ \text{Maximize } \textbf{Accuracy} $$
$$ \text{subject to } \textbf{Latency} \le 100ms $$
$$ \text{and } \textbf{Memory} \le 500MB $$

\begin{itemize}
    \item \textbf{Accuracy:} 목적 함수 (Objective Function) $\rightarrow$ Optimizing
    \item \textbf{Latency, Memory:} 제약 조건 (Constraints) $\rightarrow$ Satisficing
\end{itemize}

\begin{warningbox}{가중치 합의 함정}
"그냥 $\text{Score} = \text{Accuracy} - 0.5 \times \text{Latency}$ 처럼 합치면 안 되나요?"
\textbf{비추천합니다.}
1. 단위가 다릅니다(Accuracy는 \%, Latency는 ms).
2. 사용자 경험은 비선형적입니다. 100ms가 넘으면 '렉 걸림'을 느껴 바로 앱을 끕니다. 이를 선형 식으로 표현하기 어렵습니다. Satisficing(Cut-off)이 훨씬 자연스럽습니다.
\end{warningbox}

% --- 7. 구현 코드 ---
\section{Implementation: Model Selector}

여러 모델의 성능표를 입력받아 자동으로 최적 모델을 뽑는 로직을 구현합니다.

\begin{lstlisting}[language=Python, caption=Auto Model Selector Logic, breaklines=true]
class ModelSelector:
    def __init__(self, optimize_key, constraints):
        """
        optimize_key: 최적화할 지표 (예: 'accuracy')
        constraints: {지표명: (임계값, 연산자)} 
                     예: {'latency': (100, 'lt')} -> less than 100
        """
        self.opt_key = optimize_key
        self.constraints = constraints

    def select(self, models):
        # 1. Satisficing 단계 (Filtering)
        valid_models = []
        for m in models:
            is_valid = True
            for key, (thresh, op) in self.constraints.items():
                val = m[key]
                if op == 'lt' and val > thresh: is_valid = False
                if op == 'gt' and val < thresh: is_valid = False
            
            if is_valid:
                valid_models.append(m)
        
        if not valid_models:
            print("No models satisfied constraints!")
            return None

        # 2. Optimizing 단계 (Sorting)
        # 점수 높은 순으로 정렬
        best_model = sorted(valid_models, key=lambda x: x[self.opt_key], reverse=True)[0]
        return best_model

# --- 실행 ---
if __name__ == "__main__":
    candidates = [
        {'id': 'A', 'acc': 0.99, 'lat': 1500}, # 성능 굿, 너무 느림
        {'id': 'B', 'acc': 0.98, 'lat': 90},   # 성능 적당, 빠름 (Pass)
        {'id': 'C', 'acc': 0.90, 'lat': 50},   # 너무 빠름, 성능 별로 (Pass)
        {'id': 'D', 'acc': 0.985, 'lat': 110}, # 아깝게 느림 (Fail)
    ]
    
    # 전략: Latency < 100ms 인 것 중에서 Accuracy 최대화
    constraints = {'lat': (100, 'lt')}
    selector = ModelSelector('acc', constraints)
    
    best = selector.select(candidates)
    print(f"Best Model: {best['id']} (Acc: {best['acc']}, Lat: {best['lat']})")
    # A, D 탈락. B(0.98) vs C(0.90) -> B 승리.
\end{lstlisting}

% --- 8. FAQ ---
\section{FAQ \& Pitfalls}

\textbf{Q. 만족 지표(Satisficing)가 너무 빡빡하면 어떡하나요?} \\
\textbf{A.} 만약 `Latency < 10ms`로 걸었는데 통과하는 모델이 하나도 없다면, 하드웨어를 바꾸거나 비즈니스 요구사항(임계값)을 완화해야 합니다.

\textbf{Q. 최적화 지표를 2개로 하면 안 되나요?} \\
\textbf{A.} 안 됩니다. "정확도도 높고 속도도 빠른 것"을 찾으려 하면, A(정확도짱)와 B(속도짱) 사이에서 결정을 못 내립니다. 결국 둘을 합친 단일 지표를 만들거나, 하나를 제약 조건으로 돌려야 합니다.

% --- 9. 다음 단원 연결 ---
\section*{🔗 다음 단계 (Next Step)}
이제 평가 기준(Metric)까지 완벽하게 세팅했습니다. 모델을 열심히 개선하고 있는데, 문득 의문이 듭니다.
\textbf{"도대체 이 모델은 어디까지 좋아질 수 있는 걸까? 100\%가 가능한가?"}

이 질문에 답하기 위해서는 비교 대상, 즉 \textbf{기준점(Baseline)}이 필요합니다. 
다음 시간에는 \textbf{[Human-level Performance]}를 통해, 나의 모델이 현재 어느 수준인지, 그리고 얼마나 더 발전할 여지가 있는지(Bayes Error) 가늠하는 법을 배웁니다.

\vspace{0.5cm}

\begin{summarybox}{단원 요약 (Cheat Sheet)}
\begin{enumerate}
    \item \textbf{Optimizing:} 최대한 좋게 만들어야 하는 단 하나의 지표 (예: Accuracy).
    \item \textbf{Satisficing:} 기준선(Threshold)만 넘으면 되는 지표들 (예: Latency, Cost).
    \item \textbf{Process:} Satisficing 지표로 필터링(Cut)하고, Optimizing 지표로 줄 세운다(Rank).
    \item \textbf{Rule:} $N$개의 지표가 있다면 1개는 Optimizing, $N-1$개는 Satisficing이다.
\end{enumerate}
\end{summarybox}

\end{document}