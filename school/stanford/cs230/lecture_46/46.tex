\documentclass[a4paper, 11pt]{article}

% --- 패키지 설정 ---
\usepackage{kotex} % 한글 지원
\usepackage{geometry} % 여백 설정
\geometry{left=25mm, right=25mm, top=25mm, bottom=25mm}
\usepackage{amsmath, amssymb, amsfonts} % 수식 패키지
\usepackage{graphicx}
\usepackage{adjustbox}  % 표/박스 크기 조절 % 이미지 삽입
\usepackage{hyperref} % 하이퍼링크
\usepackage{xcolor} % 색상 지원
\usepackage{listings} % 코드 블록
\usepackage[most]{tcolorbox}
\tcbuselibrary{breakable} % 박스 디자인
\usepackage{enumitem} % 리스트 스타일
\usepackage{booktabs} % 표 디자인
\usepackage{array} % 표 정렬

% --- 색상 정의 ---
\definecolor{conceptblue}{RGB}{60, 100, 160}
\definecolor{analogygreen}{RGB}{80, 160, 100}
\definecolor{alertred}{RGB}{200, 60, 60}
\definecolor{exampleorange}{RGB}{230, 120, 30}
\definecolor{codegray}{rgb}{0.5,0.5,0.5}
\definecolor{backcolour}{rgb}{0.96,0.96,0.96}

% --- 코드 스타일 설정 ---
\lstdefinestyle{mystyle}{
    backgroundcolor=\color{backcolour},   
    commentstyle=\color{analogygreen},
    keywordstyle=\color{conceptblue},
    numberstyle=\tiny\color{codegray},
    stringstyle=\color{exampleorange},
    basicstyle=\ttfamily\footnotesize,
    breakatwhitespace=false,         
    breaklines=true,                 
    captionpos=b,                    
    keepspaces=true,                 
    numbers=left,                    
    numbersep=5pt,                  
    showspaces=false,                
    showstringspaces=false,
    showtabs=false,                  
    tabsize=4,
    frame=single
}
\lstset{style=mystyle}

% --- 박스 스타일 정의 ---
\newtcolorbox{summarybox}[1]{
    colback=conceptblue!5!white,
    colframe=conceptblue!80!black,
    fonttitle=\bfseries,
    title=📌 #1
}

\newtcolorbox{analogybox}[1]{
    colback=analogygreen!5!white,
    colframe=analogygreen!80!black,
    fonttitle=\bfseries,
    title=💡 #1 (직관적 비유)
}

\newtcolorbox{warningbox}[1]{
    colback=alertred!5!white,
    colframe=alertred!80!black,
    fonttitle=\bfseries,
    title=⚠️ #1 (오해 방지 가이드)
}

\newtcolorbox{architecturebox}[1]{
    colback=exampleorange!5!white,
    colframe=exampleorange!80!black,
    fonttitle=\bfseries,
    title=🏗️ #1 (구조 분석)
}

% --- 문서 정보 ---
\title{\textbf{[CS230] Sequence Models: \\ Sequence-to-Sequence (Seq2Seq)}}
\author{Lecturer: Gemini (Integrated Editor)}
\date{}

\begin{document}

\maketitle

% --- 1. 전체 목차 (TOC) ---
\section*{📚 Course Table of Contents}
\begin{itemize}
    \item[Chapter 1-9.] Deep Learning Fundamentals \& CNNs \textit{- Completed}
    \item[\textbf{Chapter 10.}] \textbf{Sequence Models (Current Part)}
    \begin{itemize}
        \item 10.1-10.7 RNN, Embedding, Debiasing \textit{- Completed}
        \item \textbf{10.8 Sequence-to-Sequence (Seq2Seq)}
        \begin{itemize}
            \item The Problem: Different Lengths ($T_x \neq T_y$)
            \item Encoder-Decoder Architecture
            \item Context Vector (The Information Bottleneck)
            \item Teacher Forcing (Training Trick)
        \end{itemize}
        \item 10.9 Beam Search \textit{- Upcoming}
    \end{itemize}
    \item[Chapter 11.] Attention Mechanism \textit{- Next Part}
\end{itemize}

\vspace{0.5cm}
\hrule
\vspace{0.5cm}

% --- 3. 이전 단원 연결 ---
\section*{🔗 지난 시간 복습 및 연결}
지난 시간까지 우리는 단어를 벡터로 바꾸는 법(Embedding)을 배웠습니다. 이제 단어들을 조립해 문장을 만들고, 번역하는 시스템을 만들 차례입니다.
그런데 문제가 있습니다.
\textbf{입력:} "I love you" (3단어) $\rightarrow$ \textbf{출력:} "Je t'aime" (2단어).
길이가 다릅니다. 기존 RNN은 이를 처리할 수 없습니다. 이를 해결하기 위해 구글이 제안한 혁명적인 아키텍처, \textbf{Seq2Seq (Encoder-Decoder)}를 배웁니다.

% --- 4. 개요 ---
\section{Unit Overview}
\begin{summarybox}{핵심 목표}
이 단원은 입력과 출력의 길이가 다른 시퀀스 변환 모델을 다룹니다.
\begin{itemize}
    \item \textbf{구조:} 입력 시퀀스를 압축하는 \textbf{인코더}와, 이를 풀어내는 \textbf{디코더} 구조를 이해합니다.
    \item \textbf{압축:} 인코더의 마지막 은닉 상태가 문장 전체의 의미를 담은 \textbf{문맥 벡터(Context Vector)}가 됨을 파악합니다.
    \item \textbf{전달:} 문맥 벡터가 디코더의 초기 상태로 전달되는 흐름을 코드로 구현합니다.
    \item \textbf{한계:} 문장이 길어지면 정보가 손실되는 \textbf{병목(Bottleneck)} 현상을 인지합니다.
\end{itemize}
\end{summarybox}

% --- 5. 용어 정리 ---
\section{Essential Terminology}
\begin{center}
\begin{tabular}{|c|l|l|}
\hline
\textbf{용어} & \textbf{역할} & \textbf{비유} \\ \hline
\textbf{Encoder} & 입력 문장을 읽고 이해함. & 문장을 읽는 번역가. \\ \hline
\textbf{Decoder} & 이해한 내용을 바탕으로 문장을 생성함. & 번역문을 쓰는 번역가. \\ \hline
\textbf{Context Vector} & 인코더가 넘겨주는 핵심 정보 (마지막 $h$). & 머릿속에 정리된 문장의 의미. \\ \hline
\end{tabular}
\end{center}

% --- 6. 핵심 개념 상세 설명 ---
\section{Core Concepts: 다 듣고 말하기}

\subsection{1. The Architecture: Encoder-Decoder}
Seq2Seq는 두 개의 RNN을 이어 붙인 구조입니다.



\begin{itemize}
    \item \textbf{Encoder:} 문장 $x^{\langle 1 \rangle} \dots x^{\langle T_x \rangle}$를 순서대로 읽습니다. 출력을 내지 않고 은닉 상태($h$)만 업데이트합니다. 마지막 상태 $h^{\langle T_x \rangle}$에 모든 정보를 압축합니다.
    \item \textbf{Decoder:} 인코더가 준 문맥 벡터를 \textbf{자신의 초기 상태($h^{\langle 0 \rangle}_{dec}$)}로 받습니다. 이를 바탕으로 번역문을 생성합니다.
\end{itemize}

\begin{analogybox}{번역가의 작업 방식}
\begin{itemize}
    \item \textbf{Read (Encoder):} 문장을 끝까지 다 읽습니다. 중간에 번역하지 않습니다. 머릿속에 핵심 의미(Context)를 정리합니다.
    \item \textbf{Write (Decoder):} 머릿속 의미를 바탕으로 프랑스어 문장을 처음부터 써 내려갑니다.
\end{itemize}
\end{analogybox}

\subsection{2. Training Trick: Teacher Forcing}
학습할 때는 디코더가 실수로 엉뚱한 단어를 예측했더라도, 다음 스텝 입력으로는 \textbf{정답 단어(Ground Truth)}를 넣어줍니다. 이를 \textbf{교사 강요(Teacher Forcing)}라고 합니다. (초반 학습 안정화용)

\vspace{0.5cm}\hrule\vspace{0.5cm}

\section{Implementation: Keras Functional API}

인코더의 상태를 디코더로 넘겨주는 핵심 코드를 작성합니다.

\begin{lstlisting}[language=Python, caption=Seq2Seq Model Definition, breaklines=true]
import tensorflow as tf
from tensorflow.keras.layers import Input, LSTM, Dense, Embedding
from tensorflow.keras.models import Model

def build_seq2seq(vocab_size, embedding_dim, latent_dim):
    
    # --- 1. Encoder ---
    enc_inputs = Input(shape=(None,))
    enc_emb = Embedding(vocab_size, embedding_dim)(enc_inputs)
    
    # return_state=True: 마지막 은닉 상태(h)와 셀 상태(c)를 반환
    encoder_lstm = LSTM(latent_dim, return_state=True)
    _, state_h, state_c = encoder_lstm(enc_emb)
    
    # 문맥 벡터 (Context Vector): 문장의 모든 정보를 압축한 것
    encoder_states = [state_h, state_c]
    
    # --- 2. Decoder ---
    dec_inputs = Input(shape=(None,))
    dec_emb = Embedding(vocab_size, embedding_dim)(dec_inputs)
    
    # return_sequences=True: 모든 타임 스텝 출력 (번역문 생성)
    decoder_lstm = LSTM(latent_dim, return_sequences=True, return_state=True)
    
    # 핵심: initial_state에 인코더의 상태를 주입!
    dec_outputs, _, _ = decoder_lstm(dec_emb, initial_state=encoder_states)
    
    decoder_dense = Dense(vocab_size, activation='softmax')
    dec_outputs = decoder_dense(dec_outputs)
    
    # 모델 정의
    model = Model([enc_inputs, dec_inputs], dec_outputs)
    return model
\end{lstlisting}

% --- 8. FAQ ---
\section{FAQ \& Pitfalls}

\begin{warningbox}{The Bottleneck Problem (병목 현상)}
Seq2Seq의 치명적 단점은 인코더가 문장이 길든 짧든 \textbf{고정된 크기의 벡터 하나}에 모든 정보를 우겨 넣어야 한다는 점입니다.
문장이 길어지면(50단어 이상) 정보 손실이 발생해 번역 품질이 급격히 떨어집니다. 이를 해결하기 위해 \textbf{Attention Mechanism}이 등장했습니다.
\end{warningbox}

\textbf{Q. 추론(Inference) 때는 어떻게 하나요?} \\
\textbf{A.} 학습 때와 달리 정답을 모르므로, 디코더가 방금 예측한 단어를 다음 스텝의 입력으로 넣어주는 루프(Loop)를 직접 구현해야 합니다. (Auto-regressive)

% --- 9. 다음 단원 연결 ---
\section*{🔗 다음 단계 (Next Step)}
이제 모델을 만들었습니다. 그런데 디코더가 단어를 생성할 때, 매 순간 가장 확률 높은 단어 하나만(Greedy) 고르면 최고의 문장이 될까요?
"The" 뒤에 "nice"가 올 확률이 높다고 골랐는데, 나중에 보니 "The huge building..."이 더 좋은 번역일 수도 있습니다.

다음 시간에는 번역 품질을 결정하는 결정적 탐색 알고리즘, \textbf{[Beam Search]}에 대해 알아보고, Seq2Seq의 병목 문제를 해결하는 \textbf{[Attention Mechanism]}으로 넘어가겠습니다.

\vspace{0.5cm}

\begin{summarybox}{단원 요약 (Cheat Sheet)}
\begin{enumerate}
    \item \textbf{Structure:} Encoder가 압축하고 Decoder가 푼다.
    \item \textbf{Context:} 인코더의 마지막 상태가 문장의 핵심 의미를 담은 매개체다.
    \item \textbf{Transfer:} $h_{enc}$를 $h_{dec}$의 `initial_state`로 전달한다.
    \item \textbf{Limit:} 긴 문장을 하나의 벡터로 압축하면 정보 손실이 발생한다.
\end{enumerate}
\end{summarybox}

\end{document}