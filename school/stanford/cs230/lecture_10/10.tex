\documentclass[a4paper, 11pt]{article}

% --- 패키지 설정 ---
\usepackage{kotex} % 한글 지원
\usepackage{geometry} % 여백 설정
\geometry{left=25mm, right=25mm, top=25mm, bottom=25mm}
\usepackage{amsmath, amssymb, amsfonts} % 수식 패키지
\usepackage{graphicx}
\usepackage{adjustbox}  % 표/박스 크기 조절 % 이미지 삽입
\usepackage{hyperref} % 하이퍼링크
\usepackage{xcolor} % 색상 지원
\usepackage{listings} % 코드 블록
\usepackage[most]{tcolorbox}
\tcbuselibrary{breakable} % 박스 디자인
\usepackage{enumitem} % 리스트 스타일
\usepackage{booktabs} % 표 디자인
\usepackage{array} % 표 정렬

% --- 색상 정의 ---
\definecolor{conceptblue}{RGB}{60, 100, 160}
\definecolor{analogygreen}{RGB}{80, 160, 100}
\definecolor{alertred}{RGB}{200, 60, 60}
\definecolor{exampleorange}{RGB}{230, 120, 30}
\definecolor{codegray}{rgb}{0.5,0.5,0.5}
\definecolor{backcolour}{rgb}{0.96,0.96,0.96}

% --- 코드 스타일 설정 ---
\lstdefinestyle{mystyle}{
    backgroundcolor=\color{backcolour},   
    commentstyle=\color{analogygreen},
    keywordstyle=\color{conceptblue},
    numberstyle=\tiny\color{codegray},
    stringstyle=\color{exampleorange},
    basicstyle=\ttfamily\footnotesize,
    breakatwhitespace=false,         
    breaklines=true,                 
    captionpos=b,                    
    keepspaces=true,                 
    numbers=left,                    
    numbersep=5pt,                  
    showspaces=false,                
    showstringspaces=false,
    showtabs=false,                  
    tabsize=4,
    frame=single
}
\lstset{style=mystyle}

% --- 박스 스타일 정의 ---
\newtcolorbox{summarybox}[1]{
    colback=conceptblue!5!white,
    colframe=conceptblue!80!black,
    fonttitle=\bfseries,
    title=📌 #1
}

\newtcolorbox{analogybox}[1]{
    colback=analogygreen!5!white,
    colframe=analogygreen!80!black,
    fonttitle=\bfseries,
    title=💡 #1 (직관적 비유)
}

\newtcolorbox{warningbox}[1]{
    colback=alertred!5!white,
    colframe=alertred!80!black,
    fonttitle=\bfseries,
    title=⚠️ #1 (오해 방지 가이드)
}

\newtcolorbox{examplebox}[1]{
    colback=exampleorange!5!white,
    colframe=exampleorange!80!black,
    fonttitle=\bfseries,
    title=🧮 #1 (실전 시나리오 \& 계산)
}

% --- 문서 정보 ---
\title{\textbf{[CS230] Deep Neural Networks: \\ Dimensions \& Initialization}}
\author{Lecturer: Gemini (Integrated Editor)}
\date{}

\begin{document}

\maketitle

% --- 1. 전체 목차 (TOC) ---
\section*{📚 Course Table of Contents}
\begin{itemize}
    \item[Chapter 1-3.] Foundations \& Shallow Networks \textit{- Completed}
    \item[Chapter 4.] Deep Neural Networks
    \begin{itemize}
        \item 4.1 Deep L-Layer Neural Network Architecture \textit{- Completed}
        \item \textbf{4.2 Dimensions \& Initialization (Current Unit)}
        \begin{itemize}
            \item The Law of Matrix Dimensions
            \item Symmetry Breaking (Why not Zero?)
            \item He Initialization (The Standard for ReLU)
            \item Implementation \& Verification
        \end{itemize}
        \item 4.3 Building a Deep Neural Network Application \textit{- Upcoming}
    \end{itemize}
\end{itemize}

\vspace{0.5cm}
\hrule
\vspace{0.5cm}

% --- 3. 이전 단원 연결 ---
\section*{🔗 지난 시간 복습 및 연결}
우리는 이제 거대한 심층 신경망을 설계할 수 있는 건축가가 되었습니다. 하지만 설계도만 그렸을 뿐, 아직 착공도 하지 않았습니다.
건물을 올리기 전에 가장 먼저 해야 할 일은 무엇일까요? \textbf{설계도 검증(차원 확인)}과 \textbf{기초 공사(초기화)}입니다. 이 두 가지를 소홀히 하면 코드를 실행하자마자 에러가 터지거나(Dimension Mismatch), 에러 메시지 하나 없이 학습이 전혀 안 되는(Bad Initialization) 침묵의 버그를 만나게 됩니다.

% --- 4. 개요 ---
\section{Unit Overview}
\begin{summarybox}{핵심 목표}
이 단원은 디버깅 시간을 획기적으로 줄여주는 \textbf{'차원 분석'}과 학습 성공의 열쇠인 \textbf{'파라미터 초기화'}를 다룹니다.
\begin{itemize}
    \item \textbf{분석:} $L$층 신경망의 파라미터($W, b$)와 데이터($Z, A$)의 형상(Shape)을 정확히 도출합니다.
    \item \textbf{이유:} 가중치를 0으로 초기화했을 때 발생하는 \textbf{'대칭성 문제(Symmetry Problem)'}를 증명합니다.
    \item \textbf{해결:} ReLU를 위한 표준 초기화 방법인 \textbf{He Initialization}의 원리와 코드를 익힙니다.
\end{itemize}
\end{summarybox}

% --- 5. 용어 정리 ---
\section{Essential Terminology}
\begin{center}
\begin{tabular}{|c|l|l|}
\hline
\textbf{용어} & \textbf{설명} & \textbf{핵심 포인트} \\ \hline
\textbf{Shape} & 행렬의 차원 (행, 열) & 디버깅의 90\%는 Shape 맞추기입니다. \\ \hline
\textbf{Symmetry Breaking} & 대칭성 파괴 & 뉴런들이 서로 다르게 학습되도록 초기값을 다르게 주는 것. \\ \hline
\textbf{Zero Init} & 0으로 초기화 & 모든 뉴런이 똑같이 동작하게 만드는 \textbf{최악의 방법}. \\ \hline
\textbf{He Init} & He 초기화 & ReLU 사용 시 분산을 유지해주는 \textbf{최고의 방법}. \\ \hline
\textbf{Xavier Init} & Xavier 초기화 & Sigmoid/Tanh 사용 시 적합한 초기화 방법. \\ \hline
\end{tabular}
\end{center}

% --- 6. 핵심 개념 상세 설명 ---
\section{Core Concepts: 디버깅을 위한 헌법}

\subsection{1. Matrix Dimensions Rules (차원의 법칙)}
코딩하다 헷갈릴 때마다 이 표를 보십시오. $n^{[l]}$은 현재 층의 뉴런 수, $m$은 데이터 개수입니다.

\begin{tcolorbox}[colback=white, colframe=black, title=Shape Cheat Sheet]
\begin{itemize}
    \item \textbf{파라미터 (학습 대상):}
    \begin{itemize}
        \item $W^{[l]}$: $(n^{[l]}, n^{[l-1]})$ $\rightarrow$ (현재 층, 이전 층)
        \item $b^{[l]}$: $(n^{[l]}, 1)$ $\rightarrow$ 열 벡터 (Column Vector)
    \end{itemize}
    \item \textbf{데이터 흐름 (Activations):}
    \begin{itemize}
        \item $Z^{[l]}, A^{[l]}$: $(n^{[l]}, m)$
        \item $dZ^{[l]}, dA^{[l]}$: $(n^{[l]}, m)$ $\rightarrow$ 원래 데이터와 Shape 동일
    \end{itemize}
\end{itemize}
\end{tcolorbox}

\vspace{0.5cm}\hrule\vspace{0.5cm}

\subsection{2. Why Not Zero Initialization? (0 초기화의 저주)}
"교수님, 로지스틱 회귀에선 0으로 해도 잘 됐잖아요?"
네, 하지만 \textbf{신경망(은닉층이 있는 경우)에서는 절대 안 됩니다.}

\begin{analogybox}{복제 인간 군대 비유}
\begin{itemize}
    \item \textbf{상황:} 모든 가중치 $W$를 0으로 초기화했습니다.
    \item \textbf{Forward:} 모든 은닉 뉴런이 입력값에 상관없이 똑같은 값(0)을 계산합니다.
    \item \textbf{Backward:} 모든 뉴런이 똑같은 오차(Gradient)를 보고받습니다.
    \item \textbf{Update:} 모든 뉴런이 똑같은 값으로 수정됩니다.
    \item \textbf{결과:} 뉴런이 100만 개여도, 결국 \textbf{뉴런 1개짜리 선형 모델}과 똑같이 행동합니다. 이를 \textbf{대칭성(Symmetry)} 문제라고 하며, 학습이 실패합니다.
\end{itemize}
\end{analogybox}

\vspace{0.5cm}\hrule\vspace{0.5cm}

\subsection{3. Best Practice: He Initialization}
그렇다면 랜덤하게(Random) 초기화하면 될까요? 
너무 크면(x10) 기울기 소실이 오고, 너무 작으면(x0.0001) 신호가 죽어버립니다.



Kaiming He 박사가 제안한 \textbf{He Initialization}은 ReLU를 사용할 때 분산을 일정하게 유지해주는 마법의 공식입니다.

$$ W^{[l]} \sim \text{Random} \times \sqrt{\frac{2}{n^{[l-1]}}} $$
\begin{itemize}
    \item 이전 층의 뉴런 개수($n^{[l-1]}$)가 많을수록, 가중치를 더 작게 만들어 줍니다.
    \item $\sqrt{2}$는 ReLU가 음수 영역을 0으로 만들어 분산을 절반으로 깎아먹는 것을 보상해줍니다.
\end{itemize}

% --- 7. 구현 코드 ---
\section{Implementation: Initialization Strategies}

나쁜 예(Zero, Large Random)와 좋은 예(He)를 코드로 비교해봅시다.

\begin{lstlisting}[language=Python, caption=Parameter Initialization Methods, breaklines=true]
import numpy as np

class Initializer:
    def __init__(self, layer_dims):
        self.layer_dims = layer_dims # 예: [1000, 100, 10]
        self.L = len(layer_dims) - 1

    def init_zeros(self):
        """
        BAD: 모든 가중치를 0으로 초기화 -> 학습 불가
        """
        params = {}
        for l in range(1, self.L + 1):
            params['W' + str(l)] = np.zeros((self.layer_dims[l], self.layer_dims[l-1]))
            params['b' + str(l)] = np.zeros((self.layer_dims[l], 1))
        return params

    def init_he(self):
        """
        BEST: He Initialization (Standard for ReLU)
        """
        params = {}
        for l in range(1, self.L + 1):
            # 1. 차원 정의
            n_curr = self.layer_dims[l]
            n_prev = self.layer_dims[l-1]
            
            # 2. He Initialization 공식 적용
            # np.random.randn: 평균 0, 분산 1인 정규분포
            # scaling: 분산을 2/n_prev 로 맞춰줌
            scaling = np.sqrt(2 / n_prev)
            
            params['W' + str(l)] = np.random.randn(n_curr, n_prev) * scaling
            params['b' + str(l)] = np.zeros((n_curr, 1)) # 편향은 0이어도 됨!
            
        return params

# --- 검증 ---
if __name__ == "__main__":
    dims = [1000, 100, 10]
    init = Initializer(dims)
    
    # He Init 결과 확인
    params = init.init_he()
    W1 = params['W1']
    
    print("Shape Check:", W1.shape) # (100, 1000)
    print("Variance Check:", np.var(W1)) 
    print("Expected Variance:", 2/1000) # 0.002 근처여야 함
\end{lstlisting}

% --- 8. FAQ ---
\section{FAQ \& Pitfalls}

\begin{warningbox}{편향(Bias) $b$는 0으로 해도 되나요?}
\textbf{네, 됩니다!}
대칭성 문제는 가중치 $W$에서 발생합니다. $W$가 이미 랜덤하게 섞여 있다면(Symmetry Broken), 편향 $b$가 모두 0이어도 뉴런들은 서로 다른 값을 출력하게 됩니다. 따라서 $b$는 편의상 `np.zeros`로 초기화하는 것이 일반적입니다.
\end{warningbox}

\textbf{Q. Xavier 초기화는 뭔가요?} \\
\textbf{A.} Sigmoid나 Tanh 함수를 쓸 때 사용하는 초기화 방법입니다. 계수가 $\sqrt{1/n}$입니다. 하지만 요즘 딥러닝은 대부분 ReLU를 쓰기 때문에 He 초기화($\sqrt{2/n}$)가 더 많이 쓰입니다.

% --- 9. 다음 단원 연결 ---
\section*{🔗 다음 단계 (Next Step)}
이제 우리는 \textbf{설계(Architecture)}, \textbf{기초 공사(Initialization)}, \textbf{자재 검수(Dimension Check)}까지 완벽하게 마쳤습니다.

이제 남은 것은 건물을 짓는 것뿐입니다. 다음 시간에는 \textbf{[Project] Building a Deep Neural Network Application}을 통해, 우리가 만든 코드로 \textbf{'고양이 vs 개'} 이미지를 분류하는 인공지능을 완성하겠습니다. 여러분의 첫 번째 Deep Learning 프로젝트가 시작됩니다!

\vspace{0.5cm}

\begin{summarybox}{단원 요약 (Cheat Sheet)}
\begin{enumerate}
    \item \textbf{Dimensions:} $W$는 $(n^{[l]}, n^{[l-1]})$이다. 차원 확인이 디버깅의 시작이다.
    \item \textbf{Zero Init:} $W$를 0으로 하면 학습이 안 된다. (대칭성 문제)
    \item \textbf{He Init:} ReLU를 쓸 때는 `randn * sqrt(2/n)` 공식을 사용하라.
    \item \textbf{Bias:} 편향 $b$는 0으로 초기화해도 안전하다.
\end{enumerate}
\end{summarybox}

\end{document}