\documentclass[a4paper, 11pt]{article}

% --- 패키지 설정 ---
\usepackage{kotex} % 한글 지원
\usepackage{geometry} % 여백 설정
\geometry{left=25mm, right=25mm, top=25mm, bottom=25mm}
\usepackage{amsmath, amssymb, amsfonts} % 수식 패키지
\usepackage{graphicx}
\usepackage{adjustbox}  % 표/박스 크기 조절 % 이미지 삽입
\usepackage{hyperref} % 하이퍼링크
\usepackage{xcolor} % 색상 지원
\usepackage{listings} % 코드 블록
\usepackage[most]{tcolorbox}
\tcbuselibrary{breakable} % 박스 디자인
\usepackage{enumitem} % 리스트 스타일
\usepackage{booktabs} % 표 디자인
\usepackage{array} % 표 정렬

% --- 색상 정의 ---
\definecolor{conceptblue}{RGB}{60, 100, 160}
\definecolor{analogygreen}{RGB}{80, 160, 100}
\definecolor{alertred}{RGB}{200, 60, 60}
\definecolor{exampleorange}{RGB}{230, 120, 30}
\definecolor{codegray}{rgb}{0.5,0.5,0.5}
\definecolor{backcolour}{rgb}{0.96,0.96,0.96}

% --- 코드 스타일 설정 ---
\lstdefinestyle{mystyle}{
    backgroundcolor=\color{backcolour},   
    commentstyle=\color{analogygreen},
    keywordstyle=\color{conceptblue},
    numberstyle=\tiny\color{codegray},
    stringstyle=\color{exampleorange},
    basicstyle=\ttfamily\footnotesize,
    breakatwhitespace=false,         
    breaklines=true,                 
    captionpos=b,                    
    keepspaces=true,                 
    numbers=left,                    
    numbersep=5pt,                  
    showspaces=false,                
    showstringspaces=false,
    showtabs=false,                  
    tabsize=4,
    frame=single
}
\lstset{style=mystyle}

% --- 박스 스타일 정의 ---
\newtcolorbox{summarybox}[1]{
    colback=conceptblue!5!white,
    colframe=conceptblue!80!black,
    fonttitle=\bfseries,
    title=📌 #1
}

\newtcolorbox{analogybox}[1]{
    colback=analogygreen!5!white,
    colframe=analogygreen!80!black,
    fonttitle=\bfseries,
    title=💡 #1 (직관적 비유)
}

\newtcolorbox{warningbox}[1]{
    colback=alertred!5!white,
    colframe=alertred!80!black,
    fonttitle=\bfseries,
    title=⚠️ #1 (오해 방지 가이드)
}

\newtcolorbox{mathbox}[1]{
    colback=exampleorange!5!white,
    colframe=exampleorange!80!black,
    fonttitle=\bfseries,
    title=🧮 #1 (수학적 증명)
}

% --- 문서 정보 ---
\title{\textbf{[CS230] Convolutional Neural Networks: \\ ResNet (Residual Networks)}}
\author{Lecturer: Gemini (Integrated Editor)}
\date{}

\begin{document}

\maketitle

% --- 1. 전체 목차 (TOC) ---
\section*{📚 Course Table of Contents}
\begin{itemize}
    \item[Chapter 1-8.] Deep Learning Strategy \& Architecture \textit{- Completed}
    \item[\textbf{Chapter 9.}] \textbf{Convolutional Neural Networks (Current Part)}
    \begin{itemize}
        \item 9.1-9.3 CNN Basics \& Classic Networks \textit{- Completed}
        \item \textbf{9.4 ResNet (Residual Networks)}
        \begin{itemize}
            \item The Degradation Problem
            \item Skip Connections: The Shortcut
            \item Why $H(x) = F(x) + x$?
            \item Implementation: Identity \& Conv Blocks
        \end{itemize}
        \item 9.5 Inception Network \textit{- Upcoming}
    \end{itemize}
\end{itemize}

\vspace{0.5cm}
\hrule
\vspace{0.5cm}

% --- 3. 이전 단원 연결 ---
\section*{🔗 지난 시간 복습 및 연결}
지난 시간에 배운 VGG-16은 16개 층으로 뛰어난 성능을 보였습니다. 그렇다면 질문이 생깁니다.
\textbf{"층을 100개, 1000개로 늘리면 성능이 더 좋아지지 않을까요?"}
이론적으로는 그래야 합니다. 하지만 실제로는 층이 20개를 넘어가면 성능이 급격히 떨어지는 \textbf{퇴보(Degradation)} 현상이 발생했습니다. 깊은 망을 학습시키는 것 자체가 너무 어려웠던 것입니다.
이 난제를 해결하고 딥러닝 역사를 새로 쓴 모델이 바로 \textbf{ResNet}입니다. 핵심은 \textbf{"지름길(Shortcut)"}을 뚫어주는 것입니다.

% --- 4. 개요 ---
\section{Unit Overview}
\begin{summarybox}{핵심 목표}
이 단원은 152층 이상의 초심층 신경망 학습을 가능하게 한 ResNet의 핵심 원리를 다룹니다.
\begin{itemize}
    \item \textbf{문제:} 망이 깊어질수록 학습이 안 되는 이유(기울기 소실 등)를 이해합니다.
    \item \textbf{해결:} 입력 $x$를 출력에 더해주는 \textbf{Skip Connection} 구조를 파악합니다.
    \item \textbf{원리:} 잔차 블록이 어떻게 항등 매핑($H(x)=x$)을 쉽게 학습하는지 수식으로 증명합니다.
    \item \textbf{구현:} Keras로 Identity Block과 Convolutional Block을 구현합니다.
\end{itemize}
\end{summarybox}

% --- 5. 용어 정리 ---
\section{Essential Terminology}
\begin{center}
\begin{tabular}{|c|l|l|}
\hline
\textbf{용어} & \textbf{의미} & \textbf{핵심} \\ \hline
\textbf{Skip Connection} & 입력을 몇 층 건너뛰어 출력에 더하는 연결선 & 지름길 (Shortcut) \\ \hline
\textbf{Residual (잔차)} & 학습해야 할 차이 ($F(x)$) & $H(x) - x$ \\ \hline
\textbf{Identity Mapping} & 입력을 그대로 출력하는 것 ($H(x)=x$) & 기본은 한다. \\ \hline
\end{tabular}
\end{center}

% --- 6. 핵심 개념 상세 설명 ---
\section{Core Concepts: 지름길의 마법}

\subsection{1. The Degradation Problem (성능 저하)}

일반적인 네트워크(Plain Network)에 층을 계속 추가하면, 학습 데이터에 대한 에러조차 높아집니다. 기울기(Gradient)가 입력층까지 도달하지 못하고 사라져버리기 때문입니다.

\subsection{2. Skip Connection (잔차 연결)}


ResNet의 아이디어는 간단합니다. \textbf{"정보가 흐르는 고속도로를 뚫어주자."}
\begin{itemize}
    \item \textbf{Main Path:} 합성곱 층을 통과하여 $F(x)$를 계산합니다.
    \item \textbf{Shortcut:} 입력 $x$를 그대로 가져와서 더합니다.
    \item \textbf{Output:} $H(x) = F(x) + x$
\end{itemize}

\begin{analogybox}{교과서 암기 비유}
\begin{itemize}
    \item \textbf{Plain:} "백지상태에서 교과서 전체($H(x)$)를 다 외워라." (어렵다)
    \item \textbf{ResNet:} "너는 이미 $x$만큼 알고 있으니, 교과서 내용과 네 지식의 \textbf{차이($F(x)$)}만 추가로 공부해라." (쉽다)
\end{itemize}
\end{analogybox}

\vspace{0.5cm}\hrule\vspace{0.5cm}

\section{Deep Dive: Why does it work?}

\begin{mathbox}{Gradient Superhighway}
역전파 시 미분값이 전달되는 과정을 봅시다.
$$ H(x) = F(x) + x $$
$$ \frac{\partial H}{\partial x} = \frac{\partial F(x)}{\partial x} + \mathbf{1} $$
\textbf{의미:} 복잡한 합성곱 경로($F$)의 미분값이 0이 되어도(Vanishing), 지름길 경로($+1$)가 살아있습니다. 기울기가 소실되지 않고 네트워크 앞단까지 그대로 전달됩니다.
\end{mathbox}

\subsection{3. Dimension Matching (차원 일치)}
$F(x)$와 $x$를 더하려면 두 행렬의 크기가 같아야 합니다.
\begin{itemize}
    \item \textbf{Identity Block:} 입출력 크기가 같을 때. 그냥 더함.
    \item \textbf{Convolutional Block:} 크기가 다를 때(Pooling 등). $x$에도 $1 \times 1$ Conv를 적용해 크기를 맞춰준 뒤 더함.
\end{itemize}

% --- 7. 구현 코드 ---
\section{Implementation: ResNet Block}

Keras Functional API를 사용한 구현입니다.

\begin{lstlisting}[language=Python, caption=ResNet Identity Block, breaklines=true]
from tensorflow.keras import layers, models

def identity_block(X, f, filters):
    """
    X: 입력 텐서
    f: 커널 크기 (중간층)
    filters: 필터 개수 리스트 [F1, F2, F3]
    """
    F1, F2, F3 = filters
    X_shortcut = X # 입력 저장 (지름길용)
    
    # --- Main Path (3개의 Conv 층) ---
    # 1. 1x1 Conv (차원 축소/확장용)
    X = layers.Conv2D(filters=F1, kernel_size=(1, 1), padding='valid')(X)
    X = layers.BatchNormalization()(X)
    X = layers.Activation('relu')(X)
    
    # 2. fxf Conv (메인 연산)
    X = layers.Conv2D(filters=F2, kernel_size=(f, f), padding='same')(X)
    X = layers.BatchNormalization()(X)
    X = layers.Activation('relu')(X)

    # 3. 1x1 Conv (차원 복원)
    X = layers.Conv2D(filters=F3, kernel_size=(1, 1), padding='valid')(X)
    X = layers.BatchNormalization()(X)
    
    # --- Skip Connection (핵심) ---
    # Main Path 결과와 지름길(Original X)을 더함
    X = layers.Add()([X, X_shortcut])
    
    # 더한 뒤에 ReLU 적용 (중요!)
    X = layers.Activation('relu')(X)
    
    return X
\end{lstlisting}

% --- 8. FAQ ---
\section{FAQ \& Pitfalls}

\begin{warningbox}{ReLU 위치 주의}
Skip Connection을 더하는 `Add()` 연산은 마지막 ReLU \textbf{이전}에 수행되어야 합니다.
(X + Shortcut) $\rightarrow$ ReLU. 순서가 바뀌면 성능이 떨어집니다.
\end{warningbox}

\textbf{Q. ResNet-50의 'Bottleneck' 구조가 뭔가요?} \\
\textbf{A.} $1 \times 1$, $3 \times 3$, $1 \times 1$ 순서로 쌓은 블록입니다.
$1 \times 1$로 채널을 줄였다가(압축), 연산하고, 다시 늘립니다. 연산량을 줄이면서 깊이를 늘리기 위한 테크닉입니다.

% --- 9. 다음 단원 연결 ---
\section*{🔗 다음 단계 (Next Step)}
ResNet을 통해 우리는 깊이에 대한 두려움을 극복했습니다.
그런데 비슷한 시기에 구글에서는 깊이가 아니라 \textbf{"너비(Width)"}와 \textbf{"다양성"}에 집중한 모델을 내놓았습니다. $1 \times 1$, $3 \times 3$, $5 \times 5$ 필터를 한 층에서 동시에 쓴다면 어떨까요?

다음 시간에는 \textbf{Network within a Network}라고 불리는 $1 \times 1$ Convolution의 마법과, 이를 활용한 \textbf{[Inception Network]}에 대해 알아보겠습니다.

\vspace{0.5cm}

\begin{summarybox}{단원 요약 (Cheat Sheet)}
\begin{enumerate}
    \item \textbf{Problem:} 너무 깊으면 학습이 안 된다 (Degradation).
    \item \textbf{Solution:} Skip Connection ($H(x) = F(x) + x$).
    \item \textbf{Math:} 미분 시 $+1$ 항이 생겨 기울기 소실을 막는다.
    \item \textbf{Block:} 차원이 같으면 Identity Block, 다르면 Conv Block을 쓴다.
\end{enumerate}
\end{summarybox}

\end{document}