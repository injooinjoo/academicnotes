\documentclass[a4paper, 11pt]{article}

% --- 패키지 설정 ---
\usepackage{kotex} % 한글 지원
\usepackage{geometry} % 여백 설정
\geometry{left=25mm, right=25mm, top=25mm, bottom=25mm}
\usepackage{amsmath, amssymb, amsfonts} % 수식 패키지
\usepackage{graphicx}
\usepackage{adjustbox}  % 표/박스 크기 조절 % 이미지 삽입
\usepackage{hyperref} % 하이퍼링크
\usepackage{xcolor} % 색상 지원
\usepackage{listings} % 코드 블록
\usepackage[most]{tcolorbox}
\tcbuselibrary{breakable} % 박스 디자인
\usepackage{enumitem} % 리스트 스타일
\usepackage{booktabs} % 표 디자인
\usepackage{array} % 표 정렬

% --- 색상 정의 ---
\definecolor{conceptblue}{RGB}{60, 100, 160}
\definecolor{analogygreen}{RGB}{80, 160, 100}
\definecolor{alertred}{RGB}{200, 60, 60}
\definecolor{exampleorange}{RGB}{230, 120, 30}
\definecolor{codegray}{rgb}{0.5,0.5,0.5}
\definecolor{backcolour}{rgb}{0.96,0.96,0.96}

% --- 코드 스타일 설정 ---
\lstdefinestyle{mystyle}{
    backgroundcolor=\color{backcolour},   
    commentstyle=\color{analogygreen},
    keywordstyle=\color{conceptblue},
    numberstyle=\tiny\color{codegray},
    stringstyle=\color{exampleorange},
    basicstyle=\ttfamily\footnotesize,
    breakatwhitespace=false,         
    breaklines=true,                 
    captionpos=b,                    
    keepspaces=true,                 
    numbers=left,                    
    numbersep=5pt,                  
    showspaces=false,                
    showstringspaces=false,
    showtabs=false,                  
    tabsize=4,
    frame=single
}
\lstset{style=mystyle}

% --- 박스 스타일 정의 ---
\newtcolorbox{summarybox}[1]{
    colback=conceptblue!5!white,
    colframe=conceptblue!80!black,
    fonttitle=\bfseries,
    title=📌 #1
}

\newtcolorbox{analogybox}[1]{
    colback=analogygreen!5!white,
    colframe=analogygreen!80!black,
    fonttitle=\bfseries,
    title=💡 #1 (직관적 비유)
}

\newtcolorbox{warningbox}[1]{
    colback=alertred!5!white,
    colframe=alertred!80!black,
    fonttitle=\bfseries,
    title=⚠️ #1 (오해 방지 가이드)
}

\newtcolorbox{featurebox}[1]{
    colback=purple!5!white,
    colframe=purple!80!black,
    fonttitle=\bfseries,
    title=🔍 #1 (핵심 특징)
}

% --- 문서 정보 ---
\title{\textbf{[CS230] Convolutional Neural Networks: \\ Pooling Layers}}
\author{Lecturer: Gemini (Integrated Editor)}
\date{}

\begin{document}

\maketitle

% --- 1. 전체 목차 (TOC) ---
\section*{📚 Course Table of Contents}
\begin{itemize}
    \item[Chapter 1-8.] Deep Learning Strategy \& Architecture \textit{- Completed}
    \item[\textbf{Chapter 9.}] \textbf{Convolutional Neural Networks (Current Part)}
    \begin{itemize}
        \item 9.1 CNN Foundations: Convolution, Padding, Strides \textit{- Completed}
        \item \textbf{9.2 Pooling Layers (Max/Average)}
        \begin{itemize}
            \item Concept: Downsampling \& Invariance
            \item Max Pooling vs Average Pooling
            \item Channel Independence Rule
            \item Implementation
        \end{itemize}
        \item 9.3 CNN Example: LeNet-5 \textit{- Upcoming}
    \end{itemize}
\end{itemize}

\vspace{0.5cm}
\hrule
\vspace{0.5cm}

% --- 3. 이전 단원 연결 ---
\section*{🔗 지난 시간 복습 및 연결}
지난 시간에 우리는 합성곱(Conv) 연산으로 이미지의 특징을 추출하는 법을 배웠습니다. 하지만 합성곱 층만 계속 쌓으면 연산량이 너무 많아지고, 모델이 이미지의 미세한 변화(1픽셀 이동 등)에 너무 민감해집니다.
우리는 \textbf{"중요한 특징만 남기고, 크기는 줄여서 효율적으로"} 처리하고 싶습니다. 이 두 마리 토끼를 잡는 기술이 바로 \textbf{풀링(Pooling)}입니다.

% --- 4. 개요 ---
\section{Unit Overview}
\begin{summarybox}{핵심 목표}
이 단원은 CNN의 핵심 구성 요소인 풀링 층의 원리와 종류를 다룹니다.
\begin{itemize}
    \item \textbf{다운샘플링:} 이미지 크기($n_H, n_W$)를 줄여 계산량을 낮추는 원리를 이해합니다.
    \item \textbf{불변성:} 풀링이 어떻게 평행 이동에 대한 \textbf{강건함(Invariance)}을 제공하는지 배웁니다.
    \item \textbf{비교:} 가장 널리 쓰이는 \textbf{Max Pooling}과 과거에 쓰였던 \textbf{Average Pooling}을 비교합니다.
    \item \textbf{특징:} 풀링 층에는 \textbf{학습 파라미터($W, b$)가 없다}는 점을 명심합니다.
\end{itemize}
\end{summarybox}

% --- 5. 용어 정리 ---
\section{Essential Terminology}
\begin{center}
\begin{tabular}{|c|l|l|}
\hline
\textbf{용어} & \textbf{의미} & \textbf{핵심 역할} \\ \hline
\textbf{Max Pooling} & 영역 내 최댓값 선택 & 가장 강한 특징만 남김 (대세). \\ \hline
\textbf{Average Pooling} & 영역 내 평균값 계산 & 정보를 부드럽게 요약함. \\ \hline
\textbf{Invariant} & 불변성 & 입력이 조금 바뀌어도 출력은 변하지 않음. \\ \hline
\textbf{Hyperparameters} & $f$(필터 크기), $s$(스트라이드) & 풀링 동작을 결정하는 설정값. \\ \hline
\end{tabular}
\end{center}

% --- 6. 핵심 개념 상세 설명 ---
\section{Core Concepts: 요약의 기술}

\subsection{1. Max Pooling (최대 풀링)}


가장 널리 쓰이는 방식입니다. 필터 영역($f \times f$) 내에서 \textbf{가장 큰 숫자 하나}만 골라냅니다.
\begin{itemize}
    \item \textbf{동작:} $2 \times 2$ 윈도우로 이미지를 훑으며 최댓값을 뽑습니다.
    \item \textbf{의미:} "이 구역에 고양이 눈(특징)이 있는가?" $\rightarrow$ \textbf{YES (높은 값)}. 정확히 어디(좌상단? 우하단?)에 있는지는 중요하지 않습니다. 존재 유무만 강조합니다.
\end{itemize}

\subsection{2. Average Pooling (평균 풀링)}


필터 영역 내의 숫자를 모두 더해 평균을 냅니다.
\begin{itemize}
    \item \textbf{의미:} 특징을 부드럽게(Smoothing) 만듭니다.
    \item \textbf{용도:} 과거에는 많이 썼으나 최근에는 잘 안 씁니다. 단, 모델의 맨 마지막단(Global Average Pooling)에서는 여전히 유용합니다.
\end{itemize}

\vspace{0.5cm}\hrule\vspace{0.5cm}

\section{Deep Dive: Channel Independence}

이 부분이 합성곱(Convolution)과 가장 헷갈리는 지점입니다.

\begin{featurebox}{채널 독립성의 법칙}
\begin{itemize}
    \item \textbf{Convolution:} 입력 채널(RGB 3개)을 \textbf{모두 합쳐서(Sum)} 하나의 숫자로 만듭니다. 채널 수가 변합니다.
    \item \textbf{Pooling:} 각 채널을 \textbf{독립적으로(Independently)} 처리합니다.
    \begin{itemize}
        \item 입력: $32 \times 32 \times \mathbf{10}$
        \item 풀링: $16 \times 16 \times \mathbf{10}$
        \item \textbf{결과:} 높이/너비는 줄지만, \textbf{채널 수(깊이)는 그대로 유지}됩니다.
    \end{itemize}
\end{itemize}
\end{featurebox}

\begin{warningbox}{파라미터 개수는 몇 개?}
풀링 층에는 학습해야 할 가중치($W$)나 편향($b$)이 있을까요?
\textbf{정답: 0개입니다.}
풀링은 우리가 정해준 규칙($f, s$, Max/Avg)대로만 계산하는 고정된 함수입니다. 역전파 때 업데이트될 대상이 없습니다.
\end{warningbox}

% --- 7. 구현 코드 ---
\section{Implementation: NumPy Pooling}

원리 이해를 위해 4중 루프를 사용하여 직접 구현해 봅니다.

\begin{lstlisting}[language=Python, caption=Max \& Average Pooling Implementation, breaklines=true]
import numpy as np

class Pooling:
    def __init__(self, f=2, s=2, mode='max'):
        self.f = f
        self.s = s
        self.mode = mode

    def forward(self, A_prev):
        """
        A_prev: (m, n_H, n_W, n_C)
        """
        (m, n_H_prev, n_W_prev, n_C_prev) = A_prev.shape
        
        # 출력 크기 계산 (공식 적용)
        n_H = int((n_H_prev - self.f) / self.s) + 1
        n_W = int((n_W_prev - self.f) / self.s) + 1
        n_C = n_C_prev # 채널 수는 그대로!
        
        A = np.zeros((m, n_H, n_W, n_C))
        
        for i in range(m):              # 데이터 샘플
            for h in range(n_H):        # 세로 이동
                for w in range(n_W):    # 가로 이동
                    for c in range(n_C):# 채널 (독립적)
                        
                        # 윈도우 슬라이싱
                        vert_start = h * self.s
                        vert_end   = vert_start + self.f
                        horiz_start= w * self.s
                        horiz_end  = horiz_start + self.f
                        
                        slice_A = A_prev[i, vert_start:vert_end, horiz_start:horiz_end, c]
                        
                        if self.mode == 'max':
                            A[i, h, w, c] = np.max(slice_A)
                        elif self.mode == 'average':
                            A[i, h, w, c] = np.mean(slice_A)
        return A

# --- 실행 ---
if __name__ == "__main__":
    # 4x4 이미지, 채널 1개
    img = np.array([[[[1],[3],[2],[1]],
                     [[2],[9],[1],[1]],
                     [[1],[3],[2],[3]],
                     [[5],[6],[1],[2]]]]) # shape (1,4,4,1)
    
    pool = Pooling(f=2, s=2, mode='max')
    out = pool.forward(img)
    
    print("Input:\n", img[0,:,:,0])
    print("Max Pool Output:\n", out[0,:,:,0])
    # 예상: [[9, 2], [6, 3]]
\end{lstlisting}

% --- 8. FAQ ---
\section{FAQ \& Pitfalls}

\textbf{Q. 풀링을 하면 정보가 사라지는데 괜찮나요?} \\
\textbf{A.} 네, 그게 목적입니다. 불필요한 배경이나 노이즈는 버리고, \textbf{"여기 특징이 있다!"(Max Value)}라는 핵심 정보만 남겨서 다음 층으로 전달하는 것이 CNN의 추상화 과정입니다.

\textbf{Q. $f=3, s=2$ 같은 건 언제 쓰나요?} \\
\textbf{A.} 보통은 $f=2, s=2$ (크기 절반 축소)가 국룰입니다. 하지만 겹치는 영역을 두고 싶을 때(Overlapping Pooling) $f=3, s=2$를 쓰기도 합니다 (예: AlexNet).

% --- 9. 다음 단원 연결 ---
\section*{🔗 다음 단계 (Next Step)}
이제 우리는 CNN을 만드는 3대 요소인 \textbf{Convolution, ReLU(Activation), Pooling}을 모두 배웠습니다. 레고 블록이 다 모였습니다.

이제 이것들을 어떻게 조립해야 할까요?
다음 시간에는 이 블록들을 결합하여 숫자 필기체(MNIST)를 인식하는 전설적인 CNN 아키텍처, \textbf{[LeNet-5 Example]}을 통해 첫 번째 완전한 CNN 모델을 구축해보겠습니다.

\vspace{0.5cm}

\begin{summarybox}{단원 요약 (Cheat Sheet)}
\begin{enumerate}
    \item \textbf{Max Pooling:} 영역 내 최댓값만 남겨 특징을 강조한다. (가장 많이 씀)
    \item \textbf{Dimension:} 가로/세로는 줄어들지만, \textbf{채널 수($n_C$)는 유지된다.}
    \item \textbf{Parameters:} 학습할 가중치($W$)가 없다. (Parameter-free)
    \item \textbf{Effect:} 연산량을 줄이고, 이동 불변성(Invariance)을 얻는다.
\end{enumerate}
\end{summarybox}

\end{document}