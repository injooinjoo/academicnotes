\documentclass[a4paper, 11pt]{article}

% --- 패키지 설정 ---
\usepackage{kotex} % 한글 지원
\usepackage{geometry} % 여백 설정
\geometry{left=25mm, right=25mm, top=25mm, bottom=25mm}
\usepackage{amsmath, amssymb, amsfonts} % 수식 패키지
\usepackage{graphicx}
\usepackage{adjustbox}  % 표/박스 크기 조절 % 이미지 삽입
\usepackage{hyperref} % 하이퍼링크
\usepackage{xcolor} % 색상 지원
\usepackage{listings} % 코드 블록
\usepackage[most]{tcolorbox}
\tcbuselibrary{breakable} % 박스 디자인
\usepackage{enumitem} % 리스트 스타일
\usepackage{booktabs} % 표 디자인
\usepackage{array} % 표 정렬

% --- 색상 정의 ---
\definecolor{conceptblue}{RGB}{60, 100, 160}
\definecolor{analogygreen}{RGB}{80, 160, 100}
\definecolor{alertred}{RGB}{200, 60, 60}
\definecolor{exampleorange}{RGB}{230, 120, 30}
\definecolor{codegray}{rgb}{0.5,0.5,0.5}
\definecolor{backcolour}{rgb}{0.96,0.96,0.96}

% --- 코드 스타일 설정 ---
\lstdefinestyle{mystyle}{
    backgroundcolor=\color{backcolour},   
    commentstyle=\color{analogygreen},
    keywordstyle=\color{conceptblue},
    numberstyle=\tiny\color{codegray},
    stringstyle=\color{exampleorange},
    basicstyle=\ttfamily\footnotesize,
    breakatwhitespace=false,         
    breaklines=true,                 
    captionpos=b,                    
    keepspaces=true,                 
    numbers=left,                    
    numbersep=5pt,                  
    showspaces=false,                
    showstringspaces=false,
    showtabs=false,                  
    tabsize=4,
    frame=single
}
\lstset{style=mystyle}

% --- 박스 스타일 정의 ---
\newtcolorbox{summarybox}[1]{
    colback=conceptblue!5!white,
    colframe=conceptblue!80!black,
    fonttitle=\bfseries,
    title=📌 #1
}

\newtcolorbox{analogybox}[1]{
    colback=analogygreen!5!white,
    colframe=analogygreen!80!black,
    fonttitle=\bfseries,
    title=💡 #1 (직관적 비유)
}

\newtcolorbox{warningbox}[1]{
    colback=alertred!5!white,
    colframe=alertred!80!black,
    fonttitle=\bfseries,
    title=⚠️ #1 (오해 방지 가이드)
}

\newtcolorbox{examplebox}[1]{
    colback=exampleorange!5!white,
    colframe=exampleorange!80!black,
    fonttitle=\bfseries,
    title=🧮 #1 (실전 시나리오 \& 계산)
}

% --- 문서 정보 ---
\title{\textbf{[CS230] Shallow Neural Networks: \\ Forward \& Backward Propagation}}
\author{Lecturer: Gemini (Integrated Editor)}
\date{}

\begin{document}

\maketitle

% --- 1. 전체 목차 (TOC) ---
\section*{📚 Course Table of Contents}
\begin{itemize}
    \item[Chapter 1.] Deep Learning Big Picture \textit{- Completed}
    \item[Chapter 2.] Logistic Regression as a Neural Network \textit{- Completed}
    \item[\textbf{Chapter 3.}] \textbf{Shallow Neural Networks (Current Unit)}
    \begin{itemize}
        \item 3.1 Concept of Hidden Layer \& Architecture \textit{- Completed}
        \item 3.2 Activation Functions \textit{- Completed}
        \item \textbf{3.3 Forward \& Backward Propagation (Math Heavy!)}
        \begin{itemize}
            \item Forward: The Flow of Prediction
            \item Backward: The "Blame Game" (Gradient Calculation)
            \item The 6 Magic Equations
            \item Implementation with Dimensions Check
        \end{itemize}
        \item 3.4 Random Initialization
    \end{itemize}
    \item[Chapter 4.] Deep Neural Networks \textit{- Upcoming}
\end{itemize}

\vspace{0.5cm}
\hrule
\vspace{0.5cm}

% --- 3. 이전 단원 연결 ---
\section*{🔗 지난 시간 복습 및 연결}
지난 시간에 우리는 신경망의 '구조(Layer)'와 '스위치(Activation Function)'를 장착했습니다. 이제 자동차는 완성되었습니다.
하지만 자동차를 앞으로 달리게만 해서는 운전을 배울 수 없습니다. 사고를 냈을 때(오차가 발생했을 때), 무엇이 잘못되었는지 파악하고 핸들을 돌리는 법(수정하는 법)을 배워야 합니다.
오늘 배울 \textbf{역전파(Backpropagation)}가 바로 그 과정입니다. 수학 기호가 쏟아지겠지만, 포기하지 마십시오. 이것이 딥러닝의 심장입니다.

% --- 4. 개요 ---
\section{Unit Overview}
\begin{summarybox}{핵심 목표}
이 단원은 딥러닝 학습 메커니즘인 '순전파'와 '역전파'를 수식과 코드로 구현합니다.
\begin{itemize}
    \item \textbf{순전파 (Forward):} 입력 $X$가 은닉층을 거쳐 출력 $\hat{y}$가 되는 과정을 행렬로 정의합니다.
    \item \textbf{역전파 (Backward):} 예측이 틀렸을 때, 비용 함수(Cost)의 기울기(Gradient)를 뒤쪽에서 앞쪽으로 계산합니다.
    \item \textbf{도구:} 미적분의 \textbf{연쇄 법칙(Chain Rule)}과 행렬의 \textbf{전치(Transpose)}가 왜 필요한지 이해합니다.
    \item \textbf{구현:} 차원(Dimension) 오류 없이 역전파 알고리즘을 코딩합니다.
\end{itemize}
\end{summarybox}

% --- 5. 용어 정리 ---
\section{Essential Terminology}
\begin{center}
\begin{tabular}{|c|l|l|}
\hline
\textbf{기호} & \textbf{의미} & \textbf{비유 (역할)} \\ \hline
$Z^{[l]}$ & 선형 출력 ($WX+b$) & 뉴런이 받아들인 원시 점수 \\ \hline
$A^{[l]}$ & 활성화 출력 ($g(Z)$) & 점수를 확률/신호로 변환한 최종 리포트 \\ \hline
$dZ^{[l]}$ & 오차항 ($\partial J / \partial Z$) & "얼마나 틀렸니?" (책임의 크기) \\ \hline
$dW^{[l]}$ & 가중치 기울기 & "가중치를 얼마나 수정할까?" \\ \hline
$*$ & \textbf{요소별 곱 (Element-wise)} & 행렬 곱이 아니라, 같은 위치끼리 곱함 \\ \hline
\end{tabular}
\end{center}

% --- 6. 핵심 개념 상세 설명 ---
\section{Core Concepts: 흐름의 이해}

\subsection{1. Forward Propagation (예측의 흐름)}


데이터가 강물처럼 입력층에서 출력층으로 흐릅니다.
$$ Input(X) \xrightarrow{W^{[1]}, b^{[1]}} Hidden(A^{[1]}) \xrightarrow{W^{[2]}, b^{[2]}} Output(A^{[2]}) $$
이 과정은 직관적입니다. "입력받아서, 계산하고, 넘겨준다." 끝입니다.

\vspace{0.5cm}\hrule\vspace{0.5cm}

\subsection{2. Backward Propagation (학습의 흐름)}


예측값($A^{[2]}$)과 실제값($Y$)의 차이, 즉 \textbf{비용(Cost)}을 줄이기 위해 미분을 사용합니다.
문제는 우리가 수정하고 싶은 파라미터($W^{[1]}$)가 출력층에서 멀리 떨어져 있다는 것입니다.

\begin{analogybox}{프로젝트 실패의 책임 소재 따지기 (The Blame Game)}
여러분이 팀장(출력층)이고 프로젝트가 실패(Error)했다고 가정해봅시다.
\begin{enumerate}
    \item \textbf{Step 1 (Output Layer):} 먼저 최종 결과물($A^{[2]}$)을 보고 "얼마나 부족했는지($dZ^{[2]}$)" 파악합니다.
    \item \textbf{Step 2 (Hidden Layer):} 팀장은 자신의 실패 원인을 분석하여, 중간 관리자(은닉층, $A^{[1]}$)에게 책임을 묻습니다. "네가 준 보고서가 잘못돼서 결과가 이렇게 됐잖아!" ($dZ^{[1]}$ 전파)
    \item \textbf{Step 3 (Parameters):} 중간 관리자는 다시 자신의 업무 도구($W^{[1]}$)를 탓하며 수정합니다. "이 가중치가 문제였군, 고치자." ($dW^{[1]}$ 계산)
\end{enumerate}
역전파는 이처럼 \textbf{오차(책임)를 뒤에서 앞으로 전달하며} 파라미터를 수정하는 과정입니다.
\end{analogybox}

\vspace{0.5cm}\hrule\vspace{0.5cm}

\section{Deep Dive: The 6 Magic Equations}

이 6개의 수식은 딥러닝 엔지니어의 구구단입니다. \textbf{연쇄 법칙(Chain Rule)}에 의해 유도됩니다.

\subsection{Phase 1: 출력층 (Layer 2) - 역전파의 시작}

\textbf{1. 오차 계산 ($dZ^{[2]}$)}
가장 직관적인 수식입니다. 예측값과 정답의 차이입니다.
$$ dZ^{[2]} = A^{[2]} - Y $$
\textit{(참고: Cross-Entropy와 Sigmoid 미분이 만나면 이렇게 깔끔하게 정리됩니다.)}

\textbf{2. 가중치 기울기 ($dW^{[2]}$)}
오차($dZ^{[2]}$)에 입력값($A^{[1]}$)을 곱합니다.
$$ dW^{[2]} = \frac{1}{m} dZ^{[2]} A^{[1]T} $$
\begin{warningbox}{왜 전치($T$)를 하나요?}
행렬 곱셈의 차원을 맞추기 위해서입니다.
$dZ^{[2]}$는 $(1, m)$, $A^{[1]}$은 $(n^{[1]}, m)$입니다. 곱하려면 $A^{[1]}$을 뒤집어야 $(1, m) \times (m, n^{[1]}) = (1, n^{[1]})$이 되어 $W^{[2]}$와 크기가 같아집니다.
\end{warningbox}

\textbf{3. 편향 기울기 ($db^{[2]}$)}
오차들의 평균입니다.
$$ db^{[2]} = \frac{1}{m} \sum_{rows} dZ^{[2]} $$

\vspace{0.5cm}\hrule\vspace{0.5cm}

\subsection{Phase 2: 은닉층 (Layer 1) - 핵심 구간}

\textbf{4. 은닉층 오차 ($dZ^{[1]}$)}
여기가 가장 어렵습니다. 출력층의 오차를 가중치 비율만큼 가져오고(Linear), 활성화 함수의 미분값(Non-linear)을 곱합니다.
$$ dZ^{[1]} = \underbrace{W^{[2]T} dZ^{[2]}}_{\text{오차 전파}} \quad \underbrace{*}_{\text{요소별 곱}} \quad \underbrace{g'^{[1]}(Z^{[1]})}_{\text{활성화 미분}} $$

\begin{itemize}
    \item $W^{[2]T} dZ^{[2]}$: 출력층의 오차를 은닉층으로 역송신합니다.
    \item $*$: 행렬 곱이 아닙니다! \textbf{Element-wise product}입니다.
    \item $g'(Z^{[1]})$: 만약 Tanh를 썼다면 $(1 - A^{[1]2})$입니다.
\end{itemize}

\textbf{5, 6. 파라미터 기울기 ($dW^{[1]}, db^{[1]}$)}
Layer 2와 동일한 패턴입니다.
$$ dW^{[1]} = \frac{1}{m} dZ^{[1]} X^T $$
$$ db^{[1]} = \frac{1}{m} \sum dZ^{[1]} $$

% --- 7. 구현 코드 ---
\section{Implementation (Python with NumPy)}

수식을 코드로 옮길 때 가장 중요한 것은 \textbf{차원(Shape) 확인}입니다.

\begin{lstlisting}[language=Python, caption=Full Backpropagation Implementation, breaklines=true]
import numpy as np

def backward_propagation(parameters, cache, X, Y):
    """
    parameters: W1, b1, W2, b2
    cache: Z1, A1, Z2, A2 (Forward 단계에서 저장해둔 값)
    X, Y: 입력 데이터 및 정답
    """
    m = X.shape[1] # 데이터 개수
    
    # 1. 파라미터 및 캐시 로드
    W2 = parameters["W2"]
    A1 = cache["A1"]
    A2 = cache["A2"]
    
    # --- Layer 2 (Output) ---
    # 수식 1: dZ2 = A2 - Y
    dZ2 = A2 - Y
    
    # 수식 2: dW2 (행렬 곱 주의: dZ2 @ A1.T)
    dW2 = (1 / m) * np.dot(dZ2, A1.T)
    
    # 수식 3: db2 (행 방향 합계, keepdims 필수)
    db2 = (1 / m) * np.sum(dZ2, axis=1, keepdims=True)
    
    # --- Layer 1 (Hidden) ---
    # 수식 4: dZ1 계산 (가장 중요!)
    # g'(z) for Tanh = 1 - a^2
    # '*' 연산자는 요소별 곱(Element-wise)임에 유의
    dZ1 = np.dot(W2.T, dZ2) * (1 - np.power(A1, 2))
    
    # 수식 5: dW1
    dW1 = (1 / m) * np.dot(dZ1, X.T)
    
    # 수식 6: db1
    db1 = (1 / m) * np.sum(dZ1, axis=1, keepdims=True)
    
    grads = {"dW1": dW1, "db1": db1, "dW2": dW2, "db2": db2}
    return grads
\end{lstlisting}

% --- 8. 예시 시나리오 ---
\section{Numerical Example: 계산 흐름 추적}

\begin{examplebox}{간단한 오차 역전파 예시}
\textbf{상황:} 정답 $y=1$인데, 모델이 예측 $a^{[2]}=0.8$을 내놓았습니다.
\begin{enumerate}
    \item \textbf{오차 발생 ($dZ^{[2]}$):} $0.8 - 1.0 = -0.2$. (0.2만큼 부족함)
    \item \textbf{은닉층 전달:} $W^{[2]}$가 $0.5$라고 가정합시다. 은닉층으로 오차를 보냅니다.
    $$ \text{전달된 오차} \approx 0.5 \times (-0.2) = -0.1 $$
    \item \textbf{활성화 미분 반영:} 만약 은닉층 활성화 미분값이 $0.5$라면?
    $$ dZ^{[1]} = -0.1 \times 0.5 = -0.05 $$
    \item \textbf{결론:} 은닉층의 오차는 -0.05입니다. 이 값을 줄이는 방향으로 $W^{[1]}$을 업데이트합니다.
\end{enumerate}
\end{examplebox}

% --- 9. FAQ ---
\section{FAQ \& Pitfalls}

\textbf{Q1. $dZ^{[1]}$ 구할 때 왜 행렬 곱(dot)이 아니라 요소별 곱(*) 인가요?} \\
\textbf{A.} 연쇄 법칙 $\frac{\partial A}{\partial Z}$ 부분 때문입니다. 활성화 함수의 미분은 각 뉴런마다 개별적으로 적용됩니다. 행렬 전체를 섞는(Linear mixing) 과정이 아니므로 같은 위치의 원소끼리만 곱해야 합니다.

\textbf{Q2. 전치($T$)는 언제 하나요? 외워야 하나요?} \\
\textbf{A.} 외우지 마세요. \textbf{차원(Dimensions)을 그려보면 됩니다.}
$dW$는 $W$와 모양이 같아야 합니다. $(n, m)$과 $(1, m)$을 곱해서 $(n, 1)$을 만들려면 뒤의 것을 뒤집어야 한다는 것이 자연스럽게 보입니다.

% --- 10. 다음 단원 연결 ---
\section*{🔗 다음 단계 (Next Step)}
고생하셨습니다. 여러분은 방금 딥러닝에서 가장 험난한 고개인 '역전파'를 넘었습니다. 이제 모델을 학습시킬 준비가 거의 다 되었습니다.

그런데, 학습을 시작할 때 \textbf{가중치($W$)를 처음에 어떻게 설정하느냐}가 학습의 성패를 좌우한다는 사실을 아십니까?
다음 시간에는 [Practice] 세션으로, \textbf{랜덤 초기화(Random Initialization)}의 중요성을 다루고, 왜 0으로 초기화하면 이 모든 역전파 알고리즘이 무용지물이 되는지 증명하겠습니다.

\vspace{0.5cm}

\begin{summarybox}{단원 요약 (Cheat Sheet)}
\begin{enumerate}
    \item \textbf{역전파:} 출력층의 오차($dZ$)를 구해 입력층 방향으로 전파하며 기울기($dW$)를 구한다.
    \item \textbf{Chain Rule:} 층을 건너갈 때마다 미분값을 곱한다 (미분의 연쇄).
    \item \textbf{Transpose:} 행렬 곱셈 시 차원을 맞추기 위해 전치 행렬($A^T$)을 사용한다.
    \item \textbf{Element-wise:} 활성화 함수의 미분값은 반드시 요소별 곱($*$)으로 계산한다.
\end{enumerate}
\end{summarybox}

\end{document}