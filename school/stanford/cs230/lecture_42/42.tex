\documentclass[a4paper, 11pt]{article}

% --- 패키지 설정 ---
\usepackage{kotex} % 한글 지원
\usepackage{geometry} % 여백 설정
\geometry{left=25mm, right=25mm, top=25mm, bottom=25mm}
\usepackage{amsmath, amssymb, amsfonts} % 수식 패키지
\usepackage{graphicx}
\usepackage{adjustbox}  % 표/박스 크기 조절 % 이미지 삽입
\usepackage{hyperref} % 하이퍼링크
\usepackage{xcolor} % 색상 지원
\usepackage{listings} % 코드 블록
\usepackage[most]{tcolorbox}
\tcbuselibrary{breakable} % 박스 디자인
\usepackage{enumitem} % 리스트 스타일
\usepackage{booktabs} % 표 디자인
\usepackage{array} % 표 정렬

% --- 색상 정의 ---
\definecolor{conceptblue}{RGB}{60, 100, 160}
\definecolor{analogygreen}{RGB}{80, 160, 100}
\definecolor{alertred}{RGB}{200, 60, 60}
\definecolor{exampleorange}{RGB}{230, 120, 30}
\definecolor{codegray}{rgb}{0.5,0.5,0.5}
\definecolor{backcolour}{rgb}{0.96,0.96,0.96}

% --- 코드 스타일 설정 ---
\lstdefinestyle{mystyle}{
    backgroundcolor=\color{backcolour},   
    commentstyle=\color{analogygreen},
    keywordstyle=\color{conceptblue},
    numberstyle=\tiny\color{codegray},
    stringstyle=\color{exampleorange},
    basicstyle=\ttfamily\footnotesize,
    breakatwhitespace=false,         
    breaklines=true,                 
    captionpos=b,                    
    keepspaces=true,                 
    numbers=left,                    
    numbersep=5pt,                  
    showspaces=false,                
    showstringspaces=false,
    showtabs=false,                  
    tabsize=4,
    frame=single
}
\lstset{style=mystyle}

% --- 박스 스타일 정의 ---
\newtcolorbox{summarybox}[1]{
    colback=conceptblue!5!white,
    colframe=conceptblue!80!black,
    fonttitle=\bfseries,
    title=📌 #1
}

\newtcolorbox{analogybox}[1]{
    colback=analogygreen!5!white,
    colframe=analogygreen!80!black,
    fonttitle=\bfseries,
    title=💡 #1 (직관적 비유)
}

\newtcolorbox{warningbox}[1]{
    colback=alertred!5!white,
    colframe=alertred!80!black,
    fonttitle=\bfseries,
    title=⚠️ #1 (오해 방지 가이드)
}

\newtcolorbox{formulabox}[1]{
    colback=exampleorange!5!white,
    colframe=exampleorange!80!black,
    fonttitle=\bfseries,
    title=🧮 #1 (핵심 수식)
}

% --- 문서 정보 ---
\title{\textbf{[CS230] Sequence Models: \\ LSTM (Long Short-Term Memory)}}
\author{Lecturer: Gemini (Integrated Editor)}
\date{}

\begin{document}

\maketitle

% --- 1. 전체 목차 (TOC) ---
\section*{📚 Course Table of Contents}
\begin{itemize}
    \item[Chapter 1-9.] Deep Learning Fundamentals \& CNNs \textit{- Completed}
    \item[\textbf{Chapter 10.}] \textbf{Sequence Models (Current Part)}
    \begin{itemize}
        \item 10.1-10.3 RNN Basics \& GRU \textit{- Completed}
        \item \textbf{10.4 LSTM (Long Short-Term Memory)}
        \begin{itemize}
            \item Difference: Cell State vs Hidden State
            \item The Three Gates (Forget, Update, Output)
            \item Mathematical Equations
            \item GRU vs LSTM Comparison
        \end{itemize}
        \item 10.5 NLP \& Word Embeddings \textit{- Upcoming}
    \end{itemize}
\end{itemize}

\vspace{0.5cm}
\hrule
\vspace{0.5cm}

% --- 3. 이전 단원 연결 ---
\section*{🔗 지난 시간 복습 및 연결}
지난 시간에 배운 GRU는 \textbf{'기울기 소실'}을 해결하기 위해 게이트를 도입한 훌륭한 모델이었습니다.
하지만 GRU는 2014년에 나온 모델이고, 그보다 훨씬 전인 1997년에 제안되어 지금까지 \textbf{시퀀스 모델의 제왕(Standard)}으로 군림하고 있는 모델이 있습니다.
바로 \textbf{LSTM}입니다. GRU가 2개의 게이트로 효율성을 추구했다면, LSTM은 3개의 게이트로 기억을 더욱 정교하게 제어합니다. 앤드류 응 교수님은 말합니다.
\textbf{"무엇을 쓸지 모르겠다면, 일단 LSTM부터 시작하십시오."}

% --- 4. 개요 ---
\section{Unit Overview}
\begin{summarybox}{핵심 목표}
이 단원은 시퀀스 모델의 업계 표준인 LSTM의 구조와 동작 원리를 파헤칩니다.
\begin{itemize}
    \item \textbf{구조:} \textbf{Cell State($c$)}와 \textbf{Hidden State($a$)}가 분리된 이중 트랙 구조를 이해합니다.
    \item \textbf{게이트:} Forget, Update, Output이라는 3개의 게이트 역할을 파악합니다.
    \item \textbf{수식:} 과거를 잊고($\Gamma_f$) 새로운 기억을 더하는($\Gamma_u$) 독립적 제어 공식을 익힙니다.
    \item \textbf{비교:} GRU와 LSTM의 장단점을 비교하고 선택 기준을 세웁니다.
\end{itemize}
\end{summarybox}

% --- 5. 용어 정리 ---
\section{Essential Terminology}
\begin{center}
\begin{tabular}{|c|l|l|}
\hline
\textbf{용어} & \textbf{기호} & \textbf{역할} \\ \hline
\textbf{Cell State} & $c^{\langle t \rangle}$ & \textbf{장기 기억 고속도로.} 내부에서만 순환하며 정보를 보존함. \\ \hline
\textbf{Hidden State} & $a^{\langle t \rangle}$ & \textbf{단기 상태 및 출력.} Cell State를 가공하여 외부로 내보냄. \\ \hline
\textbf{Forget Gate} & $\Gamma_f$ & 과거의 기억을 삭제하는 비율 (0~1). \\ \hline
\textbf{Update Gate} & $\Gamma_u$ & 새로운 기억을 저장하는 비율 (0~1). \\ \hline
\textbf{Output Gate} & $\Gamma_o$ & 현재 상태를 다음 층으로 내보내는 비율 (0~1). \\ \hline
\end{tabular}
\end{center}

% --- 6. 핵심 개념 상세 설명 ---
\section{Core Concepts: 기억의 정교한 제어}

\subsection{1. The Key Difference: $c$ and $a$}
GRU는 기억($c$)과 출력($a$)이 통합되어 있었습니다. 반면 LSTM은 이를 엄격히 분리합니다.



\begin{itemize}
    \item \textbf{Cell State ($c^{\langle t \rangle}$):} 기억의 핵심입니다. 기울기가 소실되지 않도록 보호받는 경로입니다.
    \item \textbf{Hidden State ($a^{\langle t \rangle}$):} Cell State에 $\tanh$를 씌우고 Output Gate를 통과시켜 만든, "지금 당장 필요한 정보"입니다.
\end{itemize}

\subsection{2. The Three Gates (3개의 문지기)}
LSTM은 기억을 관리하기 위해 3단계 검문을 수행합니다.

\begin{analogybox}{LSTM의 기억 관리 시스템}
\begin{itemize}
    \item \textbf{Forget Gate ($\Gamma_f$): [쓰레기통]} "이전 기억 중 쓸모없는 건 버려라." (예: 문단이 바뀌었으니 이전 주제 삭제)
    \item \textbf{Update Gate ($\Gamma_u$): [기록장]} "새로운 정보 중 중요한 것만 적어라." (예: 새로운 주어 등록)
    \item \textbf{Output Gate ($\Gamma_o$): [스피커]} "지금 당장 필요한 정보만 말해라." (예: 다음 단어 예측에 필요한 정보만 발설)
\end{itemize}
\end{analogybox}

\vspace{0.5cm}\hrule\vspace{0.5cm}

\section{Deep Dive: LSTM Equations}

이 수식들이 LSTM의 작동 원리를 보여주는 지도입니다.


\subsection{1. Gate Calculation}
이전 상태($a^{\langle t-1 \rangle}$)와 현재 입력($x^{\langle t \rangle}$)을 보고 게이트를 얼마나 열지 결정합니다.
$$ \Gamma_f = \sigma(W_f[a^{\langle t-1 \rangle}, x^{\langle t \rangle}] + b_f) $$
$$ \Gamma_u = \sigma(W_u[a^{\langle t-1 \rangle}, x^{\langle t \rangle}] + b_u) $$
$$ \Gamma_o = \sigma(W_o[a^{\langle t-1 \rangle}, x^{\langle t \rangle}] + b_o) $$

\subsection{2. Memory Update (핵심)}
\begin{formulabox}{Cell State Update}
$$ c^{\langle t \rangle} = \underbrace{\Gamma_f \cdot c^{\langle t-1 \rangle}}_{\text{과거 기억 삭제}} + \underbrace{\Gamma_u \cdot \tilde{c}^{\langle t \rangle}}_{\text{새 기억 추가}} $$
\begin{itemize}
    \item \textbf{GRU와의 차이:} GRU는 $\Gamma_u$ 하나로 과거와 현재의 비율을 시소처럼 조절했습니다 ($1-\Gamma_u$).
    \item \textbf{LSTM:} $\Gamma_f$와 $\Gamma_u$가 \textbf{독립적}입니다. 과거를 기억하면서($\Gamma_f=1$) 동시에 새로운 중요한 정보도 추가($\Gamma_u=1$)할 수 있어 표현력이 더 풍부합니다.
\end{itemize}
\end{formulabox}

\subsection{3. Output Generation}
$$ a^{\langle t \rangle} = \Gamma_o \cdot \tanh(c^{\langle t \rangle}) $$

\vspace{0.5cm}\hrule\vspace{0.5cm}

\section{Comparison: GRU vs LSTM}

\begin{center}
\begin{tabular}{|c|c|c|}
\hline
\textbf{특징} & \textbf{GRU} & \textbf{LSTM} \\ \hline
\textbf{게이트 수} & 2개 (Update, Reset) & 3개 (Forget, Update, Output) \\ \hline
\textbf{복잡도} & 단순함 (Simpler) & 복잡함 (Powerful) \\ \hline
\textbf{데이터 양} & 적을 때 유리 & 많을 때 유리 (대용량 학습) \\ \hline
\textbf{위상} & 경량화 모델 & \textbf{업계 표준 (Default Choice)} \\ \hline
\end{tabular}
\end{center}

% --- 7. 구현 코드 ---
\section{Implementation: LSTM Cell}

\begin{lstlisting}[language=Python, caption=LSTM Cell Forward Implementation, breaklines=true]
import numpy as np

def sigmoid(x):
    return 1 / (1 + np.exp(-x))

def lstm_cell_forward(xt, a_prev, c_prev, parameters):
    """
    xt: 현재 입력
    a_prev: 이전 은닉 상태 (단기)
    c_prev: 이전 셀 상태 (장기)
    """
    # 파라미터 로드 (Wf, Wu, Wc, Wo 등)
    concat = np.concatenate((a_prev, xt), axis=0)
    
    # 1. Gates 계산
    ft = sigmoid(np.dot(parameters["Wf"], concat) + parameters["bf"]) # Forget
    it = sigmoid(np.dot(parameters["Wu"], concat) + parameters["bu"]) # Update(Input)
    ot = sigmoid(np.dot(parameters["Wo"], concat) + parameters["bo"]) # Output
    
    # 2. 새로운 기억 후보 (Candidate)
    c_tilde = np.tanh(np.dot(parameters["Wc"], concat) + parameters["bc"])
    
    # 3. Cell State 업데이트 (핵심!)
    # 과거를 잊을 건 잊고(ft), 새것을 더함(it)
    c_next = ft * c_prev + it * c_tilde
    
    # 4. Hidden State 업데이트 (출력)
    a_next = ot * np.tanh(c_next)
    
    return a_next, c_next
\end{lstlisting}

% --- 8. FAQ ---
\section{FAQ \& Pitfalls}

\textbf{Q. Peephole Connection이란 뭔가요?} \\
\textbf{A.} 기본 LSTM에서 게이트($\Gamma$)는 $a^{\langle t-1 \rangle}$와 $x^{\langle t \rangle}$만 봅니다. Peephole 변형은 게이트가 \textbf{Cell State($c^{\langle t-1 \rangle}$)도 훔쳐보고(Peep)} 결정을 내리게 합니다. "기억통이 얼마나 찼는지 보고 문을 열지 말지 정한다"는 개념입니다.

\textbf{Q. 왜 LSTM이 기울기 소실에 강한가요?} \\
\textbf{A.} $c^{\langle t \rangle} = c^{\langle t-1 \rangle} + \dots$ 형태의 \textbf{덧셈 연산} 때문입니다. 역전파 시 덧셈은 기울기를 그대로($\times 1$) 전달하는 특성이 있어, 깊은 시간까지 오차가 잘 전달됩니다. (ResNet과 유사 원리)

% --- 9. 다음 단원 연결 ---
\section*{🔗 다음 단계 (Next Step)}
이제 우리는 시퀀스 데이터를 처리하는 가장 강력한 엔진(LSTM)을 만들었습니다. 이제 이 엔진에 넣을 \textbf{연료(데이터)}를 가공할 차례입니다.

컴퓨터는 '사과', '바나나'를 이해하지 못합니다. 숫자로 바꿔야 하는데, 단순히 $1, 2$로 바꾸면 단어 간 의미 관계가 사라집니다.
다음 시간에는 단어의 의미를 벡터 공간에 매핑하여 \textbf{"왕 - 남자 + 여자 = 여왕"}이라는 연산을 가능하게 하는 마법, \textbf{[Word Embedding (Word2Vec, GloVe)]}에 대해 다루겠습니다.

\vspace{0.5cm}

\begin{summarybox}{단원 요약 (Cheat Sheet)}
\begin{enumerate}
    \item \textbf{LSTM:} 3개의 게이트로 기억을 관리하는 시퀀스 모델의 표준.
    \item \textbf{Structure:} 장기 기억($c$)과 단기 상태($a$)를 분리하여 운영한다.
    \item \textbf{Gates:} Forget(삭제), Update(추가), Output(출력).
    \item \textbf{Strategy:} 일단 LSTM을 기본으로 쓰고, 경량화가 필요하면 GRU를 고려하라.
\end{enumerate}
\end{summarybox}

\end{document}