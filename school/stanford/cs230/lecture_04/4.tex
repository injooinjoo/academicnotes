\documentclass[a4paper, 11pt]{article}

% --- 패키지 설정 ---
\usepackage{kotex} % 한글 지원
\usepackage{geometry} % 여백 설정
\geometry{left=25mm, right=25mm, top=25mm, bottom=25mm}
\usepackage{amsmath, amssymb, amsfonts} % 수식 패키지
\usepackage{graphicx}
\usepackage{adjustbox}  % 표/박스 크기 조절 % 이미지 삽입
\usepackage{hyperref} % 하이퍼링크
\usepackage{xcolor} % 색상 지원
\usepackage{listings} % 코드 블록
\usepackage[most]{tcolorbox}
\tcbuselibrary{breakable} % 박스 디자인
\usepackage{enumitem} % 리스트 스타일
\usepackage{booktabs} % 표 디자인

% --- 색상 정의 ---
\definecolor{conceptblue}{RGB}{60, 100, 160}
\definecolor{analogygreen}{RGB}{80, 160, 100}
\definecolor{alertred}{RGB}{200, 60, 60}
\definecolor{exampleorange}{RGB}{230, 120, 30}
\definecolor{codegray}{rgb}{0.5,0.5,0.5}
\definecolor{backcolour}{rgb}{0.96,0.96,0.96}

% --- 코드 스타일 설정 ---
\lstdefinestyle{mystyle}{
    backgroundcolor=\color{backcolour},   
    commentstyle=\color{analogygreen},
    keywordstyle=\color{conceptblue},
    numberstyle=\tiny\color{codegray},
    stringstyle=\color{exampleorange},
    basicstyle=\ttfamily\footnotesize,
    breakatwhitespace=false,         
    breaklines=true,                 
    captionpos=b,                    
    keepspaces=true,                 
    numbers=left,                    
    numbersep=5pt,                  
    showspaces=false,                
    showstringspaces=false,
    showtabs=false,                  
    tabsize=4,
    frame=single
}
\lstset{style=mystyle}

% --- 박스 스타일 정의 ---
\newtcolorbox{summarybox}[1]{
    colback=conceptblue!5!white,
    colframe=conceptblue!80!black,
    fonttitle=\bfseries,
    title=📌 #1
}

\newtcolorbox{analogybox}[1]{
    colback=analogygreen!5!white,
    colframe=analogygreen!80!black,
    fonttitle=\bfseries,
    title=💡 #1 (직관적 비유)
}

\newtcolorbox{warningbox}[1]{
    colback=alertred!5!white,
    colframe=alertred!80!black,
    fonttitle=\bfseries,
    title=⚠️ #1 (오해 방지 가이드)
}

\newtcolorbox{examplebox}[1]{
    colback=exampleorange!5!white,
    colframe=exampleorange!80!black,
    fonttitle=\bfseries,
    title=🧮 #1 (실전 시나리오 \& 벤치마크)
}

% --- 문서 정보 ---
\title{\textbf{[CS230] Foundations of Neural Networks: \\ Python \& Vectorization}}
\author{Lecturer: Gemini (Integrated Editor)}
\date{}

\begin{document}

\maketitle

% --- 1. 전체 목차 (TOC) ---
\section*{📚 Course Table of Contents}
\begin{itemize}
    \item[Chapter 1.] Deep Learning Big Picture \textit{- Completed}
    \item[Chapter 2.] Logistic Regression as a Neural Network
    \begin{itemize}
        \item 2.1 Architecture \& Forward Propagation \textit{- Completed}
        \item 2.2 Cost Function \& Gradient Descent \textit{- Completed}
        \item \textbf{2.3 Python \& Vectorization (Current Unit)}
        \begin{itemize}
            \item Overview: Why Vectorization?
            \item SIMD: The Hardware Magic
            \item Broadcasting Rules
            \item Implementation \& Benchmark
        \end{itemize}
    \end{itemize}
    \item[Chapter 3.] Shallow Neural Networks \textit{- Upcoming}
\end{itemize}

\vspace{0.5cm}
\hrule
\vspace{0.5cm}

% --- 3. 이전 단원 연결 ---
\section*{🔗 지난 시간 복습 및 연결}
우리는 지난 시간까지 딥러닝의 \textbf{'이론적 토대(비용 함수, 경사 하강법)'}를 완성했습니다. 이론적으로는 완벽합니다. 하지만 이 수식을 컴퓨터에게 그대로 주면 학습하는 데 수십 년이 걸릴지도 모릅니다. 이제 이 이론에 \textbf{'제트 엔진'}을 달아줄 시간입니다. 초심자와 전문가를 가르는 가장 결정적인 기술, \textbf{벡터화(Vectorization)}를 배워봅시다.

% --- 4. 개요 ---
\section{Unit Overview}
\begin{summarybox}{핵심 목표}
이 단원은 딥러닝 코드를 수백 배 빠르게 만드는 \textbf{'최적화 기술'}을 다룹니다.
\begin{itemize}
    \item \textbf{개념:} `for-loop`를 죄악시하고, 행렬 단위 연산(Vectorization)을 해야 하는 이유를 배웁니다.
    \item \textbf{원리:} CPU/GPU의 SIMD(병렬 처리) 아키텍처가 어떻게 연산을 가속하는지 이해합니다.
    \item \textbf{기술:} NumPy의 핵심 기능인 \textbf{브로드캐스팅(Broadcasting)}의 규칙과 위험성을 파악합니다.
    \item \textbf{검증:} 실제 코드로 100만 개의 데이터를 연산해보며 속도 차이를 눈으로 확인합니다.
\end{itemize}
\end{summarybox}

% --- 5. 용어 정리 ---
\section{Essential Terminology}
\begin{center}
\begin{tabular}{|c|l|l|}
\hline
\textbf{용어} & \textbf{설명} & \textbf{한 줄 핵심 요약} \\ \hline
\textbf{Vectorization} & 벡터화 & 반복문 없이 데이터를 통째로(행렬로) 연산하는 기법 \\ \hline
\textbf{SIMD} & Single Instruction, Multiple Data & 명령어 하나로 여러 데이터를 동시에 처리하는 CPU 기술 \\ \hline
\textbf{Broadcasting} & 브로드캐스팅 & 모양이 다른 배열끼리 연산할 때 자동으로 크기를 맞춰주는 기능 \\ \hline
\textbf{NumPy} & 넘파이 & 파이썬의 느린 속도를 C언어 레벨 최적화로 극복한 수치 연산 라이브러리 \\ \hline
\textbf{Rank-1 Array} & 랭크-1 배열 & `(5,)` 처럼 행도 열도 아닌 애매한 배열 (버그의 주범) \\ \hline
\end{tabular}
\end{center}

% --- 6. 핵심 개념 상세 설명 ---
\section{Core Concepts: 속도의 비밀}

\subsection{1. Vectorization (벡터화란 무엇인가?)}
\textbf{한 줄 요약:} 하나씩 처리하지 말고, 트럭에 실어서 한 번에 옮기십시오.

\begin{analogybox}{이사짐 옮기기 비유}
100만 개의 벽돌(데이터)을 옮겨야 합니다.
\begin{itemize}
    \item \textbf{For-loop (Non-vectorized):} 인부가 벽돌을 \textbf{손에 하나씩 들고} 100만 번 왕복합니다. (파이썬이 데이터를 하나씩 꺼내서 처리함)
    \item \textbf{Vectorization:} 100만 개의 벽돌을 \textbf{거대한 덤프트럭(행렬)}에 싣고 단 한 번에 이동합니다. (NumPy가 데이터를 통째로 메모리에 올려 처리함)
\end{itemize}
\end{analogybox}

\textbf{기술적 정의:}
$z = w^T x + b$를 계산할 때, $w_1x_1, w_2x_2 \dots$를 순회하지 않고, $w$와 $x$ 전체 벡터를 한 번에 내적(Dot Product)하는 것입니다.

\vspace{0.5cm}\hrule\vspace{0.5cm}

\subsection{2. Under the Hood: SIMD (하드웨어의 마법)}
왜 NumPy(`np.dot`)가 `for`문보다 빠를까요? 단순히 C언어로 짜여서가 아닙니다. 컴퓨터 구조적인 이유가 있습니다.



\begin{itemize}
    \item \textbf{SISD (Single Instruction, Single Data):} 일반적인 `for`문입니다. CPU가 "가져와", "곱해", "저장해"를 데이터 하나마다 반복합니다.
    \item \textbf{SIMD (Single Instruction, Multiple Data):} 최신 CPU는 "이 8개의 데이터를 동시에 곱해!"라는 명령을 내릴 수 있습니다. 벡터화는 이 병렬 처리 기능을 활용합니다.
\end{itemize}

\vspace{0.5cm}\hrule\vspace{0.5cm}

\subsection{3. Broadcasting (브로드캐스팅)}
\textbf{한 줄 요약:} 작은 행렬을 큰 행렬 크기에 맞게 자동으로 '늘려서(Stretch)' 연산합니다.



\begin{itemize}
    \item \textbf{상황:} $(4 \times 1)$ 행렬에 숫자 $100$(스칼라)을 더하고 싶습니다.
    \item \textbf{원칙:} 수학적으로는 불가능하지만, Python은 $100$을 자동으로 $(4 \times 1)$ 크기로 복사하여 더해줍니다.
    \item \textbf{예시:} 
    $$ \begin{bmatrix} 1 \\ 2 \\ 3 \\ 4 \end{bmatrix} + 100 \rightarrow \begin{bmatrix} 1 \\ 2 \\ 3 \\ 4 \end{bmatrix} + \begin{bmatrix} 100 \\ 100 \\ 100 \\ 100 \end{bmatrix} = \begin{bmatrix} 101 \\ 102 \\ 103 \\ 104 \end{bmatrix} $$
\end{itemize}

% --- 7. 실전 벤치마크 ---
\section{Implementation \& Benchmark (성능 검증)}

"백문이 불여일견"입니다. 100만 개의 데이터를 곱하는 시간을 직접 측정해 봅시다.

\begin{examplebox}{For-loop vs Vectorization 속도 대결}
\begin{lstlisting}[language=Python, breaklines=true]
import numpy as np
import time

# 데이터 준비: 100만 개의 난수 생성
a = np.random.rand(1000000)
b = np.random.rand(1000000)

# --- 1. For-loop (느린 방법) ---
c = 0
tic = time.time()
for i in range(1000000):
    c += a[i] * b[i]
toc = time.time()

print(f"For-loop: {c:.4f}")
print(f"Time: {1000 * (toc - tic):.2f} ms") # 약 400~500ms 소요

# --- 2. Vectorization (빠른 방법) ---
tic = time.time()
c_vec = np.dot(a, b) # SIMD 병렬 처리
toc = time.time()

print(f"Vectorized: {c_vec:.4f}")
print(f"Time: {1000 * (toc - tic):.2f} ms") # 약 1~2ms 소요
\end{lstlisting}
\textbf{결과 분석:} 벡터화 코드가 약 \textbf{300~500배} 더 빠릅니다. 딥러닝 모델 학습 시간이 1달 걸릴 것을 2시간으로 줄여주는 마법입니다.
\end{examplebox}

% --- 8. 주의사항 및 FAQ ---
\section{Pitfalls \& FAQ}

\begin{warningbox}{Rank-1 Array의 함정}
NumPy에서 `a = np.random.randn(5)`를 하면 모양(Shape)이 `(5,)`가 됩니다.
이것은 행 벡터도, 열 벡터도 아닌 애매한 상태라 전치(Transpose)가 안 됩니다.
\begin{itemize}
    \item \textbf{나쁜 예:} `a = np.random.randn(5)` $\rightarrow$ 버그 발생 위험 높음.
    \item \textbf{좋은 예:} `a = np.random.randn(5, 1)` (열 벡터) 또는 `(1, 5)` (행 벡터)로 명시하십시오.
    \item \textbf{습관:} 코드 중간에 `assert(a.shape == (5, 1))`을 넣어 차원을 확인하십시오.
\end{itemize}
\end{warningbox}

\subsection*{FAQ: 자주 묻는 질문}
\textbf{Q1. GPU는 언제 쓰나요?} \\
A. NumPy는 기본적으로 CPU를 사용합니다. 나중에 배울 TensorFlow나 PyTorch는 이 벡터화 연산을 GPU(그래픽 카드)에서 수행하여, CPU보다 훨씬 더 많은 병렬 처리(수천 개의 코어)를 가능하게 합니다. 원리는 똑같습니다.

\textbf{Q2. 모든 코드를 벡터화할 수 있나요?} \\
A. 대부분의 수학 연산은 가능합니다. 하지만 복잡한 조건문(`if-else`)이 데이터마다 다르게 적용되어야 한다면 벡터화가 어려울 수 있습니다. 그럼에도 99\%의 딥러닝 연산은 벡터화가 가능합니다.

% --- 9. 다음 단원 연결 ---
\section*{🔗 다음 단계 (Next Step)}
축하합니다! 여러분은 이제 \textbf{'고속 연산 엔진(Vectorization)'}을 장착했습니다. 

지금까지는 뉴런이 딱 하나(로지스틱 회귀)뿐이었습니다. 다음 장 \textbf{[Chapter 3. Shallow Neural Networks]}에서는 이 강력한 엔진을 활용해 뉴런을 수백 개로 늘려보겠습니다. 이제 진짜 '신경망'다운 신경망을 만들 차례입니다.

\vspace{0.5cm}

\begin{summarybox}{단원 요약 (Cheat Sheet)}
\begin{enumerate}
    \item \textbf{Vectorization:} `for`문은 죄악이다. `np.dot` 등을 써서 행렬 단위로 계산하라.
    \item \textbf{SIMD:} 벡터화는 CPU의 병렬 처리 명령어를 사용하여 속도를 수백 배 높인다.
    \item \textbf{Broadcasting:} 차원이 달라도 NumPy가 알아서 맞춰주지만, 버그를 조심해야 한다.
    \item \textbf{Shape Check:} `(n,)` 대신 `(n, 1)`을 사용하여 차원을 명시하는 습관을 들여라.
\end{enumerate}
\end{summarybox}

\end{document}