\documentclass[a4paper, 11pt]{article}

% --- 패키지 설정 ---
\usepackage{kotex} % 한글 지원
\usepackage{geometry} % 여백 설정
\geometry{left=25mm, right=25mm, top=25mm, bottom=25mm}
\usepackage{amsmath, amssymb, amsfonts} % 수식 패키지
\usepackage{graphicx}
\usepackage{adjustbox}  % 표/박스 크기 조절 % 이미지 삽입
\usepackage{hyperref} % 하이퍼링크
\usepackage{xcolor} % 색상 지원
\usepackage{listings} % 코드 블록
\usepackage[most]{tcolorbox}
\tcbuselibrary{breakable} % 박스 디자인
\usepackage{enumitem} % 리스트 스타일
\usepackage{booktabs} % 표 디자인
\usepackage{array} % 표 정렬

% --- 색상 정의 ---
\definecolor{conceptblue}{RGB}{60, 100, 160}
\definecolor{analogygreen}{RGB}{80, 160, 100}
\definecolor{alertred}{RGB}{200, 60, 60}
\definecolor{exampleorange}{RGB}{230, 120, 30}
\definecolor{codegray}{rgb}{0.5,0.5,0.5}
\definecolor{backcolour}{rgb}{0.96,0.96,0.96}

% --- 코드 스타일 설정 ---
\lstdefinestyle{mystyle}{
    backgroundcolor=\color{backcolour},   
    commentstyle=\color{analogygreen},
    keywordstyle=\color{conceptblue},
    numberstyle=\tiny\color{codegray},
    stringstyle=\color{exampleorange},
    basicstyle=\ttfamily\footnotesize,
    breakatwhitespace=false,         
    breaklines=true,                 
    captionpos=b,                    
    keepspaces=true,                 
    numbers=left,                    
    numbersep=5pt,                  
    showspaces=false,                
    showstringspaces=false,
    showtabs=false,                  
    tabsize=4,
    frame=single
}
\lstset{style=mystyle}

% --- 박스 스타일 정의 ---
\newtcolorbox{summarybox}[1]{
    colback=conceptblue!5!white,
    colframe=conceptblue!80!black,
    fonttitle=\bfseries,
    title=📌 #1
}

\newtcolorbox{analogybox}[1]{
    colback=analogygreen!5!white,
    colframe=analogygreen!80!black,
    fonttitle=\bfseries,
    title=💡 #1 (직관적 비유)
}

\newtcolorbox{warningbox}[1]{
    colback=alertred!5!white,
    colframe=alertred!80!black,
    fonttitle=\bfseries,
    title=⚠️ #1 (오해 방지 가이드)
}

\newtcolorbox{formulabox}[1]{
    colback=exampleorange!5!white,
    colframe=exampleorange!80!black,
    fonttitle=\bfseries,
    title=🧮 #1 (수학적 원리)
}

% --- 문서 정보 ---
\title{\textbf{[CS230] Sequence Models: \\ Attention Mechanism}}
\author{Lecturer: Gemini (Integrated Editor)}
\date{}

\begin{document}

\maketitle

% --- 1. 전체 목차 (TOC) ---
\section*{📚 Course Table of Contents}
\begin{itemize}
    \item[Chapter 1-9.] Deep Learning Fundamentals \& CNNs \textit{- Completed}
    \item[\textbf{Chapter 10.}] \textbf{Sequence Models (Current Part)}
    \begin{itemize}
        \item 10.1-10.10 RNN, LSTM, Word2Vec, BLEU \textit{- Completed}
        \item \textbf{10.11 Attention Mechanism}
        \begin{itemize}
            \item The Bottleneck Problem in Seq2Seq
            \item Dynamic Context Vector ($c^{\langle t \rangle}$)
            \item Alignment Scores \& Attention Weights ($\alpha$)
            \item Implementation: Bahdanau Attention
        \end{itemize}
    \end{itemize}
    \item[Chapter 11.] Transformer Network \textit{- Next Part}
\end{itemize}

\vspace{0.5cm}
\hrule
\vspace{0.5cm}

% --- 3. 이전 단원 연결 ---
\section*{🔗 지난 시간 복습 및 연결}
지난 시간에 배운 Seq2Seq 모델은 혁신적이었지만, 문장이 길어지면 성능이 급격히 떨어지는 \textbf{'병목(Bottleneck)'} 현상을 겪었습니다. 인코더가 100단어짜리 긴 문장을 고정된 크기의 벡터 하나에 억지로 압축해야 했기 때문입니다.

생각해 봅시다. 여러분이 긴 문장을 번역할 때, 문장 전체를 한 번 읽고 외운 다음 눈을 감고 번역합니까? 아닙니다. \textbf{"지금 번역하려는 단어와 관련된 원문 부분을 그때그때 다시 쳐다보면서(Attend)"} 번역합니다. 이 과정을 구현한 것이 바로 \textbf{어텐션(Attention)}입니다.

% --- 4. 개요 ---
\section{Unit Overview}
\begin{summarybox}{핵심 목표}
이 단원은 긴 시퀀스 정보 손실을 해결하고 성능을 극대화하는 어텐션 메커니즘을 다룹니다.
\begin{itemize}
    \item \textbf{동적 문맥:} 매 타임 스텝마다 다르게 계산되는 \textbf{문맥 벡터 $c^{\langle t \rangle}$}의 개념을 이해합니다.
    \item \textbf{가중치:} 인코더의 특정 단어에 얼마나 집중할지 나타내는 \textbf{어텐션 가중치($\alpha$)}를 정의합니다.
    \item \textbf{구조:} 인코더의 모든 은닉 상태를 가중 합(Weighted Sum)하여 정보를 취합하는 원리를 배웁니다.
    \item \textbf{학습:} 가중치를 결정하는 정렬 모델(Alignment Model)도 역전파로 함께 학습됨을 파악합니다.
\end{itemize}
\end{summarybox}

% --- 5. 용어 정리 ---
\section{Essential Terminology}
\begin{center}
\begin{tabular}{|c|l|l|}
\hline
\textbf{용어} & \textbf{의미} & \textbf{비유} \\ \hline
\textbf{Attention Weight} & $\alpha^{\langle t, t' \rangle}$: $t$번째 출력 시 $t'$번째 입력의 중요도. & 돋보기로 비추는 빛의 세기. \\ \hline
\textbf{Context Vector} & $c^{\langle t \rangle}$: 인코더 상태들의 가중 평균. & 번역 중인 단어의 연관 참고 자료. \\ \hline
\textbf{Alignment Score} & 단어 간의 유사도/관련성 점수. & 두 단어가 얼마나 '찰떡궁합'인가. \\ \hline
\end{tabular}
\end{center}

% --- 6. 핵심 개념 상세 설명 ---
\section{Core Concepts: 필요한 곳만 쳐다보기}

\subsection{1. The Intuition (직관)}


기존 Seq2Seq는 인코더의 \textbf{마지막 정보} 하나만 디코더에게 던져줍니다. 반면, 어텐션 모델은 디코더가 단어를 생성할 때마다 인코더의 \textbf{모든 은닉 상태}($a^{\langle 1 \rangle}, a^{\langle 2 \rangle}, \dots$)를 다 참고합니다. 단, 중요한 것은 진하게(높은 가중치), 안 중요한 것은 흐리게 봅니다.

\begin{analogybox}{번역가의 힐끔거리기}
\begin{itemize}
    \item \textbf{기존:} 문장을 한 번 읽고 책을 덮은 뒤 암기해서 번역함. (병목 발생)
    \item \textbf{어텐션:} 번역문을 쓰면서 원문에서 지금 단어와 관련된 부분(예: 주어, 목적어)을 계속 \textbf{힐끔힐끔} 다시 보며 번역함.
\end{itemize}
\end{analogybox}

\subsection{2. The Architecture: Context Vector $c^{\langle t \rangle}$}
문맥 벡터 $c^{\langle t \rangle}$는 인코더의 모든 은닉 상태 $a^{\langle t' \rangle}$의 \textbf{가중 평균}입니다.

\begin{formulabox}{Context Vector 공식}
$$ c^{\langle t \rangle} = \sum_{t'} \alpha^{\langle t, t' \rangle} a^{\langle t' \rangle} $$
\begin{itemize}
    \item $\alpha^{\langle t, t' \rangle}$: 디코더 $t$ 시점에 인코더 $t'$ 시점을 보는 비중.
    \item $\sum_{t'} \alpha^{\langle t, t' \rangle} = 1$ (확률값).
\end{itemize}
\end{formulabox}

\vspace{0.5cm}\hrule\vspace{0.5cm}

\section{Deep Dive: Computing Attention (수학적 원리)}

그렇다면 가장 중요한 $\alpha$ (가중치)는 누가 정할까요? 사람이 정하는 게 아니라, 이조차 \textbf{작은 신경망(Alignment Model)}이 학습합니다.

\begin{enumerate}
    \item \textbf{Alignment Score ($e^{\langle t, t' \rangle}$):} 디코더의 이전 상태 $s^{\langle t-1 \rangle}$와 인코더 상태 $a^{\langle t' \rangle}$가 얼마나 어울리는지 점수를 매깁니다.
    $$ e^{\langle t, t' \rangle} = \text{dense\_layer}([s^{\langle t-1 \rangle}, a^{\langle t' \rangle}]) $$
    
    \item \textbf{Softmax (Weights):} 점수를 확률로 변환합니다.
    $$ \alpha^{\langle t, t' \rangle} = \frac{\exp(e^{\langle t, t' \rangle})}{\sum_{k=1}^{T_x} \exp(e^{\langle t, k \rangle})} $$
\end{enumerate}



% --- 7. 구현 코드 ---
\section{Implementation: Bahdanau Attention}

TensorFlow/Keras의 서브클래싱 방식으로 구현한 예시입니다.

\begin{lstlisting}[language=Python, caption=Bahdanau Attention Layer, breaklines=true]
import tensorflow as tf
from tensorflow.keras.layers import Layer, Dense

class BahdanauAttention(Layer):
    def __init__(self, units):
        super().__init__()
        self.W1 = Dense(units) # Decoder state weight
        self.W2 = Dense(units) # Encoder state weight
        self.V = Dense(1)      # Score weight

    def call(self, query, values):
        # query: 디코더 은닉 상태 (Batch, hidden)
        # values: 인코더 은닉 상태들 (Batch, Tx, hidden)
        
        query_with_time = tf.expand_dims(query, 1)
        
        # 1. 점수 계산 (Additive Attention)
        score = self.V(tf.nn.tanh(self.W1(query_with_time) + self.W2(values)))
        
        # 2. 가중치 계산 (Softmax)
        attention_weights = tf.nn.softmax(score, axis=1)
        
        # 3. 문맥 벡터 계산 (Weighted Sum)
        context_vector = attention_weights * values
        context_vector = tf.reduce_sum(context_vector, axis=1)
        
        return context_vector, attention_weights
\end{lstlisting}

% --- 8. FAQ ---
\section{FAQ \& Pitfalls}

\textbf{Q. 어텐션을 쓰면 모델이 무거워지지 않나요?} \\
\textbf{A.} 네, 인코더의 모든 타임 스텝을 매번 연산해야 하므로 $T_x \times T_y$의 연산량이 추가됩니다. 하지만 성능 향상 폭이 워낙 커서 감수할 만한 비용입니다.

\textbf{Q. 어텐션은 번역 말고 어디에 쓰이나요?} \\
\textbf{A.} \textbf{이미지 캡셔닝}에서도 쓰입니다. 단어를 생성할 때마다 이미지의 특정 구역을 쳐다보는 식이죠.

% --- 9. 다음 단원 연결 ---
\section*{🔗 다음 단계 (Next Step)}
어텐션은 RNN의 한계를 부수고 성능을 극대화했습니다. 하지만 사람들은 깨달았습니다.
\textbf{"어텐션이 이렇게 강력하다면, 굳이 느린 순차 방식(RNN)을 써야 할까? 그냥 어텐션만 쓰면 안 되나?"}

여기서 딥러닝 역사를 바꾼 논문, \textbf{"Attention Is All You Need"}가 등장합니다.
다음 시간, RNN과 CNN을 모두 버리고 오직 어텐션만으로 무장한 현대 AI의 심장, \textbf{[Transformer Network]}에 대해 알아보겠습니다.

\vspace{0.5cm}

\begin{summarybox}{단원 요약 (Cheat Sheet)}
\begin{enumerate}
    \item \textbf{Problem:} Seq2Seq 인코더의 고정 크기 벡터 병목 현상.
    \item \textbf{Solution:} 디코더가 매 순간 인코더의 모든 부분을 연관성에 따라 참고한다.
    \item \textbf{Formula:} $c^{\langle t \rangle} = \sum \alpha^{\langle t, t' \rangle} a^{\langle t' \rangle}$ (가중 평균).
    \item \textbf{Benefit:} 긴 문장 번역 성능 비약적 상승 + 시각적 해석 가능.
\end{enumerate}
\end{summarybox}

\end{document}