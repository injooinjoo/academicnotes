\documentclass[a4paper, 11pt]{article}

% --- 패키지 설정 ---
\usepackage{kotex} % 한글 지원
\usepackage{geometry} % 여백 설정
\geometry{left=25mm, right=25mm, top=25mm, bottom=25mm}
\usepackage{amsmath, amssymb, amsfonts} % 수식 패키지
\usepackage{graphicx}
\usepackage{adjustbox}  % 표/박스 크기 조절 % 이미지 삽입
\usepackage{hyperref} % 하이퍼링크
\usepackage{xcolor} % 색상 지원
\usepackage{listings} % 코드 블록
\usepackage[most]{tcolorbox}
\tcbuselibrary{breakable} % 박스 디자인
\usepackage{enumitem} % 리스트 스타일
\usepackage{booktabs} % 표 디자인
\usepackage{array} % 표 정렬

% --- 색상 정의 ---
\definecolor{conceptblue}{RGB}{60, 100, 160}
\definecolor{analogygreen}{RGB}{80, 160, 100}
\definecolor{alertred}{RGB}{200, 60, 60}
\definecolor{exampleorange}{RGB}{230, 120, 30}
\definecolor{codegray}{rgb}{0.5,0.5,0.5}
\definecolor{backcolour}{rgb}{0.96,0.96,0.96}

% --- 코드 스타일 설정 ---
\lstdefinestyle{mystyle}{
    backgroundcolor=\color{backcolour},   
    commentstyle=\color{analogygreen},
    keywordstyle=\color{conceptblue},
    numberstyle=\tiny\color{codegray},
    stringstyle=\color{exampleorange},
    basicstyle=\ttfamily\footnotesize,
    breakatwhitespace=false,         
    breaklines=true,                 
    captionpos=b,                    
    keepspaces=true,                 
    numbers=left,                    
    numbersep=5pt,                  
    showspaces=false,                
    showstringspaces=false,
    showtabs=false,                  
    tabsize=4,
    frame=single
}
\lstset{style=mystyle}

% --- 박스 스타일 정의 ---
\newtcolorbox{summarybox}[1]{
    colback=conceptblue!5!white,
    colframe=conceptblue!80!black,
    fonttitle=\bfseries,
    title=📌 #1
}

\newtcolorbox{analogybox}[1]{
    colback=analogygreen!5!white,
    colframe=analogygreen!80!black,
    fonttitle=\bfseries,
    title=💡 #1 (직관적 비유)
}

\newtcolorbox{warningbox}[1]{
    colback=alertred!5!white,
    colframe=alertred!80!black,
    fonttitle=\bfseries,
    title=⚠️ #1 (오해 방지 가이드)
}

\newtcolorbox{formulabox}[1]{
    colback=exampleorange!5!white,
    colframe=exampleorange!80!black,
    fonttitle=\bfseries,
    title=🧮 #1 (수학적 정의)
}

% --- 문서 정보 ---
\title{\textbf{[CS230] Convolutional Neural Networks: \\ YOLO Algorithm (Object Detection)}}
\author{Lecturer: Gemini (Integrated Editor)}
\date{}

\begin{document}

\maketitle

% --- 1. 전체 목차 (TOC) ---
\section*{📚 Course Table of Contents}
\begin{itemize}
    \item[Chapter 1-8.] Deep Learning Strategy \& Architecture \textit{- Completed}
    \item[\textbf{Chapter 9.}] \textbf{Convolutional Neural Networks (Current Part)}
    \begin{itemize}
        \item 9.1-9.6 CNN Basics \& Sliding Windows \textit{- Completed}
        \item \textbf{9.7 YOLO Algorithm (You Only Look Once)}
        \begin{itemize}
            \item The Grid System \& Bounding Box Regression
            \item IoU (Intersection over Union) Metric
            \item Non-max Suppression (Removing Duplicates)
            \item Anchor Boxes (Handling Overlap)
        \end{itemize}
    \end{itemize}
    \item[Chapter 10.] Sequence Models (RNN) \textit{- Next Part}
\end{itemize}

\vspace{0.5cm}
\hrule
\vspace{0.5cm}

% --- 3. 이전 단원 연결 ---
\section*{🔗 지난 시간 복습 및 연결}
지난 시간에 배운 슬라이딩 윈도우는 합성곱 구현으로 속도는 빨라졌지만, 여전히 \textbf{"박스 위치가 부정확하다"}는 한계가 있었습니다. 윈도우가 고정된 간격으로만 움직이기 때문입니다.
"객체의 중심을 찾고, 그 중심을 기준으로 박스 크기를 예측하면 어떨까?"
이 아이디어로 탄생한 것이 \textbf{YOLO}입니다. 이름처럼 이미지를 단 한 번만 보고(Look Once), 모든 객체의 위치와 종류를 동시에 찾아내는 혁신적인 알고리즘입니다.

% --- 4. 개요 ---
\section{Unit Overview}
\begin{summarybox}{핵심 목표}
이 단원은 실시간 객체 탐지의 표준인 YOLO 알고리즘의 원리를 완벽히 이해합니다.
\begin{itemize}
    \item \textbf{그리드:} 이미지를 격자로 나누고, 객체 중심점이 속한 셀이 책임을 지는 구조를 배웁니다.
    \item \textbf{IoU:} 두 박스가 얼마나 겹치는지를 측정하는 평가 지표를 수학적으로 정의합니다.
    \item \textbf{NMS:} 중복된 박스를 제거하는 비최대 억제(Non-max Suppression) 알고리즘을 익힙니다.
    \item \textbf{앵커:} 겹친 물체를 분리하는 앵커 박스(Anchor Box) 개념을 파악합니다.
\end{itemize}
\end{summarybox}

% --- 5. 용어 정리 ---
\section{Essential Terminology}
\begin{center}
\begin{tabular}{|c|l|l|}
\hline
\textbf{용어} & \textbf{약어} & \textbf{설명} \\ \hline
\textbf{Grid Cell} & - & 이미지를 $S \times S$로 나눈 작은 구역. \\ \hline
\textbf{IoU} & Intersection over Union & 교집합 영역 / 합집합 영역. (일치도) \\ \hline
\textbf{NMS} & Non-max Suppression & 가장 확실한 박스 하나만 남기고 나머지는 지움. \\ \hline
\textbf{Anchor Box} & - & 미리 정의된 박스 모양. (길쭉한 사람, 넓은 차 등) \\ \hline
\end{tabular}
\end{center}

% --- 6. 핵심 개념 상세 설명 ---
\section{Core Concepts: 단 한 번의 추론}

\subsection{1. Bounding Box Predictions (그리드 시스템)}


YOLO는 이미지를 $S \times S$ 그리드(보통 $19 \times 19$)로 나눕니다.
각 셀은 다음 벡터 $y$를 예측합니다.
$$ y = [p_c, b_x, b_y, b_h, b_w, c_1, c_2, \dots]^T $$
\begin{itemize}
    \item $p_c$: 객체가 있을 확률 (Confidence).
    \item $b_x, b_y$: 박스 중심 좌표 (셀 내 상대 위치, 0~1).
    \item $b_h, b_w$: 박스 높이/너비 (전체 이미지 대비 비율).
    \item $c_i$: 클래스 확률 (차, 사람 등).
\end{itemize}

\subsection{2. IoU (Intersection over Union)}


모델이 예측한 박스가 정답과 얼마나 비슷한지 평가하는 척도입니다.
\begin{formulabox}{IoU 수식}
$$ \text{IoU} = \frac{\text{교집합 영역 (Intersection)}}{\text{합집합 영역 (Union)}} $$
\begin{itemize}
    \item 보통 $\text{IoU} \ge 0.5$ 이면 "올바른 탐지"로 간주합니다.
    \item 1이면 완벽하게 일치, 0이면 전혀 겹치지 않음.
\end{itemize}
\end{formulabox}

\subsection{3. Anchor Boxes (겹친 물체 해결)}
한 셀의 중심에 사람과 차가 겹쳐 있다면?
기존 벡터로는 하나만 예측 가능합니다. 이를 위해 미리 정의된 모양(앵커)을 사용합니다.
$$ y = [\text{Anchor 1}, \text{Anchor 2}] $$
\begin{itemize}
    \item \textbf{Anchor 1 (세로로 긴):} 사람 담당.
    \item \textbf{Anchor 2 (가로로 넓은):} 자동차 담당.
\end{itemize}

\vspace{0.5cm}\hrule\vspace{0.5cm}

\section{Deep Dive: Non-max Suppression (NMS)}

YOLO는 하나의 객체에 대해 여러 셀이 "내가 찾았다!"며 박스를 칠 수 있습니다. 중복을 제거해야 합니다.

\begin{enumerate}
    \item \textbf{Filter:} $p_c < 0.6$ 인 박스(확신 없는 것)는 모두 버립니다.
    \item \textbf{Select:} 남은 박스 중 $p_c$가 가장 높은 것을 선택합니다 (Best Box).
    \item \textbf{Suppress:} 선택된 박스와 $\text{IoU} \ge 0.5$ 인(많이 겹친) 다른 박스들은 "같은 물체를 중복 탐지한 것"으로 보고 지웁니다.
    \item \textbf{Repeat:} 박스가 다 정리될 때까지 반복합니다.
\end{enumerate}

% --- 7. 구현 코드 ---
\section{Implementation: IoU Calculation}

IoU 계산은 객체 탐지 성능 평가와 NMS 구현의 핵심입니다.

\begin{lstlisting}[language=Python, caption=IoU Calculation Function, breaklines=true]
def calculate_iou(box1, box2):
    """
    box: (x1, y1, x2, y2)좌표 (좌상단, 우하단)
    """
    (b1_x1, b1_y1, b1_x2, b1_y2) = box1
    (b2_x1, b2_y1, b2_x2, b2_y2) = box2
    
    # 1. 교집합(Intersection) 좌표 계산
    xi1 = max(b1_x1, b2_x1)
    yi1 = max(b1_y1, b2_y1)
    xi2 = min(b1_x2, b2_x2)
    yi2 = min(b1_y2, b2_y2)
    
    # 넓이 (음수면 0)
    inter_area = max(0, xi2 - xi1) * max(0, yi2 - yi1)
    
    # 2. 합집합(Union) 넓이 계산
    b1_area = (b1_x2 - b1_x1) * (b1_y2 - b1_y1)
    b2_area = (b2_x2 - b2_x1) * (b2_y2 - b2_y1)
    union_area = b1_area + b2_area - inter_area
    
    # 3. IoU
    return inter_area / (union_area + 1e-6)

# --- 테스트 ---
if __name__ == "__main__":
    box_a = (1, 1, 3, 3) # 면적 4
    box_b = (2, 2, 4, 4) # 면적 4, 교집합 1
    # 합집합 = 4 + 4 - 1 = 7
    # IoU = 1/7 = 0.1428...
    
    print(f"IoU: {calculate_iou(box_a, box_b):.4f}")
\end{lstlisting}

% --- 8. FAQ ---
\section{FAQ \& Pitfalls}

\begin{warningbox}{좌표계 주의}
YOLO의 출력 $b_x, b_y$는 \textbf{그리드 셀 내부에서의 상대 위치(0~1)}입니다. 실제 이미지 위에 박스를 그리려면, 셀의 위치(인덱스)를 더하고 이미지 크기를 곱해주는 변환 과정이 필요합니다.
\end{warningbox}

\textbf{Q. 앵커 박스 크기는 어떻게 정하나요?} \\
\textbf{A.} 보통 훈련 데이터에 있는 객체들의 실제 박스 크기를 모아서 \textbf{K-Means 클러스터링}을 돌립니다. 가장 빈번하게 등장하는 대표적인 모양 5~9개를 선정하여 사용합니다 (YOLO v2부터 적용).

% --- 9. 다음 단원 연결 ---
\section*{🔗 다음 단계 (Next Step)}
이것으로 컴퓨터 비전(CNN) 파트를 마칩니다. 이제 여러분은 정지된 이미지에서 사물을 분류하고 위치까지 찾아내는 기술을 습득했습니다.

하지만 세상은 멈춰 있지 않습니다. 유튜브 영상, 음성 인식, 주가 예측, 번역 등은 \textbf{시간의 흐름(Sequence)}이 있는 데이터입니다.
다음 시간부터는 \textbf{[Part 5. Sequence Models]}의 세계로 떠납니다. 시계열 데이터를 처리하는 가장 기본적인 신경망, \textbf{RNN (Recurrent Neural Networks)}에 대해 알아보겠습니다.

\vspace{0.5cm}

\begin{summarybox}{단원 요약 (Cheat Sheet)}
\begin{enumerate}
    \item \textbf{YOLO:} 그리드 셀마다 바운딩 박스를 회귀(Regression)로 직접 예측한다.
    \item \textbf{IoU:} 교집합/합집합. 박스 위치 정확도의 척도이자 NMS의 기준.
    \item \textbf{NMS:} 중복된 박스를 제거하여 객체당 하나의 박스만 남긴다.
    \item \textbf{Anchor:} 다양한 비율의 객체를 잡기 위해 미리 정의된 박스 모양을 쓴다.
\end{enumerate}
\end{summarybox}

\end{document}