%%%%%%%%%%%%%%%%%%%%%%%%%%%%%%%%%%%%%%%%%%%%%%%%%%%%%%%%%%%%%%%%%%%%%%%%%%%%%%%
% FIN 574: 기업 수준 경제학 - 통합본
% UIUC Finance Department
% 자동 생성됨
%%%%%%%%%%%%%%%%%%%%%%%%%%%%%%%%%%%%%%%%%%%%%%%%%%%%%%%%%%%%%%%%%%%%%%%%%%%%%%%

\documentclass[11pt,a4paper]{book}

%========================================================================================
% 기본 패키지
%========================================================================================

\usepackage{kotex}
\usepackage[top=25mm, bottom=25mm, left=25mm, right=25mm]{geometry}
\usepackage{setspace}
\onehalfspacing
\setlength{\parskip}{0.5em}
\setlength{\parindent}{0pt}

\usepackage{amsmath, amssymb, amsthm}
\usepackage{graphicx}
\usepackage{booktabs}
\usepackage{tabularx}
\usepackage{array}
\usepackage{longtable}
\usepackage{adjustbox}
\renewcommand{\arraystretch}{1.1}

\usepackage{enumitem}
\setlist{nosep, leftmargin=*, itemsep=0.3em}

\usepackage{fancyhdr}
\pagestyle{fancy}
\fancyhf{}
\fancyhead[LE,RO]{\thepage}
\fancyhead[LO]{\leftmark}
\fancyhead[RE]{FIN 574}
\renewcommand{\headrulewidth}{0.5pt}
\setlength{\headheight}{15pt}

\usepackage[
    colorlinks=true,
    linkcolor=blue!80!black,
    urlcolor=blue!80!black,
    bookmarks=true,
    bookmarksnumbered=true
]{hyperref}

%========================================================================================
% 색상 정의
%========================================================================================

\usepackage[dvipsnames]{xcolor}

\definecolor{lightblue}{RGB}{220, 235, 255}
\definecolor{lightgreen}{RGB}{220, 255, 235}
\definecolor{lightyellow}{RGB}{255, 250, 220}
\definecolor{lightpurple}{RGB}{240, 230, 255}
\definecolor{lightgray}{gray}{0.95}
\definecolor{lightpink}{RGB}{255, 235, 245}
\definecolor{boxgray}{gray}{0.95}
\definecolor{boxblue}{rgb}{0.9, 0.95, 1.0}
\definecolor{boxred}{rgb}{1.0, 0.95, 0.95}

\definecolor{darkblue}{RGB}{50, 80, 150}
\definecolor{darkgreen}{RGB}{40, 120, 70}
\definecolor{darkorange}{RGB}{200, 100, 30}
\definecolor{darkpurple}{RGB}{100, 60, 150}

%========================================================================================
% 박스 환경 (tcolorbox)
%========================================================================================

\usepackage[most]{tcolorbox}
\tcbuselibrary{skins, breakable}

% 요약 박스
\newtcolorbox{summarybox}[1][]{
    enhanced,
    colback=lightblue,
    colframe=darkblue,
    fonttitle=\bfseries,
    title=핵심 요약,
    arc=2mm,
    boxrule=0.7pt,
    left=6pt, right=6pt, top=6pt, bottom=6pt,
    breakable,
    #1
}

% 주의사항 박스
\newtcolorbox{warningbox}[1][]{
    enhanced,
    colback=lightyellow,
    colframe=darkorange,
    fonttitle=\bfseries,
    title=주의사항,
    arc=2mm,
    boxrule=0.7pt,
    left=6pt, right=6pt, top=6pt, bottom=6pt,
    breakable,
    #1
}

% 일반 박스 (mybox)
\newtcolorbox{mybox}[1][]{
    enhanced,
    colback=lightgray,
    colframe=black!60,
    fonttitle=\bfseries,
    title=#1,
    arc=2mm,
    boxrule=0.7pt,
    left=6pt, right=6pt, top=6pt, bottom=6pt,
    breakable,
}

% 핵심 포인트 박스
\newtcolorbox{keypointbox}[1][]{
    enhanced,
    colback=lightgreen,
    colframe=darkgreen,
    fonttitle=\bfseries,
    title=#1,
    arc=2mm,
    boxrule=0.7pt,
    left=6pt, right=6pt, top=6pt, bottom=6pt,
    breakable,
}

%========================================================================================
% 정리 환경
%========================================================================================

\theoremstyle{definition}
\newtheorem{theorem}{정리}[chapter]
\newtheorem{lemma}[theorem]{보조정리}
\newtheorem{proposition}[theorem]{명제}
\newtheorem{corollary}[theorem]{따름정리}
\newtheorem{definition}{정의}[chapter]
\newtheorem{example}{예제}[chapter]

%========================================================================================
% 문서 정보
%========================================================================================

\title{\textbf{FIN 574: 기업 수준 경제학}}
\author{통합 강의 노트}
\date{}

\begin{document}

\maketitle
\tableofcontents
\newpage

%=======================================================================
% Chapter 1: 소비자 및 생산자 행동
%=======================================================================
\chapter{소비자 및 생산자 행동}
\label{ch:lecture1}

\begin{summarybox}
이 문서는 기업 수준 경제학의 첫 번째 모듈인 '소비자 및 생산자 행동'의 핵심 내용을 정리합니다. 경제학의 근본적인 문제인 \textbf{희소성(scarcity)}에서 출발하여, 자원이 배분되는 방식으로서의 \textbf{시장(market)}과 \textbf{가격(price)}의 역할을 탐구합니다. \textbf{수요(demand)}와 \textbf{공급(supply)} 곡선의 도출 원리와 시장 \textbf{균형(equilibrium)}이 결정되는 과정을 학습합니다. 특히, '곡선 상의 이동'과 '곡선 자체의 이동'을 구분하는 것은 시장 변화를 분석하는 데 가장 중요한 핵심입니다.
\end{summarybox}

\section{학습 로드맵 (Learning Roadmap)}

이 문서는 다음과 같은 논리적 흐름으로 구성되어 있습니다. 이 순서대로 개념을 정복해 나가는 것이 좋습니다.

\begin{enumerate}
    \item \textbf{경제학의 근본 문제 (희소성)}: 왜 경제학이 필요한지 이해합니다. "세상에 공짜는 없다"는 말의 진짜 의미를 배웁니다.
    \item \textbf{핵심 의사결정 도구 (기회비용)}: 모든 선택에 따르는 보이지 않는 비용, 즉 기회비용(opportunity cost)을 계산하는 법을 배웁니다.
    \item \textbf{구매자의 행동 (수요 곡선)}: 가격이 변할 때 소비자가 어떻게 반응하는지를 나타내는 수요 곡선(demand curve)을 이해합니다.
    \item \textbf{판매자의 행동 (공급 곡선)}: 가격이 변할 때 기업(생산자)이 어떻게 반응하는지를 나타내는 공급 곡선(supply curve)을 이해합니다.
    \item \textbf{시장의 작동 (균형)}: 구매자와 판매자가 만나 가격이 결정되고 거래가 이루어지는 '시장 균형(market equilibrium)' 지점을 찾습니다.
    \item \textbf{시장 변화 분석 (이동과 변동)}: 이 모듈의 최종 목표입니다. 외부 충격(뉴스, 정책, 재해 등)이 발생했을 때, 시장 가격과 거래량이 어떻게 변하는지 예측합니다.
\end{enumerate}

\section{핵심 용어 정리 (Key Terminology)}

경제학은 용어의 정확한 정의에서 시작합니다. 특히 '수요'와 '수요량'의 구분은 매우 중요합니다.

\begin{table}[h!]
\centering
\caption{모듈 1 핵심 용어}
\label{tab:key_terms}
\small
\begin{tabularx}{\textwidth}{>{\raggedright\arraybackslash}p{2.5cm} >{\raggedright\arraybackslash}X >{\raggedright\arraybackslash}p{3cm}}
\toprule
\textbf{용어} & \textbf{쉬운 설명} & \textbf{원어 (Eng)} \\
\midrule
\textbf{희소성} & 단순히 '드문 것'이 아니라, '사람들의 욕구에 비해 원하는 만큼 다 가질 수 없는 상태'를 의미합니다. & Scarcity \\
\textbf{기회비용} & 어떤 것을 선택했을 때 포기해야 하는 '차선책(next best alternative)의 가치'. (암묵적 비용 포함) & Opportunity Cost \\
\textbf{수요 (Demand)} & \textbf{모든} 가능한 가격 수준에서 소비자가 구매하려는 의사와 능력을 나타내는 \textbf{관계 전체 (곡선 자체)}. & Demand \\
\textbf{수요량} & \textbf{특정한} 가격 수준에서 소비자가 구매하려는 \textbf{구체적인 양 (곡선 위의 한 점)}. & Quantity Demanded \\
\textbf{공급 (Supply)} & \textbf{모든} 가능한 가격 수준에서 생산자가 판매하려는 의사와 능력을 나타내는 \textbf{관계 전체 (곡선 자체)}. & Supply \\
\textbf{공급량} & \textbf{특정한} 가격 수준에서 생산자가 판매하려는 \textbf{구체적인 양 (곡선 위의 한 점)}. & Quantity Supplied \\
\textbf{시장 균형} & 수요량과 공급량이 정확히 일치하여, 가격이나 거래량이 변할 유인이 없는 안정된 상태. (수요=공급) & Market Equilibrium \\
\textbf{초과 공급 (잉여)} & 가격이 너무 높아 공급량이 수요량을 초과하는 상태. (재고가 쌓임) & Surplus \\
\textbf{초과 수요 (부족)} & 가격이 너무 낮아 수요량이 공급량을 초과하는 상태. (줄을 서야 함) & Shortage \\
\bottomrule
\end{tabularx}
\end{table}

\section{핵심 개념 1: 희소성과 기회비용}

\subsection{희소성 (Scarcity)과 배분 메커니즘}

경제학은 \textbf{희소성(scarcity)}의 문제를 해결하는 학문입니다.
희소성이란 "모두가 원하는 만큼 가질 수 없는 상태(not enough to go around)"를 의미합니다.

\begin{itemize}
    \item 희소성은 단순히 '희귀함(rare)'과는 다릅니다. 예를 들어, 레오나르도 다빈치의 '모나리자'는 희귀하지만, 공기 중의 '깨끗한 공기'는 흔해 보이지만 희소합니다.
    \item 희소성 때문에 우리는 필연적으로 '선택'을 해야 하며, "누가 얼마나 가질 것인가"를 결정할 규칙, 즉 \textbf{배분 메커니즘(Allocation Mechanism)}이 필요합니다.
    \item 과거에는 왕이나 정부가 모든 것을 결정했습니다(명령 경제, Command Economy).
    \item 현대 서구 경제에서는 주로 \textbf{시장(Market)}이 이 역할을 합니다.
\end{itemize}

\begin{mybox}[사례: 1990년의 RAM 가격과 시장의 힘]
    1990년, Compaq 컴퓨터의 4MB RAM 가격은 약 \$2,600였습니다. 이는 당시 매우 비싼 가격이었습니다. 하지만 오늘날 512MB (4MB의 128배) RAM은 \$4.59에 불과합니다.

    이러한 극적인 가격 하락은 정부의 계획 때문이 아니었습니다.
    높은 가격($P \uparrow$)을 본 실리콘 밸리의 많은 기업가들이 "저 비용으로 나도 만들 수 있겠다"고 판단하여 시장에 진입(Entry)했습니다.
    그 결과, 공급자(producers)가 많아지면서 경쟁이 발생했고, \textbf{공급이 증가($S \uparrow$)}하여 \textbf{가격이 하락($P \downarrow$)}했습니다.
    이것이 바로 시장이 작동하는 방식입니다.
\end{mybox}

\subsection{가격의 역할: 배급 장치 (Rationing Device)}

시장은 \textbf{가격(Price)}이라는 도구를 사용하여 희소한 자원을 배분합니다.
가격은 단순히 '비용'이 아니라, 누가 그 재화를 가질 자격이 있는지를 결정하는 \textbf{배급 장치(Rationing Device)}입니다.

\begin{itemize}
    \item \textbf{가격 상승 ($\uparrow$)}: 재화가 더 희소해지면(e.g., 공급 감소), 가격이 오릅니다.
        \begin{itemize}
            \item 일부 소비자는 "이 가격을 낼 가치는 없다"며 구매를 포기합니다 (수요량 감소).
            \item 일부 생산자는 "수익성이 좋다"며 시장에 진입합니다 (공급량 증가).
        \end{itemize}
    \item \textbf{가격 하락 ($\downarrow$)}: 재화가 덜 희소해지면(e.g., 공급 증가), 가격이 내립니다.
        \begin{itemize}
            \item 더 많은 소비자가 "이 가격이면 살 만하다"며 구매를 시작합니다 (수요량 증가).
            \item 일부 생산자는 "수익성이 나쁘다"며 시장을 떠납니다 (공급량 감소).
        \end{itemize}
\end{itemize}

\begin{warningbox}
\textbf{만약 가격이 없다면? $\rightarrow$ 대기 (Queuing)}

미국 듀크(Duke) 대학의 농구팀 티켓은 학생들에게 \$0 (무료)입니다.
하지만 경기장은 10,000석 미만으로 매우 작아(희소성), 모든 학생이 들어갈 수 없습니다.

가격을 \$0로 정한다고 해서 희소성이 사라지진 않습니다.
단지 배분 메커니즘이 '가격'에서 '대기(Queuing)'로 바뀌었을 뿐입니다.
학생들은 티켓을 얻기 위해 몇 주 동안 텐트에서 생활('Krzyzewskiville')해야 합니다.
즉, '돈' 대신 '시간'이라는 엄청난 비용을 지불하는 것입니다.
\end{warningbox}

\subsection{기회비용 (Opportunity Cost)}

모든 선택에는 비용이 따릅니다. 경제학에서 말하는 '진짜' 비용은 \textbf{기회비용(Opportunity Cost)}입니다.

\textbf{정의}: "어떤 것을 선택함으로써 포기해야 하는 것 중 가장 가치가 큰 대안 (The value of the next best foregone alternative)"

\begin{itemize}
    \item 노벨상 수상자인 밀턴 프리드먼(Milton Friedman)은 "세상에 공짜 점심은 없다(There's no such thing as a free lunch)"고 말했습니다.
    \item 누군가 당신에게 공짜로 점심을 사주더라도, 당신은 그 시간에 친구를 만나거나 넷플릭스를 볼 수 있는 '시간'을 포기해야 합니다. 이 시간이 바로 기회비용입니다.
\end{itemize}

\begin{keypointbox}[핵심 예시: 명시적 비용 vs. 암묵적 비용(기회비용)]
    당신이 친구와 함께 차를 사려 한다고 가정합시다.

    \begin{itemize}
        \item \textbf{당신 (명시적 비용)}: 돈이 없어 은행에서 \$3,000를 대출받고 이자를 냅니다.
        \item \textbf{친구 (암묵적 비용)}: 모아둔 저축 \$3,000를 인출해서 차를 샀습니다. 친구는 "나는 이자를 내지 않으니 공짜다"라고 주장합니다.
    \end{itemize}

    \textbf{경제학적 분석}: 친구의 주장은 틀렸습니다.
    친구가 만약 그 돈 \$3,000를 은행에 계속 저축했다면 이자를 받을 수 있었을 것입니다.
    차를 사기 위해 돈을 인출하는 순간, 그 '받을 수 있었던 이자'를 포기한 것입니다.
    이것이 바로 눈에 보이지 않는 \textbf{암묵적 비용(Implicit Cost)}이며, 기회비용의 핵심입니다.

    \textbf{기업의 적용}: 회계 장부상 \$3백만의 '이익(Profit)'을 낸 CEO라도, 경제학적으로는 다음과 같이 생각해야 합니다. "내가 이 자산을 모두 팔고 트윙키(Twinkies)를 만들었다면 \$4백만을 벌 수 있지 않았을까?" 만약 그렇다면, 이 기업은 회계상 흑자임에도 불구하고 경제학적으로는 \$1백만의 손실을 본 것입니다.
\end{keypointbox}


\section{핵심 개념 2: 수요 곡선 (The Demand Curve)}

시장은 \textbf{구매자(Buyers)}와 \textbf{판매자(Sellers)}의 상호작용으로 이루어집니다.
구매자들의 행동을 요약한 것이 바로 \textbf{수요 곡선(Demand Curve)}입니다. (이 분석은 정부 개입이 없다고 가정합니다.)

\subsection{수요(Demand)의 정의}

\textbf{정의}: "다른 모든 조건이 일정할 때, 각각의 가격 수준에서 소비자들이 구매하고자 하는 재화의 양을 나타내는 점들의 궤적"

\begin{itemize}
    \item 수학적으로는 $Q = f(P)$ 로 표현할 수 있습니다. 즉, 구매하려는 수량(Quantity, Q)이 가격(Price, P)에 따라 결정됩니다.
    \item 수요 곡선은 \textbf{우하향(Downward sloping)}합니다.
    \item \textbf{이유 (수요의 법칙, Law of Demand)}:
        \begin{itemize}
            \item 가격이 하락하면($P \downarrow$), 더 적은 '녹색 종이(돈)'를 포기해도 되므로(기회비용 감소), 더 많이 구매합니다($Q \uparrow$).
            \item 가격이 상승하면($P \uparrow$), 더 많은 것을 포기해야 하므로(기회비용 증가), 더 적게 구매합니다($Q \downarrow$).
        \end{itemize}
\end{itemize}

\subsection{수요 곡선의 형태와 관행}

\begin{keypointbox}[주의: 축(Axis)이 반대로 되어 있습니다]
    수학적으로 $Q = f(P)$ 라면, 독립 변수인 가격(P)이 X축에, 종속 변수인 수량(Q)이 Y축에 있어야 합니다.

    하지만 100여 년 전 경제학자 알프레드 마샬(Alfred Marshall)이 처음 그래프를 그릴 때, 가격(P)을 Y축(세로축)에, 수량(Q)을 X축(가로축)에 두었습니다.
    이것이 관행으로 굳어져, 오늘날 모든 경제학 교과서가 이 방식을 따릅니다.

    엄밀히 말해 우리가 그리는 것은 $P = f^{-1}(Q)$ 인 \textbf{역수요함수(Inverse Demand Curve)}이지만, 관습적으로 그냥 "수요 곡선"이라고 부릅니다. 이 관행에 익숙해져야 합니다.
\end{keypointbox}


\section{핵심 개념 3: 공급 곡선 (The Supply Curve)}

판매자(Sellers, Firms)의 행동을 요약한 것이 \textbf{공급 곡선(Supply Curve)}입니다.

\subsection{공급(Supply)의 정의}

\textbf{정의}: "다른 모든 조건이 일정할 때, 각각의 가격 수준에서 생산자들이 판매하고자 하는 재화의 양을 나타내는 점들의 궤적"

\begin{itemize}
    \item 수학적으로는 $Q = h(P)$ 로 표현할 수 있습니다.
    \item 공급 곡선은 \textbf{우상향(Upward sloping)}합니다.
    \item \textbf{이유}:
        \begin{itemize}
            \item 생산량을 늘리기 위해서는($Q \uparrow$) 추가 노동자를 고용하거나 공장을 더 돌려야 하므로 \textbf{추가 생산 비용이 더 많이 듭니다}.
            \item 기업이 이처럼 더 비싼 비용을 감수하고 생산량을 늘리도록 유인하려면, 시장에서 더 높은 가격($P \uparrow$)을 보장받아야 합니다.
        \end{itemize}
    \item 공급 곡선 역시 수요 곡선과 마찬가지로, 분석의 편의를 위해 \textbf{선형(Linear)}으로 가정합니다.
\end{itemize}

\section{핵심 개념 4: 시장 균형 (Market Equilibrium)}

수요 곡선(구매자)과 공급 곡선(판매자)을 한 그래프에 합치면, 시장이 어떻게 작동하는지 알 수 있습니다.

\subsection{균형(Equilibrium)의 정의}

\textbf{정의}: "더 이상 변화할 유인이 없는 안정된 상태 (No tendency for change)"

\begin{itemize}
    \item \textbf{비유 (Cereal Bowl)}: 시리얼 그릇 바닥에 구슬이 멈춰 있는 상태가 '균형'입니다.
    \item 만약 누군가 그릇을 쳐서(외부 충격, Shock) 구슬이 흔들리더라도, 구슬은 결국 그릇 바닥의 새로운 지점에서 다시 멈출 것입니다. 이것이 '새로운 균형'입니다.
    \item 경제학자의 목표는 (1) 현재의 균형을 찾고, (2) 충격이 왔을 때, (3) 시장이 어떤 혼란을 거쳐 (4) 어떤 '새로운 균형'으로 이동할지 예측하는 것입니다.
\end{itemize}

\subsection{균형점의 결정}

\textbf{균형점(E)}: 수요 곡선($D_0$)과 공급 곡선($S_0$)이 교차하는 지점입니다.

\begin{itemize}
    \item \textbf{균형 가격 ($P_E$)}: 이 교차점의 가격입니다.
    \item \textbf{균형 거래량 ($Q_E$)}: 이 교차점의 수량입니다.
    \item \textbf{균형의 의미}: $P_E$ 가격에서는 소비자가 사고 싶어 하는 양(\textbf{수요량, $Q_D$})과 생산자가 팔고 싶어 하는 양(\textbf{공급량, $Q_S$})이 정확히 일치합니다 ($Q_D = Q_S$).
    \item 이 상태에서는 재고가 쌓이거나 물건이 부족해 줄을 서는 사람이 없습니다. 모든 시장 참여자(구매자, 판매자)가 만족하므로 가격이 변할 유인이 없습니다.
\end{itemize}


\newpage

%=======================================================================
% Chapter 2: 시장 구조와 완전 경쟁
%=======================================================================
\chapter{시장 구조와 완전 경쟁}
\label{ch:lecture2}

\section{개요}

본 문서는 기업 수준 경제학의 첫 번째 모듈을 요약합니다.
이전 과정에서 다룬 소비자 행동(수요 곡선)과 기업 행동(공급 곡선)을 바탕으로,
이제 \textbf{시장 구조(Market Structure)}라는 개념을 도입합니다.
기업의 의사결정은 동일한 비용 구조를 갖더라도, 경쟁자가 10,000명인지(완전 경쟁) 아니면 0명인지(독점)에 따라 완전히 달라집니다.

본 모듈에서는 완전 경쟁 시장을 중심으로 가격이 결정되는 원리를 파악하고,
개별 기업의 공급 곡선이 어떻게 도출되며, 이것이 시장 전체의 공급 곡선이 되는 과정을 학습합니다.

\section{시장 구조 (Market Structure)}

시장 구조는 산업의 "경쟁성"을 의미하며, 주로 시장 내 기업의 수(Number of firms)를 기준으로 스펙트럼을 형성합니다.

\begin{enumerate}
    \item \textbf{독점 (Monopoly):} 기업의 수가 1개입니다. 경쟁자가 전혀 없습니다.
    \item \textbf{과점 (Oligopoly):} "소수"의 기업 (2, 3, 4...)이 경쟁합니다. (예: 자동차 시장의 도요타와 혼다). 이들은 서로의 행동을 민감하게 관찰하며 전략적 결정을 내립니다.
    \item \textbf{경쟁 시장 (Competitive Market):} 기업의 수(N)가 "매우 많습니다". (N is very large).
\end{enumerate}

본 모듈에서는 양 극단의 사례인 \textbf{완전 경쟁}과 \textbf{독점}을 먼저 다루고, 그 중간인 과점을 분석합니다.

\section{완전 경쟁 시장 (Perfect Competition)}

완전 경쟁은 현실에서 정확히 들어맞는 경우는 드물지만(옥수수 시장이 가장 유사함), 시장 분석의 가장 기본이 되는 중요한 이론적 모델입니다. 이 모델이 성립하기 위해서는 4가지 엄격한 가정이 필요합니다.

\begin{mybox}[완전 경쟁의 4가지 조건]
\begin{enumerate}
    \item \textbf{수많은 소규모 구매자와 판매자}: 시장에 참여하는 개별 기업이나 소비자가 너무 많고 그 규모가 작아서, 누구도 시장 가격에 영향을 미칠 수 없습니다.
    \item \textbf{동질적 제품 (Homogeneous products)}: 모든 기업이 \textbf{완전히 동일한(identical)} 제품을 판매합니다.
    \item \textbf{자유로운 진입과 퇴출 (Free entry and exit)}: 새로운 기업이 시장에 진입하거나 기존 기업이 시장에서 나가는 데 어떠한 법적, 규제적 장벽이 없어야 합니다.
    \item \textbf{완전한 정보 (Perfect information)}: 모든 구매자와 판매자가 시장의 모든 정보(가격, 품질 등)를 완벽하게 알고 있어, 정보 부족으로 인한 실수를 하지 않습니다.
\end{enumerate}
\end{mybox}

\subsection{완전 경쟁의 핵심 결론: 가격 수용자 (Price Taker)}

위 4가지 조건이 모두 충족될 때의 핵심 결론은 다음과 같습니다:
\textbf{완전 경쟁 시장의 모든 개별 기업은 가격 수용자(Price Taker)입니다.}

즉, 개별 기업은 시장 가격에 아무런 통제력이 없으며, 시장(예: 시카고 상품 거래소)에서 결정된 가격($P_0$)을 \textbf{주어진 것(exogenously given)}으로 받아들여야 합니다.

\section{공급 곡선의 도출}

\subsection{개별 기업의 공급 곡선}

\begin{itemize}
    \item \textbf{기본 원리 (이윤 극대화):} 모든 기업은 \textbf{한계 수입(MR) = 한계 비용(MC)} 지점에서 생산량을 결정합니다.
    \item \textbf{완전 경쟁 적용:} 완전 경쟁 기업은 가격 수용자(Price Taker)입니다. 기업이 제품 1개를 더 팔 때 벌어들이는 수입(MR)은 정확히 시장 가격($P_0$)과 같습니다.
    \item 즉, $MR = P_0$ 입니다.
    \item \textbf{결론:} 완전 경쟁 기업은 \textbf{$P_0 = MC$}가 되는 지점에서 자신의 생산량($q^*$)을 결정합니다.
    \item 따라서, \textbf{개별 기업의 공급 곡선 = 한계비용(MC) 곡선}입니다.
\end{itemize}

\begin{warningbox}
\textbf{셧다운 조건 (Shutdown Condition)을 고려한 최종 정의}

기업은 생산을 할 때마다 최소한 가변 비용(재료비, 인건비 등)은 회수해야 합니다.
만약 가격(P)이 \textbf{평균가변비용(AVC)의 최저점}보다 낮아지면, 기업은 물건을 1개 만들 때마다 손해를 봅니다.
이 경우, 기업은 생산을 아예 중단(shutdown, $q=0$)하는 것이 손실을 최소화하는 길입니다.

따라서 개별 기업의 공급 곡선은 다음과 같이 정의됩니다:
\textbf{개별 기업 공급(S) 곡선 = 평균가변비용(AVC) 곡선의 최저점보다 \underline{위에 있는} 한계비용(MC) 곡선}
\end{warningbox}

\subsection{시장 공급 곡선}

시장 공급 곡선(Market Supply Curve, $S$)은 산업 내 모든 개별 기업($i=1$부터 $N$까지)의 공급 곡선(즉, MC 곡선)을 \textbf{수평으로 모두 더한(horizontal summation)} 것입니다.

$$
\text{시장 공급 곡선 } S = \sum_{i=1}^{N} MC_i \quad (\text{단, } P \ge AVC \text{ 최저점})
$$

\section{균형 분석: 단기(SR)와 장기(LR)}

\subsection{단기 균형 (Short-Run Equilibrium, SRE)}

단기 균형이 성립하기 위해서는 다음 2가지 조건이 충족되어야 합니다.

\begin{enumerate}
    \item \textbf{시장 균형:} 시장 수요량($D$)과 시장 공급량($S$)이 일치하는가? ($D=S$)
    \item \textbf{기업 균형:} 개별 기업이 이윤을 극대화하고 있는가? ($MR=MC$)
\end{enumerate}

\textbf{단기 이윤($\Pi$) 상태:}
단기 균형에서는 이윤($\Pi = q \cdot [P - ATC]$)이 0일 필요가 없습니다.
\begin{itemize}
    \item $\Pi > 0$ (양의 경제적 이윤): $P > ATC$ 일 때
    \item $\Pi < 0$ (음의 경제적 이윤 / 손실): $P < ATC$ 일 때
    \item $\Pi = 0$ (정상 이윤): $P = ATC$ 일 때
\end{itemize}

\subsection{장기 균형 (Long-Run Equilibrium, LRE)}

단기 균형에서 만약 이윤($\Pi$)이 0이 아니라면, 이 시장은 \textbf{장기적으로} 안정적이지 않습니다.
왜냐하면 완전 경쟁의 가정 3번, \textbf{자유로운 진입과 퇴출(Free Entry and Exit)}이 작동하기 때문입니다.

\begin{summarybox}
\textbf{장기 균형(LRE)의 3가지 조건}

따라서 장기 균형은 위 2가지 조건에 '진입/퇴출 유인 없음' 조건이 추가되어야 합니다.
\begin{enumerate}
    \item \textbf{시장 균형:} $D = S$
    \item \textbf{기업 균형:} $MR = MC$
    \item \textbf{장기 안정 (이윤 = 0):} $\Pi = 0$ (즉, $P = ATC$)
\end{enumerate}
장기 균형 상태에서 개별 기업은 \textbf{ATC 곡선의 최저점}에서 생산하게 됩니다 ($P = MC = \min ATC$).
\end{summarybox}

\section{주요 Q\&A}

\begin{keypointbox}[Q1: 장기 균형에서 "이윤($\Pi$) = 0" 이라는 것은 기업이 돈을 못 번다는 뜻인가요?]
\textbf{A: 아닙니다.} 이는 \textbf{경제적 이윤(Economic Profit)이 0}이라는 의미입니다.

경제적 이윤 = 회계상 이윤(Accounting Profit) - 기회비용(Opportunity Cost)

경제적 이윤이 0이라는 것은, 이 기업이 회계상으로는 \$20M의 큰 이익을 냈더라도, 이 기업이 할 수 있었던 차선책(기회비용) 역시 \$20M의 가치가 있다는 뜻입니다.
즉, \textbf{"다른 산업에 갔어도 딱 이만큼 벌 수 있었다"}는 의미이며, 따라서 굳이 이 산업을 떠나거나(퇴출) 다른 기업이 이 산업에 들어올(진입) 유인이 없는 \textbf{안정된 균형 상태}를 의미합니다.
\end{keypointbox}

\begin{keypointbox}[Q2: 개별 기업의 공급 곡선은 정확히 무엇인가요?]
\textbf{A: 한계비용(MC) 곡선입니다.}
더 정확하게는, \textbf{평균가변비용(AVC) 곡선의 최저점보다 위에 있는 MC 곡선 부분}입니다.
기업은 가격(P)이 AVC보다 낮으면 아예 생산을 중단(shutdown)하기 때문입니다.
\end{keypointbox}

\begin{keypointbox}[Q3: 주가는 왜 그렇게 변동이 심한가요?]
\textbf{A: 주가는 현재 실적이 아니라 "미래 이윤에 대한 기대($\Pi_t$)"의 현재가치 합이기 때문입니다.}
새로운 정보(뉴스, 스캔들, 신기술 발표 등)가 나올 때마다 시장 참여자들은 \textbf{미래의 기대 이윤($\Pi_t$)}을 즉각적으로 재조정합니다. 이 기대치가 변하면 주가 공식의 분자 값이 변하므로 주가가 실시간으로 변동합니다.
\end{keypointbox}

\end{document}
