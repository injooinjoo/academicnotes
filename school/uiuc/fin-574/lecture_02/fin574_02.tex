%%%%%%%%%%%%%%%%%%%%%%%%%%%%%%%%%%%%%%%%%%%%%%%%%%%%%%%%%%%%%%%%%%%%%%%%%%%%%%%
% Harvard Academic Notes - 통합 마스터 템플릿
% 모든 강의 노트에 적용되는 통일된 스타일
% 버전: 2.1 - 가독성 개선 (선택적 최적화)
% 최종 수정일: 2025-11-17
%%%%%%%%%%%%%%%%%%%%%%%%%%%%%%%%%%%%%%%%%%%%%%%%%%%%%%%%%%%%%%%%%%%%%%%%%%%%%%%

\documentclass[11pt,a4paper]{article}

%========================================================================================
% 기본 패키지
%========================================================================================

% --- 한국어 지원 ---
\usepackage{kotex}

% --- 페이지 레이아웃 ---
\usepackage[top=20mm, bottom=20mm, left=20mm, right=18mm]{geometry}
\usepackage{setspace}
\onehalfspacing                      % 1.5배 줄간격
\setlength{\parskip}{0.5em}          % 문단 간격
\setlength{\parindent}{0pt}          % 들여쓰기 없음

% --- 표 관련 ---
\usepackage{booktabs}              % 고품질 표
\usepackage{tabularx}              % 자동 너비 조절 표
\usepackage{array}                 % 표 컬럼 확장
\usepackage{longtable}             % 여러 페이지 표
\renewcommand{\arraystretch}{1.1}  % 표 행간 조절

%========================================================================================
% 헤더 및 푸터
%========================================================================================

\usepackage{fancyhdr}
\pagestyle{fancy}
\fancyhf{}
\fancyhead[L]{\small\textit{FIN 574: 기업 수준 경제학}}
\fancyhead[R]{\small\textit{Lecture 02}}
\fancyfoot[C]{\thepage}
\renewcommand{\headrulewidth}{0.5pt}
\renewcommand{\footrulewidth}{0.3pt}

% 첫 페이지는 헤더 없음
\fancypagestyle{firstpage}{
    \fancyhf{}
    \fancyfoot[C]{\thepage}
    \renewcommand{\headrulewidth}{0pt}
}

%========================================================================================
% 색상 정의 (파스텔 톤 + 다크모드 호환)
%========================================================================================

\usepackage[dvipsnames]{xcolor}

% 밝은 배경용 파스텔 색상
\definecolor{lightblue}{RGB}{220, 235, 255}      % 부드러운 파랑
\definecolor{lightgreen}{RGB}{220, 255, 235}     % 부드러운 초록
\definecolor{lightyellow}{RGB}{255, 250, 220}    % 부드러운 노랑
\definecolor{lightpurple}{RGB}{240, 230, 255}    % 부드러운 보라
\definecolor{lightgray}{gray}{0.95}              % 밝은 회색
\definecolor{lightpink}{RGB}{255, 235, 245}      % 부드러운 핑크
\definecolor{boxgray}{gray}{0.95}
\definecolor{boxblue}{rgb}{0.9, 0.95, 1.0}
\definecolor{boxred}{rgb}{1.0, 0.95, 0.95}

% 진한 색상 (테두리/제목용)
\definecolor{darkblue}{RGB}{50, 80, 150}
\definecolor{darkgreen}{RGB}{40, 120, 70}
\definecolor{darkorange}{RGB}{200, 100, 30}
\definecolor{darkpurple}{RGB}{100, 60, 150}

%========================================================================================
% 박스 환경 (tcolorbox) - 6가지 타입
%========================================================================================

\usepackage[most]{tcolorbox}
\tcbuselibrary{skins, breakable}

% 1. 개요 박스 (강의 시작 부분)
\newtcolorbox{overviewbox}[1][]{
    enhanced,
    colback=lightpurple,
    colframe=darkpurple,
    fonttitle=\bfseries\large,
    title=📚 강의 개요,
    arc=3mm,
    boxrule=1pt,
    left=8pt,
    right=8pt,
    top=8pt,
    bottom=8pt,
    breakable,
    #1
}

% 2. 요약 박스
\newtcolorbox{summarybox}[1][]{
    enhanced,
    colback=lightblue,
    colframe=darkblue,
    fonttitle=\bfseries,
    title=📝 핵심 요약,
    arc=2mm,
    boxrule=0.7pt,
    left=6pt,
    right=6pt,
    top=6pt,
    bottom=6pt,
    breakable,
    #1
}

% 3. 핵심 정보 박스
\newtcolorbox{infobox}[1][]{
    enhanced,
    colback=lightgreen,
    colframe=darkgreen,
    fonttitle=\bfseries,
    title=💡 핵심 정보,
    arc=2mm,
    boxrule=0.7pt,
    left=6pt,
    right=6pt,
    top=6pt,
    bottom=6pt,
    breakable,
    #1
}

% 4. 주의사항 박스
\newtcolorbox{warningbox}[1][]{
    enhanced,
    colback=lightyellow,
    colframe=darkorange,
    fonttitle=\bfseries,
    title=⚠️ 주의사항,
    arc=2mm,
    boxrule=0.7pt,
    left=6pt,
    right=6pt,
    top=6pt,
    bottom=6pt,
    breakable,
    #1
}

% 5. 예제 박스
\newtcolorbox{examplebox}[1][]{
    enhanced,
    colback=lightgray,
    colframe=black!60,
    fonttitle=\bfseries,
    title=📖 예제: #1,
    arc=2mm,
    boxrule=0.7pt,
    left=6pt,
    right=6pt,
    top=6pt,
    bottom=6pt,
    breakable,
}

% 6. 정의 박스
\newtcolorbox{definitionbox}[1][]{
    enhanced,
    colback=lightpink,
    colframe=purple!70!black,
    fonttitle=\bfseries,
    title=📌 정의: #1,
    arc=2mm,
    boxrule=0.7pt,
    left=6pt,
    right=6pt,
    top=6pt,
    bottom=6pt,
    breakable,
}

% 7. 중요 박스 (importantbox - warningbox와 유사)
\newtcolorbox{importantbox}[1][]{
    enhanced,
    colback=boxred,
    colframe=red!70!black,
    fonttitle=\bfseries,
    title=⚠️ 매우 중요: #1,
    arc=2mm,
    boxrule=0.7pt,
    left=6pt,
    right=6pt,
    top=6pt,
    bottom=6pt,
    breakable,
}

% 8. cautionbox (warningbox와 동일)
\let\cautionbox\warningbox
\let\endcautionbox\endwarningbox

%========================================================================================
% 코드 블록 설정 (밝은 배경)
%========================================================================================

\usepackage{listings}

\definecolor{codegray}{rgb}{0.5,0.5,0.5}
\definecolor{codepurple}{rgb}{0.58,0,0.82}
\definecolor{backcolour}{rgb}{0.95,0.95,0.95}

\lstset{
    basicstyle=\ttfamily\small,
    backgroundcolor=\color{lightgray},
    keywordstyle=\color{darkblue}\bfseries,
    commentstyle=\color{darkgreen}\itshape,
    stringstyle=\color{purple!80!black},
    numberstyle=\tiny\color{black!60},
    numbers=left,
    numbersep=8pt,
    breaklines=true,
    breakatwhitespace=false,
    frame=single,
    frameround=tttt,
    rulecolor=\color{black!30},
    captionpos=b,
    showstringspaces=false,
    tabsize=2,
    xleftmargin=15pt,
    xrightmargin=5pt,
    escapeinside={\%*}{*)}
}

% Python 코드 스타일
\lstdefinestyle{pythonstyle}{
    language=Python,
    morekeywords={self, True, False, None},
}

% SQL 코드 스타일
\lstdefinestyle{sqlstyle}{
    language=SQL,
    morekeywords={SELECT, FROM, WHERE, JOIN, GROUP, BY, ORDER, HAVING},
}

%========================================================================================
% 목차 스타일링
%========================================================================================

\usepackage{tocloft}
\renewcommand{\cftsecleader}{\cftdotfill{\cftdotsep}}
\setlength{\cftbeforesecskip}{0.4em}
\renewcommand{\cftsecfont}{\bfseries}
\renewcommand{\cftsubsecfont}{\normalfont}

%========================================================================================
% 표 및 그림
%========================================================================================

\usepackage{graphicx}              % 이미지
\usepackage{adjustbox}             % 표/박스 크기 조절

% 표 캡션 스타일
\usepackage{caption}
\captionsetup[table]{
    labelfont=bf,
    textfont=it,
    skip=5pt
}
\captionsetup[figure]{
    labelfont=bf,
    textfont=it,
    skip=5pt
}

%========================================================================================
% 수학
%========================================================================================

\usepackage{amsmath, amssymb, amsthm}

% 정리 환경
\theoremstyle{definition}
\newtheorem{theorem}{정리}[section]
\newtheorem{lemma}[theorem]{보조정리}
\newtheorem{proposition}[theorem]{명제}
\newtheorem{corollary}[theorem]{따름정리}
\newtheorem{definition}{정의}[section]
\newtheorem{example}{예제}[section]

%========================================================================================
% 하이퍼링크
%========================================================================================

\usepackage[
    colorlinks=true,
    linkcolor=blue!80!black,
    urlcolor=blue!80!black,
    citecolor=green!60!black,
    bookmarks=true,
    bookmarksnumbered=true,
    pdfborder={0 0 0}
]{hyperref}

% PDF 메타데이터는 각 문서에서 설정
\hypersetup{
    pdftitle={FIN 574: 기업 수준 경제학 - Lecture 02},
    pdfauthor={강의 노트},
    pdfsubject={Academic Notes}
}

%========================================================================================
% 기타 유용한 패키지
%========================================================================================

\usepackage{enumitem}              % 리스트 커스터마이징
\setlist{nosep, leftmargin=*, itemsep=0.3em}

\usepackage{microtype}             % 타이포그래피 개선
\usepackage{footnote}              % 각주 개선
\usepackage{url}                   % URL 줄바꿈
\urlstyle{same}

%========================================================================================
% 사용자 정의 명령어
%========================================================================================

% 강조 텍스트
\newcommand{\important}[1]{\textbf{\textcolor{red!70!black}{#1}}}
\newcommand{\keyword}[1]{\textbf{#1}}
\newcommand{\term}[1]{\textit{#1}}
\newcommand{\code}[1]{\texttt{#1}}

% 용어 설명 (인라인)
\newcommand{\defterm}[2]{\textbf{#1}\footnote{#2}}

% 섹션 시작 전 페이지 분리
\newcommand{\newsection}[1]{\newpage\section{#1}}

%========================================================================================
% 문서 제목 스타일
%========================================================================================

\usepackage{titling}
\pretitle{\begin{center}\LARGE\bfseries}
\posttitle{\par\end{center}\vskip 0.5em}
\preauthor{\begin{center}\large}
\postauthor{\end{center}}
\predate{\begin{center}\large}
\postdate{\par\end{center}}

%========================================================================================
% 섹션 제목 간격
%========================================================================================

\usepackage{titlesec}
\titlespacing*{\section}{0pt}{1.5em}{0.8em}
\titlespacing*{\subsection}{0pt}{1.2em}{0.6em}
\titlespacing*{\subsubsection}{0pt}{1em}{0.5em}

%========================================================================================
% 메타 정보 박스 명령어
%========================================================================================

\newcommand{\metainfo}[4]{
\begin{tcolorbox}[
    colback=lightpurple,
    colframe=darkpurple,
    boxrule=1pt,
    arc=2mm,
    left=10pt,
    right=10pt,
    top=8pt,
    bottom=8pt
]
\begin{tabular}{@{}rl@{}}
▣ \textbf{강의명:} & #1 \\[0.3em]
▣ \textbf{주차:} & #2 \\[0.3em]
▣ \textbf{교수명:} & #3 \\[0.3em]
▣ \textbf{목적:} & \begin{minipage}[t]{0.75\textwidth}#4\end{minipage}
\end{tabular}
\end{tcolorbox}
}

%========================================================================================
% 끝
%========================================================================================


\begin{document}

\maketitle
\thispagestyle{firstpage}

\metainfo{FIN 574: 기업 수준 경제학}{Lecture 02}{UIUC Faculty}{Lecture 02의 핵심 개념 학습}


\tableofcontents

\newpage
\section{개요}

본 문서는 기업 수준 경제학의 첫 번째 모듈을 요약합니다.
이전 과정에서 다룬 소비자 행동(수요 곡선)과 기업 행동(공급 곡선)을 바탕으로,
이제 \textbf{시장 구조(Market Structure)}라는 개념을 도입합니다.
기업의 의사결정은 동일한 비용 구조를 갖더라도, 경쟁자가 10,000명인지(완전 경쟁) 아니면 0명인지(독점)에 따라 완전히 달라집니다.

본 모듈에서는 완전 경쟁 시장을 중심으로 가격이 결정되는 원리를 파악하고,
개별 기업의 공급 곡선이 어떻게 도출되며, 이것이 시장 전체의 공급 곡선이 되는 과정을 학습합니다.
또한 단기 및 장기 균형의 개념을 이해하고, 외부 충격이 발생했을 때 시장이 어떻게 새로운 균형으로 이동하는지 분석합니다.

\section{핵심 용어 정리}

\begin{mybox}{주요 용어}
\small
\begin{tabularx}{\textwidth}{@{}>{\raggedright\arraybackslash}p{2.5cm} >{\raggedright\arraybackslash}X >{\raggedright\arraybackslash}p{2.8cm} >{\raggedright\arraybackslash}p{3cm}@{}}
\toprule
\textbf{용어} & \textbf{쉬운 설명} & \textbf{원어 (Eng)} & \textbf{비고 (예시)} \\
\midrule
시장 구조 & 산업의 "경쟁성"을 나타내며, 주로 시장 내 기업의 수로 구분함. & Market Structure & 독점, 과점, 완전 경쟁 등. \\
완전 경쟁 & 수많은 소규모 구매자와 판매자, 동질적 제품, 자유로운 진입/퇴출, 완전한 정보. & Perfect Competition & 옥수수, 밀 등 현실에 가까운 시장. \\
가격 수용자 & 시장에서 결정된 가격을 그대로 받아들여야 하는 개별 기업. & Price Taker & 완전 경쟁 시장의 기업. \\
독점 & 단 하나의 기업이 시장 전체를 지배하는 구조. & Monopoly & 특허권, 천연자원 독점. \\
과점 & 소수의 기업이 시장을 지배하는 구조. 기업 간 상호작용이 매우 중요함. & Oligopoly & 자동차 시장 (현대/기아, 도요타/혼다). \\
영구 자산 & 영구적으로 이윤을 창출할 것으로 기대되는 자산. & Asset in Perpetuity & 토지(농지), 기업 주식. \\
순현재가치 & 미래에 발생할 수익의 흐름을 현재 시점의 가치로 할인(환산)한 값. & Net Present Value (NPV) & 주가 계산의 기본 원리. \\
경제적 이윤 & 총수입에서 모든 비용(명시적 비용 + 기회비용)을 제외한 이윤. & Economic Profit & 이윤이 0이라도 기회비용만큼은 버는 것. \\
시장 실패 & 시장이 자원을 효율적으로 배분하지 못하는 상황. & Market Failure & 외부 효과, 비대칭 정보로 인해 발생. \\
외부 효과 & 한 경제 주체의 행위가 제3자에게 의도치 않은 혜택이나 손해를 주는 것. & Externality & 긍정적 예: 양봉과 과수원. 부정적 예: 공장의 오염. \\
비대칭 정보 & 거래 당사자 중 한쪽이 다른 쪽보다 더 많은 정보를 가진 상황. & Asymmetric Information & 중고차 시장. \\
\bottomrule
\end{tabularx}
\end{mybox}

\newpage
\section{핵심 개념 및 원리}

\subsection{기업가치 평가: 영구 자산과 주가}

이윤 극대화 모델은 개인 사업자(Sole Proprietorships)나 파트너십(Partnerships)에는 잘 적용됩니다.
하지만 \textbf{주식회사(Corporations)}는 \textbf{소유와 경영이 분리}되어 있다는 점에서 복잡합니다.
소유주(주주)는 불특정 다수이며, 경영자(C-Suite)는 전문 경영인입니다.

주식회사는 \textbf{영구 자산(Asset in Perpetuity)}으로 간주됩니다.
이는 해당 자산이 일회성이 아니라 미래에도 계속해서 이윤을 창출할 것이라는 의미입니다.

\begin{mybox}{예시: 영구 자산의 가치}
\begin{itemize}
    \item \textbf{농지(Farmland):} 1에이커의 옥수수밭이 작년에 \$200의 수익을 냈다고 해서 이 밭의 가격이 \$200은 아닙니다. 이 밭은 내년에도, 10년 후에도 계속 수익을 창출할 것이기 때문입니다. 따라서 밭의 가격은 미래에 발생할 모든 수익의 \textbf{현재 가치 합}입니다.
    \item \textbf{뉴욕 택시 메달리온(Taxicab Medallions):} 뉴욕에서 택시 영업을 할 수 있는 권리(면허)입니다. 이 권리는 미래 수익을 보장하므로 자산 가치를 가집니다.
    \begin{itemize}
        \item 1947년: \$2,500
        \item 2013년: \$1,300,000 (최고점)
        \item 2018년: 약 \$200,000 (우버(Uber) 효과로 인해 자산 가치 폭락)
    \end{itemize}
\end{itemize}
\end{mybox}

\subsubsection{주가(Price of a Share of Stock) 결정}
주식의 가격($P_{\text{stock}}$)은 해당 주식이 미래에 가져다줄 것으로 \textbf{기대되는} 모든 이윤($\Pi$)의 흐름을 현재 가치로 할인(Discount)한 합입니다.

$$
P_{\text{stock}} = \Pi_0 + \frac{\Pi_1}{(1+r)} + \frac{\Pi_2}{(1+r)^2} + \frac{\Pi_3}{(1+r)^3} + \dots = \sum_{t=0}^{\infty} \frac{\Pi_t}{(1+r)^t}
$$

\begin{itemize}
    \item $\Pi_t$: $t$ 시점의 \textbf{기대} 이윤 (per share)
    \item $r$: 이자율 (할인율)
\end{itemize}

주가는 현재의 실적이 아니라 \textbf{미래 기대($\Pi_t$)}에 의해 결정됩니다.
만약 어떤 기업에 대한 부정적인 뉴스(예: 폭스바겐 배출가스 조작 스캔들)가 나오면,
투자자들은 미래 기대 이윤($\Pi_t$)이 감소할 것(벌금, 브랜드 이미지 손상 등)이라고 예상합니다.
이 기대치가 낮아지면, 위 공식의 분자 값이 작아져 주가($P_{\text{stock}}$)는 즉시 하락합니다.

\subsection{시장 구조 (Market Structure)}

시장 구조는 산업의 "경쟁성"을 의미하며, 주로 시장 내 기업의 수(Number of firms)를 기준으로 스펙트럼을 형성합니다.

\begin{enumerate}
    \item \textbf{독점 (Monopoly):} 기업의 수가 1개입니다. 경쟁자가 전혀 없습니다.
    \item \textbf{과점 (Oligopoly):} "소수"의 기업 (2, 3, 4...)이 경쟁합니다. (예: 자동차 시장의 도요타와 혼다). 이들은 서로의 행동을 민감하게 관찰하며 전략적 결정을 내립니다.
    \item \textbf{경쟁 시장 (Competitive Market):} 기업의 수(N)가 "매우 많습니다". (N is very large).
\end{enumerate}

본 모듈에서는 양 극단의 사례인 \textbf{완전 경쟁}과 \textbf{독점}을 먼저 다루고, 그 중간인 과점을 분석합니다.

\subsection{완전 경쟁 시장 (Perfect Competition)}

완전 경쟁은 현실에서 정확히 들어맞는 경우는 드물지만(옥수수 시장이 가장 유사함), 시장 분석의 가장 기본이 되는 중요한 이론적 모델입니다. 이 모델이 성립하기 위해서는 4가지 엄격한 가정이 필요합니다.

\begin{mybox}{완전 경쟁의 4가지 조건}
\begin{enumerate}
    \item \textbf{수많은 소규모 구매자와 판매자 (Large number of relatively small buyers and sellers)}
    \begin{itemize}
        \item 시장에 참여하는 개별 기업이나 소비자가 너무 많고 그 규모가 작아서, 누구도 시장 가격에 영향을 미칠 수 없습니다.
        \item [예시 👍] 일리노이 최대 옥수수 농가가 생산량을 3배 늘려도, 미국 전체 옥수수 시장 가격에는 아무런 영향(0)을 주지 못합니다.
    \end{itemize}
    \item \textbf{동질적 제품 (Homogeneous products)}
    \begin{itemize}
        \item 모든 기업이 \textbf{완전히 동일한(identical)} 제품을 판매합니다. (예: 2번 적색 겨울 밀, 옥수수, 원유).
        \item [나쁜 예 👎] 제품별로 맛, 디자인, 브랜드가 다른 수제 맥주, 사탕, 치약 시장은 해당되지 않습니다.
    \end{itemize}
    \item \textbf{자유로운 진입과 퇴출 (Free entry and exit)}
    \begin{itemize}
        \item 새로운 기업이 시장에 진입하거나 기존 기업이 시장에서 나가는 데 어떠한 법적, 규제적 장벽이 없어야 합니다.
        \item (주의: 공장 설립 비용 같은 "금전적 비용"이 없다는 뜻이 아니라, "진입 허가"가 필요 없다는 의미입니다.)
        \item [나쁜 예 👎] 정부가 주류 판매 면허(liquor license) 수를 제한하는 바(Bar) 산업은 자유로운 진입이 아닙니다.
    \end{itemize}
    \item \textbf{완전한 정보 (Perfect information)}
    \begin{itemize}
        \item 모든 구매자와 판매자가 시장의 모든 정보(가격, 품질 등)를 완벽하게 알고 있어, 정보 부족으로 인한 실수를 하지 않습니다.
    \end{itemize}
\end{enumerate}
\end{mybox}

\subsubsection{완전 경쟁의 핵심 결론: 가격 수용자 (Price Taker)}

위 4가지 조건이 모두 충족될 때의 핵심 결론은 다음과 같습니다:
\textbf{완전 경쟁 시장의 모든 개별 기업은 가격 수용자(Price Taker)입니다.}

즉, 개별 기업은 시장 가격에 아무런 통제력이 없으며, 시장(예: 시카고 상품 거래소)에서 결정된 가격($P_0$)을 \textbf{주어진 것(exogenously given)}으로 받아들여야 합니다.

\begin{itemize}
    \item \textbf{이유:}
    \begin{enumerate}
        \item 만약 시장가 \$3.70인 옥수수를 한 농부가 \$4.00에 팔려고 하면? (가정 2: 동질적 제품)
        \item 소비자들은 정확히 똑같은 옥수수를 \$3.70에 파는 다른 수많은 농부에게서 살 것이므로, 이 농부는 0개의 옥수수도 팔지 못할 것입니다.
        \item 반대로 \$3.70보다 싸게 팔 이유도 없습니다. 어차피 시장 가격 \$3.70에 원하는 만큼 모두 팔 수 있기 때문입니다.
    \end{enumerate}
\end{itemize}

\subsection{공급 곡선의 도출 (Derivation of the Supply Curve)}

우리의 목표는 시장 공급곡선(Market S curve)이 어떻게 만들어지는지 분해(deconstruct)하는 것입니다.

\subsubsection{1단계: 개별 기업의 공급 곡선}

\begin{itemize}
    \item \textbf{기본 원리 (이윤 극대화):} 모든 기업은 \textbf{한계 수입(MR) = 한계 비용(MC)} 지점에서 생산량을 결정합니다.
    \item \textbf{완전 경쟁 적용:} 완전 경쟁 기업은 가격 수용자(Price Taker)입니다. 기업이 제품 1개를 더 팔 때 벌어들이는 수입(MR)은 정확히 시장 가격($P_0$)과 같습니다.
    \item 즉, $MR = P_0$ 입니다. (개별 기업이 보는 MR 곡선은 $P_0$ 높이의 수평선입니다.)
    \item \textbf{결론:} 완전 경쟁 기업은 $MR = MC$가 아닌 \textbf{$P_0 = MC$}가 되는 지점에서 자신의 생산량($q^*$)을 결정합니다.
    \item \textbf{공급 곡선의 정의:} "다양한 가격 수준에서 기업이 얼마나 생산(공급)할 의향이 있는가?"
    \item $P_1$일 때 $\to$ $P_1 = MC$에서 $q_1$ 생산.
    \item $P_2$일 때 $\to$ $P_2 = MC$에서 $q_2$ 생산.
    \item $P_3$일 때 $\to$ $P_3 = MC$에서 $q_3$ 생산.
    \item 이는 기업이 자신의 \textbf{한계비용(MC) 곡선을 따라 생산량을 결정}한다는 의미입니다.
    \item 따라서, \textbf{개별 기업의 공급 곡선 = 한계비용(MC) 곡선}입니다.
\end{itemize}

\begin{warningbox}
\textbf{셧다운 조건 (Shutdown Condition)을 고려한 최종 정의}

기업은 생산을 할 때마다 최소한 가변 비용(재료비, 인건비 등)은 회수해야 합니다.
만약 가격(P)이 \textbf{평균가변비용(AVC)의 최저점}보다 낮아지면, 기업은 물건을 1개 만들 때마다 손해를 봅니다.
이 경우, 기업은 고정 비용(임대료 등)은 어차피 손해 보더라도, 생산을 아예 중단(shutdown, $q=0$)하는 것이 손실을 최소화하는 길입니다.

따라서 개별 기업의 공급 곡선은 다음과 같이 정의됩니다.
\textbf{개별 기업 공급(S) 곡선 = 평균가변비용(AVC) 곡선의 최저점보다 \underline{위에 있는} 한계비용(MC) 곡선}
\end{warningbox}

\subsubsection{2단계: 시장 공급 곡선}

시장 공급 곡선(Market Supply Curve, $S$)은 매우 간단합니다.
이는 산업 내 모든 개별 기업($i=1$부터 $N$까지)의 공급 곡선(즉, MC 곡선)을 \textbf{수평으로 모두 더한(horizontal summation)} 것입니다.

$$
\text{시장 공급 곡선 } S = \sum_{i=1}^{N} MC_i \quad (\text{단, } P \ge AVC \text{ 최저점})
$$

이제 우리는 시장 공급 곡선($S$)이 단순히 우상향하는 선이 아니라, 그 산업 내 모든 기업의 한계 생산 비용을 합산한 것임을 알게 되었습니다.

\newpage
\section{균형 분석 방법론: 단기(SR)와 장기(LR)}

\subsection{분석 도구: Side-by-Side 그래프}

완전 경쟁 시장을 분석하기 위해 두 개의 그래프를 나란히 놓고 사용합니다.

\begin{enumerate}
    \item \textbf{왼쪽 (시장, Market):}
    \begin{itemize}
        \item X축: 시장 전체 수량 (대문자 $Q$)
        \item Y축: 가격 ($P$)
        \item 곡선: 시장 수요($D$)와 시장 공급($S$)
    \end{itemize}
    \item \textbf{오른쪽 (개별 기업, "Representative Firm"):}
    \begin{itemize}
        \item X축: 개별 기업 수량 (소문자 $q$)
        \item Y축: 가격 및 비용 ($P, MC, ATC, AVC$)
        \item 곡선: $MC$, $ATC$, $AVC$
    \end{itemize}
\end{enumerate}

\begin{mybox}{가장 중요한 연결고리}
두 그래프의 \textbf{Y축(가격)은 동일}합니다.
왼쪽 \textbf{시장(Market)} 그래프에서 $S$와 $D$가 만나 결정된 \textbf{균형 가격($P_0$)}이
오른쪽 \textbf{개별 기업(Firm)} 그래프의 \textbf{수평선인 한계수입($MR_0$) 곡선}이 됩니다.
($P_0 = MR_0$)
\end{mybox}

\subsection{단기 균형 (Short-Run Equilibrium, SRE)}

단기 균형은 "더 이상 변할 유인이 없는" 안정 상태(사발 속 구슬이 멈춘 상태)를 의미합니다.
단기 균형이 성립하기 위해서는 다음 2가지 조건이 충족되어야 합니다.

\begin{enumerate}
    \item \textbf{시장 균형:} 시장 수요량($D$)과 시장 공급량($S$)이 일치하는가? ($D=S$)
    \item \textbf{기업 균형:} 개별 기업이 이윤을 극대화하고 있는가? ($MR=MC$)
    \begin{itemize}
        \item (그리고 셧다운하지 않는가? $P \ge AVC$)
    \end{itemize}
\end{enumerate}

\textbf{단기 이윤($\Pi$) 상태:}
단기 균형에서는 이윤($\Pi = q \cdot [P - ATC]$)이 0일 필요가 없습니다.
\begin{itemize}
    \item $\Pi > 0$ (양의 경제적 이윤): $P > ATC$ 일 때
    \item $\Pi < 0$ (음의 경제적 이윤 / 손실): $P < ATC$ 일 때 (단, $P > AVC$라서 생산은 함)
    \item $\Pi = 0$ (정상 이윤): $P = ATC$ 일 때
\end{itemize}

\subsection{장기 균형 (Long-Run Equilibrium, LRE)}

단기 균형에서 만약 이윤($\Pi$)이 0이 아니라면 (즉, $\Pi > 0$ 또는 $\Pi < 0$),
이 시장은 \textbf{장기적으로} 안정적이지 않습니다.
왜냐하면 완전 경쟁의 가정 3번, \textbf{자유로운 진입과 퇴출(Free Entry and Exit)}이 작동하기 때문입니다.

\subsubsection{사례 1: 단기에 $\Pi > 0$ (초과 이윤) 발생 시}
\begin{itemize}
    \item \textbf{의미:} 이 산업의 기업들이 다른 산업(기회비용)보다 돈을 더 많이 벌고 있습니다.
    \item \textbf{행동:} 외부의 다른 기업가들이 "저기 돈 되네!"라며 이 시장으로 \textbf{진입(Entry)}합니다.
    \item \textbf{결과 (시장에):}
    \begin{enumerate}
        \item 진입 $\to$ 시장 내 기업 수($N$) 증가.
        \item $\to$ 시장 공급 곡선($S = \sum MC$)이 \textbf{오른쪽으로 이동}.
        \item $\to$ 시장 가격($P$)이 \textbf{하락}.
    \end{enumerate}
    \item \textbf{언제까지?:} 이윤($\Pi$)이 0이 될 때까지 진입과 가격 하락이 계속됩니다.
\end{itemize}

\subsubsection{사례 2: 단기에 $\Pi < 0$ (손실) 발생 시}
\begin{itemize}
    \item \textbf{의미:} 이 산업의 기업들이 다른 산업에서 벌 수 있는 돈(기회비용)보다도 못 벌고 있습니다.
    \item \textbf{행동:} 기존 기업들이 "더 이상 못 버티겠다!"라며 이 시장에서 \textbf{퇴출(Exit)}합니다.
    \item \textbf{결과 (시장에):}
    \begin{enumerate}
        \item 퇴출 $\to$ 시장 내 기업 수($N$) 감소.
        \item $\to$ 시장 공급 곡선($S = \sum MC$)이 \textbf{왼쪽으로 이동}.
        \item $\to$ 시장 가격($P$)이 \textbf{상승}.
    \end{enumerate}
    \item \textbf{언제까지?:} 이윤($\Pi$)이 0이 될 때까지 퇴출과 가격 상승이 계속됩니다.
\end{itemize}

\begin{summarybox}
\textbf{장기 균형(LRE)의 3가지 조건}

따라서 장기 균형은 위 2가지 조건에 '진입/퇴출 유인 없음' 조건이 추가되어야 합니다.
\begin{enumerate}
    \item \textbf{시장 균형:} $D = S$
    \item \textbf{기업 균형:} $MR = MC$
    \item \textbf{장기 안정 (이윤 = 0):} $\Pi = 0$ (즉, $P = ATC$)
\end{enumerate}
장기 균형 상태에서 개별 기업은 \textbf{ATC 곡선의 최저점}에서 생산하게 됩니다 ($P = MC = \min ATC$).
\end{summarybox}

\subsection{사례 분석: 외부 충격과 균형 이동}

시장에 외부 충격(예: 정부 발표, 기술 혁신)이 발생했을 때, 시장이 어떻게 반응하는지 SRE와 LRE 개념을 사용해 분석할 수 있습니다.

\begin{mybox}{분석 4단계: 수요 증가 (예: "이 제품, 암 예방에 효과" 발표)}
\begin{enumerate}
    \item \textbf{1단계 (시작): 초기 장기 균형 (Initial LRE)}
    \begin{itemize}
        \item 시장: $S_0$와 $D_0$가 만나 $P_0, Q_0$ 결정.
        \item 기업: $P_0 = MR_0$이며, $P_0 = MC = \min ATC$ 지점에서 $q_0^*$ 생산. \textbf{$\Pi = 0$}.
    \end{itemize}
    \item \textbf{2단계 (충격): 수요 증가 (Shock)}
    \begin{itemize}
        \item 시장: 수요 곡선이 \textbf{$D_0 \to D_1$ (오른쪽 이동)}.
        \item 기업: 아직 변화 없음.
    \end{itemize}
    \item \textbf{3단계 (결과): 새로운 단기 균형 (New SRE)}
    \begin{itemize}
        \item 시장: $D_1$이 $S_0$(공급은 아직 그대로)와 만나는 \textbf{새로운 단기 가격 $P_1$ 형성} ($P_1 > P_0$). 시장 수량 $Q_1$ 증가.
        \item 기업: 높아진 가격 $P_1$을 $MR_1$으로 받아들임. $MR_1 = MC$ 지점까지 생산량을 $q_1^*$로 늘림.
        \item \textbf{핵심:} $P_1$이 $ATC$보다 높아졌으므로, \textbf{$\Pi > 0$ (양의 경제적 이윤)} 발생.
    \end{itemize}
    \item \textbf{4단계 (조정): 새로운 장기 균형 (New LRE)}
    \begin{itemize}
        \item 시장: 3단계의 초과 이윤($\Pi > 0$)을 보고 \textbf{새로운 기업들이 진입(Entry)}.
        \item $\to$ 공급 곡선이 \textbf{$S_0 \to S_2$ (오른쪽 이동)}.
        \item $\to$ 공급 증가는 시장 가격을 다시 \textbf{하락}시킴 ( $P_1 \to \dots \to P_2$ ).
        \item $\to$ 이윤이 0이 될 때까지 진입과 가격 하락이 계속됨.
    \end{itemize}
\end{enumerate}

\textbf{최종 결과 (Final LRE):}
\begin{itemize}
    \item \textbf{가격($P$):} $P_2 = P_0$. 가격은 결국 원래의 장기 균형 수준(ATC 최저점)으로 복귀.
    \item \textbf{시장 수량($Q$):} $Q_2 > Q_1 > Q_0$. 시장 전체의 거래량은 영구적으로 증가.
    \item \textbf{기업 수량($q$):} $q_2^* = q_0^*$. 개별 기업의 생산량은 원래 수준(ATC 최저점)으로 복귀.
    \item \textbf{이윤($\Pi$):} $\Pi = 0$. 다시 안정 상태로 복귀.
\end{itemize}
\end{mybox}

\newpage
\section{학습 체크리스트}

외부 충격(Shock)이 발생한 시장을 분석할 때 다음 단계를 순서대로 점검하세요.

\begin{enumerate}
    \item [ ] \textbf{시작점 확인:} 분석을 시작하는 시점이 장기 균형(LRE, $\Pi=0$) 상태인가?
    \item [ ] \textbf{충격 식별:} 발생한 이벤트(예: 정부 규제, 신기술, 소비자 선호 변화)는 4개의 곡선 ($S, D, MC, ATC$) 중 무엇을, 어느 방향으로 이동시키는가?
    \begin{itemize}
        \item (예: 수요 증가 $\to$ $D$ 곡선 우측 이동)
        \item (예: 원자재 비용 증가 $\to$ $MC, ATC$ 곡선 상향 이동 $\to$ $S$ 곡선 좌측 이동)
    \end{itemize}
    \item [ ] \textbf{새로운 단기 균형(SRE) 도출:}
    \begin{itemize}
        \item [ ] 이동한 곡선과 기존 곡선이 만나는 새로운 단기 균형 가격($P_1$)과 수량($Q_1$)을 찾았는가?
        \item [ ] 새로운 가격($P_1$)이 개별 기업의 $ATC$ 곡선보다 위인가, 아래인가?
        \item [ ] 단기 이윤($\Pi$) 상태를 판별했는가? ($\Pi > 0$ 또는 $\Pi < 0$)
    \end{itemize}
    \item [ ] \textbf{장기 조정(LRE) 예측:}
    \begin{itemize}
        \item [ ] $\Pi > 0$ 인가? $\to$ \textbf{진입(Entry)} $\to$ $S$ 곡선 우측 이동.
        \item [ ] $\Pi < 0$ 인가? $\to$ \textbf{퇴출(Exit)} $\to$ $S$ 곡선 좌측 이동.
    \end{itemize}
    \item [ ] \textbf{최종 장기 균형(New LRE) 결론:}
    \begin{itemize}
        \item [ ] 진입/퇴출이 멈추는 지점($\Pi=0$, 즉 $P = \min ATC$)을 확인했는가?
        \item [ ] 최초 가격($P_0$)과 최종 가격($P_2$)을 비교했는가?
        \item [ ] 최초 시장 수량($Q_0$)과 최종 시장 수량($Q_2$)을 비교했는가?
        \item [ ] 최초 기업 수량($q_0^*$)과 최종 기업 수량($q_2^*$)을 비교했는가?
    \end{itemize}
\end{enumerate}

\section{주요 Q\&A}

\begin{tcolorbox}{Q1: 장기 균형에서 "이윤($\Pi$) = 0" 이라는 것은 기업이 돈을 못 번다는 뜻인가요?}
\textbf{A: 아닙니다.} 이는 \textbf{경제적 이윤(Economic Profit)이 0}이라는 의미입니다.

경제적 이윤 = 회계상 이윤(Accounting Profit) - 기회비용(Opportunity Cost)

경제적 이윤이 0이라는 것은, 이 기업이 회계상으로는 \$20M의 큰 이익을 냈더라도, 이 기업이 할 수 있었던 차선책(기회비용) 역시 \$20M의 가치가 있다는 뜻입니다.
즉, \textbf{"다른 산업에 갔어도 딱 이만큼 벌 수 있었다"}는 의미이며, 따라서 굳이 이 산업을 떠나거나(퇴출) 다른 기업이 이 산업에 들어올(진입) 유인이 없는 \textbf{안정된 균형 상태}를 의미합니다.
\end{tcolorbox}

\begin{tcolorbox}{Q2: 개별 기업의 공급 곡선은 정확히 무엇인가요?}
\textbf{A: 한계비용(MC) 곡선입니다.}
더 정확하게는, \textbf{평균가변비용(AVC) 곡선의 최저점보다 위에 있는 MC 곡선 부분}입니다.
기업은 가격(P)이 AVC보다 낮으면 아예 생산을 중단(shutdown)하기 때문입니다.
\end{tcolorbox}

\begin{tcolorbox}{Q3: 주가는 왜 그렇게 변동이 심한가요?}
\textbf{A: 주가는 현재 실적이 아니라 "미래 이윤에 대한 기대($\Pi_t$)"의 현재가치 합이기 때문입니다.}
새로운 정보(뉴스, 스캔들, 신기술 발표 등)가 나올 때마다 시장 참여자들은 \textbf{미래의 기대 이윤($\Pi_t$)}을 즉각적으로 재조정합니다. 이 기대치가 변하면 주가 공식의 분자 값이 변하므로 주가가 실시간으로 변동합니다.
\end{tcolorbox}

\newpage
\section{빠르게 훑어보기 (1-Page Summary)}

\begin{mybox}{1. 완전 경쟁 시장의 4가지 조건}
\begin{enumerate}
    \item 수많은 소규모 구매자와 판매자 (No market power)
    \item 동질적 제품 (Identical products)
    \item 자유로운 진입과 퇴출 (No barriers to entry/exit)
    \item 완전한 정보 (Perfect information)
\end{enumerate}
\textbf{결론: 모든 기업은 가격 수용자(Price Taker)이다. ($P = MR$)}
\end{mybox}

\begin{mybox}{2. 공급 곡선의 도출}
\begin{itemize}
    \item \textbf{개별 기업 공급 곡선 (Firm Supply):}
    \begin{itemize}
        \item 이윤 극대화: $MR = MC$
        \item 완전 경쟁: $P = MR$
        \item $\to$ 기업은 $P = MC$ 에서 생산량($q$) 결정.
        \item $\to$ \textbf{공급 곡선 = $MC$ 곡선 (단, $P \ge \min AVC$)}
    \end{itemize}
    \item \textbf{시장 공급 곡선 (Market Supply):}
    \begin{itemize}
        \item \textbf{시장 공급 ($S$) = $\sum_{i=1}^{N} MC_i$} (모든 개별 기업 MC의 수평 합)
    \end{itemize}
\end{itemize}
\end{mybox}

\begin{summarybox}{3. 단기 균형 (SRE) vs 장기 균형 (LRE)}
\begin{tabularx}{\textwidth}{@{}l|X|X@{}}
\toprule
\textbf{조건} & \textbf{단기 균형 (SRE)} & \textbf{장기 균형 (LRE)} \\
\midrule
\textbf{1. 시장} & $D = S$ (시장 청산) & $D = S$ (시장 청산) \\
\midrule
\textbf{2. 기업} & $MR = MC$ (이윤 극대화) & $MR = MC$ (이윤 극대화) \\
\midrule
\textbf{3. 이윤} & $\Pi$는 양수($>$), 0, 음수($<$) \textbf{모두 가능} & \textbf{반드시 $\Pi = 0$} (진입/퇴출 유인 없음) \\
\bottomrule
\end{tabularx}
\end{summarybox}

\begin{warningbox}{4. 장기 균형으로의 조정 메커니즘}
\begin{itemize}
    \item \textbf{상황: $\Pi > 0$ (초과 이윤)}
    \begin{itemize}
        \item $\to$ 신규 기업 \textbf{진입 (Entry)}
        \item $\to$ 시장 공급($S$) 곡선 \textbf{우측 이동}
        \item $\to$ 시장 가격($P$) \textbf{하락}
        \item $\to$ $\Pi = 0$ 될 때까지 반복
    \end{itemize}
    \item \textbf{상황: $\Pi < 0$ (손실)}
    \begin{itemize}
        \item $\to$ 기존 기업 \textbf{퇴출 (Exit)}
        \item $\to$ 시장 공급($S$) 곡선 \textbf{좌측 이동}
        \item $\to$ 시장 가격($P$) \textbf{상승}
        \item $\to$ $\Pi = 0$ 될 때까지 반복
    \end{itemize}
\end{warningbox}


% Auto-added missing environment closes:
\end{itemize}  % Auto-closed

\end{document}
