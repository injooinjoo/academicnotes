%%%%%%%%%%%%%%%%%%%%%%%%%%%%%%%%%%%%%%%%%%%%%%%%%%%%%%%%%%%%%%%%%%%%%%%%%%%%%%%
% Harvard Academic Notes - 통합 마스터 템플릿
% 모든 강의 노트에 적용되는 통일된 스타일
% 버전: 2.1 - 가독성 개선 (선택적 최적화)
% 최종 수정일: 2025-11-17
%%%%%%%%%%%%%%%%%%%%%%%%%%%%%%%%%%%%%%%%%%%%%%%%%%%%%%%%%%%%%%%%%%%%%%%%%%%%%%%

\documentclass[11pt,a4paper]{article}

%========================================================================================
% 기본 패키지
%========================================================================================

% --- 한국어 지원 ---
\usepackage{kotex}

% --- 페이지 레이아웃 ---
\usepackage[top=20mm, bottom=20mm, left=20mm, right=18mm]{geometry}
\usepackage{setspace}
\onehalfspacing                      % 1.5배 줄간격
\setlength{\parskip}{0.5em}          % 문단 간격
\setlength{\parindent}{0pt}          % 들여쓰기 없음

% --- 표 관련 ---
\usepackage{booktabs}              % 고품질 표
\usepackage{tabularx}              % 자동 너비 조절 표
\usepackage{array}                 % 표 컬럼 확장
\usepackage{longtable}             % 여러 페이지 표
\renewcommand{\arraystretch}{1.1}  % 표 행간 조절

%========================================================================================
% 헤더 및 푸터
%========================================================================================

\usepackage{fancyhdr}
\pagestyle{fancy}
\fancyhf{}
\fancyhead[L]{\small\textit{FIN 574: 기업 수준 경제학}}
\fancyhead[R]{\small\textit{Lecture 01}}
\fancyfoot[C]{\thepage}
\renewcommand{\headrulewidth}{0.5pt}
\renewcommand{\footrulewidth}{0.3pt}

% 첫 페이지는 헤더 없음
\fancypagestyle{firstpage}{
    \fancyhf{}
    \fancyfoot[C]{\thepage}
    \renewcommand{\headrulewidth}{0pt}
}

%========================================================================================
% 색상 정의 (파스텔 톤 + 다크모드 호환)
%========================================================================================

\usepackage[dvipsnames]{xcolor}

% 밝은 배경용 파스텔 색상
\definecolor{lightblue}{RGB}{220, 235, 255}      % 부드러운 파랑
\definecolor{lightgreen}{RGB}{220, 255, 235}     % 부드러운 초록
\definecolor{lightyellow}{RGB}{255, 250, 220}    % 부드러운 노랑
\definecolor{lightpurple}{RGB}{240, 230, 255}    % 부드러운 보라
\definecolor{lightgray}{gray}{0.95}              % 밝은 회색
\definecolor{lightpink}{RGB}{255, 235, 245}      % 부드러운 핑크
\definecolor{boxgray}{gray}{0.95}
\definecolor{boxblue}{rgb}{0.9, 0.95, 1.0}
\definecolor{boxred}{rgb}{1.0, 0.95, 0.95}

% 진한 색상 (테두리/제목용)
\definecolor{darkblue}{RGB}{50, 80, 150}
\definecolor{darkgreen}{RGB}{40, 120, 70}
\definecolor{darkorange}{RGB}{200, 100, 30}
\definecolor{darkpurple}{RGB}{100, 60, 150}

%========================================================================================
% 박스 환경 (tcolorbox) - 6가지 타입
%========================================================================================

\usepackage[most]{tcolorbox}
\tcbuselibrary{skins, breakable}

% 1. 개요 박스 (강의 시작 부분)
\newtcolorbox{overviewbox}[1][]{
    enhanced,
    colback=lightpurple,
    colframe=darkpurple,
    fonttitle=\bfseries\large,
    title=📚 강의 개요,
    arc=3mm,
    boxrule=1pt,
    left=8pt,
    right=8pt,
    top=8pt,
    bottom=8pt,
    breakable,
    #1
}

% 2. 요약 박스
\newtcolorbox{summarybox}[1][]{
    enhanced,
    colback=lightblue,
    colframe=darkblue,
    fonttitle=\bfseries,
    title=📝 핵심 요약,
    arc=2mm,
    boxrule=0.7pt,
    left=6pt,
    right=6pt,
    top=6pt,
    bottom=6pt,
    breakable,
    #1
}

% 3. 핵심 정보 박스
\newtcolorbox{infobox}[1][]{
    enhanced,
    colback=lightgreen,
    colframe=darkgreen,
    fonttitle=\bfseries,
    title=💡 핵심 정보,
    arc=2mm,
    boxrule=0.7pt,
    left=6pt,
    right=6pt,
    top=6pt,
    bottom=6pt,
    breakable,
    #1
}

% 4. 주의사항 박스
\newtcolorbox{warningbox}[1][]{
    enhanced,
    colback=lightyellow,
    colframe=darkorange,
    fonttitle=\bfseries,
    title=⚠️ 주의사항,
    arc=2mm,
    boxrule=0.7pt,
    left=6pt,
    right=6pt,
    top=6pt,
    bottom=6pt,
    breakable,
    #1
}

% 5. 예제 박스
\newtcolorbox{examplebox}[1][]{
    enhanced,
    colback=lightgray,
    colframe=black!60,
    fonttitle=\bfseries,
    title=📖 예제: #1,
    arc=2mm,
    boxrule=0.7pt,
    left=6pt,
    right=6pt,
    top=6pt,
    bottom=6pt,
    breakable,
}

% 6. 정의 박스
\newtcolorbox{definitionbox}[1][]{
    enhanced,
    colback=lightpink,
    colframe=purple!70!black,
    fonttitle=\bfseries,
    title=📌 정의: #1,
    arc=2mm,
    boxrule=0.7pt,
    left=6pt,
    right=6pt,
    top=6pt,
    bottom=6pt,
    breakable,
}

% 7. 중요 박스 (importantbox - warningbox와 유사)
\newtcolorbox{importantbox}[1][]{
    enhanced,
    colback=boxred,
    colframe=red!70!black,
    fonttitle=\bfseries,
    title=⚠️ 매우 중요: #1,
    arc=2mm,
    boxrule=0.7pt,
    left=6pt,
    right=6pt,
    top=6pt,
    bottom=6pt,
    breakable,
}

% 8. cautionbox (warningbox와 동일)
\let\cautionbox\warningbox
\let\endcautionbox\endwarningbox

%========================================================================================
% 코드 블록 설정 (밝은 배경)
%========================================================================================

\usepackage{listings}

\definecolor{codegray}{rgb}{0.5,0.5,0.5}
\definecolor{codepurple}{rgb}{0.58,0,0.82}
\definecolor{backcolour}{rgb}{0.95,0.95,0.95}

\lstset{
    basicstyle=\ttfamily\small,
    backgroundcolor=\color{lightgray},
    keywordstyle=\color{darkblue}\bfseries,
    commentstyle=\color{darkgreen}\itshape,
    stringstyle=\color{purple!80!black},
    numberstyle=\tiny\color{black!60},
    numbers=left,
    numbersep=8pt,
    breaklines=true,
    breakatwhitespace=false,
    frame=single,
    frameround=tttt,
    rulecolor=\color{black!30},
    captionpos=b,
    showstringspaces=false,
    tabsize=2,
    xleftmargin=15pt,
    xrightmargin=5pt,
    escapeinside={\%*}{*)}
}

% Python 코드 스타일
\lstdefinestyle{pythonstyle}{
    language=Python,
    morekeywords={self, True, False, None},
}

% SQL 코드 스타일
\lstdefinestyle{sqlstyle}{
    language=SQL,
    morekeywords={SELECT, FROM, WHERE, JOIN, GROUP, BY, ORDER, HAVING},
}

%========================================================================================
% 목차 스타일링
%========================================================================================

\usepackage{tocloft}
\renewcommand{\cftsecleader}{\cftdotfill{\cftdotsep}}
\setlength{\cftbeforesecskip}{0.4em}
\renewcommand{\cftsecfont}{\bfseries}
\renewcommand{\cftsubsecfont}{\normalfont}

%========================================================================================
% 표 및 그림
%========================================================================================

\usepackage{graphicx}              % 이미지
\usepackage{adjustbox}             % 표/박스 크기 조절

% 표 캡션 스타일
\usepackage{caption}
\captionsetup[table]{
    labelfont=bf,
    textfont=it,
    skip=5pt
}
\captionsetup[figure]{
    labelfont=bf,
    textfont=it,
    skip=5pt
}

%========================================================================================
% 수학
%========================================================================================

\usepackage{amsmath, amssymb, amsthm}

% 정리 환경
\theoremstyle{definition}
\newtheorem{theorem}{정리}[section]
\newtheorem{lemma}[theorem]{보조정리}
\newtheorem{proposition}[theorem]{명제}
\newtheorem{corollary}[theorem]{따름정리}
\newtheorem{definition}{정의}[section]
\newtheorem{example}{예제}[section]

%========================================================================================
% 하이퍼링크
%========================================================================================

\usepackage[
    colorlinks=true,
    linkcolor=blue!80!black,
    urlcolor=blue!80!black,
    citecolor=green!60!black,
    bookmarks=true,
    bookmarksnumbered=true,
    pdfborder={0 0 0}
]{hyperref}

% PDF 메타데이터는 각 문서에서 설정
\hypersetup{
    pdftitle={FIN 574: 기업 수준 경제학 - Lecture 01},
    pdfauthor={강의 노트},
    pdfsubject={Academic Notes}
}

%========================================================================================
% 기타 유용한 패키지
%========================================================================================

\usepackage{enumitem}              % 리스트 커스터마이징
\setlist{nosep, leftmargin=*, itemsep=0.3em}

\usepackage{microtype}             % 타이포그래피 개선
\usepackage{footnote}              % 각주 개선
\usepackage{url}                   % URL 줄바꿈
\urlstyle{same}

%========================================================================================
% 사용자 정의 명령어
%========================================================================================

% 강조 텍스트
\newcommand{\important}[1]{\textbf{\textcolor{red!70!black}{#1}}}
\newcommand{\keyword}[1]{\textbf{#1}}
\newcommand{\term}[1]{\textit{#1}}
\newcommand{\code}[1]{\texttt{#1}}

% 용어 설명 (인라인)
\newcommand{\defterm}[2]{\textbf{#1}\footnote{#2}}

% 섹션 시작 전 페이지 분리
\newcommand{\newsection}[1]{\newpage\section{#1}}

%========================================================================================
% 문서 제목 스타일
%========================================================================================

\usepackage{titling}
\pretitle{\begin{center}\LARGE\bfseries}
\posttitle{\par\end{center}\vskip 0.5em}
\preauthor{\begin{center}\large}
\postauthor{\end{center}}
\predate{\begin{center}\large}
\postdate{\par\end{center}}

%========================================================================================
% 섹션 제목 간격
%========================================================================================

\usepackage{titlesec}
\titlespacing*{\section}{0pt}{1.5em}{0.8em}
\titlespacing*{\subsection}{0pt}{1.2em}{0.6em}
\titlespacing*{\subsubsection}{0pt}{1em}{0.5em}

%========================================================================================
% 메타 정보 박스 명령어
%========================================================================================

\newcommand{\metainfo}[4]{
\begin{tcolorbox}[
    colback=lightpurple,
    colframe=darkpurple,
    boxrule=1pt,
    arc=2mm,
    left=10pt,
    right=10pt,
    top=8pt,
    bottom=8pt
]
\begin{tabular}{@{}rl@{}}
▣ \textbf{강의명:} & #1 \\[0.3em]
▣ \textbf{주차:} & #2 \\[0.3em]
▣ \textbf{교수명:} & #3 \\[0.3em]
▣ \textbf{목적:} & \begin{minipage}[t]{0.75\textwidth}#4\end{minipage}
\end{tabular}
\end{tcolorbox}
}

%========================================================================================
% 끝
%========================================================================================


\begin{document}

\maketitle
\thispagestyle{firstpage}

\metainfo{FIN 574: 기업 수준 경제학}{Lecture 01}{UIUC Faculty}{Lecture 01의 핵심 개념 학습}


\begin{summarybox}
이 문서는 기업 수준 경제학의 첫 번째 모듈인 '소비자 및 생산자 행동'의 핵심 내용을 정리합니다. 경제학의 근본적인 문제인 \textbf{희소성(scarcity)}에서 출발하여, 자원이 배분되는 방식으로서의 \textbf{시장(market)}과 \textbf{가격(price)}의 역할을 탐구합니다. \textbf{수요(demand)}와 \textbf{공급(supply)} 곡선의 도출 원리와 시장 \textbf{균형(equilibrium)}이 결정되는 과정을 학습합니다. 특히, '곡선 상의 이동'과 '곡선 자체의 이동'을 구분하는 것은 시장 변화를 분석하는 데 가장 중요한 핵심입니다.
\end{summarybox}

\tableofcontents

\newpage

\section{학습 로드맵 (Learning Roadmap)}

이 문서는 다음과 같은 논리적 흐름으로 구성되어 있습니다. 이 순서대로 개념을 정복해 나가는 것이 좋습니다.

\begin{enumerate}
    \item \textbf{경제학의 근본 문제 (희소성)}: 왜 경제학이 필요한지 이해합니다. "세상에 공짜는 없다"는 말의 진짜 의미를 배웁니다.
    \item \textbf{핵심 의사결정 도구 (기회비용)}: 모든 선택에 따르는 보이지 않는 비용, 즉 기회비용(opportunity cost)을 계산하는 법을 배웁니다.
    \item \textbf{구매자의 행동 (수요 곡선)}: 가격이 변할 때 소비자가 어떻게 반응하는지를 나타내는 수요 곡선(demand curve)을 이해합니다.
    \item \textbf{판매자의 행동 (공급 곡선)}: 가격이 변할 때 기업(생산자)이 어떻게 반응하는지를 나타내는 공급 곡선(supply curve)을 이해합니다.
    \item \textbf{시장의 작동 (균형)}: 구매자와 판매자가 만나 가격이 결정되고 거래가 이루어지는 '시장 균형(market equilibrium)' 지점을 찾습니다.
    \item \textbf{시장 변화 분석 (이동과 변동)}: 이 모듈의 최종 목표입니다. 외부 충격(뉴스, 정책, 재해 등)이 발생했을 때, 시장 가격과 거래량이 어떻게 변하는지 예측합니다.
\end{enumerate}

\section{핵심 용어 정리 (Key Terminology)}

경제학은 용어의 정확한 정의에서 시작합니다. 특히 '수요'와 '수요량'의 구분은 매우 중요합니다.

\begin{table}[h!]
\centering
\caption{모듈 1 핵심 용어}
\label{tab:key_terms}
\small
\begin{tabularx}{\textwidth}{>{\raggedright\arraybackslash}p{2.5cm} >{\raggedright\arraybackslash}X >{\raggedright\arraybackslash}p{3cm}}
\toprule
\textbf{용어} & \textbf{쉬운 설명} & \textbf{원어 (Eng)} \\
\midrule
\textbf{희소성} & 단순히 '드문 것'이 아니라, '사람들의 욕구에 비해 원하는 만큼 다 가질 수 없는 상태'를 의미합니다. & Scarcity \\
\textbf{기회비용} & 어떤 것을 선택했을 때 포기해야 하는 '차선책(next best alternative)의 가치'. (암묵적 비용 포함) & Opportunity Cost \\
\textbf{수요 (Demand)} & \textbf{모든} 가능한 가격 수준에서 소비자가 구매하려는 의사와 능력을 나타내는 \textbf{관계 전체 (곡선 자체)}. & Demand \\
\textbf{수요량} & \textbf{특정한} 가격 수준에서 소비자가 구매하려는 \textbf{구체적인 양 (곡선 위의 한 점)}. & Quantity Demanded \\
\textbf{공급 (Supply)} & \textbf{모든} 가능한 가격 수준에서 생산자가 판매하려는 의사와 능력을 나타내는 \textbf{관계 전체 (곡선 자체)}. & Supply \\
\textbf{공급량} & \textbf{특정한} 가격 수준에서 생산자가 판매하려는 \textbf{구체적인 양 (곡선 위의 한 점)}. & Quantity Supplied \\
\textbf{시장 균형} & 수요량과 공급량이 정확히 일치하여, 가격이나 거래량이 변할 유인이 없는 안정된 상태. (수요=공급) & Market Equilibrium \\
\textbf{초과 공급 (잉여)} & 가격이 너무 높아 공급량이 수요량을 초과하는 상태. (재고가 쌓임) & Surplus \\
\textbf{초과 수요 (부족)} & 가격이 너무 낮아 수요량이 공급량을 초과하는 상태. (줄을 서야 함) & Shortage \\
\bottomrule
\end{tabularx}
\end{table}

\newpage

\section{핵심 개념 1: 희소성과 기회비용 (Scarcity and Opportunity Cost)}

\subsection{희소성 (Scarcity)과 배분 메커니즘}

경제학은 \textbf{희소성(scarcity)}의 문제를 해결하는 학문입니다.
희소성이란 "모두가 원하는 만큼 가질 수 없는 상태(not enough to go around)"를 의미합니다.

\begin{itemize}
    \item 희소성은 단순히 '희귀함(rare)'과는 다릅니다. 예를 들어, 레오나르도 다빈치의 '모나리자'는 희귀하지만($\approx$희소함), 공기 중의 '깨끗한 공기'는 흔해 보이지만 희소합니다. (모두가 원하는 만큼의 깨끗한 공기를 가질 수 없기 때문)
    \item 희소성 때문에 우리는 필연적으로 '선택'을 해야 하며, "누가 얼마나 가질 것인가"를 결정할 규칙, 즉 \textbf{배분 메커니즘(Allocation Mechanism)}이 필요합니다.
    \item 과거에는 왕이나 정부가 모든 것을 결정했습니다(명령 경제, Command Economy).
    \item 현대 서구 경제에서는 주로 \textbf{시장(Market)}이 이 역할을 합니다.
\end{itemize}

\begin{mybox}{사례: 1990년의 RAM 가격과 시장의 힘}
    1990년, Compaq 컴퓨터의 4MB RAM 가격은 약 \$2,600였습니다. 이는 당시 매우 비싼 가격이었습니다. 하지만 오늘날 512MB (4MB의 128배) RAM은 \$4.59에 불과합니다.

    이러한 극적인 가격 하락은 정부의 계획 때문이 아니었습니다.
    높은 가격($P \uparrow$)을 본 실리콘 밸리의 많은 기업가들이 "저 비용으로 나도 만들 수 있겠다"고 판단하여 시장에 진입(Entry)했습니다.
    그 결과, 공급자(producers)가 많아지면서 경쟁이 발생했고, \textbf{공급이 증가($S \uparrow$)}하여 \textbf{가격이 하락($P \downarrow$)}했습니다.
    이것이 바로 시장이 작동하는 방식입니다.
\end{mybox}

\subsection{가격의 역할: 배급 장치 (Rationing Device)}

시장은 \textbf{가격(Price)}이라는 도구를 사용하여 희소한 자원을 배분합니다.
가격은 단순히 '비용'이 아니라, 누가 그 재화를 가질 자격이 있는지를 결정하는 \textbf{배급 장치(Rationing Device)}입니다.

\begin{itemize}
    \item \textbf{가격 상승 ($\uparrow$)}: 재화가 더 희소해지면(e.g., 공급 감소), 가격이 오릅니다.
        \begin{itemize}
            \item 일부 소비자는 "이 가격을 낼 가치는 없다"며 구매를 포기합니다 (수요량 감소).
            \item 일부 생산자는 "수익성이 좋다"며 시장에 진입합니다 (공급량 증가).
        \end{itemize}
    \item \textbf{가격 하락 ($\downarrow$)}: 재화가 덜 희소해지면(e.g., 공급 증가), 가격이 내립니다.
        \begin{itemize}
            \item 더 많은 소비자가 "이 가격이면 살 만하다"며 구매를 시작합니다 (수요량 증가).
            \item 일부 생산자는 "수익성이 나쁘다"며 시장을 떠납니다 (공급량 감소).
        \end{itemize}
\end{itemize}

\begin{warningbox}
\textbf{만약 가격이 없다면? $\rightarrow$ 대기 (Queuing)}

미국 듀크(Duke) 대학의 농구팀 티켓은 학생들에게 \$0 (무료)입니다.
    하지만 경기장은 10,000석 미만으로 매우 작아(희소성), 모든 학생이 들어갈 수 없습니다.

    가격을 \$0로 정한다고 해서 희소성이 사라지진 않습니다.
    단지 배분 메커니즘이 '가격'에서 '대기(Queuing)'로 바뀌었을 뿐입니다. 학생들은 티켓을 얻기 위해 몇 주 동안 텐트에서 생활('Krzyzewskiville')해야 합니다.
    즉, '돈' 대신 '시간'이라는 엄청난 비용을 지불하는 것입니다.
\end{warningbox}

\subsection{기회비용 (Opportunity Cost)}

모든 선택에는 비용이 따릅니다. 경제학에서 말하는 '진짜' 비용은 \textbf{기회비용(Opportunity Cost)}입니다.

\textbf{정의}: "어떤 것을 선택함으로써 포기해야 하는 것 중 가장 가치가 큰 대안 (The value of the next best foregone alternative)"

\begin{itemize}
    \item 노벨상 수상자인 밀턴 프리드먼(Milton Friedman)은 "세상에 공짜 점심은 없다(There's no such thing as a free lunch)"고 말했습니다.
    \item 누군가 당신에게 공짜로 점심을 사주더라도, 당신은 그 시간에 친구를 만나거나 넷플릭스를 볼 수 있는 '시간'을 포기해야 합니다. 이 시간이 바로 기회비용입니다.
\end{itemize}

\begin{keypointbox}{핵심 예시: 명시적 비용 vs. 암묵적 비용(기회비용)}
    당신이 친구와 함께 차를 사려 한다고 가정합시다.

    \begin{itemize}
        \item \textbf{당신 (명시적 비용)}: 돈이 없어 은행에서 \$3,000를 대출받고 이자를 냅니다.
        \item \textbf{친구 (암묵적 비용)}: 모아둔 저축 \$3,000를 인출해서 차를 샀습니다. 친구는 "나는 이자를 내지 않으니 공짜다"라고 주장합니다.
    \end{itemize}

    \textbf{경제학적 분석}: 친구의 주장은 틀렸습니다.
    친구가 만약 그 돈 \$3,000를 은행에 계속 저축했다면 이자를 받을 수 있었을 것입니다.
    차를 사기 위해 돈을 인출하는 순간, 그 '받을 수 있었던 이자'를 포기한 것입니다.
    이것이 바로 눈에 보이지 않는 \textbf{암묵적 비용(Implicit Cost)}이며, 기회비용의 핵심입니다.

    \textbf{기업의 적용}: 회계 장부상 \$3백만의 '이익(Profit)'을 낸 CEO라도, 경제학적으로는 다음과 같이 생각해야 합니다. "내가 이 자산을 모두 팔고 트윙키(Twinkies)를 만들었다면 \$4백만을 벌 수 있지 않았을까?" 만약 그렇다면, 이 기업은 회계상 흑자임에도 불구하고 경제학적으로는 \$1백만의 손실을 본 것입니다.
\end{keypointbox}


\newpage

\section{핵심 개념 2: 수요 곡선 (The Demand Curve)}

시장은 \textbf{구매자(Buyers)}와 \textbf{판매자(Sellers)}의 상호작용으로 이루어집니다.
구매자들의 행동을 요약한 것이 바로 \textbf{수요 곡선(Demand Curve)}입니다. (이 분석은 정부 개입이 없다고 가정합니다.)

\subsection{수요(Demand)의 정의}

\textbf{정의}: "다른 모든 조건이 일정할 때, 각각의 가격 수준에서 소비자들이 구매하고자 하는 재화의 양을 나타내는 점들의 궤적 (Locus of points showing how much consumers wish to purchase at different prices)"

\begin{itemize}
    \item 수학적으로는 $Q = f(P)$ 로 표현할 수 있습니다. 즉, 구매하려는 수량(Quantity, Q)이 가격(Price, P)에 따라 결정됩니다.
    \item 수요 곡선은 \textbf{우하향(Downward sloping)}합니다.
    \item \textbf{이유 (수요의 법칙, Law of Demand)}:
        \begin{itemize}
            \item 가격이 하락하면($P \downarrow$), 더 적은 '녹색 종이(돈)'를 포기해도 되므로(기회비용 감소), 더 많이 구매합니다($Q \uparrow$).
            \item 가격이 상승하면($P \uparrow$), 더 많은 것을 포기해야 하므로(기회비용 증가), 더 적게 구매합니다($Q \downarrow$).
        \end{itemize}
\end{itemize}

\subsection{수요 곡선의 형태와 관행}

\begin{keypointbox}{주의: 축(Axis)이 반대로 되어 있습니다}
    수학적으로 $Q = f(P)$ 라면, 독립 변수인 가격(P)이 X축에, 종속 변수인 수량(Q)이 Y축에 있어야 합니다.

    하지만 100여 년 전 경제학자 알프레드 마샬(Alfred Marshall)이 처음 그래프를 그릴 때, 가격(P)을 Y축(세로축)에, 수량(Q)을 X축(가로축)에 두었습니다.
    이것이 관행으로 굳어져, 오늘날 모든 경제학 교과서가 이 방식을 따릅니다.

    엄밀히 말해 우리가 그리는 것은 $P = f^{-1}(Q)$ 인 \textbf{역수요함수(Inverse Demand Curve)}이지만, 관습적으로 그냥 "수요 곡선"이라고 부릅니다. 이 관행에 익숙해져야 합니다.
\end{keypointbox}

\subsection{선형(Linear)으로 그리는 이유}

현실 세계의 수요 곡선은 복잡한 곡선(non-linear)일 것입니다.
하지만 우리는 이 수업에서 수요 곡선을 \textbf{직선(Linear)}으로 단순화하여 그립니다.

\begin{itemize}
    \item \textbf{이유}: 분석의 편의성(Tractability) 때문입니다.
    \item 직선으로 그려도 "가격이 오르면 수요량이 줄어든다"는 핵심 직관(Intuition)은 동일하게 얻을 수 있습니다.
    \item 곡선으로 그리면 시장 균형을 찾기 위해 복잡한 이차방정식의 근의 공식(quadratic rule) 등을 사용해야 하지만, 직선으로 그리면 간단한 연립방정식으로 해결할 수 있습니다.
\end{itemize}

\subsection{실제 수요 곡선은 어디에서 오는가?}

현실의 기업들은 수요 곡선을 어떻게 추정할까요?

\begin{itemize}
    \item \textbf{불확실성과 기대값}: 기업이 특정 가격($P_0$)을 책정할 때, 실제로 팔릴 양은 불확실합니다. 통계적으로 이는 '정규 분포(Normal Distribution)' 또는 '벨 커브(Bell Curve)'를 따릅니다.
    \item \textbf{수요 곡선의 진짜 의미}: 우리가 그리는 수요 곡선 상의 한 점($P_0, Q_0$)은, 사실 해당 가격($P_0$)에서 팔릴 것으로 예상되는 \textbf{기대값(Expected Value, $\mu$)}, 즉 벨 커브의 가장 높은 지점을 의미합니다.
    \item \textbf{데이터 수집 (Big Data)}: 기업들은 '빅데이터'를 활용해 이 기대값을 추정합니다.
        \begin{enumerate}
            \item \textbf{고객 로열티 카드}: 식료품점은 고객 카드를 통해 '가격이 할인될 때(P$\downarrow$) 래리 교수가 마요네즈를 얼마나 더 사는지(Q$\uparrow$)' 데이터를 수집합니다.
            \item \textbf{데이터 결합}: 이 구매 데이터를 Zillow(주택 가격 $\to$ 부의 대리 변수), 인구조사 데이터(가구 구성, 자녀 수) 등과 결합합니다.
            \item \textbf{결과}: 통계 분석을 통해 '마요네즈'나 '냉동 피자'에 대한 정교한 수요 곡선을 추정할 수 있습니다.
        \end{enumerate}
\end{itemize}

\newpage

\section{핵심 개념 3: 공급 곡선 (The Supply Curve)}

판매자(Sellers, Firms)의 행동을 요약한 것이 \textbf{공급 곡선(Supply Curve)}입니다.

\subsection{공급(Supply)의 정의}

\textbf{정의}: "다른 모든 조건이 일정할 때, 각각의 가격 수준에서 생산자들이 판매하고자 하는 재화의 양을 나타내는 점들의 궤적 (How much will firms be willing to sell at different prices?)"

\begin{itemize}
    \item 수학적으로는 $Q = h(P)$ 로 표현할 수 있습니다.
    \item 공급 곡선은 \textbf{우상향(Upward sloping)}합니다.
    \item \textbf{이유}:
        \begin{itemize}
            \item 생산량을 늘리기 위해서는($Q \uparrow$) 추가 노동자를 고용하거나 공장을 더 돌려야 하므로 \textbf{추가 생산 비용이 더 많이 듭니다}.
            \item 기업이 이처럼 더 비싼 비용을 감수하고 생산량을 늘리도록 유인하려면, 시장에서 더 높은 가격($P \uparrow$)을 보장받아야 합니다.
            \item (이 직관적인 이유는 이후 모듈에서 '이윤 극대화', '생산 함수' 등을 통해 엄밀하게 증명될 것입니다.)
        \end{itemize}
    \item 공급 곡선 역시 수요 곡선과 마찬가지로, 분석의 편의를 위해 \textbf{선형(Linear)}으로 가정합니다.
\end{itemize}

\newpage

\section{핵심 개념 4: 시장 균형 (Market Equilibrium)}

수요 곡선(구매자)과 공급 곡선(판매자)을 한 그래프에 합치면, 시장이 어떻게 작동하는지 알 수 있습니다.

\subsection{균형(Equilibrium)의 정의}

\textbf{정의}: "더 이상 변화할 유인이 없는 안정된 상태 (No tendency for change)"

\begin{itemize}
    \item \textbf{비유 (Cereal Bowl)}: 시리얼 그릇 바닥에 구슬이 멈춰 있는 상태가 '균형'입니다.
    \item 만약 누군가 그릇을 쳐서(외부 충격, Shock) 구슬이 흔들리더라도, 구슬은 결국 그릇 바닥의 새로운 지점에서 다시 멈출 것입니다. 이것이 '새로운 균형'입니다.
    \item 경제학자의 목표는 (1) 현재의 균형을 찾고, (2) 충격이 왔을 때, (3) 시장이 어떤 혼란을 거쳐 (4) 어떤 '새로운 균형'으로 이동할지 예측하는 것입니다.
\end{itemize}

\subsection{균형점의 결정}

\textbf{균형점(E)}: 수요 곡선($D_0$)과 공급 곡선($S_0$)이 교차하는 지점입니다.

\begin{itemize}
    \item \textbf{균형 가격 ($P_E$)}: 이 교차점의 가격입니다.
    \item \textbf{균형 거래량 ($Q_E$)}: 이 교차점의 수량입니다.
    \item \textbf{균형의 의미}: $P_E$ 가격에서는 소비자가 사고 싶어 하는 양(\textbf{수요량, $Q_D$})과 생산자가 팔고 싶어 하는 양(\textbf{공급량, $Q_S$})이 정확히 일치합니다 ($Q_D = Q_S$).
    \item 이 상태에서는 재고가 쌓이거나 물건이 부족해 줄을 서는 사람이 없습니다. 모든 시장 참여자(구매자, 판매자)가 만족하므로 가격이 변할 유인이 없습니다.
\end{itemize}

\subsection{반증법(Proof by Contradiction)을 통한 균형 증명}

만약 $P_E$가 균형 가격이 아니라면 어떤 일이 벌어질까요?

\begin{mybox}{상황 1: 가격이 균형보다 높을 때 ($P_1 > P_E$)}
    \begin{itemize}
        \item \textbf{소비자 반응}: 가격이 비싸므로 조금만 사려 합니다 (수요량 $Q_D$는 적음).
        \item \textbf{생산자 반응}: 가격이 높으므로 신나서 많이 생산합니다 (공급량 $Q_S$는 많음).
        \item \textbf{결과}: $Q_S > Q_D$. 시장에 물건이 팔리지 않고 쌓입니다. $\rightarrow$ \textbf{초과 공급 (Surplus)}
        \item \textbf{시장의 조정}: 재고를 없애기 위해 기업들은 가격을 \textbf{인하($\downarrow$)}하기 시작합니다. 가격은 $P_E$를 향해 내려갑니다.
        \item \textbf{결론}: $P_1$은 균형이 될 수 없습니다.
    \end{itemize}
\end{mybox}

\begin{mybox}{상황 2: 가격이 균형보다 낮을 때 ($P_2 < P_E$)}
    \begin{itemize}
        \item \textbf{소비자 반응}: 가격이 싸므로 많이 사려 합니다 (수요량 $Q_D$는 많음).
        \item \textbf{생산자 반응}: 가격이 낮으므로 수익성이 나빠 조금만 생산합니다 (공급량 $Q_S$는 적음).
        \item \textbf{결과}: $Q_D > Q_S$. 물건이 부족해서 사지 못하는 사람이 생깁니다. $\rightarrow$ \textbf{초과 수요 (Shortage)}
        \item \textbf{시장의 조정}: 소비자들은 물건을 사기 위해 "내가 돈을 더 낼게"라며 경쟁합니다. 가격은 \textbf{인상($\uparrow$)}되기 시작합니다. 가격은 $P_E$를 향해 올라갑니다.
        \item \textbf{결론}: $P_2$는 균형이 될 수 없습니다.
    \end{itemize}
\end{mybox}

오직 $P_E$에서만 가격이 오르거나 내릴 압력이 없는 유일한 '균형' 상태가 됩니다.

\newpage

\section{절차/방법론: 시장 변화 분석 (Analyzing Market Changes)}

현실 세계의 뉴스(날씨, 정책, 기술)가 시장에 어떤 영향을 미치는지 분석하는 것은 이 모듈의 최종 목표입니다.
이를 위해서는 \textbf{두 가지 용어를 명확히 구별}해야 합니다. 이것이 이 모듈에서 가장 중요한 핵심입니다.

\subsection{가장 중요한 구분: '곡선 자체의 이동' vs '곡선 상의 이동'}

\begin{warningbox}{치명적인 용어 혼동 (Jargon Warning)}
    "수요가 감소했다"와 "수요량이 감소했다"는 완전히 다른 말이며, 혼용할 경우 0점 처리될 수 있습니다.
\end{warningbox}

\begin{table}[h!]
\centering
\caption{수요/공급 변화 용어 비교}
\label{tab:shift_vs_move}
\begin{tabularx}{\textwidth}{>{\raggedright}p{3.5cm} >{\raggedright}X >{\raggedright}X}
\toprule
\textbf{구분} & \textbf{...량(Quantity)의 변화} & \textbf{...자체의 변화} \\
\midrule
\textbf{용어 (Jargon)} & $\bullet$ \textbf{수요량}의 변화 \newline $\bullet$ \textbf{공급량}의 변화 & $\bullet$ \textbf{수요}의 변화 \newline $\bullet$ \textbf{공급}의 변화 \\
\textbf{그래프 상의 의미} & $\bullet$ \textbf{곡선 상에서의 이동} \newline (Movement along the curve) & $\bullet$ \textbf{곡선 자체의 이동} \newline (Shift of the entire curve) \\
\textbf{유일한 원인} & $\bullet$ 해당 상품의 \textbf{가격(P) 변화} & $\bullet$ \textbf{가격 이외의} 모든 요인 \newline (e.g., 소득, 유행, 날씨, 기술, 세금) \\
\textbf{예시 (가솔린)} & [나쁜 예 👎] \newline "가솔린 \textbf{가격이 올라서} 가솔린 \textbf{수요가} 줄었다." \newline (틀림!) & [좋은 예 👍] \newline "가솔린 \textbf{가격이 올라서} 가솔린 \textbf{수요량이} 줄었다." \newline (옳음!) \\
\textbf{예시 (토마토)} & [좋은 예 👍] \newline "토마토 \textbf{가격이 하락하자} 토마토 \textbf{수요량이} 증가했다." \newline (곡선 상에서 이동) & [좋은 예 👍] \newline "토마토가 심장병에 좋다는 \textbf{연구 결과}가 나와 토마토 \textbf{수요가} 증가했다." \newline (곡선 자체가 오른쪽으로 이동) \\
\bottomrule
\end{tabularx}
\end{table}

\subsection{분석 프레임워크: 오렌지 주스(O.J.) 시장 예시}

\textbf{질문}: "어느 날 아침 뉴스에서 플로리다 오렌지 농장에 얼음이 1인치 두께로 얼어붙었다는(냉해) 보도가 나왔다. 오렌지 주스(O.J.) 시장의 가격과 거래량은 어떻게 변할까?"

\textbf{분석 단계}:

\begin{enumerate}
    \item \textbf{시작 (Start)}: 시장은 원래 $S_0$ (공급 곡선)와 $D_0$ (수요 곡선)가 만나는 균형점 $(P_0, Q_0)$에 있었습니다.

    \item \textbf{충격 식별 (Shock)}: 플로리다(주요 공급처)의 냉해로 오렌지 작물이 전멸했습니다.
        \begin{itemize}
            \item 이것은 '가격' 변화가 아니라 '생산 조건'의 변화입니다.
            \item 따라서 '공급량'이 아니라 \textbf{'공급(Supply)'}이 변합니다.
        \end{itemize}

    \item \textbf{곡선 이동 (Shift)}: 공급 능력이 파괴되었으므로, 모든 가격 수준에서 판매할 수 있는 O.J.의 양이 줄어듭니다.
        \begin{itemize}
            \item \textbf{공급 곡선이 왼쪽으로 이동합니다. ($S_0 \rightarrow S_1$)}
            \item (수요 곡선($D_0$)은 변하지 않습니다. 사람들의 O.J.에 대한 취향이 변한 것은 아닙니다.)
        \end{itemize}

    \item \textbf{즉각적 효과 (Disequilibrium)}: 공급 곡선이 $S_1$으로 이동한 직후, 시장 가격은 아직 $P_0$에 머물러 있습니다.
        \begin{itemize}
            \item 이 $P_0$ 가격에서 소비자들은 여전히 $Q_0$만큼 사려 하지만 (수요 곡선 $D_0$ 기준),
            \item 생산자들은 이제 $S_1$ 곡선에 따라 훨씬 적은 양만 팔려 합니다.
            \item $Q_D > Q_S$ $\rightarrow$ 시장에 O.J.가 동나는 \textbf{초과 수요 (Shortage)}가 발생합니다.
        \end{itemize}

    \item \textbf{새 균형으로의 이동 (Movement)}:
        \begin{itemize}
            \item 물건이 부족(Shortage)하니 소비자들이 가격을 올려 부르기 시작합니다. \textbf{가격이 상승($P \uparrow$)}합니다.
            \item 가격이 $P_0$에서 $P_1$으로 오르는 동안 \textbf{두 가지 일이 동시에} 일어납니다:
                \begin{enumerate}
                    \item \textbf{(공급)}: 가격이 오르자, 텍사스나 캘리포니아의 남은 공급자들이 "수익성이 좋다"며 공급을 늘립니다. (새로운 $S_1$ 곡선을 \textit{따라} 우상향 \textbf{이동})
                    \item \textbf{(수요)}: 가격이 오르자, 소비자들은 "너무 비싸다"며 구매를 포기하고 다이어트 콜라로 바꿉니다. (원래 $D_0$ 곡선을 \textit{따라} 좌상향 \textbf{이동})
                \end{enumerate}
        \end{itemize}

    \item \textbf{결론 (New Equilibrium)}: 이 과정은 가격이 $P_1$에 도달할 때까지 계속됩니다.
        \begin{itemize}
            \item $P_1$은 $D_0$와 \textbf{새로운 $S_1$}이 만나는 새 균형점입니다.
            \item \textbf{최종 결과}: 원래 균형 $(P_0, Q_0)$과 비교할 때, \textbf{가격은 상승($P_1 > P_0$)}하고 \textbf{거래량은 감소($Q_1 < Q_0$)}합니다.
        \end{itemize}
\end{enumerate}

\newpage

\section{학습 체크리스트 (Checklist)}

이 모듈을 완료한 후, 다음 질문에 스스로 답할 수 있는지 확인하세요.

\begin{itemize}
    \item [ ] '희소성'의 경제학적 정의가 "드물다(rare)"와 어떻게 다른지 설명할 수 있는가?
    \item [ ] "세상에 공짜 점심은 없다"는 말을 '기회비용'을 사용해 설명할 수 있는가?
    \item [ ] 은행 예금으로 차를 산 친구가 실제로는 '이자 비용(암묵적 비용)'을 지불하고 있음을 증명할 수 있는가?
    \item [ ] 수학적으로 $Q=f(P)$임에도 불구하고 왜 경제학자들이 가격(P)을 Y축에 그리는지 설명할 수 있는가?
    \item [ ] 수요 곡선이 왜 우하향하고, 공급 곡선이 왜 우상향하는지 '직관적'으로 설명할 수 있는가?
    \item [ ] 시장 가격이 균형 가격보다 높을 때 (초과 공급), 왜 가격이 하락할 수밖에 없는지 설명할 수 있는가?
    \item [ ] 시장 가격이 균형 가격보다 낮을 때 (초과 수요), 왜 가격이 상승할 수밖에 없는지 설명할 수 있는가?
    \item [ ] (매우 중요) "가솔린에 세금이 붙어 가격이 올랐다"는 뉴스가 '수요의 변화'인지 '수요량의 변화'인지 구별하고, 그 이유를 설명할 수 있는가?
    \item [ ] (매우 중요) "AI 기술의 발전으로 반도체 생산 비용이 절반이 되었다"는 뉴스가 '공급의 변화'인지 '공급량의 변화'인지 구별하고, 그 이유를 설명할 수 있는가?
\end{itemize}

\section{주요 Q\&A (FAQ)}

\begin{keypointbox}{Q: 왜 수요 곡선을 그릴 때 가격(P)을 Y축에 두나요? 수학적으로 틀린 것 아닌가요?}
    \textbf{A}: 맞습니다. 수학적으로 $Q=f(P)$이므로 P가 X축(독립변수)에 와야 하지만, 100여 년 전 경제학자 알프레드 마샬의 관행이 굳어진 것입니다. 우리는 기술적으로 '역수요함수'를 그리고 있지만, 모두가 이를 '수요 곡선'이라고 부르기로 약속했습니다. 불편하더라도 이 관행을 따라야 합니다.
\end{keypointbox}

\begin{keypointbox}{Q: 수요/공급 곡선은 현실에서 정말 직선(Linear)인가요?}
    \textbf{A}: 아닐 가능성이 높습니다. 현실에서는 복잡한 곡선 형태일 것입니다. 하지만 "가격이 오르면 덜 산다" (우하향) 또는 "가격이 오르면 더 판다" (우상향)는 핵심 직관(Intuition)은 동일합니다. 우리는 복잡한 수학 없이 이 핵심 직관을 배우기 위해 분석하기 쉬운 '직선' 모델을 사용하는 것입니다.
\end{keypointbox}

\begin{keypointbox}{Q: '수요의 변화'와 '수요량의 변화'를 구분하는 것이 왜 그렇게 중요한가요?}
    \textbf{A}: 시장 분석의 전부이기 때문입니다.
    \begin{itemize}
        \item "가격이 올라서 수요\textbf{량}이 줄었다" $\rightarrow$ 단순히 곡선 위에서 $A \rightarrow B$로 이동한 것입니다. 시장의 근본적인 룰(수요 곡선)은 그대로입니다.
        \item "유행이 변해서 수요가 줄었다" $\rightarrow$ 시장의 룰 자체가 변한 것입니다 ($D_0 \rightarrow D_1$). 이는 완전히 새로운 균형 가격과 균형 거래량을 만듭니다.
    \end{itemize}
    원인이 '가격'인지 '가격 외 요인'인지 구분하지 못하면, 시장이 앞으로 어떻게 변할지 전혀 예측할 수 없습니다.
\end{keypointbox}

\newpage

\section{빠르게 훑어보기 (1-Page Summary)}

\begin{tcolorbox}[title=\textbf{1. 희소성 (Scarcity) \& 기회비용 (Opportunity Cost)}]
    $\bullet$ \textbf{희소성}: 원하는 만큼 가질 수 없는 상태 $\rightarrow$ '배분 메커니즘' 필요.
    \newline
    $\bullet$ \textbf{시장}: '가격'을 배급 장치로 사용하여 희소성 문제를 해결.
    \newline
    $\bullet$ \textbf{기회비용}: 선택 시 포기하는 차선책의 가치. (e.g., 공짜 점심의 '시간' 비용, 저축으로 산 차의 '이자' 비용)
\end{tcolorbox}

\begin{tcolorbox}[title=\textbf{2. 수요 (Demand, $D_0$)}]
    $\bullet$ \textbf{정의}: 소비자가 각 가격에서 얼마나 사려 하는가? ($Q = f(P)$)
    \newline
    $\bullet$ \textbf{형태}: \textbf{우하향} (Downward Sloping)
    \newline
    $\bullet$ \textbf{이유 (수요의 법칙)}: 가격($P \uparrow$) $\rightarrow$ 기회비용($\uparrow$) $\rightarrow$ 구매($Q \downarrow$)
    \newline
    $\bullet$ \textbf{관행}: P를 Y축, Q를 X축에 그린다 (원래는 반대여야 함).
\end{tcolorbox}

\begin{tcolorbox}[title=\textbf{3. 공급 (Supply, $S_0$)}]
    $\bullet$ \textbf{정의}: 생산자가 각 가격에서 얼마나 팔려 하는가? ($Q = h(P)$)
    \newline
    $\bullet$ \textbf{형태}: \textbf{우상향} (Upward Sloping)
    \newline
    $\bullet$ \textbf{이유}: 생산량($Q \uparrow$) $\rightarrow$ 추가 생산 비용($\uparrow$) $\rightarrow$ 요구 가격($P \uparrow$)
\end{tcolorbox}

\begin{tcolorbox}[title=\textbf{4. 시장 균형 (Equilibrium, $E$)}]
    $\bullet$ \textbf{정의}: 수요 곡선과 공급 곡선의 교차점. "변화할 유인이 없는 상태"
    \newline
    $\bullet$ \textbf{균형}: 수요량($Q_D$) = 공급량($Q_S$)
    \newline
    $\bullet$ \textbf{가격 \textgreater{} 균형가} $\rightarrow$ 초과 공급 (Surplus) $\rightarrow$ 가격 하락($\downarrow$) 압력
    \newline
    $\bullet$ \textbf{가격 \textless{} 균형가} $\rightarrow$ 초과 수요 (Shortage) $\rightarrow$ 가격 상승($\uparrow$) 압력
\end{tcolorbox}

\begin{warningbox}{★ 모듈 1의 핵심: 용어 구분 ★}
\begin{tabularx}{\linewidth}{>{\raggedright}X >{\raggedright}X}
    \textbf{...량 (Quantity)의 변화} & \textbf{...자체 (Demand/Supply)의 변화} \\
    \midrule
    $\bullet$ 곡선 \textbf{위에서} 점이 이동 & $\bullet$ 곡선 \textbf{자체가} 평행 이동 \\
    $\bullet$ \textbf{원인: 가격(P) 변화} (Only) & $\bullet$ \textbf{원인: 가격 외} 모든 것 (날씨, 유행, 소득...) \\
    $\bullet$ 예: "가솔린 가격이 오르자 \textit{수요량이} 줄었다." & $\bullet$ 예: "전기차가 유행하자 가솔린 \textit{수요가} 줄었다." \\
\end{tabularx}
\end{warningbox}

\end{document}
